\documentclass[
	%noproofs
]{mycourse}

\title{Numerische Mathematik 2}
\author{}

\begin{document}

\maketitle
\tableofcontents

\chapter{Gewöhnliche Differentialgleichungen}


\section{Einführung und Beispiele}


\subsection{Mathematische Beispiele gewöhnlichere DGLs}

Eine DGL ist eine Gleichung, die eine unbekannte Funktion zusammen mit ihren Ableitungen enthält.

\begin{ex} \label{1.1}
	\begin{enumerate}[a)]
		\item
			Betrachte
			\[
				y'(x) = 0
			\]
			oder kurz $y' = 0$.
			Gesucht ist
			\begin{align*}
				y: \R &\to \R \\
				x \mapsto y(x)
			\end{align*}
			mit $y$ differenzierbar und $y'(x) = 0$ für alle $x \in \R$.
			\begin{itemize}
				\item
					Spezielle Lösungen sind z.B. $y(x)=0$, $y(x)=1$, $y(x) = 37$.
				\item
					Die \emph{allgemeine} Lösung ist $y(x) = C$, $C \in \R$, d.h. man kann zeigen, dass jede solche Funktion die DGL löst und jede Lösung in dieser Form geschrieben werden kann
			\end{itemize}
		\item
			Betrachte
			\[
				y'(x) = r y(x),
				\qquad r\in \R
			\]
			oder kurz $y' = ry$.
			\begin{itemize}
				\item
					Spezielle Lösungen sind $y(x)=e^{rx}$, $y(x)=0$, $y(x)=3e^{rx}$.
				\item
					Die \emph{allgemeine} Lösung ist $y(x) = C e^{rx}$, $C\in \R$.
			\end{itemize}
		\item
			Betrachte die DGL höherer Ordnung
			\[
				y''(x) = -y(x)
			\]
			oder kurz $y'' = -y$.
			\begin{itemize}
				\item
					Spezielle Lösungen sind $y(x) = 0$, $y(x) = \sin x$, $y(x)=\cos x$.
				\item
					Die allgemeine Lösung ist
					\[
						y(x) = C_1 \sin x + C_2 \cos x,
						\qquad C_1, C_2 \in \R
					\]
			\end{itemize}
		\item
			Betrachte das DGL-System
			\begin{align*}
				y_1''(x) &= -y_1(x) \\
				y_2''(x) &= y_3(x) \\
				y_3''(x) &= y_3(x) \\
			\end{align*}
			\begin{itemize}
				\item
					Eine spezielle Lösung ist
					\begin{align*}
						y_1(x) &= \sin x \\
						y_2(x) &= 37 \\
						y_3(x) &= 0
					\end{align*}
				\item
					Die allgemeine Lösung ist (für $C_1,C_2,C_3,C_4 \in \R$)
					\begin{align*}
						y_1(x) &= C_1 \sin x + C_2 \cos x \\
						y_2(x) &= C_3 e^x + C_4 \\
						y_3(x) &= C_3 e^x
					\end{align*}
			\end{itemize}
		\item
			Betrachte die DGL $y' = y$ für eine vektorwertige Funktion.
			Gesucht ist also
			\begin{align*}
				y: \R &\to \R^2 \\
				& x \mapsto y(x) = \begin{pmatrix}
					y_1(x) \\ y_2(x)
				\end{pmatrix}
			\end{align*}
			mit
			\begin{align*}
				\begin{pmatrix}
					y_1'(x) \\ y_2'(x)
				\end{pmatrix}
				=
				\begin{pmatrix}
					y_1(x) \\ y_2(x)
				\end{pmatrix}
			\end{align*}
			Man sieht hier sofort die Äquivalenz zu einem entsprechenden DGL-System.
	\end{enumerate}
\end{ex}

Bei gewöhnlichen DGLs beziehen sich alle Ableitungen auf die selbe eindimensionale Variable.

\begin{ex} \label{1.2}
	Betrachte $y : \R^2 \to \R : (x_1,x_2) \mapsto y(x) = y(x_1,x_2)$ mit einer sogenannten partielle Differentialgleichung (PDGL):
	\[
		\f{\d^2y}{\d x_1^2} + \f{\d^2 y}{\d x_2^2 }
	\]
	\begin{itemize}
		\item
			Spezielle Lösungen sind $y(x) = 0, y(x) = x_1 + x_2$.
		\item
			Allgemeine Lösungen sind schwierig.
	\end{itemize}
\end{ex}

\subsection{Anwendungsbeispiele für gewöhnliche Differentialgleichungen}

\subsubsection{Continuous compounding-stetige Verzinsung}

Beschreibe $y(t)$ das Sparguthaben zum Zeitpunkt $t$.

\fixme[Bild Zinsen]

Die Bank zahlt Zinsen proportional zum Guthaben und Zeitspanne
\[
	\Delta y(t) = y(t + \Delta t) - y(t) = r y(t) \Delta t
\]
mit festgelegtem Zinssatz $r$.
Für $\Delta t \to 0$ ergibt sich
\[
	\dot y (t) = ry(t),
\]
eine DGL mit der Lösung
\[
	y(t) = Ce^{rt}
\]

\subsubsection{Populationsdynamik}

Beschreibe $y(t)$ die Anzahl von Individuen in einer Population.

\begin{description}
	\item[Model 1 (Malthus)]
		Die Population wächst mit konstanter Rate $r \in \R$.
		In einer kurzen Zeiteinheit $\Delta t$ ist
		\[
			\Delta y \approx ry \Delta t
			\quad \implies \quad
			\dot y = r y
		\]
	\item[Model 2 (Verhulst)]
		Es existiert eine Maximalbevölkerung $M > 0$.
		Die Population wächst mit von $y$ abhängiger Rate
		\[
			\begin{cases}
				\text{Wachstum mit Rate $r$} & y << M \\
				\text{Wachstum mit Rate $0$} & y \approx M
			\end{cases}
		\]
		Im Zeitintervall $\Delta t$ wächst die Population um $\Delta y \approx r(1-\f yM) M \Delta t$.
		Es ergibt sich die DLG
		\[
			\dot y = r \Big(1 - \f yM\Big) y = ry \underbrace{- \f rM y^2}_{\text{Todesrate aufgrund zu hoher Population}}
		\]
	\item[Model 3 (Lofka-Volterra, Räuber-Beute Modell)]
		Betrachte zwei Populationen $y_1(t), y_2(t)$ (Räuber und Beutetiere).
		\begin{align*}
			\dot y_1 &= r_1 y_1 - f_1y_1y_2
			\qquad r_1,f_1 \in \R \\
			\dot y_2 &= -r_2 y_2 + f_2 y_1y_2
		\end{align*}
\end{description}


\subsubsection{Chemische Reaktionen}


Seien $A(t), B(t), C(t)$ Konzentrationen der Chemikalien $A,B,C$.
Betrachte die Reaktion
\[
	A \overset{k}{\longrightarrow} B.
\]
In einem kurzen Zeitinvervall $\Delta t$ wandeln sich $k A(t) \Delta t$ Moleküle von $A$ in $B$ um.
\begin{align*}
	\dot A(t) &= -k A(t) \\
	\dot B(t) &= k A(t)
\end{align*}
Betrachte die Reaktion
\[
	A + B \overset{k}{\longrightarrow} C + 2D.
\]
\begin{align*}
	\dot A &= -kAB,  \\
	\dot B &= -kAB, \\
	\dot C &= kAB, \\
	\dot D &= 2kAB
\end{align*}


\subsubsection{Newtonsche Gesetze}

\fixme[Bild]

Beschreibe $x(t)$ die Position, $v(t) = \dot x(t)$ die Geschwindigkeit und $a(t) = \dot v(t) = \ddot x(t)$ die Beschleunigung eines Teilchens im $\R^3$ zu einem Zeitpunkt $t$.

Newton: Wirkt eine Kraft $F(t)$ auf einen Körper der Masse $m$, so ist
\[
	F(t) = m a(t) = m\ddot x(t)
\]


\subsubsection{Elektrische Schaltkreise}

\fixme[Bild: Schaltkreis]

Betrachte eine RLC-Reihenschaltung.
\begin{itemize}
	\item
		Stromstärke $I_R$ durch den Widerstand
		\[
			I_R(t) = \f {U_R(t)}R
		\]
		für $R \in \R$.
	\item
		Stromstärke $I_C$ durch den Kondensator
		\[
			I_C(t) = \dot  U_C(t) C
		\]
		für $C \in \R$.
	\item
		Stromstärke $I_L$ durch die Spule
		\[
			\dot I_L(t) = \f {U_L(t)}L
		\]
		für $L \in \R$.
\end{itemize}

Die Kirchhoffschen Gesetze besagen
\begin{gather*}
	U(t) = U_R(t) + U_C(t) + U_L(t), \qquad \text{(Spannungsbilanz)}
	I_L(t) =  I_C(t)0= I_R(t). \qquad \text{Strombilanz}
\end{gather*}
Es ergibt sich mit obigen Gleichungen
\begin{align*}
	U_R(t) &= RC \dot U_C(t) \\
	U_L(t) &= L \dot I_L(t) = LC \ddot U_C(t) \\
\end{align*}
also ergibt sich die DGL
\[
	U(t) = U_C(t) + RC \dot U_C(t) + LC \ddot U_C(t)
\]


\subsection{Anfangswertprobleme}


Die Lösung einer gewöhnlichen DGL ist üblicherweise nicht eindeutig bestimmt.
Die allgemeine Lösung von
\[
	\dot y(t) = ry(t)
	\qquad \text{ist} \qquad
	y(t) = Ce^{rt}
\]
mit Parameter $C \in \R$.
In viele Anwendungen sind die Parameter eindeutig bestimmt durch die Anfangswerte von $y$.
Beispielsweise bei stetiger Verzinsung gibt $C=y(0)$ das Anfangsguthaben an.

Intuitiv erwarten wir, dass die Lösung eindeutig durch folgende Vorgaben bestimmt wird:
\begin{itemize}
	\item
		Bei der Populationsdynamik durch die anfängliche Population $y(0)$, bzw. $y_1(0)$ und $y_2(0)$.
	\item
		Bei der Chemischen Reaktion durch die anfänglichen Konzentrationen $A(0), B(0)$.
	\item
		Bei den Newtonschen Gesetzen durch die anfängliche Position $x_1(0),x_2(0),x_3(0)$, und die anfänglichen Geschwindigkeiten $v_1(0), v_2(0), v_3(0)$.
	\item
		Bei den Elektrischen Schaltkreisen durch die anfängliche Spannung $U_C(0)$ und Stromstärke $\dot U_C(0)$ am Kondensator.
\end{itemize}
Eine gewöhnliche DGL zusammen mit Anfangsbedingungen heißt \emph{Anfangswertproblem} (AWP).


\subsection{Elementare Lösungsmethoden}


\subsubsection{Raten/Wissen der Lösung}

Siehe \ref{1.1}.

\subsection{Separation der Variablen}


Für eine gewöhnliche DGL der Form
\[
	y'(x) = g(x) h(y(x))
\]
schreiben wir formal
\[
	\f {dy}{dx} = g(x) h(y)
	\qquad \leadsto \qquad
	\int \f 1{h(y)} dy = \int g(x) dx
\]

\begin{ex} \label{1.3}
	Betrachte Modell 2 aus der Populationsdynamik ($M=1=r$).
	\begin{align*}
		\f {dy}{dt} = \dot y(t) = (1-y)y
		\qquad \leadsto \qquad
		\int \f 1{(1-y)y} dy = \int 1 dt
	\end{align*}
	Für einen Startwert $y(0) > 1$ erwarten wir $y(t) \ge 1$ für $t \ge 0$.
	Sei num $y(0) > 1$.

	Es ergibt sich durch Partialbruchzerlegung
	\begin{align*}
		\int \f 1{(1-y)y} dy 
		= \int \f 1y dy + \int 1{1-y} dy
		= - \ln (y-1) + \ln(y) + \const
		= \ln \f {y}{y-1} + \const
	\end{align*}
	Also 
	\begin{align*}
		\ln \f {y}{y-1} &= t + \const \\
		\implies \qquad	\f y{y-1} &= C e^t \qquad C \ge 0 \\
		\implies \qquad y(t)= \f {Ce^t}{Ce^t - 1}
	\end{align*}
	Das Vorgehen war \emph{formal} und beweist weder, dass dies eine Lösung ist, noch, dass alle Lösungen so aussehen.
	Erst Einsetzen von $y(t)$ in die DGL beweist, dass dies eine Lösung (für jedes $C>0$) ist.

	$C$ kann aus Anfangswerten bestimmt werden:
	\[
		y_0 := y(0) = \f C{C-1}
		\qquad \implies \qquad
		C = \f {y_0}{y_0-1}
	\]
\end{ex}

\subsubsection{Variation der Konstanten}

Betrachte eine lineare inhomogene DGL
\[
	y'(x) = ry(x) + z(x).
\]
Die allgemeine Lösung der \emph{homogenen Gleichung} $y'(x) = ry(x)$ ist $y(x) = Ce^{rx}$.

Ansatz: Ersetze zur Lösung der inhomogenen Gleichung die Konstante $C$ durch eine Funktion $C(x)$:
\[
	y(x) = C(x) e^{rx}
\]
Löst $y$ die DGL, so ist
\[
	rC(x)e^{rx} + C'(x)e^{rx} = y'(x) = ry(x) + z(x) = rC(x)e^{rx} + z(x)
\]
also $C'(x) e^{rx} = z(x)$ und damit
\[
	C'(x) = z(x) e^{-rx}
\]
Integration liefert $C(x)$ und Nachprüfen zeigt, dass die so gefundene Funktion tatsächlich die DGL löst.



\section{Theorie gewöhnlicher Differentialgleichungen}



\subsection{Eine allgemeine Form}


\begin{ex} \label{1.4}
	Die DGL $y'' = -y$ kann in ein System erster Ordnung transformiert werden.
	Setze dazu
	\begin{align*}
		u = \begin{pmatrix}
			u_1(x) \\ u_2(x)
		\end{pmatrix} = \begin{pmatrix}
			y(x) \\ y'(x)
		\end{pmatrix}
	\end{align*}
	Die DGL lässt sich jetzt schreiben als
	\begin{align*}
		u'(x) =
		\begin{pmatrix}
			y'(x) \\ y''(x)
		\end{pmatrix}
		= \begin{pmatrix}
			y'(x) \\ -y(x)
		\end{pmatrix}
		= \begin{pmatrix}
			u_2(x) \\ -u_1(x)
		\end{pmatrix}
		=: f(x, u(x))
	\end{align*}
\end{ex}

\end{document}


