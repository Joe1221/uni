\documentclass[
	%noproofs
]{mycourse}

\title{Numerische Mathematik 2}
\author{}

\begin{document}

\maketitle
\tableofcontents

\chapter{Gewöhnliche Differentialgleichungen}


\section{Einführung und Beispiele}


\subsection{Mathematische Beispiele gewöhnlichere DGLs}

Eine DGL ist eine Gleichung, die eine unbekannte Funktion zusammen mit ihren Ableitungen enthält.

\begin{ex} \label{1.1}
	\begin{enumerate}[a)]
		\item
			Betrachte
			\[
				y'(x) = 0
			\]
			oder kurz $y' = 0$.
			Gesucht ist
			\begin{align*}
				y: \R &\to \R \\
				x \mapsto y(x)
			\end{align*}
			mit $y$ differenzierbar und $y'(x) = 0$ für alle $x \in \R$.
			\begin{itemize}
				\item
					Spezielle Lösungen sind z.B. $y(x)=0$, $y(x)=1$, $y(x) = 37$.
				\item
					Die \emph{allgemeine} Lösung ist $y(x) = C$, $C \in \R$, d.h. man kann zeigen, dass jede solche Funktion die DGL löst und jede Lösung in dieser Form geschrieben werden kann
			\end{itemize}
		\item
			Betrachte
			\[
				y'(x) = r y(x),
				\qquad r\in \R
			\]
			oder kurz $y' = ry$.
			\begin{itemize}
				\item
					Spezielle Lösungen sind $y(x)=e^{rx}$, $y(x)=0$, $y(x)=3e^{rx}$.
				\item
					Die \emph{allgemeine} Lösung ist $y(x) = C e^{rx}$, $C\in \R$.
			\end{itemize}
		\item
			Betrachte die DGL höherer Ordnung
			\[
				y''(x) = -y(x)
			\]
			oder kurz $y'' = -y$.
			\begin{itemize}
				\item
					Spezielle Lösungen sind $y(x) = 0$, $y(x) = \sin x$, $y(x)=\cos x$.
				\item
					Die allgemeine Lösung ist
					\[
						y(x) = C_1 \sin x + C_2 \cos x,
						\qquad C_1, C_2 \in \R
					\]
			\end{itemize}
		\item
			Betrachte das DGL-System
			\begin{align*}
				y_1''(x) &= -y_1(x) \\
				y_2''(x) &= y_3(x) \\
				y_3''(x) &= y_3(x) \\
			\end{align*}
			\begin{itemize}
				\item
					Eine spezielle Lösung ist
					\begin{align*}
						y_1(x) &= \sin x \\
						y_2(x) &= 37 \\
						y_3(x) &= 0
					\end{align*}
				\item
					Die allgemeine Lösung ist (für $C_1,C_2,C_3,C_4 \in \R$)
					\begin{align*}
						y_1(x) &= C_1 \sin x + C_2 \cos x \\
						y_2(x) &= C_3 e^x + C_4 \\
						y_3(x) &= C_3 e^x
					\end{align*}
			\end{itemize}
		\item
			Betrachte die DGL $y' = y$ für eine vektorwertige Funktion.
			Gesucht ist also
			\begin{align*}
				y: \R &\to \R^2 \\
				& x \mapsto y(x) = \begin{pmatrix}
					y_1(x) \\ y_2(x)
				\end{pmatrix}
			\end{align*}
			mit
			\begin{align*}
				\begin{pmatrix}
					y_1'(x) \\ y_2'(x)
				\end{pmatrix}
				=
				\begin{pmatrix}
					y_1(x) \\ y_2(x)
				\end{pmatrix}
			\end{align*}
			Man sieht hier sofort die Äquivalenz zu einem entsprechenden DGL-System.
	\end{enumerate}
\end{ex}

Bei gewöhnlichen DGLs beziehen sich alle Ableitungen auf die selbe eindimensionale Variable.

\begin{ex} \label{1.2}
	Betrachte $y : \R^2 \to \R : (x_1,x_2) \mapsto y(x) = y(x_1,x_2)$ mit einer sogenannten partielle Differentialgleichung (PDGL):
	\[
		\f{\d^2y}{\d x_1^2} + \f{\d^2 y}{\d x_2^2 }
	\]
	\begin{itemize}
		\item
			Spezielle Lösungen sind $y(x) = 0, y(x) = x_1 + x_2$.
		\item
			Allgemeine Lösungen sind schwierig.
	\end{itemize}
\end{ex}

\subsection{Anwendungsbeispiele für gewöhnliche Differentialgleichungen}

\subsubsection{Continuous compounding-stetige Verzinsung}

Beschreibe $y(t)$ das Sparguthaben zum Zeitpunkt $t$.

\fixme[Bild Zinsen]

Die Bank zahlt Zinsen proportional zum Guthaben und Zeitspanne
\[
	\Delta y(t) = y(t + \Delta t) - y(t) = r y(t) \Delta t
\]
mit festgelegtem Zinssatz $r$.
Für $\Delta t \to 0$ ergibt sich
\[
	\dot y (t) = ry(t),
\]
eine DGL mit der Lösung
\[
	y(t) = Ce^{rt}
\]

\subsubsection{Populationsdynamik}

Beschreibe $y(t)$ die Anzahl von Individuen in einer Population.

\begin{description}
	\item[Model 1 (Malthus)]
		Die Population wächst mit konstanter Rate $r \in \R$.
		In einer kurzen Zeiteinheit $\Delta t$ ist
		\[
			\Delta y \approx ry \Delta t
			\quad \implies \quad
			\dot y = r y
		\]
	\item[Model 2 (Verhulst)]
		Es existiert eine Maximalbevölkerung $M > 0$.
		Die Population wächst mit von $y$ abhängiger Rate
		\[
			\begin{cases}
				\text{Wachstum mit Rate $r$} & y << M \\
				\text{Wachstum mit Rate $0$} & y \approx M
			\end{cases}
		\]
		Im Zeitintervall $\Delta t$ wächst die Population um $\Delta y \approx r(1-\f yM) M \Delta t$.
		Es ergibt sich die DLG
		\[
			\dot y = r \Big(1 - \f yM\Big) y = ry \underbrace{- \f rM y^2}_{\text{Todesrate aufgrund zu hoher Population}}
		\]
	\item[Model 3 (Lofka-Volterra, Räuber-Beute Modell)]
		Betrachte zwei Populationen $y_1(t), y_2(t)$ (Räuber und Beutetiere).
		\begin{align*}
			\dot y_1 &= r_1 y_1 - f_1y_1y_2
			\qquad r_1,f_1 \in \R \\
			\dot y_2 &= -r_2 y_2 + f_2 y_1y_2
		\end{align*}
\end{description}


\subsubsection{Chemische Reaktionen}


Seien $A(t), B(t), C(t)$ Konzentrationen der Chemikalien $A,B,C$.
Betrachte die Reaktion
\[
	A \overset{k}{\longrightarrow} B.
\]
In einem kurzen Zeitinvervall $\Delta t$ wandeln sich $k A(t) \Delta t$ Moleküle von $A$ in $B$ um.
\begin{align*}
	\dot A(t) &= -k A(t) \\
	\dot B(t) &= k A(t)
\end{align*}
Betrachte die Reaktion
\[
	A + B \overset{k}{\longrightarrow} C + 2D.
\]
\begin{align*}
	\dot A &= -kAB,  \\
	\dot B &= -kAB, \\
	\dot C &= kAB, \\
	\dot D &= 2kAB
\end{align*}



\end{document}
