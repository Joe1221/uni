% Kapitel 4
\chapter{Homologie}

\Timestamp{2016-01-15}

Verschiedene Vorgehensweisen/Geschmacksrichtungen:
\begin{itemize}
    \item
        Simpliziale Homologie
    \item
        Singuläre Homologie
        \begin{itemize}
            \item
                Singuläre simpliziale Homologie
            \item
                Singuläre kubische Homologie
        \end{itemize}
    \item
        Axiomatische Homologie
    \item
        Zelluläre Homologie
\end{itemize}

Sei $K$ ein Simplizialkomplex mit Eckenmenge $\Omega = \bigcup K$.
Ziel: orientierte Simplizes.

Punkt, Strecke, Dreieck:
\begin{math}
    \<v_0\>, & \\
    \<v_0, v_1\>, && \Boundary \<v_0, v_1\> &= \<v_1\> - \<v_0\>, \\
    \<v_1, v_0\> = -\<v_0, v_1\>, & \\
    \<v_0, v_1, v_2\>, && \Boundary \<v_0, v_1, v_2\> &= \<v_0, v_1\> + \<v_1, v_2\> + \<v_2, v_0\>, \\
    && &= + \<\hat v_0, v_1, v_2\> - \<v_0, \hat v_1, v_2\> + \< v_0, v_1, \hat v_2\>, \\
\end{math}
Es gilt
\begin{math}
    \<v_1, v_2, v_0\> &= \<v_0, v_1, v_2\>, \\
    \<v_0, v_2, v_1\> &= -\<v_0, v_1, v_2\>,
\end{math}
Tetraeder:
\begin{math}
    \Boundary\<v_0, v_1, v_2, v_3\>
    = \<v_1, v_2, v_3\> - \<v_0, v_2, v_3\> + \<v_0, v_1, v_3\> - \<v_0, v_1, v_2\>.
\end{math}

Konstruktion:

Sei $R$ ein kommutativer Ring mit Eins, etwa $R = \Z$, oder $\Z / p$, $\Q$, $\R$, $\C$.
\begin{math}
    \tilde K_n &:= \Set{(v_0, \dotsc, v_n) \in \Omega^{n+1} & \Set{v_0, \dotsc, v_n} \in K}, \\
    R\tilde K_n &:= \bigoplus_{\sigma \in \tilde K_n} R\sigma \qquad \text{freier $R$-Modul über $\tilde K_n$} \\
    &= R^{(\tilde K_n)} \\
    D_n := D_n K &:= R( \Set{(v_0, \dotsc, v_i, v_{i+1}, \dotsc, v_n) & v_i = v_{i+1}} \\
    & \qquad \cup \Set{(v_0, \dotsc, v_i, v_{i+1}, \dotsc, v_n) + (v_0, \dotsc, v_{i+1}, v_i, \dotsc, v_n)})
\end{math}

\begin{df}
    Wir setzen
    \begin{math}
        C_n := C_n K := C_n(K, R) := R\tilde K_n / D_n K
    \end{math}
    mit Quotientenabbildung
    \begin{math}
        \quot: R\tilde K_n &\to C_n \\
        (v_0, \dotsc, v_n) &\mapsto \<v_0, \dotsc, v_n\>.
    \end{math}
\end{df}

\begin{prop}
    \begin{itemize}
        \item
            $\<v_0, \dotsc,v_i, v_{i+1}, \dotsc, v_n\> = - \<v_0, \dotsc, v_{i+1}, v_i, \dotsc, v_n\>$,
        \item
            $\<v_{\pi 0}, v_{\pi 1}, \dotsc, v_{\pi n}\> = \sign(\pi) \<v_0, v_1, \dotsc, v_n\>$,
        \item
            $\<v_0, \dotsc, v_n\> = 0$ für $v_i = v_j$ mit $i \neq j$.
    \end{itemize}
\end{prop}

\begin{conv}
    Setze $C_n = 0$ für $n < 0$.
\end{conv}

\begin{df}
    Setze $\Boundary_i := \Boundary_i^{(n)}: \tilde K_n \to \tilde K_{n-1}$ als
    \begin{math}
        (v_0, \dotsc, v_n) \mapsto (v_0, \dotsc, \hat v_i, \dotsc, v_n) := (v_0, \dotsc, v_{i-1}, v_{i+1}, \dotsc, v_n).
    \end{math}
    Definiere damit $\Boundary = \Boundary^{(n)}: R \tilde K_n \to R \tilde K_{n-1}$ linear durch
    \begin{math}
        \Boundary := \sum_{i=0}^n (-1)^i \Boundary_i:
        (v_0, \dotsc, v_n) \mapsto \sum_{i=0}^n (-1)^i (v_0, \dotsc, \hat v_i, \dotsc, v_n).
    \end{math}
\end{df}

\begin{prop}
    $\Boundary(D_n) \subset D_{n-1}$
    \begin{proof}
        Nachrechnen.
    \end{proof}
\end{prop}

Also
\begin{math}
    \begin{tikzcd}
        R\tilde K_n \ar[r,"\del"] \ar[d,"\quot"] \ar[dr] & R \tilde K_{n-1} \ar[d,"\quot"] \\
        C_n K \ar[r,dashed,"\exists! \del"] & C_{n-1} K
    \end{tikzcd}
\end{math}
Dank voriger Proposition ist also $\del$ wohldefiniert auf $C_n K$:
\begin{math}
    \<v_0, \dotsc, v_n\>
    \mapsto \sum_{i=0}^n (-1)^i \<v_0, \dotsc, \hat v_i, \dotsc, v_n\>.
\end{math}

\begin{lem}
    Die Komposition $C_{n+1} K \xto{\Boundary^{n+1}} C_n K \xto{\Boundary^{n}} C_{n-1} K$ ist Null, kurz: $\Boundary^2 = 0$.
    \begin{proof}
        \begin{math}
            \Boundary^2(v_0, \dotsc, v_{n+1})
            &= \Boundary \sum_{i=0}^{n+1} (i-1)^i (v_0, \dotsc, \hat v_i, \dotsc, v_{n+1}) \\
            &= \sum_{i=0}^{n+1} (-1)^i \sum_{j=0}^{i-1} (-1)^j (v_0, \dotsc, \hat v_j, \dotsc, \hat v_i, \dotsc, v_{n+1}) \\
            &\qquad + \sum_{i=0}^{n+1} (-1)^{i} \sum_{j=i+1}^{n+1} (-1)^{j-1} (v_0, \dotsc, \hat v_j, \dotsc, \hat v_i, \dotsc, v_{n+1}) \\
            &= \sum_{i<j} (-1)^{i+j-1} (v_0, \dotsc, \hat v_i, \dotsc, \hat v_j, \dotsc, v_{n+1}) \\
            &\qquad + \sum_{i>j} (-1)^{i+j} (v_0, \dotsc, \hat v_j, \dotsc, \hat v_i, \dotsc, v_{n+1}) \\
            &= 0
        \end{math}
    \end{proof}
    \begin{note}
        Geometrische Analogie für Mannigfaltigkeiten: $\Boundary^2 M = \emptyset$.
    \end{note}
\end{lem}


\begin{df}
    Ein \emphdef{Kettenkomplex} über $R$ ist eine Familie $(C_n, \Boundary_n)_{n \in \Z}$ von $R$-Moduln $C_n$ und $R$-Homomorphismen $\Boundary_n: C_n \to C_{n-1}$ mit $\Boundary \circ \Boundary = 0$.
    Typischerweise $C_k = 0$ für $k < 0$.
    \begin{math}
        0 \xto* C_0 \xto*{\Boundary_1} C_1 \xto*{\Boundary_2} C_2 \xto* \dotsc
    \end{math}
    also $\im(\Boundary_{n+1}) \subset \ker(\Boundary_n)$.
    Wir nennen $Z_n := \ker(\Boundary_n)$ \emph{$n$-Zykel} und $B_n := \im(\Boundary_{n+1})$ \emph{$n$-Ränder}.

    Die \emphdef{$n$-Homologie} ist definiert als
    \begin{math}
        H_n(C_*, \Boundary_*) := Z_n(C_*, \Boundary_*) / B_n(C_*, \Boundary_*).
    \end{math}
    Für $z \in Z_n$ gilt $\Boundary z = 0$.
    Wir nennen $[z] := z + B_n \in H_n$ die \emphdef{Homologieklasse} von $z$.
    $z \sim z' \iff z - z' = \Boundary b_{n+1} \in B_n$: \emphdef{homologe Zykel} $z, z' \in Z_n$.
\end{df}

\begin{ex}
    Simplizialkomplex: Kreisring, innen Dreieck $z$, außen Viereck $z'$.
    \begin{math}
        z - z' &= \Boundary b, \\
        b &= \sigma_1 + \dotsb + \sigma_7.
    \end{math}
\end{ex}

Erste Beispiele:

\begin{itemize}
    \item
        $K = K_1$ mit Kettenkomplex
        \begin{math}
            0 \xto* R \<v_0\> \xto* 0 \xto* \dotsc
        \end{math}
        Homologie:
        \begin{math}
            H_n(\ast, R) = \begin{cases}
                R[v_0] \isomorphic R & \text{für $n = 0$,} \\
                0 & \text{für $n \neq 0$.}
            \end{cases}
        \end{math}
    \item
        $K = K_2$ mit Kettenkomplex
        \begin{math}
            0 \xto* R\<v_0\> \oplus R\<v_1\> &\xto* R\<v_0,v_1\> \xto* 0 \xto* \dotsc \\
            \<v_1\> - \<v_0\> &\mapsfrom \<v_0, v_1\>
        \end{math}
        Homologie:
        \begin{math}
            Z_0 &= C_0, \\
            B_0 &= R(\<v_1\> - \<v_0\>), \\
            H_0 &= Z_0 / B_0 \isomorphic R[v_0] = R[v_1] \isomorphic R,
        \end{math}
        da
        \begin{math}
            \begin{tikzcd}
                H_0 \ar[r,"\homeomorphic"] & R \\
                C_0 \ar[u] \ar[ru,"{\<v_0\>,\<v_1\> \mapsto 1}"']
            \end{tikzcd}.
        \end{math}
        Also
        \begin{math}
            H_n &= \begin{cases}
                R & \text{für $n = 0$}, \\
                0 & \text{sonst}.
            \end{cases}
        \end{math}
    \item
        $K : v_0 v_1 v_2$.
        \begin{math}
            0 \xto* R\<v_0\> \oplus R\<v_1\> \oplus R\<v_2\> \xto* R\<v_0,v_1\> \oplus R\<v_1,v_2\> \xto* 0  \xto* \dotsc
        \end{math}
        $H_n$ wie zuvor.
    \item
        $K$: Rand eines Dreiecks
        \begin{math}
            C_0 &= R\<v_0\> \oplus R\<v_1\> \oplus R \<v_2\> \\
            C_1 &= R\<v_0, v_1\> \oplus R\<v_1, v_2\> \oplus R\<v_2, v_0\> \\
            C_n &= 0 \quad \text{für $n \ge 2$}
        \end{math}
        \begin{math}
            Z_0 &= C_0, B_0 = R\Set{\<v_1\> - \<v_0\>, \<v_2\> - \<v_1\>, \underbrace{\<v_0\> - \<v_2\>}_{\text{redundant}}}, \\
            H_0 &= R[v_0] = R[v_1] = R[v_2].
        \end{math}
        $Z_1$ wird frei erzeugt von $Z := \Boundary\<v_0,v_1,v_2\>$ (nutze allgemeinen Ansatz mit Elementen aus $C_1$)
        \begin{math}
            Z_1 &= R_Z = R(\<v_0,v_1\> + \<v_1,v_2\> + \<v_2,v_0\>),&
            B_1 &= 0, \\
            H_1 &= Z_1 / B_1 = R[z] \isomorphic R.
        \end{math}
        Also
        \begin{math}
            H_n(\Delta, R) = \begin{cases}
                R[v_0] & \text{für $n = 0$}, \\
                R[z] & \text{für $n = 1$}, \\
                0 & \text{sonst}.
            \end{cases}
        \end{math}
\end{itemize}

\begin{st}
    Für jeden Simplizialkomplex $K$ gilt
    \begin{math}
        H_0(K, R) &\xto{\isomorphic} R \pi_0(K), \\
        \<v_0\> + B_0 = [v_0] &\mapsto [v_0] = \Set{w \in \Omega & \text{$\exists$ Weg $v_0 \dotsc v_n = w$}}.
    \end{math}
    Ist $K$ zusammenhängend, so gilt
    \begin{math}
        H_1(K, \Z) &\xto{\isomorphic} \pi_1(K,x_0)_{\text{ab}}.
    \end{math}
    Genauer: Wähle einen Spannbaum $T \subset K$
    \begin{math}
        \begin{tikzcd}
            H_1(K, \Z) \ar[r,"h","\isomorphic"'] & \pi_1(K, x_0)_{\text{ab}} \\
            Z_1(K, \Z) \ar[u,"\quot"] \ar[ru] \ar[d,"\subset"] & \pi_1(K, x_0) \ar[u] \\
            C_1(K, \Z) \ar[ruu] & \<S \mathrel{|} R\> \ar[u,"\isomorphic"] \ar[l] \\
            \Z \tilde K_1 \ar[u,"\quot"] \ar[ruuu,"\tilde h"] & \<S \mathrel{|} -\> \ar[u] \ar[l,"\tilde k"] \ar[lu]
        \end{tikzcd}
    \end{math}
    mit
    \begin{math}
        \tilde h(v_0,v_1)
        &:= [\underbrace{x_0 \dotsc v_0}_{\text{in $T$}} \ast v_0v_1 \ast \underbrace{v_1 \dotsc x_0}_{\text{in $T$}}] \\
        \tilde k(v_0v_1) &:= (v_0,v_1)
    \end{math}
    $S$: Kanten, $R$: Dreiecke.

    Für beliebige $K$ folgt dann
    \begin{math}
        \begin{tikzcd}
            H_1(K, \Z) \ar[r,"\isomorphic"] & \bigoplus_{[x_0] \in \pi_0(K)} \pi_1([x_0], x_0)_{\text{ab}}.
        \end{tikzcd}
    \end{math}
    \begin{proof}
        Übung
    \end{proof}
\end{st}

\begin{lem}
    Jede simpliziale Abbildung $f: K \to L$ induziert
    \begin{math}
        \begin{tikzcd}
            f_*: R \tilde K_n \ar[r,"f_n"] \ar[d,"\quot"] \ar[dr] & \tilde R \tilde L_n \ar[d] \\
            C_n K \ar[r,dashed,"f_n"] & C_n L
        \end{tikzcd}
    \end{math}
    mit
    \begin{math}
        f_n: (v_0, \dotsc, v_n) &\mapsto (f(v_0), \dotsc, f(v_n)) \\
        \<v_0, \dotsc, v_n\> &\mapsto \<f(v_0), \dotsc, f(v_n)\> \\
    \end{math}
    Es gilt $f_{n-1} \circ \Boundary_n^k = \Boundary_n^L \circ f_n$, kurz $f \circ \Boundary = \Boundary \circ f$ d.h.
    \begin{math}
        \begin{tikzcd}
            0 & C_0 K \ar[l] \ar[d,"f_0"] & C_1 K \ar[l,"\Boundary_1^K"] & C_2 K \ar[l,"\Boundary_2^k"] \ar[d,"f_2"] & \dotsc\ar[l] \\
            0 & C_0 L \ar[l] & C_1 L \ar[l,"\Boundary_1^L"] & C_2 L \ar[l,"\Boundary_2^L"] & \dotsc\ar[l]
        \end{tikzcd}
    \end{math}
    Dies nennt man einen \emphdef{Kettenhomomorphismus}.
\end{lem}

\begin{prop}
    Jeder Kettenhomomorphismus $g: (C_*, \Boundary_*) \to (C_*', \Boundary_*')$ induziert
    \begin{math}
        g_n = H_n(g): H_n(C_*, \Boundary_*) & \to H_n(C_*', \Boundary_*') \\
        [z] &\mapsto [g(z)].
    \end{math}
    \begin{proof}
        Wohldefiniertheit:
        \begin{enumerate}[i)]
            \item
                $g_n(Z_n) \subset Z_n'$: $\Boundary z = 0 \implies \Boundary(g(z) = g(\Boundary z) = g(0) = 0$.
            \item
                $g_n(B_n) \subset B_n'$: $z-z' = \Boundary b \implies g(z) - g(z') = g(\Boundary b) = \Boundary g(b)$.
        \end{enumerate}
    \end{proof}
\end{prop}

Kurzum: Wir erhalten Funktoren
\begin{math}
    \Cat{SComp} &\to \Cat{CComp}_R, \\
    K &\mapsto (C_* K, \Boundary_*), \\
    (f: K \to L) &\mapsto (f_*: C_* K \to C_* L).
\end{math}
\begin{math}
    \Cat{CComp}_R &\to \Cat{GrMod}_R, \\
    (C_n, \Boundary_n)_{n\in \Z} &\mapsto H_n(C_*, \Boundary_*)_{n\in \Z}, \\
    (f_n: C_n \to C_n')_{n\in \Z} &\mapsto H_n(f)_{n\in \Z}.
\end{math}

\begin{st}
    Sei $K = \Set{v} \ast L$ der Kegel über $L$ mit Spitze $v$, d.h.
    \begin{math}
        K = L \cup \Set{\Set{v} \sqcup \sigma & \sigma \in L}.
    \end{math}
    Dann gilt
    \begin{math}
        H_n K = H_n \Set{v}
        = \begin{cases}
            R[v_0] & \text{für $n = 0$}, \\
            0 & \text{sonst}.
        \end{cases}
    \end{math}
    \begin{proof}
        Definiere $T: C_n K \to C_{n+1} K$ durch $\<v_0, \dotsc, v_n\> \mapsto \<v, v_0, \dotsc, v_n\>$.
        \begin{math}
            \Boundary T\<v_0, \dotsc, v_n\>
            &= +\<v_0, \dotsc, v_n\> - \sum_{i=0}^n (-1)^i \<v, v_0, \dotsc, \hat v_i, \dotsc, v_n\> \\
            &= \id \<v_0, \dotsc, v_n\> - T \Boundary \<v_0, \dotsc, v_n\>,
        \end{math}
        d.h. $\Boundary T + T \Boundary = \id - 0$.
    \end{proof}
\end{st}

\begin{df}
    Eine \emphdef{Kettenhomotopie} $T: f \sim g$ ist eine Familie von Abbildungen $T_n: C_n \to C_{n+1}'$ mit $\Boundary_{n+1} T_n + T_{n-1} \Boundary_n = f_n - g_n$.
    \begin{math}
        \begin{tikzcd}
            C_{n-1} & C_n \ar[l,"\Boundary_n"] \ar[d,"f_n"',shift right] \ar[d,"g_n",shift left] & C_{n+1} \ar[l,"\Boundary_{n+1}"] \ar[d,"f_{n+1}"',shift right] \ar[d,"g_{n+1}",shift left] \\
            C_{n-1}' & C_n' \ar[l,"\Boundary_n"] & C_{n+1} \ar[l,"\Boundary_{n+1}"]
        \end{tikzcd}
    \end{math}
\end{df}

\begin{prop}
    Aus $T: f \sim g$ folgt $H_*(f) = H_*(g)$, also
    \begin{math}
        H_n(f) = H_n(g): H_n(C) \to H_n(C')
    \end{math}
    \begin{proof}
        $[(f_n - g_n)(z)] = 0$
        \begin{math}
            (f-g)(z)
            = (\Boundary T + T \Boundary)(z)
            = \Boundary T z
        \end{math}
        Also $f(z) \sim g(z)$.
    \end{proof}
\end{prop}
