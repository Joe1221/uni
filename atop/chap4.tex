% Kapitel 4
\chapter{Simpliziale Homologie}

\Timestamp{2016-01-15}

Verschiedene Vorgehensweisen/Geschmacksrichtungen:
\begin{itemize}
    \item
        Simpliziale Homologie
    \item
        Singuläre Homologie
        \begin{itemize}
            \item
                Singuläre simpliziale Homologie
            \item
                Singuläre kubische Homologie
        \end{itemize}
    \item
        Axiomatische Homologie
    \item
        Zelluläre Homologie
\end{itemize}

Sei $K$ ein Simplizialkomplex mit Eckenmenge $\Omega = \bigcup K$.
Ziel: orientierte Simplizes.

Punkt, Strecke, Dreieck:
\begin{math}
    \<v_0\>, & \\
    \<v_0, v_1\>, && \Boundary \<v_0, v_1\> &= \<v_1\> - \<v_0\>, \\
    \<v_1, v_0\> = -\<v_0, v_1\>, & \\
    \<v_0, v_1, v_2\>, && \Boundary \<v_0, v_1, v_2\> &= \<v_0, v_1\> + \<v_1, v_2\> + \<v_2, v_0\>, \\
    && &= + \<\hat v_0, v_1, v_2\> - \<v_0, \hat v_1, v_2\> + \< v_0, v_1, \hat v_2\>, \\
\end{math}
Es gilt
\begin{math}
    \<v_1, v_2, v_0\> &= \<v_0, v_1, v_2\>, \\
    \<v_0, v_2, v_1\> &= -\<v_0, v_1, v_2\>,
\end{math}
Tetraeder:
\begin{math}
    \Boundary\<v_0, v_1, v_2, v_3\>
    = \<v_1, v_2, v_3\> - \<v_0, v_2, v_3\> + \<v_0, v_1, v_3\> - \<v_0, v_1, v_2\>.
\end{math}

Konstruktion:

Sei $R$ ein kommutativer Ring mit Eins, etwa $R = \Z$, oder $\Z / p$, $\Q$, $\R$, $\C$.
\begin{math}
    \tilde K_n &:= \Set{(v_0, \dotsc, v_n) \in \Omega^{n+1} & \Set{v_0, \dotsc, v_n} \in K}, \\
    R\tilde K_n &:= \bigoplus_{\sigma \in \tilde K_n} R\sigma \qquad \text{freier $R$-Modul über $\tilde K_n$} \\
    &= R^{(\tilde K_n)} \\
    D_n := D_n K &:= R( \Set{(v_0, \dotsc, v_i, v_{i+1}, \dotsc, v_n) & v_i = v_{i+1}} \\
    & \qquad \cup \Set{(v_0, \dotsc, v_i, v_{i+1}, \dotsc, v_n) + (v_0, \dotsc, v_{i+1}, v_i, \dotsc, v_n)})
\end{math}

\begin{df}
    Wir setzen
    \begin{math}
        C_n := C_n K := C_n(K, R) := R\tilde K_n / D_n K
    \end{math}
    mit Quotientenabbildung
    \begin{math}
        \quot: R\tilde K_n &\to C_n \\
        (v_0, \dotsc, v_n) &\mapsto \<v_0, \dotsc, v_n\>.
    \end{math}
\end{df}

\begin{prop}
    \begin{itemize}
        \item
            $\<v_0, \dotsc,v_i, v_{i+1}, \dotsc, v_n\> = - \<v_0, \dotsc, v_{i+1}, v_i, \dotsc, v_n\>$,
        \item
            $\<v_{\pi 0}, v_{\pi 1}, \dotsc, v_{\pi n}\> = \sign(\pi) \<v_0, v_1, \dotsc, v_n\>$,
        \item
            $\<v_0, \dotsc, v_n\> = 0$ für $v_i = v_j$ mit $i \neq j$.
    \end{itemize}
\end{prop}

\begin{conv}
    Setze $C_n = 0$ für $n < 0$.
\end{conv}

\begin{df}
    Setze $\Boundary_i := \Boundary_i^{(n)}: \tilde K_n \to \tilde K_{n-1}$ als
    \begin{math}
        \del_i: (v_0, \dotsc, v_n) \mapsto (v_0, \dotsc, \hat v_i, \dotsc, v_n) := (v_0, \dotsc, v_{i-1}, v_{i+1}, \dotsc, v_n).
    \end{math}
    Definiere damit $\Boundary = \Boundary^{(n)}: R \tilde K_n \to R \tilde K_{n-1}$ linear durch
    \begin{math}
        \Boundary := \sum_{i=0}^n (-1)^i \Boundary_i:
        (v_0, \dotsc, v_n) \mapsto \sum_{i=0}^n (-1)^i (v_0, \dotsc, \hat v_i, \dotsc, v_n).
    \end{math}
\end{df}

\begin{prop}
    $\Boundary(D_n) \subset D_{n-1}$
    \begin{proof}
        Nachrechnen.
    \end{proof}
\end{prop}

Also
\begin{math}
    \begin{tikzcd}
        R\tilde K_n \ar[r,"\del"] \ar[d,"\quot"] \ar[dr] & R \tilde K_{n-1} \ar[d,"\quot"] \\
        C_n K \ar[r,dashed,"\exists! \del"] & C_{n-1} K
    \end{tikzcd}
\end{math}
Dank voriger Proposition ist also $\del$ wohldefiniert auf $C_n K$:
\begin{math}
    \<v_0, \dotsc, v_n\>
    \mapsto \sum_{i=0}^n (-1)^i \<v_0, \dotsc, \hat v_i, \dotsc, v_n\>.
\end{math}

\begin{lem}
    Die Komposition $C_{n+1} K \xto{\Boundary^{n+1}} C_n K \xto{\Boundary^{n}} C_{n-1} K$ ist Null, kurz: $\Boundary^2 = 0$.
    \begin{proof}
        Nachrechnen liefert
        \begin{math}
            \Boundary^2(v_0, \dotsc, v_{n+1})
            &= \Boundary \sum_{i=0}^{n+1} (-1)^i (v_0, \dotsc, \hat v_i, \dotsc, v_{n+1}) \\
            &= \sum_{i=0}^{n+1} (-1)^i \sum_{j=0}^{i-1} (-1)^j (v_0, \dotsc, \hat v_j, \dotsc, \hat v_i, \dotsc, v_{n+1}) \\
            &\qquad + \sum_{i=0}^{n+1} (-1)^{i} \sum_{j=i+1}^{n+1} (-1)^{j-1} (v_0, \dotsc, \hat v_i, \dotsc, \hat v_j, \dotsc, v_{n+1}) \\
            &= \sum_{i>j} (-1)^{i+j} (v_0, \dotsc, \hat v_j, \dotsc, \hat v_i, \dotsc, v_{n+1}) \\
            &\qquad + \sum_{i<j} (-1)^{i+j-1} (v_0, \dotsc, \hat v_i, \dotsc, \hat v_j, \dotsc, v_{n+1}) \\
            &= 0.
        \end{math}
    \end{proof}
    \begin{note}
        Geometrische Analogie für Mannigfaltigkeiten: $\Boundary^2 M = \emptyset$.
    \end{note}
\end{lem}


\begin{df}
    Ein \emphdef{Kettenkomplex} über $R$ ist eine Familie $(C_n, \Boundary_n)_{n \in \Z}$ von $R$-Moduln $C_n$ und $R$-Homomorphismen $\Boundary_n: C_n \to C_{n-1}$ mit $\Boundary \circ \Boundary = 0$.

    Typischerweise ist $C_k = 0$ für $k < 0$, also
    \begin{math}
        \begin{tikzcd}
            0 & C_0 \ar[l] & C_1 \ar[l,"\del_1"] & C_2 \ar[l,"\del_2"] & \dotsc \ar[l,"\del_3"]
        \end{tikzcd}
    \end{math}
    Es gilt stets $\im(\del_{n+1}) \subset \ker(\del_n)$.
    Wir bezeichnen \emphdef[Zykel]{$n$-Zykel} $Z_n := Z_n(C_*, \del_*) := \ker(\del_n)$ und \emphdef[Ränder]{$n$-Ränder} $B_n := B_n(C_*, \del_*) := \im(\del_{n+1})$.

    Die \emphdef{$n$-Homologie} ist definiert als
    \begin{math}
        H_n := H_n(C_*, \Boundary_*) := Z_n / B_n.
    \end{math}
    Für einen Zykel $z \in Z_n$ gilt $\Boundary z = 0$ und wir nennen $[z] := z + B_n \in H_n$ die \emphdef{Homologieklasse} von $z$.
    Entsprechend nennen wir zwei Zykel $z, z' \in Z_n$ \emphdef{homolog}, geschrieben $z \sim z'$, falls $z - z' = \del b_{n+1} \in B_n$.
\end{df}

\begin{ex}
    Skizze: Triangulierter Kreisring begrenzt durch inneres Dreieck $z$ (negativ orientiert) und äußeres Viereck $z'$ (positiv orientiert), Dreiecke jeweils positiv orientiert.
    \begin{math}
        z - z' &= \Boundary b, \\
        b &= \sigma_1 + \dotsb + \sigma_7.
    \end{math}
\end{ex}

\begin{ex}
    Erste Beispiele zum Ausrechnen:
    \begin{itemize}
        \item
            $K$:
            \begin{tikzpicture}[every node/.style={draw,fill,circle,inner sep=0.5pt},baseline]
                \draw (0,0) node["$v_0$"] (0) {};
            \end{tikzpicture}
            \begin{math}
                \begin{tikzcd}
                    0 & R \<v_0\> \ar[l,"\del_0"'] & 0 \ar[l,"\del_1"'] & \dotsc \ar[l]
                \end{tikzcd}
            \end{math}
            Homologie:
            \begin{math}
                H_n = \begin{cases}
                    R[v_0] \isomorphic R & \text{für $n = 0$,} \\
                    0 & \text{für $n \neq 0$.}
                \end{cases}
            \end{math}
        \item
            $K$:
            \begin{tikzpicture}[every node/.style={draw,fill,circle,inner sep=0.5pt},baseline]
                \draw (0,0) node["$v_0$"] (0) {}
                      (1,0) node["$v_1$"] (1) {} edge[<-] (0);
            \end{tikzpicture}
            \begin{math}
                \begin{tikzcd}[row sep=tiny]
                    0 & R\<v_0\> \oplus R\<v_1\> \ar[l,"\del_0"'] & R\<v_0, v_1\> \ar[l,"\del_1"'] & 0 \ar[l,"\del_2"'] & \dotsc \ar[l] \\
                    & \<v_1\> - \<v_0\> & \<v_0, v_1\> \ar[l,|->]
                \end{tikzcd}
                %0 \xto* R\<v_0\> \oplus R\<v_1\> &\xto* R\<v_0,v_1\> \xto* 0 \xto* \dotsc \\
                %\<v_1\> - \<v_0\> &\mapsfrom \<v_0, v_1\>
            \end{math}
            Homologie:
            \begin{math}
                Z_0 &= R\<v_0\> \oplus R\<v_1\>, \\
                B_0 &= R(\<v_1\> - \<v_0\>), \\
                H_0 &= Z_0 / B_0 \isomorphic R[v_0] = R[v_1] \isomorphic R,
            \end{math}
            da
            \begin{math}
                \begin{tikzcd}
                    H_0 \ar[r,"\homeomorphic"] & R \\
                    Z_0 = C_0 \ar[u] \ar[ru,"{\<v_0\>,\<v_1\> \mapsto 1}"']
                \end{tikzcd}.
            \end{math}
            Also zusammen mit $H_1 = 0$ wegen $\ker \del_1 = 0$ haben wir
            \begin{math}
                H_n &= \begin{cases}
                    R & \text{für $n = 0$}, \\
                    0 & \text{sonst}.
                \end{cases}
            \end{math}
        \item
            $K$:
            \begin{tikzpicture}[every node/.style={draw,fill,circle,inner sep=0.5pt},baseline]
                \draw (0,0) node["$v_0$"] (0) {}
                      (1,0) node["$v_1$"] (1) {} edge[<-] (0)
                      (2,0) node["$v_2$"] (2) {} edge[<-] (1);
            \end{tikzpicture}
            \begin{math}
                \begin{tikzcd}
                    0 & \bigoplus_{k=0}^2 R \<v_k\> \ar[l,"\del_0"'] & R \<v_0, v_1\> \oplus R\<v_1, v_2\> \ar[l,"\del_1"'] & 0 \ar[l,"\del_2"'] & \dotsc \ar[l]
                \end{tikzcd}
            \end{math}
            $H_n$ wie zuvor.
        \item
            $K$:
            \begin{tikzpicture}[every node/.style={draw,fill,circle,inner sep=0.5pt},baseline]
                \draw (210:0.5) node["$v_0$" left] (0) {}
                      (330:0.5) node["$v_1$" right] (1) {} edge[<-] (0)
                      (90:0.5) node["$v_2$"] (2) {} edge[<-] (1) edge[->] (0);
            \end{tikzpicture}
            \begin{math}
%                \resizebox{\textwidth}{!}{
                \begin{tikzcd}
                    0 & \bigoplus_{k=0}^2 R \<v_k\> \ar[l,"\del_0"'] & R \<v_0, v_1\> \oplus R\<v_1, v_2\> \oplus R\<v_2, v_0\> \ar[l,"\del_1"'] & 0 \ar[l,"\del_2"'] & \dotsc \ar[l]
                \end{tikzcd}
%                }
            \end{math}
            \begin{math}
                C_0 &= R\<v_0\> \oplus R\<v_1\> \oplus R \<v_2\> \\
                C_1 &= R\<v_0, v_1\> \oplus R\<v_1, v_2\> \oplus R\<v_2, v_0\> \\
                C_n &= 0 \quad \text{für $n \ge 2$}
            \end{math}
            \begin{math}
                Z_0 &= C_0, B_0 = R\Set{\<v_1\> - \<v_0\>, \<v_2\> - \<v_1\>, \underbrace{\<v_0\> - \<v_2\>}_{\mathclap{\text{redundant}}}}, \\
                H_0 &= R[v_0] = R[v_1] = R[v_2].
            \end{math}
            $Z_1$ wird frei erzeugt von $z := \Boundary\<v_0,v_1,v_2\>$ (nutze allgemeinen Ansatz mit Elementen aus $C_1$)
            \begin{math}
                Z_1 &= R z = R(\<v_0,v_1\> + \<v_1,v_2\> + \<v_2,v_0\>),&
                B_1 &= 0, \\
                H_1 &= Z_1 / B_1 = R[z] \isomorphic R.
            \end{math}
            Also
            \begin{math}
                H_n(\Delta, R) = \begin{cases}
                    R[v_0] & \text{für $n = 0$}, \\
                    R[z] & \text{für $n = 1$}, \\
                    0 & \text{sonst}.
                \end{cases}
            \end{math}
    \end{itemize}
\end{ex}


\begin{st}
    Für jeden Simplizialkomplex $K$ gilt
    \begin{math}
        H_0(K, R) &\xto{\isomorphic} R \pi_0(K), \\
        \<v_0\> + B_0 = [v_0] &\mapsto [v_0] = \Set{w \in \Omega & \text{$\exists$ Weg $v_0 \dotsc v_n = w$}}.
    \end{math}
    Ist $K$ zusammenhängend, so gilt
    \begin{math}
        \begin{tikzcd}
            H_1(K, \Z) \ar[r,"\isomorphic"] & \pi_1(K,x_0)_{\text{ab}}.
        \end{tikzcd}
    \end{math}
    Für beliebige $K$ folgt dann
    \begin{math}
        \begin{tikzcd}
            H_1(K, \Z) \ar[r,"\isomorphic"] & \bigoplus_{[x_0] \in \pi_0(K)} \pi_1([x_0], x_0)_{\text{ab}}.
        \end{tikzcd}
    \end{math}
    \begin{proof}
        Siehe Übung

        Skizze (nicht ganz korrekt): Wähle einen Spannbaum $T \subset K$
        \begin{math}
            \begin{tikzcd}
                H_1(K, \Z) \ar[r,"h","\isomorphic"'] & \pi_1(K, x_0)_{\text{ab}} \\
                Z_1(K, \Z) \ar[u,"\quot"] \ar[ru] \ar[d,"\subset"] & \pi_1(K, x_0) \ar[u] \\
                C_1(K, \Z) \ar[ruu] & \<S \mathrel{|} R\> \ar[u,"\isomorphic"] \ar[l] \\
                \Z \tilde K_1 \ar[u,"\quot"] \ar[ruuu,"\tilde h"] & \<S \mathrel{|} -\> \ar[u] \ar[l,"\tilde k"] \ar[lu]
            \end{tikzcd}
        \end{math}
        mit
        \begin{math}
            \tilde h(v_0,v_1)
            &:= [\underbrace{x_0 \dotsc v_0}_{\text{in $T$}} \ast v_0v_1 \ast \underbrace{v_1 \dotsc x_0}_{\text{in $T$}}] \\
            \tilde k(v_0v_1) &:= (v_0,v_1)
        \end{math}
        $S$: Kanten, $R$: Dreiecke.
    \end{proof}
\end{st}

\begin{lem}
    Jede simpliziale Abbildung $f: K \to L$ induziert $f_*$:
    \begin{math}
        \begin{tikzcd}
            R \tilde K_n \ar[r,"f_n"] \ar[d,"\quot"] \ar[dr] & R \tilde L_n \ar[d] \\
            C_n K \ar[r,dashed,"f_n"] & C_n L
        \end{tikzcd}
    \end{math}
    mit
    \begin{math}
        f_n: (v_0, \dotsc, v_n) &\mapsto (f(v_0), \dotsc, f(v_n)) \\
        \<v_0, \dotsc, v_n\> &\mapsto \<f(v_0), \dotsc, f(v_n)\> \\
    \end{math}
    Es gilt $f_{n-1} \circ \Boundary_n^K = \Boundary_n^L \circ f_n$, kurz $f \circ \Boundary = \Boundary \circ f$ d.h.
    \begin{math}
        \begin{tikzcd}
            0 & C_0 K \ar[l] \ar[d,"f_0"] & C_1 K \ar[l,"\Boundary_1^K"] \ar[d,"f_1"] & C_2 K \ar[l,"\Boundary_2^K"] \ar[d,"f_2"] & \dotsc\ar[l] \\
            0 & C_0 L \ar[l] & C_1 L \ar[l,"\Boundary_1^L"] & C_2 L \ar[l,"\Boundary_2^L"] & \dotsc\ar[l]
        \end{tikzcd}
    \end{math}
    $f$ nennt man einen \emphdef{Kettenhomomorphismus}.
\end{lem}

\begin{prop}
    Jeder Kettenhomomorphismus $g: (C_*, \Boundary_*) \to (C_*', \Boundary_*')$ induziert
    \begin{math}
        g_n = H_n(g): H_n(C_*, \Boundary_*) & \to H_n(C_*', \Boundary_*') \\
        [z] &\mapsto [g(z)].
    \end{math}
    \begin{proof}
        Wohldefiniertheit:
        \begin{enumerate}[i)]
            \item
                $g_n(Z_n) \subset Z_n'$: $\Boundary z = 0 \implies \Boundary(g(z) = g(\Boundary z) = g(0) = 0$.
            \item
                $g_n(B_n) \subset B_n'$: $z-z' = \Boundary b \implies g(z) - g(z') = g(\Boundary b) = \Boundary g(b)$.
        \end{enumerate}
    \end{proof}
\end{prop}

Kurzum: Wir erhalten Funktoren
\begin{math}
    \Cat{SComp} &\to \Cat{CComp}_R, \\
    K &\mapsto (C_* K, \Boundary_*), \\
    (f: K \to L) &\mapsto (f_*: C_* K \to C_* L).
\end{math}
\begin{math}
    \Cat{CComp}_R &\to \Cat{GrMod}_R, \\
    (C_n, \Boundary_n)_{n\in \Z} &\mapsto H_n(C_*, \Boundary_*)_{n\in \Z}, \\
    (f_n: C_n \to C_n')_{n\in \Z} &\mapsto H_n(f)_{n\in \Z}.
\end{math}

\begin{st}
    Sei $K = \Set{v} \ast L$ der Kegel über $L$ mit Spitze $v$, d.h.
    \begin{math}
        K = L \cup \Set{\Set{v} \sqcup \sigma & \sigma \in L}.
    \end{math}
    Dann gilt
    \begin{math}
        H_n K = H_n \Set{v}
        = \begin{cases}
            R[v_0] & \text{für $n = 0$}, \\
            0 & \text{sonst}.
        \end{cases}
    \end{math}
    \begin{proof}
        Definiere $T_n: C_n K \to C_{n+1} K$ durch $\<v_0, \dotsc, v_n\> \mapsto \<v, v_0, \dotsc, v_n\>$.
        Für $n \ge 1$ gilt
        \begin{math}
            \Boundary_{n+1} T_n\<v_0, \dotsc, v_n\>
            &= \<v_0, \dotsc, v_n\> - \sum_{i=0}^n (-1)^i \<v, v_0, \dotsc, \hat v_i, \dotsc, v_n\> \\
            &= \id \<v_0, \dotsc, v_n\> - T_{n-1} \Boundary_n \<v_0, \dotsc, v_n\>,
        \end{math}
        Für $n = 0$ gilt $\Boundary_1 T_0 \<v_0\> = \Boundary_1 \<v, v_0\> = \<v_0\> - \<v\>$.

        Setze $g: K \to K$, $g(\Omega) = \Set{v}$ und $g_*: C_* K \to C_* K$, $\<v_0,\dotsc, v_n\> = \<g(v_0), \dotsc, g(v_n)\>$.
        Damit gilt
        \begin{math}
            \del T + T \del = \id - g.
        \end{math}
    \end{proof}
\end{st}

\Timestamp{2016-01-22}

\begin{df}
    Seien $(C_n, \del_n)_{n\in\Z}$ und $(C_n', \del_n')_{n \in \Z}$ zwei Kettenkomplexe und $f, g$ Kettenhomomorphismen, d.h. $f_n, g_n: C_n \to C_n'$ mit $f_{n-1} \circ \del_{n} = \del_n' \circ f_n$ und $g\circ \del = \del' \circ g$.

    Eine \emphdef{Kettenhomotopie} $T: f \sim g$ ist eine Familie von Abbildungen $T_n: C_n \to C_{n+1}'$, $n \in \Z$ mit $\del_{n+1}' T_n + T_{n-1} \del_n = f_n - g_n$.
    \begin{math}
        \begin{tikzcd}[column sep=large]
            ~ & C_{n-1} \ar[l] \ar[d,"f_{n-1}"',shift right] \ar[d,"g_{n-1}",shift left] \ar[dr,"T_{n-1}"] & C_n \ar[dr,"T_n"] \ar[l,"\Boundary_n"'] \ar[d,"f_n"',shift right] \ar[d,"g_n",shift left] & C_{n+1} \ar[l,"\Boundary_{n+1}"'] \ar[d,"f_{n+1}"',shift right] \ar[d,"g_{n+1}",shift left] & ~ \ar[l] \\
            ~ & C_{n-1}' \ar[l] & C_n' \ar[l,"\Boundary_n'"] & C_{n+1} \ar[l,"\Boundary_{n+1}'"] & ~ \ar[l]
        \end{tikzcd}
    \end{math}
\end{df}

\begin{prop}
    Aus $T: f \sim g$ folgt $H_*(f) = H_*(g)$, also
    \begin{math}
        H_n(f) = H_n(g): H_n(C) \to H_n(C')
    \end{math}
    \begin{proof}
        Sei $z \in Z_n$, d.h. $z \in C_n$ und $\del z = 0$.
        Es gilt
        \begin{math}
            f_n(z) - g_n(z)
            = \underbrace{\del T z}_{\in B_n} + T \underbrace{\del z}_{=0}
            \in B_n
        \end{math}
        Also gilt $H_*(f)[z] = [f(z)] = [g(z)] = H_*(g)[z]$.
        %$[(f_n - g_n)(z)] = 0$
        %\begin{math}
        %    (f-g)(z)
        %    = (\Boundary T + T \Boundary)(z)
        %    = \Boundary T z
        %\end{math}
        %Also $f(z) \sim g(z)$.
    \end{proof}
\end{prop}

\begin{nt}[Folgerung für Kegel]
    Für einen Kegel $K = \Set{v} \ast L$ ist
    \begin{math}
        H_*(\id_K) = H_*(g): H_*(K) \to H_*(K).
    \end{math}
    Nun ist
    \begin{math}
        \im H_*(g) = \begin{cases}
            R\<v\> & \text{für $n = 0$}, \\
            0 & \text{für $n \neq 0$}.
        \end{cases}
    \end{math}
    Also $H_* K = H_* \Set{v}$.

    Vornehmer: Sei $M = \Set{\emptyset, \Set{v}}$ Simplizialkomplex mit $\Omega = \Set{v}$.
    Wir haben $i: M \xto[injective] K$, also ist $i_*: C_* M \xto[injective] C_* K$ ein Kettenhomomorphismus.
    Zudem haben wir eine Retraktion $r: K \to M$, also $r_*: C_* K \to C_* M$ ein Kettenhomomorphismus.
    Es gilt.
    \begin{math}
        r \circ i &= \id_M &&\implies &  r_* \circ i_* &= \id_{C_* M}, \\
        i \circ r &= g &&\implies & i_* \circ r_* &= g_* \sim \id_{C_* K}
    \end{math}
    Kurz:
    \begin{math}
        \begin{tikzcd}
            C_* M  \ar[r,inj,"i_*",shift left] & C_* K \ar[l,"r_*",shift left]
        \end{tikzcd}
    \end{math}
    ist ein Homotopieretrakt.

    Auf Homologieniveau $H_*(r) \circ H_*(i) = \id_{H_* M}$ und $H_*(i) \circ H_*(r) = \id_{H_* K}$ folgt Isomorphie
    \begin{math}
        \begin{tikzcd}
            H_* M \ar[r,"H_*(i)",shift left,"\isomorphic"'] & H_* K \ar[l,"H_*(r)",shift left]
        \end{tikzcd}
    \end{math}
\end{nt}

\begin{ex}
    Sei $\Delta^n$ ein $n$-Simplex.
    Es gilt
    \begin{math}
        \Delta^n = \scr P(\Set{v_0, \dotsc, v_n})
        = \Delta^{n-1} \ast \Set{v_n}
    \end{math}
    Damit ist
    \begin{math}
        H_q(\Delta^n) = H_q(\Set{v_n})
        = \begin{cases}
            R = R[v_n]  & \text{für $q = 0$}, \\
            0 & \text{sonst}.
        \end{cases}
    \end{math}
\end{ex}

\begin{ex}
    Berechne für den Rand eines Simplex $H_q(\Boundary \Delta^{n+1})$.
    \begin{math} % fixme: isomorphismen, 0
        \begin{tikzcd}[column sep=small]
            0 & C_0(\Boundary \Delta^{n+1}) \ar[l,"\del'"] \ar[d,"i"] & \dotsc \ar[l,"\del'"] & C_{n-1}(\Boundary \Delta^{n+1}) \ar[l,"\del'"] & C_{n}(\Boundary \Delta^{n+1}) \ar[l,"\del'"] & C_{n+1}(\Boundary \Delta^{n+1}) \ar[l,"\del'"] & 0 \\
            0 & C_0( \Delta^{n+1}) \ar[l,"\del"] & \dotsc \ar[l,"\del"] & C_{n-1}(\Delta^{n+1}) \ar[l,"\del"] & C_{n}( \Delta^{n+1}) \ar[l,"\del"] & C_{n+1}( \Delta^{n+1}) \ar[l,"\del"] & 0 \\
        \end{tikzcd}
    \end{math}
    Setze $b = \<v_0, \dotsc, v_n, v_{n+1}\>$.
    Setze $z := \del b$, dann ist $\del z = \del \del b = 0$.
    Also ist $z \in C_n(\Delta^{n+1})$ ein Zykel.
    Ebenso $i_n^{-1}(z) =: z' \in C_n(\Boundary \Delta^{n+1})$, denn
    \begin{math}
        i_{n-1} \del z'
        = \del_{n-1} i_{n-1} z'
        = \del_{n-1} z
        = 0
    \end{math}
    Also $\del_n' z' = 0$ und
    \begin{math}
        H_n(\Boundary \Delta^{n+1}) =
        \begin{cases}
            R[\<v_0\>] & \text{für $q = 0$}, \\
            R[z] & \text{für $q = n$}, \\
            0 & \text{für $q \not\in \Set{0,n}$}.
        \end{cases}
    \end{math}
\end{ex}


\section{Relative Homologie}


Sei $A \subset K$ ein Teilkomplex, z.B. $A = \Boundary K$.
$i: A \injto K$ induziert $i_*: C_* A \injto C_* K$, kurz: $C_* A \subset C_* K$.

Wir setzen $C_q(K, A) := C_q K / C_q A$ als Quotientenmodul.
Randoperator
\begin{math}
    \_\del_q: C_q(K, A) &\to C_{q-1}(K,A), \\
    c + C_q A &\mapsto \del_q c + C_{q-1} A.
\end{math}
Wir erhalten den Kettenkompleex $(C_q(K, A), \_\del_q)_{q \in \Z}$.
Seine Homologie bezeichnen wir mit $H_*(K, A)$, gesprochen: \emphdef[Homologie!relative]{Homologie von $K$ relativ zu $A$}.

\begin{ex}
    $K$ ausgefülltes Dreieck, $A = \Boundary K$.

    Für $z = \<v_1, v_2\> - \< v_0, v_2\> + \<v_0,v_1\>$ dann
    \begin{math}
        H_q K &= \begin{cases}
            R\<v_0\> & \text{für $q = 0$}, \\
            0 & \text{für $q \neq 0$},
        \end{cases} \\
        H_q A &= \begin{cases}
            R\<v_0\> & \text{für $q = 0$}, \\
            R[z] & \text{für $q = 1$},
        \end{cases} \\
        H_q(K, A) &= \begin{cases}
            0 & \text{für $q = 0$}, \\
            0 & \text{für $q = 1$}, \\
            R\<v_0, v_1, v_2\>& \text{für $q = 2$}, \\
            0 & \text{für $q \ge 3$}.
        \end{cases}
    \end{math}
    Denn $C_2(K, A) = C_2(K) = R\<v_0, v_1, v_2\>$, $\<v_0, v_1, v_2\>$ ist Zykel.
\end{ex}

Allgemein: Sei $0 \to (C', \del') \to (C, \del) \to (C'', \del'') \to 0$ eine kurze exakte Sequenz von Kettenkomplexen, d.h.
\begin{math}
    \begin{tikzcd}
        0 \ar[r] & C_{q+1}' \ar[r,"i"] \ar[d,"\del'"] & C_{q+1} \ar[r,"p"] \ar[d,"\del"] & C_{q+1}'' \ar[d,"\del''"] \ar[r] & 0 \\
        0 \ar[r] & C_{q}' \ar[r,"i"] \ar[d,"\del'"] & C_{q} \ar[r,"p"] \ar[d,"\del"] & C_{q}'' \ar[r] \ar[d,"\del''"] & 0 \\
        0 \ar[r] & C_{q-1}' \ar[r,"i"] & C_{q-1} \ar[r,"p"] & C_{q-1}'' \ar[r] & 0
    \end{tikzcd}
\end{math}
und jede Zeile ist eine kurze exakte Sequenz.

\begin{lem}
    Dies induziert
    \begin{math}
        H_* C' \xto{i_*} H_* C \xto{p_*} H_* C''
    \end{math}
    mit $\im i_* = \ker p_*$.
    \begin{note}
        Im Allgemeinen gilt Exaktheit also nur in der Mitte, nicht links oder rechts, siehe voriges Beispiel.
    \end{note}
    \begin{proof}
        $\subset$ ist klar, denn es gilt $p \circ i = 0$, also $p_* \circ i_* = 0$ dank Funktorialität.

        Zeige also $\supset$.
        Sei $[z] \in \ker p_*$, d.h. $z \in C_q$, $\del z = 0$ und $p_*[z] = 0$.
        Zu zeigen: es existiert $y \in C_q'$, $\del y = 0$, $i_*[y] = [z]$.

        Es gilt $p_q z \in B_q'$, d.h. $\exists x''. \in C_{q+1}'' : \del x'' = p_q z$.

        Es existiert $x \in C_q: p_{q+1} x = x''$, also
        \begin{math}
            p_q(z-\del x)
            = p_q z - p_q \del x
            = p_q z - \del'' p_{q+1} x
            = p_q z - \del x''
            = 0,
        \end{math}
        d.h. $z - \del x \in \ker p_q$.
        Also existiert $y \in C_q': i_q y = z - \del x$.
        Es gilt
        \begin{math}
            i_{q-1} \del' y
            = \del i_q y
            = \del (z - \del x)
            = \del z - \del\del x
            = 0 - 0
            = 0
        \end{math}
        Da $\ker i_{q-1} = 0$ folgt $\del y = 0$ und es gilt
        \begin{math}
            i_*[y]
            = [i_q y]
            = [z - \del x]
            = [z].
        \end{math}
    \end{proof}
\end{lem}

Wir definieren
\begin{math}
    \del_q: H_q C'' \to H_{q-1} C', \\
    [z] \mapsto [\del z]
\end{math}
Genauer: Sei $z \in C_q''$ mit $\del'' z = 0$.
Es existiert $y \in C_q$ mit $p_q y = z$.
\begin{math}
    p_{q-1} \del y
    = \del p_q y
    = \del q
    = 0,
\end{math}
d.h. $\del y \in \ker(p_{q-1})$, also existiert genau ein $x \in C_{q-1}': i_{q-1} x = \del y$.
Setze nun $\del_q[z] := [x]$.

Zunächst ist $x$: $i_{q-2} \del x = \del i_{q-1} = \del \del y = 0$, da $\ker(i_{q-2}) = 0$ folgt $\del x = 0$, also $x \in Z_{q-1}'$.
Somit definiert $x$ eine Homologieklasse $[x] \in H_{q-1} C'$.
Zeige: $[x]$ ist wohldefiniert.
\begin{itemize}
    \item
        Sei $\tilde y \in C_q: p_q \tilde y = z$, also $p_q(\tilde y - y) = 0$, es existiert $\tilde w \in C_q'$ mit $i_q \tilde w = \tilde y - y$.
        Es existiert genau ein $\tilde x \in C_{q-1}'$ mit $i_{q-1} \tilde x = \del \tilde y$.
        Insgesamt
        \begin{math}
            i_{q-1} \del \tilde w
            = \del i_{q+1} \tilde w
            = \del(\tilde y - y)
            = \del\tilde y - \del y
            = i_{q-1}\tilde x - i_{q-1} x
            = i_{q-1}(\tilde x - x).
        \end{math}
        Also $\tilde x - x = \del \tilde w \in B_{n-1}'$, also $[\tilde x] = [x]$.
    \item
        Sei $\tilde z \sim z$, d.h. $\tilde z - z = \del'' w \in B_q''$.
        Es existiert $u \in C_{q+1}$ mit $p_{q+1} u = w$.
        Betrachte $\tilde y := y + \del u$ und erhalte
        \begin{math}
            p_q(\tilde y)
            = p_q y + p_q \del u
            = z + \del p_{q+1} u
            = z + \del w
            = z + \tilde z - z
            = \tilde z.
        \end{math}
        Die Konstruktion mit $\tilde y$ statt $y$ liefert das selbe $x$.
\end{itemize}

\begin{ex}
    $K$ ausgefülltes Dreieck, $A = \Boundary K$.
    \begin{math}
        C' = C(A) \xto{i} C = C(K) \xto{p} C'' = C(K) / C(A), \\
        H' = H(A) \xto{i_*} H = H(K) \xto{p_*} H'' = H(K) / H(A).
    \end{math}
    Wir haben
    \begin{math}
        \begin{tikzcd}
            H_2' \ar[r,"i_*"] & H_2 \ar[r,"p_*"] & H_2'' \ar[dll,"\del_2"] \\
            H_1' \ar[r,"i_*"] & H_1 \ar[r,"p_*"] & H_1'' \ar[dll,"\del_1"] \\
            H_0' \ar[r,"i_*"] & H_0 \ar[r,"p_*"] & H_0'' \ar[r] & 0
        \end{tikzcd}
        \qquad
        \begin{tikzcd}
            0 \ar[r,"i_*"] & 0 \ar[r,"p_*"] & R[z] \ar[dll,"\del_2"] \\
            R[z] \ar[r,"i_*"] & 0 \ar[r,"p_*"] & 0 \ar[dll,"\del_1"] \\
            R\<v_0\> \ar[r,"i_*"] & R\<v_0\> \ar[r,"p_*"] & 0 \ar[r] & 0
        \end{tikzcd}
    \end{math}
\end{ex}

\begin{st}
    Jede kurze exakte Sequenz von Kettenkomplexen
    \begin{math}
        0 \to (C', \del') \xto{i} (C, \del) \xto{p} (C'', \del'')
    \end{math}
    induziert einen Randoperator $\del_q: H_q C'' \to C_{q-1} C'$ wie oben definiert.

    Die so entstehende lange Sequenz
    \begin{math}
        \dotsc \xto{i_*} H_q C' \xto{p_*} H_q C'' \xto{\del_*} H_{q-1} C' \xto{i_*} H_{q-1} C \xto{p_*} \dotsc
    \end{math}
    ist exakt.
    \begin{proof}
        \begin{enumerate}[1)]
            \item
                Die Exaktheit bei $H_q C$ wurde im Lemma gezeigt.
            \item
                Exaktheit in $H_q C''$, d.h. $\im p_q = \ker \del_q$:
                $\subset$ ist klar: aus $\del y = 0$ folgt $x = 0$, also $\del_q[z] = [0]$.

                Zeige $\supset$.
                Sei $z \in Z_q$ mit $\del_q[z] = [x] = 0$, d.h. $x = \del \tilde w \in B_{q-1}'$.
                Betrachte $\tilde y = y - i_q \tilde w$ und erhalte
                \begin{math}
                    p_q(\tilde y) = p_q(y) - \underbrace{p_q i_q \tilde w}_{=0},
                \end{math}
                sowie
                \begin{math}
                    \del \tilde y
                    = \del y - \del i_q \tilde w
                    = \del y - i_{q-1} \del \tilde w
                    = \del y + i_{q-1} x
                    = \del y - \del y
                    = 0.
                \end{math}
                D.h. $\tilde y$ ist ein Zykel
                \begin{math}
                    H_q C \ni [\tilde y]
                    \stack{p_*}{\mapsto}
                    [p_q \tilde y] = [p_q y] = [z].
                \end{math}
            \item
                Exaktheit in $H_{q-1}C'$, d.h. $\im \del_q = \ker i_{q-1}$.
                $\subset$ ist klar: zu $x = \del_q z$ wie im Diagramm ist $i_{q-1} x = \del y$, also $i_*[x] = [i_{q-1} x] = [\del y] = 0$.

                Zeige $\supset$: Übung.
        \end{enumerate}
    \end{proof}
\end{st}


\Timestamp{2016-01-29}

\section{Abschluss des Kapitels}


Wir haben einem Simplizialkomplex $K$ einen Kettenkomplex $C_*(K,R)$ zugeordnet und aus dieser die Homologie $H_*(K,R)$ bestimmt, sowie erste Eigenschaften nachgewiesen.

Im Folgenden weitere wichtige Eigenschaften (hier nur genannt, aber zunächst nicht bewiesen).

\begin{st}
    Für jede Unterteilung $K' \prec K$ haben wir einen Kettenhomomorphismus $C_* K \to C_* K'$ und dieser induziert einen Isomorphismus $H_* K \to H_* K'$.
\end{st}

\begin{st}
    Sind $f, g: K \to L$ homotop, d.h. $|f| \homotopic |g|: |K| \to |L|$, so folgt $H_*(f) = H_*(g): H_*(K) \to H_*(L)$.
\end{st}

\begin{kor}
    Jede stetige Abbildung $\phi: |K| \to |L|$ können wir simplizial approximieren durch $f: K' \to L$ mit $|f| \homotopic \phi$.
    Wir definieren $H_*(\phi) := H_*(f) : H_*(K) \xto{\isomorphic} H_*(K') \to H_*(L)$.

    Dies ist wohldefiniert, d.h. unabhängig von der Wahl der Approximation:
    \begin{proof}
        Ist $g: K'' \to K$ eine weitere Approximation zu $\phi$, betrachte Unterteilung $K'''$ von $K'$ und $K''$ und $f', g': K''' \to L$.
        Damit ist
        \begin{math}
            \begin{tikzcd}
                H_*(K) \ar[r,"\isomorphic"] & H_*(K''') \ar[r,shift left,"H_*(f')"] \ar[r,shift right,"H_*(g')"'] & H_*(L)
            \end{tikzcd}.
        \end{math}
        Wegen $|f'| \homotopic \phi \homotopic |g'|$ sind also $H_*(f') = H_*(g')$.
    \end{proof}
\end{kor}

Gegeben sei ein topologischer Raum $X$ mit Triangulierung $h: |K| \xto{\homeomorphic} X$.

Dann können wir definieren
\begin{math}
    H_*(X) := H_*(K).
\end{math}
Das ist wohldefiniert bis auf Isomorphie
\begin{math}
    \begin{tikzcd}
        {|K|} \ar[r,"\phi","\isomorphic"'] & X & {|L|} \ar[l,"\psi"',"\isomorphic"] \\
        H_*(K) \ar[rr,shift left,"H_*(\psi^{-1} \circ \phi)"] && H_*(L) \ar[ll,shift left,"H_*(\phi^{-1} \circ \psi)","\isomorphic"']
    \end{tikzcd}.
\end{math}

\begin{ex}
    Sei $X = \S^n$, $|K| = \Boundary \Delta^{n+1}$, $K = \scr P \Set{0,1,\dotsc,n+1} \setminus \Set{\Set{0,1,\dotsc, n+1}}$.
    \begin{math}
        H_q(\S^n) := H_q(K)
        = \begin{cases}
            R & \text{für $q = 0$}, \\
            R & \text{für $q = n$}, \\
            0 & \text{sonst}.
        \end{cases}
    \end{math}
    Beachte für $\S^0$ ist $H_0(\S^0) = R \oplus R$ und $H_q(\S^0) = 0$ für $q \neq 0$.

    Zu jeder stetigen Abbildung $f: \S^n \to \S^n$ erhatlen wir $H_*(f): H_*(\S^n) \to H_*(S^n)$, insbesondere für $R = \Z$ in oberster Dimension
    \begin{math}
        \begin{tikzcd}
            H_n(\S^n, \Z) \ar[r,"H_*(f)"] \ar[d,"\isomorphic"] & H_n(\S^n, \Z) \ar[d,"\isomorphic"] \\
            \Z \ar[r,"\cdot a"] & \Z
        \end{tikzcd}
    \end{math}
    Wir definieren den Abbildungsgrad von $f$ durch $\deg(f) := a$.

    Eigenschaften:
    \begin{itemize}
        \item
            $f \homotopic g \implies \deg(f) = \deg(g)$,
        \item
            $f \homotopic \const \implies \deg(f) = 0$,
        \item
            $f \homotopic \id \implies \deg(f) = 1$,
        \item
            $f \homotopic -\id \implies \deg(f) = (-1)^{n+1}$,
            denn
            \begin{itemize}
                \item
                    $f \in \SO(\R^{n+1}) \implies f \homotopic \id$,
                \item
                    Punktsymmetrische Triangulierung wählen, Permutation der Ecken.
            \end{itemize}
        \item
            $\deg(f \circ g) = \deg(f) \deg(g)$,
        \item
            $f \circ g \homotopic \id \implies \deg(f) \deg(g) = 1 \implies \deg(f) = \deg(g) \in \Set{\pm 1}$.
    \end{itemize}
    Damit haben wir alle schönen Anwendungen aus der Grundvorlesung Topologie.
\end{ex}
