% Kapitel 1
\chapter{Fundamentalgruppe}

\begin{df}
    Sei $X$ ein topologischer Raum, $x_0 \in X$ ein Punkt.
    \begin{math}
        P(X) &= \scr C([0,1], X), \\
        P(X, a, b) &= \Set{\gamma \in P(X) & \gamma(0) = a, \gamma(1) = b}.
    \end{math}
    Spezielle Wege, bzw Wegoperationen:
    \begin{itemize}
        \item
            $1_a \in P(X, a, a)$, $1_a(t) := a$ für $0 \le t \le 1$,
        \item
            $\_\argdot: P(X, a, b) \to P(X, b, a)$, $\gamma \mapsto \_\gamma$, $\_\gamma(t) := \gamma(1-t)$,
        \item
            $\ast: P(X, a,b) \times P(X, b,c) \to P(X,a,c)$,
            \begin{math}
                (\gamma_1 \ast \gamma_2)(t) := \begin{cases}
                    \gamma_1(2t) & \text{für $0 \le t \le \frac{1}{2}$} \\
                    \gamma_2(2t - 1) & \text{für $\frac{1}{2} \le t \le 1$}
                \end{cases}
            \end{math}
    \end{itemize}
    Zwei Wege $\alpha, \alpha' \in P(X,a,b)$ heißen \emphdef{äquivalent}, genauer \emphdef{homotop bei festen Endpunkten}, wenn eine stetige Abbildung $H: [0,1] \times [0,1] \to X$ existiert mit $H(0, t) = \alpha(t)$, $H(1, t) = \alpha'(t)$, sowie $H(s, 0) = a$, $H(s, 1) = b$ für alle $s,t \in [0,1]$.
    Wir schreiben dann $H: \alpha \sim \alpha'$ oder kurz $\alpha \sim \alpha'$.
\end{df}

\begin{prop}
    \begin{itemize}
        \item
            Die Äquivalenz $\sim$ ist eine Äquivalenzrelation.
        \item
            Aus $\alpha \sim \beta$ folgt $\_\alpha \sim \_\beta$.
        \item
            Aus $\alpha \sim \alpha'$ und $\beta \sim \beta'$ folgt $\alpha \ast \beta \sim \alpha' \sim \beta'$.
    \end{itemize}
\end{prop}

Wir erhalten wohldefinierte Abbildungen auf $\Pi(X, a, b) := P(X, a, b) / \sim$ durch
\begin{itemize}
    \item
        $\_\argdot: \Pi(X,a,b) \to \Pi(X,a,b)$, $\_{[\gamma]} := [\_\gamma]$.
    \item
        $\ast: \Pi(X,a,b) \times \Pi(X,b,c) \to \Pi(X,a,c)$, $[\alpha] \ast [\beta] := [\alpha \ast \beta]$.
\end{itemize}

\begin{st}
    Jeder topologische Raum $X$ definiert so seine \emphdef{Wegekategorie} $\Pi(X)$ (auch \emphdef{Fundamentalgruppoid} genannt).
    \begin{enumerate}[a)]
        \item
            Objekte sind die punkte $a, b, c, \dotsc \in X$,
        \item
            Morphismen $[\alpha]: a \to b$ sind Wegeklassen von $a$ nach $b$.
        \item
            Verknüpfung $\ast$ ist die Konkatenation wie oben.
    \end{enumerate}
    Die Verknüpfung erfüllt
    \begin{enumerate}[1)]
        \item
            Identität: Für $\alpha: a \to b$ gilt
            \begin{math}
                1_a \ast \alpha \sim \alpha \sim \alpha \ast 1_b,
            \end{math}
            also
            \begin{math}
                [1_a] \sim [\alpha] = [\alpha] = [\alpha] \ast [1_b].
            \end{math}
        \item
            Inversion: Für $\alpha: a \to b$ und $\_\alpha: b \to a$ gilt
            \begin{math}
                \alpha \ast \_\alpha &\sim 1_a, &
                \_\alpha \ast \alpha &\sim 1_b
            \end{math}
            also
            \begin{math}
                [\alpha] \ast [\_\alpha] &= [1_a], &
                [\_\alpha] \ast [\alpha] &= [1_b],
            \end{math}
        \item
            Assoziativität: Für $a \xto{\alpha} b \xto{\beta} c \xto{\gamma} d$ gilt $(\alpha \ast \beta) \ast \gamma \sim \alpha \ast (\beta \ast \gamma)$, also
            \begin{math}
                ([\alpha] \ast [\beta]) \ast [\gamma] = [\alpha] \ast ([\beta] \ast [\gamma]).
            \end{math}
    \end{enumerate}
    \begin{proof}
        Skizzenbeweis.
    \end{proof}
\end{st}

\begin{st}
    Jede stetige Abbildung $f: X \to Y$ induziert einen Funktor
    \begin{math}
        f_\#: \Pi(X) &\to \Pi(Y) \\
        a &\mapsto f(a) \\
        [\alpha: a \to b] &\mapsto [f \circ \alpha: f(a) \to f(b)]
    \end{math}
\end{st}

\begin{df}
    Vom Fundamentalgruppoid zur Fundamentalgruppe, definiere
    \begin{math}
        \pi_1(X, x_0) := \Pi(X, x_0, x_0)
        = \frac{\Set{\text{Schleifen $\alpha: ([0,1], \Set{0,1}) \to (X,x_0)$}}}{\text{Homotopie relativ $\Set{0,1}$}}.
    \end{math}
    Dies ist eine Gruppe (Übung) mit der Verknüpfung $[\alpha] \ast [\beta] = [\alpha \ast \beta]$.

    Jede stetige Abbildung $f: (X, x_0) \to (Y, y_0)$ induziert einen Gruppenhomomorphismus $f_\# = \pi_1(f): \pi_1(X, x_0) \to \pi_1(Y, y_0)$ mit $[\alpha] \mapsto f_\#([\alpha]) = [f \circ \alpha]$.
    Wir erhalten einen Funktor
    \begin{math}
        \pi_1: \Cat{Top}_* &\to \Cat{Grp} \\
        (X, x_0) &\mapsto \pi_1(X, x_0) \\
        (f: (X,x_0) \to (Y, y_0)) &\mapsto (\pi_1(f): \pi_1(X, x_0) \to \pi_1(Y, y_0)).
    \end{math}
\end{df}

\begin{ex}
    Sei $X = \R^n$ oder $X \subset \R^n$ konvex oder sternförmig bezüglich $x_0$.
    Dann ist $X$ wegzusammenhängend, d.h. $\pi_0(X) = \Set{[x_0]}$, und sogar einfach zusammenhängend, d.h. zudem
    \begin{math}
        \pi_1(X, x_0) = \Set{[1_{x_0}]},
    \end{math}
    kurz $\pi_1(X, x_0) = \Set{1}$.
    \begin{proof}
        Zu $\alpha: [0,1] \to X$ betrachte $H(s, t) = (1-s)\alpha(t) + s x_0$.
        $H: [0,1] \times [0,1] \to X$ ist eine Abbildung, da $X$ sternförmig ist.
        Sie ist stetig und erfüllt $H: \alpha \sim 1_{x_0}$.
    \end{proof}
\end{ex}

\paragraph{Offene Mengen $X \subset \R^n$ und polygonale Fundamentalgruppe}

Sei $X \subset \R^n$ und $x_0 \in X$. Wir definieren die polygonale Fundamentalgruppe
\begin{math}
    \pi_1^{\text{pl}}(X,x_0) := \Pi^{\text{pl}}(X, x_0, x_0)
    = \frac{\Set{\text{geschlossene Polygonzüge in $(X, x_0)$}}}{\text{polygonale Homotopie in $X$}}.
\end{math}

\begin{st}
    Sei $X \subset \R^n$ offen und $x_0 \in X$.
    Wir haben einen Gruppenisomorphismus
    \begin{math}
        \phi: \pi_1^{\text{pl}}(X, x_0) \to \pi_1(X, x_0)
    \end{math}
    \begin{proof}
        (Skizze: stetiger/polygonaler Weg)
        Surjektivität: Jede stetige Abbildung lässt sich beliebig genau durch Polygone approximieren.
        Injektivität: Stetige Homotopie lässt sich durch polygonale Homotopie approximieren.
    \end{proof}
\end{st}

\begin{ex}
    Sei $X := \R^2 \setminus \Set{0}$, $x_0 := (1, 0)$ und $\gamma$ ein geschlossener polygonaler Weg in $X$ von $x_0$ (Skizze).
    Durch zählen der Übergänge über die negative reelle Achse erhalten wir $\deg: \pi_1^{\text{pl}}(X, x_0) \to \Z$.
    Dies ist ein Gruppenisomorphismus
    \begin{proof}
        Wohldefiniertheit: Übergang von Wegen zu Wegeklassen.
        Homomorphismus.
        Surjektivität: Konstruktion.
        Injektivität: Umlaufzahl $0$ betrachten: neg/pos Übergänge eliminieren (Punkte sternförmig um $x_0$).
    \end{proof}
    Kurz:
    \begin{math}
        \deg: \pi_1(X, x_0) \isomorphic \pi_1^{\text{pl}}(X, x_0) \isomorphic \Z.
    \end{math}
\end{ex}

\begin{ex}
    Sei $X := \C \setminus \Set{0, -1, \dotsc, 1 - n}$, $x_0 = 1$ (Skizze mit Weg).
    Kodiere Übergänge: $s_1, \dotsc, s_n$.

    Wir erhalten $\phi: \pi_1^{\text{pl}}(X, x_0) \to \Gen{ s_1, \dotsc, s_n & - }$.
    Dies ist ein Gruppenisomorphismus.
    \begin{proof}
        Wohldefiniertheit: Übergang von Wegen zu Wegeklassen: Kürzung.
        Homomorphismus: klar.
        Surjektivität: Konstruktion.
        Injektivität: Betrachte Wege, die auf $1$ abgebildet werden, Induktion über Wortlänge durch Kürzen.
    \end{proof}
    \begin{note}
        Die Gruppe ist für $n \ge 2$ nicht kommutativ!
    \end{note}
\end{ex}

\paragraph{Präsentation von Gruppen durch Erzeuger und Relationen}

Kurzfassung: Sei $A$ eine Menge. $A^* := \bigcup_{n \in \N} A^n$ ist die Menge aller Wörter über dem Alphabet $A$.
Für $n = 0$ ist $e = ()$ das leere Wort.
Für $n = 1$ identifizieren wir $(a) \in A^*$ mit $a \in A$.

Die Verkettung $\circ: A^* \times A^*$ ist gegeben durch die Konkatenation der Wörter
\begin{math}
    (a_1, \dotsc, a_n)(a_1', \dotsc, a_m') := (a_1, \dotsc, a_m, a_1', \dotsc, a_n').
\end{math}
Damit ist $(A^*, \circ, e)$ ein Monoid, genannt das \emphdef[freies Monoid]{freie Monoid} über $A$.

Wir wollen Relationen der Form $w_1 = w_2$ einführen.
Hierzu sei $K \subset A^* \times A^*$.
Auf $A^*$ sei $\equiv$ die Äquivalenzrelation, die erzeugt wird durch die elementaren Umformungen
\begin{math}
    u \circ w_1 \circ v \equiv u \circ w_2 \circ v, && \text{für $(w_1, w_2) \in K$},
\end{math}
Diese Kongruenz ist verträglich mit $\circ$, d.h. $u \equiv u'$ und $v \equiv v'$, dann ist $u \circ v \equiv u' \circ v'$.

Auf $Q := A^* / K := A^* / \equiv$ erhalten wir $\argdot: Q \times Q \to Q$, $[u] \cdot [v] := [u \circ v]$.
Damit ist auch $(Q, \cdot, [e])$ ein Monoid.


\begin{df}
    Das durch $(A, K)$ \emphdef{präsentierte Monoid} ist
    \begin{math}
        \GenMonoid{A & K} := A^* / K.
    \end{math}
\end{df}

\begin{ex}
    \begin{itemize}
        \item
            \begin{math}
                (N = \GenMonoid{a & -}, \cdot) &\xto[homeomorphic] (\N, +) \\
                a^n &\mapsto n \\
                a^n &\mapsfrom n
            \end{math}
        \item
            \begin{math}
                \GenMonoid{a,b & -}
                = \Set{e, a, b, aa, ab, ba, bb, \dotsc}
            \end{math}
        \item
            \begin{math}
                C = \GenMonoid{s^+, s^- & s^+s^- = e, s^-s^+ = e},
            \end{math}
            d.h. $A = \Set{s^+, s^-}$, $K = \Set{(s^+s^-, e), (s^-s^+, e)}$.
            Dies ist eine Gruppe.
            Definiere
            \begin{math}
                \phi: (\Z, +) &\to (C, \cdot) \\
                k &\mapsto \begin{cases}
                    (s^+)^k & \text{für $k > 0$}, \\
                    e & \text{für $k = 0$}, \\
                    (s^-)^{-k} & \text{für $k < 0$}.
                \end{cases}
            \end{math}
            Dies ist ein Gruppenhomomorphismus, surjektiv (auf Wortklasssen).
            Inverse:
            \begin{math}
                \psi: (C, \cdot) &\to (\Z, +), \\
                s^{\eps_1} s^{\eps_2} \dotsb s^{\eps_l} &\mapsto \eps_1 + \dotsb + \eps_l.
            \end{math}
            Dies ist wohldefiniert, Gruppenhomomorphismus, surjektiv.
            Es gilt $\psi \circ \phi: \id_\Z$, aber auch $\phi \circ \psi = \id_C$.
        \item
            $C_n := C_{n, 0} := \GenMonoid{a & a^n = 1}$ (Skizze: Kreis).
            Es gilt $(C_n, \cdot) \isomorphic (\Z / n, +)$.
        \item
            $C_{n,m} := \GenMonoid{a & a^n = a^m}$ für $0 \le m < n$ (Skizze: Anfang + Schleife).
    \end{itemize}
\end{ex}

Speziell für Gruppen:
Zur Menge $S$ wählen wir das Alphabet
\begin{math}
    A = S \times \Set{\pm} = \Set{s^+, s^- & s \in S}
\end{math}
Zu $R \subset A^*$ setzen wir $K = \Set{ r = 1 & r \in R} \cup \Set{s^+s^- = 1, s^-s^+ = 1 & s \in S}$.
Formal:
\begin{math}
    K = \Set{(r, e) & r \in R} \cup \Set{(s^+s^-, e), (s^-s^+, e) & s \in S}.
\end{math}
Die durch $(S, R)$ \emphdef{präsentierte Gruppe} ist $\Gen{S & R} := \GenMonoid{A & K} = A^* / K$.

\begin{nt}
    In jeder Gruppe lässt sich $a = b$ umformen als $ab^{-1} = 1$.
\end{nt}

\begin{ex}
    \begin{itemize}
        \item
            $\Gen{s & -} := \GenMonoid{s^+ s^- & s^+s^- = 1, s^-s^+ = 1} \isomorphic (\Z, +)$,
        \item
            $\Gen{s & s^n} = \Gen{s & s^n = 1} = \GenMonoid{s^+, s^- & (s^+)^n, s^+s^- = 1, s^-s^+ = 1} \isomorphic (\Z / n, +)$,
        \item
            $\Gen{a,b & ab = ba} = \Gen{a,b & aba^{-1}b^{-1}} \isomorphic (\Z^2, +)$
        \item
            $\Gen{a,b & -} = \Set{e, a, a^{-1}, b, b^{-1}, a^2, a^{-2}, ab, ab^{-1}, a^{-1}b^{-1}, b^2, b^{-2}, ba, ba^{-1}, b^{-1}a, b^{-1}a^{-1}}$

            Skizze: Baum in der Ebene, $a$ nach rechts, $b$ nach oben.

            Im Kontrast dazu $\Z^2 = \Gen{a,b & ab = ba}$: Cayley-Graph.
    \end{itemize}
\end{ex}


\Timestamp{2015-10-23}


\section{Simplizialkomplexe}


Kombinatorische Kodierung eines Simplizialkomplexes:
\begin{math}
    \Set{\emptyset, \Set{a}, \dotsc, \Set{f}, \Set{a,b}, \dotsc, \Set{g,f}, \Set{c,e,f}, \dotsc, \Set{c,g,f}, \Set{c,e,f,g}}.
\end{math}

\begin{df}
    Ein (abstrakter) \emphdef{Simplizialkomplex} $K$ ist ein System endlicher Mengen mit
    \begin{enumerate}[i)]
        \item
            $\emptyset \in K$,
        \item
            $T \subset S \in K \implies T \in K$,
    \end{enumerate}
    Wir setzen
    \begin{math}
        \dim S &:= \card(S) - 1,\\
        \dim K &:= \sup\Set{\dim S & S \in K}, \\
        \Omega(K) &:= \bigcup K = \bigcup_{S \in K} S.
    \end{math}
    $a \in \Omega(K)$ heißt \emphdef{Ecke}, $S \in K$ heißt \emphdef{Simplex} von $K$.
    \begin{math}
        K_{\le n} := \Set{S \in K & \dim S \le n}
    \end{math}
\end{df}

\begin{df}
    Eine \emphdef{Darstellung} $f: K \to V$ in einen $\R$-Vektorraum ist eine Abbildung $f: \Omega(K) \to V$, sodass
    \begin{enumerate}[i)]
        \item
            Für $S \in K$ ist $f(S)$ affin unabhängig.
        \item
            Für $S, T \in K$ gilt $[f(S)] \cap [f(T)] = [f(S\cap T)]$.
    \end{enumerate}
    Die \emphdef{kanonische Darstellung} von $K$ ist $f: K \to \R^{(\Omega)}$, $s \mapsto e_s$.
    \begin{note}
        Hierbei ist
        \begin{math}
            \R^{(\Omega)} = \Set{x: \Omega \to \R & \text{$\supp x$ endlich}}.
        \end{math}
        Dieser hat als kanonische Basis $(e_s)_{s\in\Omega}$ mit $e_s: \Omega \to \R$, $e_s(s') = \delta_{s,s'}$.

        Wir identifizieren $s$ mit $e_s$.
        Dann schreibt sich jedes Element $x \in \R^{(\Omega)}$ als formale Linearkombination
        \begin{math}
            x = \sum_{s \in \Omega} x(s) e_s
            = \sum_{s \in \Omega} x(s) s.
        \end{math}
        Man nennt $\R^{(\Omega)}$ den Vektorraum „frei über $\Omega$“.
    \end{note}
\end{df}

\begin{df}
    Sei $f: K \to V$ eine Darstellung.
    $[f(S)]$ ist ein affiner Simplex in $V$ mit $\dim [f(S)] = \dim S$.
    $\Set{[f(S)] & S \in K}$ ist ein \emphdef{affiner Simplizialkomplex} in $V$, d.h.
    ein System affiner Simplizies, sodass sich je zwei höchstens in einer gemeinsamen Seite schneiden.

    Das Polyeder
    \begin{math}
        |K|_f| := \bigcup_{S \in K} [f(S)] \subset V
    \end{math}
    versehen wir mit der \emphdef{simplizialen Topologie}.
    Eine Teilmenge $U \subset |K|$ ist offen genau dann, wenn $U \cap [f(S)]$ offen ist in $[f(S)]$ für alle $S \in K$.
    \begin{note}
        Für $K$ endlich genügt $f: K \to \R^n$ und die Teilraumtopologie von $|K|_f \subset \R^n$ ist die simpliziale Topologie.

        Für $\Omega$ unendlich ist die simpliziale Topologie wesentlich.
        Betrachte (Skizze: diskrete Variante der Sinuskurve des Topologen)
        \begin{math}
            \Omega &:= \Set{a,b} \cup \N,
            K &:= \Set{\emptyset} \cup \binom{\Omega}{1} \Set{\Set{k, k+1} & k \in \N}
        \end{math}
        mit Darstellung $f: K \to \R^2$, $a \mapsto (0,1)$, $b \mapsto (0,-1)$,
        \begin{math}
            f(k) = \frac{\frac{1}{k}}{(-1)^k}.
        \end{math}
        Wir erhalten $|K|_f \subset \R^2$.
        $[f(a), f(b)]$ ist offen in der simplizialen Topologie, aber nicht offen in der Teilraumtopologie.
    \end{note}
\end{df}


\section{Simpliziale Fundamentalgruppen}

\begin{df}
    Sei $K$ ein Simplizialkomplex mit $\Omega = \Omega(K)$.
    \begin{itemize}
        \item
            Ein Kantenzug $v_0v_1 \dotsc v_n$ ist eine endlich Folge von Eckpunkten mit $\Set{v_0, v_1}, \dotsc, \Set{v_{n-1}, v_n} \in K$.
        \item
            Zwei Kantenzüge $w = v_0 \dotsc v_n$ und $w' = v_0' \dotsc v_m'$ heißen \emphdef{verknüpfbar}, wenn $v_n = v_0'$.
            In diesem Fall ist $w \ast w' := v_0 \dotsc v_n v_1' \dotsc v_m'$ die \emphdef{Verknüpfung} beider Kantenzüge.
        \item
            Zwei Kantenzüge $w = v_0 \dotsc v_{k-1} v_k v_{k+1} \dotsc v_n$ und $w' = v_0 \dotsc v_{k-1} v_{k+1} \dotsc v_n$ heißen äquivelent, geschrieben $w \approx w'$, falls $\Set{v_{k-1}, v_k, v_{k+1}} \in K$.
        \item
            Zu $w = v_0 v_1 \dotsc v_n$ setze $\_w := v_n \dotsc v_1 v_0$.
            Es gilt $w \ast \_w = v_0 v_1 \dotsc v_{n-1} v_n v_{n-1} \dotsc v_1 v_0 \approx v_0$.
        \item
            Simpliziales Fundamentalgruppoid:
            \begin{math}
                \Pi(K) = \frac{\Set{\text{Kantenzüge in $K$}}}{\approx}.
            \end{math}
        \item
            Simpliziale Fundamentalgruppe:
            \begin{math}
                \pi_1(K, x_0) := \Pi(K, x_0, x_0) = \frac{\Set{\text{Kantenzüge in $K$}}}{\approx}
            \end{math}
            Wir erhalten einen Gruppenisomorphismus
            \begin{math}
                \phi: \pi_1(K, x_0) &\xto[isomorphic] \pi_1(|K|, x_0)
                [w] &\mapsto [|w|].
            \end{math}
            \begin{proof}[Beweisidee]
                Prüfe: Wohldefiniertheit (Verträglichkeit der Äquivalenzen),
                Surjektivität: simpliziale Approximation von $\gamma:[0,1] \to (K, x_0)$,
                Injektivität: simpliziale Approximation von $H: [0,1]^2 \to (K, x_0)$.
            \end{proof}
    \end{itemize}
\end{df}

\begin{ex}
    \begin{itemize}
        \item
            Ein (simplizialer) Graph ist ein Simplizialkomplex $K$ mit $\dim K \le 1$.

            Für Kantenzüge gibt es dann nur die Äquivalenzen (Kürzungen/Erweiterungen) der Art
            \begin{math}
                uu &\approx u, &
                uvu &\approx u.
            \end{math}
            Sind keine solchen Kürzungen möglich, so nennen wir den Kantenzug \emphdef{gekürzt} (eigentlich \emph{lokal} gekürzt, greedy).
        \item
            Ein \emphdef{Baum} ist ein Graph, der zusammenhängend und zykelfrei ist, d.h.
            \begin{enumerate}[i)]
                \item
                    Zu je zwei Ecken $a,b \in \Omega(K)$ existiert ein Kantenzug von $a$ nach $b$.
                \item
                    Für jede Kante $\Set{a,b} \in K$, $a \neq b$ sind die Ecken $a,b$ in $K \setminus \Set{\Set{a,b}}$ nicht mehr verbindbar.
            \end{enumerate}
        \item
            Für jeden nicht-leeren endlichen Graphen $K$ sind äquivalent:
            \begin{enumerate}[i)]
                \item
                    $K$ ist ein Baum,
                \item
                    $|K|$ ist zusammenziehbar,
                \item
                    $K$ ist zusammenhängend und $\chi(K) = 1$,
                \item
                    $K$ ist zykelfrei und $\chi(K) = 1$.
                \item
                    Zu je zwei Ecken $a, b \in \Omega(K)$ existiert genau ein gekürzter Kantenzug von $a$ nach $b$.
            \end{enumerate}
            Für unendliche Graphen gilt die Äquivalenz noch zwischen i), ii) und v).
        \item
            Sei $K$ ein zusammenhängender Graph und $T \subset K$ ein Teilgraph, der alle Ecken von $K$ enthält, kurz: $\Omega(T) = \Omega(K)$.
            Dann sind äquivalent:
            \begin{enumerate}[i)]
                \item
                    $T$ ist ein Baum,
                \item
                    $T$ ist zykelfrei und maximal,
                \item
                    $T$ ist zusammenhängend und minimal,
            \end{enumerate}
            In diesem Fall nennen wir $T$ \emphdef{Spannbaum}.
    \end{itemize}
\end{ex}

\begin{st}
    Für jeden Baum $T$ gilt
    \begin{math}
        \pi_1(T, x_0) = \pi_0(|T|, x_0) = \Set{1}.
    \end{math}
    \begin{proof}
        $\pi_1$ durch Kürzung.
        $\pi_0$ durch Zusammenziehen.
    \end{proof}
\end{st}

\begin{st}
    Sei $K$ ein zusammenhängender Graph, $x_0 \in \Omega(K)$, $T \subset K$ ein Spannbaum.
    Dann ist $\pi_1(K, x_0)$ frei über $|K \setminus T|$ Erzeugern.

    Genauer: $\psi: \pi_1(K, x_0) \to F = \GenMonoid{S & R} = \Gen{S & R}$ mit Erzeugern $S = \Set{s_{ab} & \Set{a,b} \in K \setminus T}$ und Relationen $R = \Set{s_{ab} s_{ba} & \Set{a,b} \in K \setminus T}$.

    Ist $K$ zudem endlich, so hat $\pi_1(K, x_0)$ den Rang $|K \setminus T| = 1 - \chi(K)$.
    \begin{proof}
        Siehe Verallgemeinerung unten
    \end{proof}
\end{st}

\begin{st}
    Sei $K$ ein zusammenhängender Simplizialkomplex, $x_0 \in \Omega(K)$, $T \subset K$ ein Spannbaum (im 1-Skelett).
    Dann gilt $\pi_1(K, x_0) \isomorphic \Gen{S & R} = G$ mit
    \begin{math}
        S &= \Set{s_{ab} & \Set{a,b} \in K}, \\
        R &= \Set{s_{ab} & \Set{a,b} \in T} \cup \Set{s_{ab} s_{ba} & \Set{a,b} \in K}
        \cup \Set{s_{ab} s_{bc} s_{ca} & \Set{a,b,c} \in K }
    \end{math}
    Genauer: existieren zueinander inverse Gruppenisomorphismen
    \begin{math}
        \psi&:& \pi_1(K, x_0) &\to G, &
        [v_0 \dotsc v_n] &\mapsto s_{v_0v_1} \dotsb s_{v_{n-1} v_n}, \\
        \phi&:& G &\to \pi_1(K,x_0), &
        s_{ab} &\mapsto [x_0 \dotsc a \ast ab \ast b \dotsc x_0].
    \end{math}
    \begin{proof}
        Durch Nachrechnen: $\psi$ wohldefiniert, $\phi$ wohldefiniert.
        Es gilt $\psi \circ \phi = \id_G$, denn $\psi(\phi(s_{ab})) = s_{ab}$.
        Ebenso $\phi \circ \psi = \id_{\pi_1}$ (nach Kürzen der antisymmetrischen Wege in $T$).
    \end{proof}
\end{st}

\begin{ex}
    \begin{itemize}
        \item
            Skizze: Triangulierter Torus $T$ mit 9 Ecken, Spannbaum $U$.
            Plausibilität: $\chi(T) = 9 - 27 + 18 = 0$.
            Nicht-triviale Elemente $s_{xy} \in T \setminus U$ (wende Relationen an) bilden Erzeuger: $s := s_{ac}$, $t := s_{ea}$.
            Wir erhalten
            \begin{math}
                \pi_1(T, a) = \Gen{S & R}
                \xto* \Gen{s, t & st = ts} \isomorphic \Z^2.
            \end{math}
            Surjektiv nach Bild: Alle Erzeuger werden getroffen.
            Injektiv nach Bild: Alle Relationen wurden verwendet.

            Wie erwartet
            \begin{math}
                \pi_1(|T|, x_0)
                \isomorphic \pi_1(\S^1 \times \S^1, x_0)
                \isomorphic \pi_1(\S^1, x_0) \times \pi_1(\S^1, x_0)
                \isomorphic \Z \times \Z
                \isomorphic \Z^2.
            \end{math}
    \end{itemize}
\end{ex}


\Timestamp{2015-10-30}

\section{Der Satz von Seifert-van-Kampen}

Sei $X = \bigcup_{i \in I} U_i$ eine offene Überdeckung.
Ziel: Wie berechnet man $\pi_1(X, x_0)$ aus den Teilen $(U_i)_{i \in I}$?

Einfaches Beispiel (Skizze: drei Mengen mit paarweisen Schnitten)
Fordere
\begin{itemize}
    \item
        $U_i$ einfach zusammenhängend
    \item
        $U_i \cap U_j$ einfach zusammenhängend
    \item
        $U_i \cap U_j \cap U_k$ einfach zusammenhängend
\end{itemize}
Man denke an $U_i \subset \R^n$ konvex, dann sind alle weiteren Schnitte konvex (ebenso $U_i \subset M$ in einer Riemannschen Mannigfaltigkeit).

Wir bilden folgenden Simplizialkomplex, genannt der \emphdef{Nerv} von $\scr U = (U_i)_{i \in I}$.
Für $S = \Set{s_0, \dotsc, s_n} \subset I$ setze $U_S := U_{s_0} \cap \dotsb \cap U_{s_n}$, sowie $U_{\emptyset} := X$.
Der \emphdef{Nerv} ist
\begin{math}
    N(\scr U) = \Set{\text{$S \subset I$ endlich} & U_S \neq \emptyset}.
\end{math}

Im Beispiel $I = \Set{1, 2, 3}$, $\scr U$ wie skizziert,
\begin{math}
    N(\scr U) = \Set{\emptyset, \Set 1, \Set 2, \Set 3, \Set{1, 2}, \Set{1, 3}, \Set{2, 3}}
\end{math}

\begin{prop}
    $N(\scr U)$ ist ein (abstrakter) Simplizialkomplex.
\end{prop}

\begin{st}
    Sei $X$ ein topologischer Raum, $\scr U = (U_i)_{i \in I}$ eine offene Überdeckung sodass $U_i$ einfach zusammenhängend und $U_i \cap U_j$ wegzusammenhängend ist.
    Wähle $i_0 \in I$ und $x_0 \in U_{i_0}$.

    Dann existiert ein Gruppenisomorphismus
    \begin{math}
        \Phi: \pi_1(N(\scr U), i_0) &\xto \pi_1(X, x_0), \\
        [(i_0, i_1, \dotsc, i_n)] &\mapsto [\gamma_{i_0, i_1} \ast \dotsb \ast \gamma_{i_{n-1}, i_n}].
    \end{math}
    Genauer:
    Hierzu wählen wir $x_i \in U_i$ für $i \in I$ sowie $x_{ij} \in U_{ij} := U_i \cap U_j$ für $i \neq j$ mit $U_{ij} \neq \emptyset$, $x_{ij} = x_{ji}$.
    Sei $\gamma_{ij}$ ein Weg von $x_i$ nach $x_{ij}$ in $U_i$ und dann von $x_{ij}$ nach $x_j$ in $U_j$.
    Damit ist $[\gamma_{ij}]$ eindeutig festgelegt (da $U_i$, $U_j$ einfach zusammenhängend, $U_{ij}$ wegzusammenhängend).
    % \gamma_{ij} \sim \gamma_{ij} \iff \gamma_{ij}\_{\gamma_{ij}} \sim *
    \begin{proof}[Skizze]
        \begin{enumerate}[1)]
            \item
                $\Phi$ surjektiv (Skizze: Weg $\omega$ über endlich viele $U_i$ von $x_{i_0}$ nach $x_{i_n} = x_{i_0}$, homotop zu zweitem Weg):
                Sei $\omega:[0,1] \to X$ ein Weg von $x_0$ nach $x_0$.
                Wir haben eine offene Überdeckung $[0,1] = \bigcup_{i \in I} \omega^{-1}(U_i)$.
                Es existiert eine Lebesgue-Zahl $\frac{1}{n}$, $n \in \nu$, sodass $\omega([\frac{k-1}{n}, \frac{k}{n}]) \subset U_{i_k}$ für $k = 1, \dotsc, n$.
                Für $i_k \neq i_{k+1}$ liegt $\omega(\frac{k}{n}$ in $U_{i_k} \cap U_{i_{k+1}}$.
                Wähle $\beta_k$ von $\omega(\frac{k}{n})$ nach $x_{i_k}{i_{k+1}}$ in $U_{i_k} \cap U_{i_{k+1}}$.
                Dann gilt $\gamma \ast \_\omega \sim 1_{x_0}$.
                %Nutze Kompaktheit des Weges (endliche Überdeckung, Intervallteilung mit Lebesgue-Zahl), konstruiere so die Folge von Mengen $U_i$.
            \item
                $\Phi$ injektiv: später
        \end{enumerate}
    \end{proof}
\end{st}

Im Allgemeinen sind unsere Überdeckungen jedoch nicht so schön.

Sei $X$ ein topologischer Raum und $\scr U = (U_s)_{s \in \Omega}$ eine offene Überdeckung ($U_s$ muss nicht wegzusammenhängend sein, ebensowenig $U_s \cap U_t$).

Der \emphdef{Wegnerv} von $\scr U$ ist definiert durch
\begin{math}
    N^\circ(\scr U) = \Set{(S, C) & \text{$S \subset \Omega$ endlich, $C \in \pi_0(U_s)$} }
\end{math}
($\pi_0(U_s)$ Menge der Wegzusammenhangskomponenten).
Setze $\dim(S, C) = |S| - 1$.
Definiere $(S, C) \to (T, D)$ durch $T \subset S$ und $D \supset C$.

Skizze: $U_1, U_2$, $U_1$ mit zwei Komponenten, mit jeweils $3$ Schnitten $D_1, D_2, D_3$ und einem Schnitt $D_4$ mit $U_2$, $U_2$ einfach zusammenhängend.
\begin{math}
    \begin{tikzcd}
        & (\Set{1,2}, D_1) \ar[ld] \ar[rd]& \\
        (\Set 1, C) & (\Set{1,2}, D_2) \ar[l] \ar[r] & (\Set{2}, U_2) \\
        & (\Set{1,2}, D_3) \ar[lu] \ar[ru] & \\
        (\Set{1}, C') & (\Set{1,2}, D_4) \ar[l] \ar[ruu]
    \end{tikzcd}
\end{math}
Damit ist $I = N^\circ(\scr U)$ mit $\to$ ein Poset, d.h. reflexiv ($i \to i$) und transitiv ($i \to j \to k \implies i \to k$).
Jedem Index $i = (S, C)$ ordnen wir den Teilraum $X_i = C$ zu.
Wir wählen $x_i \in X_i$.
Für $i \to j$ gilt $X_i \subset X_j$.
Da $X_j$ wegzusammenhängend ist, wähle einen Weg $\gamma_{ij}: [0,1] \to X_j$ von $\gamma_{ij}(0) = x_i$ nach $\gamma_{ij}(1) = x_j$.
Für $i = j$ setze $\gamma_{ii} = 1_{x_i}$.
Für $i \to j$, $j \to i$ gilt $X_i = X_j$ und evtl. $x_i \neq x_j$, wir wollen dann $\gamma_{ji} = \_{\gamma_{ij}}$.

Aus $\scr U = (U_s)_{s \in \Omega}$ erhalten wir $(I, \to, (X_i)_{i \in I}, (x_i)_{i \in I}, (\gamma_{ij})_{i \to j})$.

Zu $i \in I$ setzen wir $G_i := \pi_1(X_i, x_i)$.
Für $i \to j$ induziert $\iota_{ij}: X_i \injto X_j$ einen Gruppenhomomorphismus $h_{ij} : G_i \to G_j$ durch
\begin{math}
    h_{ij}([\alpha]) := \_{\gamma_{ij}} \ast (\iota_{ij} \circ \alpha) \ast \gamma_{ij}
\end{math}
(Skizze: !!)
Für $i \to j \to k$ (Skizze: Venn-Diagramm mit Drei Mengen, $i \to j \to k$ und $i \to k$)
setze $g_{ijk} := [\_{\gamma_{jk}} \ast \_{\gamma_{ij}} \ast \gamma_{ik}] \in G_k$.

Es gilt
\begin{math}
    h_{jk} \circ h_{ij} = g_{ijk} h_{ik} g_{ijk}^{-1}.
\end{math}
D.h. $g_{ijk}$ misst die Abweichung von $h_{jk} \circ h_{ij}$ zu $h_{ik}$.

Wir erhalten hieraus den Gruppenkomplex
\begin{math}
    \Gamma &= (I, \to, G_\argdot, h_{\argdot, \argdot}, g_{\argdot, \argdot, \argdot}) \\
    &= (I, \to, (G_i)_{i\in I}, (h_{ij})_{i \to j}, (g_{ijk})_{i \to j \to k}).
\end{math}
Hierin betrachten wir \emphdef{Kantenzüge}
\begin{math}
    w = (i_0 \xto[lr]{g_1} i_1 \xto[lr] \dotsb \xto[lr] i_n)
\end{math}
Hierbei seien $i_0, i_1, \dotsc, i_n \in I$ und $i \xto[lr]{g} j$ steht entweder für $i \to j$ oder $i \xto* j$ in Graphen $(I, \to)$ oder aber $i \xto{g} i$ oder $i \xto*{g} i$ mit $g \in G_i$.
Die Verknüpfung $w \ast w'$ ist die Aneinanderhängung.
Wir nutzen folgende Relationen
\begin{math}
    (i \to i) \approx (i \xto{1} i) &\approx (i), \\
    (i \to j \xto* i) &\approx (i), \\
    (j \xto* i \xto{g} i \to j) &\approx (j \xto{h_{ij}(g)} j), \\
    (i \xto{g} i) &\approx (i \xto*{g^{-1}} i), \\
    (i \xto{g} i \xto{h} i) &\approx (i \xto{gh} i), \\
    (i \to j \to k) &\approx (i \to k \xto{g_{ijk}^{-1}} k).
\end{math}
Die Kantengruppe des Gruppenkomplexes $\Gamma$ ist
\begin{math}
    \pi_1(\Gamma, i_0) = \frac{\text{geschl. Kantenzüge in $\Gamma$ von $i_0$ nach $i_0$}}{\text{Äquivalenz $\approx$}}.
\end{math}

\begin{st}[Seifert-van-Kampen]
    Jede offene Überdeckung $X = \bigcup{s \in \Omega} U_s$ definiert einen Gruppenkomplex $\Gamma$ (nach Wahl von Fußpunkten und Verbindungswegen).
    Seine Kantengruppe $\pi_1(\Gamma, i_0)$ ist isomorph zu $\pi_1(X, x_0)$.

    Genauer: Für das $1$-Skelett $\Gamma_{\le 1}$ liefert die topologische Realisierung eine Surjektion
    \begin{math}
        \Phi_1: \pi_1(\Gamma_{\le 1}, i_0) \xto[surjective] \pi_1(X, x_0).
    \end{math}
    Für das $2$-Skelett $\Gamma_{\le 2}$ erhalten wir einen Isomorphismus $\Phi_2 : \pi_1(\Gamma_{\le 2}, i_0) \xto[isomorphic] \pi_1(X, x_0)$.
\Timestamp{2015-11-06}
    \begin{proof}[Skizze]
        Betrachte $\Phi_1: \pi_1(\Gamma_{\le 1}, i_0) \xto[surjective] \pi_1(X, x_0)$, zeige Surjektivität.
        Sei dazu $[\omega] \in \pi_1(X, x_0)$, d.h. $\omega:[0,1] \to X$ eine Schleife in $x_0$.
        Zeige: $\omega$ homotop zu einem Weg $\Phi_1(w)$ mit $w \in \pi_1(\Gamma_{\le 1}, i_0)$.
        \begin{math}
            w = (i_0 \xto{g_0} i_0 \xto* i_{0,1} \to i_1 \xto{g_1} i_1 \xto* i_{12} \to i_2 \xto{g_2} i_2 \dotsb \xto* j_{n-1} \to i_n \xto{g_n} i_n).
        \end{math}
        Die topologische Realisierung von $w$ ist homotop zu $\omega$ nach Konstruktion.

        Betrachte nun $\Phi_2: \pi_2(\Gamma_{\le 2}, i_0) \to \pi_1(X, x_0))$, zeige Bijektivität.
        Wohldefiniert: Man vergewissere sich, dass alle definierten Äquivalenzen entsprechende Homotopien erlauben.
        Surjektivität wie zuvor.
        Zeige nun Injektivität: Werden zwei Wörter $w_1, w_2$ durch homotope Wege $w_1, w_2$ realisiert, dann sind $w_1$ und $w_2$ äquivalent.
        Sei $H: [0,1]^2 \to X$ eine Homotopie von $w_1$ nach $w_2$.
        Idee: Modifziere $H$ derart, dass sich eine Triangulierung ergibt und jedes Dreieck einer Relation entspricht.
        Wir nutzen $X = \bigcup_{s \in \Omega} U_s$ für $[0,1]^2 = \bigcup_{s \in \Omega} H^{-1}(U_S)$.
        Dank Kompaktheit existiert eine Lebesgue-Zahl dieser Überdeckung.
        Wir unterteilen $[0,1]^2$ wie folgt (Skizze: Ziegel-Mauerwerk):
        Nach hinreichend feiner Unterteilung gilt für jeden Ziegel $\Z_\alpha$ die Bedingung $H(Z_\alpha) \subset U_{s(\alpha)}$ für eine geeignete Abbildung $s$.
        Wir dicken die Fugen auf (z.B. erst horizontale Fugen, dann vertikale durch konstante Homotopien, es entstehen kleine Quadrate, in denen die Homotopie konstant ist).
        Lokal in einem kleinen Quadrat ist $H$ konstant $c$ und es gilt $H(Q) \subset U_1 \cap U_2 \cap U_3$.
        Wähle $i \in I$ sodass $H(Q) \subset X_i$, füge $x_i$ im Quadrat ein mit „radialer Homotopie“.
        Zusätzlich $x_{01}, x_{12}, x_{12}$ an den Kantenmittelpunkten.
        Hilfspunkte für $g_{012}$.

        Von Fugen-Quadraten zu den Fugen-Rechtecken.
        Damit können wir jede Fuge auffüllen durch
        Schließlich Ziegel.

        (Skizzen sind hilfreich)

        Zusammenfassung:
        Wir beginnen mit der gegebenen Homotopie $H$.
        \begin{enumerate}[1.]
            \item
                Aufdicken von horizontalen und vertikalen Fugen.
            \item
                Korrektur um Eckpunkten.
            \item
                Korrektur auf Fugenstücken.
            \item
                Korrektur auf Ziegeln.
        \end{enumerate}
        Ablesen dieser einfachen Teile liefert die kombinatorische Äquivalenz von $w_1$ nach $w_2$.
    \end{proof}
\end{st}

\begin{ex}
    \begin{itemize}
        \item
            Kreisring aus zwei Mengen gebildet.
            $\pi_1(\Gamma, x_1) \isomorphic \Z$.
            Wir benötigen nur $U_1, U_2$ einfach zusammenhängend, $U_1 \cap U_2$ hat zwei Wegkomponenten.

            Betrachte $U_1, U_2$ mit $U_1 \cap U_2 = C_1 \dunion C_2$ und $C_1 \isomorphic C_2 \isomorphic \S^1 \times \B^2$.
        \item
            $U_1$ Kreisring, $U_2$ Kreisscheibe, $U_1 \cap U_2$ Kreisring.
        \item
            Venn-Diagramm, ohne Mittelteil, $\pi_1(\Gamma, x_0) \isomorphic \Z$.
        \item
            Wie voriges, mit mittlerer Kreisscheibe, $\pi_1(\Gamma, x_0) \isomorphic \Set e$.
        \item
            $\pi_0(U_1) = \pi_0(U_2) = \pi_0(U_1 \cap U_2) = \Set *$.
            $\pi_1(U_1) = G_1$, $\pi_1(U_2) = G_2$ beliebig, $\pi_1(U_1 \cap U_2) = \Set e$.
            \begin{math}
                \pi(\Gamma, x_0) = G_1 \ast G_2
            \end{math}
            (freies Produkt).
        \item
            $\pi_1(U_1) = G_1$, $\pi_1(U_2) = G_2$ beliebig, $U_1 \cap U_2$ wegzusammenhängend, $\pi_1(U_1 \cap U_2) = K$.
            Sei $U_{12} := U_1 \cap U_2$,
            \begin{math}
                i: U_{12} &\injto U_1, &i_\#: \pi_1(U_{12}) &\injto \pi_1(U_1), \\
                j: U_{12} &\injto U_2, &j_\#: \pi_1(U_{12}) &\injto \pi_1(U_2). \\
            \end{math}
            Das führt zum amalgamierten Produkt $G_1 \ast_K G_2$.

            Ausführlich und etwas allgemeiner:
            Sei $X = U_1 \cup U_2$, $U_1, U_2$ offen und wegzusammenhängend,
            $U_{12} = U_1 \cap U_2$ (offen und) wegzusammenhängend.
            Wir wählen $x_0 \in U_{12} \subset U_1, U_2$.
            Sei $\pi_1(U_i, x_0) = G_i = \Gen{S_i & R_i}$ für $i \in \Set{1, 2, (1,2)}$.
            (Skizze: $\Gamma$)
            Dann gilt
            \begin{math}
                \pi_1(X, x_0) = \pi_1(\Gamma, x_0)
                = \Gen{S_1, S_2 & R_1 \dunion R_2 \dunion T}
            \end{math}
            mit
            \begin{math}
                T = \Set{h_1(s) h_2(s)^{-1} & s \in S_{12}}
            \end{math}
        \item
            $g_{ijk}$ nichttrivial:
            $U_1 \isomorphic U_2 \isomorphic U_3 \isomorphic \S^1 \times (0,1)$ mit geeigneter Wahl der Fußpunkte.
    \end{itemize}
    \begin{note}
        Gruppenkomplexe dienen zur Analyse von Gruppen, siehe Serre, Trees.
    \end{note}
\end{ex}


\Timestamp{2015-11-13}

\section{Zellkomplexe}


\begin{df}[Anheften von $n$-Zellen]
    Ein Paar $(X, A)$ topologischer Räume entsteht durch \emphdef{Anheften von $n$-Zellen} an $A$, wenn es eine (diskrete) Indexmenge $I$ gibt und eine Abbildung $g: I \times \D^n \to X$, so dass gilt:
    \begin{enumerate}[(1)]
        \item
            $g$ bildet $I \times \S^{n-1}$ nach $A$ ab und $I \times \B^n$ bijektiv auf $X \setminus A$.
        \item
            Genaue dann ist $U \subset X$ offen in $X$, wenn $U \cap A$ offen in $A$ ist und $g^{-1}(U)$ offen in $I \times \D^n$ ist
    \end{enumerate}
    Wir nennen $e_i := g(\Set i \times \B^n)$ eine \emphdef{offene Zelle}, $\_{e_i} := g(\Set i \times \D^n)$ eine \emphdef{abgeschlossene Zelle}.
    Sei $g_i: \D^n \to X$, $g_i(x) = (i,x)$ die \emphdef{charakteristische Abbildung}, $f_i = g_i|_{\S^{n-1}}$ eine \emphdef{anhefende Abbildung} für die Zelle $e_i$.
\end{df}

\begin{nt}
    \begin{itemize}
        \item
            $X$ ist der Quotientenraum
            \begin{math}
                g \sqcup \Id_A: (I \times \D^n) \sqcup A \to X
            \end{math}
            nach (2).
        \item
            $g$ ist stetig.
        \item
            $e_i = g_i(\B^n)$ ist offen in $X$.
        \item
            $g_i|_{\B^n}: \B^n \to e_i$ ist ein Homöomorphismus.
        \item
            $\pi_0(X \setminus A) = \Set{e_i & i \in I}$
        \item
            $X \setminus A$ ist offen in $X$, $A$ ist abgeschlossen in $X$.
        \item
            $\_{e_i} = g_i(\D^n)$ in $X$ ist homöomorph zu $\D^n / f_i$.
        \item
            Mit $\D^n$ ist auch $\_{e_i}$ kompakt.
            Ist $A$ hausdorffsch, so ist $\_{e_i}$ abgeschlossen und $X$ ist auch hausdorffsch.
    \end{itemize}
\end{nt}

\begin{df}
    Sei $(X, A)$ ein Paar topologischer Räume.
    Eine \emphdef{Zellstruktur} auf $(X, A)$ ist eine \emphdef{Filtrierung} $\scr X = (X_n)_{n \in \N}$, also $A = X_{-1} \subset X_0 \subset X_1 \subset \dotsb \subset X$ mit
    \begin{enumerate}[(1)]
        \item
            Für jedes $n \in \N$ entsteht $(X_n, X_{n-1})$ durch Anheften von $n$-Zellen.
        \item
            Der Raum $X = \bigcup_{n \in \N} X_n$ trägt die von $\scr X$ induzierte finale Topologie, d.h. genau dann ist $U \subset X$ offen, wenn $U \cap X_n$ offen in $X_n$ ist für alle $n \in \N$.
    \end{enumerate}
    Das Paar $(X, \scr X)$ heißt \emphdef{Zellkomplex} (auch \emphdef{$CW$-Komplex}), $X_n$ heißt \emphdef{$n$-Skelett}, $\dim (X, \scr X) := \dim \scr X := \inf \Set{n \in \Z & X_n = X}$

    Ein Zellkomplex $(X, \scr X)$, bzw. eine Zellstruktur $\scr X$ heißt \emphdef{endlich}, wenn er nur endlich viele Zellen hat, d.h. $\bigcup_{k \in \N} \pi_0(X_k \setminus X_{k-1})$ ist endlich.

    Im Folgenden betrachten wir meist den absoluton Fall $X_{n-1} = A := \emptyset$.
\end{df}

\begin{ex}
    \begin{itemize}
        \item
            $\dim(X, \scr X) = 0$ genau dann, wenn $X$ diskret ist.
        \item
            Für $\dim \scr X = 1$ haben wir einen (zellulären) \emphdef[zellulärer Graph]{Graphen}.
            $X_0$ bilden die Ecken, $X_1 \setminus X_0$ bilden die offenen Kanten.
    \end{itemize}
\end{ex}

\begin{nt}
    Jeder Simplizialkomplex ist ein Zellkomplex.
    Für $X = |K|$ setzt man dazu $X_n = |K_{\le n}|$.
    Als charakteristische Abbildung können wir die affinen Koordinaten wählen.
\end{nt}

\begin{nt}
    Der $\R^n$ erlaubt eine Zellstruktur.
    Für $\R^2$ exemplarisch:
    \begin{math}
        X_0 &= \Z^2, \\
        X_1 &= (\R \times \Z) \cup (\Z \times \R), \\
        X_2 &= \R^2.
    \end{math}
\end{nt}

\begin{nt}
    Eine offene Zelle $e_i$ mit $\dim e_i < \dim \scr X$ ist im Allgemeinen nicht offen in $X$.
\end{nt}

\begin{ex}
    $X = \S^n$ erlaubt Zellstrukturen
    \begin{enumerate}[a)]
        \item
            \begin{math}
                X_0 = \Set{p} = X_1 = X_2 = \dotsb = X_{n-1} \subset X_n = \S^n
            \end{math}
            Tatsächlich ist $X_n \setminus X_{n-1} = \S^n \setminus \Set{p} \homeomorphic \R^n \homeomorphic \B^n$ mit charakteristischen Abbildung $\D^n \xto{\text{quot}} X = \S^n = \D^n // \S^{n-1}$.
        \item
            Ankleben von nördlichem/südlichem Teil
            \begin{math}
                X_0 = \S^0,
                X_1 = \S^1,
                \dotsc,
                X_n = \S^n.
            \end{math}
    \end{enumerate}
\end{ex}

\begin{ex}
    $\RP^n = \S^n / \Set{\pm 1}$ erlaubt eine Zellstruktur
    \begin{math}
        X_0 = \RP^0,
        X_1 = \RP^1,
        \dotsc,
        X_n = \RP^n.
    \end{math}
    Wir haben
    \begin{math}
        \begin{tikzcd}
            \S^0 & \S^1 & \dotsb & \S^n, \\
            \RP^0 & \RP^1 & \dotsb & \RP^n,
        \end{tikzcd}
    \end{math}
    Es gilt $\S^k \setminus \S^{k-1} = \S^k_{>0} \sqcup \S^k_{<0}$.
    Charakteristische Abbildung
    \begin{math}
        g_{\pm}: \D^k &\to \S^k, \\
        x &\mapsto (x, \pm \sqrt{1-|x|^2})
    \end{math}
    $g_\pm(\S^{k-1} \subset \S^{k-1}$, hier die Identität.
    Daraus bilden wir nun den Quotienten modulo $\pm 1$.
    $\RP^k \setminus \RP^{k-1}$ besteht aus genau einer $k$-Zelle $\_g: \D^k \to \S^k \xto{q} \RP^k$,
    \begin{math}
        \_g(x) &= q(x, \sqrt{1-|x|^2}) = \l[x_0 : \dotsb : x_{k-1}: \sqrt{1 - x_0^2 - \dotsb - x_{k-1}^2 } \r],
    \end{math}
    $\_f = \_g|_{\S^{k-1}}: \S^{k-1} \xto{\Id} \S^{k-1} \xto{q} \RP^{k-1}$.
    Die Anheftende Abbildung auf dem Rand ist $q: \S^{k-1} \to \RP^{k-1}$.
\end{ex}

\begin{prop}
    Jeder Zellkomplex $(X, \scr X)$ ist lokal zusammenziehbar.
\end{prop}

\begin{ex}
    Folgende Räume erlauben keine Zellstruktur:
    \begin{itemize}
        \item
            $\Q \subset \R$,
        \item
            $X = (\R \times \Set{\pm 1}) \cup (\Q \times [-1, +1])$,
        \item
            $X := \bigcup_{k=1}^\infty \frac{1}{k} (\S^1 + 1)$.
            Versucht man $X_0 = \Set{0}, X_1 = X$,
            \begin{math}
                X_1 \setminus X_0 = \bigcup_{k=1}^\infty \frac{1}{k} ((\S^1 + 1) \setminus \Set 0),
            \end{math}
            so erhält man die falsche Topologie:
            \begin{math}
                U = \bigcup_{k=1}^\infty \frac{1}{k} (\S_{>0}^1 + 1)
            \end{math}
            ist offen in der zellulären Topologie, aber nicht in der euklidischen.
    \end{itemize}
\end{ex}

\begin{ex}
    Flächen $F_g^\pm$.

    Eine Zellstruktur für den Torus $F_1^+$ besteht aus einer Ecke, zwei Kanten und einer FLäche,
    \begin{math}
        X_2 \setminus X_1 \homeomorphic \B^2.
    \end{math}
\end{ex}

\begin{prop}
    Sind $(X, \scr X)$ und $(Y, \scr Y)$ Zellkomplexe, dann erlaubt $X \times Y$ die Zellstruktur $\scr Z = \scr X \otimes \scr Y$ mit
    \begin{math}
        Z_n = \bigcup_{k+l=n} X_k \times Y_l
    \end{math}
    (sind $\scr X$ und $\scr Y$ nicht lokal endlich, so statten wir $X \times Y$ mit der kompakt erzeugten Topologie aus).
\end{prop}

\begin{ex}
    \begin{itemize}
        \item
            $(\R, (\emptyset \subset \Z \subset \R)) \times (\R, (\emptyset \subset \Z \subset \R))$ ergibt
            \begin{math}
                (\R^2, (\emptyset \subset \Z \times \Z \subset (\R \times \Z) \cup (\Z \times \R) \subset \R \times \R).
            \end{math}
        \item
            $(\S^1, (\emptyset \subset \Set p \subset \S^1)) \times (\S^1, (\emptyset \subset \Set q \subset \S^1))$ ergibt
            \begin{math}
                (\S^1 \times \S^1, (\emptyset \subset \Set{(p\times q)} \subset (\S^1 \times \Set q) \cup (\Set p \times \S^1) \subset \S^1 \times \S^1))
            \end{math}
    \end{itemize}
\end{ex}

\begin{nt}
    $\Z^2 \xto[cap] \R^2 \xto[surjective] \S^1 \times \S^1$.
    Die Zellstruktur auf $\R^2$ ist äquivariant bezüglich der Operation von $\Z^2 \xto[cap] \R^2$.
    Wir erhalten eine Zellstruktur auf dem Quotienten $\R^2 / \Z^2 \homeomorphic \S^1 \times \S^1$.
\end{nt}

\begin{ex}
    Fläche $F_g^+$: eine Ecke, $2g$ Kanten und eine Zelle.
\end{ex}

\subsection{Teilkomplexe}

\begin{df}
    Sei $(X, \scr X)$ ein Zellkomplex.
    Ein \emphdef{Teilkomplex} $(Y, \scr Y)$ ist ein Teilraum $Y$ mit einer Zellstruktur $\scr Y$ mit $Y_n = X_n \cap Y$ und $\pi_0(Y_n \setminus Y_{n-1}) \subset \pi_0(X_n \setminus X_{n-1})$.
\end{df}

\begin{st}
    Sei $(X, \scr X)$ ein Zellkomplex.
    Jede kompakte Teilmenge $K \subset X$ ist in einem endlichen Teilkomplex enthalten.
\end{st}

\begin{kor}
    Sei $(X, \scr X)$ ein Zellkomplex.
    Genau dann ist $X$ kompakt, wenn $\scr X$ endlich ist.
\end{kor}

\begin{df}
    Ein Zellkomplex $(X, \scr X)$ heißt \emphdef{regulär}, wenn gilt
    \begin{enumerate}[1)]
        \item
            Jede anheftende Abbildung $f_i: \S^n \to X_{n-1}$ kann injektiv gewählt werden.
        \item
            Das Bild $f_i(\S^{n-1}) \subset X_{n-1}$ ist ein endlicher Teilkomplex.
    \end{enumerate}
\end{df}

\begin{ex}
    \begin{itemize}
        \item
            Graphen (Schlingen nicht erlaubt für reguläre Zellkomplexe).
        \item
            Simplizialkomplexe sind immer reguläre Zellkomplexe.
        \item
            $\R^n$ wie oben ist regulär, $\S^0 \subset \dotsb \subset \S^n$ ist regulär.
            $\RP^0 \subset \dotsb \subset \RP^n$ ist nicht regulär.
    \end{itemize}
\end{ex}

\begin{st}
    Jeder reguläre Zellkomplex erlaubt eine baryzentrische Triangulierung.
\end{st}

\begin{df}
    Die Euler-Charakteristik eines endlichen Zellkomplexes $(X, \scr X)$ ist
    \begin{math}
        \chi(X, \scr X) = \chi(\scr X)
        := \sum_{k=0}^\infty (-1)^k|\pi_0(X_k \setminus X_{k-1})|.
    \end{math}
\end{df}

\begin{ex}
    \begin{math}
        \chi(\S^n, (\Set p \subset \dotsb \subset \Set p \subset \S^n))
        &:= 1 + (-1)^n&
        &= \begin{cases}
            2 & \text{für $n$ gerade}, \\
            0 & \text{fur $n$ ungerade}.
        \end{cases}, \\
        \chi(\S^n, (\S^0 \subset \dotsb \subset \S^n))
        &:= 2 - 2 \pm \dotsb + (-1)^n \cdot 2&
        &= \begin{cases}
            2 & \text{für $n$ gerade}, \\
            0 & \text{fur $n$ ungerade}.
        \end{cases}
    \end{math}
\end{ex}

\begin{st}
    Seien $(X, \scr X)$ und $(Y, \scr Y)$ Zellkomplexe.
    Aus $X \homeomorphic Y$ oder sogar $X \homotopic Y$ folgt $\chi(\scr X) = \chi(\scr Y)$.
\end{st}


\subsection{Berechnung von \texorpdfstring{$\pi_1(X, \ast)$}{π₁(X, *)} mit Zellstruktur \texorpdfstring{$\scr X$}{𝒳} auf \texorpdfstring{$X$}{X}}

$\ast \subset X_0 \subset X_1$ ist ein Graph: $\pi_1(X_1, \ast)$ ist frei.

$(X_1, *) \injto (X_2, *)$ induziert $\pi_1(X_1, \ast) \surto \pi_1(X_2, \ast)$ surjektiv mit einem Kern, der normal erzeugt ist von den anheftenden Abbildungen $f_i: \S^1 \to X_1$.
Anschließend ist $\pi_1(X_k, \ast) \to \pi_1(X_{k+1}, \ast)$ ein Isomorphismus für $k \ge 2$.


\Timestamp{2015-11-20}

\section{Zelluläre Präsentation der Fundamentalgruppe}

Beobachtung: Für jeden Zellkomplex $(X, \scr X)$ haben wir
\begin{math}
    \pi_0(X_0) \xto[surjective] \pi_0(X_1) \xto[homeomorphic] \pi_0(X_2) \xto[homeomorphic] \dotsb
\end{math}
Wege könnten Punkten verbinden, aber $n$-Zellen am Rand von $\S^{n-1}$-Zellen zu kleben verbindet nichts neues für $n \ge 2$.

Für $\pi_1$ ist dann $\pi_1(X_k, x_0) \xto \pi_1(X, x_0)$ bijektiv für $k > 0$, aber für $k = 0$ nur surjektiv.

\begin{st}
    Für jeden Zellkomplex $(X, \scr X)$ können wir eine Präsentation der Fundamentalgruppe $\pi_1(X, x_0) = \Gen{s_i, i \in I & r_j, j \in J}$ wie folgt ablesen:
    \begin{enumerate}[(1)]
        \item
            Sei $X$ (weg-)zusammenhängend, $x_0 \in X_0$ eine Ecke, $T \subset X_1$ ein Spannbaum.
            Dann ist $\pi_1(X, x_0)$ frei über $S = \pi_0(X_1 \setminus T)$.

            Genauer wählen wir für jede Kante $s_i \in S$ eine Parametrisierung $g_i: \D^1 = [-1, 1] \to \_{s_i}$ und somit eine (willkürliche) Orientierung.
            Damit ist
            \begin{math}
                \pi_1(X_1, x_0) = \Gen{s_i, i \in I & -}.
            \end{math}
        \item
            Die Inklusion $\iota: X_1 \injto X_2$ induziert $\iota_\#: F = \pi_1(X_1, x_0) \surto \pi_1(X_2, x_0)$ mit
            \begin{math}
                \ker(\iota_\#) = \< r_j^F : j \in J\> =: N \normdiv F
            \end{math}
            Genauer: Für jede 2-Zelle $e_j \in \pi_0(X_2 \setminus X_1)$, $j \in J$ wählen wir eine Parametrisierung
            \begin{math}
                g_j : \D^2 \to \_{e_j}.
            \end{math}
            Die anheftende Abbildung $\partial g_j := g_j|_{\partial \D^2}: \S^1 \to X_1$ definiert nach Verbindung zum Fußpunkt $x_0$ einen Weg in $(X_1, x_0)$ und somit ein Gruppenelement $r_j := [\partial g_j] \in \pi_1(X_1, x_0)$ (Skizze!).
        \item
            Jede weitere Inklusion $X_n \injto X_{n+1}$, $n \ge 2$ induziert eienen Isomorphismus $\pi_1(X_n, x_0) \xto[isomorphic] \pi_1(X_{n+1}, x_0)$.
        \item
            Ebenso induziert $X_n \injto X$ einen Isomorphismus
            \begin{math}
                \pi_1(X_n, x_0) \injto \pi_1(X, x_0).
            \end{math}
    \end{enumerate}
    \begin{proof}
        Seifert-van-Kampen:
        \begin{enumerate}[(1)]
            \item
                Wir haben $X_1 = U \cup V$ mit $U = X_1 \setminus \Set{g_i(0) & i \in I}$, $V = X_1 \setminus T \isomorphic I \times \B^1$ und
                $U \homotopic T \homotopic \ast$.
                Wegzusammenhangskomponenten sind $g_i(\D^1_{<0})$, $g_i(\D^1_{>0})$.
            \item
                $X_2 = U \cup V$ mit
                \begin{math}
                    U &= X_2 \setminus \Set{g_j(0) & j \in J} \homotopic X_1, \\
                    V &= X_2 \setminus X_1 \homeomorphic J \times \B^2.
                \end{math}
            \item
                $J \times \S^1 \homotopic J \times (\B^2 \setminus \Set 0) \xto[homeomorpic]{g} U \cap V = W$.
                $\Set{j} \times \frac{1}{2} \S^1 \injto V$ ist zusammenziehbar.
                $\Set{j} \times \frac{1}{2} \S^1 \injto U$ ist homotopieäquivalent zu $\partial g_j$.
                Also
                \begin{math}
                    \pi_1(X_2, x_0) &= \pi_1(U, x_0) / \<r_j^F : j \in J\> \\
                    &= \pi_1(X_1, x_0) / \<r_j^F: j \in J\> \\
                    &= F / N.
                \end{math}
            \item
                $X_{n+1} = U \cup V$ mit $U = X_{n+1} \setminus \Set{g_k(0) & k \in K} \homotopic X_n$, $V = X_{n+1} \setminus X_n \homeomorphic K \times \B^{n+1}$.
                $W = U \cap V \homeomorphic K \times (\B^{n+1} \setminus \Set{0}) \homotopic K \times \S^n$.
                Für $n \ge 2$ gilt $\pi_1(\S^n, \ast) = \Set{1}$.
                Damit klar, wie zuvor.
            \item
                Jeder Weg $\gamma: [0,1] \to X$ in $X$ und jede Homotopie $H: [0,1]^2 \to X$ hat ein kompaktes Bild, liegt also in einem endlichen Teilkomplex $K$, also $K \subset X_N$ für $N$ groß.
                Damit klar.
        \end{enumerate}
    \end{proof}
\end{st}

\begin{kor}
    Jede Gruppe ist Fundamentalgruppe eines topologischen Raumes, sogar eines Zellkomplexes mit $\dim \le 2$.
\end{kor}


