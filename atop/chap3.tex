% Kapitel 3
\chapter{Höhere Homotopiegruppen}

Betrachte $X := \R^n \setminus B_1(0)$, $Y := \R^n$: die Fundamentalgruppe kann für $n = 2$ $X$ und $Y$ unterscheiden, für $n \ge 3$ nicht mehr.
Wir suchen ein Analogon um dies zu ermöglichen.

Sei $I^n = [0,1]^n \subset \R^n$, $\Boundary I^n$ sein Rand.
Zu jedem topologischen Raum $X$ mit Fußpunkt $x_0$ definieren wir
\begin{math}
    \pi_n(X, x_0) = \Set{f: (I^n, \Boundary I^n) \to (X, x_0) \text{ stetig}} / \text{ Homotopie}
\end{math}
(für $n = 0, 1$ stimmt das überein mit den bisherigen Definitionen).

Verknüpfung:
Zu $f, g : (I^n, \Boundary I^n) \to (X, x_0)$ definieren wir $(f \ast g): (I^n, \Boundary I^n) \to (X, x_0)$ durch
\begin{math}
    (f \ast g)(s_1, \dotsc, s_n) = \begin{cases}
        f(2s_1, s_2, \dotsc, s_n) & \text{für $0 \le s \le \frac{1}{2}$}, \\
        g(2s_1 - 1, s_2, \dotsc, s_n) & \text{für $\frac{1}{2} \le s_1 \le 1$}.
    \end{cases}
\end{math}

Dies ist verträglich mit Homotopie:
$H: f \homotopic f'$, $K: g \homotopic g'$, dann ist $H \ast K: f \ast g \homotopic f' \ast g'$.

Wir erhalten eine Verknüpfung $\cdot: \pi_n(X, x_0) \times \pi_n(X, x_0) \to \pi_n(X, x_0)$ durch $[f]\cdot [g] = [f\ast g]$.

Dies ist eine Gruppe: Assoziativität durch Umskalieren, neutrales Element $e = [\const]$, Inverse $\_f(s_1, \dotsc, s_n) := f(1-s_1, s_2, \dotsc, s_n)$.
Beweise wie für $n = 1$.

\begin{prop}
    Für $n \ge 2$ ist $\pi_n(X, x_0)$ abelsch.
    \begin{proof}
        Skizze.
    \end{proof}
\end{prop}

\begin{prop}
    Jede Überlagerung $p: (\tilde X, \tilde x_0) \to (X, x_0)$ induziert Isomorphismen $p_\#: \pi_n(\tilde X, \tilde x_n) \to \pi_n(X, x_0)$, $[f] \mapsto p_\#([f]) = [p \circ f]$ für $n \ge 2$.
    \begin{proof}
        Übung
    \end{proof}
\end{prop}

\begin{ex}
    \begin{itemize}
        \item
            $\pi_n(\R^n, 0) = 0$ für $n \ge 0$.
        \item
            \begin{math}
                \pi_n(\S^1, 1) = \begin{cases}
                    \Z & \text{für $n = 1$} \\
                    0 & \text{für $n \ge 2$}
                \end{cases}
            \end{math}
        \item
            Simpliziale Approximation, Abbildungsgrad
            \begin{math}
                \pi_n(\S^k, 1) = \begin{cases}
                    0 & \text{für $n < k$}, \\
                    \Z & \text{für $n = k$}, \\
                    ??? & \text{für $n > k$}.
                \end{cases}
            \end{math}
    \end{itemize}
\end{ex}

