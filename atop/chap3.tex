% Kapitel 3
\chapter{Höhere Homotopiegruppen}

\section{Homotopiegruppen}

Betrachte $X := \R^n \setminus B_1(0)$, $Y := \R^n$: die Fundamentalgruppe kann für $n = 2$ $X$ und $Y$ unterscheiden, für $n \ge 3$ nicht mehr.
Wir suchen ein Analogon um dies zu ermöglichen.

Sei $I^n = [0,1]^n \subset \R^n$, $\Boundary I^n$ sein Rand.
Zu jedem topologischen Raum $X$ mit Fußpunkt $x_0$ definieren wir
\begin{math}
    \pi_n(X, x_0)
    &= \Set{f: (I^n, \Boundary I^n, 0) \to (X, \{x_0\}, x_0) \text{ stetig}} / \text{ Homotopie} \\
    &= \Set{f: (I^n // \Boundary I^n, \ast) \to (X, x_0) \text{ stetig}} / \text{ Homotopie} \\
    &= \Set{f: (\S^n, \ast) \to (X, x_0) \text{ stetig}} / \text{ Homotopie}.
    &= [\S^n, \ast; X, x_0].
\end{math}
(für $n = 0, 1$ stimmt das überein mit den bisherigen Definitionen).

Verknüpfung:
Zu $f, g : (I^n, \Boundary I^n) \to (X, x_0)$ definieren wir $(f \ast g): (I^n, \Boundary I^n) \to (X, x_0)$ durch
\begin{math}
    (f \ast g)(s_1, \dotsc, s_n) = \begin{cases}
        f(2s_1, s_2, \dotsc, s_n) & \text{für $0 \le s \le \frac{1}{2}$}, \\
        g(2s_1 - 1, s_2, \dotsc, s_n) & \text{für $\frac{1}{2} \le s_1 \le 1$}.
    \end{cases}
\end{math}

Dies ist verträglich mit Homotopie:
$H: f \homotopic f'$, $K: g \homotopic g'$, dann ist $H \ast K: f \ast g \homotopic f' \ast g'$.

Wir erhalten eine Verknüpfung $\cdot: \pi_n(X, x_0) \times \pi_n(X, x_0) \to \pi_n(X, x_0)$ durch $[f]\cdot [g] = [f\ast g]$.

Dies ist eine Gruppe: Assoziativität durch Umskalieren, neutrales Element $e = [\const]$, Inverse $\_f(s_1, \dotsc, s_n) := f(1-s_1, s_2, \dotsc, s_n)$.
Beweise wie für $n = 1$.

\begin{prop}
    Für $n \ge 2$ ist $\pi_n(X, x_0)$ abelsch.
    \begin{proof}
        Skizze.
    \end{proof}
\end{prop}

\begin{prop}
    Jede Überlagerung $p: (\tilde X, \tilde x_0) \to (X, x_0)$ induziert Isomorphismen $p_\#: \pi_n(\tilde X, \tilde x_n) \to \pi_n(X, x_0)$, $[f] \mapsto p_\#([f]) = [p \circ f]$ für $n \ge 2$.
    \begin{proof}
        Übung
    \end{proof}
\end{prop}

\begin{ex}
    \begin{itemize}
        \item
            $\pi_n(\R^n, 0) = 0$ für $n \ge 0$.
        \item
            \begin{math}
                \pi_n(\S^1, 1) = \begin{cases}
                    \Z & \text{für $n = 1$} \\
                    0 & \text{für $n \ge 2$}
                \end{cases}
            \end{math}
        \item
            Simpliziale Approximation, Abbildungsgrad
            \begin{math}
                \pi_n(\S^k, 1) = \begin{cases}
                    0 & \text{für $n < k$}, \\
                    \Z & \text{für $n = k$}, \\
                    ??? & \text{für $n > k$}.
                \end{cases}
            \end{math}
    \end{itemize}
\end{ex}

\Timestamp{2015-12-18}

% §3.1 Homotopiegruppe

% §3.2 Relative Homotopiegruppe und lange exakte Homotopiesequenz

% §3.3 Faserungen & Faserbündel und lange exakte Homotopiesequenz

% §3.4 Einhängungssatz

Operation von $\pi_1$ auf $\pi_n$ ($n \ge 1$).
\begin{math}
    \pi_1(X, x_0) \times \pi_n(X, x_0) & \to \pi_n(X, x_0) \\
    ([\gamma], [\alpha]) & \mapsto [\gamma] \cdot [\alpha] := [\gamma \cdot \alpha]
\end{math}
(Skizze: Verkleinern von $\alpha$, Einspannen von $\gamma$, explizit mit „Karton“-Homöomorphismus).
Formal: $I^n \homeomorphic \Boundary I^{n+1} \setminus (\Int I^n) \times \Set{1}$.
Anschaulich: Verschiebung des „Ballons“ entlang $\gamma$.

Allgemeiner: $\Pi(X, x_0, x_1) \times \pi_n(X,x_1) \to \pi_n(X, x_0)$.

\begin{ex}
    Was sind $\pi_1$ und $\pi_2$ für folgenden Raum?
    $\S^1 \vee \S^2$ (Bouquet).
    \begin{math}
        \pi_1(X, x_0) & \isomorphic \Z \\
        \pi_2(X, x_0) & \isomorphic \Z^n
    \end{math}
    Betrachte dazu die universelle Überlagerung $(\tilde X, \tilde x_0) \to (X, x)$.
    \begin{math}
        \pi_1(\tilde X, \tilde x_0) &= \Set{1} \\
        \pi_2(\tilde X, \tilde x_0) &= \bigoplus_{i \in \Z} \Z = \Z^{(\Z)}.
    \end{math}
    Operation: $\pi_1(X, x_0) \times \pi_2(X, x_0) \to \pi_2(X, x_0)$ äquivalent zu $\Z \times \Z^{(\Z)} \to \Z^{(\Z)}$:
    \begin{math}
        k \cdot (a_i)_{i \in \Z} = (a_{i-k})_{i \in \Z}.
    \end{math}
\end{ex}

\begin{note}
    Für $n = 1$ erhalten wir die Konjugation:
    \begin{math}
        [\gamma] \cdot [\alpha] = \gamma \alpha \_\gamma.
    \end{math}
\end{note}

\begin{df}
    Ein Raum $(X, x_0)$ heißt \emphdef{abelsch}, wenn alle Operationen $\pi_1 \times \pi_n \to \pi_n$ trivial sind.
\end{df}

\begin{ex}
    $\pi_1 = \Set{1}$, aber auch alle topologische Gruppen (Fundamentalgruppe abelsch).
    \begin{proof}
        Übung!
    \end{proof}
\end{ex}


% §3.2
\section{Relative Homotopiegruppen}

\begin{df}
    Sei $x_0 \in A \subset X$.
    Wir definieren die relative Homotopiegruppe $\pi_n(X, A, x_0)$ wie folgt:

    Wir identifizieren $\R^{n-1}$ mit $\R^{n-1} \times \Set{0}$ in $\R^n$.
    Damit $\R^0 \subset \R^1 \subset \R^2 \subset \dotsb$.

    Sei $I^n = [0,1]^n$ und $I^{n-1} = I^n \cap \R^{n-1}$ und $I^{n-1} = I^n \cap \R^{n-1}$.
    Wir identifizieren also $I^{n-1}$ mit $I^{n-1} \times \Set{0}$.

    Sei $J^{n-1} := \Boundary I^n \setminus \Int I^{n-1}$.
    Setze
    \begin{math}
        \pi_n(X, A, x_0)
        &= \Set{\alpha: (I^n, \Boundary I^n, J^{n-1}) \to (X, A, x_0)} / \text{Homotopie}.
    \end{math}
    Verknüpfung wie zuvor.

    \begin{note}
        \begin{itemize}
            \item
                $\pi_0(X, A, x_0)$ wird hier nicht definiert.
            \item
                Für $n = 1$ ist
                \begin{math}
                    \pi_1(X, A, x_0) = \Set{([0,1], \Set{0,1}, \Set{1}) \to (X, A, x_0)} / \text{Homotopie}.
                \end{math}
                als Menge definiert, trägt jedoch keine sinnvolle Verknüpfung und ist damit keine Gruppe
            \item
                Für $n \ge 2$ ist $\pi_n(X, A, x_0)$ eine Gruppe.
            \item
                Für $n \ge 3$ ist $\pi_n(X, A, x_0)$ abelsch.
            \item
                Es ist $(I^n, \Boundary I^n, J^{n-1}) // J^{n-1} \homeomorphic (\D^n, \S^{n-1}, \ast)$.
                Bildlich: Skizze
            \item
                $\pi_n(X, x_ ) = \pi_n(X, \Set{x_0}, x_0)$.
                Die absolute Homotopiegruppen sind Spezialfälle der relative Homotopiegruppen, mit etwas mehr Struktur.
        \end{itemize}
    \end{note}
\end{df}

\begin{lem}[Kompressionslemma]
    Eine Abbildung $\alpha: (I^n, \Boundary I^n, J^{n-1}) \to (X, A, x_0)$ ist genau dann nullhomotop, also Null in $\pi_n(X, A, x_0)$, wenn $\alpha$ homotop ist relativ $\Boundary I^{n-1}$ (d.h. bleibt festgehalten bei der Homotopie) zu einer Abbildung $\alpha': (I^n, \Boundary I^n, J^{n-1}) \to (A, A, x_0)$. 
    \begin{proof}
        \begin{seg}{\ProofImplication*}
            Die $n$-Schleife $\alpha'$ ist nullhomotop vermöge $H: I^n \times I \to A \subset X$ mit $H_s(t) = H(t,s)$,
            \begin{math}
                H(t_1, \dotsc, t_n, s) = \alpha'(t_1, \dotsc, t_{n-1}, 1 - s(1-t_n)).
            \end{math}
            Es gilt $H_0 = 1_{x_0}$, $H_1 = \alpha'$, sowie $H_s(\Boundary I^n) \subset A$, $H_s(J^{n-1}) = \Set{x_0}$.

            Ist also $K: \alpha \sim \alpha'$ relativ $\Boundary I^{n-1}$, adnn folgt $K \ast H: \alpha \sim 1_{x_0}$ wie behauptet.
        \end{seg}
        \begin{seg}{\ProofImplication}
            Sei $H: I^n \times I \to X$, $H_s(t) = H(t,s)$ mit $H_0 = \alpha$ nach $H_1 = 1_{x_0}$ mit $H_s(\Boundary I^n) \subset A$, $H_s(J^{n-1}) = \Set{x_0}$ für alle $s \in I = [0,1]$.

            Wir suchen $K: I^n \times I \to X$ mit $K_0 = \alpha$ und $K_1 = \alpha'$, $\alpha'(I^n) \subset A$ und $K_s(t) = K_0(t)$ für alle $t \in \Boundary I^n$, $s \in I$.
            Skizze!
        \end{seg}
    \end{proof}
\end{lem}


Sei $x_0 \subset A \subset X$.
Die Inklusion $i: (A, x_0) \injto (X, x_0)$ und $j: (X, x_0) = (X, \Set{x_0}, x_0) \injto (X, A, x_0)$ induzieren $i_\#: \pi_n(A,x_0) \to \pi_n(X,x_0)$, $[\alpha] \mapsto [i \circ \alpha]$ und $j_\#: \pi_n(X, x_0) \to \pi_n(X, A, x_0)$, $[\beta] \mapsto [j \circ \beta]$.
Zudem haben wir $\partial_\#: \pi_{n+1}(X, A, x_0) \to \pi_n(A, x_0)$ mit $\partial_\#[\alpha] = [\alpha|_{I^n}]$.
Dies sind Gruppenhomomorphismen (für $n \ge 1$).

\begin{st}
    Die Sequenz
    \begin{math}
        \begin{tikzcd}
            & \dotsc \ar[r,"i_\#"] & \pi_3(X, A, x_0) \ar[dll,"\partial_\#"] \\
            \pi_2(A,x_0) \ar[r,"i_\#"] & \pi_2(X, x_0) \ar[r,"i_\#"] & \pi_2(X, A, x_0) \ar[dll,"\partial_\#"] \\
            \pi_1(A,x_0) \ar[r,"i_\#"] & \pi_1(X, x_0) \ar[r,"i_\#"] & \pi_1(X, A, x_0) \ar[dll,"\partial_\#"] \\
            \pi_0(A,x_0) \ar[r,"i_\#"] & \pi_0(X, x_0) \ar[r,"i_\#"] & \pi_0(X, A, x_0)
        \end{tikzcd}
    \end{math}
    ist exakt, d.h. an jeder Stelle gilt „Bild = Kern“.
    \begin{proof}
        Zeige Exaktheit an jeder Stelle:
        \paragraph{Exaktheit in $\pi_n(X,x_0)$}
        \begin{itemize}
            \item
                $\im(i_\#) \subset \ker(j_\#)$, d.h. $j_\# \circ i_\# = 1$ ist klar (Homotopie wie in vorigem Lemma)
            \item
                $\ker(j_\#) \subset \im(i_\#)$:
                Sei $\beta: (I^n, \Boundary I^n, J^{n-1}) \to (X, \Set{x_0}, x_0)$ mit $j_\#([\beta]) = 1$, d.h. als $n$-Schleife in $(X, A, x_0)$ ist $\beta$ nullhomotop.
                Nach dem Kompressionslemma ist $\beta$ homotop relativ $\Boundary I^n$ zu $\beta': (I^n, \Boundary I^n) \to (A, x_0)$.
                Damit gilt $[\beta] = i_\# [\beta']$.
        \end{itemize}
        \paragraph{Exaktheit in $\pi_n(A, x_0)$}
        \begin{itemize}
            \item
                $\im \partial_\# \subset \ker i_\#$, d.h. $i_\# \circ \partial_\# = 1$:
                Für $\alpha: (I^{n+1}, \Boundary I^{n+1}, J^n) \to (X, A, x_0)$ und $\beta = \alpha|_{I^n}$ gilt:
                In $(X, x_0)$ ist $i \circ \beta$ nullhomotop vermöge $\alpha: i \circ \beta \sim 1_{x_0}$.
            \item
                $\ker(i_\#) \subset \ker \partial_\#$:
                Genauso, umgekehrt.
        \end{itemize}
        \paragraph{Exaktheit in $\pi_n(X, A, x_0)$}
        \begin{itemize}
            \item
                $\im j_\# \subset \ker \partial_\#$, d.h. $\partial_\# \circ i_\# = 1$:
                Für $\beta: (I^n, \Boundary I^n, J^{n-1}) \to (X, \Set{x_0}, x_0)$ gilt $(j \circ \beta)|_{I^{n-1}} = 1_{x_0}$.
            \item
                $\ker \partial_\# \subset j_\#$.
                Sei $\alpha: (I^n, \Boundary I^n, J^{n-1}) \to (X, A, x_0)$ mit $\Boundary_\#[\alpha] = 1$.
                D.h. $H: \alpha|_{I^{n-1}} \sim 1_{x_0}$ in $(A, x_0)$.
                Dank Kompressionslemma ist $\alpha$ homotop in $(X, A, x_0)$ zu $\alpha'(I^n, \Boundary I^n, J^{n-1}) \to (X, \Set{x_0}, x_0)$.
                Damit ist $j_\#[\alpha'] = [\alpha]$.
        \end{itemize}
    \end{proof}
    \begin{note}
        Die Exaktheit gilt allgemeiner für $x_0 \in B \subset A \subset X$ und
        \begin{math}
            \begin{tikzcd}
                \pi_n(A, B, x_0) \xto{i_\#} \pi_n(X, B, x_0) \xto{j_\#} \pi_n(X, A, x_0) \xto{\partial_\#} \pi_{n-1}(A, B, x_0).
            \end{tikzcd}
        \end{math}
        Beweis genauso, genauer hinschauen.
    \end{note}
\end{st}

% §3.3
\section[Faserungen, Faserbündel, exakte Homotopiesequenzen]{Faserungen, Faserbündel und lange exakte Homotopiesequenzen}

\begin{df}
    Ein \emphdef{Bündel} $(E, p, B)$ ist eine stetige Surjektion $p: E \surto B$.
    \begin{math}
        \begin{tikzcd}
            E = \bigcup_{x\in B} p^{-1}(x) \ar[d] \\
            B
        \end{tikzcd}.
    \end{math}
\end{df}

Bündel über den Basisraum $B$ bilden eine Kategorie
\begin{math}
    \begin{tikzcd}
        E \ar[rr,"f"] \ar[rd,"p"] && E' \ar[ld,"p'"] \\
        & B
    \end{tikzcd}
    \begin{tikzcd}
        E \ar[r,"f"] \ar[rd,"p"] & E' \ar[r,"g"] \ar[d,"p'"] & E'' \ar[ld,"p''"] \\
        & B
    \end{tikzcd}
    \begin{tikzcd}
        E \ar[rr,"\id_E"] \ar[dr,"p"] && E \ar[dl,"p"] \\ 
        & B
    \end{tikzcd}
\end{math}

\begin{df}
    $(E, p, B)$ heißt \emphdef{trivial} mit Faser $F$, wenn es homöomorph ist zur Projektion des Produktbündels $B \times F \xto{q} B$, $q(x,y) = x$, d.h.
    \begin{math}
        \begin{tikzcd}
            E \ar[rr,"h",shift left] \ar[dr,"p"] & & B \times F \ar[ll,"k",shift left] \ar[ld,"q"] \\
            & B
        \end{tikzcd}
    \end{math}
    Es existieren $(h,k): E \homeomorphic B \times F$ mit $q \circ h = p$, $p \circ k = q$.
\end{df}

\begin{ex}
    Triviale und nicht-triviale Bündel im $\R^2$
\end{ex}

\begin{df}
    Ein Bündel $p: E \to B$ heißt \emphdef{lokal trivial}, wenn jeder Punkt $b \in B$ eine offene Umgebung $U \subset B$ besitzt, sodass die Einschränkung $p \over U = p |_{p^{-1}(U)}^U : p^{-1}(U) \to U$ trivial ist.

    Spezieller heißt $p$ \emphdef{Faserbündel} mit Faser $F$, wenn zu $b \in B$ eine Umgebung $U \subset B$ existiert, sodass $p \over U$ homöomorph ist zum Produktbündel $U \times F \to U$.
\end{df}

\begin{ex}
    Ein Faserbündel mit diskreter Faser $F$ ist eine Überlagerung.
    Jede Überlagerung mit konstanter Blätterzahl $|F|$ ist ein Faserbündel.
\end{ex}

\begin{ex}
    Zylinder und Möbiusband über $\S^1$.
\end{ex}

\begin{ex}
    Tangentialbündel $p: \operatorname{T} M \to M$.
    Insbesondere $\operatorname{T} \S^n \to \S^n$.
    Für $n = 2, 4, 6, \dotsc$ ist dies nicht-trivial.
\end{ex}

\begin{lem}
    Seien $t_- < t_0 < t_+$ und $p: B = A \times [t_-, t_+]$, $B_- = A \times [t_-, t_0]$, $B_+ = A \times [t_0, t_+]$.
    Also $B = B_- \cup B_+$ und $B_0 := B_- \cap B_+ = A \times \Set{t_0}$.
    Sei $p: E \to B$ ein Bündel.
    Die eingeschränkten Bündel $p_\pm: E_\pm \to B_\pm$ seien trivial.

    Dann ist $p$ trivial.
    \begin{proof}
        \begin{math}
            \begin{tikzcd}[column sep=small]
                & E_- \ar[dd,"p_-"] && E_0 \ar[rr,inj] \ar[ll,inj] \ar[dd] && E_+ \ar[dd,"p_+"] & \\
                B_- \times F \ar[ru,"\tau_-","\homeomorphic"'] \ar[rd,"q_-"'] && B_0 \times F \ar[ll,inj,crossing over] \ar[rr,shift left,"\sigma_+^{-1} \circ \sigma_-",crossing over] \ar[ru,"\sigma_-"] \ar[rd,"q_0"] && B_0 \times F \ar[ll,shift left,crossing over] \ar[rr,inj,crossing over] \ar[lu,"\sigma_+"'] \ar[ld,"q_0"] && B_+ \times F \ar[lu,"\tau_+"'] \ar[ld,"q_+"] \\
                & B_- && B_0 \ar[rr,inj] \ar[ll,inj] && B_+ &
            \end{tikzcd}
%           \begin{tikzcd}
%                &E_- && E_0 && E_+ \\
%                B_- \times F \ar[ru,"\tau_-","\homeomorphic"'] \ar[rd,"q_-"] && B_0 \times F \ar[d,"p_-"] & & B_0 \times F \ar[d,"p_0"] & \ar[d,"p_+"] \\
%                &B_- && B_0 && B_+
%            \end{tikzcd}
        \end{math}
        Ersetze $\tau_+$ durch $\tau_+'(a,t,y) = \tau_+(a,t,\sigma(a,y))$ mit $\sigma_+^{-1} \circ \sigma_-: B_0 \times F \to B_0 \times F$, $(a,y) \mapsto (a, \sigma(a,y))$.
        Verklebe $\tau_-$ und $\tau_+'$ zu einer Trivialisierung $\tau: B \times F \xto[homeomorphic] E$ mit $p \circ \tau = q$.
    \end{proof}
\end{lem}

\begin{st}
    Jedes lokal triviale Bündel $p: E \to B = [0,1]^n$ ist trivial.
    \begin{proof}
        Es existiert eine offene Überdeckung $B = \bigcup_{i \in I} U_i$ mit $p\over U_i$ trivial.
        Unterteile $B = [0,1]^n$ in $k^n$ gleich große Würfel, sodass jeder in einem $U_i$ liegt (Kompaktheit, Lebesgue-Zahl).
        Über jedem Würfel ist $p$ trivial.
        Wiederholte Anwendung des Lemmas liefert denn Satz.
    \end{proof}
\end{st}

\subsection{Zurückziehen von Bündeln}

\begin{math}
    \begin{tikzcd}
        E' \ar[d,"p'"] \ar[r,"f"] & E \ar[d,"p"] \\
        B' \ar[r,"g"] & B
    \end{tikzcd}
\end{math}

\begin{df}
    Sei $p: E \to B$ ein Bündel und $g: B' \to B$ stetig.
    Setze
    \begin{math}
        E' := \Set{(b', e) \in B' \times E & g(b') = p(e)}
    \end{math}
    und $p': E' \to B'$, $p'(b', e') = b'$, sowie $f: E' \to E$, $f(b', e) = e$.
    Schreibweise: $p' = g^*(p)$.
\end{df}

\begin{ex}
    Spezialfall $B' = [0,1]^n$:
    \begin{math}
        \begin{tikzcd}~
            [0,1]^n \times F & E' \ar[r,"f"] \ar[d,"p'"] & E \ar[d,"p"] \\
            &~[0,1]^n \ar[r,"g"] \ar[ru,"\exists \tilde g",dashed] & B
        \end{tikzcd}
    \end{math}
    Es existiert mindestens eine Hochhebung $\tilde g$.
\end{ex}

