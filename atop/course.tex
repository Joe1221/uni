% This work is licensed under the Creative Commons
% Attribution-NonCommercial-ShareAlike 3.0 Unported License. To view a copy of
% this license, visit http://creativecommons.org/licenses/by-nc-sa/3.0/ or send
% a letter to Creative Commons, 444 Castro Street, Suite 900, Mountain View,
% California, 94041, USA.

\documentclass{mycourse}

\ExplSyntaxOn

%\cs_gset_nopar:Npn \thechapter { \Alph { chapter } }
%\cs_gset_nopar:Npn \thesection { \thechapter \relax \arabic { section } }

\DeclareDocumentCommand \Cat { m } { \underline{\mathtt{#1}} }
\DeclareDocumentCommand \S { } { \mathbb{S} }
\DeclareDocumentCommand \H { } { \mathbb{H} }
\DeclareDocumentCommand \id { } { \Id }
\DeclareDocumentCommand \dunion { } { \sqcup }
\DeclareDocumentCommand \B { } { \mathbb{B} }
\DeclareDocumentCommand \D { } { \mathbb{D} }
\DeclareDocumentCommand \RP { } { \mathbb{RP} }
\DeclareDocumentCommand \homeomorphic { } { \cong }
\DeclareDocumentCommand \homotopic { } { \simeq }
\DeclareDocumentCommand \normdiv { } { \mathrel{\triangleleft} }
\DeclareDocumentCommand \ops { G{} } { \xto[cap]{#1} }
\DeclareDocumentCommand \b { } { \flat }
\DeclareDocumentCommand \bs { } { \mathrel{\backslash} }
\DeclareDocumentCommand \quot { } { \mathrm{quot} }
\DeclareDocumentCommand \semiprod { G{} } { \mathrel{\stack{#1}{\rtimes}} }
\DeclareDocumentCommand \twprod { G{} } { \mathrel{\stack{#1}{\rtimes}} }
\DeclareDocumentCommand \rang { } { \operatorname{rang} }
\DeclareDocumentCommand \homto { G{} } { \xto[homeomorphic]{#1} }
\DeclareDocumentCommand \Homeo { } { \operatorname{Homeo} }
\DeclareDocumentCommand \Int { } { \operatorname{Int} }

\DeclareDocumentCommand \over { } { \divs }

\ExplSyntaxOff
\tikzset {
    commutative diagrams/ops/.style = { bend left, shift right=1.5ex },
}
\ExplSyntaxOn


\DeclareDocumentCommand { \GenMonoid } { om } {
    \tl_clear_new:N \l_content_tl
    \tl_set:Nn \l_content_tl { #2 }

    \tl_replace_all:Nnn \l_content_tl { & } { \c_mymath_set_delim_tl }

    \tl_if_in:NnTF \l_content_tl { \c_mymath_set_delim_tl } {
	\mathchoice {
	    \left[ \big. \mkern1mu \mymath_set_vertbar:nn{\l_content_tl}{\middle|} \mkern2mu \right]
	} {
	    [ \mkern1mu \mymath_set_vertbar:nn{\l_content_tl}{\mkern-2mu | \mkern-2mu} \mkern1mu ]
	} {
	    [ \mymath_set_vertbar:nn{\l_content_tl}{\mkern-1mu | \mkern-1mu} ]
	} {
	    [ \mymath_set_vertbar:nn{\l_content_tl}{\mkern-1mu | \mkern-1mu} ]
	}
    } {
        \mathchoice {
            \left[ \big.\mkern-1mu \l_content_tl \right]
        } {
            [ \l_content_tl ]
        } {
            [ \l_content_tl ]
        } {
            [ \l_content_tl ]
        }
    }
}

\DeclareDocumentCommand { \Gen } { om } {
    \tl_clear_new:N \l_content_tl
    \tl_set:Nn \l_content_tl { #2 }

    \tl_replace_all:Nnn \l_content_tl { & } { \c_mymath_set_delim_tl }

    \tl_if_in:NnTF \l_content_tl { \c_mymath_set_delim_tl } {
	\mathchoice {
	    \left\< \big. \mkern1mu \mymath_set_vertbar:nn{\l_content_tl}{\middle|} \mkern2mu \right\>
	} {
	    \< \mkern1mu \mymath_set_vertbar:nn{\l_content_tl}{\mkern-2mu | \mkern-2mu} \mkern1mu \>
	} {
	    \< \mymath_set_vertbar:nn{\l_content_tl}{\mkern-1mu | \mkern-1mu} \>
	} {
	    \< \mymath_set_vertbar:nn{\l_content_tl}{\mkern-1mu | \mkern-1mu} \>
	}
    } {
        \mathchoice {
            \left\< \big.\mkern-1mu \l_content_tl \right\>
        } {
            \< \l_content_tl \>
        } {
            \< \l_content_tl \>
        } {
            \< \l_content_tl \>
        }
    }
}

\ExplSyntaxOff

\title{Algebraische Topologie}

\begin{document}

\maketitle

\tableofcontents

\Timestamp{2015-10-13}

\section*{Was ist Differentialgeometrie?}

\begin{itemize}
    \item
        Ein mathematisches Teilgebiet, in dem geometrische Objekte mit Hilfe von (mehrdimensionaler) Differential- und Integralrechnung studiert werden.
    \item
        „Die Lehre von der Krümmung“
    \item
        Studium glatter Objekte (= glatte Mannigfaltigkeiten) und geometrischer Strukturen.
    \item
        Verallgemeinerung der elementaren Differentialgeometrie, d.h. des Studiums von Kurven und Flächen in der Ebene und dem dreidimensionalen Raum, ihrer Krümmung und globalen Eigenschaften.
\end{itemize}

Angrenzende Teilgebiete:
\begin{itemize}
    \item
        Topologie: insbesondere die Differentialtopologie.
    \item
        Differentialgleichungen und -ungleichungen: z.B. Geodätengleichung, Krümmungsbedingungen (z.B. pos/neg).
    \item
        Liegruppen: die Gruppe der Isometrien einer Riemannschen Mannigfaltigkeit ist eine Liegruppe, Beschreibung von homogenen und symmetrischen Räumen.
    \item
        Variationsrechnung: z.B. Geodätische, Minimalflächen.
    \item
        Funktionentheorie: komplexe Analysis, z.B. Weierstraß-Darstellung einer Minimalfläche.
    \item
        Algebraische Geometrie
    \item
        Kontrolltheorie, etc.
\end{itemize}

Starke Bezüge zur Physik:
Einsteins Allgemeine Relativitätstheorie wird beschrieben mit Begriffen der Differentialgeometrie: die \emph{Raumzeit} ist eine gekrümmte $4$-dimensionale pseudo-riemannsche Mannigfaltigkeit.

Beispiele für typische Sätze/Probleme aus der Differentialgeometrie:

\begin{st}[Gauß-Bonnet]
    Sei $M$ eine zweidimensionale, kompakte, orienterbare riemannsche Mannigfaltigkeit.
    Dann gilt für das Integral über die Gaußkrümmung:
    \begin{math}
        \int_M \kappa = 2 \pi \chi(M),
    \end{math}
    wobei $\chi(M)$ die Eulercharakteristik von $M$ ist.
\end{st}

\begin{prob}
    Klassifikation von positiv gekrümmten riemannschen Mannigfaltigkeiten (bis jetzt nicht vollständig bekannt).
\end{prob}





% Kapitel 1
\chapter{Fundamentalgruppe}

\begin{df}
    Sei $X$ ein topologischer Raum, $x_0 \in X$ ein Punkt.
    \begin{math}
        P(X) &= \scr C([0,1], X), \\
        P(X, a, b) &= \Set{\gamma \in P(X) & \gamma(0) = a, \gamma(1) = b}.
    \end{math}
    Spezielle Wege, bzw Wegoperationen:
    \begin{itemize}
        \item
            $1_a \in P(X, a, a)$, $1_a(t) := a$ für $0 \le t \le 1$,
        \item
            $\_\argdot: P(X, a, b) \to P(X, b, a)$, $\gamma \mapsto \_\gamma$, $\_\gamma(t) := \gamma(1-t)$,
        \item
            $\ast: P(X, a,b) \times P(X, b,c) \to P(X,a,c)$,
            \begin{math}
                (\gamma_1 \ast \gamma_2)(t) := \begin{cases}
                    \gamma_1(2t) & \text{für $0 \le t \le \frac{1}{2}$} \\
                    \gamma_2(2t - 1) & \text{für $\frac{1}{2} \le t \le 1$}
                \end{cases}
            \end{math}
    \end{itemize}
    Zwei Wege $\alpha, \alpha' \in P(X,a,b)$ heißen \emphdef{äquivalent}, genauer \emphdef{homotop bei festen Endpunkten}, wenn eine stetige Abbildung $H: [0,1] \times [0,1] \to X$ existiert mit $H(0, t) = \alpha(t)$, $H(1, t) = \alpha'(t)$, sowie $H(s, 0) = a$, $H(s, 1) = b$ für alle $s,t \in [0,1]$.
    Wir schreiben dann $H: \alpha \sim \alpha'$ oder kurz $\alpha \sim \alpha'$.
\end{df}

\begin{prop}
    \begin{itemize}
        \item
            Die Äquivalenz $\sim$ ist eine Äquivalenzrelation.
        \item
            Aus $\alpha \sim \beta$ folgt $\_\alpha \sim \_\beta$.
        \item
            Aus $\alpha \sim \alpha'$ und $\beta \sim \beta'$ folgt $\alpha \ast \beta \sim \alpha' \sim \beta'$.
    \end{itemize}
\end{prop}

Wir erhalten wohldefinierte Abbildungen auf $\Pi(X, a, b) := P(X, a, b) / \sim$ durch
\begin{itemize}
    \item
        $\_\argdot: \Pi(X,a,b) \to \Pi(X,a,b)$, $\_{[\gamma]} := [\_\gamma]$.
    \item
        $\ast: \Pi(X,a,b) \times \Pi(X,b,c) \to \Pi(X,a,c)$, $[\alpha] \ast [\beta] := [\alpha \ast \beta]$.
\end{itemize}

\begin{st}
    Jeder topologische Raum $X$ definiert so seine \emphdef{Wegekategorie} $\Pi(X)$ (auch \emphdef{Fundamentalgruppoid} genannt).
    \begin{enumerate}[a)]
        \item
            Objekte sind die punkte $a, b, c, \dotsc \in X$,
        \item
            Morphismen $[\alpha]: a \to b$ sind Wegeklassen von $a$ nach $b$.
        \item
            Verknüpfung $\ast$ ist die Konkatenation wie oben.
    \end{enumerate}
    Die Verknüpfung erfüllt
    \begin{enumerate}[1)]
        \item
            Identität: Für $\alpha: a \to b$ gilt
            \begin{math}
                1_a \ast \alpha \sim \alpha \sim \alpha \ast 1_b,
            \end{math}
            also
            \begin{math}
                [1_a] \sim [\alpha] = [\alpha] = [\alpha] \ast [1_b].
            \end{math}
        \item
            Inversion: Für $\alpha: a \to b$ und $\_\alpha: b \to a$ gilt
            \begin{math}
                \alpha \ast \_\alpha &\sim 1_a, &
                \_\alpha \ast \alpha &\sim 1_b
            \end{math}
            also
            \begin{math}
                [\alpha] \ast [\_\alpha] &= [1_a], &
                [\_\alpha] \ast [\alpha] &= [1_b],
            \end{math}
        \item
            Assoziativität: Für $a \xto{\alpha} b \xto{\beta} c \xto{\gamma} d$ gilt $(\alpha \ast \beta) \ast \gamma \sim \alpha \ast (\beta \ast \gamma)$, also
            \begin{math}
                ([\alpha] \ast [\beta]) \ast [\gamma] = [\alpha] \ast ([\beta] \ast [\gamma]).
            \end{math}
    \end{enumerate}
    \begin{proof}
        Skizzenbeweis.
    \end{proof}
\end{st}

\begin{st}
    Jede stetige Abbildung $f: X \to Y$ induziert einen Funktor
    \begin{math}
        f_\#: \Pi(X) &\to \Pi(Y) \\
        a &\mapsto f(a) \\
        [\alpha: a \to b] &\mapsto [f \circ \alpha: f(a) \to f(b)]
    \end{math}
\end{st}

\begin{df}
    Vom Fundamentalgruppoid zur Fundamentalgruppe, definiere
    \begin{math}
        \pi_1(X, x_0) := \Pi(X, x_0, x_0)
        = \frac{\Set{\text{Schleifen $\alpha: ([0,1], \Set{0,1}) \to (X,x_0)$}}}{\text{Homotopie relativ $\Set{0,1}$}}.
    \end{math}
    Dies ist eine Gruppe (Übung) mit der Verknüpfung $[\alpha] \ast [\beta] = [\alpha \ast \beta]$.

    Jede stetige Abbildung $f: (X, x_0) \to (Y, y_0)$ induziert einen Gruppenhomomorphismus $f_\# = \pi_1(f): \pi_1(X, x_0) \to \pi_1(Y, y_0)$ mit $[\alpha] \mapsto f_\#([\alpha]) = [f \circ \alpha]$.
    Wir erhalten einen Funktor
    \begin{math}
        \pi_1: \Cat{Top}_* &\to \Cat{Grp} \\
        (X, x_0) &\mapsto \pi_1(X, x_0) \\
        (f: (X,x_0) \to (Y, y_0)) &\mapsto (\pi_1(f): \pi_1(X, x_0) \to \pi_1(Y, y_0)).
    \end{math}
\end{df}

\begin{ex}
    Sei $X = \R^n$ oder $X \subset \R^n$ konvex oder sternförmig bezüglich $x_0$.
    Dann ist $X$ wegzusammenhängend, d.h. $\pi_0(X) = \Set{[x_0]}$, und sogar einfach zusammenhängend, d.h. zudem
    \begin{math}
        \pi_1(X, x_0) = \Set{[1_{x_0}]},
    \end{math}
    kurz $\pi_1(X, x_0) = \Set{1}$.
    \begin{proof}
        Zu $\alpha: [0,1] \to X$ betrachte $H(s, t) = (1-s)\alpha(t) + s x_0$.
        $H: [0,1] \times [0,1] \to X$ ist eine Abbildung, da $X$ sternförmig ist.
        Sie ist stetig und erfüllt $H: \alpha \sim 1_{x_0}$.
    \end{proof}
\end{ex}

\paragraph{Offene Mengen $X \subset \R^n$ und polygonale Fundamentalgruppe}

Sei $X \subset \R^n$ und $x_0 \in X$. Wir definieren die polygonale Fundamentalgruppe
\begin{math}
    \pi_1^{\text{pl}}(X,x_0) := \Pi^{\text{pl}}(X, x_0, x_0)
    = \frac{\Set{\text{geschlossene Polygonzüge in $(X, x_0)$}}}{\text{polygonale Homotopie in $X$}}.
\end{math}

\begin{st}
    Sei $X \subset \R^n$ offen und $x_0 \in X$.
    Wir haben einen Gruppenisomorphismus
    \begin{math}
        \phi: \pi_1^{\text{pl}}(X, x_0) \to \pi_1(X, x_0)
    \end{math}
    \begin{proof}
        (Skizze: stetiger/polygonaler Weg)
        Surjektivität: Jede stetige Abbildung lässt sich beliebig genau durch Polygone approximieren.
        Injektivität: Stetige Homotopie lässt sich durch polygonale Homotopie approximieren.
    \end{proof}
\end{st}

\begin{ex}
    Sei $X := \R^2 \setminus \Set{0}$, $x_0 := (1, 0)$ und $\gamma$ ein geschlossener polygonaler Weg in $X$ von $x_0$ (Skizze).
    Durch zählen der Übergänge über die negative reelle Achse erhalten wir $\deg: \pi_1^{\text{pl}}(X, x_0) \to \Z$.
    Dies ist ein Gruppenisomorphismus
    \begin{proof}
        Wohldefiniertheit: Übergang von Wegen zu Wegeklassen.        
        Homomorphismus.
        Surjektivität: Konstruktion.
        Injektivität: Umlaufzahl $0$ betrachten: neg/pos Übergänge eliminieren (Punkte sternförmig um $x_0$).
    \end{proof}
    Kurz:
    \begin{math}
        \deg: \pi_1(X, x_0) \isomorphic \pi_1^{\text{pl}}(X, x_0) \isomorphic \Z.
    \end{math}
\end{ex}

\begin{ex}
    Sei $X := \C \setminus \Set{0, -1, \dotsc, 1 - n}$, $x_0 = 1$ (Skizze mit Weg).
    Kodiere Übergänge: $s_1, \dotsc, s_n$.

    Wir erhalten $\phi: \pi_1^{\text{pl}}(X, x_0) \to \Gen{ s_1, \dotsc, s_n & - }$.
    Dies ist ein Gruppenisomorphismus.
    \begin{proof}
        Wohldefiniertheit: Übergang von Wegen zu Wegeklassen: Kürzung.
        Homomorphismus: klar.
        Surjektivität: Konstruktion.
        Injektivität: Betrachte Wege, die auf $1$ abgebildet werden, Induktion über Wortlänge durch Kürzen.
    \end{proof}
    \begin{note}
        Die Gruppe ist für $n \ge 2$ nicht kommutativ!
    \end{note}
\end{ex}

\paragraph{Präsentation von Gruppen durch Erzeuger und Relationen}

Kurzfassung: Sei $A$ eine Menge. $A^* := \bigcup_{n \in \N} A^n$ ist die Menge aller Wörter über dem Alphabet $A$.
Für $n = 0$ ist $e = ()$ das leere Wort.
Für $n = 1$ identifizieren wir $(a) \in A^*$ mit $a \in A$.

Die Verkettung $\circ: A^* \times A^*$ ist gegeben durch die Konkatenation der Wörter
\begin{math}
    (a_1, \dotsc, a_n)(a_1', \dotsc, a_m') := (a_1, \dotsc, a_m, a_1', \dotsc, a_n').
\end{math}
Damit ist $(A^*, \circ, e)$ ein Monoid, genannt das \emphdef[freies Monoid]{freie Monoid} über $A$.

Wir wollen Relationen der Form $w_1 = w_2$ einführen.
Hierzu sei $K \subset A^* \times A^*$.
Auf $A^*$ sei $\equiv$ die Äquivalenzrelation, die erzeugt wird durch die elementaren Umformungen
\begin{math}
    u \circ w_1 \circ v \equiv u \circ w_2 \circ v, && \text{für $(w_1, w_2) \in K$},
\end{math}
Diese Kongruenz ist verträglich mit $\circ$, d.h. $u \equiv u'$ und $v \equiv v'$, dann ist $u \circ v \equiv u' \circ v'$.

Auf $Q := A^* / K := A^* / \equiv$ erhalten wir $\argdot: Q \times Q \to Q$, $[u] \cdot [v] := [u \circ v]$.
Damit ist auch $(Q, \cdot, [e])$ ein Monoid.


\begin{df}
    Das durch $(A, K)$ \emphdef{präsentierte Monoid} ist
    \begin{math}
        \GenMonoid{A & K} := A^* / K.
    \end{math}
\end{df}

\begin{ex}
    \begin{itemize}
        \item
            \begin{math}
                (N = \GenMonoid{a & -}, \cdot) &\xto[homeomorphic] (\N, +) \\
                a^n &\mapsto n \\
                a^n &\mapsfrom n
            \end{math}
        \item
            \begin{math}
                \GenMonoid{a,b & -}
                = \Set{e, a, b, aa, ab, ba, bb, \dotsc}
            \end{math}
        \item
            \begin{math}
                C = \GenMonoid{s^+, s^- & s^+s^- = e, s^-s^+ = e},
            \end{math}
            d.h. $A = \Set{s^+, s^-}$, $K = \Set{(s^+s^-, e), (s^-s^+, e)}$.
            Dies ist eine Gruppe.
            Definiere
            \begin{math}
                \phi: (\Z, +) &\to (C, \cdot) \\
                k &\mapsto \begin{cases}
                    (s^+)^k & \text{für $k > 0$}, \\
                    e & \text{für $k = 0$}, \\
                    (s^-)^{-k} & \text{für $k < 0$}.
                \end{cases}
            \end{math}
            Dies ist ein Gruppenhomomorphismus, surjektiv (auf Wortklasssen).
            Inverse:
            \begin{math}
                \psi: (C, \cdot) &\to (\Z, +), \\
                s^{\eps_1} s^{\eps_2} \dotsb s^{\eps_l} &\mapsto \eps_1 + \dotsb + \eps_l.
            \end{math}
            Dies ist wohldefiniert, Gruppenhomomorphismus, surjektiv.
            Es gilt $\psi \circ \phi: \id_\Z$, aber auch $\phi \circ \psi = \id_C$.
        \item
            $C_n := C_{n, 0} := \GenMonoid{a & a^n = 1}$ (Skizze: Kreis).
            Es gilt $(C_n, \cdot) \isomorphic (\Z / n, +)$.
        \item
            $C_{n,m} := \GenMonoid{a & a^n = a^m}$ für $0 \le m < n$ (Skizze: Anfang + Schleife). 
    \end{itemize}
\end{ex}

Speziell für Gruppen:
Zur Menge $S$ wählen wir das Alphabet
\begin{math}
    A = S \times \Set{\pm} = \Set{s^+, s^- & s \in S}
\end{math}
Zu $R \subset A^*$ setzen wir $K = \Set{ r = 1 & r \in R} \cup \Set{s^+s^- = 1, s^-s^+ = 1 & s \in S}$.
Formal:
\begin{math}
    K = \Set{(r, e) & r \in R} \cup \Set{(s^+s^-, e), (s^-s^+, e) & s \in S}.
\end{math}
Die durch $(S, R)$ \emphdef{präsentierte Gruppe} ist $\Gen{S & R} := \GenMonoid{A & K} = A^* / K$.

\begin{nt}
    In jeder Gruppe lässt sich $a = b$ umformen als $ab^{-1} = 1$.
\end{nt}

\begin{ex}
    \begin{itemize}
        \item
            $\Gen{s & -} := \GenMonoid{s^+ s^- & s^+s^- = 1, s^-s^+ = 1} \isomorphic (\Z, +)$,
        \item
            $\Gen{s & s^n} = \Gen{s & s^n = 1} = \GenMonoid{s^+, s^- & (s^+)^n, s^+s^- = 1, s^-s^+ = 1} \isomorphic (\Z / n, +)$,
        \item
            $\Gen{a,b & ab = ba} = \Gen{a,b & aba^{-1}b^{-1}} \isomorphic (\Z^2, +)$
        \item
            $\Gen{a,b & -} = \Set{e, a, a^{-1}, b, b^{-1}, a^2, a^{-2}, ab, ab^{-1}, a^{-1}b^{-1}, b^2, b^{-2}, ba, ba^{-1}, b^{-1}a, b^{-1}a^{-1}}$

            Skizze: Baum in der Ebene, $a$ nach rechts, $b$ nach oben.

            Im Kontrast dazu $\Z^2 = \Gen{a,b & ab = ba}$: Cayley-Graph.
    \end{itemize}
\end{ex}


\Timestamp{2015-10-23}


\section{Simplizialkomplexe}


Kombinatorische Kodierung eines Simplizialkomplexes:
\begin{math}
    \Set{\emptyset, \Set{a}, \dotsc, \Set{f}, \Set{a,b}, \dotsc, \Set{g,f}, \Set{c,e,f}, \dotsc, \Set{c,g,f}, \Set{c,e,f,g}}.
\end{math}

\begin{df}
    Ein (abstrakter) \emphdef{Simplizialkomplex} $K$ ist ein System endlicher Mengen mit
    \begin{enumerate}[i)]
        \item
            $\emptyset \in K$,
        \item
            $T \subset S \in K \implies T \in K$,
    \end{enumerate}
    Wir setzen
    \begin{math}
        \dim S &:= \card(S) - 1,\\
        \dim K &:= \sup\Set{\dim S & S \in K}, \\
        \Omega(K) &:= \bigcup K = \bigcup_{S \in K} S.
    \end{math}
    $a \in \Omega(K)$ heißt \emphdef{Ecke}, $S \in K$ heißt \emphdef{Simplex} von $K$.
    \begin{math}
        K_{\le n} := \Set{S \in K & \dim S \le n}
    \end{math}
\end{df}

\begin{df}
    Eine \emphdef{Darstellung} $f: K \to V$ in einen $\R$-Vektorraum ist eine Abbildung $f: \Omega(K) \to V$, sodass
    \begin{enumerate}[i)]
        \item
            Für $S \in K$ ist $f(S)$ affin unabhängig.
        \item
            Für $S, T \in K$ gilt $[f(S)] \cap [f(T)] = [f(S\cap T)]$.
    \end{enumerate}
    Die \emphdef{kanonische Darstellung} von $K$ ist $f: K \to \R^{(\Omega)}$, $s \mapsto e_s$.
    \begin{note}
        Hierbei ist
        \begin{math}
            \R^{(\Omega)} = \Set{x: \Omega \to \R & \text{$\supp x$ endlich}}.
        \end{math}
        Dieser hat als kanonische Basis $(e_s)_{s\in\Omega}$ mit $e_s: \Omega \to \R$, $e_s(s') = \delta_{s,s'}$.

        Wir identifizieren $s$ mit $e_s$.
        Dann schreibt sich jedes Element $x \in \R^{(\Omega)}$ als formale Linearkombination
        \begin{math}
            x = \sum_{s \in \Omega} x(s) e_s
            = \sum_{s \in \Omega} x(s) s.
        \end{math}
        Man nennt $\R^{(\Omega)}$ den Vektorraum „frei über $\Omega$“.
    \end{note}
\end{df}

\begin{df}
    Sei $f: K \to V$ eine Darstellung.
    $[f(S)]$ ist ein affiner Simplex in $V$ mit $\dim [f(S)] = \dim S$.
    $\Set{[f(S)] & S \in K}$ ist ein \emphdef{affiner Simplizialkomplex} in $V$, d.h.
    ein System affiner Simplizies, sodass sich je zwei höchstens in einer gemeinsamen Seite schneiden.

    Das Polyeder
    \begin{math}
        |K|_f| := \bigcup_{S \in K} [f(S)] \subset V
    \end{math}
    versehen wir mit der \emphdef{simplizialen Topologie}.
    Eine Teilmenge $U \subset |K|$ ist offen genau dann, wenn $U \cap [f(S)]$ offen ist in $[f(S)]$ für alle $S \in K$.
    \begin{note}
        Für $K$ endlich genügt $f: K \to \R^n$ und die Teilraumtopologie von $|K|_f \subset \R^n$ ist die simpliziale Topologie.

        Für $\Omega$ unendlich ist die simpliziale Topologie wesentlich.
        Betrachte (Skizze: diskrete Variante der Sinuskurve des Topologen)
        \begin{math}
            \Omega &:= \Set{a,b} \cup \N,
            K &:= \Set{\emptyset} \cup \binom{\Omega}{1} \Set{\Set{k, k+1} & k \in \N}
        \end{math}
        mit Darstellung $f: K \to \R^2$, $a \mapsto (0,1)$, $b \mapsto (0,-1)$,
        \begin{math}
            f(k) = \frac{\frac{1}{k}}{(-1)^k}.
        \end{math}
        Wir erhalten $|K|_f \subset \R^2$.
        $[f(a), f(b)]$ ist offen in der simplizialen Topologie, aber nicht offen in der Teilraumtopologie.
    \end{note}
\end{df}


\section{Simpliziale Fundamentalgruppen}

\begin{df}
    Sei $K$ ein Simplizialkomplex mit $\Omega = \Omega(K)$.
    \begin{itemize}
        \item
            Ein Kantenzug $v_0v_1 \dotsc v_n$ ist eine endlich Folge von Eckpunkten mit $\Set{v_0, v_1}, \dotsc, \Set{v_{n-1}, v_n} \in K$.
        \item
            Zwei Kantenzüge $w = v_0 \dotsc v_n$ und $w' = v_0' \dotsc v_m'$ heißen \emphdef{verknüpfbar}, wenn $v_n = v_0'$.
            In diesem Fall ist $w \ast w' := v_0 \dotsc v_n v_1' \dotsc v_m'$ die \emphdef{Verknüpfung} beider Kantenzüge.
        \item
            Zwei Kantenzüge $w = v_0 \dotsc v_{k-1} v_k v_{k+1} \dotsc v_n$ und $w' = v_0 \dotsc v_{k-1} v_{k+1} \dotsc v_n$ heißen äquivelent, geschrieben $w \approx w'$, falls $\Set{v_{k-1}, v_k, v_{k+1}} \in K$.
        \item
            Zu $w = v_0 v_1 \dotsc v_n$ setze $\_w := v_n \dotsc v_1 v_0$.
            Es gilt $w \ast \_w = v_0 v_1 \dotsc v_{n-1} v_n v_{n-1} \dotsc v_1 v_0 \approx v_0$.
        \item
            Simpliziales Fundamentalgruppoid:
            \begin{math}
                \Pi(K) = \frac{\Set{\text{Kantenzüge in $K$}}}{\approx}.
            \end{math}
        \item
            Simpliziale Fundamentalgruppe:
            \begin{math}
                \pi_1(K, x_0) := \Pi(K, x_0, x_0) = \frac{\Set{\text{Kantenzüge in $K$}}}{\approx}
            \end{math}
            Wir erhalten einen Gruppenisomorphismus
            \begin{math}
                \phi: \pi_1(K, x_0) &\xto[isomorphic] \pi_1(|K|, x_0)
                [w] &\mapsto [|w|].
            \end{math}
            \begin{proof}[Beweisidee]
                Prüfe: Wohldefiniertheit (Verträglichkeit der Äquivalenzen),
                Surjektivität: simpliziale Approximation von $\gamma:[0,1] \to (K, x_0)$,
                Injektivität: simpliziale Approximation von $H: [0,1]^2 \to (K, x_0)$.
            \end{proof}
    \end{itemize}
\end{df}

\begin{ex}
    \begin{itemize}
        \item
            Ein (simplizialer) Graph ist ein Simplizialkomplex $K$ mit $\dim K \le 1$.

            Für Kantenzüge gibt es dann nur die Äquivalenzen (Kürzungen/Erweiterungen) der Art
            \begin{math}
                uu &\approx u, &
                uvu &\approx u.
            \end{math}
            Sind keine solchen Kürzungen möglich, so nennen wir den Kantenzug \emphdef{gekürzt} (eigentlich \emph{lokal} gekürzt, greedy).
        \item
            Ein \emphdef{Baum} ist ein Graph, der zusammenhängend und zykelfrei ist, d.h.
            \begin{enumerate}[i)]
                \item
                    Zu je zwei Ecken $a,b \in \Omega(K)$ existiert ein Kantenzug von $a$ nach $b$.
                \item
                    Für jede Kante $\Set{a,b} \in K$, $a \neq b$ sind die Ecken $a,b$ in $K \setminus \Set{\Set{a,b}}$ nicht mehr verbindbar.
            \end{enumerate}
        \item
            Für jeden nicht-leeren endlichen Graphen $K$ sind äquivalent:
            \begin{enumerate}[i)]
                \item
                    $K$ ist ein Baum,
                \item
                    $|K|$ ist zusammenziehbar,
                \item
                    $K$ ist zusammenhängend und $\chi(K) = 1$,
                \item
                    $K$ ist zykelfrei und $\chi(K) = 1$.
                \item
                    Zu je zwei Ecken $a, b \in \Omega(K)$ existiert genau ein gekürzter Kantenzug von $a$ nach $b$.
            \end{enumerate}
            Für unendliche Graphen gilt die Äquivalenz noch zwischen i), ii) und v).
        \item
            Sei $K$ ein zusammenhängender Graph und $T \subset K$ ein Teilgraph, der alle Ecken von $K$ enthält, kurz: $\Omega(T) = \Omega(K)$.
            Dann sind äquivalent:
            \begin{enumerate}[i)]
                \item
                    $T$ ist ein Baum,
                \item
                    $T$ ist zykelfrei und maximal,
                \item
                    $T$ ist zusammenhängend und minimal,
            \end{enumerate}
            In diesem Fall nennen wir $T$ \emphdef{Spannbaum}.
    \end{itemize}
\end{ex}

\begin{st}
    Für jeden Baum $T$ gilt
    \begin{math}
        \pi_1(T, x_0) = \pi_0(|T|, x_0) = \Set{1}.
    \end{math}
    \begin{proof}
        $\pi_1$ durch Kürzung.
        $\pi_0$ durch Zusammenziehen.
    \end{proof}
\end{st}

\begin{st}
    Sei $K$ ein zusammenhängender Graph, $x_0 \in \Omega(K)$, $T \subset K$ ein Spannbaum.
    Dann ist $\pi_1(K, x_0)$ frei über $|K \setminus T|$ Erzeugern.

    Genauer: $\psi: \pi_1(K, x_0) \to F = \GenMonoid{S & R} = \Gen{S & R}$ mit Erzeugern $S = \Set{s_{ab} & \Set{a,b} \in K \setminus T}$ und Relationen $R = \Set{s_{ab} s_{ba} & \Set{a,b} \in K \setminus T}$.

    Ist $K$ zudem endlich, so hat $\pi_1(K, x_0)$ den Rang $|K \setminus T| = 1 - \chi(K)$.
    \begin{proof}
        Siehe Verallgemeinerung unten
    \end{proof}
\end{st}

\begin{st}
    Sei $K$ ein zusammenhängender Simplizialkomplex, $x_0 \in \Omega(K)$, $T \subset K$ ein Spannbaum (im 1-Skelett).
    Dann gilt $\pi_1(K, x_0) \isomorphic \Gen{S & R} = G$ mit
    \begin{math}
        S &= \Set{s_{ab} & \Set{a,b} \in K}, \\
        R &= \Set{s_{ab} & \Set{a,b} \in T} \cup \Set{s_{ab} s_{ba} & \Set{a,b} \in K}
        \cup \Set{s_{ab} s_{bc} s_{ca} & \Set{a,b,c} \in K }
    \end{math}
    Genauer: existieren zueinander inverse Gruppenisomorphismen
    \begin{math}
        \psi&:& \pi_1(K, x_0) &\to G, &
        [v_0 \dotsc v_n] &\mapsto s_{v_0v_1} \dotsb s_{v_{n-1} v_n}, \\
        \phi&:& G &\to \pi_1(K,x_0), &
        s_{ab} &\mapsto [x_0 \dotsc a \ast ab \ast b \dotsc x_0].
    \end{math}
    \begin{proof}
        Durch Nachrechnen: $\psi$ wohldefiniert, $\phi$ wohldefiniert.
        Es gilt $\psi \circ \phi = \id_G$, denn $\psi(\phi(s_{ab})) = s_{ab}$.
        Ebenso $\phi \circ \psi = \id_{\pi_1}$ (nach Kürzen der antisymmetrischen Wege in $T$).
    \end{proof}
\end{st}

\begin{ex}
    \begin{itemize}
        \item
            Skizze: Triangulierter Torus $T$ mit 9 Ecken, Spannbaum $U$.
            Plausibilität: $\chi(T) = 9 - 27 + 18 = 0$.
            Nicht-triviale Elemente $s_{xy} \in T \setminus U$ (wende Relationen an) bilden Erzeuger: $s := s_{ac}$, $t := s_{ea}$.
            Wir erhalten
            \begin{math}
                \pi_1(T, a) = \Gen{S & R}
                \xto* \Gen{s, t & st = ts} \isomorphic \Z^2.
            \end{math}
            Surjektiv nach Bild: Alle Erzeuger werden getroffen.
            Injektiv nach Bild: Alle Relationen wurden verwendet.

            Wie erwartet
            \begin{math}
                \pi_1(|T|, x_0)
                \isomorphic \pi_1(\S^1 \times \S^1, x_0)
                \isomorphic \pi_1(\S^1, x_0) \times \pi_1(\S^1, x_0)
                \isomorphic \Z \times \Z
                \isomorphic \Z^2.
            \end{math}
    \end{itemize}
\end{ex}


\Timestamp{2015-10-30}

\section{Der Satz von Seifert-van-Kampen}

Sei $X = \bigcup_{i \in I} U_i$ eine offene Überdeckung.
Ziel: Wie berechnet man $\pi_1(X, x_0)$ aus den Teilen $(U_i)_{i \in I}$?

Einfaches Beispiel (Skizze: drei Mengen mit paarweisen Schnitten)
Fordere
\begin{itemize}
    \item
        $U_i$ einfach zusammenhängend
    \item
        $U_i \cap U_j$ einfach zusammenhängend
    \item
        $U_i \cap U_j \cap U_k$ einfach zusammenhängend
\end{itemize}
Man denke an $U_i \subset \R^n$ konvex, dann sind alle weiteren Schnitte konvex (ebenso $U_i \subset M$ in einer Riemannschen Mannigfaltigkeit).

Wir bilden folgenden Simplizialkomplex, genannt der \emphdef{Nerv} von $\scr U = (U_i)_{i \in I}$.
Für $S = \Set{s_0, \dotsc, s_n} \subset I$ setze $U_S := U_{s_0} \cap \dotsb \cap U_{s_n}$, sowie $U_{\emptyset} := X$.
Der \emphdef{Nerv} ist
\begin{math}
    N(\scr U) = \Set{\text{$S \subset I$ endlich} & U_S \neq \emptyset}.
\end{math}

Im Beispiel $I = \Set{1, 2, 3}$, $\scr U$ wie skizziert,
\begin{math}
    N(\scr U) = \Set{\emptyset, \Set 1, \Set 2, \Set 3, \Set{1, 2}, \Set{1, 3}, \Set{2, 3}}
\end{math}

\begin{prop}
    $N(\scr U)$ ist ein (abstrakter) Simplizialkomplex.
\end{prop}

\begin{st}
    Sei $X$ ein topologischer Raum, $\scr U = (U_i)_{i \in I}$ eine offene Überdeckung sodass $U_i$ einfach zusammenhängend und $U_i \cap U_j$ wegzusammenhängend ist.
    Wähle $i_0 \in I$ und $x_0 \in U_{i_0}$.

    Dann existiert ein Gruppenisomorphismus
    \begin{math}
        \Phi: \pi_1(N(\scr U), i_0) &\xto \pi_1(X, x_0), \\
        [(i_0, i_1, \dotsc, i_n)] &\mapsto [\gamma_{i_0, i_1} \ast \dotsb \ast \gamma_{i_{n-1}, i_n}].
    \end{math}
    Genauer:
    Hierzu wählen wir $x_i \in U_i$ für $i \in I$ sowie $x_{ij} \in U_{ij} := U_i \cap U_j$ für $i \neq j$ mit $U_{ij} \neq \emptyset$, $x_{ij} = x_{ji}$.
    Sei $\gamma_{ij}$ ein Weg von $x_i$ nach $x_{ij}$ in $U_i$ und dann von $x_{ij}$ nach $x_j$ in $U_j$.
    Damit ist $[\gamma_{ij}]$ eindeutig festgelegt (da $U_i$, $U_j$ einfach zusammenhängend, $U_{ij}$ wegzusammenhängend).
    % \gamma_{ij} \sim \gamma_{ij} \iff \gamma_{ij}\_{\gamma_{ij}} \sim *
    \begin{proof}[Skizze]
        \begin{enumerate}[1)]
            \item
                $\Phi$ surjektiv (Skizze: Weg $\omega$ über endlich viele $U_i$ von $x_{i_0}$ nach $x_{i_n} = x_{i_0}$, homotop zu zweitem Weg):
                Sei $\omega:[0,1] \to X$ ein Weg von $x_0$ nach $x_0$.
                Wir haben eine offene Überdeckung $[0,1] = \bigcup_{i \in I} \omega^{-1}(U_i)$.
                Es existiert eine Lebesgue-Zahl $\frac{1}{n}$, $n \in \nu$, sodass $\omega([\frac{k-1}{n}, \frac{k}{n}]) \subset U_{i_k}$ für $k = 1, \dotsc, n$.
                Für $i_k \neq i_{k+1}$ liegt $\omega(\frac{k}{n}$ in $U_{i_k} \cap U_{i_{k+1}}$.
                Wähle $\beta_k$ von $\omega(\frac{k}{n})$ nach $x_{i_k}{i_{k+1}}$ in $U_{i_k} \cap U_{i_{k+1}}$.
                Dann gilt $\gamma \ast \_\omega \sim 1_{x_0}$.
                %Nutze Kompaktheit des Weges (endliche Überdeckung, Intervallteilung mit Lebesgue-Zahl), konstruiere so die Folge von Mengen $U_i$.
            \item
                $\Phi$ injektiv: später
        \end{enumerate}
    \end{proof}
\end{st}

Im Allgemeinen sind unsere Überdeckungen jedoch nicht so schön.

Sei $X$ ein topologischer Raum und $\scr U = (U_s)_{s \in \Omega}$ eine offene Überdeckung ($U_s$ muss nicht wegzusammenhängend sein, ebensowenig $U_s \cap U_t$).

Der \emphdef{Wegnerv} von $\scr U$ ist definiert durch
\begin{math}
    N^\circ(\scr U) = \Set{(S, C) & \text{$S \subset \Omega$ endlich, $C \in \pi_0(U_s)$} }
\end{math}
($\pi_0(U_s)$ Menge der Wegzusammenhangskomponenten).
Setze $\dim(S, C) = |S| - 1$.
Definiere $(S, C) \to (T, D)$ durch $T \subset S$ und $D \supset C$.

Skizze: $U_1, U_2$, $U_1$ mit zwei Komponenten, mit jeweils $3$ Schnitten $D_1, D_2, D_3$ und einem Schnitt $D_4$ mit $U_2$, $U_2$ einfach zusammenhängend.
\begin{math}
    \begin{tikzcd}
        & (\Set{1,2}, D_1) \ar[ld] \ar[rd]& \\
        (\Set 1, C) & (\Set{1,2}, D_2) \ar[l] \ar[r] & (\Set{2}, U_2) \\
        & (\Set{1,2}, D_3) \ar[lu] \ar[ru] & \\
        (\Set{1}, C') & (\Set{1,2}, D_4) \ar[l] \ar[ruu]
    \end{tikzcd}
\end{math}
Damit ist $I = N^\circ(\scr U)$ mit $\to$ ein Poset, d.h. reflexiv ($i \to i$) und transitiv ($i \to j \to k \implies i \to k$).
Jedem Index $i = (S, C)$ ordnen wir den Teilraum $X_i = C$ zu.
Wir wählen $x_i \in X_i$.
Für $i \to j$ gilt $X_i \subset X_j$.
Da $X_j$ wegzusammenhängend ist, wähle einen Weg $\gamma_{ij}: [0,1] \to X_j$ von $\gamma_{ij}(0) = x_i$ nach $\gamma_{ij}(1) = x_j$.
Für $i = j$ setze $\gamma_{ii} = 1_{x_i}$.
Für $i \to j$, $j \to i$ gilt $X_i = X_j$ und evtl. $x_i \neq x_j$, wir wollen dann $\gamma_{ji} = \_{\gamma_{ij}}$.

Aus $\scr U = (U_s)_{s \in \Omega}$ erhalten wir $(I, \to, (X_i)_{i \in I}, (x_i)_{i \in I}, (\gamma_{ij})_{i \to j})$.

Zu $i \in I$ setzen wir $G_i := \pi_1(X_i, x_i)$.
Für $i \to j$ induziert $\iota_{ij}: X_i \injto X_j$ einen Gruppenhomomorphismus $h_{ij} : G_i \to G_j$ durch
\begin{math}
    h_{ij}([\alpha]) := \_{\gamma_{ij}} \ast (\iota_{ij} \circ \alpha) \ast \gamma_{ij}
\end{math}
(Skizze: !!)
Für $i \to j \to k$ (Skizze: Venn-Diagramm mit Drei Mengen, $i \to j \to k$ und $i \to k$)
setze $g_{ijk} := [\_{\gamma_{jk}} \ast \_{\gamma_{ij}} \ast \gamma_{ik}] \in G_k$.

Es gilt
\begin{math}
    h_{jk} \circ h_{ij} = g_{ijk} h_{ik} g_{ijk}^{-1}.
\end{math}
D.h. $g_{ijk}$ misst die Abweichung von $h_{jk} \circ h_{ij}$ zu $h_{ik}$.

Wir erhalten hieraus den Gruppenkomplex
\begin{math}
    \Gamma &= (I, \to, G_\argdot, h_{\argdot, \argdot}, g_{\argdot, \argdot, \argdot}) \\
    &= (I, \to, (G_i)_{i\in I}, (h_{ij})_{i \to j}, (g_{ijk})_{i \to j \to k}).
\end{math}
Hierin betrachten wir \emphdef{Kantenzüge}
\begin{math}
    w = (i_0 \xto[lr]{g_1} i_1 \xto[lr] \dotsb \xto[lr] i_n)
\end{math}
Hierbei seien $i_0, i_1, \dotsc, i_n \in I$ und $i \xto[lr]{g} j$ steht entweder für $i \to j$ oder $i \xto* j$ in Graphen $(I, \to)$ oder aber $i \xto{g} i$ oder $i \xto*{g} i$ mit $g \in G_i$.
Die Verknüpfung $w \ast w'$ ist die Aneinanderhängung.
Wir nutzen folgende Relationen
\begin{math}
    (i \to i) \approx (i \xto{1} i) &\approx (i), \\
    (i \to j \xto* i) &\approx (i), \\
    (j \xto* i \xto{g} i \to j) &\approx (j \xto{h_{ij}(g)} j), \\
    (i \xto{g} i) &\approx (i \xto*{g^{-1}} i), \\
    (i \xto{g} i \xto{h} i) &\approx (i \xto{gh} i), \\
    (i \to j \to k) &\approx (i \to k \xto{g_{ijk}^{-1}} k).
\end{math}
Die Kantengruppe des Gruppenkomplexes $\Gamma$ ist
\begin{math}
    \pi_1(\Gamma, i_0) = \frac{\text{geschl. Kantenzüge in $\Gamma$ von $i_0$ nach $i_0$}}{\text{Äquivalenz $\approx$}}.
\end{math}

\begin{st}[Seifert-van-Kampen]
    Jede offene Überdeckung $X = \bigcup{s \in \Omega} U_s$ definiert einen Gruppenkomplex $\Gamma$ (nach Wahl von Fußpunkten und Verbindungswegen).
    Seine Kantengruppe $\pi_1(\Gamma, i_0)$ ist isomorph zu $\pi_1(X, x_0)$.

    Genauer: Für das $1$-Skelett $\Gamma_{\le 1}$ liefert die topologische Realisierung eine Surjektion
    \begin{math}
        \Phi_1: \pi_1(\Gamma_{\le 1}, i_0) \xto[surjective] \pi_1(X, x_0).
    \end{math}
    Für das $2$-Skelett $\Gamma_{\le 2}$ erhalten wir einen Isomorphismus $\Phi_2 : \pi_1(\Gamma_{\le 2}, i_0) \xto[isomorphic] \pi_1(X, x_0)$.
\Timestamp{2015-11-06}
    \begin{proof}[Skizze]
        Betrachte $\Phi_1: \pi_1(\Gamma_{\le 1}, i_0) \xto[surjective] \pi_1(X, x_0)$, zeige Surjektivität.
        Sei dazu $[\omega] \in \pi_1(X, x_0)$, d.h. $\omega:[0,1] \to X$ eine Schleife in $x_0$.
        Zeige: $\omega$ homotop zu einem Weg $\Phi_1(w)$ mit $w \in \pi_1(\Gamma_{\le 1}, i_0)$.
        \begin{math}
            w = (i_0 \xto{g_0} i_0 \xto* i_{0,1} \to i_1 \xto{g_1} i_1 \xto* i_{12} \to i_2 \xto{g_2} i_2 \dotsb \xto* j_{n-1} \to i_n \xto{g_n} i_n).
        \end{math}
        Die topologische Realisierung von $w$ ist homotop zu $\omega$ nach Konstruktion.

        Betrachte nun $\Phi_2: \pi_2(\Gamma_{\le 2}, i_0) \to \pi_1(X, x_0))$, zeige Bijektivität.
        Wohldefiniert: Man vergewissere sich, dass alle definierten Äquivalenzen entsprechende Homotopien erlauben.
        Surjektivität wie zuvor.
        Zeige nun Injektivität: Werden zwei Wörter $w_1, w_2$ durch homotope Wege $w_1, w_2$ realisiert, dann sind $w_1$ und $w_2$ äquivalent.
        Sei $H: [0,1]^2 \to X$ eine Homotopie von $w_1$ nach $w_2$.
        Idee: Modifziere $H$ derart, dass sich eine Triangulierung ergibt und jedes Dreieck einer Relation entspricht.
        Wir nutzen $X = \bigcup_{s \in \Omega} U_s$ für $[0,1]^2 = \bigcup_{s \in \Omega} H^{-1}(U_S)$.
        Dank Kompaktheit existiert eine Lebesgue-Zahl dieser Überdeckung.
        Wir unterteilen $[0,1]^2$ wie folgt (Skizze: Ziegel-Mauerwerk):
        Nach hinreichend feiner Unterteilung gilt für jeden Ziegel $\Z_\alpha$ die Bedingung $H(Z_\alpha) \subset U_{s(\alpha)}$ für eine geeignete Abbildung $s$.
        Wir dicken die Fugen auf (z.B. erst horizontale Fugen, dann vertikale durch konstante Homotopien, es entstehen kleine Quadrate, in denen die Homotopie konstant ist).
        Lokal in einem kleinen Quadrat ist $H$ konstant $c$ und es gilt $H(Q) \subset U_1 \cap U_2 \cap U_3$.
        Wähle $i \in I$ sodass $H(Q) \subset X_i$, füge $x_i$ im Quadrat ein mit „radialer Homotopie“.
        Zusätzlich $x_{01}, x_{12}, x_{12}$ an den Kantenmittelpunkten.
        Hilfspunkte für $g_{012}$.

        Von Fugen-Quadraten zu den Fugen-Rechtecken.
        Damit können wir jede Fuge auffüllen durch
        Schließlich Ziegel.

        (Skizzen sind hilfreich)

        Zusammenfassung:
        Wir beginnen mit der gegebenen Homotopie $H$.
        \begin{enumerate}[1.]
            \item
                Aufdicken von horizontalen und vertikalen Fugen.
            \item
                Korrektur um Eckpunkten.
            \item
                Korrektur auf Fugenstücken.
            \item
                Korrektur auf Ziegeln.
        \end{enumerate}
        Ablesen dieser einfachen Teile liefert die kombinatorische Äquivalenz von $w_1$ nach $w_2$.
    \end{proof}
\end{st}

\begin{ex}
    \begin{itemize}
        \item
            Kreisring aus zwei Mengen gebildet.
            $\pi_1(\Gamma, x_1) \isomorphic \Z$.
            Wir benötigen nur $U_1, U_2$ einfach zusammenhängend, $U_1 \cap U_2$ hat zwei Wegkomponenten.

            Betrachte $U_1, U_2$ mit $U_1 \cap U_2 = C_1 \dunion C_2$ und $C_1 \isomorphic C_2 \isomorphic \S^1 \times \B^2$.
        \item
            $U_1$ Kreisring, $U_2$ Kreisscheibe, $U_1 \cap U_2$ Kreisring.
        \item
            Venn-Diagramm, ohne Mittelteil, $\pi_1(\Gamma, x_0) \isomorphic \Z$.
        \item
            Wie voriges, mit mittlerer Kreisscheibe, $\pi_1(\Gamma, x_0) \isomorphic \Set e$.
        \item
            $\pi_0(U_1) = \pi_0(U_2) = \pi_0(U_1 \cap U_2) = \Set *$.
            $\pi_1(U_1) = G_1$, $\pi_1(U_2) = G_2$ beliebig, $\pi_1(U_1 \cap U_2) = \Set e$.
            \begin{math}
                \pi(\Gamma, x_0) = G_1 \ast G_2
            \end{math}
            (freies Produkt).
        \item
            $\pi_1(U_1) = G_1$, $\pi_1(U_2) = G_2$ beliebig, $U_1 \cap U_2$ wegzusammenhängend, $\pi_1(U_1 \cap U_2) = K$.
            Sei $U_{12} := U_1 \cap U_2$,
            \begin{math}
                i: U_{12} &\injto U_1, &i_\#: \pi_1(U_{12}) &\injto \pi_1(U_1), \\
                j: U_{12} &\injto U_2, &j_\#: \pi_1(U_{12}) &\injto \pi_1(U_2). \\
            \end{math}
            Das führt zum amalgamierten Produkt $G_1 \ast_K G_2$.

            Ausführlich und etwas allgemeiner:
            Sei $X = U_1 \cup U_2$, $U_1, U_2$ offen und wegzusammenhängend,
            $U_{12} = U_1 \cap U_2$ (offen und) wegzusammenhängend.
            Wir wählen $x_0 \in U_{12} \subset U_1, U_2$.
            Sei $\pi_1(U_i, x_0) = G_i = \Gen{S_i & R_i}$ für $i \in \Set{1, 2, (1,2)}$.
            (Skizze: $\Gamma$)
            Dann gilt
            \begin{math}
                \pi_1(X, x_0) = \pi_1(\Gamma, x_0)
                = \Gen{S_1, S_2 & R_1 \dunion R_2 \dunion T}
            \end{math}
            mit
            \begin{math}
                T = \Set{h_1(s) h_2(s)^{-1} & s \in S_{12}}
            \end{math}
        \item
            $g_{ijk}$ nichttrivial:
            $U_1 \isomorphic U_2 \isomorphic U_3 \isomorphic \S^1 \times (0,1)$ mit geeigneter Wahl der Fußpunkte.
    \end{itemize}
    \begin{note}
        Gruppenkomplexe dienen zur Analyse von Gruppen, siehe Serre, Trees.
    \end{note}
\end{ex}




% Kapitel B
\chapter{Knotengruppen}

\Timestamp{2015-05-06}

Ziel: Zu Knoten $K \subset \R^3$ wollen wir $\pi_1(\R^3 \setminus K, *)$ berechnen und nutzen.


% §B1
\section{Erinnerung: Präsentation von Gruppen}

\begin{ex}
    \begin{itemize}
        \item
            Zyklische Gruppe:
            \begin{math}
                G = \Set{g, g^2, g^3, \dotsc, g^n = 1},
            \end{math}
            wobei $g^i \neq g^j$ für $0 \le i < j \le n$.
            Wir nutzen dafür die Schreibweise
            \begin{math}
                G = \<g | g^n = 1\>
                = \<g | g^n\>.
            \end{math}
            Dann erhalten wir den Gruppenisomorphismus $\Z / n \to G$, $k + n\Z \mapsto g^k$.
        \item
            Unendliche zyklische Gruppe:
            \begin{math}
                G = \Set{g^k & k \in \Z},
            \end{math}
            mit $g^i \neq g^j$ für $i \neq j$ in $\Z$.
            Wir nutzen die Schreibweise
            \begin{math}
                G = \< g | - \>.
            \end{math}
            Dann haben wir den Gruppenisomorphismus $\Z \to G, k \mapsto g^k$.
        \item
            Wir wollen folgende Notationen nutzen können:
            \begin{math}
                G = \Gen{a,b & ab = ba}
            \end{math}
            Wir haben einen Gruppenisomorphismus $Z^2 \to G$, $(k,l) \mapsto a^kb^l$.
    \end{itemize}
\end{ex}

\subsection{Freie Gruppen}

\begin{df}
    Sei $∈(G, \cdot)$ eine Gruppe, $S \subset G$.
    Die von $S$ erzeugte Untergruppe ist
    \begin{math}
        \<S\> = \Set{s_1^{e_1} \dotsc s_n^{e_n} & n \in \N, s_i \in S, e_i \in \Z}.
    \end{math}
    Wir nennen $s_1^{e_1} \dotsc s_n^{e_n}$, genauer $(s_1,e_1; \dotsc; s_n, e_n) \in (S\times \Z)^n$ ein \emphdef{Wort} über $S$.
    Ein Wort heißt \emphdef{reduziert}, wenn $s_i \neq s_{i+1}$ und $e_i \neq 0$.
\end{df}

\begin{df}
    Eine Gruppe $G$ heißt \emphdef{frei}, über $S \subset G$, wenn sich jedes $g \in G$ eindeutig schreiben lässt als reduziertes Wort über $S$.
\end{df}

\begin{ex}
    \begin{itemize}
        \item
            $G \isomorphic \Z/5$ ist nicht frei über $S = \Set{g}$, weil $1 = g^0 = g^5 = g^{10} = \dotsc$.
        \item
            $G \isomorphic \Z$ ist frei über $S = \Set{g}$.
        \item
            $G \isomorphic \Z^2$ ist nicht frei über $S = \Set{a,b}$, denn $ab = ba$, also
            \begin{math}
                (a,1;b,1) \neq (b,1;a,1).
            \end{math}
    \end{itemize}
\end{ex}

\begin{st}
    Zu jeder Menge $S$ existiert eine freie Gruppe $F(S) = \<S| - \>$ über $S$.
    \begin{proof}
        Übung.
    \end{proof}
\end{st}

\begin{st}[universelle Abbildungseigenschaft]
    Eine Gruppe $F$ ist genau dann frei über $S \subset F$, wenn gilt:
    zu jeder Abbildung $f: S \to G$ in eine Gruppe $G$ existiert genau ein Gruppenhomomorphismus $h: F \to G$ mit $h|_S = f$.
    %\begin{note}
    %    \Hom(F,G) \stack\isomorphic\to \App(S,G),
    %    h \mapsto h|_S.
    %\end{note}
    \begin{proof}
        Sei $F$ frei über $S$, dann besteht $F$ aus reduzierten Wörtern der Form $s_1^{e_1} \dotsc s_n^{e_n}$.
        Setze $h: F \to G$ durch $h(s_1^{e_1} \dotsc, s_n^{e_n}) = f(s_1)^{e_1} \dotsc f(s_n)^{e_n}$, dies ist die einzige Möglichkeit, eine solche Abbildung zu definieren.
        Sie ist wohldefiniert und multiplikativ.

        Die Umkehrung ist abstract general nonsense:
        Angenommen $f$ besitzt die universelle Abbildungseigenschaft.
        Dann existiert genau ein Gruppenhomomorphismus $h: F \to F(S)$ mit $h|_S = \id_S$ und genau ein $k: F(s) \to F$ mit $k|_S = \id_S$.
        Für diese gilt $k \circ h = \id_F$ und $h \circ k = \id_{F(S)}$.
    \end{proof}
\end{st}

\begin{df}
    Sei $S$ eine Menge, $F = F(S)$ eine freie Gruppe über $S$.
    Sei $R \subset F$ eine Menge von reduzierten Gruppenwörtern über $S$.
    Wir nennen $(S, R)$ eine \emphdef{Präsentation} mit Erzeugern $S$ und Relationen $R$.
    Die hierdurch \emphdef{prästentierte Gruppe} ist
    \begin{math}
        \<S |R\> := F / \<R^F\>.
    \end{math}
    Hierbei ist $\<R^F\>$ die von $R$ normal erzeugte Untergruppe in $F$, d.h.
    \begin{math}
        \<R^F\> = \Gen{ r^f & r \in R, f \in F }
    \end{math}
    wird erzeugt von allen Konjugierten von $r \in R$ in $F$.
    Dies ist die kleinste normale Untergruppe, die $R$ enthält.
\end{df}

\begin{ex}
    \begin{enumerate}[1)]
        \item
            Für $R = \emptyset$ ist $\<S | \emptyset\> = \Gen{S & -} = F(S)$.
        \item
            $\Gen{a & -} \leftarrow \Z, a^k \mapsfrom k$,
        \item
            $\Gen{a & a^n} \leftarrow \Z /n, a^k \mapsfrom k$.
        \item
            $\Gen{a, b & aba^{-1}b^{-1}} \leftarrow \Z^2, a^kb^l \mapsfrom (k,l)$
        \item
            Zopfgruppen, symmetrische Gruppen
    \end{enumerate}
\end{ex}

\begin{st}[Universelle Abbildungseigenschaft]
    Sei $(S,R)$ wie oben, $f : S \to G$.
    Dann sind äquivalent:
    \begin{enumerate}[1)]
        \item
            Der Gruppenhomomorphismus $h: F(S) \to G$ mit $h|_S = f$ erfüllt $h(R) = \Set{1}$ (und faktorisiert somit).
        \item
            Es existiert ein Gruppenhomomorphismus $\_h: \<S|R\> \to G$ mit $\_h \circ q = f$ ($q$ sei hierbar Quotientenhomomorphismus).
    \end{enumerate}
    \begin{proof}
        Leichte Übung.
    \end{proof}
\end{st}

\begin{prop}
    Jede Gruppe $G$ erlaubt eine Präsentation, d.h. ein Tripel $(S,R,h)$ mit $h: \<S | R\> \stack\isomorphic\to G$.
\end{prop}

\begin{ex}
    Sei $G = \Z / R = \Set{0,1,2,3,4}$.
    $h: \< a | a^5\> \to \Z^5$, $a \mapsto 1$ (wohldefiniert, surjektiv, injektiv).
    Aber auch $k: \<a|a^5\> \to \Z^5$, $a \mapsto 2$ ist ein Gruppenisomorphismus.
\end{ex}

Zu einer gegebenen Gruppe $G$ gibt es stets unendlich viele Präsentationen!
Die folgenden Tietze-Operationen ändern die Präsentation, nicht aber die präsentierte Gruppe.
\begin{enumerate}[(T1)]
    \item
        Hinzufügen/Entfernen einer redundanten Relation: $(S,R) \leadsto (S,R')$ mit $R' = R \cup \Set{r}$, $r \in \<R^F\> \setminus R$.
    \item
        Hinzufügen/Entfernen eines redundanten Erzeugers: $(S,R) \leadsto (S',R')$ mit $S' = S \dotcup \Set{s}$, $R' = R \cup \Set{s^{-1} w}$, $w \in \<S\>$.
\end{enumerate}

\begin{st}[Tietze, 1908]
    Zwei (endliche) Präsentationen $(S,R)$ und $(S',R')$ präsentieren genau dann isomorphe Gruppen, wenn sie sich durch (T1), (T2) ineinander überführen lassen.
\end{st}

\Timestamp{2015-05-11}

Ziel: Zu jedem Knoten $K \subset \R^3$ wollen wir die Knotengruppe $\pi_K := \pi_1(\R^3 \setminus K, *)$ „berechnen“, d.h. präsentieren und auswerten.


% B2
\section{Wirtinger-Präsentation}


Stelle $K$ durch ein ebenes Diagramm $D$ dar, $K$ und $D$ seien orientiert.
Erzeuger sind die Bögen $x_1, \dotsc, x_n$ von $D$.
Relationen sind Kreuzungen: $x_ix_j = x_jx_{i+1}$, $x_i^{-1}x_i x_j = x_{i+1}$, $x_jx_i = x_{i+1}x_j$, $x_jx_ix_j^{-1} = x_{i+1}$, oder zusammengefasst
\begin{math}
    x^{-\eps(i)}_{j(i)} x_i x_{j(i)}^{\eps(i)} = x_{i+1},
\end{math}
die Daten $\eps: \Set{1,\dotsc, n} \to \Set{\pm 1}$ und $j: \Set{1, \dotsc, n} \to \Set{1, \dotsc, n}$ liest man leicht am Diagramm ab.

\begin{df}
    Wir setzen
    \begin{math}
        \pi_D := \Gen{x_1, \dotsc, x_n & x_{j(i)}^{-\eps(i)} x_i x_{j(i)}^{\eps(i)} = x_{i+1}, i = 1,\dotsc, n}
    \end{math}
\end{df}

\begin{ex}
    \begin{itemize}
        \item
            $\pi_{\KnotTriv} = \Gen{x_1 & -} \isomorphic \Z$.
        \item
            \begin{math}
                \pi_{\KnotKlee} &= \Gen{a,b,c & ac = cb, cb = ba, ba = ac} \\
                &= \Gen{a,b,c & a^c = b, b^a = c, c^b = a}
            \end{math}
    \end{itemize}
\end{ex}

\begin{note}
    $\pi_D$ ist unendlich, denn wir haben die Abelschmachung $(\pi_D,\cdot) \to (\Z,+), x_i \mapsto 1$.

    Für Verschlingungen, bzw. Schlingel mit $n$ Komponenten entsprechend $\pi_D \to \Z^n$.
\end{note}

\begin{ex}
    $\pi_{\KnotKlee}$ ist nicht abelsch, also $\pi_{\KnotKlee} \not\isomorphic \pi_{\KnotTriv}$.
    \begin{proof}
        Betrachte $h: \pi_D \to S_3$ mit $a \mapsto (12)$, $b \mapsto (23)$, $c \mapsto (13)$.
        Man rechnet:
        \begin{math}
            a^c &= (23) = b, &
            b^a &= (13) = c, &
            c^b &= (12) = a,
        \end{math}
        Da $h$ surjektiv und $S_3$ nicht abelsch, ist auch $\pi_D$ nicht abelsch.
    \end{proof}
\end{ex}

Noch zu zeigen: $\pi_D$ ist eine Invariante des Knotentyps.
Es gibt folgende Möglichkeiten:
\begin{enumerate}[1)]
    \item
        Reidemeister-Züge verändern die Gruppe nicht: die Präsentationen unterscheiden sich um Tietze-Transformationen.
    \item
        Es existiert ein Isomorphismus $\pi_D \isomorphic \pi_1(\R^3 \setminus K, *)$.
\end{enumerate}


% todo: Appendix:

%\subsection*{Erinnerung: Permutationen, Zykelschreibweise, Konjugation}
%
%In $S_n$ nutzen wir folgende Schreibweise:
%Seien $i_1, \dotsc, i_l \in \Set{1, \dotsc, n}$ verschieden.
%Definiere die Zykel: $c := (i_1, \dotsc, i_l)$ durch $i_1 \mapsto i_2 \mapsto \dotsb \mapsto i_l \mapsto i_1$.
%
%\begin{prop}
%    Jedes $\sigma \in S_i$ ist Produkt disjunkter Zykel.
%    Dieses ist eindeutig bis auf Umordnung der Faktoren.
%\end{prop}
%
%\begin{ex}
%    $\sigma = (1352)(476) = (476)(1352)$.
%\end{ex}
%
%Für die Konjugation gilt
%\begin{math}
%    (i_1, \dotsc, i_l)^\sigma &=
%    \sigma^{-1} (i_1, \dotsc, i_l) \sigma \\
%    &= (\sigma(i_1), \dotsb, \sigma(i_l))
%\end{math}
%

\begin{st}[Wirtinger, <1900]
    Es existiert ein Gruppenisomorphismus $\pi_D \to \pi_1(\R^3 \setminus K, *)$.
    \begin{proof}
        \begin{enumerate}[1),start=0]
            \item
                Konstruktion von $h$:
                Wie in der Skizze, ordnen wir jedem Bogen $b_i$ von $D$ einen (polygonalen) Weg $\gamma_i: [0,1] \to \R^3 \setminus K$ zu.
                Dieser definiert ein Gruppenelement $w_i = [\gamma_i] \in \pi_1(\R^3 \setminus K)$.
                An jeder Kreuzung gilt die Wirtinger-Relation:
                \begin{math}
                    x_i x_j = x_j x_{i+1}
                \end{math}
                und analog die anderen.
                Wir haben nun einen Gruppenhomomorphismus $h: \pi_D \to \pi_1(\R^3 \setminus K, *)$ mit $x_i \mapsto w_i = [\gamma_i]$.
            \item
                $h$ ist surjektiv, d.h. $\pi_1(\R^3 \setminus K, *)$ wird erzeugt von $w_1, \dotsc, w_n$:

                Wir nutzen die polygonale Fundamentalgruppe (mittels polygonaler Approximation)
                \begin{math}
                    \pi_1(\R^3 \setminus K, *)
                    = \frac{\Set{\text{Schleifen}}}{\text{Homotopie}}
                    = \frac{\Set{\text{polygonale Schleifen}}}{\text{polygonale Homotopie}}.
                \end{math}
                Ohne Einschränkung betrachten wir also polygonale Schleifen $\gamma$ in $\R^3 \setminus K$.
                Zu zeigen ist $\gamma \homotopic \gamma_{i_1}^{e_1} \dotsb \gamma_{i_l}^{e_l}$.
                Trick: Betrachte den „Schatten“ des Knotens $K \subset \R^3$ unter senkrecht von oben einfallendem Licht.
                Genauer: Zu $K \subset \R^3$ ist der Schatten $\hat K = \Set{(x,y,z) \in \R^3 & \exists z' \ge z: (x,y,z') \in K}$.
                Dies ist die Vereinigung über alle Schatten $\hat A$ der Kanten $A$ von $K$.
                \begin{prop}
                    Es gilt $\R^3 \setminus \hat K \homequiv *$
                    \begin{proof}
                        Übung: explizite Formel, vgl. Sternförmig bei Zentralprojektion.
                    \end{proof}
                \end{prop}
                Ablesen an $w = [\gamma]$ eines Wortes in $w_i = [\gamma_i]$.
                Wir nehmen an, dass $\gamma$ die Wände $\hat A$ transversal im Inneren trifft.
                Jeder Durchgang liefert einen Erzeuger $w_i^{\pm 1}$.
                Damit gilt $\gamma = \gamma_{i_1}^{e_1} \dotsb \gamma_{i_l}^{e_l}$.
            \item
                $h$ ist injektiv, d.h. die Wirtinger-Relationen erzeugen alle Relationen.
                Polygonale Homotopie:
                \begin{enumerate}[1)]
                    \item
                        Keine Wand wird getroffen: Keine Änderung des Wortes.
                    \item
                        Eine Wand wird geschnitten:
                        Zwei Unterfälle: Jeweils keine Änderung des Wortes.
                    \item
                        Schatten einer Kreuzung wird geschnitten.
                        Dank Wirtinger-Relation keine Änderung des Wortes.
                \end{enumerate}
        \end{enumerate}
    \end{proof}
\end{st}




\chapter{Kryptographie, Primzahltests}



% 3.1
\section{Das RSA-Verfahren}

Das Verfahren wurde 1977/1978 am MIT von Rivest, Shamir und Adelman entwickelt (soll aber bereits vorher, zumindest in ähnlicher Form MI6 bekannt gewesen sein).

Das RSA-Verfahren ist ein sogenanntes asymmetrisches kryptographisches Verfahren.
Solche finden Anwendung bei Verschlüsselung von Nachrichten, bei digitalen Signaturen, Bezahlung mit Kreditkarten, Pay-TV.

Mathematisch gesehen war es der Retter der Königin der Mathematik (der elementaren Zahlentheorie).
Diese hatte bis dahin nämlich kaum praktischen Anwendungsgebiete gehabt.


\paragraph{Ein Szenario}
Ein Kunde möchte seiner Bank über einen unsicheren/öffentlichen Weg eine geheime Botschaft übermitteln.
Bezeichne den Kunden mit $K$, die Bank mit $B$.
$K$ teilt $B$ mit, dass er eine Nachricht $m$ übermitteln will.
$B$ wählet zwei (große) Primzahlen $p$ und $q$, $p \neq q$ und berechnet $n := pq$ und $\phi(n) = (p-1)(q-1)$.
$B$ wählt $e$ mit $(e, \phi(n)) = 1$.
Dann berechnet $B$ ein $d \in \N$ mit $ed \equiv 1 \bmod \phi(n)$.
Soein $d$ existiert, da $(e, \phi(n)) = 1$.
Wie berechnet man $d$?
Mit dem Lemma von Beszout ist $z_1 e + z_2 \phi(n) = 1$ und der euklidische Algorithmus liefert $z_1 = d$.
$B$ sendet das Paar $(n, e)$ (“public key”) öffentlich an $K$, $d$ bleibt geheim.
$K$ sendet $m$ (als Restklasse modulo $n$ betrachtet) mittels Chiffrierung
\[
	m \mapsto m^e \mod n
\]
an $B$.
$(n, e)$ und $m^e$ sind also öffentlich bekannt, während $m$ geheim bleibt.
$B$ dechiffriert mittels $d$
\[
	m \equiv m^{ed} \mod n
\]
Warum funktioniert dies?


Zunächst eine Art Verallgemeinerung des kleinen Satzes von Fermat.

\setcounter{thm}{1}
% Lem 3.2
\begin{lem} \label{3.2}
	In $\Z / n \Z$ gilt für $n = pq$ mit $p,q \in \P$, dass
	\[
		x^{k\phi(n) + 1} \equiv x \mod n
	\]
	für jedes $x$.
	\begin{proof}
		Mit \ref{1.14} ist $Z / n\Z = \Z /p\Z \times \Z q\Z$ und $x = (x_1, x_2)$, also $x^{\phi(n)} = (x_1^{\phi(n)}, x_2^{\phi(n)})$ und
		\[
			x_1^{\phi(n)}
			= x_1^{(p-1)(q-1)}
			= (x_1^{p-1})^{q-1}
			\equiv 1^{q-1}
			= 1,
		\]
		wenn $x_1 \neq 0$, oder $x_1^{\phi(n)} \equiv 0$, wenn $x_1 = 0$.
		Analog gilt
		\[
			x_2^{\phi(n)} \equiv \begin{cases}
				1 & x_2 \neq 0 \\
				0 & x_2 = 0
			\end{cases},
		\]
		also ist
		\[
			x^{k\phi(n)} = \begin{cases}
				(1, 0) & x_2=0, x_1 \neq 0 \\
				(0,1) & x_1=0, x_2 \neq 0 \\
				(1,1) & x_1\neq 0, x_2 \neq 0
			\end{cases}
		\]
		und
		\[
			x^{k\phi(n) + 1} = \begin{cases}
				(x_1, 0) & x_2=0, x_1 \neq 0 \\
				(0,x_2) & x_1=0, x_2 \neq 0 \\
				(x_1,x_2) & x_1\neq 0, x_2 \neq 0
			\end{cases},
		\]
		also in allen Fällen $x^{k\phi(n) + 1} = x$.
	\end{proof}
\end{lem}

Damit ein Außenstehender an die geheime Nachricht $m$ kommt, benötigt er $d$.

% Lem 3.3
\begin{lem} \label{3.3}
	Die Kenntnis von $n$ und $\phi(n)$ ist äquivalent zur Kenntnis von $p$ und $q$.
	\begin{proof}
		\begin{segnb}{$\impliedby$}
			Klar, denn für bekanntes $p, q$ ist $n = pq$ und $\phi(n) = (p-1)(q-1)$.
		\end{segnb}
		\begin{segnb}{$\implies$}
			Seien $n$ und $\phi(n)$ bekannt.
			\[
				\phi(n)
				= (p-1)(q-1)
				= pq - p - q + 1
				= n - p - q + 1,
			\]
			also
			\[
				n = pq = nq - \phi(n)q - q^2 + q,
			\]
			bzw.
			\[
				q^2 + (\phi(n) - n - 1)q + n = 0.
			\]
			Aus dieser quadratischen Gleichung ergeben sich $p, q$ als Lösungen.
		\end{segnb}
	\end{proof}
\end{lem}

Um $d$ zu berechnen benötigt man $\phi(n)$ und für $\phi(n)$ benötigt man $p$ und $q$.
Man benötigt also Verfahren, um große Zahlen in Primfaktoren zu zerlegen.
Dies ist schwer, oft praktisch unmöglich.

\coursetimestamp{12}{05}{2014}


% Ex 3.4
\begin{ex} \label{3.4}
	\begin{enumerate}[a)]
		\item
			Sei $n = 11 \cdot 31 = 341$, dann ist $\phi(n) = 300$.
			Die Mitternachtsformel liefert für
			\[
				q^2 + q(300 - 342) + 341 = 0
			\]
			die Lösung $q_{1,2} = \f{42 \pm \sqrt{42^2 - 4\cdot 341}}2 = 21 \pm 10$.
		\item
			Sei $(n, e) = (667, 15)$.
			Gesendet wird $424$, wie lautet die Nachricht von K an B?.
	\end{enumerate}
\end{ex}

Das RSA-Verfahren liefert einen Grund gegegben Zahlen $n \in \N$ zu faktorisieren, oder zumindest zu prüfen, ob $n$ eine Primzahl ist.
Dies motiviert Primzahltests.

% Prop 3.5
\begin{prop}[Fermat'scher Primzahltest] \label{3.5}
	Sei $n \in \N$.
	Falls $a \in \N$ existiert mit $(a,n) = 1$ und $a^{n-1} \not\equiv 1 \bmod n$, dann ist $n$ keine Primzahl.
	\begin{proof}
		Umkehrung des kleinen Satzes von Fermat \ref{2.9} und Übung P4.4 auf Übungsblatt 4.
	\end{proof}
\end{prop}

% Def 3.6
\begin{df} \label{3.6}
	\begin{enumerate}[a)]
		\item
			$n \in \N$ nennt man \emphdef[Pseudoprimzahl]{Pseudoprimzahl zur Basis $a$}, wenn $n$ keine Primzahl ist, aber $a^{n-1} \equiv 1 \bmod n$ gilt.
		\item
			$n \in \N$ nennt man \emphdef{Carlmichaelzahl}, wenn $n$ keine Primzahl ist, aber für alle $a \in \N$ mit $(a,n) = 1$ gilt, dass $a^{n-1} \equiv 1 \bmod n$.
	\end{enumerate}
	\begin{note}
		Carlmichaelzahlen  kommen von der Umkehrung des kleinen Fermats: ist $p$ Primzahl, dann ist $a^{p-1} \equiv 1 \bmod p$ für alle $(a,p) = 1$.
		Man hat sich also die Frage gestellt, für welche Zahlen die Umkehrung nicht gilt, diese nennt man Carlmichaelzahlen.
	\end{note}
\end{df}

% Ex 3.7
\begin{ex} \label{3.7}
	\begin{enumerate}[a)]
		\item
			Sei $n=9, a= 2$.
			Dann ist $a^3 = 8 \equiv -1 \bmod 9$, also ist $a^6 \equiv 1 \bmod 9$ und $a^8 \not\equiv 1 \bmod 9$
			Also ist $9$ keine Primzahl.
		\item
			$341$ ist eine Pseudoprimzahl zur Basis $2$, denn $2^10 = 1024 = 3\cdot 341 + 1$, also ist $2^10 \equiv 1 \bmod 341$ und $2^{340} \equiv 1 \bmod 341$.
		\item
			$561 = 51 \cdot 11 = 3 \cdot 17 \cdot 11$ ist eine Carmichaelzahl (sogar die kleinste).

			Dies manuell nachzuprüfen ist mühsam.
			Glücklicherweise liefert der folgende Satz eine einfacherer Charakterisierung.
	\end{enumerate}
\end{ex}

% St 3.8
\begin{st}[Korselt, 1899] \label{3.8}
	$n$ ist genau dann eine Carmichaelzahl, wenn $n$ quadratfrei ist (d.h. kein Quadrat als Teiler hat) und für alle ihre Primteile $p$ gilt, dass $p-1$ ein Teiler von $n-1$ ist und $n = p_1 p_2 \cdot p_k$ mit $k \ge 3$ und $p_i$ ungerade Primzahl.
	\begin{proof}
		\begin{enumerate}[1.]
			\item
				Wenn $n = \prod_{i=1}^k p_i$ mit $p_i$ Primzahl, $p_i \neq p_j$ für $i \neq j$, dann ist mit \ref{1.14}
				\[
					U(\Z / n\Z) = U(\Z / p_1\Z) \times \dotsb \times U(\Z / p_k \Z).
				\]
				Aus $u \in U(\Z / n\Z)$ folgt $u^m = 1$ für $m = \kgV \{p_i-1 : 1 \le i \le k\}$.
				Nach Voraussetzung wird $n - 1$ von $m$ geteilt, also $u^{n-1} = 1$.
				$u$ repräsentiert jedes $a$ mit $(a,n) = 1$.
			\item
				Zeige: ist $n$ nicht quadratfrei, dann ist $n$ keine Carmichaelzahl.

				Sei $n = p^m a$ mit $(p,a) = 1$, Primzahl $p$ und $m \ge 2$.
				Aus $b^{n-1} \equiv 1 \bmod n$ folgt $b^{n-1} \equiv 1 \bmod p^m$, also $\phi(p^m)$ teilt $n-1$, denn $U(\Z / p^m \Z)$ ist zyklisch von Ordnung $\phi(p^m)$.
				Es folgt $p^{m-1} (p-1)$ teilt $n-1$.
				Da aber $n$ von $p$ geteilt wird und wegen $m \ge 2$ $p$ auch $n-1$ teilt, folgt dann $p$ teilt $n - (n-1) = 1$, ein Widerspruch.
			\item
				Ist $n$ eine Carmichaelzahl, also $p-1$ teilt $n-1$, wenn $p$ ein Teiler von $n$ ist.
				Zerlege $U(\Z / n\Z) = U(\Z / p_1 \Z) \times \dotsb \times U(\Z / p_k \Z)$.
				Also $1 = a^{n-1} = (a_1^{n-1}, \dotsc, a_k^{n-1})$.
				Insbesondere kann man die $a_i$ so wählen, dass sie $U(\Z/p_i \Z)$ erzeugen.
				$p_i - 1$ teilt $n-1$ für alle $1 \le i \le k$.
			\item
				$n$ ist ungerade, wenn $n$ Carmichaelzahl ist.
				Angenommen $2$ teile $n$, dann ist $-1 = (-1)^{n-1} \equiv 1 \bmod n$, da $n$ Carmichael, also $n \divs 2$ und somit $n = 2$, ein Widerspruch.
		\end{enumerate}
	\end{proof}
\end{st}

\begin{ex*}
	Sei $561 = 3 \cdot 17 \cdot 11$ die Zahl aus \ref{3.7} c).
	Offenbar ist $561$ quadratfrei und $2 \divs 560, 10 \divs 560, 16 \divs 560$, also ist $561$ eine Carmichaelzahl.
\end{ex*}

% Lem 2.9
\begin{lem} \label{2.9}
	Ist $n$ eine ungerade Pseudoprimzahl zur Basis $2$, dann ist auch $2^n - 1$ eine Pseudoprimzahl zur Basis $2$.
	\begin{proof}
		Es gilt $2^{n-1} \equiv 1 \bmod n$, also $2^{n-1} - 1 = kn$, also $2^{2^n-2} = 2^{2kn}$ und
		\[
			2^{2^n -2} - 1 = (2^n)^{2k} - 1.
		\]
		Also $2^n - 1 \divs 2^{2^n -2} - 1 = 2^{2^n - 1 - 1}$.
		Es bleibt zu zeigen, dass $2^{n-1}$ keine Primzahl ist.
		Sei $n$ keine Primzahl, dann folgt aus $d \divs n$, dass $2^d - 1 \divs 2^n - 1$ mit $d < n$.
		Dann ist $2^d - 1 < 2^n - 1$ und ein nicht trivialer Teiler.
	\end{proof}
	\begin{note}
		Im Beweis wurde mehrfach verwendet, dass
		\[
			2^{am} - 1
			= (2^q - 1)\Big(1 + 2^a + 2^{2a} + \dotsb + 2^{(m-1)a} \Big).
		\]
	\end{note}
\end{lem}

\begin{nt*}
	\begin{enumerate}[a)]
		\item
			Nach $3.9$ existieren unendlich viele Pseudoprimzahlen zur Basis 2.
			Lange zeit war es offen, ob es unendlich viele Carmichaelzahlen gibt.
			Seit 1992 (Granville) weiß man, dass es unendlich viele gibt.
		\item
			Den Fermat'schen Primzahltest kann man verbessern.
			Es gilt: $n$ ist eine Primzahl, wenn für alle $1 < m < n$ gilt $m^{n-1} \equiv 1 \bmod n$.
			Rückrichtung mit $| U(\Z/n\Z) | = n-1$, wenn $n$ eine Primzahl ist, Hinrichtung mit kleinem Fermat.

			Auch für Carmichaelzahlen $n$ gilt:
			Ist $p$ ein echter Teiler von $n$, dann ist $\_p = p \bmod n \not\in U(\Z / n\Z)$.
	\end{enumerate}
\end{nt*}

% St 3.10
\begin{st}[Lucas] \label{3.10}
	Sei $n > 1$, $n \in \N$.
	Wenn für jeden Primteiler $p$ von $n - 1$ eine ganze Zahl $a = a(p)$ existiert mit $a^{n-1} \equiv 1 \bmod n$ und $a^{\f {n-1}p} \not \equiv 1 \bmod n$, dann ist $n$ eine Primzahl.

	\begin{proof}
		Um zu zeigen, dass $n$ eine Primzahl ist, genügt es $\phi(n) = n - 1$ zu zeigen.
		Sei $p$ eine Primzahl und $p^r$ der maximale $p$-Potenzteiler von $n-1$.
		Sei $a = a(p)$ und $e$ die Ordnung von $a \bmod n$.
		Dann ist $e$ ein Teiler von $n-1$ und $e$ teilt nicht $\f {n-1}p$, also ist $p^r \divs e$.
		Sicherlich $e \divs \phi(n)$, dann $p^r \divs \phi(n)$.
		Dies gilt für jeden Primteiler von $n-1$, also folgt $n - 1 \divs \phi(n)$ und damit $\phi(n) = n-1$.
	\end{proof}
\end{st}

% Nt + Ex 3.11
\begin{nt} \label{3.11}
	Um \ref{3.10} anwenden zu können, sollte man die Primzahlen von $n-1$ kennen.
	In Spezialfallen ist dies der Fall, z.B. für $n = 2^{16} + 1 = 65537$.

	Nach \ref{3.10} genügt zum Primzahlnachweis eine Zahl $a$ zu finden mit $a^{2^{16}} \equiv 1 \bmod 2^{16} + 1$ und $a^{2^{15}} \not\equiv 1 \bmod 2^{16} + 1$.
	Man kann dann ausrechnen, dass dies für $a = 3$ der Fall ist.
\end{nt}

Nachstes Thema sind Primzahltests im Zusammenhang mit Fermatzahlen und Mersennezahlen.

% Lem 3.12
\begin{lem} \label{3.12}
	\begin{enumerate}[a)]
		\item
			Ist $2^k - 1$ eine Primzahl (Mersenneprimzahl), dann ist $k$ eine Primzahl.
		\item
			Ist $2^k + 1$ eine Primzahl, dann ist $k = 2^n$.
	\end{enumerate}
	\begin{proof}
		\begin{enumerate}[a)]
			\item
				Ist $k = uv$, dann ist $2^k - 1 = (2^n)^v - 1$.

				$x-1 \divs x^v - 1$, da $1$ Nullstelle, also $2^n -1 \divs 2^k - 1$.
			\item
				$k = 2^r \_ u$ mit $\_ u$ ungerade, dann ist $2^k + 1 = (2^{2r})^u + 1$.
				$x + 1 \divs x^{\_ u} + 1$, da $-1$ Nullstelle von $x^{\_ u} + 1$, also $2^{2r} + 1 \divs 2^k + 1$.
		\end{enumerate}
	\end{proof}
\end{lem}

% Df 3.13
\begin{df} \label{3.13}
	\begin{enumerate}[a)]
		\item
			Die Zahlen $M_p = 2^p - 1$ heißen \emphdef[Mersenne-Zahl]{Mersenne-Zahlen} (nach Marin Mersenne, 1588-1648).
		\item
			Die Zahlen $F_n = 2^{2^n} + 1$ heißen \emphdef[Fermat-Zahl]{Fermat-Zahlen} (nach P. de Fermat, 1601-1665).
	\end{enumerate}
\end{df}

% St 3.14
\begin{st} \label{3.14}
	$F_k$ für $k \ge 1$ ist genau dann eine Primzahl, wenn
	\[
		3^{\f{F_k - 1}2} \equiv -1 \mod F_k
	\]
	\begin{note}
		Dies zeigt, dass $3$ eine Primitivwurzel ist für Fermat'sche Primzahlen.
		\ref{3.14} verifiziert \ref{3.11}.
		\ref{3.10} steckt mit $a = 3$ im Beweis von \ref{3.14}.

		Angenommen $3$ wäre nur endlich oft Primitivwurzel von $\Z / p\Z$ mit Primzahl $p$, dann existieren nur endlich viele Fermatzahlen, die Primzahlen sind.
		Man kennt $F_0 = 3, F_1 = 5, F_2 = 17, F_3 = 257, F_4 = 65537$, aber \emph{andererseits} legt die Artin'sche Vermutung nache, dass so ein Beweis nicht zu führen ist.
	\end{note}
\coursetimestamp{15}{05}{2014}
	\begin{proof}
		\begin{segnb}{$\impliedby$}
			Es gilt $F_k - 1 = 2^{2^k}$.
			Nach \ref{3.10} genügt es für $a = 3$ zu zeigen $3^{\f{F_k - 1}2} \not\equiv 1 \bmod F_k$ und $3^{F_k-1} \equiv 1 \bmod F_k$.
			Ersteres gilt nach Voraussetzung.
			Zweiteres folgt aus der Voraussetzung durch quadrieren.
		\end{segnb}
		\begin{segnb}{$\implies$}
			Umgekehrt sei $F_k$ eine Primzahl, dann gilt wegen $k \ge 1$
			\[
				2^{2^k} + 1
				\equiv (-1)^{2^k} + 1
				\equiv 1 + 1
				\equiv -1
				\mod 3.
			\]
			Also ist $\legsym{F_k}{3} = \legsym{-1}{3} = (-1)^{\f{3-1}2} = -1$.
			Nach dem quadratischen Reziprozitätsgesetz gilt aber auch mit $F_k \equiv 1 \bmod 4$ (für $k \ge 1$)
			\[
				\legsym{F_k}{3}
				= \legsym{3}{F_k}
			\]
			nach \ref{??} gilt $3^{\f{F_k}2} \equiv \legsym{3}{F_k} \bmod F_k$, also folgt
			\[
				3^{\f{F_k-1}2} \equiv -1 \mod F_k.
			\]
		\end{segnb}
	\end{proof}
\end{st}

% St 3.15
\begin{st} \label{3.15}
	Sei $n \ge 2$ und $p$ ein Primteiler von $F_n$, dann gilt $p \equiv 1 \bmod 2^{n+2}$.
	\begin{proof}
		Wegen $p \divs F_n$ ist $2^{2^n} + 1 \equiv 0 \bmod p$, also $2^{2^n} \equiv - 1 \bmod p$.
		Damit ist $2^{2^n \cdot 2} = 2^{2^{n+1}} \equiv 1 \bmod p$.
		Die Ordnung von $\_ 2$ in $U(\Z / p\Z)$ ist $2^{n+1}$, also $2^{n+1} \divs p - 1$ ($U(\Z / p\Z)$ ist zyklisch von Ordnung $p-1$) und $p \equiv 1 \bmod 2^{n+1}$.
		Wegen $n \ge 2$ folgt aus $p \equiv 1 \bmod 2^{n+1}$ insbesondere $p \equiv 1 \bmod 8$.
		Mit dem zweiten Ergänzungssatz ist $\legsym{2}{p} = 1$ (denn wenn $p \equiv 1 \bmod 8$, also $p = 8k + 1$ und $p^2 = 64k^2 + 16k + 1$ und $p^2 - 1 = 64k^2 + 16k$ und $\f{p^2 - 1}{8}$ ist gerade).

		Es existiert $x$ mit $x^2 \equiv 2 \bmod p$, also $x^{2^{n+1}} \equiv 2^{2^n} \equiv -1 \bmod p$ und $x^{2^{n+2}} \equiv 1 \bmod p$.
		Also ist die Ordnung $\_ x$ in $U(\Z / p\Z) = 2^{n+2}$ und $2^{n+2} \divs p-1$, also $p \equiv 1 \bmod 2^{n+2}$.
	\end{proof}
\end{st}

% Bem 3.16
\begin{nt} \label{3.16}
	In einem Brief um 1640 vermutete Fermat, dass alle Zahlen $F_n$ Primzahlen sind.
	$F_5$ ist keine Primzahl (gezeigt von Euler 1732), denn $641 \divs F_5$.

	Nach \ref{3.15} gilt für einen Primteiler von $F_5$, dass $p \equiv 1 \bmod 2^7 = 128$.
	Daraus folgt $p \in \{129, 257, 385, 513, 641, \dotsc\}$.
	Es gilt $3 \divs 139, 5 \divs 385, 3 \divs 513$.
	Man errechnet, dass $257$ kein Teiler von $F_5$ ist, aber $641$ schon.
	Es gilt
	\[
		F_5 = 641 \cdot (2^7 \cdot 52347 + 1).
	\]
	Es ist unbekannt, ob es nach $F_1, \dotsc, F_4$ weitere Primzahlen gibt.
\end{nt}

% St 3.17
\begin{st}[Lucas-Lehmer Test] \label{3.17}
	Sei $p > 2$ Primzahl und die Folgt $(s_n)_{n\in\N}$ sei rekursiv definiert durch
	\begin{align*}
		s_1 &:= 4, &
		s_{n} &:= s_{n-1}^2 - 2.
	\end{align*}
	Dann ist $M_p$ genau dann eine Primzahl, wenn $M_p$ ein Teiler von $s_{p-1}$ ist.
	\begin{proof}
		Setze $w := 2 + \sqrt 3, \tilde w := 2 - \sqrt 3$, dann ist $w\tilde w = 1$.
		Damit sind $w, \tilde w \in \Z[\sqrt{3}] = \{ z_1 + z_2 \sqrt{3} : z_1, z_2 \in \Z \}$ Einheiten in dem Teilring $\Z[\sqrt{3}]$ von $\R$.
		Induktion nach $m$ zeigt \Exercise
		\[
			s_m = w^{2^{m-1}} + \tilde w^{2^{m-1}}.
		\]
		$M_p$ teilt $s_{p-1}$ genau dann, wenn $w^{2^{p-2}} + \tilde w^{2^{p-2}} \equiv 0 \bmod M_p$ und genau dann, wenn $w^{2^{p-1}} + 1 \equiv 0 \bmod M_p$.
		Annahme $M_p$ hätte einen nicht-trivialen Primteiler $q$, dann auch einen mit $q^2 \le M_p$ und sicherlich $2 < q$.
		Betrachte den Faktorring $R := \Z[\sqrt{3}] / q\Z[\sqrt{3}]$.
		Also ist
		\[
			R = \{ a + b\sqrt{3} : a,b \in \Z / q\Z \},
		\]
		d.h. $|R| = q^2$.
		Dann ist $|U(R)| \le q^2 - 1$.
		Es gilt $\_{1} \neq \_{-1} \bmod \_m$ sind Einheiten von $R$ und $\_w^{2^{p-1}} = \_{-1}$, also $\_w^{2^p} = \_ 1$ und die Ordnung von $\_w$ in $U(R)$ ist $2^p$.
		Damit ist $2^p \divs |U(R)|$, also insbesondere
		\[
			2^p \le |U(R)| \le q^2 - 1 \le M_p - 1 \le 2^{p}-2,
		\]
		ein Widerspruch.
		Also hat $M_p$ keinen nicht-trivialen Primteiler und ist damit Primzahl.

		% fixme: segments
		Sei umgekehrt $M_p$ eine Primzahl.
		Setze $a = \f{1 + \sqrt{3}}{\sqrt{2}}$ und $\tilde a = \f{1 - \sqrt{3}}{\sqrt{2}}$
		Man sieht $a^2 = \f{1 + 2\sqrt{3} + 3}2 = 2 + \sqrt{3} = w$ und analog $\tilde a^2 = \tilde w$.
		Zu zeigen ist $M_p$ teilt $s_{p-1}$, oder äquivalent $a^{2^p} + 1 \equiv 0 \bmod M_p$ (siehe andere Richtung des Beweises).
		Schreibe $\tilde q = M_p$, dann ist $\sqrt{2} a = 1 + \sqrt{3}$ und
		\[
			(\sqrt{2} q)^{\tilde q} = a^{\tilde q} 2^{\f{\tilde q - 1}2} \sqrt{2}
			\in \Z[\sqrt{3}].
		\]
		Modulo $\tilde q$ (genauso modulo $\tilde q \Z[\sqrt{3}]$) gilt
		\[
			(\sqrt 2 a)^{\tilde q}
			= (1 + \sqrt 3)^{\tilde q}
			\equiv 1 + \sqrt{3}^{\tilde q}
			\equiv 1 + 3^{\f{\tilde q -1}2} \sqrt{3}
			\mod \tilde q
		\]
		Sicherlich ist $\tilde \equiv -1 \bmod 8$ und mit dem quadratischen Reziprozitätsgesetz
		\[
			2^{\f{\tilde q - 1}2} \equiv \legsym{2}{q}
			\equiv 1 \bmod \tilde q.
		\]
		Außerdem ist $\tilde q \equiv -1 \bmod 4, 3 \equiv -1 \bmod 4$ und $\tilde q \equiv 1 \bmod 3$.
		Damit ist $\legsym{3}{\tilde q} = - \legsym{\tilde q}{3} = -1$.
		Ferner gilt $\legsym{3}{\tilde q} \equiv 3^{\f{\tilde q-1}2} \bmod \tilde q$.

		Damit ist $a^{\tilde q} \sqrt{2} \equiv 1 - \sqrt{3} \bmod \tilde q$ und somit
		\[
			2 a^{\tilde q + 1}
			\equiv \underbrace{\sqrt{2} a}_{=1+\sqrt{3}} (1 - \sqrt{}∫3)
			\equiv -2 \bmod \tilde q.
		\]
		$2$ ist invertierbar modulo $\tilde q$, also ist $a^{\tilde q + 1} \equiv -1 \bmod \tilde q$, d.h. $a^{2p} \equiv -1 \bmod M_p$.
	\end{proof}
\end{st}

\begin{ex*}
	Ist $M_7 = 2^7 - 1 = 127$ eine Primzahl?.
	Berechne die Folge $(s_p)_{p\in \N}$ modulo $127$.
	\begin{align*}
		s_1 &\equiv 4, &
		s_2 &\equiv 14, &
		s_3 &\equiv 67, &
		s_4 &\equiv 42, &
		s_5 &\equiv 111, &
		s_6 &\equiv 0
		\mod 127
	\end{align*}
	also ist $M_7$ eine Primzahl.
\end{ex*}

% Prop 3.19
\begin{prop}[1. Teil des Miller-Rabin Tests]
	Sei $n \in \P, a \in \Z$ mit $(a, n) = 1$.
	Setze $n - 1 = 2^s d$ mit maximalem 2-Potenzteiler $2^s$ von $n-1$.
	Dann gilt entweder
	\[
		a^d \equiv 1 \mod n
	\]
	oder
	\[
		\exists \tilde r \in \{0, \dotsc, s-1\} :
		2^{\tilde r} d \equiv -1 \bmod n.
	\]
\end{prop}





\printindex[lectures]
\printindex[terms]
%\printbibliography


\end{document}
