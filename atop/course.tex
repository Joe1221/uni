% This work is licensed under the Creative Commons
% Attribution-NonCommercial-ShareAlike 3.0 Unported License. To view a copy of
% this license, visit http://creativecommons.org/licenses/by-nc-sa/3.0/ or send
% a letter to Creative Commons, 444 Castro Street, Suite 900, Mountain View,
% California, 94041, USA.

\documentclass{mycourse}

\ExplSyntaxOn

%\cs_gset_nopar:Npn \thechapter { \Alph { chapter } }
%\cs_gset_nopar:Npn \thesection { \thechapter \relax \arabic { section } }

\DeclareDocumentCommand \Cat { m } { \underline{\mathtt{#1}} }
\DeclareDocumentCommand \S { } { \mathbb{S} }
\DeclareDocumentCommand \H { } { \mathbb{H} }
\DeclareDocumentCommand \id { } { \Id }
\DeclareDocumentCommand \dunion { } { \sqcup }
\DeclareDocumentCommand \B { } { \mathbb{B} }
\DeclareDocumentCommand \D { } { \mathbb{D} }
\DeclareDocumentCommand \RP { } { \mathbb{RP} }
\DeclareDocumentCommand \homeomorphic { } { \cong }
\DeclareDocumentCommand \homotopic { } { \simeq }
\DeclareDocumentCommand \normdiv { } { \mathrel{\triangleleft} }
\DeclareDocumentCommand \ops { G{} } { \xto[cap]{#1} }
\DeclareDocumentCommand \b { } { \flat }
\DeclareDocumentCommand \bs { } { \mathrel{\backslash} }
\DeclareDocumentCommand \quot { } { \mathrm{quot} }
\DeclareDocumentCommand \semiprod { G{} } { \mathrel{\stack{#1}{\rtimes}} }
\DeclareDocumentCommand \twprod { G{} } { \mathrel{\stack{#1}{\rtimes}} }
\DeclareDocumentCommand \rang { } { \operatorname{rang} }
\DeclareDocumentCommand \homto { G{} } { \xto[homeomorphic]{#1} }
\DeclareDocumentCommand \Homeo { } { \operatorname{Homeo} }
\DeclareDocumentCommand \Int { } { \operatorname{Int} }
%[\alpha] \ast [\beta] looks weird with this?
%\cs_new_eq:NN \ast_old \ast
%\DeclareDocumentCommand \ast { } { \mathop{\ast_old} }
\cs_new_eq:NN \sim_old \sim
\DeclareDocumentCommand \sim { } { \mathrel{\sim_old} }


\DeclareDocumentCommand \over { } { \divs }

\cs_new_eq:NN \text_old \text
\DeclareDocumentCommand \text { m } { \mathord{\text_old{#1}} }

\ExplSyntaxOff
\tikzset {
    commutative diagrams/ops/.style = { bend left, shift right=1.5ex },
}
\ExplSyntaxOn


\DeclareDocumentCommand { \GenMonoid } { om } {
    \tl_clear_new:N \l_content_tl
    \tl_set:Nn \l_content_tl { #2 }

    \tl_replace_all:Nnn \l_content_tl { & } { \c_mymath_set_delim_tl }

    \tl_if_in:NnTF \l_content_tl { \c_mymath_set_delim_tl } {
	\mathchoice {
	    \left[ \big. \mkern1mu \mymath_set_vertbar:nn{\l_content_tl}{\middle|} \mkern2mu \right]
	} {
	    [ \mkern1mu \mymath_set_vertbar:nn{\l_content_tl}{\mkern-2mu | \mkern-2mu} \mkern1mu ]
	} {
	    [ \mymath_set_vertbar:nn{\l_content_tl}{\mkern-1mu | \mkern-1mu} ]
	} {
	    [ \mymath_set_vertbar:nn{\l_content_tl}{\mkern-1mu | \mkern-1mu} ]
	}
    } {
        \mathchoice {
            \left[ \big.\mkern-1mu \l_content_tl \right]
        } {
            [ \l_content_tl ]
        } {
            [ \l_content_tl ]
        } {
            [ \l_content_tl ]
        }
    }
}

\DeclareDocumentCommand { \Gen } { om } {
    \tl_clear_new:N \l_content_tl
    \tl_set:Nn \l_content_tl { #2 }

    \tl_replace_all:Nnn \l_content_tl { & } { \c_mymath_set_delim_tl }

    \tl_if_in:NnTF \l_content_tl { \c_mymath_set_delim_tl } {
	\mathchoice {
	    \left\< \big. \mkern1mu \mymath_set_vertbar:nn{\l_content_tl}{\middle|} \mkern2mu \right\>
	} {
	    \< \mkern1mu \mymath_set_vertbar:nn{\l_content_tl}{\mkern-2mu | \mkern-2mu} \mkern1mu \>
	} {
	    \< \mymath_set_vertbar:nn{\l_content_tl}{\mkern-1mu | \mkern-1mu} \>
	} {
	    \< \mymath_set_vertbar:nn{\l_content_tl}{\mkern-1mu | \mkern-1mu} \>
	}
    } {
        \mathchoice {
            \left\< \big.\mkern-1mu \l_content_tl \right\>
        } {
            \< \l_content_tl \>
        } {
            \< \l_content_tl \>
        } {
            \< \l_content_tl \>
        }
    }
}

\ExplSyntaxOff

\title{Algebraische Topologie}

\begin{document}

\maketitle

\tableofcontents

% Kap 0
\chapter{Einführende Motivation}

\coursetimestamp{16}{10}{2013}

% 0.1.
\section{Stochastik = „Lehre vom Zufall“}


Zufall ist eine Eigenschaft des Lebens.
Es gibt „gewollten Zufall“ (z.B. Glücksspiele) und „ungewollten Zufall“ (z.B. Katastrophen).
Die Stochastik (artgr., „scharfsinniges Vermuten“) besitzt viele Anwendungen, basiert jedoch auf einer strengen axiomatischen Theorie (1933, Kolmogorov).

Die Frage nach der Existenz des Zufalls ist nicht eindeutig zu beantworten (und ist hier auch nicht von Interesse), jedoch ist der Zufall oft ein gutes Modell für deterministische Geschehnisse, die (z.B. auf Grund ihrer hohen Komplexität) nicht (bzw. nicht komplett) vorhersagbar sind.

Teilgebiete der Stochastik sind die Wahrscheinlichkeitstheorie und die Statistik.
Die Wahrscheinlichkeitstheorie behandelt dabei die theoretische Seite der Stochastik, während die Statistik eher (!) Anwendungen beinhalten.

% fixme: kapitel-referenz
Die Wahrscheinlichkeit $\P(A)$ eines Ereignisses $A$ ist eine reelle Zahl zwischen Null und Eins (jeweils inkl.), die die Plausibilität des Eintreffens von $A$ anzeigt („misst“, siehe Maßtheorie).


\section{Zwei Beispiele und Grundbegriffe}

% Bsp 0.2.1
\begin{ex}[Würfeln mit zwei fairen Würfeln] \label{0.2.1}
	Wie groß ist die Wahrscheinlichkeit, dass $A := \text{die Augensumme $\ge 10$}$ ist?
	Wir zeigen zwei Lösungswege
	\begin{description}
		\item[1. Weg]
			Beide Würfel sind unterscheidbar, wir können den Wurf also als Element von
			\begin{align*}
				\Omega
				&= \{ 1,2,3,4,5,6 \} \times \{ 1,2,3,4,5,6 \} \\
				&= \{ (a,b): a,b \in \{ 1, \dotsc, 6 \}
			\end{align*}
			betrachten.
			Alle diese Paare haben die selbe Wahrscheinlichkeit, nämlich $\f 1{\# \Omega} = \f 1{36}$.
			\begin{align*}
				\P(A)
				&= \f 1{36} \# \{ (4,6), (5,5), (6,4), (5,6), (6,5), (6,6) \} \\
				&= \f {\# A}{\# \Omega}
				= \f {6}{36}
				= \f 16
			\end{align*}
		\item[2. Weg]
			Betrachte gleich die Augensummen
			\[
				\Omega
				= \{ 2, 3, \dotsc, 12 \}.
			\]
			Diese Elemente aus $\Omega$ haben nicht mehr die gleiche Wahrscheinlichkeit.
			\begin{table}
				\caption{Wahrscheinlichkeiten der auftretenden Augensummen}
				\centering
				\begin{tabular}{r|ccccccccccc}
					$\omega$ & 2 & 3 & 4 & 5 & 6 & 7 & 8 & 9 & 10 & 11 & 12 \\ \hline
					$\P(\omega)$ & $\f 1{36}$ & $\f 2{36}$ & $\f 3{36}$ & $\f 4{36}$ & $\f 5{36}$ &$\f 6{36}$ &$\f 5{36}$ &$\f 4{36}$ &$\f 3{36}$ &$\f 2{36}$ &$\f 1{36}$
				\end{tabular}
			\end{table}
			Addieren der Wahrscheinlichkeiten für $\omega \ge 10$ ($A = \{10,11,12\}$) ergibt
			\[
				\P(A)
				= \f {3+2+1}{36}
				= \f 16
			\]
	\end{description}
\end{ex}

\paragraph{Fazit:}
\begin{enumerate}[1.]
	\item
		Man definiert eine Menge $\Omega$, den Raum der möglichen Ereignisse.
	\item
		Dafür gibt es i.A. mehrere Möglichkeiten.
	\item
		Oft ist es günstig, wenn \emph{alle} Elemente von $\Omega$ die selbe Wahrscheinlichkeit besitzen.
		In diesem Fall nennen wir den Raum \emph{Laplace-Wahrscheinlichkeitsraum}.
	\item
		Das interessierende Ereignis identifiziert man mit einer Teilmenge $A \subset \Omega$.
	\item
		Die Wahrscheinlichkeit von $A$ ist dann die Summe der Einzelwarscheinlichkeiten der Elemente von $A$, d.h.
		\[
			\P(A) = \sum_{\omega \in A} \P(\{\omega\}).
		\]
\end{enumerate}

\begin{df} \label{0.2.2}
	Sei $\Omega \neq \emptyset$ eine endliche, abzählbare Menge und $\scr P(\Omega)$ ihre Potenzmenge.
	Ein \emph{diskreter Wahrscheinlichkeitsraum} ist ein Paar $(\Omega, p)$, wobei $p: \Omega \to [0,1]$ mit der Eigenschaft
	\[
		\sum_{\omega \in \Omega} p(\omega) = 1.
	\]
	Wir nennen $\Omega$ \emph{Ereignissumme}, ein Element $\omega \in \Omega$ \emph{Elementarereignis} und eine Teilmenge $A \subset \Omega$ \emph{Ereignis}.
\end{df}

\begin{df} \label{0.2.3}
	Die Abbildung $\P : \scr P(\Omega) \to [0,1]$ definiert durch
	\[
		\P(A) := \sum_{\omega \in A} p(\omega)
	\]
	heißt das \emph{von $p$ induzierte Wahrscheinlichkeitsmaß}.
\end{df}

\begin{nt*}
	In einer Summe $\sum_{\omega \in A} p(\omega)$ spielt die Reihenfolge der Summation keine Rolle, da $p(\omega) \ge 0$.
	Streng genommen ist nämlich
	\[
		\sum_{\omega \in A} p(\omega)
		= \lim_{n\to \infty} \sum_{j=1}^n p(\omega_i),
	\]
	wobei $\omega_1, \omega_2, \dotsc$ eine Abzählung von $A$ ist.
\end{nt*}

Die Definitionen \ref{0.2.2} und \ref{0.2.3} sind Spezialfälle eines allgemeines Konzepts.

\begin{nt}[Kolmogoroff'sche Axiome] \label{0.2.4}
	Sei $(\Omega, p)$ ein diskreter Wahrscheinlichkeitsraum.
	Dann hat das Wahrscheinlichkeitsmaß $\P$ folgende Eigenschaften
	\begin{enumerate}[(i)]
		\item
			\emph{Normierung}:
			$\P(\emptyset) = 0, \P(\Omega) = 1$
		\item
			\emph{$\sigma$-Additivität}:
			Für alle paarweise disjunkten Folgen $(A_i)_{i \in \N}$, $A_i \subset \Omega$ gilt
			\[
				\P \bigg( \bigcup_{i=1}^\infty \bigg) = \sum_{i=1}^\infty \P(A_i).
			\]
	\end{enumerate}
	\begin{proof}
		Übung
	\end{proof}
\end{nt}

\begin{nt} \label{0.2.5}
	Umgekehrt gilt:
	Ist $\Omega \neq \emptyset$ abzählbar und $\P : \scr P(\Omega) \to [0,1]$ eine Abbildung mit Eigenschaften (i) und (ii) aus \ref{0.2.4}, so definiert $p(\omega) := \P({\omega})$, $\omega \in \Omega$ einen diskreten Wahrscheinlichkeitsraum $(\Omega, p)$.
\end{nt}

\paragraph{Sprechweisen für Ereignisse:}
Seien $A, B, C \subset \Omega$ Ereignisse.
\begin{description}
	\item[$A \cap B$]
		$A$ und $B$ treten ein.
	\item[$A \cup B$]
		$A$ oder $B$ treten ein.
	\item[$A^C = \Omega \setminus A$]
		$A$ tritt nicht ein.
	\item[$A \cap B = \emptyset$]
		$A$ und $B$ schließen einander aus.
	\item[$A \subset B$]
		$A$ zieht $B$ nach sich.
	\item[$\emptyset$]
		Unmögliches Ereignis.
	\item[$\Omega$]
		Sicheres Ereignis.
\end{description}

\begin{ex} \label{0.2.6}
	Tropfen fallen gleichmäßig auf ein Quadrat $Q$.
	Wir betrachten eine Teilmenge $A \subset \Omega$, also
	\[
		%fixme: entspricht \hat =
		\P(A) = \text{nächster Tropfen, der in $Q$ fällt trifft $A$}.
	\]
	Es ist naheliegend
	\[
		\P(A)
		= \f {\Area(A)}{\Area(Q)}
	\]
	zu setzen.

	Für welche Mengen $A$ ist der Flächeninhalt definiert?
	Wir werden sehen, dass dies nicht für alle Mengen $A \in \scr P(\Omega)$ der Fall ist.
	$\P$ wird daher auf einer echten Teilmenge von $\scr P(\Omega)$, einer sogenannten $\sigma$-Algebra definiert.
\end{ex}


\chapter{Grundlagen}

\section{Definitionen und Notationen}

\begin{df}[Multiindex und partielle Ableitung] \label{1.1}
	Sei $u: \R^d \to \R$ hinreichend oft differenzierbar.
	Wir nennen $\beta = (\beta_1, \dotsc, \beta_d)^T \in \N_0^d$ mit $k := |\beta| := \sum_{i=1}^d \beta_i$ einen \emphdef[Multiindex]{Multiindex der Ordnung $k$}.

	Wir definieren
	\[
		\partial^\beta u := (\pddx[x_1])^{\beta_1} \dotsb (\pddx[x_d])^{\beta_d} u,
	\]
	die \emphdef[partielle Ableitung]{partielle Ableitung von $u$ zum Index $\beta$}.

	Sei $\mathbb{B}_k := \Set{\beta \in \N_0^d & |\beta| = k}$ die Menge aller Multiindizes der Ordnung $k$ und
	\[
		\D^k u := (\partial^\beta u)_{\beta \in \mathbb{B}_k}
	\]
	der Vektor aller partieller Ableitungen der Ordnung $k$ (mit beliebiger Reihenfolge).
\end{df}

\begin{df}[Ableitungsoperatoren] \label{1.2}
	Für $u: \R^d \to \R$ hinreichend oft differenzierbar definieren wir den \emphdef[Gradient]{Gradienten}
	\[
		\grad u(x) := \nabla u(x) := \Vector{ \partial_{x_1} u(x) & \dots & \partial_{x_d} u(x) },
	\]
	wobei $x = (x_1, \dotsc, x_d)$ und $\partial_{x_i} := \pddx[x_i]$ für $i = 1, \dotsc, d$.

	Für ein hinreichend oft differenzierbares Vektorfeld $v: \R^d \to \R^d$ definieren wir die \emphdef{Divergenz} durch
	\[
		\div v(x) := \nabla \cdot v(x) = \sum_{i=1}^d \partial_{x_i} v_i (x)
	\]
	und im Fall $d = 3$ zusätzlich die \emphdef{Rotation} durch
	\[
		\rot v(x) := \nabla \times v(x) = \Vector{ \partial_{x_2} v_3 - \partial_{x_3} v_2 & \partial_{x_3} v_1 - \partial_{x_1} v_3 & \partial_{x_1} v_2 - \partial_{x_2} v_1 }.
	\]
	Wir nutzen die Abkürzung $\partial_{x_i}^2 := (\partial_{x_i})^2$ und definieren den \emphdef{Laplace-Operator} durch
	\[
		\Laplace u(x) := \nabla \cdot (\nabla u) = \div( \grad  u(x) ) = \sum_{i=1}^d \partial_{x_i}^2 u(x).
	\]
	Skalare Operatoren werden für vektorielle Funktionen komponentenweise definiert, z.B.
	\[
		\Laplace v(x) := \Vector{\Laplace v_1(x) & \dots & \Laplace v_d(x)},
	\]
	und für $b \in \R^d$
	\[
		(b \cdot \nabla) v := \Big(\sum_{i=1}^d b_i \partial_{x_i}\Big) v
		= \Vector { \sum_{i=1}^d b_i \partial_{x_i} v_1 & \dots & \sum_{i=1}^d b_i \partial_{x_i} v_d }.
	\]
\end{df}

\begin{df}[Räume stetig differenzierbarer Funktionen] \label{1.3}
	Sei $\Omega \subset \R^d$ offen und beschränkt.
	Wir bezeichnen mit $C^m(\_\Omega, \R^n)$ den Raum der $m$-mal stetig differenzierbaren Funktionen (differenzierbar auf $\Omega$, sodass die $m$-ten Ableitungen stetig auf $\_\Omega$ fortsetzbar sind) von $\_\Omega$ nach $\R^n$.

	Für $n = 1$ schreiben wir auch kurz $C^m(\_\Omega) = C^m(\_\Omega, \R^1)$ und definieren hier für $u \in C^0(\_\Omega)$ die \emphdef{Supremumsnorm}
	\[
		\|u\|_\infty := \sup_{x\in \_\Omega} = |u(x)|.
	\]
	Auf $C^m(\_\Omega)$ definieren wir damit eine Norm:
	\[
		\|u\|_{C^m(\_\Omega)} := \sum_{|\beta| \le m} \|\partial^\beta u\|_\infty,
	\]
	wobei $u \in C^m(\_\Omega)$.
	\begin{note}
		\begin{itemize}
			\item
				$C^m(\_\Omega)$ ist ein Banachraum, d.h. ein vollständiger, normierter Raum (Alt, Lemma 1.8)
			\item
				Man kann auch $C^m(\Omega)$ für offenes oder potentiell unbeschränktes $\Omega$ und auch $m = \infty$ definieren.
				Statt einer Norm wird dann eine Metrik (Frechét-Metrik) eingeführt, bzgl. der $C^m(\Omega)$ immernoch vollständig ist (Alt, Abschnitt 1.6).
		\end{itemize}
	\end{note}
\end{df}

\begin{df}[$L^p$-Räume] \label{1.4}
	Für $p \in [1, \infty)$ definieren wir
	\[
		\tilde L^p(\Omega) := \Set{ u: \Omega \to \R \text{ Lebesgue-messbar} & \big(\mathsmaller{\int}_\Omega |u|^p \big)^{\f 1p} < \infty}
	\]
	mit Seminorm
	\[
		\|u\|_p := \|u\|_{L^p(\Omega)}  := \Big(\int_\Omega |u|^p \Big)^{\f 1p}.
	\]
	Für $p = \infty$ definieren wir
	\[
		\tilde L^\infty(\Omega) := \Set{ u: \Omega \to \R \text{ Lebesgue-messbar} & \esssup_{x\in \Omega} |u(x)| < \infty }
	\]
	mit $\|u\|_\infty := \|u\|_{L^\infty(\Omega)} := \esssup_{x\in \Omega} |u(x)|$.

	Sei $\sim$ die Äquivalenzrelation auf $\tilde L^p(\Omega)$ via
	\[
		u \sim v \defiff \exists N \subset \Omega \text{ Nullmenge} : \forall x \in \Omega \setminus \N : u(x) = v(x).
	\]
	Dann definieren wir $L^p(\Omega)$ als die Menge der Äquivalenzklassen
	\[
		L^p(\Omega) := \tilde L^p(\Omega) / \sim.
	\]
	$\|u\|_p$ ist auf jeder einzelnen Äquivalenzklasse konstant und ist daher ohne weiteres auf $L^p(\Omega)$ erweiterbar.
	Wir nennen $(L^p(\Omega), \|\argdot\|_p)$ \emphdef[$L^p$-Raum]{normierter $L^p$-Raum}.
	\begin{note}
		\begin{itemize}
			\item
				$L^p(\Omega)$ ist vollständig bezüglich $\|\argdot\|_p$, also ein Banachraum (Alt, Lemma 1.1, Satz 1.14),
			\item
				Elemente von $L^p(\Omega)$ sind also Äquivalenzklassen von Funktionen, die sich nur auf einer Nullmenge unterscheiden.
				Konsequente Unterscheidung zwischen Funktionen und Äquivalenzklassen wäre mühsam.
				Daher folgende praktische Konvention: $u \in L^p(\Omega)$ soll heißen $u \in U \in L^p(\Omega)$ für eine geeignete Äquivalenzklasse $U$ mit $u: \Omega \to \R$ als Repräsentant von $U$.
			\item
				Wir nennen $L^p(\Omega)$ trotz der Äquivalenzklassen einen \emph{Funktionenraum}.
				Beim Arbeiten mit $L^p$-Räumen muss jedoch immer bedacht/hinterfragt werden, ob die betrachteten Operationen sinnvoll definiert sind, d.h. unabhängig vom Repräsentanten sind.
				Beispielsweise ist die Punktauswertung $u(x)$ nicht wohldefiniert, eine Mittelung über alle Funktionswerte jedoch schon.
\Timestamp{2014-10-17}
			\item
				$(u,v) := \<u,v\>_{L^(\Omega)} := \int_\Omega uv$ ist ein Skalarprodukt auf $L^2(\Omega)$ und $\|u\|_2 = \sqrt{(u,v)}$, also $L^2(\Omega)$ ist vollständig bezüglich einer aus einem Skalarprodukt induzierten Norm, also ein sogenannter \emphdef{Hilbertraum}.
			\item
				Zu einem Banachraum $V$ ist der \emph{Dualraum} $V'$
				\[
					V' := \Set{ \phi: V \to \R & \phi \text{ linear und stetig} }
				\]
				mit der induzierten Norm
				\[
					\|\phi\|_{V'} := \sup_{u\in V\setminus \Set 0} \frac{\phi((u)}{\|u\|_V}
				\]
				wieder ein Banachraum.
			\item
				Für $1 < p,q < \infty$ mit $\f 1p + pq = 1$ ist $L^p(\Omega)$ ist isomorph zu $(L^q(\Omega))'$.
			\item
				$L^2(\Omega)$ ist alsowegen $\f 12 + \f 12 = 1$ isomorph zu $L^2(\Omega)'$.
		\end{itemize}
	\end{note}
\end{df}

\begin{df}[lokal integrierbare Funktionen] \label{1.5}
	Wir definieren
	\[
		L^1_{\text{loc}}(\Omega) :=
		\Set{ u : \Omega \to \R \text{ lebesgue-messbar} & \forall  K \subset \Omega \text{ kompakt } : \int_k |u(x)| \di < \infty }
	\]
\end{df}

\begin{ex*}
	\begin{itemize}
		\item
			$L^1(\Omega) \subset L^1_{\text{loc}}(\Omega)$.
			Aus $u(x) = 1, x \in \Omega := \R$ folgt $u \not\in L^1$, aber $u \in L^1_{\text{loc}}(\Omega)$.
	\end{itemize}
\end{ex*}

\begin{df}[Funktionen mit kompaktem Träger] \label{1.6}
	Wir definieren für $\Omega \subset \R^d$ offen (möglicherweise nubeschränkt), $m \in \N_0 \cup \Set \infty$.
	\[
		C_0^m(\Omega)
		:= \Set{ u \in C^m(\Omega) & \supp(u) \subset \Omega \text{ ist beschränkt} },
	\]
	wobei $\supp(u) := \_{\Set{x \in \Omega & u(x) \neq 0}}$ (also abgeschlossen) den \emphdef{Träger} (engl. “support”) von $u$ bezeichnet.
\end{df}

\begin{st}[Fundamentalsatz der Variatonsrechnung] \label{1.7}
	Sei $u \in L^1_{\text{loc}}, \Omega \in \R^d$ offen. Dann sind äquivalent
	\begin{enumerate}[i)]
		\item
			$\forall v \in C_0^\infty(\Omega) : \int_\Omega uv = 0$
		\item
			$u = 0$ fast überall in $\Omega$.			
	\end{enumerate}
	\begin{proof}
		2.11 in Alt.
	\end{proof}
\end{st}

\begin{df}[Skalare PDE] \label{1.8}
	Sei $F: \R^{|\mathbb{B}_k} \times \R^{|B_{k-1}} \times \dotsb \times \R^d \times \R \times \Omega \to \R$ gegeben.
	Dann ist
	\[ \label{eq:1.1}
		F(\D^k u(x), D^{k-1} u(x), \dotsc, D^1 u(x), u(x), x) = 0,
		\qquad x \in \Omega
	\]
	eine \emphdef[partielle Differentialgleichung!skalare]{skalare partielle Differentialgleichung der Ordnung $k$} für eine unbekannte Lösung $u: \Omega \to \R$.
\end{df}

\begin{df}[Lineare/Nichtlineare PDE] \label{1.9}
	Die PDE \eqref{1.1} ist
	\begin{enumerate}[i)]
		\item
			\emphdef{linear}, falls sie die Form $\sum_{|\beta| \le k} a_\beta(x) \partial^\beta u(x) = f(x)$ für Multiindex $\beta \in \N_0^d$ und gegebenen Funktionen $a_\beta, f$ besitzt.
			Die PDE heißt \emphdef{homogen}, falls $f(x) = 0$, sonst \emphdef{inhomogen}.
		\item
			\emphdef{semilinear}, falls sie die Form
			\[
				\sum_{|\beta| = k} a_\beta(x) \partial^\beta u(x) + a (D^{k-1} u(x), \dotsc, D^1 u(x), u, x) = 0
			\]
			besitzt.
		\item
			\emphdef{quasilinear}, falls sie die Form
			\[
				\sum_{|\beta| = k} a_\beta(D^{k-1}u, \dotsc, D^1 u, u, x) \partial^\beta uu(x) + a(D^{k-1} u(x), \dotsc, u, x) = 0
			\]
			besitzt.
		\item
			\emphdef{voll nichtlinear} falls sie die nichtlinear von $D^k$ abhängt.
	\end{enumerate}
	\begin{note}[Systeme]
		Ein System von PDEs ist eine Sammliung mehrerer skalarer PDEs für mehrere unbekannte Funktionen $u = (u_1, \dotsc, u_n)^T$.
		Typischerweise sind die einzelnen PDEs miteinander gekoppelt und die Anzahl der Gleichungen und der Unbekannten stimmen überein. 
	\end{note}
	\begin{note}[zeitäbhängige Probleme]
		Alle Notationen und Definitionen für $\Omega \subset \R^d$ mit $x = (x_1, \dotsc, x_d)^T \in \Omega$ erweitern wir auf Orts-Zeit-Zylinder $\Omega_T := \Omega \times (0, T) \subset \R^d \times \R$ mit $(x,t) \in \Omega_T$, $T \in \R^+ \cup \Set \infty$.
		Ortsvariable $x$, Zeitvariable $t$.
		Insbesondere $\partial_t := \pddx[t], \partial_t^2 := \pddx[t^2]$.
		Dann bezeichnet für $u \in C^1(\Omega_t)$
		\begin{align*}
			\nabla_x u(x) &:= \Vector{ \partial_{x_1} u(x) & \dots & \partial_{x_d} u(x) }, \\
			\Laplace_x u(x) &:= \sum_{i=1}^d \partial_{x_i}^2 u(x)
		\end{align*}
		Falls keine Verwechslungsgefahr besteht, lässt man $x$ bei $\nabla_x, \Laplace_x$ auch weg.
	\end{note}
\end{df}

\begin{ex}[Lineare PDEs] \label{1.10}
	Einige häufig begegneten PDEs (Koeffizienten der Einfachheit halber $1$).
	\begin{itemize}
		\item
			\emphdef{Laplace-Gleichung}: $-\Laplace u = 0$,
		\item
			\emphdef{Poisson-Gleichung}: $-\Laplace u = f$,
		\item
			\emphdef{Helmholtz-Gleichung}: $-\Laplace u - \lambda u = 0$ für $\lambda > 0$,
		\item
			\emphdef{Advektions-Gleichung}: $\partial_t u + b \cdot \nabla u = 0$ für $b \in \R^d$
		\item
			\emphdef{Wärmeleitungs-Gleichung} oder \emphdef{Diffusionsgleichung}: $\partial_t u - \Laplace u = 0$.
		\item
			\emphdef{Schrödinger-Gleichung}: $i \partial_t u + \Laplace u = 0$, wobei $i = \sqrt{-1} \in \C$.
		\item
			\emphdef{Wellengleichung}: $\partial_t^2 u - \Laplace u = 0$,
		\item
			\emphdef{Airy's-Gleichung}: $\partial_t u + \partial_x^3 u = 0$
		\item
			\emphdef{Balken-Gleichung}: $\partial_t u + \partial_x^4 u = 0$.
		\item
			\emphdef{Allgemeine Diffusions-Advektions-Reaktions-Gleichung}:
			\[
				- \div \cdot (A \nabla u)0+ b \cdot \nabla u + c u = f,
			\]
			wobei $A \in \R^{d\times d}, b \in \R^d, c \in \R$.
			% fixme: bezeichnung : Diffusion, Advektion, Reaktions, Quellterm
	\end{itemize}
\end{ex}

\begin{ex}[Nichtlineare PDEs] \label{1.11}
	\begin{itemize}
		\item
			\emphdef{Nichtlineare Poission-Gleichung}: $-\Laplace u = f(u)$,
		\item
			\emphdef{$p$-Laplace-Gleichung}: $\div (|\nabla u|^{p-2} \nabla u) = 0$,
		\item
			\emphdef{Minimalflächen-Gleichung}: $\div ( \f{\nabla u}{\sqrt{1 + |\nabla u|^2}} ) = 0$,
		\item
			\emphdef{Hamilton-Jacobi-Gleichung}: $\partial_t u + H(\nabla u, u) = 0$,
		\item
			\emphdef{Burgers-Gleichung}: $\partial_t u + \partial_x(\f 12 u^2) = 0$,
		\item
			\emphdef{skalare Erhaltungsgleichung}: $\partial_t u + \nabla \cdot(f(u, \nabla u)) = 0$,
		\item
			\emphdef{Korteweg de Vries Gleichung (KdV)}: $\partial_t u + u \partial_x u + \partial_x^3 u = 0$,
		\item
			\emphdef{allgemeine Transport-Reaktions-Gleichung}: $\partial_t u + \div(f(u, \nabla u)) = g(u)$.
	\end{itemize}
\end{ex}

\begin{ex}[Lineare Systeme] \label{1.12}
	\begin{itemize}
		\item
			\emphdef{Maxwell-Gleichungen}:
			\begin{align*}
				\partial_t E &= \rot B, \\
				\partial_t B &= - \rot E, \\
				0 &= \div B = \div E.
			\end{align*}
		\item
			\emphdef{Oseen-Gleichungen}:
			\begin{align*}
				(b \cdot \nabla) u - \mu \Laplace u + \nabla p &= 0, \\
				\div u &= 0.
			\end{align*}
			Für $b = 0$ sind dies die \emphdef{Stokes-Gleichungen}.
		\item
			\emphdef[Poission-Gleichung!gemischte Formulierung]{gemischte Formulierung der Poission-Gleichung}:
			\begin{align*}
				\div v &= f, \\
				v + \nabla u &= 0
			\end{align*}
	\end{itemize}
\end{ex}

\begin{ex}[Nichtlineare Systeme] \label{1.13}
	\begin{itemize}
		\item
			System von Erhaltungsgleichungen
			\[
				\partial_t u + \div(F(u)) = 0
			\]
			für $F: \R^d \to \R^d$
		\item
			\emphdef{Navier-Stokes-Gleichungen}
			\begin{align*}
				\partial_t u + (u\cdot \nabla) u - \mu \Laplace u + \nabla p &= 0
				\div u &= 0
			\end{align*}
			Für $\mu = 0$ ergeben sich die \emphdef{Euler-Gleichungen} für ein nichtviskoses, inkompressibles Fluid.
	\end{itemize}
\end{ex}


\section{Klassifikation linearer PDEs zweiter Ordnung}


\begin{df}[linearer Differentialoperator zweiter Ordnung] \label{1.14}
	Sei $\Omega \subset \R^d$ offen, $A = (a_{ij})_{i,j = 1}^d \in C^0 (\Omega, \R^{d\times d}), b = (b_i)_{i=1}^d \in C^0(\Omega)^d, c \in C^0(\Omega)$.
	Dann nennen wir $\scr L: C^2(\Omega) \to C^0(\Omega)$ mit
	\[ \label{eq:1.2}
		(\scr L u)(x) := - \sum_{i,j=1}^d a_{ij}(x) \partial_{x_i} \partial_{x_j} u(x) + \sum_{i=1}^d b_i \partial_{x_i} u(x) + c(x) u(x)
	\]
	\emphdef[Differentialoperator!allgemein, linear, zweiter Ordnung]{allgemeiner linearer Differentialoperator zweiter Ordnung}.
	\begin{note}
		\begin{itemize}
			\item
				$\scr L$ erfasst die Differential-Operatoren in Laplace-, Poisson-, Helmholtz, Wärmeleitungs-, Diffusions  und allgeimener Diffusions-Advektions-Reaktionsgleichung aus \ref{1.10}.
			\item
				Zu $f \in C^0(\Omega)$ ergibt sich eine entsprechende PDE
				\[ \label{eq:1.3}
					\scr L u(x) = f(x), x \in \Omega.
				\]
			\item
				Wir nennen $-\sum_{i,j=1}^d a_{ij}(x) \partial_{x_i} \partial_{x_j} u$ \emphdef{Hauptteil} von $\scr L$.
			\item
				Ohne Einschränkung kann $A$ als symmetrisch vorausgesetzt werden, denn $\partial_{x_i} \partial_{x_j} u = \partial_{x_j} \partial_{x_i} u$.
				Falls $A$ nicht symmetrisch ist, so ergibt $A_s := \f 12 (A + A^T)$ identisches $\scr L$. \Exercise
				$A$ hat somit ohne Einschränkung nur reelle Eigenwerte.
			\item
				Auch geläufig ist die sogennante \emphdef{Divergenzform}, welche man bei differenzierbaren $a_{ij}, b_i$ leicht in obige Form bringen kann.
				Sei
				\begin{align*}
					(\_{\scr L} u)(x) &:= - \nabla \cdot (A(x) \cdot \nabla u) + \nabla \cdot (b \cdot u) + c u \\
					&= - \sum_{i,j=1}^d \partial_{x_i} (a_{ij} \partial_{x_j} u) + \sum_{i=1}^d \partial_{x_i} (b_i u) + cu \\
					&= -\sum_{i,j=1}^d a_{ij} \partial_{x_i} \partial_{x_j} u - \sum_{j=1}^d \sum_{i=1}^d (\partial_{x_i} a_{ij}) \partial_{x_j} + \sum_{i=1}^d b_i \partial_{x_i} u + \sum_{i=1}^d (\partial_{x_i} b_i) \cdot u + cu \\
					&= %fixme
				\end{align*}
				mit der Wahl
				\[
					\tilde b_i := b_i - \sum_{j=1}^d (\partial_{x_j} a_{ji})
					%fixme: \tilde c_i
				\]
		\end{itemize}
	\end{note}
\end{df}

\begin{df}[Klassifikation] \label{1.15}
	Der Operator $\scr L$ aus \eqref{eq:1.2} ist
	\begin{itemize}
		\item
			\emphdef{elliptisch} in $x$, falls alle Eigenwerte von $A(x)$ positiv,
		\item
			\emphdef{parabolisch} in $x$, falls $d-1$ Eigenwerte von $A(x)$ positiv, ein Eigenwert Null ist, aber $\rg([A(x),b(x)]) = d$.
		\item
			\emphdef{hyperbolisch} in $x$, falls $d-1$ Eigenwerte von $A(x)$ positiv und ein Eigenwert negativ ist.
	\end{itemize}
	$\scr L$ \emphdef{elliptisch}, \emphdef{parabolisch}, bzw. \emphdef{hyperbolisch}, wenn er es in jedem $x \in \Omega$ ist.
	Die PDE \eqref{eq:1.3} ist \emphdef{elliptisch}, \emphdef{parabolisch}, bzw. \emphdef{hyperbolisch}, wenn $\scr L$ dies ist.
	\begin{note}
		\begin{itemize}
			\item
				Die Begriffe sind motiviert aus Kegelschnitten/Quadriken:
				\[
					z^T A(x) z = 1
				\]
				beschreibt unter den genannten Voraussetzungen ein Ellipsoid/Paraboloid/Hyperboloid.
		\end{itemize}
	\end{note}
\end{df}



\chapter{Finite Differenzen Verfahren für elliptische Probleme} \label{chap:2}



\section{Finite Differenzen für Poisson-Gleichung}


\begin{df}[Finite Differenz] \label{2.1}
	Sei $h \in \R$, $e_j \in \R^d$ Einheitsvektor für $j = 1, \dotsc, d$ und $u: \Set{x, x \pm e_j h & j = 1, \dotsc, d} \to \R$, dann definieren wir \emphdef[Vorwärtsdifferenz]{Vorwärts-} oder \emphdef{rechtsseitige Differenz} durch
	\[
		(\partial_{x_j}^{+h}u)(x) := \frac{u(x+he_j) -u(x)}{h}
	\]
	analog \emphdef[Rückwärtsdifferenz]{Rückwärts-} oder \emphdef{linksseitige Differenz} durch
	\[
		(\partial_{x_j}^{-h}u)(x) := \frac{u(x) - u(x-he_j)}{h}
	\]
	und \emphdef[symmetrische Differenz]{symmetrische} oder \emphdef{zentrale Differenz} durch
	\[
		(\partial_{x_j}^{ch} u)(x)
		:= \f 12 (\partial_{x_j}^{+h} u(x) + \partial_{x_j}^{-h} u(x))
		= \frac{u(x+he_j) - u(x-he_j)}{2h}
	\]
\end{df}


\begin{st}[Approximationsgüte] \label{2.2}
	Sei $u: \Omega \to \R, x \in \Omega \subset \R^d, r \in \N^+$ mit $B_r(x) \subset \Omega$.
	Für $h < r$ gilt dann
	\begin{enumerate}[i)]
		\item
			$|\partial_{x_j} u(x) - \partial_{x_j}^{\pm h} u(x)| \le \f h2 \|\partial_{x_j}^2 u\|_\infty$ für $u \in C^2(\_\Omega)$
		\item
			$|\partial_{x_j} u(x) - \partial_{x_j}^{c, h} u(x)| \le \f {h^2}6 \|\partial_{x_j}^3 u\|_\infty$ für $u \in C^3(\_\Omega)$
		\item
			$|\partial_{x_j}^2 u(x) - \partial_{x_j}^{-h} \partial_{x_j}^{+h} u(x)| \le \f {h^2}{12} \|\partial_{x_j}^4 u\|_\infty$ für $u \in C^4(\_\Omega)$
	\end{enumerate}
	\begin{proof}
		Es genügt dies für $d = 1$ zu zeigen, denn  sei $u \in C^k(\_\Omega), v(t) = u(x + th e_j)$.
		Dann ist $\|\ddx[t^k] v(t)\|_\infty \le \|\partial_{x_j}^k u\|_{C^0(\_\Omega)}$.
		\begin{enumerate}[i)]
			\item
				Taylor liefert für ein $\xi \in (x, x + h)$
				\[
					u(x+h) = u(x) + hu'(x) + \f {h^2}2 u''(\xi)
				\]
				Es folgt
				\[
					\partial_x u(x) - \partial_x^{+h} u(x)
					= u'(x) - \f{u(x+h) - u(x)}{h}
					= - \f h2 u''(\xi).
				\]
				Analog für die linksseitige Differenz.
			\item
				Subtraktion von
				\begin{align*}
					u(x+h) &= u(x) + hu'(x) + \f {h^2}2 u''(x) + \f {h^3}6 u'''(\xi) \\
					u(x-h) &= u(x) - hu'(x) + \f {h^2}2 u''(x) - \f {h^3}6 u'''(\_\xi) \\
				\end{align*}
				mit $\xi \in (x,x+h), \_\xi \in (x-h,x)$ liefert
				\[
					u(x+h) - u(x-h) = 2h u'(x) + \f{h^3}6 \big(u'''(\xi) + u'''(\_\xi)\big),
				\]
				also
				\[
					u'(x) - \f{u(x+h) - u(x-h)}{2h}
					= - \f{h^2}{12} \big( u'''(\xi) + u'''(\_\xi) \big)
					\le \f{h^2}6 \|u'''\|_\infty.
				\]
			\item
				Addition von
				\begin{align*}
					u(x+h) &= u(x) + hu'(x) + \f{h^2}2 u''(x) + \f {h^3}6 u'''(x) + \f{h^4}{24} u''''(\xi) \\
					-2u(x) &= -2u(x) \\
					u(x+h) &= u(x) - hu'(x) + \f{h^2}2 u''(x) - \f {h^3}6 u'''(x) + \f{h^4}{24} u''''(\_\xi) \\
				\end{align*}
				mit $\xi \in (x,x+h), \_\xi \in (x-h, x)$ liefert
				\begin{align*}
					\partial_x^{-h} \partial_x^{+h} u(x)
					&= \dfrac{\frac{u(x+h)-u(x)}{h}-\frac{u(x)-u(x-h)}{h}}{h} \\
					&= \frac{u(x+h) - 2u(x) +  u(x+h)}{h^2} \\
					&= \f 1{h^2} \big( \f {h^2}2 + \f {h^2}2 \big) u''(x) + \f 1{h^2} \f {h^4}{24} \big( u''''(\xi) + u''''(\_\xi) \big)
				\end{align*}
		\end{enumerate}
	\end{proof}
	\begin{note}
		\begin{itemize}
			\item
				Die Approximation in iii) ist also eine zweite zentrale Differenz
				\[
					\partial_{x_j}^{-h} \partial_{x_j}^{+h} u(x)
					= \partial_{x_j}^{+h} \partial_{x_j}^{-h} u(x)
					= \partial_{x_j}^{c, \f h2} \partial_{x_j}^{c, \f h2} u(x)
					= \f{u(x+h) - 2u(x) + u(x-h)}{h^2}.
				\]
			\item
				Aus dem Beweis folgt, dass $\partial_x^{-h} \partial_{x}^{+h} u(x) = u''(x)$ falls $u^{(4)} = 0$, z.B. für $u \in \P_3$.
			\item
				Man kann zentrale Differenzen für höhere Ableitungen verallgemeinern:
				\[
					\partial_{x_j}^{h,m} u(x)
					:= (\partial_{x_{j}}^{c, \f h2})^m u(x)
				\]
				falls $u: \Set{u+(k-\f m2) h e_j & k = 0, \dotsc, m } \to \R$.
				Dann ist
				\[
					\partial_{x_j}^{h,m} u(x)
					= \f 1{h^m} \sum_{k=0}^m \binom{m}{k} (-1)^{k+m} u\big( x + (k-\f m2) h e_j \big).
				\]
		\end{itemize}
	\end{note}
\end{st}

\begin{kor}[FD-Approximation für Laplace] \label{2.3}
	Sei $u : \Set{x, x \pm h e_j} \to \R$.
	Dann definieren
	\begin{equation} \label{eq:2.1}
		\Laplace_h u(x) :=
		\Big(\sum_{i=1}^d \partial_{x_j}^{-h} \partial_{x_j}^{+h} u \Big) (x)
	\end{equation}
	und es gilt unter Voraussetzungen von \ref{2.2}
	\[
		|\Laplace_h u(x) - \Laplace u(x)| \le C h^2
	\]
	für $u \in C^4(\_\Omega)$.
	\begin{proof}
		Dreiecksungleichung und \ref{2.2} iii) liefert
		\begin{align*}
			|\Laplace u(x) - \Laplace_h u(x)|
			&= \Big| \sum_{i=1}^d \partial_{x_j}^2 u(x) - \partial_{x_j}^{-h} \partial_{x_j}^{+h} u(x) \Big| \\
			&\le \sum_{i=1}^d | \partial_{x_j}^2 u(x) - \partial_{x_j}^{-h} \partial_{x_j}^{+h} u(x) \Big| \\
			&\le \sum_{j=1}^d \f {h^2}{12} \|\partial_{x_j}^{(4)} u\|_\infty \\
			&\le d \f {h^2}{12} \|u\|_{C^4(\_\Omega)}.
		\end{align*}
	\end{proof}
	\begin{note}
		\begin{itemize}
			\item
				Für $p(x) := \prod_{i=1}^d p_i(x_i)$ mit $p_i \in \P_3$ ist $\Laplace_h$ exakt, d.h. $\Laplace_h p(x) = \Laplace p(x)$.
		\end{itemize}
	\end{note}
\end{kor}


\begin{df}[Würfelgebiet] \label{2.4}
	Sei $\Omega \subset \R^d$ offen, beschränkt.
	$\Omega$ heißt \emphdef{Würfelgebiet} zu $h \in \R^+$, falls $Z \subset \Z^d$ sodass $\Omega = W \setminus \Boundary W =: \mathring W$ mit $W := \bigcup_{z \in Z} W(z)$ und $W(z) := [z_1h , (z_1+1)h] \times \dotsc \times [z_d h, (z_d+1)h] \subset \R^d$.
	\begin{note}
		\begin{itemize}
			\item
				Ist $\Omega$ ein Würfelgebiet zu $h$, dann ist $\Omega$ auch ein Würfelgebiet zu $\f hn$ für alle $n \in \N$.
		\end{itemize}
	\end{note}
\end{df}

Wir wollen uns im Folgenden nur mit Würfelgebieten beschäftigen

\begin{df}[FD-Gitter] \label{2.5}
	Sei $\Omega \subset \R^d$ ein Würfelgebiet zu $h \in \R^+$, $\Gamma := \Boundary \Omega$, also $\_\Omega = \Omega \cup \Gamma$.
	Wir definieren das \emphdef{Gitter} $\_\Omega_h$ durch \emphdef{innere Punkte} $\Omega_h$ und \emphdef{Randpunkt} $\Gamma_h$, wobei
	\begin{align*}
		\Omega_h &:= \Set{ x \in \Omega & \exists z \in \Z^d : x = hz } \\
		\Gamma_h &:= \Set{ x \in \Gamma & \exists z \in \Z^d : x = hz } \\
		\_\Omega_h &:= \Omega_h \cup \Gamma_h.
	\end{align*}
	\begin{note}
		\begin{itemize}
			\item
				Jeder innere Punkt hat genau $2d$ Nachbarn im Abstand von $h$ in $\_\Omega_h$
			\item
				Erweiterung für allgemeine Gebiete später.
		\end{itemize}
	\end{note}
\end{df}

\begin{df}[Gitterfunktionen] \label{2.6}
	Zu einem Gitter $\_\Omega_h$ definieren wir den Raum der \emphdef{Gitterfunktionen} $X_h := \Set{ v : \_\Omega_h \to \R }$
	und den Teilraum der Funktionen mit Nullrandwerten $X_h^0 := \Set{ v \in X_h & \forall x \in \Gamma_h : v(x) = 0 } \subset X_h$
	und den Raum der Funktionen auf inneren Punkten $Y_h := \Set{v: \Omega_h \to \R}$ mit Maximumsnorm $\|v\|_{\_\Omega_h} := \max_{x\in \_\Omega_h} |v(x)|$ und $\|v\|_{\Omega_h} := \max_{x\in\Omega_h} |v(x)|$.
\end{df}

\begin{nt*}[Nebenbemerkungen]
	\begin{itemize}
		\item
			Also sind $(X_h, \|\argdot\|_{\_\Omega_h}), (X_h^0, \|\argdot\|_{\Omega_h}), (X_h^ , \|\argdot\|_{\_\Omega_h}), (Y_h, \|\argdot\|_{\Omega_h})$ Banachräume, weil endlichdimensional und damit vollständig.
		\item
			Man kann auch $X_h$ mit einer Hilbertraumstruktur versehen, indem man das \emphdef[diskretes $l_2$-Skalarprodukt]{diskrete $L_2$-Skalarprodukt} definiert:
			\[
				\<u,v\>_{l_2} := h^d \sum_{x\in\_\Omega_h} u(x) v(x),
			\]
			welches die Norm $\|u\|_{l_,} := \sqrt{h^d \sum_{x\in\_\Omega} u(x)^2}$ induziert.
			Dann ist $X_h$ auch vollständig bezüglich $\|\argdot\|_{l_2}$, weiter gilt: $\lim_{h\to 0} \|u\|_{l_2} = \|u\|_{L^2(\Omega)}$ für $u \in C^0(\Omega)$.
		\item
			Man kann $X_h$ auch mit einer Seminorm versehen, welche auch die Ableitungen miteinbezieht
			\[
				|u|_{h_1}
				:= \Big( h^d \sum_{x\in\Omega} \sum_{j=1}^d  \big(\partial_{x_j}^{+h} u(x)\big)^2 \Big)^{\f 12},
			\]
			die \emphdef{diskrete $h_1$ Seminorm}.
			Dies ist eine Norm auf $X_h^0$, aber nicht auf $X_h$.
			Damit erhält man durch Kombination mit der diskreten $l_2$-Norm eine Norm auf $X_h$ („diskrete $h_1$ Norm“):
			\[
				\|u\|_{h_1}
				:= \sqrt{\|u\|_{l_2}^2 + |u|_{h_1}^2},
			\]
			bezüglich welcher $X_h$ ein Hilbertraum ist.
	\end{itemize}
\end{nt*}

Für $v \in X$ ist mit \eqref{eq:2.1} der Operator $-\Laplace_h v(x)$ für $x \in \Omega_h$ wohldefiniert, d.h. wir können $\Laplace_h : X_h \to Y_h$ als linearen Operator sehen.

\begin{df}[FD-Approximation für Poisson-RWP] \label{2.7}
	Sei ein Gitter $\_\Omega_h$ gegeben.
	Dann nennen wir $u_h \in X_h$ \emphdef{Finite Differenzen Lösung} des Poisson-RWPs aus \ref{1.23}, falls
	\begin{align} \label{eq:2.2}
		\Laplace_h u_h(x) &= f(x) && \text{$x \in \Omega_h$} \\
		u_h(x) &= g(x) && \text{$x\in \Gamma_h$}. \notag
	\end{align}
\end{df}

\begin{nt*}[Berechnung via LGS]
	\begin{itemize}
		\item
			Lege eine Aufzählung $\Set{x_1, \dotsc, x_n} = \Omega_h$ fest.
			Dann ist \eqref{eq:2.2} äquivalent zu einem LGS für Unbekannte $\underbar{u}_h = (u_i)_{i=1}^n$ mit $u_i = u_h(x_i)$ für $i=1,\dotsc, n$, denn $u_h(x)$ für $x \in \Gamma$ ist schon festgelet durch $g$.
		\item
			Sei FD-Operator in $x_i \in \Omega_h$ gegeben durch
			\[
				\Laplace_h u(x_i) = \sum_{j=1}^n \alpha_{ij} u(x_j) + \sum_{x\in \Gamma_h} \beta_{ix} u(x).
			\]
			Dann ist das LGS gegeben durch $A_h \underbar{u}_h = b_h$ mit $(A_h)_{ij} = \alpha_{ij}$ und $(b_h)_i = f(x_i) - \sum_{x\in \Gamma_h} \beta_{ix} g(x)$.
		\item
			$A_h$ ist dünn besetzt (sparse), da sie nur sehr wenige nichtnull-Einträge pro Zeile enthält.
	\end{itemize}
\end{nt*}

\Timestamp{2014-10-31}

\begin{ex} \label{2.8}
	Sei $d=1, \Omega = (0,1), n \in \N, h := \f 1{n+1}, x_i := i h, i \in \Set{0, \dotsc, n+1}$.
	Dann ist $\Omega_h := \Set{x_1, \dotsc, x_n}, \Gamma_h := \Set{x_0, x_{n+1}}$.
	Betrachte die Poisson-Gleichung
	\begin{align*}
		-u''(x) &= f(x) && \text{in $\Omega$} \\
		u(0) &= \alpha \\
		u(1) &= \beta
	\end{align*}
	Sei $u_i \approx u(x_i)$ für $i=1, \dotsc, n$, $u_0 = \alpha, u_{n+1} = \beta$.
	Diskretisierung ergibt
	\[
		- \f{u_{i+1} - 2u_i + 2u_{i-1}}{h^2} = f(x_i)
	\]
	für $i = 1, \dotsc, n$.
	Das LGS ergibt sich als
	\[
		\overbrace{\f 1{h^2} \underbrace{\Matrix{2 & -1 & & \\ -1 & \ddots & \ddots & \\ & \ddots & \ddots & -1 \\ & & -1 & 2}}_{\tilde A_n}}^{A_h}
		\Vector{u_1 & \dots & u_n}
		= \Vector{f(x_1) + \f{\alpha}{h^2} & f(x_2) & \dots & f(x_{n-1}) & f(x_n) + \f{\beta}{h^2}}.
	\]
	\begin{note}
		\begin{itemize}
			\item
				$A_h$ ist tridiagonal, symmetrisch, $A_h = \f 1{h^2} \tilde A_n$
			\item
				$A_h$ ist regulär, denn per Induktion folgt $\det \tilde A_{n} = n + 1$:
				\begin{proof}
					Für $n = 1$ ist $\det(\tilde A_n) = \det(2) = n+1$.
					Die Aussage gelte für $n-1$, es gilt
					\begin{align*}
						\det \tilde A_n
						&= \det \Matrix{\tilde A_{n-1} & & \\ & & -1 \\ & -1 & 2} \\
						&= 2 \det \tilde A_{n-1} - (-1) \det \Matrix{\tilde A_{n-2} & & \\ & & & \\ & -1 & -1} \\
						&= 2 n + (-1) \det A_{n-2}
						= 2n - (n-1)
						= n+1
					\end{align*}
				\end{proof}
			\item
				$A_h$ ist positiv definit, denn Gerschgorin liefert $\lambda_i(\tilde A_n) \in [0,4]$ und wegen Regularität $\lambda_i(\tilde A_n) \neq 0$.
				Es folgt also $\lambda_i > 0$ für $i = 1, \dotsc, n$.
			\item
				Also existiert eine eindeutige FD-Lösung für das Poisson-RWP in einer Dimension.
			\item
				Wegen $A_h$ symmetrisch, positiv definit kann das CG oder das PCG Verfahren zum iterativen Lösen des LGS verwendet werden.
			\item
				Man kann hier auch direkt stetige Abhängigkeit von den Daten beweisen:
				Seien $u, u_h$ Lösungen zu $f, \alpha, \beta$, bzw. $\_f, \_\alpha, \_\beta$.
				Dann existiert $C > 0$ unabhängig von $f, \_f, \alpha, \_\alpha, \beta, \_\beta$ sodass
				\[
					\|u_h - \_u_h\| \le C \Big( \|f-\_f\| + |\alpha - \_\alpha| + |\beta - \_\beta| \Big).
				\]
		\end{itemize}
	\end{note}
\end{ex}

\begin{ex} \label{2.9}
	Sei $d = 2, \Omega = (0,1)^2$ und betrachte das Poisson-RWP mit $g(x) = 0$, $m \in \N, h:= \f 1{m+1}, n := m^2$.
	Statt Einzelindex ist ein Doppelindex übersichtlicher, setze dazu
	\begin{align*}
		x_{ij} &= (ih, jh),&
		u_{ij} &\approx u(x_{ij}) \quad \text{für $0 \le i,j \le m+1$}.
	\end{align*}
	Die Diskretisierung des RWP liefert für jeden inneren Punkt eine Gleichung
	\[
		\f 1{h^2} \Big( 4 u_{ij} - u_{i-1,j} - u_{i+1,j} - u_{i,j-1} - u_{i,j+1}\Big)
		= f(ih, jh).
	\]
	Am Rand gilt
	\[
		u_{0,j} = u_{m+1,j} = u_{j,0} = u_{j,m+1}
	\]
	für $0 < i,j < m$.
	Für jede beliebige Wahl einer Aufzählung der $\Set{u_{ij}}$ erhält man ein System $A_h u_h = b_h$ mit $A_h$ symmetrisch, $\f 4{h^2}$ auf der Diagonalen und $-\f 1{h^2}$ an bis zu $4$ Einträgen pro Zeile/Spalte.

	Falls eine lexikographische Aufzahlung gewählt wird:
	\[
		\_u_h = \Big(u_{11}, u_{12}, \dotsc, u_{1m}, u_{21}, \dotsc, u_{2m}, \dotsc, u_{mm} \Big)
	\]
	so erhält $A_h$ eine Bandstruktur, jedoch nicht mehr tridiagonal wie in \ref{2.8}, sondern „Block-tridiagonal“:
	\begin{align*}
		A_h &= \f 1{h^2} \Matrix{B & C & & \\C & \ddots & \ddots & \\ & \ddots & \ddots & C \\ & & C & B}, &
		B &= \Matrix{4 & -1 & & \\ -1 & \ddots & \ddots & \\ & \ddots & \ddots & -1 \\ & & -1 & 4}, &
		C &= \Matrix{-1 & & \\ & \ddots & \\ & & -1},
	\end{align*}
	\begin{note}[Beliebige Gebiete]
		\begin{itemize}
			\item
				Falls $\Omega$ beschränkt ist, aber kein Würfelgebiet, muss das Gitter modifiziert werden, indem Schnittpunkte von $\Gamma$ mit den $h \Z^d$ Würfelkanten hinzugenommen werden:
				\[
					\Gamma_h := \Set{ x \in \Gamma & \exists z\in \Z^d, j =1,\dotsc, d : x \in z + \R e_j }
				\]
				Dann wird wie gehabt $\_\Omega_h := \Omega_h \cup \Gamma_h$ mit $\Omega_h$ aus \ref{2.5} definiert.
			\item
				% Rand in Süd/West richtung
				Koeffizienten der FD-Diskretisierung werden angepasst.
				Die Taylor-Entwicklung liefert
				\begin{align*}
					u''(x) &= \f{2}{h_W(h_O + h_W)} u(x-h_W) - \f 2{h_Oh_W} u(x) + \f 2{h_O(h_O+h_W)} u(x+h_O) \\
					&\quad + \LandauO(h) \\
					\Laplace(u) &= \f{2}{h_W(h_O+h_W)} u(x-e_1h_W) - \f 2{h_Oh_N} u(x) + \f 2{h_O(h_O+h_W)} u(x+h_O e_1) \\
					&\quad + \f 2{h_S(h_N+h_S)} u(x-e_2 h_S) - \f 2{h_S h_N} u(x) + \f 2{h_N(h_S+h_N)} u(x+ e_2 h_N) \\
					&\quad + \LandauO(h)
				\end{align*}
				„Shortley-Weller Approximation“.
		\end{itemize}
	\end{note}
	\begin{note}[Andere Randbedingungen]
		Neben Dirichlet- können auch andere Randbedingungen realisiert werden, z.B. Neumann-Randbedingungen:
		\begin{align*}
			(\Nabla u) \cdot n &= g_N
			&& \text{auf $\Gamma_N \subset \Gamma$}.
		\end{align*}
		Wir nennen $\Gamma_N$ \emphdef{Neumann-Randteil}.
		Wir nehmen an, dass $x$ auf einer Kante liegt und nicht auf einer Ecke von $\Gamma$ (sonst ist keine Normale $n$ definiert).
		Sei $n = \pm e_j$ (da Würfelgebiet) äußerer Normalenvektor für ein $j = 1, \dotsc, d$.
		Wir Approximieren $\Nabla u$ durch
		\[
			(\Nabla_h u)_i :=
			\begin{cases}
				\partial_{x_i}^{c,h} u & i \neq j \\
				\partial_{x_j}^{-h} u & i = j
			\end{cases}.
		\]
		Nun sind $u_h(x), x \in \Gamma_h \cap \Gamma_N$ Unbekannte.
		Es fällt eine weitere Gleichung für das LGS an:
		\[
			(\Nabla_h u_h(x)) \cdot n = g_N(x)
		\]
		für $x \in \Gamma_h \cap \Gamma_N$.
	\end{note}
\end{ex}


\section{Allgemeine Elliptische PDEs zweiter Ordnung}

\begin{df}[Allgemeine Elliptische RWP] \label{2.10}
	Zu $\Omega \subset \R^d$ beschränkt, $f \in C^0(\Omega), g \in C^0(\Boundary \Omega)$ sei
	\begin{equation} \label{eq:2.3}
		(\scr L u)(x)
		= - \sum_{i,j=1}^d a_{ij}(x) \partial_{x_j} \partial_{x_i} u(x)
		+ \sum_{i=1}^d b_i(x) \partial_{x_i} u(x) + c(x) u(x)
	\end{equation}
	gleichmäßig elliptisch und $a_{ij}, b_i, c \in C^0(\_\Omega)$.
	Gesucht ist $u \in C^2(\Omega) \cap C^0(\_\Omega)$.
	\begin{align*}
		\scr L u(x) &= f(x) && \text{$x \in \Omega$}, \\
		u(x) &= g(x) && \text{$x\in\Gamma$}.
	\end{align*}
\end{df}

\begin{df}[FD-Approximation] \label{2.11}
	Für $\scr L$ aus \eqref{eq:2.3}, $x \in \Omega$ mit $\_B_h(x)  \subset \_\Omega$ definiere
	\begin{equation} \setcounter{equation}{5} \label{eq:2.5}
		\begin{aligned}
			\scr L_h u(x)
			&:= - \sum_{i=1}^d a_{ii} (x) \partial_{x_i}^{-h} \partial_{x_i}^{+h} u(x)
			- \sum_{\substack{i,j=1 \\ i\neq j}}^d a_{ij} (x) \partial_{x_i}^{c,h} \partial_{x_j}^{c,h} u(x) \\
			&\qquad + \sum_{i=1}^d b_i(x) \partial_{x_i}^{c,h} u(x)	+ c(x) u(x).
		\end{aligned}
	\end{equation}
	\begin{note}
		Wir werden sehen, dass
		\begin{itemize}
			\item
				\ref{2.11} nur unter weiteren Annahmen an $a_{ij}, b_i, c, h$ eine „stabile Diskretisierung“ ergibt,
			\item
				Eiene etwas sorgfältigere Diskretisierung des Hauptteils eine erweiterte Klasse von Problemen stabil diskretisiert.
		\end{itemize}
	\end{note}
\end{df}

\begin{st}[FD-Approximationsfehler für $\scr L_h$] \label{2.12}
	Sei $u \in C^4(\_\Omega), x \in \Omega$ sodass $x + \sum_{i=1}^d \sigma_i e_i h \in \_\Omega$ für $\sigma_i = \Set{0,+1, -1}$, $i = 1, \dotsc, d$.
	Dann existiert ein $C$ (unabhängig von $x,h$) sodass
	\[
		\big|\scr L u(x) - \scr L_h u(x)\big| \le C h^2.
	\]
	\begin{proof}
		Taylor analog zu \ref{2.2} und \ref{2.3}.
	\end{proof}
\end{st}

\begin{note}[FD-Stern]
	\begin{itemize}
		\item
			Die Diskretisierung $\scr L_h$ kann man anschaulicher notieren: z.B. für $d = 2$.
			Falls $\scr L_h u(x_1, x_2) = \f 1{h^2} \sum_{i,j=-m}^m \alpha_{ij} u(x_1 + ih, x_2 + jh)$ dann ist
			\begin{equation} \setcounter{equation}{4} \label{eq:2.4}
				\Matrix[{
					\alpha_{-m,m} & \hdots & \alpha_{0,m} & \hdots & \alpha_{m,m} \\
					\vdots && \vdots && \vdots \\
					\alpha_{-m, 0} & \hdots & \alpha_{0,0} & \hdots & \alpha_{m,0} \\
					\vdots && \vdots && \vdots \\
					\alpha_{-m,-m} & \hdots & \alpha_{0,-m} & \hdots & \alpha_{m,-m}
				}_*
			\end{equation}
			Für $\scr L_h = -\Laplace_h$ aus \ref{2.3} ergibt sich für $m = 1$ der \emphdef{5-Punkt-FD-Stern}
			\[
				\Matrix[{ 0 & -1 & 0 \\ -1 & 4 & -1 \\ 0 & -1 & 0 }_*
			\]
			Für $\scr L_h$ aus \ref{2.11} (ohne Einschräkung $a_{12} = a_{21}$):
			\[
				\f 12 \Matrix[{ a_{12}(x) & - 2 a_{22}(x) & - a_{12}(x) \\ -2 a_{11}(x) & 4(a_{11}(x) + a_{22}(x)) & -2a_{11}(x) \\ -a_{12}(x) & -2 a_{22}(x) & a_{12}(x)}_*
				+ \f h2 \Matrix[{ 0 & b_2(x) & 0 \\ -b_1(x) & 0 & b_1(x) \\ 0 & -b_2(x) & 0 }_*
				+ h^2 \Matrix[{0 & 0 & 0 \\ 0 & c(x) & 0 \\ 0 & 0 & 0}_*.
			\]
		\item
			Für $m = 1$ (also $3\times 3$ Sterne) ist höchstens Approximationsordnung 2 erreichbar, wie in \ref{2.12} und \ref{2.3} für spezielle $\scr L_h$ gesehen
		\item
			Für $m > 1$ sind bessere Approximationsordnungen erreichbar
	\end{itemize}
\end{note}

\begin{df}[FD-Approximation für elliptisches RWP] \label{2.13}
	Sei $\Omega$ Würfelgebiet zu $h \in \R^+$, $\_\Omega_h$ zugehöriges Gitter.
	Dann ist $u_h \in X_h$ FD-Lösung falls
	\begin{align*}
		\scr L_h u_h(x) &= f(x) && \text{$x \in \Omega_h$}, \\
		u_h(x) &= g(x) && \text{$x\in \Gamma_h$}.
	\end{align*}
\end{df}

\begin{df}[Diskreter Zusammenhang] \label{2.14}
	Ein Gitter $\_\Omega_h$ heißt \emphdef{diskret zusammenhängend} falls es für alle $x,y \in \Omega_h$ (innere Punkte) eine Punktfolge $z_i \in \Omega_h$, $i = 0, \dotsc, k$ gibt mit $z_0 = x, z_k = y$ und $|z_i - z_{i-1}| = h$ für $i = 1, \dotsc, k$.
	\begin{note}
		Wenn zu grob gesampled wird, kann es vorkommen, dass $\Omega$ zwar zusammenhängend ist, $\Omega_h$ jedoch nicht.
		Falls $\Omega$ zusammenhängend, ist jedoch für hinreichend kleines $h$ auch $\_\Omega_h$ diskret zusammenhängend.
	\end{note}
\end{df}

\begin{lem}[Sternlemma] \label{2.15}
	Sei $k \ge 1$, $\Set{\alpha_i}_{i=0}^k \subset \R$ und $\Set{p_i}_{i=0}^n \subset \R$ gegeben und es gelte
	\begin{enumerate}[i)]
		\item
			$\alpha_0 > 0$ und $\alpha_i < 0$ für $i = 1, \dotsc, k$2
		\item
			$\sum_{i=0}^k \alpha_i = 0$,
		\item
			$\sum_{i=0}^k \alpha_i p_i \le 0$.
	\end{enumerate}
	Dann folgt aus $p_0 \ge \max_{i=1,\dotsc, k} p$ die Gleichheit $p_0 = p_1 = \dotsb = p_k$.
	\begin{proof}
		Es gilt
		\[
			\sum_{i=1}^k \alpha_i(p_i-p_0)
			= \sum_{i=0}^k \alpha_i (p_i-p_0)
			= \underbrace{\sum_{i=0}^k \alpha_i p_i}_{\le 0} - \underbrace{p_0 \sum_{i=0}^k \alpha_i}_{=0}
			\le 0.
		\]
		Wegen $p_0 \ge p_i, \alpha_i < 0$ für $i = 1, \dotsc, k$ sind die Summanden links nicht-negativ, also null und es folgt $p_i = p_0$.
	\end{proof}
\end{lem}

\begin{st}[Diskretes Maximumsprinzip] \label{2.16}
	Sei $u_h \in X_h$ eine FD-Lösung zu dem RWP \ref{2.13} mit $f(x) \le 0$ für $x \in \Omega_h$.
	Der Differenzenstern \eqref{eq:2.4} zu $m = 1$ (also ein $3\times 3$ Stern) erfülle in allen Punkten
	\begin{enumerate}[i)]
		\item
			$\sum_{i,j=-1}^1 \alpha_{i,j} = 0$,
		\item
			$\alpha_{0,0} > 0$,
		\item
			$\forall (i,j) \neq (0,0) : \alpha_{i,j} \le 0$,
		\item
			$\alpha_{1,0}, \alpha_{0,1}, \alpha_{0,-1}, \alpha_{-1,0} < 0$, „Koeffizienten in Hauprichtungen negativ“.
	\end{enumerate}
	Dann gilt
	\[
		\max_{x\in\_\Omega_h} u_h(x) = \max_{x\in \Gamma_h} u_h(x).
	\]
\Timestamp{2014-11-04}
	\begin{proof}
		Sei $x \in \_\Omega_h$ mit $u_h(x) = \max_{\_x\in \_\Omega_h} u_h (\_x)$.
		Falls $x \in \Gamma_h$, so sind wir fertig.
		Falls $x \in \Omega_h$: setze $p_0 := u_h(x)$, $(p_i)_{i=1}^k$ als „Nachbarwerte“ von $u_h(x)$, $\alpha_0 := \alpha_{0,0}$, $(\alpha_i)_{i=1}^k$ als Nicht-Null-Koeffizienten des FD-Sterns.
		Es gilt also $\f 1{h^d} \sum_{k=0}^k \alpha_i p_i = \scr L_h u(x) = f(x) \le 0$.
		Das Sternlemma \ref{2.15} hier angewandt ergibt $p_0 = p_1 = \dotsc = p_k$, also $u_h$ konstant in $x$ und seinen Nachbarn, welche im Differenzenstern auftreeten.
		Wiederholung dieses Argumentes in alle $2d$ Hauptrichtungen führt zum Rand (Beschränktheit von $\Omega$)
	\end{proof}
\end{st}

Falls $\Omega_h$ diskret zusammenhängend ist, führt das Argument aus dem Beweis von \ref{2.16} zu allen Punkten in $\Omega_h$, also folgt

\begin{kor}[Lösung $u_h$ konstant] \label{2.17}
	Falls $\_\Omega_h$ diskret zusammenhängend und die FD-Lösung für $f \le 0$ nimmt ihr Maximum im Inneren an, so ist $u_h$ konstant.
\end{kor}

\begin{note}
	\begin{itemize}
		\item
			Obiges diskretes Maximum-Prinzip gilt für $\scr L_h = - \Laplace_h$, weil Voraussetzungen erfüllt sind.
		\item
			Für $\scr L_h$ aus \eqref{eq:2.5} mit $a_{1,2} = 0$, $c = 0$, $b \neq 0$ sind die Voraussetzungen von \ref{2.16} erfüllt, falls $h$ hinreichend klein gewählt wird.
			Aus gleichmäßiger Elliptizität folgt für $z = e_i$
			\[
				0 < \tilde \alpha \underbrace{\|z\|^2}_{= 1} \le z^T A(x) z = a_{ii}(x),
			\]
			also $\alpha_{0,0} = 2(a_{11} + a_{22}) > 0$.
			Falls $|b_i| \le B$ und $h < \f 2B \tilde \alpha$, dann gilt für $\alpha_{1,0}$
			\[
				\alpha_{1,0}
				= -a_{11} + \f {h2} b_1
				< -\tilde \alpha + \f 2B \tilde \alpha \f 12 B
				= 0.
			\]
			Analog für $\alpha_{-1,0} = \alpha_{0,-1} = \alpha_{0,1} = 0$.
		\item
			Falls $c(x) > 0$ in \eqref{eq:2.5}, so ist $\sum_{i,j = -1}^1 \alpha_{ij} > 0$.
			Für diesen Fall lassen sich Abschwächungen des Sternlemmas und des Maximumsprinzips formulieren und beweisen.
		\item
			Falls $a_{1,2} \neq 0$, so ist eine Modifikation der Diskretisierung notwendig.
	\end{itemize}
\end{note}

\begin{kor}[Diskretes Vergleichsprinzip] \label{2.18}
	Seien $u_h, v_h \in X_h$ und es gelte $\scr L_h u_h \le \scr L_h v_h$ in $\Omega_h$ und $u_h \le v_h$ auf $\Gamma_h$ und es gelte das diskrete Maximumsprinzip.
	Dann gilt $u_h \le v_h$ in $\_\Omega_h$.
	\begin{proof}
		Für $w_h := u_h - v_h$ gilt $\scr L_h w_h = \scr L_h u_h - \scr L_h v_h \le 0$ in $\Omega_h$ und $w_h \le 0$ auf $\Gamma_h$.
		Mit dem diskreten Maximumsprinzip wird das Maximum auf dem Rand $\Gamma_h$ angenommen und daher $\max_{x\in \_\Omega_h} w_h(x) \le \max_{x\in\Gamma_h}w_h(x) \le 0$, also $u_h \le v_h$ auf $\_\Omega_h$.
	\end{proof}
\end{kor}

\begin{kor}[Existenz und Eindeutigkeit] \label{2.19}
	Sei ein diskretes RWP gemäß \ref{2.13} gegeben und es gelte das Maximumsprinzip.
	Dann existiert eine eindeutige FD-Lösung $u_h \in X_h$.
	\begin{proof}
		\begin{seg}{Eindeutigkeit}
			Seien $u_h, \_u_h \in X_h$ zwei Lösungen, setze $v := u_h - \_u_h$.
			Es gilt
			\begin{align*}
				\scr L_h v &= \scr L_h u_h - \scr L_h \_u_h = f - f = 0 && \text{in $\Omega_h$} \\
				v &= u_h - \_u_h = g - g = 0 && \text{auf $\Gamma_h$}.
			\end{align*}
			Das Maximumsprinzip liefert $v(x) \le \sup_{x\in\Gamma_h} v(x) = 0$ für alle $x \in \_\Omega_h$.
			Analoge Argumentation für $-v$ liefert $-v(x) \le 0$ in $\_\Omega_h$, folglich $v(x) = 0$ in ganz $\_\Omega_h$
		\end{seg}
		\begin{seg}{Existenz}
			Die FD-Diskretisierung führt auf ein $n \times n$ System $A_h \underbar{u}_h = b_h$ mit $\ker A_h = \Set 0$ wegen der Eindeutigkeit.
			Folglich hat $A_h$ vollen Rang und ist regulär, also $\underbar{u}_h := A_k^{-1} b_k$ ist eindeutiger DOF-Vektor von $u_h \in X_h$.
		\end{seg}
	\end{proof}
\end{kor}

\begin{kor}[Stetige Abhängigkeit von Randdaten] \label{2.20}
	Seien $u_h, \_u_h \in X_h$ FD-Lösungen zum RWP \ref{2.13} mit identischem $f$ aber unterschiedlichen Randdaten $g, \_g$ und es gelte das diskrete Maximumsprinzip.
	Dann gilt
	\[
		\|u_h - \_u_h\|_{\Omega_h} = \|g-\_g\|_{\Gamma_h} := \sup_{x\in\Gamma_h} |g(x) - \_g(x)|.
	\]
	\begin{proof}
		Setze $v :=: u_h - \_u_h$, dann ist $\scr L_h v = 0$ in $\Omega_h$ mit diskretem Maximumsprinzip, also
		\[
			v(x) \le \max_{\_x \in \Gamma_h} v(x)
			\le \max_{x \in \Gamma_h} |v(x)|
			= \|g - \_g\|_{\Gamma_h},
		\]
		analog für $-v(x)$, es folgt die Behauptung.
	\end{proof}
	\begin{note}
		\begin{itemize}
			\item
				Anschauliche Bedeutung: Leichte Änderung in Daten ergibt nur leichte Änderung in Lösung.
			\item
				Ähnlich: Stetige Abhängige bezüglich der rechten Seite $f$.
		\end{itemize}
	\end{note}
\end{kor}

\begin{df}[Stabilität, Konsistenz, Konvergenz] \label{2.21}
	Sei $u \in X := C^2(\Omega) \cap C^0(\_\Omega)$ Lösung des ellptischen RWP \ref{2.10} und $u_h \in X_h$ FD-Approximation aus \ref{2.13}.
	Die FD-Diskretisierung ist
	\begin{enumerate}[i)]
		\item
			\emphdef{konsistenz mit Ordnung $p$}, wenn für ein $C_c = C_c(x)$ unabhängig von $h$ gilt
			\[
				\|\scr L_h u - \scr L u\|_{\Omega_h} \le C_c h^p.
			\]
		\item
			\emphdef{stabil} (genauer: $(X_h^0, Y_h)$-stabil), wenn $C_s$ unabhängig von $h$ existiert, sodass
			\[
				\|v_h\|_{\_\Omega_h} \le C_s \|\scr L_h v_h\|_{\Omega_h}
			\]
			für alle $v_h \in X_h^0$.
		\item
			\emphdef{konvergent mit Ordnung $p$}, wenn für ein $C = C(u)$ unabhängig von $h$ gilt:
			\[
				\|u - u_h\|_{\_\Omega_h} \le C h^p.
			\]
	\end{enumerate}
	\begin{note}[Konsistenz der FD-Approximation]
		\begin{itemize}
			\item
				Punktweise Fehlerschranke aus \ref{2.3}, oder \ref{2.12} besagten $|\scr L_h u(x) - \scr L u(x)| \le C h^2$ mit $C$ unabhängig von $x$.
				Also liegt eine \emphdef{uniforme Schranke in $x$} vor, falls $u \in C^4(\_\Omega)$.
				Mit $C_c := C$ gilt Konsistenz der FD-Approximation mit Ordnung 2 für $\Omega$ Würfelgebiet und Ordnung $1$ für Nicht-Würfelgebiete, z.B. durch die Shortley-Weller-Approximation.
		\end{itemize}
	\end{note}
	\begin{note}[Stabilität]
		\begin{itemize}
			\item
				Stabilität bedeutet anschaulisch, dass die Lösung der PDE durch die rechte Seite beschränkt bleibt, unabhängig von $h$.
				Sei $w_h: \Omega_h \to \R$, also $w_h \in Y_h$ und $v_h \in X_h^0$ Lösung von $\scr L_h v_h = w_h$ in $\Omega_h$, $v_h = 0$ auf $\Gamma_h$.
				Dann ist also
				\[
					\|v_h\|_{\_\Omega_h} \le C_s \|\scr L_h v_h\| = C_s \|w_h\|_{\Omega_h}.
				\]
		\end{itemize}
	\end{note}
\end{df}

\begin{st}[Hinreichende Bedingung für Stabilität] \label{2.22}
	Sei $A_h \in \R^{n\times n}$ die FD-System-Matrix.
	Falls $C_s$ unabhängig von $h$ existiert, sodass $\|A_h^{-1}\|_\infty \le C_s$.
	Dann ist das FD-Verfahren stabil.
	\begin{proof}
		Seien $\underbar{v}_h, \underbar{w}_h \in \R^n$ Vektor der inneren Knotenwerten für $v_h \in X_h^0$, $w_h \in Y_h$, also $A_h \underbar{v}_h = \underbar{w}_h$.
		Dann gilt ($v_h$ hat Nullrandwerte)
		\[
			\|v_h\|_{\_\Omega_h}
			= \|\underbar{v}_h\|_\infty
			= \|A_h^{-1} \underbar{w}_h\|_\infty
			\le \|A_h^{-1}\|_\infty \|\underbar{w}_h\|_\infty
			\le C_s \|w_h\|_{\Omega_h}
			= C_s \|\scr L_h v_h\|
		\]
	\end{proof}
\end{st}

\begin{st}[Stabiltität für Poisson-RWP, FD-Diskretisierung] \label{2.23}
	Sei $\Omega \subset \R^d$ beschränktes Gebiet mit $\Omega \subset B_R(0)$ für ein $R > 0$.
	Dann gilt für alle $v_h \in X_h^0$, dass
	\[
		\|v_h\|_{\_\Omega_h} \le \f {R^2}{2d} \|\Laplace_h v_h\|_{\Omega_h}
	\]
	also das FD-Verfahren stabil mit $C_s := \f {R^2}{2d}$.
	\begin{proof}
		Sei $v_h \in X_h^0$ und $w_h$ aus \ref{2.24}.
		Dann ist für $x \in \Omega_h$
		\[
			- \frac{\Laplace_h v_h(x)}{\|\Laplace_h v_h\|_{\Omega_h}}
			\le \frac{\Laplace_h v_h(x)}{\|\Laplace_h v_h\|_{\Omega_h}}
			\le 1
			= - \Laplace_h w_h(x)
		\]
		Für $x \in \Gamma_h$ gilt $- \frac{v_h(x)}{\|\Laplace_h v_h\|} = 0 = w_h(x)$.
		Also mit diskretem Vergleichsprinzip \ref{2.18} für alle $x \in \Omega_h$
		\[
			\frac{v_h(x)}{\|\Laplace_h v_h\|_{\Omega_h}}
			\le w_h(x)
			\stack{\ref{2.14}}{\le} \f 1{2d} (R^2 - \|x\|^2)
			\le \f {R^2}{2d}.
		\]
		Eine analoge Argumentation für $-v_h$ liefert
		\[
			- \frac{v_h(x)}{\|\Laplace_h v_h\|_{\Omega_h}}
			\le \frac{R^2}{2d},
		\]
		also $\|v_h\|_{\_\Omega_h} \le \f {R^2}{2d} \|\Laplace_h v_h\|_{\Omega_h}$.
	\end{proof}
\end{st}

\begin{lem} \label{2.24}
	Sei $w_h \in X_h$ Lösung von $-\Laplace_h w_h = 1$ in $\Omega_h$, $w_h = 0$ auf $\Gamma_h$.
	Dann gilt
	\begin{equation} \setcounter{equation}{6} \label{eq:2.6}
		0 \le w_h(x) \le \f 1{2d} (R^2 - \|x\|_2^2)
	\end{equation}
	für $x \in \_\Omega_h$.
	\begin{proof}
		Sei $w(x) := \f 1{2d} (R^2 - \|x\|_2^2)$.
		Dann ist $w$ Polynom zweiten Grades, also $\Laplace_h$ exakt für $w$ gemäß Bemerkung nach \ref{2.3}.
		Für $x \in \Omega_h$ gilt somit
		\begin{align*}
			-\Laplace_h w(x)
			= - \Laplace w(x)
			&= - \sum_{i=1}^d \partial_{x_i}^2 \Big(R^2 - \sum_{j=1}^d x_j^2 \Big) \f 1{2d} \\
			&= - \sum_{i=1}^d (-2) \f 1{2d}
			= 1
			= - \Laplace_h w_h(x).
		\end{align*}
		Weiter ist $w \ge 0 = w_h$ auf $\Gamma_h$ nach Wahl von $R$.
		Aus dem diskreten Vergleichsprinzip \ref{2.18} folgt $w \ge w_h$ auf $\_\Omega_h$, also die zweite Gleichung in der Behauptung.
		Die erste Ungleichung folgt aus dem diskreten Maximumsprinzip für $-w_h$:
		\[
			- \Laplace_h (-w_h) =  -1 \le 0
		\]
		für $x \in \Omega_h$, also $\max_{x\in\_\Omega_h} (-w_h(x)) \le \max_{x\in \Gamma_h} (-w_h(x)) = 0$ und damit $w_h \ge 0$ auf $\Omega_h$.
	\end{proof}
\end{lem}

\Timestamp{2014-11-07}

\begin{st}[Konvergenz] \label{2.25}
	Sei ein FD-Verfahren für ein elliptisches RWP gemäß \ref{2.10} gegeben.
	Falls das Verfahren staabil und konsistenz mit Ordnung $p$ ist, so auch konvergent mit Ordnung $p$.
	\begin{proof}
		Seien $u \in X$ die exakte und $u_h \in X_h$ die FD-Lösung.
		Dann hat $u - u_h$ Nullrandwerte auf $\Gamma_h$, also folgt mit Stabilität:
		\[
			\|u-u_h\|_{\_\Omega_h}
			\le C_S \| \scr L_h (u-u_h) \|_{\Omega_h}
			= C_s \|\scr L_h u - \scr L_h u_h\|_{\Omega_h}
		\]
		Wegen $(\scr L_h u_h)(x) = f(x) = (\scr L u)(x)$ für alle $x \in \Omega_h$ folgt mit Konsistenz:
		\[
			\|u - u_h\|_{\_\Omega_h}
			\le C_s \|\scr L_h u - \scr L u\|_{\Omega_h}
			\le \underbrace{C_s C_c}_{=:C} h^p.
		\]
	\end{proof}
\end{st}

\ref{2.25} ist auf die FD-Diskretisierung des Poisson-RWPs anwendbar, denn wir haben Konsistenz (siehe Bemerkung nach \ref{2.21}) und Stabilität in \ref{2.23} nachgewiesen.

\begin{kor}[Konvergenz für FD-Diskretisierung, Poisson-RWP] \label{2.26}
	Sei $\Omega \subset \R^d$ beschränktes Gebiet und die Lösung $u$ des Poisson-RWP erfülle $u \in C^4(\_\Omega)$.
	Dann konvergiert das FD-Verfahren, d.h.
	\[
		\|u - u_h\|_{\_\Omega_h}
		\le C h^p
	\]
	mit $p = 2$ für Würfelgebiete und $p = 1$ für allgemeine Gebiete.
\end{kor}

Ein Weg, Stabilität zu zeigen, führt über das diskrete Maximumsprinzip.
Ein alternativer Weg bietet \ref{2.22}:
es genügt $\|A_h^{-1}\|_\infty \le C_s$ für geeignetes $C_s$ unabhängig von $h$ zu zeigen.
Dies ist mit der sogenannten „M-Matrix-Theorie“ möglich.

\begin{df} \label{2.27}
	\begin{enumerate}[i)]
		\item
			Eine Matrix $A = (a_{ij})_{i,j=1}^n \in \R^{n\times n}$ heißt \emphdef{$L_0$-Matrix}, falls $a_{ij} \le 0$ für alle $i \neq j$.
		\item
			Eine Matrix $A = (a_{ij})_{i,j=1}^n \in \R^{n\times n}$ heißt \emphdef{$L$-Matrix}, falls $A$ ein $L_0$-Matrix ist und $a_{ii} > 0$ für $i= 1, \dotsc, n$.
		\item
			Eine $L_0$-Matrix $A$, für die $A^{-1}$ existiert und $A^{-1} \ge 0$ (komponentenweise größergleich Null) heißt \emphdef{M-Matrix}.
	\end{enumerate}
	\begin{note}
		\begin{itemize}
			\item
				Wenn $A^{-1}$ existiert und $A^{-1} \ge 0$, so nennt man $A$ auch \emphdef{inversmonoton}.
			\item
				Eine M-Matrix ist also eine inversmonotone $L_0$-Matrix.
			\item
				Ziel: für gegebenen $L_0$- oder $L$-Matrix, finde Zusatzbedingungen, welche $M$-Matrix-Eigenschaft implizieren und $\|A^{-1}\|_\infty$ abzuschätzen erlauben.
		\end{itemize}
	\end{note}
\end{df}

\begin{st}[$M$-Kriterium] \label{2.28}
	Sei $A \in \R^{n\times m}$ ein $L_0$-Matrix.
	\begin{enumerate}[i)]
		\item
			$A$ ist inversmonoton (also $M$-Matrix) genau dann, wenn $e \in \R^n$ existiert mit $e > 0$ und $Ae > 0$.
		\item
			Falls $A$ eine $M$-Matrix und $e$ wie in i), so gilt
			\[
				\|A^{-1}\|_{\infty} \le \frac{\|e\|_\infty}{\min_{k}(Ae)_k}.
			\]
	\end{enumerate}
	\begin{proof}
		\begin{enumerate}[i)]
			\item
				\begin{segnb}{\ProofImplication}
					Setze $e := A^{-1} \Vector{1 & \dots & 1}$, dann ist $e > 0$ und $Ae = \Vector{1 & \dots & 1} > 0$.
				\end{segnb}
				\begin{segnb}{\ProofImplication*}
					Sei $e > 0$ und $Ae > 0$, d.h. $\sum_{j} a_{ij} e_j > 0$ für alle $i = 1, \dotsc, n$.
					Weil $a_{ij} e_j \le 0$ für $i \neq j$, da $A$ $L_0$-Matrix, muss $a_{ii} e_i > 0$, also $a_{ii} > 0$ und $A$ ist eine $L$-Matrix.
					Setze $D := \diag(a_{11}, \dotsc, a_{nn})$, diese ist offenbar invertierbar.
					Setze $P d= D^{-1}(D-A) = I - D^{-1} A$, also $A = D(I-P)$ und $P \ge 0$ (wegen $D - A \ge 0$ und $D^{-1} \ge 0$).
					Weiter ist $(I-P)e = D^{-1}A e > 0$, also folgt
					\begin{equation} \label{eq:2.7}
						e = Ie > Pe
					\end{equation}
					Führe eine spezielle Norm ein: $\|x\|_e := \max_i \frac{|x_i|}{e_i}$ mit induzierter Matrixnorm $\|P\|_e := \sup_{\|x\|_e = 1} \|Px\|_e$.
					Es gilt $\|e\|_e = \max \frac{|e_i|}{e_i} = 1$, also $\|P\|_e \ge \|Pe\|_e$, andererseits ist für $y \in \R^n$ mit $\|y\|_e = 1$, d.h. $\max_i \frac{|y_i|}{e_i} = 1$ auch $y \le e$ und es folgt wegen $P \ge 0$, dass $Py \le Pe$, also $\|Py\|_e \le \|Pe\|_e$.
					Somit ist
					\[
						\|P\|_e = \sup_{\|x\|_e = 1} \|Px\|_e = \|Pe\|_e = \max_i \frac{(Pe)_i}{e_i}.
					\]
					Wegen \eqref{eq:2.7} ist $Pe < e$ und somit $\|P\|_e < 1$.
					Damit existiert $(I - P)^{-1}$ und es gilt die Darstellung als Neumannsche Reihe:
					\[
						(I - P)^{-1} = \sum_{j=0}^\infty P^j.
					\]
					Wegen $A = D(I-P)$ existiert auch $A^{-1} = (I-P)^{-1}D^{-1}$ und wegen $P \ge 0$ auch $P^j \ge 0$, also mittel Neumannscher Reihe $(I-P)^{-1} \ge 0$, und somit $A^{-1} \ge 0$ inversmonoton.
				\end{segnb}
			\item
				Sei $Aw = f$, d.h. $w = A^{-1} f$ und
				\[
					w_i = (A^{-1}f)_i = \sum_{j} (A^{-1})_{ij} f_j \le \|f\|_\infty \sum_{j=1}^n (A^{-1})_{ij}.
				\]
				Also gilt
				\begin{equation} \label{eq:2.8}
					w \le \|f\|_\infty A^{-1} \Vector{1 & \dots & 1}
				\end{equation}
				und analog $-w \le \|f\|_\infty A^{-1} \Vector{1 & \dots & 1}$.
				Es gilt $Ae \ge (\min_k (Ae_k)) \Vector{1 & \dots & 1}$, mit $Ae > 0$ folgt
				\[
					\frac{Ae}{\min_k (Ae)_k} \ge \Vector{1 & \dots & 1}.
				\]
				Also ist mit \eqref{eq:2.8} $\pm w \le \|f\|_\infty A^{-1} \frac{Ae}{\|\min_k (Ae)_k\|} = \|f\|_\infty \frac{c}{\min_k (Ae)_k}$, es folgt $\|w\|_\infty \le \|f\|_\infty \frac{\|e\|_\infty}{\min_k(Ae)_k}$, also
				\[
					\|A^{-1}\|_\infty
					= \sup_{f\neq 0} \frac{\|A^{-1} f\|_\infty}{\|f\|_\infty}
					= \sup_{f \neq 0} \frac{\|w\|_\infty}{\|f\|_\infty}
					\le \frac{\|e\|_\infty}{\min_k (Ae)_k}.
				\]
		\end{enumerate}
	\end{proof}
\end{st}

\begin{ex*}
	Betrachte die FD für das Poisson-RWP, $d = 1$ wie in \ref{2.8}, $h := \f 1{n+1}$, $A_h = \Matrix{2 & -1 &  & \\ -1 & \ddots & \ddots & \\ &\ddots & \ddots & -1 \\ & & -1 & 2} \in \R^{n\times n}$.
	$A_h$ ist eine $L_0$-Matrix.
	Finde $e \in \R^n, e > 0$ und z.B. $A_h e = \Vector{1 & \dots & 1} > 0$.
	Die Lösung ist $e = (e_k)_{k=1}^n$ mit $e_k = \f 12 k h  (1-kh) = \f 12 kh - \f 12 k^2h^2$.
	Es gilt
	\[
		(A_h e_h)_1 = \frac{1}{h^2} (2e_1 - e_2)
	\]
	mit $e_1 = \f 12 h - \f 12 h^2, e_2 = h - 2h^2$ folgt
	\[
		(A_h e_h)_1 = \f 1{h^2}(h - h^2 - h + 2h^2) = 1
	\]
	und analog $(Ae)_n = 1$ wegen Symmetrie in $A_h$ und $e$.
	Für $1 < k < n$ haben wir
	\begin{align*}
		h^2 (Ae)_k
		&= (-e_{k-1} + 2e_k - e_{k+1}) \\
		&= -\f 12 (k-1)h + \f 12(k-1)^2 h^2 + kh -k^2 h^2 - \f 12 (k+1)h + \f 12 (k+1)^2 h^2 \\
		&= -\f 12 kh + \f 12 h + \f 12(k^2 - 2k + 1)h^2 + kh - k^2h^2 \\
		&\qquad - \f 12 kh - \f 12 h + \f 12 (k^2 + 2k+1)h^2 \\
		&= h^2( \f 12 k^2 - k^2 + \f 12 k^2 - k + \f 12 + k + \f 12) \\
		&= h^2
	\end{align*}
	und damit ist $A_h$ eine $M$-Matrix.

	Es gilt $\|e\|_\infty = \max_k \f 12 kh(1-kh) \le \f 12 \f 12 ( 1 - \f 12) = \f 18$.
	Wegen $Ae = \Vector{1 & \dots & 1}$ ist $\min_k (Ae)_k = 1$ und aus \ref{2.28} ii) folgt
	\[
		\|A_k^{-1}\| \le \frac{\|e\|_\infty}{\min_k (Ae)_k} = \f 18.
	\]
	Vergleich zu $C_S = \frac{R^2}{2d}$ aus \ref{2.23} mit $R = 1, d=1$ liefert $C_s = \f 12$.
	Die Konstante $\f 18$ ist also besser als die in \ref{2.23}.
\end{ex*}

\begin{df}[Diagonaldominanz, Irreduzibilität]
	\begin{enumerate}[i)]
		\item
			$A \in \R^{n\times n}$ ist \emphdef[diagonaldominant!stark]{stark (zeilen-)diagonaldominant}, falls $|a_{ii}| > \sum_{i \neq j}^n |a_{ij}|$ für alle $i = 1, \dotsc, n$.
		\item
			$A \in \R^{n\times n}$ ist \emphdef[diagonaldominant!schwach]{schwach (zeilen-)diagonoldominant}, falls $|a_{ii}| \ge \sum_{i \neq j}^n |a_{ij}|$ für alle $i = 1, \dotsc, n$ mit mindestens einem $k \in \Set{1,\dotsc, n}$ mit $|a_{kk}| > \sum_{j \neq k} |a_{kj}|$.
		\item
			$A$ ist \emphdef{irreduzibel}, falls keine Permutationsmatrix $P$ existiert, sodass
			\[
				P A P^T = \Matrix{ A_{11} & 0 \\ A_{21} & A_{22} }
			\]
			mit $A_{11} \in \R^{p\times p}$ mit $p \in {1, \dotsc, n-1}$.
	\end{enumerate}
\end{df}

\begin{st} \label{2.30}
	Sei $A_h$ eine $L$-Matrix.
	Falls $A_h$ stark diagonaldominant, oder schwach diagonaldominant und irreduzibel, so ist $A_h$ ein $M$-Matrix.
	\begin{proof}
		Siehe Großmann/Roos, Satz 2.8.
	\end{proof}
\end{st}

\begin{note}
	Falls $A_h$ strikt diagonaldominant, so ist $e = \Vector{1 & \dots & 1}$ ein geeigneter Vektor für \ref{2.28}, denn $e > 0$ und $A_h e > 0$.
	Also $\|A_h^{-1}\|_\infty \le \f 1{\min_k(Ae)_k}$ (aber potentiell von $h$ abhängig).
\end{note}

\begin{note}[gemischte Ableitung]
	\begin{itemize}
		\item
			In \ref{2.11} wurde $-\sum_{i\neq j} a_{ij}(x) \partial_{x_i}^{c,h} \partial_{x_j}^{c,h} u(x)$ gewählt, was zu $\f 12 \Matrix[{a_{12} & 0 & -a_{12} \\ 0 & 0 & 0 \\ -a_{12} & 0 & a_{12}}_*$ führte.
			Weder das Maximumsprinzip, noch $M$-Matrix-Eigenschaft können mit unseren Techniken gezeigt werden, wegen unterschiedlichen Vorzeichen im Fall $a_{12} \neq 0$.
		\item
			Man kann FD-Sterne 2. Ordnung mit nichtpositiven Eck-Koeffizienten für $-2a_{12} \partial_{x_1} \partial_{x_2} u$ konstruieren:
			\begin{enumerate}[i)]
				\item
					Falls $- a_{12} > 0$:
					\[
						\Matrix[{ a_{12} & -a_{12} & 0 \\ a_{12} & 2 a_{12} & - a_{12} \\ 0 & - a_{12} & a_{12} }_*
					\]
				\item
					Falls $- a_{12} < 0$:
					\[
						\Matrix[{ 0 & a_{12} & -a_{12} \\ a_{12} & - 2a_{12} & a_{12} \\ - a_{12} & a_{12} & 0}_*
					\]
			\end{enumerate}
			Mit der Konvention $a_{12}^+ := \max\{0, a_{12}\}, a_{12}^- := \min\{a_{12}, 0\}$ folgt
			\begin{equation} \label{eq:2.9}
				\Matrix{
					a_{12}^- & -(a_{22}-|a_{12}|) & -a_{12}^+ \\
					-(a_{11}-|a_{12}|) & 2(a_{11} + a_{22}-|a_{12}|) & - (a_{11} -|a_{12}|) \\
					-a_{12}^+ & -(a_{22} - |a_{12}|) & a_{12}^-
				}_*
				+ \f 12 \Matrix{
					0 & b_2 & \\
					-b_1 & 2hc & b_1 \\
					0 & -b_2 & 0
				}_*
			\end{equation}
	\end{itemize}
\end{note}

\Timestamp{2014-11-11}

\begin{kor} \label{2.31}
	Falls $a_{ii} > |a_{12}| + \f h2 |b_i|$ und $c > 0$ für $i= 1,2$, so ist die FD-Systemmatrix $A_h$ zum FD-Stern \eqref{eq:2.9} eine $M$-Matrix.
	\begin{proof}
		Die Nichtdiagonalelemente von $A_h$ sind nicht-positiv.
		Die Diagonale von $A_h$ ist echt positiv, also ist $A_h$ eine $L$-Matrix.
		$A_h$ ist streng diagonaldominant (Summe aller FD-Stern-Einträge ist größer 0, FD-Stern wird in jeweils eine Zeile von $A_h$ geschrieben).
	\end{proof}
	\begin{note}
		\begin{itemize}
			\item
				Man kann zeigen, dass $\|A_h^{-1}\|_\infty$ unabhängig von $h$ beschränkt ist.
			\item
				Die Bedingung $a_{ii} > |a_{12}| + \f h2 |b_i|$ liefert eine Bedingung für $h$, d.h. eine hinreichend kleine Gitterweite ist erforderlich für Stabilität im Fall $b \neq 0$.
			\item
				Falls $b_i$ sehr „groß“ (sogenannter \emphdef{konvektionsdominanter Fall}, kann dies zu impraktikablen Gitterweiten führen).
			\item
				Falls $A(x) = 0, c = 0$ (also reine Advektion), $b \neq 0$ kann man leicht analytisch sehen, dass der FD-Vektor mit zentralen Differenzen nicht stabil ist: „\emphdef{hyperbolische Gleichung erster Ordnung}“.

				Betrachte $\Omega = (0,1)^2, b = \Vector{1 & 1}, g(x_1, x_2) = (x_1 - x_2)^2$ mit $\partial_{x_1} u + \partial_{x_2} u = 0$ in $\Omega$ und $u = g$ auf $\Gamma$.
				Dies hat die exakte Lösung $u(x_1, x_2) = (x_1 - x_2)^2 = x_1^2 - 2x_1x_2 + x_2^2$.

				Mit $h = \f 12$ folgt $|\Omega_h| = 1$ wegen $\Omega_h = \Set{(\f 12, \f 12)}$, $|\Gamma_h| = 8$.
				Es ergibt sich ein $1\times 1$-System für $u_h(\f 12, \f 12)$.
				Mit zentralen Differenzen (wie in \eqref{eq:2.5}, \eqref{eq:2.9}) für $\partial_{x_1} u, \partial_{x_2} u$ liefert folgende Gleichung für das LGS:
				\[
					\dfrac{u(\f 12 + \f 12, \f 12) - u_h(\f 12 - \f 12, \f 12)}{2 \cdot \f 12}
					+ \dfrac{u(\f 12, \f 12 + \f 12) - u_h(\f 12, \f 12 - \f 12)}{2 \cdot \f 12}
					= 0.
				\]
				Alle Punktauswertungen liegen in $\Gamma_h$, also setzen wir die Werte für $g$ ein:
				\[
					g(1, \f 12) - g(0, \f 12) + g(\f 12, 1) - g(\f 12, 0) = 0,
				\]
				was zu $0 = 0$ führt, die Systemmatrix $A_h = (0)$ ist singulär, insbesondere ist nicht $\|A_h^{-1}\| \le C_s$ mit $C_s$ unabhängig von $h$, das FD-Verfahren also nicht stabil.
			\item
				Für konvektionsdominante Probleme oder hyperbolische Probleme erster Ordnung sind sorgfältige Diskretisierungen erforderlich (siehe \ref{chap:5}, FV-Verfahren).
		\end{itemize}
	\end{note}
\end{kor}

\begin{ex*}
	Betrachte das Poisson-RWP auf $\Omega = (0,1)^2$.
	\begin{enumerate}[i)]
		\item
			Gebe exakte Lösung vor: $u(x_1, x_2) := x_1(1-x_1)x_2(1-x_2)$.
			Für die Daten wählen wir dann entsprechend: $g(x) := 0 = u(x)$ auf $\Gamma$ und $f(x) := 2x_2(1-x_2) + 2x_1(1-x_1) = -\Laplace u$
			\begin{table}
				\centering
				\begin{tabular}{l|c|l}
					$h$ & $n$ & $\|u - u_h\|_{\_\Omega_h}$ \\ \hline
					0.5 & 1 & 0 \\
					0.25 & 9 & $6 \cdot 10^{-18}$ \\
					\vdots & \vdots & \vdots \\
					0.03125 & 961 & $4.8 \cdot 10^{16}$
				\end{tabular}
				\caption{\texttt{elliptic\textunderscore fd\textunderscore demos(4)}}
			\end{table}
			Wir sehen, dass $u_h$ für jedes $h$ exakt ist (bis auf numerische Rundungseffekte).
			Dies deckt sich mit der Beobachtung, dass $-\Laplace_h u = - \Laplace u$ für Polynome.
		\item
			Setze als exakte Lösung: $u(x_1, x_2) := \sin(2\pi x_1) \sin(2\pi x_2) \in C^\infty$ für Daten $f(x) := 8\pi^2 \sin(2\pi x_1) \sin(2\pi x_2), g(x) := 0$.
			\begin{table}
				\centering
				\begin{tabular}{l|cl}
					h & h & $\|u - u_h\|_{\_\Omega_h}$ \\ \hline
					0.25 & 9 & $0.2337$ \\
					0.125 & 49 & $0.053024$ \\
					\vdots & \vdots & \vdots \\
					0.00097656 & 1046529 & $3.1375 \cdot 10^{-6}$
				\end{tabular}
				\caption{\texttt{elliptic\textunderscore fd\textunderscore demos(5)}}
			\end{table}
			Das LGS mit $A \in 10^{10^6 \times 10^6}$ ist sehr schnell lösbar dank sparse-Matrizen.
			Konvergenzordnung $2$ ist erkennbar in Übereinstimmung mit der Theorie.
	\end{enumerate}
\end{ex*}

\begin{note}[Relevanz der FD-Verfahren]
	\begin{itemize}
		\item
			Bis Mitte des 20. Jahrhunderts wurden FD-Diskretisierungen als Allzweckwerkzeug gesehen, weil sie eine Vielzahl von Problemen sehr leicht mit ausreichender Genauigkeit diskretisieren.
			Allmählich kamen FEM-Methoden (Finite-Elemente-Methoden), welche eine wesentlich aufwändigere Assemblierung von $A_h$ und $b_h$ erfordern, aber bei gleicher Gitterfeinheit bessere Ergebnisse liefern.
		\item
			Die Konvergenzanalysis von FD-Verfahren macht häufig starke (unrealistische) Glattheits-Annahmen an die Lösung.
			Das wird bei FEM- und FV-Verfahren etwas abgeschwächt.
	\end{itemize}
\end{note}

\chapter{Restringierte Optimierung}



\section{Lineare Optimierung}


\subsection{Motivation}

Zur Schweinefütterung stehen zwei Möglichkeiten zur Verfügung:
 Soja zu 1€ pro Einheit, enthält 2 Proteiene und 4 Fett; Kartoffeln zu 2€ pro Einheit, enthält 2 Proteine und 2 Fett.

Futter soll $\ge 10$ Proteine und $\le 12$ Fett besitzen.
Dies führt auf die Minimierungsaufgabe
\[
	x + 2y \to \min!
	\udN
	\begin{cases}
		2x + 2y &\ge 10 \\
		4x + 2y &\le 12 \\
		x &\ge 0 \\
		y &\ge 0
	\end{cases}.
\]

\coursetimestamp{11}{12}{2013}

Jede Ungleichung kann in die Form
\[
	\sum_{j} a_{ij} x_j \le b_i
\]
gebracht werden.
Jede Ungleichung kann in die Form
\begin{align*}
	\sum a_{ij} x_j + \xi &= b_i \\
	\xi &\ge 0
\end{align*}
gebracht werden.
Mit $x_j = x_j' - x_j'', x_j' \ge 0, x_j'' \ge 0$
kann für jede Variable $x_j \ge 0$ gefordert werden.

\begin{df}[Lineares Optimierungsproblem, Normalform] \label{3.1}
	Sei $A \in \R^{n\times n}, b\in \R^m, c \in \R^n$.
	Das Minimierungsproblem
	\[
		f(x) := c^T x \to \min!
		\udN
		Ax = b \land x \ge 0
	\]
	heißt \emph{lineares Optimierungsproblem (engl. linear program) in Normalform}.
	Wir nennen $f$ \emph{Zielfunktion} und die Menge
	\[
		K := \{x \in \R^n : Ax = b \land x \ge 0 \}
	\]
	\emph{zulässiger Bereich}.
\end{df}

\begin{conv*}
	Im Folgenden nutzen wir die Subskript-Notation $x = (x_1, \dotsc, x_n)^T \in \R^n$ für Komponenteneinträge eines Vektors und schreiben für $x_1, \dotsc, x_n \ge 0$ auch kurz $x \ge 0$.
\end{conv*}

\begin{nt} \label{3.2}
	\begin{itemize}
		\item
			Der zulässige Bereich $K$ kann einelementig sein.
			In diesem Fall ist dieses Element die eindeutige Lösung.
		\item
			Auch für nicht-leere zulässige Bereiche muss keine Lösung existieren.
			Betrachte beispielsweise das nach unten unbeschränkte Problem: $n=m=1, A = 0, b=0, c=-1$.
		\item
			Ist der zulässige Bereich $K$ nicht-leer und beschränkt, so existiert eine Lösung.
	\end{itemize}
\end{nt}

\subsection{Geometrische Interpretation}

\begin{df} \label{3.3}
	\begin{enumerate}[(a)]
		\item
			Zu gegebenen Punkten $y^1, \dotsc, y^n \in \R^n$ heißt
			\[
				\sum_{i=1}^m \lambda_i y^i
			\]
			mit $\lambda_i \in [0,1]$ und $\sum_{i=1}^n \lambda_i = 1$ \emph{Konvexkombination}.
			Die Menge aller Konvexkombinationen
			\[
				S(y^1, \dotsc, y^m) := \Set{
					\sum_{i=1}^m \lambda_i y^i |
					\lambda_i \in [0,1], \sum_{i=1}^n \lambda_i = 1
				}
			\]
			heißt \emph{der von $y^1,\dotsc, y^m$ aufgespannte Simplex}.

			\begin{note}
				Damit ist $S(y^1, y^2) = [y^1, y^2]$ die Verbindungsstrecke aus \ref{2.1}.
			\end{note}
		\item
			Für $\emptyset \neq K \subset \R^n$ nennen wir $x \in K$ eine \emph{Ecke von $K$}, falls
			\[
				\forall y^1, y^2 \in \R^n, x \in [y^1, y^2] \subset K : x = y^1 \lor x = y^2.
			\]
	\end{enumerate}
\end{df}

Betrachte stets $A \in \R^{m\times n}, b\in \R^m, c \in \R^n$ und
\[
	K:= \{ x \in \R^n : Ax = b, x \ge 0 \}.
\]
$K$ ist offenbar konvex, d.h. $\forall y^1, y^2 \in K : [y^1, y^2] \subset K$.

\begin{ex} \label{3.4}
	Es gilt $0 \in K \iff b = 0$.
	Für $0 \in K$ ist $x = 0$ Ecke von $K$, denn für alle $y^1, y^2 \in K$ mit
	\[
		0 = (1-\lambda) y^1 + \lambda y^2,
		\qquad \lambda \in [0,1]
	\]
	ist $y^1 = 0$ oder $y^2 = 0$.
\end{ex}

\begin{df} \label{3.5}
	Teilmengen $I \subset \{1, \dotsc, n\}$ nennen wir \emph{Indexmengen} und bezeichnen die Anzahl der Elemente in $I$ mit $|I|$.
	Für $I \neq \emptyset$ bilden wir für $A \in \R^{m\times n}, v \in \R^n$.
	\begin{align*}
		A_I &:= (a_{ij})_{\substack{i=1,\dotsc,m \\ j \in I}}
			\in \R^{m\times |I|}, \\
		v_I &:= (v_j)_{j\in I}
			\in \R^{|I|}.
	\end{align*}
	Jedem $x \in \R^n$ ordnen wir zu
	\[
		I_x = \Big\{ j \in \{1, \dotsc, n\} : x_j > 0 \Big\}
		\subset \{1, \dotsc, n\}
	\]
	Für $x \neq 0$ schreiben wir auch
	\[
		A_x := A_{I_x}
	\]
	und $v_x$ statt $v_{I_x}$.
	Es gilt offenbar für $0 \neq x \in K$
	\[
		b = Ax = A_x x_x.
	\]
	Es gilt auch
	\[
		A_x v_x = A v
	\]
	für alle $v \in \R^n$ mit $I_v \subset I_x$.
\end{df}

\begin{lem} \label{3.6}
	\begin{enumerate}[(a)]
		\item
			Ist $0 \neq x \in K$ keine Ecke von $K$, so existieren $y^1, y^2 \in K$ mit $y^1 \neq x \neq y^2$ und $y^1 \neq y^2$, so dass
			\[
				x \in [y^1, y^2], \quad I_{y^1}, I_{y^2} \subset I_x.
			\]
		\item
			$0 \neq x \in K$ ist genau dann eine Ecke von $K$, wenn $A_x$ injektiv ist, also $\rg(A_x) = |I_x|$.
	\end{enumerate}
	\begin{proof}
		\begin{enumerate}[(a)]
			\item
				Ist $x \in K$ keine Ecke, dann existieren $y^1, y^2 \in K, y^1 \neq y^2$ und für $\lambda \in (0,1)$
				\[
					x = (1-\lambda) y^1 + \lambda y^2.
				\]
				Für jedes $j$ gilt
				\[
					y_j^1 > 0 \implies x_j > 0 \qquad
					y_j^2 > 0 \implies x_j > 0
				\]
				also $I_{y^1}, I_{y^2} \subset I_x$.
			\item
				Außerdem ist
				\[
					A_x (y_x^1 - y_x^2)
					= Ay^1 - Ay^2
					= b - b = 0.
				\]
				Wegen $y^1 \neq y^2 \implies y_x^1 \neq  y_x^2$.
				Damit ist $A_x$ nicht injektiv, wenn $x$ keine Ecke ist.

				Sei $A_x$ nicht injektiv, dann existiert $0 \neq y_x \in \R^{|I_x|}$ mit $A_x y_x = 0$, den wir durch Nullen zu $y \in \R^n$ fortsetzen.
				Wähle $\eps > 0$ so klein, dass $\eps |y_j| < x_j$ für alle $j \in I_x$.
				Setze
				\[
					y^1 := x - \eps y,
					\quad
					y^2 := x + \eps y
				\]
				Offenbar ist $x \in [y^1, y^2]$ und $y^1 \neq x, y^2 \neq x$.
				Aus $Ay = A_x y_x = 0$ folgt
				\[
					Ay^1 = Ax = b = Ax = Ay^2.
				\]
				und wegen der Wahl von $\eps$ ist
				\[
					y^1, y^2 \ge 0.
				\]
		\end{enumerate}
	\end{proof}
\end{lem}

\coursetimestamp{16}{12}{2013}
\begin{kor} \label{3.7}
	Ein Punkt $x \in K$ mit nichtleerer Indexmenge $I_x \subset \{1, \dotsc, n\}$ ist eine Ecke von $K$ genau dann wenn $A_{I_x} \tilde x = b$ mit $\tilde x \in \R^{|I_x|}$ genau eine Lösung mit nicht-negativen Komponenten besitzt.

	In diesem Fall ergibt sich die Ecke $x \in \R^n$ durch Nullfortsetzung aus $\tilde x \in \R^{|I_x|}$.

	Insbesondere existieren nur endlich viele Ecken.
	\begin{proof}
		\begin{segnb}[„$\implies$“]
			Sei also $x \in K$ eine Ecke mit nichtleerer Indexmenge $I_x \subset \{1, \dotsc, n\}$.
			Dann gilt
			\[
				A_{I_x} x_{I_x} = A_x x_x = Ax = b
			\]
			und wegen \ref{3.6} b) ist $x_x$ auch die einzige Lösung von $A_{I_x} \tilde x = b$ mit nichtnegativen Komponenten.
		\end{segnb}
		\begin{segnb}[„$\impliedby$“]
			Ist nun also $A_{I_x} \tilde x = A_x \tilde x = b$ eindeutig lösbar, so ist insbesondere $A_x$ injektiv, besitzt also linear unabhängige Spalten.
			Bezeichne $x \in \R^n$ die Nullfortsetzung von $\tilde x$, dann ist $I_x \subset \{1, \dotsc, n\}$ und wenn die Komponenten von $\tilde x$ nicht-negativ sind, so ist $x \in \R^n$ ein zulässiger Punkt und nach \ref{3.6} b) eine Ecke von $K$.
		\end{segnb}
	\end{proof}
\end{kor}

\begin{lem} \label{3.8}
	\begin{enumerate}[(a)]
		\item
			Jeder Punkt $x \in K$ mit minimaler Indexmenge, d.h. $|I_x| \le |I_y|$ für alle $y \in K$, ist eine Ecke.
			Insbesondere besitzt jeder nicht-leere zulässige Bereich mindestens eine Ecke.
		\item
			Existiert auf dem zulässigen Bereich ein Minimum der Zielfunktion $f(x) = c^T x$, so ist einer der Minimierer eine Ecke.
	\end{enumerate}
	\begin{proof}
		\begin{enumerate}[(a)]
			\item
				Ist $0 = x \in K$, so ist $I_x = \emptyset$ und $x = 0$ ist wegen \ref{3.4} auch eine Ecke.

				Sei also $x \neq 0$ mit minimaler Indexmenge, aber $x$ sei \emph{keine} Ecke.
				Nach \ref{3.6} a) existieren dann paarweise von $x$ verschiedene $y^1, y^2 \in K$ mit $x \in [y^1, y^2]$ und $I_{y^1}, I_{y^2} \subset I_x$.

				Betrachte $v := y^2 - y^1$ und für ein $\lambda \in \R$ den Punkt $z_\lambda := x + \lambda v$.
				Dann gilt für alle $\lambda \in \R$
				\[
					A z_\lambda
					= A (x + \lambda v)
					= b + \lambda (b-b)
					= b.
				\]
				Wir wollen nun zeigen, dass ein $\_\lambda$ existiert, sodass $z_{\_\lambda} \in K$ ist (d.h. $z_\lambda \ge 0$) und $|I_{z_\lambda}| < |I_x|$ gilt.

				Da $y^1, y^2$ paarweise verschieden waren, gilt $\emptyset \neq I_v \subset I_x$.
				Für jedes $j \in I_v$ setzen wir
				\[
					\lambda_j
					:= - \f {x_j}{v_j}
					= - \f {x_j}{y^2_j - y^1_j}
					\neq 0
				\]
				und wählen $k \in I_v$, sodass $|\lambda_k| \le |\lambda_j|$ für alle $j \in I_v$.
				Wähle $\_\lambda := \lambda_k$ und $z := z_{\_\lambda} = x + \lambda_k v$.
				$z$ erfüllt
				\begin{itemize}
					\item
						$z_k = x_k + \lambda_k v_k = x_k - x_k = 0$
					\item
						$z_j = x_j + \lambda_k v_j \ge 0$ für $j \in I_v$, denn wäre $z_j$ negativ, so müsste die Funktion $\lambda \mapsto x_j + \lambda v_j$ eine betragskleinere Nullstelle $\lambda_j$ als $\lambda_k$ besitzen.
				\end{itemize}
				Damit ist $z \in K$ zulässig.
				Außerdem gilt für $\lambda_k > 0$ dass $x \in [y^1, z]$ und für $\lambda_k < 0$ dass $x \in [z, y^2]$.
				Da also $x$ keine Ecke von $K$ ist, existieren $y, z \in K, \lambda \in (0,1)$, sodass $x = (1-\lambda) y + \lambda z$ und es gilt $|I_z| < |I_i| \le |I_x|$, ein Widerspruch zu $|I_x| \le |I_y|$ für alle $y \in K$.
			\item
				Sei also $x \in K$ ein Minimierer mit (unter den Minimierern) minimaler Indexmenge, d.h.
				\[
					x \in \argmin_{\xi \in K} c^T \xi
					\land \forall x' \in \argmin_{\xi \in K} c^T \xi : |I_x| \le |I_{x'}|.
				\]
				Falls $x$ \emph{keine} Ecke ist, so existieren wie im a)-Teil wieder $y, z \in K, \lambda \in (0, 1)$ mit $x = (1 - \lambda)y + \lambda z$ und $|I_z| < |I_x|$.
				Noch zu zeigen ist: $z$ ist Minimierer von $f$.

				Da $x$ Minimerer von $f$ ist, gilt $c^Tx \le c^Ty, c^Tx \le c^Tz$.
				Es gilt
				\[
					c^T x = (1-\lambda)c^T y + \lambda c^T z
					\le c^Ty
				\]
				und damit auch $c^T z \le c^T y$.
				Analog zeigt man $c^T y \le c^T z$, also $c^T y = c^T z$.
				Schließlich folgt
				\[
					c^T x = (1-\lambda)c^T y + \lambda c^Tz = c^T y = c^T z.
				\]
				Damit ist auch $z$ Minimierer von $f$ auf $K$, ein Widerspruch zu $|I_z| < |I_x|$.
		\end{enumerate}
	\end{proof}
\end{lem}

\begin{nt} \label{3.9}
	\begin{itemize}
		\item
			Das betrachtete Optimierungsproblem ist damit gelöst.
			Es genügt, für jede (der endlich vielen, höchstens $m$-elementigen) Teilmengen von $\{1, \dotsc, n\}$ gemäß \ref{3.7} zu prüfen, ob es eine dazugehörige Ecke $x \in K$ gibt (für $b = 0$ muss noch $x = 0$ als Ecke hinzugenommen werden).
		\item
			Gibt es einen Minimierer, so ist dies eine der Ecken und es müssen nur noch die Werte der Zielfunktion für die endlich vielen Ecken berechnet und verglichen werden.
		\item
			Falls das Minimierungsproblem lösbar ist, erhalten wir so nach endlich vielen Schritten eine Lösung.
	\end{itemize}
\end{nt}

\coursetimestamp{13}{01}{2014}

\subsection{Das Simplexverfahren}

Im folgenden setzen wir voraus, dass alle Ecken nicht-entartet sind, d.h. dass für alle Ecken $x$ von $K$ gilt, dass $|I_x| = m$.

Wir formulieren das Verfahren direkt auf der Indexmenge.
Zu einer Ecke $x$ bezeichnen wir die Indexmenge $I_x$ mit $B \subset \{1, \dotsc, n\}$ (Basisvariablen) und $N := \{1, \dotsc, n\} \setminus B$ (Nichtbasisvariable).
Entsprechend definieren wir $A_B, A_N, v_B$ und $v_N$ zu $A \in \R^{m\times n}, v \in \R^n$.

Für einen Algorithmus, der sich von Ecke zu Ecke „hangelt“ und dabei das Zielfunktional verbessert sind folgende Dinge nötig:
\begin{enumerate}[1.]
	\item
		Bestimmung einer Startecke (genauer: ein $B$, das zu einer Ecke gehört).
	\item
		Wie kommen wir von einer Ecke zu einer neuen Ecke mit besserem Zielfunktionalswert?
	\item
		Wie erkennen wir eine optimale Ecke?
\end{enumerate}

\subsubsection{Das Stoppkriterium}

Gehöre $B \subset \{1, \dotsc, n\}$ zu einer Ecke.
Nach \ref{3.6} ist $A_B \in \R^{m\times |B|}$ injektiv und dank der vereinfachten Voraussetzung $A_B$ invertierbar.
Jedes $x \in K \subset \R^n$ ist eindeutig festgelegt durch $x_B \in \R^n$ und $x_N \in \R^{n-m}$.
Es gilt
\[
	b = Ax = A_Bx_B + A_Nx_N,
\]
also $x_B = A_B^{-1} (b - A_N x_N)$ für alle $x \in K$.

In der zu $B$ gehörigen Ecke $x \in K$ gilt $x_N = 0, x_B = A_B^{-1}b$ und
\[
	c^T x = c_B^T x_B + c_N^T x_N = c_B^T A_B^{-1}b
\]
in jedem anderen zulässigen Punkt $x \in K$ ist
\begin{align*}
	c^Tx
	&= c_B^T x_B + c_N^T x_N \\
	&= c_B^T A_B^{-1}(b - A_Nx_N) + c_N^T x_N \\
	&= c_B^T A_B^{-1}b + (c_N^T - c_B^T A_B^{-1}A_N) x_N.
\end{align*}
$c_N^T - c_B^T A_B^{-1} A_N$ heißt \emph{Vektor der reduzierten Kosten}.
Für $c_N^T c_B^T A_B^{-1} A_N \ge 0$, dann ist $c^T x \ge c_B^T A_B^{-1} b$, für alle $x \in K$ also die zu $B$ gehörige Ecke optimal.

Demnach ist das Stoppkriterium für das Simplexverfahren die Bedingung $c_N^T - c_B^T A_B^{-1} A_N \ge 0$.

\subsection{Der Pivotschritt}

Es gelte
\[
	c_N^T - c_B^T A_B^{-1} A_N \not\ge 0,
\]
etwa $(c_N^T - c_B^T A_B^{-1} A_N)_j < 0$ für ein $j \in \{1, \dotsc, n-m\}$.
Jedes $x_N = (x_k)_{k\in\N}$ mit $x_k = 0$ für $k \neq j$ und $x_j > 0$ führt mit
\[
	x_B := A_B^{-1}(b - A_N x_N)
\]
zu einem $x \in \R^n$ mit geringerem Zielfunktionalswert $c^T x$ und $Ax = b$ und $x$ ist zulässig genau dann, wenn $x_B \ge 0$.

$x_B$ hängt stetig von $x_j \ge 0$ ab.
Für $x_j = 0$ ist $x_B > 0$, d.h. es existiert $\eps > 0$ mit $x_B \ge 0$ für alle $x_j \in [0, \eps]$.

Entweder bleibt $x_B$ positiv für alle $x_j > 0$, dann ist das Zielfunktional nach unten unbeschränkt.
Ansonsten existiert ein kleinstes $\hat x_j > 0$, sodass $\hat x_B \ge 0$ und für alle $x_j > \hat x_j$ gilt $x_B \not\ge 0$.
Ersetze dann den zugehörigen Index in $B$ (der jetzt 0 geworden ist) durch $j$ und definiere so $B'$.
$B'$ ist damit eine Indexmenge einer Ecke mit geringerem Zielfunktionalswert.

\coursetimestamp{15}{01}{2014}

\paragraph{Implementierung}

Sei \oBdA $N = \{1, \dotsc, n-m \}, B = \{n-m+1, \dotsc, n\}$.
Bezeichne mit $(\hat a_{kl})$ die Einträge von $\hat A := A_B^{-1} A_N \in \R^{m \times (n-m)}$ und mit $\hat b_k$ die Einträge von $\hat b := A_B^{-1}b \in \R^m$.

Für die zu $B$ gehörige Ecke $x$ gilt
\[
	x_N = 0
	\quad\land\quad
	x_B = A_B^{-1} b = \hat b.
\]
Vektor der reduzierten Kosten
\[
	c_N^T - c_B^T A_B^{-1} A_N
\]
mit
\begin{align*}
	c_N &= (c_k)_{k=1}^{n-m}, &
	c_B &= (c_{n-m+k})_{k=1}^{m}
\end{align*}
Der $j$-te Eintrag von $c_n^T - c_B^T A_B^{-1} A_N$ ist
\[
	c_j - \sum_{k=1}^m c_{n-m+k} \hat a_{kj}
\]
also existiert $j \in \{1, \dotsc, n-m \}$ mit
\begin{align*}
	&c_j - \sum_{k=1}^m c_{n-m+k} \hat a_{kj} < 0 \\
	&\qquad \iff c_N^T - c_B^T A_B^{-1} A_N \not\ge 0.
\end{align*}
Sei $j$ ein solcher Index.
Aus $x_B = A_B^{-1} (b - A_N x_N)$ folgt
\[
	x_{n-m+k} = \hat b_k - \sum_{l=1}^{n-m} \hat a_{kl} x_l
\]
für alle $k = 1, \dotsc, m$.
Wähle jetzt $x_l = 0$ für alle $l = 1, \dotsc, j-1, j+1, \dotsc, n-m$, also
\[
	x_{n-m+k} = \hat b_k - \hat a_{kj} x_j.
\]
Da die Ecke zu $B$ zulässig und nicht-entartet war, muss $\hat b = x_B > 0$ gelten.
Falls $\hat a_{kj} \le 0$ für alle $k \in \{1, \dotsc, m\}$, dann ist das Zielfunktional nach unten unbeschränkt.
Sei also $\hat a_{kj} > 0$ für mindestens ein $k \in \{1, \dotsc, m\}$.
Dann gilt
\[
	0 = x_{n-m+k}
	= \hat b_k - \hat a_{kj} x_j
	\quad\iff\quad
	x_j = \f {\hat b_k}{\hat a_{kj}}.
\]
Wähle also gemäß (a)
\[
	\hat k := \argmin_{\substack{k=1,\dotsc, m \\ \hat a_{kj} > 0}} \f {\hat b_k}{\hat a_{kj}}.
\]
und ersetze den in $N$ enthaltenen Index $j$ durch $n - m + \hat k$ und den in $B$ enthaltenen Index $n - m + \hat k$ durch $j$.

\begin{nt} \label{3.10}
	Ohne die Sortierungsannahme ist der $j$-te Eintrag von $c_N^T - c_B^T A_B^{-1} A_N$ gerade
	\[
		c_{N(j)} - \sum_{k=1}^m c_{B(k)} \hat a_{kj},
	\]
	wobei $N(j)$ das $j$-te Element von $N$ und $B(j)$ das $j$-te Element von $B$ bezeichne.

	Die Elemente von $x_B = A_B^{-1} (b - A_N x_N)$ sind
	\[
		x_{B(k)} = \hat b_k - \sum_{l=1}^{n-m} \hat a_{kl} x_{N(l)}.
	\]
	Mit $x_{N(l)} = 0$ für alle $l \neq j$ folgt
	\[
		x_{B(k)} = \hat b_k - \hat a_{kj} x_{N(j)}.
	\]
	Wähle also
	\[
		\hat k := \argmin_{\substack{k=1,\dotsc,m \\ \hat a_{kj > 0}}} \f {\hat b_{k}}{\hat a_{kj}}
	\]
	und ersetze das $\hat k$-te Element von $B$ durch das $j$-te Element von $N$ und vice versa.
\end{nt}

\subsection{Bestimmung einer Startecke}

Sei \oBdA $b \ge 0$ (evtl. durch Negieren der Zeilen von $A$).
Betrachte das Ersatzproblem:
suche $(x^T,y^T)^T \in \R^n \times \R^m$ mit
\begin{align*}
	\begin{pmatrix}
		0 & \cdots & 0 & 1 & \cdots & 1
	\end{pmatrix}
	\begin{pmatrix}
		x \\ y
	\end{pmatrix}
	\to \min! \\
	\udN
	\begin{pmatrix}
		A & I_{m\times m}
	\end{pmatrix}
	\begin{pmatrix}
		x \\ y
	\end{pmatrix}
	= b
	\quad \land \quad
	\begin{pmatrix}
		x \\ y
	\end{pmatrix}
	\ge 0.
\end{align*}
Ist der zulässig Bereich des ursprünglichen Bereichs nicht leer, also $K = \{x \in \R^n : Ax = b, x \ge 0 \} \neq \emptyset$, dann sind die Minimierer des Ersatzproblems genau diese zulässigen Punkte.

Wir setzen voraus, dass auch das Ersatzproblem keine entarteten Ecken besitzt.
$(0, b^T)$ ist Ecke des Ersatzproblems, da er höchstens $m$ Nicht-Null-Einträge enthält.
Die Minimierung des Ersatzproblems liefert
\begin{enumerate}[(a)]
	\item
		eine zulässige Ecke von $k$, falls das Minimum von der Form $(x^T, 0)^T$ ist,
	\item
		oder $K = \emptyset$, falls das Minimum nicht die Form $(x^T, 0)^T$ hat.
\end{enumerate}

\coursetimestamp{20}{01}{2014}

\section{Restringierte nichtlineare Optimierung}


Betrachte
\[
	f(x) \to \min! \udN g(x) \le 0, h(x) = 0.
\]
mit
\begin{itemize}
	\item
		$f: \R^n \to \R$
	\item
		Ungleichungsrestriktionen $g: \R^n \to \R^m$
	\item
		Gleichungsrestriktionen $h: \R^n \to \R^p$
	\item
		$f,g,h$ seine stes stetig differenzierbar
\end{itemize}

\begin{df} \label{3.11}
	Definiere den \emph{zulässigen Bereich} als
	\[
		X = \Set{ x \in \R^n | g(x) \le 0, h(x) = 0 }.
	\]
	Für zulässige Punkte $x \in X$ heißt
	\[
		I_x := \Set{ j \in \{1, \dotsc, m\} | g_j(x) < 0 }
	\]
	die \emph{Indexmenge inaktiver Ungleichungsnebenbedingungen}.
\end{df}

\subsection{Optimalitätsbedingungen}

Die notwendige Optimalitätsbedingung für unrestringierte Probleme war $\nabla f(\hat x) = 0$.

Jedes $d \in \R^n$ mit $\nabla f(\hat x)^T d < 0$ ist eine Abstiegsrichtung und es gilt
\[
	f(\hat x + sd) < f(\hat x)
\]
für hinreichend kleine $s$ (siehe auch \ref{2.20}, \ref{2.21}).
Die Bedingung $\nabla f(\hat x) = 0$ bedeutet also: es gibt keine Abstiegsrichtung.

\paragraph{Tangentialkegel}

\begin{ex} \label{3.12}
	Das Problem
	\[
		f(x) := x \to \min! \udN  x\in \R, x \ge 0
	\]
	besitzt offenbar ein Minimum bei $\hat x = 0$, aber $f'(\hat x) = 1$, d.h. $d = -1$ ist Abstiegsrichtung, ($\hat x + sd$ gehört aber für kein $s > 0$ zum zulässigen Bereich).
\end{ex}

\begin{df} \label{3.13}
	\begin{enumerate}[(a)]
		\item
			Eine Menge $K \subset \R^n$ heißt \emph{Kegel}, falls $\forall x \in K, \lambda \ge 0 : \lambda x \in K$.
		\item
			Für $\emptyset \neq X \subset \R^n$ heißt $d \in \R^n, \|d\| = 1$ \emph{Tangentialrichtung in $x \in X$}, falls eine Folge $(x^k)_{k\in\N}$ existiert mit
			\[
				x^k \to x, x^k \neq x, \lim_{k\to\infty} \f{x^k - x}{\|x^k-x\|} = d
			\]
			Ist $X$ der zulässige Bereich eines Optimierungsproblems, so heißen diese $d$ auch \emph{zulässige Richtungen}.
		\item
			Der von den Tangentialrichtungen aufgespannte Kegel heißt \emph{Tangentialkegel}:
			\[
				T(X, x) := \{ \lambda d : \lambda \ge 0, d \text{ Tangentialrichtung} \} \subset \R^n.
			\]
	\end{enumerate}
	\begin{note}
		In der Literatur wird für einen Kegel auch manchmal die Abgeschlossenheit der Addition gefordert.
	\end{note}
\end{df}

\begin{ex} \label{3.14}
	\begin{enumerate}[(a)]
		\item
			Ist $X$ offen, dann ist für alle $x \in X$, $T(X, x) = \R^n$.
		\item
			Ist $x$ isolierter Punkt von $X$, so ist $T(X, x) = \emptyset$.
		\item
			Auch für Häufungspunkte $x$ einer Menge $X$ kann $T(X, x) = \emptyset$ gelten (Übung).
	\end{enumerate}
\end{ex}

\begin{st} \label{3.15}
	Sei $f: \R^n \to \R$ stetig differenzierbar und $X \subset \R^n$ eine beliebige Menge.
	In jedem lokalen Minimum $\hat x \in X$ von $f$ in $X$ gilt
	\[
		\nabla f(\hat x)^T d \ge 0
	\]
	für alle $d \in T(X, \hat x)$.
	\begin{proof}
		Sei \oBdA $\|d\| = 1$.
		Sei $(x^k)_{k\in\N} \subset X$ mit $x^k \to \hat x, x^k \neq \hat x, \lim_{k\to \infty} \f{x^k-\hat x}{\|x^k - \hat x\|} = d$.

		Für hinreichend große $k$ gilt
		\[
			f(x^k) \ge f(\hat x)
		\]
		und damit
		\begin{align*}
			0 \le \f {f(x^k) - f(\hat x)}{\|x^k - \hat x\|}
			&= \f{\nabla f(\hat x)^T (x^k - \hat x) + o(x^k - \hat x)}{\|x^k - \hat x\|} \\
			&\to \nabla f(\hat x)^T d.
		\end{align*}
	\end{proof}
\end{st}

\paragraph{Der linearisierte Tangentialkegel}

Betrachte wieder
\[
	f(x) \to \min! \udN g(x) \le 0, h(x) = 0.
\]

\begin{df} \label{3.16}
	Zu $x \in x$ definieren wir den linearisierten Tangentialkegel durch
	\[
		T_l(g,h,x) := \{d \in \R^n : \nabla h_i(x)^T d = 0, \nabla g_i(x)^T d \le 0 \forall i \in \{1,\dotsc,p\}, j \not\in I_x \}
	\]
\end{df}

\begin{lem} \label{3.17}
	Es gilt $T(X, x) \subsetneq T_l(g,h,x)$.
\coursetimestamp{22}{01}{2014}
	\begin{proof}
		Es gilt offenbar $\lambda T_l(g,h,x) \subset T_l(g,h,x)$ für alle $\lambda \ge 0$, also ist $T_l(g,h,x)$ ein Kegel.
		Sei $d \in T(X,x)$ und \oBdA (da $T_l(g,h,x)$ Kegel) $\|d\| = 1$.
		Sei $(x^k)_{k\in \N} \subset X$ eine Folge mit
		\begin{align*}
			x^k &\to x, &
			x^k &\neq x, &
			\lim_{k\to\infty} \f{x^k - x}{\|x^k-x\|} &= d.
		\end{align*}
		Es gilt
		\[
			0
			= \f{h_i(x^k) - h_i(x)}{\|x^k - x\|}
			= \f{\nabla h_i(x)^T(x^k-x) + \rho_{h_i}(x^k-x)}{\|x^k-x\|}
			\to \nabla h_i(x)^T d
		\]
		für alle $i = 1, \dotsc, p$.
		\[
			0
			\ge \f{g_j(x^k) - g_j(x)}{\|x^k-x\|}
			\to \nabla g_j(x)^T d
			\qquad j \not\in I_x
		\]
		und damit $d \in T_l(g,h,x)$.

		Für das Gegenbeispiel betrachte \ref{3.18}.
	\end{proof}
\end{lem}

\begin{ex} \label{3.18}
	Betrachte
	\begin{align*}
		X &:= \Set{ x \in \R^2 | -1 \le x_1 \le 1, x_2 = 0 } \\
		&:= \Set{ x \in \R^2 | (x_1 + 1)^3 \ge x_2, x_1 \le 1, x_2 = 0 }.
	\end{align*}
	$X$ lässt sich beschreiben durch
	\[
		g(x) = \begin{pmatrix}
			- x_1 - 1
			x_1 - 1
		\end{pmatrix},
		h(x) = x_2,
	\]
	aber auch durch
	\[
		\tilde g(x) = \begin{pmatrix}
			x_2 - (x_1 + 1)^3 \\
			x_1 - 1
		\end{pmatrix},
		\tilde h(x) = x_2.
	\]
	Für $x \in (-1, 0)^T \in X$ gilt
	\[
		T(X, x) = T_l(g,h,x) \subsetneq T_l(\tilde g, \tilde h, x)
	\]
	\begin{proof}
		Siehe Übung
	\end{proof}
\end{ex}

\paragraph{Constraint Qualifications}

\begin{df} \label{3.19}
	Die Bedingung
	\begin{equation*} \label{acq} \tag{ACQ}
		T_l(g,h,x) = T(X,x)
	\end{equation*}
	heißt \emph{Abadic Constraint Qualification (ACQ)} für $x \in X$.

	Gilt (ACQ) in einem lokalen Minimum $\hat x \in X$, so ist nach \ref{3.15}
	\[
		\nabla f(\hat x)^T d \ge 0
	\]
	für alle $d \in T_l(g,h,x)$.
\end{df}

\begin{df} \label{3.20}
	\begin{enumerate}[(a)]
		\item
			Zu einem Kegel $K \neq \emptyset$ definieren wir den \emph{Polarkegel} durch
			\[
				K^0 := \Set{ v \in \R^n | \forall d \in K : v^T d \le 0 }.
			\]
		\item
			Die Bedingung
			\begin{equation} \label{gcq} \tag{GCQ}
				T_l(g,h,x)^0 = T(X,x)^0
			\end{equation}
			heißt Guignard Constraint Qualification (GCQ).
		\item
			Jede Bedingung, die (GCQ) impliziert, heißt \emph{Constraint Qualification (CQ)}.
	\end{enumerate}
\end{df}

\begin{ex} \label{3.21}
	\begin{enumerate}[(a)]
		\item
			Es gilt $\eqref{acq} \implies \eqref{gcq}$, d.h \eqref{acq} ist eine CQ.
		\item
			Die Forderung
			\[
				\text{$g_i$ konkav für alle $i \not\in I_x$ und $h$ affin linear}
			\]
			ist eine CQ.
		\item
			Ein Punkt $x \in X$ heißt \emph{regulär}, wenn die Vektoren
			\[
				\Big\{ \nabla h_i(x), \nabla g_j(x) : i \in \{1, \dotsc, p\}, j \not\in I_x \Big\} \subset \R^n
			\]
			linear unabhängig sind.

			„$x$ ist regulär“ ist eine CQ.
	\end{enumerate}
	\begin{proof}
		\begin{enumerate}[(a)]
			\item
				klar
			\item
				siehe Übung
			\item
				siehe Ulbrich, 16.2 % fixme: ref
		\end{enumerate}
	\end{proof}
\end{ex}

\begin{st} \label{3.22}
	In jedem lokalen Minimum $\hat x \in X$ von $f$ in $X$, das eine CQ erfüllt, gilt
	\[
		\nabla f(\hat x)^T d \ge 0
	\]
	für alle $d \in T_l(g,h,\hat x)$.
	\begin{proof}
		Nach \ref{3.15} gilt $-\nabla f(\hat x)^T d \le 0$ für alle $d \in T(X, \hat x)$.
		Dies ist äquivalent zu
		\[
			-\nabla f(\hat x) \in T(X,\hat x)^0 = T_l (g,h,\hat x)^0
		\]
		und äquivalent zu $\nabla f(\hat x)^T d \ge 0$ für alle $d \in T_l(g, h, \hat x)$.
	\end{proof}
\end{st}

\subsection{Karush-Kuhn-Tucker Bedingungen}


\begin{lem}[Trennungssatz von Hahn-Banach] \label{3.23}
	Sei $\emptyset \neq K \subset \R^n$ abgeschlossen und konvex und sei $x \in K$.
	Dann existieren $\nu \in \R^n$ und $r \in \R$ mit
	\begin{align*}
		\nu^T x &> r, &
		\forall y \in K : \nu^T y \le r.
	\end{align*}
	Ist $K$ ein abgeschlossener, konvexer Kegel, dann gilt die Aussage mit $r = 0$.
	\begin{proof}
		Es existiert $\eta \in K$ mit $\|x - \eta\| = \inf_{y\in K} \|x-y\| > 0$.

		Wähle $\nu := x - \eta$ und $r := \nu^T \eta$.
		Dann ist
		\[
			\nu^T x := \nu^T (x- \eta) + \nu^T \eta = \|\nu\|^2 + r > r.
		\]
		Zeige für $y \in K$ noch $\nu^T y \le r$.
		Da $K$ konvex, ist $[y, \eta] \subset K$.
		Es gilt
		\[
			\|t(y - \eta) - \nu\|^2
			 = \|ty + (1-t)\eta - x \|^2
			 \ge \|\eta - x\|^2
			= \|\nu\|^2
		\]
		und damit
		\[
			t^2 \|y-\eta\|^2 - 2t(y-\eta)^T \nu \ge 0.
		\]
		Für $t \to 0$ folgt $(y-\eta)^T \nu \le 0$, also $y^T \nu \le r$.
	\end{proof}
\end{lem}





\printindex[lectures]
\printindex[terms]
%\printbibliography


\end{document}
