\chapter{Überlagerungen topologischer Räume}


\section{Überlagerungen}


\begin{df}
    Seien $X, \tilde X$ topologische Räume, $p: \tilde X \surto X$ stetig.
    Eine offene Menge $U \subset X$ wird durch $p$ \emphdef{trivial überlagert}, wenn $p^{-1}(U) = \bigsqcup \tilde U_i$ disjunkte Vereinigung offener Mengen $\tilde U_i$ in $\tilde X$ ist, sodass für $i \in I$ die Einschränkung $p_i := p|_{\tilde U_i}^U$ ein Homöomorphismus ist.

    Die Abbildung $p: \tilde X \to X$ heißt \emphdef{Überlagerung}, wenn jeder Punkt $x \in X$ eine offene Umgebung $U \subset X$ besitzt, die trivial durch $p$ überlagert wird.
\end{df}

\begin{ex}
    \begin{itemize}
        \item
            $\exp: (\C, +) \to (\C^*, \cdot)$
        \item
            $p: (\R, +) \to (\S^1, \cdot)$, $p(t) = e^{it}$.
        \item
            $p_n: (\C^*, \cdot) \to (\C^*, \cdot)$, $p_n(z) = z^n$.
            Alternativ auch in $\S^1$.
    \end{itemize}
\end{ex}

\section{Hochhebung von Wegen und Homotopien}

%Folgende Situation:
%\begin{math}
%    \begin{tikzcd}
%        & \tilde x \ar[d,"p"] \\
%        w \ar[ru,"gh"] & x \ar[r,"f"]
%    \end{tikzcd}
%\end{math}
%also $f = p \circ g = p \circ h$.

\begin{prop}
    Seien $f: W \to X$, $g,h: W \to \tilde X$ stetig mit $f = p \circ g = p \circ h$.
    Dann bilden
    \begin{math}
        A &= \Set{w \in W & g(w) = h(w)}, \\
        B &= \Set{w \in W & g(w) \neq h(w)} =: A^C
    \end{math}
    eine offene Zerlegung $W = A \sqcup B$.
    \begin{proof}
        Zu $w \in W$ und $x \in f(w) \in X$ sei $U \subset X$ trivial überlagert.
        Wegen $pg(w) = ph(w) = x$ existiert $j, k \in I$, sodass
        \begin{math}
            g(w) \in \tilde U_j, h(w) \in \tilde U_k.
        \end{math}
        Dann ist $V = g^{-1}(\tilde U_j) \cap h^{-1}(\tilde U_k) \subset W$ offene Umgebung von $w$.
        Es gilt $g|_V = p_j^{-1} \circ f|_V$ und $h|_V = p_k^{-1} \circ f|_V$.
        Aus $w \in A$ folgt $j = k$, somit $g|_V = h|_V$, also $V \subset A$, also $A$ offen.
        Aus $w \in B$ folgt $j \neq k$, somit $g(V) \subset \tilde U_j$ und $h(V) \subset \tilde U_k$ disjunkt, also $V \subset B$.
    \end{proof}
\end{prop}

\begin{kor}
    Sei $p: (\tilde X, \tilde{x_0}) \to (X, x_0)$ eine Überlegung.
    Zu jeder stetigen Abbildung $f: ([0,1]^n, 0) \to (X, x_0)$ existiert genau eine Hochhebung $\tilde f: ([0,1]^n, 0) \to (\tilde X, \tilde x_0)$ stetig mit $p \circ \tilde f = f$.
    \begin{proof}
        Skizze: Lebesgue-Zahl und Hochhebung auf den kleinen Quadraten.
    \end{proof}
\end{kor}

\begin{note}
    Hochhebungen existieren nicht immer!
    Betrachte die Überlagerung $p: (\R, 0) \to (\S^1, 1)$ und die Identität auf $(\S^1, 0)$.
    \begin{proof}
        Angenommen es gäbe $\tilde f$ stetig, $p \circ \tilde f = f$.
        Dann ist $\tilde f \homotopic \ast$, also $f = p \circ \tilde f \homotopic \ast$, ein Widerspruch.
        Ebenso für $f = p_n$, $n \neq 0$.
    \end{proof}
\end{note}

\begin{kor}
    \begin{enumerate}[(1)]
        \item
            Ist $p: \tilde X \to X$ eine Überlagerung und $\tilde X$ wegzusammenhängend, $X$ einfach zusammenhängend, dann ist $p$ ein Homöomorphismus, also eine triviale Überlagerung mit einem Blatt.
        \item
            Ist $X$ lokal wegzusammenhängend und global einfach zusammenhängend, so ist jede Überlagerung $p: \tilde X \to X$ trivial.
            Es existiert ein Homöomorphismus $\tau: X \times F \homto \tilde X$ mit $(p \circ \tau)(x,u) = x$, genannt \emphdef{Trivialisierung} ($F$ diskrete Fasermenge):
            \begin{math}
                \begin{tikzcd}
                    \tilde X \ar[d,"p"] & X \times F \ar[l,"\tau","\homeomorphic"'] \ar[ld,"\mathrm{pr}"] \\
                    X
                \end{tikzcd}
            \end{math}
    \end{enumerate}
    \begin{proof}
        \begin{enumerate}[(1)]
            \item
                Seien $a, b \in \tilde X$ mit $p(a) = p(b) = x_0$.
                Da $\tilde X$ wegzusammenhängend, existiert $\tilde \gamma: [0,1] \to \tilde X$ mit $\gamma(0) = a$, $\gamma(1) = b$.
                Nun ist $\gamma := p(\tilde \gamma) \sim \ast$ mittels einer Homotopie $H$.
                Eindeutige Hochhebung der Homotopie $H$ auf eine Homotopie $\tilde H: \tilde \gamma \sim \ast$ liefert zwangsläufig $a = b$.
            \item
        \end{enumerate}
    \end{proof}
\end{kor}

\subsection{Anwendungen}

Sätze von Gelfond-Mazur und Hopf.
\begin{math}
    \exp(\R, +, 0) &\homto (\R_+, \cdot, 1) \\
    \exp(\C, +, 0) &\homto (\C^*, \cdot, 1) \\
\end{math}
Sei $(A, \cdot, 1)$ eine Banach-Algebra über $\R$.
Die Gruppe $(A^*, \cdot, 1)$ ist offen in $A$, denn für $x \in A$, $|x| < 1$ gilt $(1-x)^{-1} = \sum_{k=0}^\infty x^k$.

Für $x \in A$ konvergiert $\exp(x) = \sum_{k=0}^\infty \frac{x^k}{k!}$.

Für $x,y \in A$ mit $xy = yx$ gilt $\exp(x + y) = \exp(x) \exp(y)$, insbesondere $\exp(x) \exp(-x) = 1$, also $\exp(x)^{-1} = \exp(-x)$.

Für $x \in A$ mit $|x| < 1$ konvergiert
\begin{math}
    \ln(1+x) = \sum_{k=1}^\infty (-1)^{k+1} \frac{x^k}{k!}
\end{math}
Die so definierte Funktion $\ln: B(1,1) \to A$ ist stetig, sogar $C^\infty$.
Es gilt $\exp'(x) = \exp(x)$ und $\ln'(a) = a^{-1}$.

Für alle $a \in B(1,1)$ gilt $\exp(\ln(a)) = a$, denn sei $f(a) = a \exp(-\ln(a))$, dann ist
\begin{math}
    f'(a) = \exp(-\ln(a)) + a \exp(-\ln(a))(-a^{-1})
    = 0,
\end{math}
also $f$ konstant.
Nun ist $f(1) = \exp(0) = 1$ und daher für jedes $a$ auch
\begin{math}
    1 = f(a) = a \exp(-\ln(a)),
\end{math}
also $a = \exp(-\ln(a))^{-1} = \exp(\ln(a))$.

\begin{st}
    Für jede kommutative Banach-Algebra $(A, \cdot, 1)$ gilt
    \begin{enumerate}[(1)]
        \item
            $\exp: (A, +, 0) \to (A^*, \cdot, 1)$ ist ein Gruppenhomomorphismus mit diskrete Kern.
        \item
            Das Bild $\exp(A) = A_1^*$ ist die Komponente der $1$ ein $A^*$.
        \item
            $\exp: (A, +, 0) \to (A_1^*, \cdot, 1)$ ist eine universelle Überlagerung (d.h. $A$ einfach zusammenhängend) und lokaler Diffeomorphismus.
    \end{enumerate}
\end{st}

\begin{st}[Gelfand-Mazur]
    Ist eine Banach-Algebra $(A, \cdot, 1)$ ein Körper, so folgt $\dim_\R(A) \in \Set{1,2}$, also $A \homeomorphic \R$ oder $A \homeomorphic \C$ als Körper.
    \begin{proof}
        Aus $\dim_\R A = 1$, oder $\dim_\R A = 2$ folgt $A \homeomorphic \R$ respektive $A \homeomorphic \C$.

        Angenommen $\dim_\R A \ge 3$ und $A$ ein Körper.
        $A^* = A \setminus \Set{0}$ ist einfach zusammenhängend.
        Somit ist $\exp: A \to A^*$ ein Homöomorphismus.
        Insbesondere ist dann $\exp: (A, +, 0) \homto (A^*, \cdot, 1)$ ein Gruppenisomorphismus.
        Aber $-1 \in A^*$ hat Ordnung 2 und $(A, +, 0)$ als $\R$-Vektorraum hat keine Elemente der Ordnung $2$, ein Widerspruch.
    \end{proof}
\end{st}

Erinnerung:
Für $n \ge 2$ gilt $\S^n \not\homeomorphic \RP^n$.
Für $n = 1$ gilt $\S^1 \homeomorphic \S^1$ mit Hilfe von
\begin{math}
    \begin{tikzcd}
        \S^1 \ar[r,"z \mapsto z^2"] \ar[d] & \S^1 \\
        \RP^1 \ar[ru,"\homeomorphic"].
    \end{tikzcd}
\end{math}

\begin{st}[Hopf]
    Sei $(A, +, \cdot)$ eine endlich-dimensionale $\R$-Algebra mit $1$.
    \begin{enumerate}[1)]
        \item
            Ist $(A, +, \cdot)$ assoziativ, kommutativ und nullteilerfrei, dann ist $(A, +, \cdot)$ ein Körper und somit $(A,+,\cdot) \homeomorphic (K, +, \cdot)$ mit $K \in \Set{\R, \C}$.
        \item
            Ist $(A, +, \cdot)$ nur kommutativ und nullteilerfrei, so gilt das selbe.
    \end{enumerate}
    \begin{proof}
        $\dim_R A \in \Set{1,2}$ ist wieder klar.
        Sei $n \ge 3$, $A = \R^n$, $\cdot: \R^n \to \R^n$ bilinear, stetig.
        Sei $q: \R^n \to \R^n$ mit $q(x) = x^2 = x x$.
        $q$ ist stetig und $q(x) = q(-x)$.
        Ist $A$ kommutativ und nullteilerfrei, so gilt $q(x) = q(y)$ nur für $x = \pm y$, denn:
        \begin{math}
            &x^2 - y^2 = 0 \\
            \stack{\text{komm.}}\iff \quad &(x+y)(x-y) = 0 \\
            \stack{\text{nulltf.}}\iff \quad &x = \pm y
        \end{math}
        Sei $f: \S^{n-1} \to \S^{n-1}$ mit $f(x) = x^2$
        \begin{math}
            \begin{tikzcd}
                \S^1 \ar[r,"f"] \ar[d] & \S^1 \\
                \RP^1 \ar[ru,"h"].
            \end{tikzcd}
        \end{math}
        $h$ ist stetig und injektiv.
        Da $\RP^{n-1}$ und $\S^{n-1}$ zusammenhängend geschlossene Mannigfaltigkeiten der Dimension $n-1$ sind, ist $h$ surjektiv (Invarianz des Gebietes).
        Da $\RP^{n-1}$ kompakt und $\S^{n-1}$ hausdorffsch, ist $h$ ein Homöomorphismus.
        Das ist nur für $n - 1 = 1$ möglich, also $n = 2$, ein Widerspruch.
    \end{proof}
\end{st}


\Timestamp{2015-11-27}

\section{Gruppenoperationen und Galois-Überlagerungen}


\begin{ex}
    \begin{itemize}
        \item
            $\Set{\pm 1} \ops \S^n \to \RP^n$
        \item
            $\Set{\pm 1} \ops F_g^+ \to F_g^-$
    \end{itemize}
\end{ex}

\begin{df}
    Sei $G \times X \to X$ eine (Links-)Operation.
    Für $x \in X$ ist $Gx := \Set{gx & g \in G}$ die \emphdef{Bahn} (oder \emphdef{Orbit}) von $x$ in $G$.
    Wir erhalten $q: X \to G \bs X = \Set{Gx & x \in X}$.
    Bequemer ist die Notation
    \begin{math}
        G \ops{\phi} X \xto[surjective]{q} G \bs X.
    \end{math}
    Allgemeiner:
    \begin{math}
        \begin{tikzcd}
            G \ar[r,"\phi",bend left,yshift=-0.7em] & X \ar[r,"\phi",sur] \ar[d,"q",sur] & Y \\
            & G \bs X \ar[ur,"h",hom].
        \end{tikzcd}
    \end{math}
\end{df}

\begin{ex}
    \begin{itemize}
        \item
            $2\pi \Z \ops{+} \C \surto{\exp} \C^*$,
        \item
            $W_n \ops \C^* \surto{p_n} \C^*$, $p_n(z) = z^n$ und
            \begin{math}
                W_n = \Set{z \in \C & z^n = 1}
                = \Set{e^{2\pi i \frac{k}{n}} & k = 0, \dotsc, n-1}.
            \end{math}
        \item
            $2 \pi \Z \ops{+} \R \surto{p} \S^1$, $p(t) = e^{it}$.
    \end{itemize}
\end{ex}

\begin{df}
    Die Operation $\phi: G \times X \to X$ heißt \emphdef{frei}, wenn $gx \neq x$ für alle $x \in X$, $g \in G \setminus \Set{1}$.

    $\phi$ heißt \emphdef{frei diskontinuierlich} (oder \emphdef{topologisch frei}), wenn zu jedem Punkt $x \in X$ eine offene Umgebung $U \subset X$ existiert, sodass $U \cap gU = \emptyset$ für alle $g \in G \setminus \Set 1$.
    \begin{note}
        Ist $\phi$ frei diskontinuierlich, so ist der Orbit eines jeden Punktes insbesondere diskret.
    \end{note}
\end{df}

\begin{ex}
    \begin{itemize}
        \item
            $Z^2 \ops \R^2$ ist frei diskontinuirlich.
        \item
            $\Q \ops \R$ ist frei, aber nicht frei diskontinuierlich.
    \end{itemize}
\end{ex}

\begin{st}
    Für jede frei diskontinuierliche Operation $G \ops{\phi} \tilde X \xto{q} X$ ist $q$ eine Überlagerung.
    \begin{proof}
        klar.
    \end{proof}
\end{st}

\begin{df}
    Wir nennen $G \ops{\phi} \tilde X \xto{q} X$ eine \emphdef{Galois-Überlagerung}, wenn $\phi$ frei diskontinuierlich ist und $\tilde X$ wegzusammenhängend ist.
    Ist $\tilde X$ sogar einfach zusammenhängend ($\pi_0 = \pi_1 = \Set{*}$), so nennen wir dies eine \emphdef{universelle Überlagerung}.
\end{df}


% 2.4
\section{Kurze exakte Sequenz einer Galois-Überlagerung}


\begin{st}
    Jede Überlagerung $p: (\tilde X, \tilde x_0) \to (X, x_0)$ induziert
    \begin{math}
        \begin{tikzcd}
            P(\tilde X, \tilde x_0) \ar[d,"p_\#",xshift=-0.5em,swap] \ar[r,"\quot"] & \Pi(\tilde X, \tilde x_0) \ar[d,"p_\#",xshift=-0.5em,swap] \\
            P(X, x_0) \ar[u,"p_\b",xshift=0.5em,swap] \ar[r,"\quot"] & \Pi(X, x_0) \ar[u, "p_\b",xshift=0.5em,swap]
        \end{tikzcd},
    \end{math}
    wobei $p_\#(\tilde \gamma) = p \circ \tilde \gamma$ und $p_b(\gamma) = \tilde \gamma$ durch Hochhebung.
    Wir erhalten zueinander inverse Bijektionen.
\end{st}

\begin{kor}
    Der Gruppenhomomorphismus $p_\#: \pi_1(\tilde X, \tilde x_0) \to \pi_1(X, x_0)$ ist injektiv.
\end{kor}

\begin{ex}
    \begin{itemize}
        \item
            Zweifache Überlagerung eines Graphen mit zwei Zykeln durch, $p_\#(t_1) = t^2$, $p_\#(s_2) = s$, $p_\#(s_1) = tst^{-1}$.
            $p_\#: \pi_1(\tilde X, \tilde x_0) \to \pi_1(X, x_0)$ ist injektiv.
        \item
            Wie oben mit unendlich vielen Blättern.
    \end{itemize}
\end{ex}

\begin{df}
    Fasertransport:
    Durch Hochhebung eines Weges $\alpha: [0,1] \to X$ mit $\alpha(0) = x$, $\alpha(1) = y$ wird die Faser $F_x$ kanonisch auf die Faser $F_y$ bijektiv abgebildet.

    Wir erhalten eine Bijektion $F_x \times \Pi(X,x,y) \to F_y$.

    Insbesondere erhalten wir eine Gruppenoperation $F_x \times \pi_1(X, x) \to F_x$
\end{df}

\begin{st}[Kurze exakte Sequenz]
    Für jede Galois-Überlagerung $G \ops{\phi} \tilde X \xto{q} X$ gilt
    \begin{enumerate}[(1)]
        \item
            Die Operation von $G$ auf $\tilde X$ kommutiert mit dem Fasertransport:
            \begin{math}
                (g \cdot \tilde x) \cdot [\alpha] = g \cdot (\tilde x \cdot [\alpha])
            \end{math}
        \item
            Für $\tilde x_0 \in \tilde X$ und $x_0 = q(\tilde x_0)$ erhalten wir einen Gruppenhomomorphismus $h: \pi_1(X, x_0) \to G$
            \begin{math}
                \tilde x_0 \cdot [\alpha] = h([\alpha]) \cdot \tilde x_0.
            \end{math}
        \item
            Sein Kern ist $\ker(h) = p_\#(\pi_1(\tilde X, \tilde x_0))$ und sein Bild ist
            \begin{math}
                \im(h) = \Set{g \in G & \text{$x_0$ und $g \tilde x_0$ sind wegverbindbar}}
            \end{math}
    \end{enumerate}
    Ist $\tilde X$ wegzusammenhängend (also $G \ops \tilde X \to X$ eine Galois-Überlagerung), so erhatlen wir also eine kurze exakte Sequenz
    \begin{math}
        1 \to \pi_1(\tilde X, \tilde x_0) \xto{p_\#} \pi_1(X, x_0) \xto{h} G \to 1.
    \end{math}
    \begin{proof}
        \begin{enumerate}[(1)]
            \item
                Klar dank Skizze.
            \item
                Existenz da transitiv, Eindeutigkeit da frei.
                Gruppenhomomorphismus:
                \begin{math}
                    h([\alpha][\beta])\cdot \tilde x_0
                    &= \tilde x_0 \cdot ([\alpha][\beta]) \\
                    &= (\tilde x_0 \cdot [\alpha]) \cdot [\beta] \\
                    &= (h([\alpha])\cdot \tilde x_0) \cdot [\beta] \\
                    &= h([\alpha]) \cdot (\tilde x_0 \cdot [\beta]) \\
                    &= h([\alpha]) \cdot (h([\beta]) \cdot \tilde x_0) \\
                    &= [h([\alpha])\cdot h([\beta])] \cdot \tilde x_0.
                \end{math}
            \item
                leicht nachvollziehbar.
        \end{enumerate}
    \end{proof}
\end{st}

\begin{ex}
    \begin{itemize}
        \item
            $\Z \ops{+} (\R, 0) \xto{p} (\S^1, 1)$, $p(t) = e^{2 \pi i t}$.

            \begin{math}
                \begin{tikzcd}
                    \pi_1(\R, 0) \ar[r,inj,"p_\#"] \ar[d,"\homeomorphic"] & \pi_1(\S^1, 1) \ar[r,sur,"h"] \ar[d,"\homeomorphic"] & \Z \ar[d,"\homeomorphic"] \\
                    0 \ar[r,inj] & \Z \ar[r,bij] & \Z
                \end{tikzcd}
            \end{math}
        \item
            $W_n \ops{\cdot} (\S^1, 1) \xto{p_n} (\S^1, 1)$, $p_n(z) = z^n$
            \begin{math}
                \begin{tikzcd}
                    \pi_1(\S^1, 1) \ar[r,inj,"(p_n)_\#"] \ar[d,"\homeomorphic"] & \pi_1(\S^1, 1) \ar[r,sur,"h"] \ar[d,"\homeomorphic"] & W_n \ar[d,"\homeomorphic"] \\
                    \Z \ar[r,"\cdot n"] & \Z \ar[r,"\quot"] & \Z / n
                \end{tikzcd}
            \end{math}
        \item
            $\Set{\pm 1} \ops (\S^n, \ast) \xto (\RP^n, [\ast])$ für $n \ge 2$:
            \begin{math}
                \begin{tikzcd}
                    \pi_1(\S^n, \ast) \ar[r,inj,"p_\#"] \ar[d,"\homeomorphic"] & \pi_1(\RP^n, [\ast]) \ar[r,sur,"h"] \ar[d,"\homeomorphic"] & \Set{\pm 1} \ar[d,"\homeomorphic"] \\
                    0 \ar[r,inj] & \Z / 2 \ar[r,bij] & \Z / 2
                \end{tikzcd}
            \end{math}
    \end{itemize}
\end{ex}


% 2.5
\section{Galois-Korrespondenz}

\begin{ex}
    Überlagerungen von $\S^1$: $p_n: \S^1 \to \S^1$, $z \mapsto z^n$ für $n = 1, 2, \dotsc$.
    $p_0 : \R \to \S^1$, $p_0(t) = e^{2\pi i t}$.

    Hier gilt
    \begin{math}
        n \Z \isomorphic (p_n)_\# (\pi_1(\tilde X, \tilde x_0)) \subset \pi_1(X, x_0) \isomorphic \Z
    \end{math}
\end{ex}

\begin{st}
    Sei $X$ zusammenhängend und lokal einfach zusammenhängend.
    \begin{enumerate}[(1)]
        \item
            Jede Überlagerung $p: (Y, y_0) \to (X, x_0)$ definiert $p_\#: \pi_1(Y, y_0) \to \pi_1(X, x_0)$ injektiv und somit
            $U = \im(p_\#) \le \pi_1(X, x_0)$.
        \item
            Zu jeder Untergruppe $U \le \pi_1(X, x_0)$ existiert eine zusammenhängende Überlagerung $p: (Y, y_0) \to (X, x_0)$ mit $\im(p_\#) = U$.
            Diese ist eindeutig bis auf Isomorphie von Überlagerungen.
    \end{enumerate}
    \begin{note}
        Analogie zu Körpererweiterungen.
    \end{note}
\end{st}

\subsection{Galois-Korrespondenz für Graphen}

Sei $K$ ein simplizialer Graph, $x_0 \in \Omega$ eine Ecke, $T \subset K$ ein Spannbaum.
Dann $\pi_1(K, x_0) \xto[homeomorphic] G$ mit
\begin{math}
    G = \Gen{s_{ab}: \Set{a,b} \in K \setminus T & s_{ab}s_{ba} = 1 : \Set{a,b} \in K \setminus T}.
\end{math}
Sei $F$ eine Menge und $\phi: F \times \pi_1(K, x_0) \to F$ eine Operation, d.h. jede Kante $\Set{a,b} \in K \setminus T$ definiert eine Permutation $F \to F$, $u \mapsto u \cdot s_{ab}$, bzw. invers $u \mapsto u \cdot s_{ba}$.
Implizit: Für $\Set{a,b} \in T$ gilt $s_{ab} = 1$, $u s_{ab} = u$.

\begin{st}
    Die Daten $(K,F,\phi)$ definieren auf der Eckenmenge $\tilde \Omega = F \times \Omega$ den Graphen
    \begin{math}
        \tilde K = F \semiprod{\phi} K
        := \Set{\Set{(u,a), (v,b)} & u,v \in F, \Set{a,b} \in K, u \cdot s_{ab} = v}
    \end{math}
    Die Projektion $p: \tilde K \to K$, $p(u, a) = a$ hat eine simpliziaile Überlagerung mit Faser $F$, genauer $p^{-1}(x_0) = F \times \Set{x_0}$ und Monodromie $\phi: F \times \pi_1(K, x_0) \to F$ (Fasertransport über $x_0$).
    \begin{note}
        Über $T \subset K$ liegt $F \times T \subset \tilde K$.
    \end{note}
    \begin{proof}
        Nachrechnen.
    \end{proof}
\end{st}

\begin{kor}
    Sei $H \le G := \pi_1(K, x_0)$.
    Auf $F = H \bs G$ operiert $G$ durch Rechtsmultiplikation, $\phi: F \times G \to F$, $(Ha, g) \mapsto Hag$.

    Der Graph $\tilde K = F \semiprod{\phi} K$ ist zusammenhängend, da $G$ transitiv auf $F$ operiert.
    Die Überlagerung $p: (\tilde K, \tilde x_0) \to (K, x_0)$ mit $\tilde x_0 = (H, x_0)$ indzuiert $p_\#: \pi_1(\tilde K, \tilde x_0) \to \pi_1(K, x_0)$ mit $\im(p_\#) = H$.

    Speziell für $H = \Set{1} \le G$ erhalten wir die universelle Überlagerung von $(\tilde K, \tilde x_0) \to (K, x_0)$.
    \begin{proof}
        Nachrechnen.
    \end{proof}
\end{kor}

Eine wichtige Anwendung ist folgende.

\begin{st}[Nielsen-Schreier]
    In jeder freien Gruppe $F$ ist jede Untergruppe $G \le F$ frei.
    Hat $F$ endlichen Rang $\rang(F) < \infty$ und $G$ in $F$ endlichen Index $[F:G] < \infty$, so gilt
    \begin{math}
        \rang(G) - 1 = (\rang(F) - 1) [F: G].
    \end{math}
    \begin{proof}
        Zu $F = \Gen{S & -}$ existiert ein Graph $K$ mit $\pi_1(K, x_0) = F$.
        Zu $G \le F$ existiert eine zusammenhängende Überlagerung $p: (\tilde K, \tilde x_0) \to (K, x_0)$ mit $\im(p_\#) = G$.
        Nach Konstruktion des vorigen Satzes kann $\tilde K$ als Graph gewählt werden, also ist $\pi_1(\tilde K, \tilde x_0) \homeomorphic G$ frei.

        Ist $\rang(F) = |S| < \infty$, so ist wählen wir $K$ endlich.
        Es gilt $\rang(F) = 1 - \chi(K)$.
        Die Blätterzahl von $p$ ist $n = [F:G]$.
        Falls $n < \infty$, so gilt $\chi(\tilde K) = n \chi(K)$.
        Es ergibt sich
        \begin{math}
            1 - \rang(G) = \chi(\tilde K) = n \chi(K) = n(1 - \rang(F)).
        \end{math}
    \end{proof}
\end{st}


\subsection{Galois-Korrespondenz für Simplizialkomplexe}


Sei $K$ ein zusammenhängender Simplizialkomplex, $x_0 \in \Omega$ eine Ecke, $T \subset K$ ein Spannbaum.
Wir haben $\pi_1(K, x_0) = \Gen{S & R}$.
\begin{math}
    S &= \Set{s_{ab} & \Set{a,b} \in K}, \\
    R &= \Set{s_{ab} & \Set{a,b} \in T} \cup \Set{s_{ab}s_{ba}, s_{ab}s_{bc}s_{ca} & \Set{a,b,c} \in K }
\end{math}
Sei $F$ ein Menge, $\phi: F \times \pi_1(K, x_0) \to F$ eine Operation, d.h. jede Kante $\Set{a,b}$ operiert auf $F$ gemäß $u \mapsto u s_{ab}$, invers mit $u \mapsto u s_{ba} = u s_{ab}^{-1}$ und für jedes Dreieck $\Set{a,b,c} \in K$ gilt $u s_{ab} s_{bc} s_{ca} = u$.

\begin{st}
    Die Daten $(K, F, \phi)$ definieren auf der Eckenmenge $\tilde \Omega = F \times \Omega$ den Simplizialkomplex
    \begin{math}
        K = F \twprod{\phi} K
        := \Set{ \Set{(u_0,a_0), \dotsc, (u_n,a_n)} & \begin{aligned} u_0, \dotsc, u_n \in F, \\ \Set{a_0, \dotsc, a_n} \in K, \\ u_i \cdot s_{a_ia_j} = u_j \end{aligned} }.
    \end{math}
    Die Projektion $p: \tilde K \to K$ ist eine simpliziale Überlagerung mit Faser $F$, $F \times \Set{x_0} = p^{-1}(x_0)$ und Monodromie $\phi: F \times \pi_1(K, x_0) \to F$.
    \begin{proof}
        Nachrechnen.
    \end{proof}
\end{st}

\begin{kor}
    Sei $H \le \pi_1(K, x_0)$.
    Auf $F = H \lq G$ operiert $G$ wie oben und wir erhalten $p: (\tilde K, \tilde x_0) \to (K, x_0)$ mit $p_\#(\pi_1(\tilde K, \tilde x_0) = H \le \pi_1(K, x_0)$.
\end{kor}


\Timestamp{2015-12-11}


% §2.6
\section{Hochhebungskriterium}

\begin{math}
    \begin{tikzcd}
        & (\tilde X, \tilde x_0) \ar[d,"p"] \\
        (W, w_0) \ar[r,"f"] \ar[ur,"\tilde f"] & (X, x_0)
    \end{tikzcd}
    \stack{\pi_1}\leadsto
    \begin{tikzcd}
        & \pi_1(\tilde X, \tilde x_0) \ar[d,"p_\#"] \\
        \pi_1(W, w_0) \ar[r,"f_\#"] \ar[ur,"\tilde f_\#"] & \pi_1(X, x_0)
    \end{tikzcd}
\end{math}

Für die Eindeutigkeit der Hochhebung $\tilde f$ genügt, dass $(W, w_0)$ zusammenhängend ist.

\begin{st}
    Sei $p: (\tilde X, \tilde x_0) \to (X, x_0)$ eine Überlagerung.
    Sei $(W, w_0)$ (weg)zusammenhängend und lokal wegzusammenhängend.

    $f: (W, w_0) \to (X, x_0)$ erlaubt genau dann eine eindeutige Hochhebung $\tilde f: (W, w_0) \to (\tilde X, \tilde x_0)$, wenn
    \begin{math}[numbered] \label{eq:hhug}
        f_\# \pi_1(W,w_0) < p_\# \pi_1(\tilde X, \tilde x_0).
    \end{math}
    \begin{note}
        Der Satz lässt sich insbesondere anwenden, falls $\pi_1(W, w_0) = \Set{1}$.
        Auch für $W = [0,1]^n$ (bereits explizit bewiesen, wird hier genutzt).
    \end{note}
    \begin{proof}
        \begin{seg}{\ProofImplication}
            Klar dank Diagramm und Funktorialität: $p \circ \tilde f_\# = f$ wird zu $p_\# \circ \tilde f_\# = f_\#$.
        \end{seg}
        \begin{seg}{\ProofImplication*}
            Wir nehmen \eqref{eq:hhug} an und konstruieren $\tilde f: (W, w_0) \to (\tilde X, \tilde x_0)$:

            Konstruktion:
            Zu $w \in W$ wähle einen Weg $\alpha: [0,1] \to W$ von $\alpha(0) = w_0$ nach $\alpha(1) = w$.
            Sei $\beta := f\circ \alpha$ und $\tilde \beta$ die Hochhebung.
            Setze $\tilde f(w) := \tilde \beta(1)$.

            Wohldefiniertheit:
            Sei $\alpha'$ ein weiterer Weg von $w_0$ nach $w$ und $\beta' := f \circ \alpha'$, $\tilde \beta'$ die Hochhebung, $\tilde x' = \tilde \beta'(1)$.
            Warum gilt $\tilde x' = \tilde x$?
            Dies gilt gilt dank \eqref{eq:hhug}:
            Wir betrachten die Schleife $\gamma := \beta' \ast \_\beta = f \circ (\alpha' \ast \_\alpha)$.
            Es gilt $[\gamma] = f_\#([\alpha' \ast \_\alpha]) \in f_\# \pi_1(W, w_0) < p_\# \pi_1(\tilde X, \tilde x_0)$.
            Die Hochhebung $\tilde \gamma$ ist geschlossen, d.h. $\tilde \gamma(1) = \tilde \gamma(0) = \tilde x_0$.
            Für $0 \le t \le \frac {1}{2}$ gilt
            \begin{math}
                p(\tilde \gamma(t)) &= \beta'(2t), &
                p(\tilde \gamma(1-t)) &= \beta(2t)
            \end{math}
            Dank Eindeutigkeit gilt $\tilde \gamma(t) = \tilde \beta'(2t)$ und $\tilde \gamma(1-t) = \tilde \beta(2t)$.
            Somit ist $\tilde \gamma = \tilde \beta' \ast \_{\tilde \beta}$.
            Insbesondere gilt $\tilde \gamma(\frac{1}{2}) = \tilde \beta'(1) = \tilde \beta(1)$.

            Stetigkeit:
            Die so definierte Abbildung $\tilde f$ ist stetig dank des lokalen Wegzusammenhangs von $W$:
            Sei $w \in W$ und $\tilde x = \tilde f(w)$.
            Für jede offene Umgebung $\tilde U$ von $\tilde x$ in $\tilde X$ haben wir eine offene Umgebung $V$ von $w$ in $W$ zufinden mit $\tilde f(V) \subset \tilde U$.
            Da $p$ eine Überlagerung ist, können wir $\tilde U$ so klein wählen, dass $p|_{\tilde U}^U$ ein Homöomorphismus ist auf sein Bild $U := p(\tilde U)$.
            Da $f$ stetig ist, existiert $V \subset W$ offen, $w \in V$ mit $f(V) \subset U$.
            Da $W$ lokal wegzusammenhängend ist, können wir $V$ zudem als wegzusammenhängend annehmen.
            Wie zuvor sei $\alpha$ ein Weg von $w_0$ nach $w$, $\beta = f \circ \alpha$, $\tilde \beta$ die Hochhebung.
            Zu jedem Punkt $w' \in V$ existiert ein Weg $\alpha': [0,1] \to V$ von $\alpha(0) = w$ nach $\alpha(1) = w'$.
            Sein Bild $\gamma = f \circ (\alpha \ast \alpha') = (f \circ \alpha) \ast (f \circ \alpha') = \beta \ast \beta'$ verläuft von $x_0$ nach $x$ in $X$ und dann nach $x'$ in $f(V) \subset U$.
            Der Homöomorphismus $q := p|_{\tilde U}^U: \tilde U \to U$ liefert die Hochhebung $\tilde \beta' = q^{-1} \circ \beta'$.
            Somit ist $\tilde \gamma := \tilde \beta \ast \tilde \beta'$ eine Hochhebung von $\gamma$.
            Nach Konstruktion gilt $\tilde f(w') \subset \tilde \gamma(1) = \tilde \beta '(1) = q^{-1} \circ \beta'(1) \in q^{-1} \circ f(V) \subset \tilde U$, kurz: $\tilde f(V) \subset \tilde U$.
        \end{seg}
    \end{proof}
    \begin{nt}
        Der lokale Wegzusammenhang von $W$ ist wesentlich für die Aussage des Satzes.
        Betrachte als Gegenbeispiel den Warschauer Kreis:
        \begin{math}
            W = \Set{e^{-it} \big(2 + \sin(\frac{t}{2\pi - t})\big) & 0 \le t < 2\pi}.
        \end{math}
        $W$ ist wegzusammenhängend, aber nicht lokal wegzusammenhängend.
        Setze $w_0 := 2$.
        Es gilt $\pi_0(W) = \Set{W}$, $\pi_1(W, w_0) = \Set{1}$.

        Die Abbildung $f: W \to \S^1$, $f(w) = \frac{w}{|w|}$ ist stetig und surjektiv, aber kein Homöomorphismus (z.B. über Kompaktheit oder Fundamentalgruppe).
        Zu $p: \R \to \S^1$, $p(t) = e^{it}$ existiert zu $f$ keine Hochhebung $\tilde f$.
        \begin{math}
            \begin{tikzcd}
                & (\R, 0) \ar[d,"p"] \\
                (W, w_0) \ar[r,"f"] \ar[ru,"\tilde f"] & (\S^1, 1)
            \end{tikzcd}
        \end{math}
    \end{nt}
\end{st}

\begin{st}
    Sei $n \ge 2$.
    Jede stetige Abbildung $f: \S^n \to \S^1$ ist zusammenziehbar.
    \begin{proof}
        \begin{math}
            \begin{tikzcd}
                & (\R, 0) \ar[d,"p"] \\
                (\S^n,\ast) \ar[r,"f"] \ar[ru,"\exists ! \tilde f",dashed] & (\S^1, 1)
            \end{tikzcd}
        \end{math}
        Nun ist $\tilde f \homotopic \ast$, also $f = p \ast \tilde f \homotopic \ast$.
    \end{proof}
\end{st}

Allgemeiner:

\begin{st}
    Sei $X$ ein topologischer Raum mit zusammenziehbarer (universeller) Überlagerung $p: \tilde X \to X$.
    Sei $Y$ ein einfach zusammenhängender und lokal wegzusammenhängend.
    Dann ist jede stetige Abbildung $f: Y \to X$ zusammenziehbar.
    \begin{nt}
        Für die Wahl von $X$ gibt es viele Möglichkeiten
        \begin{itemize}
            \item
                $X$ ein Graph, dann ist $\tilde X$ ein Baum.
            \item
                $X \homeomorphic \S^1 \times \S^1$, dann ist $\tilde X \homeomorphic \R^2$ die euklidische Ebene.
            \item
                $X \homeomorphic F_g^\pm$ mit $g \ge 1$ (für $g \ge 2$ ist $\tilde X \homeomorphic \H^2 \homeomorphic \R^2$ die hyperbolische Ebene).
            \item
                $X = G \bs \tilde X$ mit $\tilde X \homotopic \ast$, $G \ops \tilde X \to X$.
        \end{itemize}
    \end{nt}
    \begin{proof}
        \begin{math}
            \begin{tikzcd}
                & \tilde X \ar[d,"p"] \\
                (Y, y_0) \ar[r,"f"] \ar[ur,"\exists! \tilde f",dashed] & (X, x_0)
            \end{tikzcd}
        \end{math}
        Es gilt $X \homotopic \ast$, also $\tilde X \homotopic \ast$, also $f = p \circ \tilde f \homotopic \ast$.
    \end{proof}
\end{st}

Morphismen von Überlagerungen:
\begin{math}
    \begin{tikzcd}
        (Y,y_0) \ar[rd,"p"] \ar[rr,"f",shift left] & & (Z,z_0) \ar[ll,"g", shift right] \ar[ld,"q"] \\
        & (X, x_0) &
    \end{tikzcd}
\end{math}

\begin{st}
    Sei $X$ wegzusammenhängend und lokal wegzusammenhängend.
    Seien $p: (Y, y_0) \to (X, x_0)$ und $q: (Z, z_0) \to (X, x_0)$ zusammenhängende Überlagerungen.
    (Insbesondere sind $Y$ und $Z$ lokal weszusammenhängend und somit wegzusammenhängend)
    \begin{enumerate}[1)]
        \item
            Es existiert genau dann ein Morphismus $f: p \to q$, d.h. $f: (Y, y_0) \to (Z, z_0)$ mit $q \circ f = p$, wenn $p_\# \pi_1(Y,y_0) < q_\# \pi_1(Z,z_0)$.
        \item
            Es existiert ein Isomorphismus genau dann, wenn Gleichheit gilt.
    \end{enumerate}
    \begin{proof}
        \begin{enumerate}[1)]
            \item
                Klar, dank Hochhebungskriterium.
            \item
                Es gelte $p_\# \pi_1(Y,y_0) = q_\# \pi_1(Z, z_0)$.
                Wir heben $p$ zu $f$ und $q$ zu $g$ hoch dank Hochhebungskriterium.
                Es gilt $p \circ (g \circ f) = (p \circ g) \circ f = q \circ f = p$, d.h. $g \circ f$ und $\id_Y$ sind Hochhebungen von $p$ über $p$, also gilt $g \circ f = \id_Y$ nach Eindeutigkeit.
                Ebenso $f \circ g = \id_Z$.
        \end{enumerate}
    \end{proof}
\end{st}


\section{Decktransformationen}

\begin{df}
    Sei $p: \tilde X \to X$ eine Überlagerung.
    Wir nennen $G = \Aut(p: \tilde X \to X) = \Aut(p) = \Aut(\tilde X, X) = \Aut(\tilde X)$
    \begin{math}
        G := \Set{g: \tilde X \xto[homeomorphic] \tilde X & p \circ g = p }
    \end{math}
    die \emphdef{Automorphismengruppe}, \emphdef{Decktransformationsgruppe} oder \emphdef{Galois-Gruppe}.
    \begin{math}
        \begin{tikzcd}
            \tilde X \ar[rr,"g"] \ar[rd,"p"] && \tilde X \ar[ld,"p"] \\
            & X &
        \end{tikzcd}
    \end{math}
    Insbesondere erhält $g$ jede Faser $p^{-1}(\Set{x})$ und permutiert die Punkte in $p^{-1}(\Set{x})$.
\end{df}

\begin{ex}
    \begin{itemize}
        \item
            $p: \R \to \S^1$, $p(t) = e^{2\pi i t}$.
            Hier ist $\Aut(p) = \Z$, da $\Z \ops \R \to \S^1$ ($\supset$ klar, $\subset$ dank Eindeutigkeit).
    \end{itemize}
\end{ex}

\begin{st}
    Sei $\tilde X$ wegzusammenhängend, $G < \Homeo(\tilde X)$ operiere topologisch frei auf $\tilde X$ und $G \ops \tilde X \xto{p} X$.
    Dann gilt $\Aut(p) = G$.
    \begin{proof}
        $\supset$ klar, $\subset$ Eindeutigkeit der Hochhebung.
    \end{proof}
\end{st}

\begin{st}
    Sei $p: \tilde X \to X$ eine wegzusammenhängend Überlagerung (d.h. $\tilde X$ wegzusammenhängend, insbesondere $X$).
    \begin{enumerate}[1)]
        \item
            Die Gruppe $G = \Aut(p)$ operiert topologisch frei auf $\tilde X$.
        \item
            Operiert $\Aut(p)$ transitiv auf (einer und damit auf) jeder Faser, so ist $p: \tilde X \to X$ äquivalent zum Quotienten $q: \tilde X \to G \bs \tilde X$, kurz:
            \begin{math}
                \begin{tikzcd}
                    G \ar[r,ops] & \tilde X \ar[r,"p"] \ar[d,"q"'] & X \\
                    & G \bs \tilde X \ar[ru,"h"',"\homeomorphic"]
                \end{tikzcd}
            \end{math}
            D.h. wir haben eine Galois-Überlagerung.
    \end{enumerate}
    \begin{proof}
        Übung: 1) Definitionen nachprüfen, 2) Bahn ist die ganze Faser, betrachte dann lokal.
    \end{proof}
\end{st}

\begin{df}
    Eine Überlagerung $p: \tilde X \to X$ heißt \emphdef{normal} oder \emphdef{galoisch}, wenn $\tilde X$ wegzusammenhängend ist und $\Aut(p)$ transitiv auf $\tilde X$ operiert.
\end{df}

\begin{ex}
    \begin{itemize}
        \item
            Dreifache Überlagerung (zyklisch) eines Graphen mit zwei Zykeln: galoisch, $\Aut(p) = \Z / 3$.
        \item
            Wie oben, aber Stockwerke nicht zyklisch verbunden: nicht galoisch, $\Aut(p) = \Set{1}$.
        \item
            Zweiblättrige Überlagerungen sind immer galoisch.
    \end{itemize}
\end{ex}

% Alle Wegzusammenhangs-Bedingungen

\begin{st}
    Seien $\tilde X$ wegzusammenhängend $X$ lokal wegzusammenhängend und $p: (\tilde X, \tilde x_0) \to (X, x_0)$ eine Überlagerung.

    $p$ ist normal genau dann, wenn $K := p_\# \pi_1(\tilde X, \tilde x_0) < \pi_1(X, x_0) =: H$ normal ist.
    \begin{proof}
        Wir wollen Existenz von $g$.
        \begin{math}
            \begin{tikzcd}
                (\tilde X, \tilde x_0) \ar[rr,"\exists! g"] \ar[rd,"p"] & & (\tilde X, \tilde x_1) \ar[ld,"p"] \\
                & (X, x_0) &
            \end{tikzcd}
        \end{math}
        Sei $\gamma: [0,1] \to \tilde X$ ein Weg von $\tilde x_0$ nach $\tilde x_1$.
        Wir haben einen Isomorphismus $\gamma_\#: \pi_1(\tilde X, \tilde x_0) \to \pi_1(\tilde X, \tilde x_1)$ mit $[\alpha] \mapsto [\_\gamma \ast \alpha \ast \gamma]$.

        Für $[p \circ \alpha] \in \pi_1(X, x_0)$ gilt also
        \begin{math}
            [p \circ \gamma_\# \alpha] = [(p \circ \_\gamma)] [(p \circ \alpha)] [(p \circ \gamma)]
            = [p \circ \alpha]^{[p \circ \gamma]}.
        \end{math}
        Demnach sind $p_\# \pi_1(\tilde X, \tilde x_0)$ und $p_\# \pi_1(\tilde X, \tilde x_1)$ in $\pi_1(x, x_0)$ konjugiert durch $[p \circ \gamma] \in \pi_1(X, x_0)$.

        \ProofImplication*:
        Ist $p_\# \pi_1(\tilde X, \tilde x_0)$ normal, dann können wir $p$ hochheben zu $g$.

        \ProofImplication:
        Rückwärts lesen
        % todo: Tilden verteilen.
    \end{proof}
\end{st}
