\chapter{Singuläre Homologie}

Ziel: Für beliebige Paare topologischer Räume $(X, A)$ möchten wir einen Funktor $H_*: \Cat{Top}_{(2)} \to \Cat{GrMod}_R$, d.h. $H_q: (X, A) \mapsto H_q(X,A) \in \Cat{Mod}_R$ für $q \in \Z$ und $H_q: (f:(X, A) \to (Y, B)) \mapsto (H_q(f): H_q(X,A) \to H_q(Y,B))$.

Wünschenswerte Eigenschaften:
\begin{enumerate}[(1)]
    \item
        Homotopie-Invarianz: Aus $f \homotopic g: (X, A) \to (Y, B)$ folgt $H_*(f) = H_*(g)$.
    \item
        Lange exakte Sequenz: Für $i: (A, \emptyset) \injto (X, \emptyset)$ und $j: (X, \emptyset) \injto (X, A)$ haben wir eine lange exakte Sequenz
        \begin{math}
            \begin{tikzcd}
                {} \ar[r] & H_q(A,\emptyset) \ar[r,"H_q(i)"] & H_q(X,\emptyset) \ar[r,"H_q(j)"] & H_q(X,A) \ar[r,"\del_q"] & H_{q-1}(A,\emptyset) \ar[r] & \dotsc
            \end{tikzcd}.
        \end{math}
        Hierzu fordern wir eine \emphdef{Randabbildung}
        \begin{math}
            \del_q = \del_q^{(X,A)}: H_q(X, A) \to H_{q-1}(A, \emptyset).
        \end{math}
        Dies soll mit dem Funktor $H_*$ verträglich sein, d.h. für $f: (X, A) \to (Y, B)$ soll folgenden Diagramm kommutieren:
        \begin{math}
            \begin{tikzcd}
                H_q(X, A) \ar[r,"\del_q^{(X,A)}"] \ar[d,"H_q(f)"'] & H_{q-1}(A, \emptyset) \ar[d,"H_{q-1}(f|_{(A, \emptyset)})"] \\
                H_q(Y, B) \ar[r,"\del_q^{(Y,B)}"] & H_{q-1}(B, \emptyset)
            \end{tikzcd}.
        \end{math}
        Dies nennt man eine \emphdef{natürliche Transformation}, $\del_q: H_q \to H_{q-1} \circ S$ mit $S(X, A) = (A, \emptyset)$.
    \item
        Dimensionsaxiom: Für $X = \Set{p}$ gilt
        \begin{math}
            H_q(\Set{p}, \emptyset) \isomorphic \begin{cases}
                R & \text{für $q = 0$}, \\
                0 & \text{sonst}
            \end{cases}
        \end{math}
        Genauer: wir fordern einen natürlichen Isomorphismus.
    \item
        Additivitätsaxiom
        Für jede topologische Summe $X = \bigsqcup_{k \in I} X_k \xto*[inj]{\iota_k} X_k$ erhalten wir einen Isomorphismus
        \begin{math}
            \begin{tikzcd}
                \bigoplus_{k\in I} H_q(\iota_k): \bigoplus_{k\in I} H_q(X_k, \emptyset) \ar[r,"\isomorphic"] & H_q(X, \emptyset) \\
            \end{tikzcd}
        \end{math}
    \item
        Ausschneidungsaxiom:
        Sei $(X, A)$ ein Raumpaar und $U \subset X$ mit $\_U \subset \mathring A$.
        Dann induziert die Inklusion $i: (X \setminus U, A \setminus U) \injto (X, A)$ einen Isomorphismus
        \begin{math}
            H_q(i): H_q(X \setminus U, A \setminus U) \xto{\isomorphic} H_q(X, A).
        \end{math}
\end{enumerate}

\begin{df}[Eilenberg-Steenrod-Milnor-Axiome für Homologie]
    Sei $R$ ein Ring.
    Eine \emphdef{Homologie-Theorie} über $R$ ist ein paar $(H, \del)$ bestehend aus einem Funktor $H: \Cat{Top}_{(2)} \to \Cat{GrMod}_{R}$ und einer natürlichen Transformation $\del: H_q \to H_{q-1} \circ S$, sodass die obigen Axiomen (1)-(5) erfüllt sind.
\end{df}

\begin{ex}
    Beispiele, bzw. Versuche.
    Zur Vereinfachung $R = \Z$, $\Cat{Mod}_\Z = \Cat{AbGrp}$.
    \begin{enumerate}[1)]
        \item
            Triviales Paar $(H, \del)$, $H_q(X,A) = 0$, also $\del_q^{(X,A)} = 0$.
            Dies erfüllt (1), (2), (4), (5), aber nicht (3).
        \item
            Simpliziale Homologie: Erfüllt alles, aber nur auf Simplizialkomplexen $\Cat{SComp}_{(2)} \to \Cat{Top}_{(2)}$.
        \item
            Konstant: $H(X, A) := H(\Set{p}, \ast)$ nach (3).
            \begin{math}
                (f: (X, A) \to (Y, B)) \mapsto H_q(f) := \id.
            \end{math}
            Dies erfüllt (1), (3), (5) aber nicht (2), (4).
            Zu (2): Für $(X, A) = (\Set{p}, \Set{p})$ gilt:
            \begin{math}
                \begin{tikzcd}[row sep=tiny]
                    {} \ar[r] & H_0(A,\emptyset) \ar[r] & H_0(X, \emptyset) \ar[r] & H_0(X, A) \ar[r] & 0 \\
                    {} \ar[r] & R \ar[r] & R \ar[r] & R \ar[r] & R
                \end{tikzcd}
            \end{math}
            Nicht exakt in $H_0(X, \emptyset)$.
        \item
            Für $X$ wegzusammenhängend definiere
            \begin{math} \\
                H(X, \emptyset) &:= H(\Set{p}, \emptyset), \\
                H(X, A) &:= 0 \qquad \text{für $A \neq \emptyset$}.
            \end{math}
            Allgemein: Zerlege $X$ in Wegkomponenten $\pi_0(X)$ und setze $H(X, A) := \bigoplus_{C \in \pi_0(X)} H(C, C \cap A)$.
            Für $f: (X, A) \to (Y, B)$ setze $H_q(f)$ pro Komponente.

            Dies erfüllt (1), (2), (3), (4), aber nicht (5)?
            Übung: Was geht schief in (5)?
        \item
            $(H, \del) = (\pi, \del)$ mit $H_q = \pi_q$ für $q \ge 2$, $H_1 = \pi_1 / \mathrm{ab}$, $H_0 = \Z \pi_0$ und additiv fortgesetzt.
    \end{enumerate}
\end{ex}

\begin{st}
    Sei $R$ ein Ring.
    \begin{enumerate}[(1)]
        \item
            Es existiert eine Homologietheorie über $R$.
        \item
            Auf Simplizialkomplexen (Allgemeiner: auf Zellkomplexen) stimmen je zwei Homologietheorien überein, d.h.
            Sind $(H, \del)$ und $(H', \del')$ zwei Homologietheorien, so gibt es natürliche Isomorphismes
            \begin{math}
                \begin{tikzcd}
                    H(|K|,|A|) \ar[r,"\isomorphic"] \ar[d,"H(f)"] & H'(|K|, |A|) \ar[d,"H'(f)"] \\
                    H(|L|,|B|) \ar[r,"\isomorphic"] & H'(|L|, |B|)
                \end{tikzcd}
            \end{math}
            und natürliche Äquivalenz für $\del \isomorphic \del'$.
    \end{enumerate}
    \begin{proof}
        (2) Ausrechnen mit Axiomen, Übung für Sphären $\S^n$, Später für Komplexe.

        (1) Verlangt eine Konstruktion der \emphdef{singulären kubischen Homologie}:

    \end{proof}
\end{st}

Sei $X$ ein topologischer Raum.
Für $n \in \N$ betrachte
\begin{math}
    \scr C(I^n, X) &= \Set{\alpha: [0,1]^n \to X \text{stetig}}, \\
    \tilde S_n(X, R) &= R \scr C(I^n, X).
\end{math}
Für $i = 1, \dotsc, n$ und $s \in [0,1]$ definiere zu $\alpha: [0,1]^n \to X$ die Einschränkung $\del_i^s \alpha: [0,1]^{n-1} \to X$ durch
\begin{math}
    (\del_i^s \alpha)(t_1, \dotsc, t_{n-1}) = \alpha(t_1, \dotsc, t_{i-1}, s, t_i, \dotsc, t_{n-1}).
\end{math}
Wir hätten gerne $\alpha(1-t_1, t_2, \dotsc, t_n) = -1 \cdot \alpha(t_1, \dotsc, t_n)$, sowie $\alpha = 0$ falls $\del_i^s \alpha$ nicht von $s$ abhängt.
Um dies zu erreichen, bilden wir den Quotienten bezüglich des Untermoduls
\begin{math}
    D_n := R \Set{ \alpha + \_{\alpha}^i,  \text{$\alpha$ degeneriert}},
\end{math}
wobei
\begin{math}
    \_\alpha^i(t_1,\dotsc, t_n) = \alpha(\dotsc, 1 - t_1, \dotsc)
\end{math}
und $\alpha$ degenerirt falls $\del_i^s \alpha$ nicht von $s$ abhängt für ein $i$.

Wir setzen
\begin{math}
    S_n(X, R) := \tilde S_n(X, R) / D_n,
\end{math}
genannt \emphdef{singuläre kubische Ketten}.
Für $\alpha: I^n \to X$ stetig bezeichne $\<\alpha\>$ die Äquivalenzklasse (später kurz $\alpha$).

Setze $\tilde \del: \tilde S_n \to \tilde S_{n-1}$ durch
\begin{math}
    \tilde \del \alpha := \sum_{i=0}^n (-1)^{i} (\del_i^0  - \del_i^1) \alpha.
\end{math}

Es gilt
\begin{itemize}
    \item
        $\tilde \del \circ \tilde \del = 0$ (Beweis: Übung)
    \item
        $\tilde \del(D_n) \subset D_{n-1}$
\end{itemize}
Wir können also zum Quotienten Übergehen:
\begin{math}
    \begin{tikzcd}
        \tilde S_n \ar[r,"\del"] \ar[d,"\quot"] \ar[rd] & \tilde S_{n-1} & \tilde \del \alpha = \sum (-1)^i (\del_i^0 \alpha - \del_i^1 \alpha) \ar[d,"\quot"] \\
        S_n \ar[r,dashed,"\del"] & S_{n-1} & \del\<\alpha\> = \sum (-1)^i (\<\del_i^0 \alpha\> - \<\del_i^1 \alpha\>)
    \end{tikzcd}
\end{math}

\begin{df}
    Zu $X$ heißt $(S_n X, \del_n)_{n\in\Z}$ der \emphdef{Kettenkomplex} der singulären Kuben in $X$.
    Die zugehörige Homologie ist
    \begin{math}
        H_n(X) := \dfrac{Z_n(X)}{B_n(X)} = \dfrac{\ker \del_n}{\im \del_{n+1}}.
    \end{math}
    Zu $(X, A)$ definiere $S_n(X,A) := S_n(X) / S_n(A)$.
    Dies ist ein Kettenkomplex, seine Homologie bezeichnen wir mit $H_n(X, A)$.
    \begin{note}
        Für $A = \emptyset$ gilt $S_n(A) = 0$ für alle $n \in \Z$, also $S_n(X, \emptyset) = S_n(X)$, $H_n(X, \emptyset) = H(X)$.
    \end{note}
    Zu $f: (X, A) \to (Y, B)$ stetig erhalten wir
    \begin{math}
        \tilde S_n(f): \tilde S_n(X) &\to \tilde S_n(Y), \\
        \alpha &\mapsto f \circ \alpha
    \end{math}
    und nach Übergang zu den Äquivalenzklassen
    \begin{math}
        S_n(f): S_n(X) &\to S_n(Y), \\
        \<\alpha\> &\mapsto \<f \circ \alpha\>
    \end{math}.
    Damit
    \begin{math}
        \begin{tikzcd}
            0 \ar[r] & S_n(A) \ar[r] \ar[d,"S_n(f|_A^B)"'] & S_n(X) \ar[r] \ar[d,"S_n(f)"'] & S_n(X,A) \ar[r] \ar[d,"S_n(f)"] & 0 \\
            0 \ar[r] & S_n(B) \ar[r] & S_n(Y) \ar[r] & S_n(Y,B) \ar[r] & 0
        \end{tikzcd}.
    \end{math}
    Dies induziert $H_n(f): H_n(X, A) \to H_n(Y, B)$ durch $[z] \mapsto [S_n(f)(z)]$.

    Wir erhalten zu jedem Paar $(X, A)$ eine lange exakte Sequenz.
    Diese ist natürlich bezüglich $f: (X, A) \to (Y, B)$.
\end{df}


\begin{ex}
    \begin{itemize}
        \item
            Für $X = \Set{p}$ gilt
            \begin{math}
                \begin{tikzcd}[row sep=tiny]
                    0 & \tilde S_0 \ar[l] & \tilde S_1 \ar[l] & \tilde S_2 \ar[l] & \dotsc \ar[l] \\
                    0 & R \ar[l,"0"] & R \ar[l,"0"] & R \ar[l,"0"] & R \ar[l,"0"] &
                \end{tikzcd}.
            \end{math}
            Also
            \begin{math}
                H_n(\tilde S(\Set{p}), \tilde \del) = \begin{cases}
                    R & \text{$n \ge 0$} \\
                    0 & \text{$n < 0$}
                \end{cases},
            \end{math}
            aber alles wird gut
            \begin{math}
                \begin{tikzcd}[row sep=tiny]
                    0 & S_0 \ar[l] & S_1 \ar[l] & S_2 \ar[l] & \dotsc \ar[l] \\
                    0 & R \ar[l] & 0 \ar[l] & 0 \ar[l] & 0 \ar[l] &
                \end{tikzcd}.
            \end{math}
            Und damit
            \begin{math}
                H_n(\Set{p}) = H_n(S(\Set{p}), \del) = \begin{cases}
                    R & \text{für $n = 0$}, \\
                    0 & \text{sonst}.
                \end{cases}
            \end{math}
        \item
            Sei $X = \bigsqcup_{k \in I} X_k$, dann
            \begin{math}
                \scr C(I^n, X) &= \sqcup_{k\in I} \scr C(I^n, X_k), \\
                \tilde S_n(X) &= \bigoplus_{k\in I} \tilde S_n(X_k), \\
                S_n(X) &= \bigoplus_{k\in I} S_n(X_k), \\
                H_n(X) &= \bigoplus_{k\in I} H_n(X_k).
            \end{math}
    \end{itemize}
\end{ex}
