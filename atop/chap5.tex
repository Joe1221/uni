\chapter{Singuläre Homologie}

\section{Homologie-Axiome}


Ziel: Für beliebige Paare topologischer Räume $(X, A)$ möchten wir einen Funktor $H_*: \Cat{Top}_{(2)} \to \Cat{GrMod}_R$, d.h. $H_q: (X, A) \mapsto H_q(X,A) \in \Cat{Mod}_R$ für $q \in \Z$ und $H_q: (f:(X, A) \to (Y, B)) \mapsto (H_q(f): H_q(X,A) \to H_q(Y,B))$.

Wünschenswerte Eigenschaften:
\begin{enumerate}[(1)]
    \item
        \emphdef[Homologie-Axiome!Homotopie-Invarianz]{Homotopie-Invarianz}: Aus $f \homotopic g: (X, A) \to (Y, B)$ folgt $H_*(f) = H_*(g)$.
    \item
        \emphdef[Homologie-Axiome!Lange exakte Sequenz]{Lange exakte Sequenz}: Für $i: (A, \emptyset) \injto (X, \emptyset)$ und $j: (X, \emptyset) \injto (X, A)$ haben wir eine lange exakte Sequenz
        \begin{math}
            \begin{tikzcd}
                {} \ar[r] & H_q(A,\emptyset) \ar[r,"H_q(i)"] & H_q(X,\emptyset) \ar[r,"H_q(j)"] & H_q(X,A) \ar[r,"\del_q"] & H_{q-1}(A,\emptyset) \ar[r] & \dotsc
            \end{tikzcd}.
        \end{math}
        Hierzu fordern wir eine \emphdef{Randabbildung}
        \begin{math}
            \del_q = \del_q^{(X,A)}: H_q(X, A) \to H_{q-1}(A, \emptyset).
        \end{math}
        Dies soll mit dem Funktor $H_*$ verträglich sein, d.h. für $f: (X, A) \to (Y, B)$ soll folgendes Diagramm kommutieren:
        \begin{math}
            \begin{tikzcd}
                H_q(X, A) \ar[r,"\del_q^{(X,A)}"] \ar[d,"H_q(f)"'] & H_{q-1}(A, \emptyset) \ar[d,"H_{q-1}(f|_{(A, \emptyset)})"] \\
                H_q(Y, B) \ar[r,"\del_q^{(Y,B)}"] & H_{q-1}(B, \emptyset)
            \end{tikzcd}.
        \end{math}
        Dies nennt man eine \emphdef{natürliche Transformation}, $\del_q: H_q \to H_{q-1} \circ S$ mit $S(X, A) = (A, \emptyset)$.
    \item
        \emphdef[Homologie-Axiome!Dimensionsaxiom]{Dimensionsaxiom}:
        Für $X = \Set{p}$ gilt
        \begin{math}
            H_q(\Set{p}, \emptyset) \isomorphic \begin{cases}
                R & \text{für $q = 0$}, \\
                0 & \text{sonst}
            \end{cases}
        \end{math}
        Genauer: wir fordern einen natürlichen Isomorphismus.
    \item
        \emphdef[Homologie-Axiome!Additivitsaxiom]{Additivitätsaxiom}:
        Für jede topologische Summe $X = \bigsqcup_{k \in I} X_k \xto*[injective]{\iota_k} X_k$ erhalten wir einen Isomorphismus
        \begin{math}
            \begin{tikzcd}
                \displaystyle \bigoplus_{k\in I} H_q(\iota_k): \bigoplus_{k\in I} H_q(X_k, \emptyset) \ar[r,"\isomorphic"] & H_q(X, \emptyset) \\
            \end{tikzcd}
        \end{math}
    \item
        \emphdef[Homologie-Axiome!Ausschneidungsaxiom]{Ausschneidungsaxiom}:
        Sei $(X, A)$ ein Raumpaar und $U \subset X$ mit $\_U \subset \mathring A$.
        Dann induziert die Inklusion $i: (X \setminus U, A \setminus U) \injto (X, A)$ einen Isomorphismus
        \begin{math}
            H_q(i): H_q(X \setminus U, A \setminus U) \xto{\isomorphic} H_q(X, A).
        \end{math}
\end{enumerate}

\begin{df}[Eilenberg-Steenrod-Milnor-Axiome für Homologie]
    Sei $R$ ein Ring.
    Eine \emphdef{Homologie-Theorie} über $R$ ist ein paar $(H, \del)$ bestehend aus einem Funktor $H: \Cat{Top}_{(2)} \to \Cat{GrMod}_{R}$ und einer natürlichen Transformation $\del: H_q \to H_{q-1} \circ S$, sodass die obigen Axiomen (1)-(5) erfüllt sind.
\end{df}

\begin{ex}
    Beispiele, bzw. Versuche.
    Zur Vereinfachung $R = \Z$, $\Cat{Mod}_\Z = \Cat{AbGrp}$.
    \begin{enumerate}[1)]
        \item
            Triviales Paar $(H, \del)$, $H_q(X,A) = 0$, also $\del_q^{(X,A)} = 0$.
            Dies erfüllt (1), (2), (4), (5), aber nicht (3).
        \item
            Simpliziale Homologie: Erfüllt alles, aber nur auf Simplizialkomplexen $\Cat{SComp}_{(2)} \to \Cat{Top}_{(2)}$.
        \item
            Konstant: $H(X, A) := H(\Set{p}, \ast)$ nach (3).
            \begin{math}
                (f: (X, A) \to (Y, B)) \mapsto H_q(f) := \id.
            \end{math}
            Dies erfüllt (1), (3), (5) aber nicht (2), (4).
            Zu (2): Für $(X, A) = (\Set{p}, \Set{p})$ gilt:
            \begin{math}
                \begin{tikzcd}[row sep=tiny]
                    {} \ar[r] & H_0(A,\emptyset) \ar[r] & H_0(X, \emptyset) \ar[r] & H_0(X, A) \ar[r] & 0 \\
                    {} \ar[r] & R \ar[r] & R \ar[r] & R \ar[r] & R
                \end{tikzcd}
            \end{math}
            Nicht exakt in $H_0(X, \emptyset)$.
        \item
            Für $X$ wegzusammenhängend definiere
            \begin{math} \\
                H(X, \emptyset) &:= H(\Set{p}, \emptyset), \\
                H(X, A) &:= 0 \qquad \text{für $A \neq \emptyset$}.
            \end{math}
            Allgemein: Zerlege $X$ in Wegkomponenten $\pi_0(X)$ und setze $H(X, A) := \bigoplus_{C \in \pi_0(X)} H(C, C \cap A)$.
            Für $f: (X, A) \to (Y, B)$ setze $H_q(f)$ pro Komponente.

            Dies erfüllt (1), (2), (3), (4), aber nicht (5)?
            Übung: Was geht schief in (5)?
        \item
            $(H, \del) = (\pi, \del)$ mit $H_q = \pi_q$ für $q \ge 2$, $H_1 = \pi_1 / \mathrm{ab}$, $H_0 = \Z \pi_0$ und additiv fortgesetzt.
    \end{enumerate}
\end{ex}

\begin{st}
    Sei $R$ ein Ring.
    \begin{enumerate}[(1)]
        \item
            Es existiert eine Homologietheorie über $R$.
        \item
            Auf Simplizialkomplexen (Allgemeiner: auf Zellkomplexen) stimmen je zwei Homologietheorien überein, d.h.
            sind $(H, \del)$ und $(H', \del')$ zwei Homologietheorien, so gibt es natürliche Isomorphismen
            \begin{math}
                \begin{tikzcd}
                    H(|K|,|A|) \ar[r,"\isomorphic"] \ar[d,"H(f)"] & H'(|K|, |A|) \ar[d,"H'(f)"] \\
                    H(|L|,|B|) \ar[r,"\isomorphic"] & H'(|L|, |B|)
                \end{tikzcd}
            \end{math}
            und natürliche Äquivalenz für $\del \isomorphic \del'$.
    \end{enumerate}
    \begin{proof}
        (2) Ausrechnen mit Axiomen: Übung für Sphären $\S^n$, Später für Komplexe.

        (1) Verlangt eine Konstruktion der \emphdef{singulären kubischen Homologie}:

    \end{proof}
\end{st}

\section{Konstruktion der singulären kubischen Homologie}

Sei $X$ ein topologischer Raum.
Für $n \in \N$ betrachte
\begin{math}
    \scr C(I^n, X) &= \Set{\alpha: [0,1]^n \to X \text{stetig}}, \\
    \tilde S_n(X, R) &= R \scr C(I^n, X).
\end{math}
Für $i = 1, \dotsc, n$ und $s \in [0,1]$ definiere zu $\alpha: [0,1]^n \to X$ die Einschränkung $\del_i^s \alpha: [0,1]^{n-1} \to X$ durch
\begin{math}
    (\del_i^s \alpha)(t_1, \dotsc, t_{n-1}) = \alpha(t_1, \dotsc, t_{i-1}, s, t_i, \dotsc, t_{n-1}).
\end{math}
Wir hätten gerne $\alpha(1-t_1, t_2, \dotsc, t_n) = -1 \cdot \alpha(t_1, \dotsc, t_n)$, sowie $\alpha = 0$ falls $\del_i^s \alpha$ nicht von $s$ abhängt.
Um dies zu erreichen, bilden wir den Quotienten bezüglich des Untermoduls
\begin{math}
    D_n := R \Set{ \alpha + \_{\alpha}^i,  \text{$\alpha$ degeneriert}},
\end{math}
wobei
\begin{math}
    \_\alpha^i(t_1,\dotsc, t_n) = \alpha(\dotsc, 1 - t_1, \dotsc)
\end{math}
und $\alpha$ degenerirt falls $\del_i^s \alpha$ nicht von $s$ abhängt für ein $i$.

Wir setzen
\begin{math}
    S_n(X, R) := \tilde S_n(X, R) / D_n,
\end{math}
genannt \emphdef{singuläre kubische Ketten}.
Für $\alpha: I^n \to X$ stetig bezeichne $\<\alpha\>$ die Äquivalenzklasse (später kurz $\alpha$).

Setze $\tilde \del: \tilde S_n \to \tilde S_{n-1}$ durch
\begin{math}
    \tilde \del \alpha := \sum_{i=1}^n (-1)^{i} (\del_i^0  - \del_i^1) \alpha.
\end{math}

Es gilt
\begin{itemize}
    \item
        $\tilde \del \circ \tilde \del = 0$ (Beweis: Übung)
    \item
        $\tilde \del(D_n) \subset D_{n-1}$
\end{itemize}
Wir können also zum Quotienten Übergehen:
\begin{math}
    \begin{tikzcd}
        \tilde S_n \ar[r,"\del"] \ar[d,"\quot"] \ar[rd] & \tilde S_{n-1} \ar[d,"\quot"] & \tilde \del \alpha = \sum (-1)^i (\del_i^0 \alpha - \del_i^1 \alpha) \ar[d,"\quot"] \\
        S_n \ar[r,dashed,"\del"] & S_{n-1} & \del\<\alpha\> = \sum (-1)^i (\<\del_i^0 \alpha\> - \<\del_i^1 \alpha\>)
    \end{tikzcd}
\end{math}

\begin{df}
    Zu $X$ heißt $(S_n X, \del_n)_{n\in\Z}$ der \emphdef{Kettenkomplex} der singulären Kuben in $X$.
    Die zugehörige Homologie ist
    \begin{math}
        H_n(X) := \dfrac{Z_n(X)}{B_n(X)} = \dfrac{\ker \del_n}{\im \del_{n+1}}.
    \end{math}
    Zu $(X, A)$ definiere $S_n(X,A) := S_n(X) / S_n(A)$.
    Dies ist ein Kettenkomplex, seine Homologie bezeichnen wir mit $H_n(X, A)$.
    \begin{note}
        Für $A = \emptyset$ gilt $S_n(A) = 0$ für alle $n \in \Z$, also $S_n(X, \emptyset) = S_n(X)$, $H_n(X, \emptyset) = H(X)$.
    \end{note}
    Zu $f: (X, A) \to (Y, B)$ stetig erhalten wir
    \begin{math}
        \tilde S_n(f): \tilde S_n(X) &\to \tilde S_n(Y), \\
        \alpha &\mapsto f \circ \alpha
    \end{math}
    und nach Übergang zu den Äquivalenzklassen
    \begin{math}
        S_n(f): S_n(X) &\to S_n(Y), \\
        \<\alpha\> &\mapsto \<f \circ \alpha\>
    \end{math}.
    Damit
    \begin{math}
        \begin{tikzcd}
            0 \ar[r] & S_n(A) \ar[r] \ar[d,"S_n(f|_A^B)"] & S_n(X) \ar[r] \ar[d,"S_n(f)"] & S_n(X,A) \ar[r] \ar[d,"S_n(f)"] & 0 \\
            0 \ar[r] & S_n(B) \ar[r] & S_n(Y) \ar[r] & S_n(Y,B) \ar[r] & 0
        \end{tikzcd}.
    \end{math}
    Dies induziert $H_n(f): H_n(X, A) \to H_n(Y, B)$ durch $[z] \mapsto [S_n(f)(z)]$.

    Wir erhalten zu jedem Paar $(X, A)$ eine lange exakte Sequenz.
    Diese ist natürlich bezüglich $f: (X, A) \to (Y, B)$.
\end{df}


\begin{ex}
    \begin{itemize}
        \item
            \emphdef{Dimensionsaxiom}:
            Für $X = \Set{p}$ gilt
            \begin{math}
                \begin{tikzcd}[row sep=tiny]
                    0 & \tilde S_0 \ar[l] & \tilde S_1 \ar[l] & \tilde S_2 \ar[l] & \dotsc \ar[l] \\
                    0 & R \ar[l,"0"] & R \ar[l,"0"] & R \ar[l,"0"] & R \ar[l,"0"] &
                \end{tikzcd}.
            \end{math}
            Also
            \begin{math}
                H_n(\tilde S(\Set{p}), \tilde \del) = \begin{cases}
                    R & \text{$n \ge 0$} \\
                    0 & \text{$n < 0$}
                \end{cases},
            \end{math}
            aber alles wird gut
            \begin{math}
                \begin{tikzcd}[row sep=tiny]
                    0 & S_0 \ar[l] & S_1 \ar[l] & S_2 \ar[l] & \dotsc \ar[l] \\
                    0 & R \ar[l] & 0 \ar[l] & 0 \ar[l] & 0 \ar[l] &
                \end{tikzcd}.
            \end{math}
            Und damit
            \begin{math}
                H_n(\Set{p}) = H_n(S(\Set{p}), \del) = \begin{cases}
                    R & \text{für $n = 0$}, \\
                    0 & \text{sonst}.
                \end{cases}
            \end{math}
        \item
\Timestamp{2016-02-05}
            \emphdef{Additivität}:
            Sei $X = \bigsqcup_{k \in I} X_k$, dann
            \begin{math}
                \scr C(I^n, X) &= \bigsqcup_{k\in I} \scr C(I^n, X_k), \\
                \tilde S_n(X) &= \bigoplus_{k\in I} \tilde S_n(X_k), \\
                S_n(X) &= \bigoplus_{k\in I} S_n(X_k).
            \end{math}
            Wir erhalten die direkte Summe $(S_n(X), \del^X) = \bigoplus_{i \in I} (S_n(X_i), \del^{X_i})$, also
            \begin{math}
                H_n(X) &= \bigoplus_{k\in I} H_n(X_k).
            \end{math}
    \end{itemize}
\end{ex}

\begin{st}
    $H_*: \Cat{Top}_{(2)} \to \Cat{GrMod}_{R}$ ist homotopie-invariant:
    \begin{math}
        \begin{tikzcd}
            \Cat{Top}_{(2)} \ar[r,"H_*"] \ar[d] & \Cat{GrMod}_{R} \\
            \Cat{HTop}_{(2)} \ar[ru,dashed]
        \end{tikzcd}
    \end{math}
    Das heißt:
    Sei $h: f \homotopic g: (X, A) \to (Y, B)$ eine Homotopie, also $h: [0,1] \times X \to Y$ mit $h_t(A) \subset B$ für alle $t \in [0,1]$, sowie $h_0 = f$ udn $h_1 = g$.
    Dann gilt $H_*(f) = H_*(g): H_*(X, A) \to H_*(Y, B)$.
    \begin{proof}
        Definiere $T: S_n(X, A) \to S_{n+1}(Y, B)$.
        Zu $\alpha: [0,1]^n \to X$ setze $T\alpha: [0,1]^{n+1} \to Y$ durch
        \begin{math}
            (T\alpha)(t_0, \dotsc, t_n)
            := h(t, \alpha(t_1, \dotsc, t_n)).
        \end{math}
        Dann gilt
        \begin{math}
            \del T\alpha
            &= [h(1, \alpha(t_1, \dotsc, t_n) - h(0, \alpha(t_1, \dotsc, t_n))] \\
            &\qquad + \sum_{i=1}^n (-1)^i \Big[h(t_1, \alpha(t_2, \dotsc, t_i, 1, t_{i+1}, \dotsc, t_n)) \\
            &\qquad \qquad \qquad - h(t_1, \alpha(t_2, \dotsc, t_i, 0, t_{i+1}, \dotsc, t_n)) \Big] \\
            &= g \circ \alpha - f \circ \alpha - T \del \alpha,
        \end{math}
        d.h.
        \begin{math}
            \del T \alpha + T \del \alpha = g \circ \alpha - f \circ \alpha,
        \end{math}
        kurz $\del T + T \del = g_* - f_*$.
        Dies ist eine Kettenhomotopie $T: f_* \homotopic g_*$.

        Daraus folgt $H_*(f) = H_*(g)$.
    \end{proof}
\end{st}

\begin{kor}
    Jede Homotopie-Äquivalenz von Räumen $(f,g): (X, A) \homotopic (Y, B)$ induziert einen Isomorphismus
    \begin{math}
        \begin{tikzcd}
            H_*(X,A) \ar[r,"H_*(f)"] & H_*(Y,B) \ar[l,"H_*(g)"]
        \end{tikzcd}
    \end{math}
    \begin{proof}
        Links kommutieren die Dreiecke nur bis auf Homotopie.
        \begin{math}
            \begin{tikzcd}
                (X, A) \ar[r,"f"] \ar[d,"\id"] & (Y, B) \ar[ld,"g"'] \ar[d,"\id"] \\
                (X, A) \ar[r,"f"] & (Y, B)
            \end{tikzcd}
            \quad
            \xto{H_*}
            \quad
            \begin{tikzcd}
                H_*(X, A) \ar[r,"H_*(f)"] \ar[d,"\id"] & H_*(Y, B) \ar[ld,"H_*(g)"'] \ar[d,"\id"] \\
                H_*(X, A) \ar[r,"H_*(f)"] & H_*(Y, B)
            \end{tikzcd}
        \end{math}
    \end{proof}
\end{kor}

\subsection{Unterteilung}

Unterteilung von $\alpha: [0,1]^n \to X$ (Halbieren in jeder Koordinate).

Sei $\epsilon \in \Set{0,1}^n$ und $\alpha_\epsilon: [0,1]^n \to X$ mit
\begin{math}
    \alpha_\epsilon(t_1, \dotsc, t_n) := \alpha(\frac{t_1 + \epsilon_1}{2}, \dotsc, \frac{t_n + \epsilon_n}{2}).
\end{math}
Wir unterteilen dann
\begin{math}
    s: S_n(X) &\to S_n(X) \\
    \alpha &\mapsto \sum_{\epsilon \in \Set{0,1}^n} \alpha_\epsilon.
\end{math}

\begin{prop}
    $\del s(\alpha) = s(\del \alpha)$, kurz: $\del \circ s = s \circ \del$, d.h. $s$ ist ein Kettenhomomorphismus.
\end{prop}

Ziel: $s \homotopic \id$.

Setze $\rho: [0,1] \to [0,1]$,
\begin{math}
    \rho(t) := \begin{cases}
        2t & \text{für $0 \le t \le \frac{1}{2}$}, \\
        1 & \text{für $\frac{1}{2} \le t \le 1$}.
    \end{cases}
\end{math}
Setze $r(\alpha) := \alpha \circ \rho^{\times n}: [0,1]^n \to X$, d.h.
\begin{math}
    r(\alpha) := \alpha(\rho(t_1), \dotsc, \rho(t_n)).
\end{math}
Dann $r(\alpha)_{0,\dotsc, 0} = \alpha$ und für $\epsilon \neq 0$ ist $r(\alpha)_\epsilon$ degeneriert (in eine Richtung konstant).

Wir erhalten $r: S_n(X) \to S_n(X)$ mit $s \circ r = \id$.
Zudem gilt $\del \circ r = r \circ \del$.

Definiere $T: S_n(X) \to S_{n+1}(X)$ durch
\begin{math}
    (T\alpha)(t_0, \dotsc, t_n)
    := \alpha((1-t_0) \rho(t_1) + t_0 t_1, \dotsc, (1-t_0) \rho(t_n) + t_0 t_n).
\end{math}
Es gilt $\del T \alpha = \alpha - r \circ \alpha - T \del \alpha$, also $\del T + T \del = \id - r$, d.h. $T: \id \homotopic r$
Damit ist
\begin{math}
    s- \id
    = s \circ \id - s \circ r
    = s \circ (\id - r)
    = s \circ (\del T + T \del)
    = s \del T + s T \del
    = \del(sT) + (sT)\del,
\end{math}
kurz: $sT: s \homotopic \id$, ebenso $s^k \homotopic \id$.


Sei $X = \bigcup_{U \in \scr U} \mathring U$ eine offene Überdeckung.
Sei
\begin{math}
    S_n(X) \supset S_n^{\scr U}(X) &:= \sum_{U \in \scr U} S_n(U) \\
    &= R \Set{\alpha: [0,1]^n \to X & \alpha([0,1]^n) \subset U \text{ für ein $U \in \scr U$}}.
\end{math}
Dies ist ein Unterkettenkomplex $(S_n^U(X), \del_n)_{n \in \Z}$ von $(S_n(X), \del_n)_{n \in \Z}$, denn $\alpha$ klein impliziert $\del_n \alpha$ klein.

\begin{st}
    Die Inklusion $i: S_*^{\scr U}(X) \injto S_*(X)$ induziert einen Isomorphismus $i_*: H_*^{\scr U}(X) \xto{\isomorphic} H_*(X)$.
    \begin{proof}
        \begin{enumerate}[1.]
            \item
                \emph{$i_*$ ist surjektiv}:
                Sei $[z] \in H_n(X)$ eine Homologieklasse, repräsentiert durch einen Zykel $z \in Z_n(X) \subset S_n(X)$
                Es gilt $z' = s^k(z) \in S_n^{\scr U}(X)$ für $K$ groß (Kompaktheit von $[0,1]^n$ und Lebesgue-Lemma).
                Dann ist $\del z' = \del s^k(z) = s^k(\del z) = 0$, d.h. $z' \in Z_n^{\scr U}(X)$.

                Wir wissen $s^k \homotopic \id$, also $T: s^k \homotopic \id$, d.h. $\del T + T \del = s^k - \id$.
                Das bedeutet
                \begin{math}
                    z' - z
                    = (s^k - \id) z
                    = (\del T + T \del) z
                    = \del Tz + T \underbrace{\del z}_{=0}
                    \in B_n(X).
                \end{math}
                In $H_n(X) = Z_n(X) / B_n(X)$ gilt also $[z'] = [z]$.
            \item
                \emph{$i_*$ ist injektiv}:
                Sei $z \in Z_n^{\scr U}(X)$ ein Rand in $X$, d.h. $z = \del b$ mit $b \in S_{n+1}(X)$.
                Dann $z \sim s^k(z)$ in $S_n^{\scr U}(X)$ und $s^k(z) = s^k(\del b) = \del(s^k b)$ mit $s^k b \in S_{n+1}^{\scr U}(X)$ für $k$ groß.
                Folglich gilt in $S_n^{\scr U}(X)$, dass $z \sim 0$.
        \end{enumerate}
    \end{proof}
\end{st}

\begin{st}[Ausschneidung]
    Sei $(X, A)$ ein Raumpaar und $\_U \subset \mathring A$.
    Dann induziert die Inklusion $\iota: (X \setminus U, A \setminus U) \injto (X, A)$ einen Isomorphismus
    $H_*(\iota) : H_*(X \setminus U, A \setminus U) \xto{\isomorphic} H_*(X, A)$.
    \begin{proof}
        Betrachte $\scr U= \Set{X \setminus U, A}$.
        Es gilt $\mathring{\paren{X \setminus U}} = X \setminus \_U$, somit $X = \mathring{\paren{X \setminus U}} \cup \mathring A$.
        \begin{math}
            \begin{tikzcd}
                S_n(X \setminus U) \ar[r,inj] \ar[d,"\quot"] \ar[dr] &
                S_n^{\scr U}(X) = S_n(X \setminus U) + S_n(A) \ar[r,inj] \ar[d,"\quot"] \ar[dr] &
                S_n(X) \ar[d,"\quot"] \\
                \frac{S_n(X \setminus U)}{S_n(A \setminus U)} \ar[r,"\isomorphic"] &
                \frac{S_n(X \setminus U) + S_n(A)}{S_n(A)} \ar[r] &
                S_n(X, A) = S_n(X) / S_n(A)
            \end{tikzcd}
        \end{math}
    \end{proof}
\end{st}

\begin{nt}
    Damit ist der Existenzsatz bewiesen: $(H_*, \del_*)$ erfüllt alle $5$ Axiome.
\end{nt}

\begin{st}
    Sei $A \subset X$  ein Deformationsretrakt einer Umgebung $\tilde A \subset X$.
    Dann gilt
    \begin{math}
        \begin{tikzcd}
            H_*(X, A) \ar[r,"\isomorphic"] & H_*(\underbrace{X \unite A}_{Y}, \underbrace{A \unite A}_{B = \Set{A}}).
        \end{tikzcd}
    \end{math}
    \begin{proof}
        \begin{math}
            \begin{tikzcd}
                (X, A) \ar[r,inj,"\text{Hom.Äquiv.}"] \ar[d,inj,"\quot"] &
                (X, \tilde A) \ar[d,"\quot"] &
                (X \setminus A, \tilde A \setminus A) \ar[l,"\text{Ausschn.}"'] \ar[d,"\quot","\homeomorphic"'] \\
                (Y, B) \ar[r,inj] &
                (Y, \tilde B) &
                (Y \setminus B, \tilde B \setminus B) \ar[l,inj]
            \end{tikzcd}
        \end{math}
        Wende $H_*$ an und erhalte Isomorphismen
        \begin{math}
            \begin{tikzcd}
                H_*(X, A) \ar[r] \ar[d] &
                H_*(X, \tilde A) \ar[d] &
                H_*(X \setminus A, \tilde A \setminus A) \ar[l] \ar[d] \\
                H_*(Y, B) \ar[r] &
                H_*(Y, \tilde B) &
                H_*(Y \setminus B, \tilde B \setminus B) \ar[l]
            \end{tikzcd}
        \end{math}
    \end{proof}
\end{st}

\begin{ex}
    Übung: Berechnen Sie aus den Axiomen $H_*(\D^n), H_*(\D^n, \S^{n-1}), H_*(\S^n)$ per Induktion.
\end{ex}

\begin{st}[Mayer-Vietoris]
    Sei $X = \mathring U \cup \mathring V$ eine offene Überdeckung, also $\scr U = \Set{U, V}$.
    Dann erhalten wir eine kurze exakte Sequenz
    \begin{math}
        \begin{tikzcd}[row sep=tiny,columns sep=small]
            0 &
            S_n(U \cap V) \ar[r,"i"] &
            S_n(U) \oplus S_n(V) \ar[r,"j"] &
            S_n^{\scr U}(X) = S_n(U) + S_n(V) \ar[r] &
            0 \\
            {} &
            c \ar[r,|->] & (c,c) \\
            {} & {} &
            (a,b) \ar[r,|->] & b - a
        \end{tikzcd}
    \end{math}
    Diese induziert die lange exakte Sequenz
    \begin{math}
        \begin{tikzcd}[column sep=small]
            \dotsc \ar[r,"\Delta_{n+1}"] &
            H_n(U \cap V) \ar[r,"i_*"] &
            H_n(U) \oplus H_n(V) \ar[r,"j_*"] &
            H_n^{\scr U}(X) \ar[r,"\Delta_n'"] \ar[d,"\isomorphic"] &
            H_{n-1}(U \cap V) \ar[r] &
            \dotsc \\
            \dotsc \ar[r,"\Delta_{n+1}"] &
            H_n(U \cap V) \ar[r,"i_*"] &
            H_n(U) \oplus H_n(V) \ar[r,"j_*"] &
            H_n(X) \ar[r,"\Delta_n"] &
            H_{n-1}(U \cap V) \ar[r] &
            \dotsc \\
            {} &
            {[c]} \ar[r,|->] & ([c], [c]) &
            {[z]} \ar[r,|->] & {[\del c_1]} \\
            {} & {} &
            ([a], [b]) \ar[r,|->] & {[b] - [a]}
        \end{tikzcd}.
    \end{math}
    Skizze: $U, V$ mit Zykel $z$ und Teile $c_1$ und $c_2$ in $U$, bzw. $V$.
    Zu $z \in S_n(X)$ wähle $z \sim z' = c_1 + c_2$, $c_1 \in S_n(U)$, $c_2 \in S_n(V)$.
    Setze $\Delta_n(z) := \del c_1 = - \del c_2$.
    Es gilt $0 = \del z = \del z' = \del c_1 + \del c_2$.
\end{st}

\begin{ex}
    Für $n \ge 1$ ist
    \begin{math}
        H_q(\S^0) &= \begin{cases}
            R \oplus R & \text{für $q = 0$}, \\
            0 & \text{sonst}.
        \end{cases} \\
        H_q(\S^n) &= \begin{cases}
            R & \text{für $q \in \Set{0,n}$}, \\
            0 & \text{sonst}.
        \end{cases}
    \end{math}
    \begin{proof}
        Durch Induktion: für $n \ge 1$ wähle $\S^n = U \cup V$ mit $U = \S^n \setminus \Set{-p}$ und $V = \S^n \setminus \Set{p}$ zusammenziehbar.
        \begin{math}
            U \cap V = \S^n \setminus \Set{\pm p} \homeomorphic \S^{n-1} \times (-1,1) \homequiv \S^{n-1}.
        \end{math}
        Wir haben
        \begin{math}
            \begin{tikzcd}
                H_1(U \cap V) \ar[r] &
                H_1(U) \oplus H_1(V) \ar[r] &
                H_1(\S^1) \ar[lld] \\
                H_0(U \cap V) = R \oplus R \ar[r] &
                H_0(U) \oplus H_1(V) = R \oplus R \ar[r] &
                H_0(\S^1) = R
            \end{tikzcd}
        \end{math}
        mit $H_0(U \cap V) \to H_0(U) \oplus H_1(V) : (c,d) \mapsto (c+d, c+d)$.
        Durch Betrachtung des Kerns dieser Abbildung erhatlen wir $H_1(\S^1) \isomorphic R$.

        Für $n \ge 2$ ähnlich, aber leichter.
    \end{proof}
\end{ex}


Anwendungen:

$f: \S^n \to \S^n$ induziert
\begin{math}
    \begin{tikzcd}
        H_n(\S^n, \Z) \ar[r,"H_n(f)"] \ar[d,"\isomorphic"] &
        H_n(\S^n, \Z) \ar[d,"\isomorphic"] \\
        \Z \ar[r,"\cdot \deg(f)"] &
        \Z
    \end{tikzcd}
\end{math}

Zelluläre Homologie:

Sei $(X, \scr X)$ ein Zellkomplex, d.h. $\scr X = (\emptyset = X_{-1} \subset X_0 \subset X_1 \dotsc)$ mit $X_n \setminus X_{n-1} \homeomorphic \bigsqcup_{i \in I} \Set{i} \times \B^n$ und es existieren charakteristische Abbildungen.

Anschaulich: $X_n \unite X_{n-1}$.
Beispiel: gefülltes Dreieck: $X_2 \unite X_1 \homeomorphic \S^2$ und $X_1 \unite X_0$ bildet drei in einem Punkt verknüpfte Kreise.

Zellulärer Kettenkomplex:
\begin{math}
    C_n(X, \scr X) := H_n(X_n, X_{n-1}).
\end{math}
Dieser $R$-Modul ist frei über den Zellen von $X_n \setminus X_{n-1}$
\begin{math}
    \begin{tikzcd}
        \B^n \ar[r,"\homeomorphic"] \ar[d,inj] &
        e_i \in \pi_0(X_n \setminus X_{n-1}) \ar[d,inj] \\
        \D^n \ar[r,"\phi_i"] \ar[d] \ar[dr] &
        X_n \ar[d,"\quot"] \\
        \D^n \unite \S^{n-1} \ar[r] &
        X_n \unite X_{n-1}
    \end{tikzcd}
\end{math}
Dies induziert mit $H_*$
\begin{math}
    R = H_n(\D^n, \S^{n-1}) \xto{H_*(\phi_i)} H_n(X_n, X_{n-1}).
\end{math}
Wie sieht die Randabbildung aus?
Sei $\<\phi\> \in C_n(X, \scr X)$ eine $n$-Zelle und $\<\psi\> \in C_{n-1}(X, \scr X)$ eine $(n-1)$-Zelle.
Definiere die \emphdef{Inzidenzzahl} $[\phi: \psi]$, anschaulich: „Wie oft läuft $\del \phi$ über $\psi$“.

\begin{math}
    \begin{tikzcd}
        \dotsc &
        H_n(X_n, X_{n-1}) \ar[r,"\del_*"] &
        H_{n-1}(X_{n-1}) \ar[d,"="] &
        \dotsc \\
        {} & \dotsc &
        H_{n-1}(X_{n-1}) \ar[r] &
        H_{n-1}(X_{n-1}, X_{n-2}) &
        \dotsc
    \end{tikzcd}
\end{math}
Dies definiert $\delta: C_n(X, \scr X) = H_n(X_n, X_{n-1}) \to H_{n-1}(X_{n-1}, X_{n-2}) = C_{n-1}(X, \scr X)$ durch
\begin{math}
    \delta\<\phi\> = \sum_{\<\psi\>} [\phi: \psi] \<\psi\>.
\end{math}
Nachrechnen liefert, dass $(C_*(X, \scr X), \delta_*)$ ein Kettenkomplex ist, also dass $\delta \circ \delta = 0$.

\begin{ex}
    Für Simplizialkomplexe erhalten wir die simpliziale Homologie.
\end{ex}

\begin{st}
    Die Inklusionen induzieren einen Isomorphismus von zellulärer und singulärer Homologie
    \begin{math}
        H_*(X, \scr X) \xto{\isomorphic} H_*(X)
    \end{math}
\end{st}

\begin{ex}
    \begin{itemize}
        \item
            Betrachte $\S^n$, $n \ge 2$ als Kettenkomplex (ein Punkt, eine $n$-Zelle)
            Wir erhalten $C_0 = C_n = R$ und $C_k = 0$ für $k \notin \Set{0,n}$
            \begin{math}
                H_q(C(X, \scr X)) = \begin{cases}
                    R & \text{für $q \in \Set{0,n}$}, \\
                    0 & \text{sonst}.
                \end{cases}
            \end{math}
        \item
            $\S^n$ mit $\tilde{\scr X} = (\S^0 \subset \S^1 \subset \dotsc)$ und
            $\RP^n$ mit $\scr X = (\RP^0 \subset \RP^1 \subset \dotsc)$ und für $C_n(\RP^n, \scr X,\Z)$ erhalten wir
            \begin{math}
                \begin{tikzcd}
                    0 &
                    \Z \ar[l] &
                    \Z \ar[l,"0"] &
                    \Z \ar[l,"2"] &
                    \Z \ar[l,"0"] &
                    \dotsc \ar[l] &
                    \Z \ar[l] &
                    0 \ar[l]
                \end{tikzcd}
            \end{math}
            Für $H_n$ erhalten wir $0, \Z, \Z/2, 0, \Z/2, 0, \dotsc$.

            Nun ist für $R = \Z / 2$
            \begin{math}
                H_n(\RP^n,\scr X, \Z / 2)
                = \begin{cases}
                    \Z / 2 & \text{für $0 \le q \le n$}, \\
                    0 & \text{sonst}.
                \end{cases}
            \end{math}
    \end{itemize}
\end{ex}
