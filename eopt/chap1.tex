% fixme: erste Vorlesung

\coursetimestamp{16}{10}{2013}

Wir bezeichnen mit $x_i$ die in das $i$-te Produkt investierte Summe und mit $\my_i$ die erwartete Rendite.
Die erwartete Gesamtrendite ergibt sich dann als
\[
	\sum_{i=1}^n \my_i x_i = \my^T x.
\]
Sei $\Sigma \in \R^{n\times n}$ die Kovarianzmatrix der Renditen, dann ergibt sich die Gesamtvarianz als
\[
	V(x) = x^T \Sigma x.
\]
Optimierungsaufgaben wären dann
\begin{enumerate}[I.]
	\item
		Minimiere zu gegebener Mindestrendite die Varianz:
		$V(x) \to \min!$ unter der Nebenbedingung $E(x) \ge \my_0$.
	\item
		Zu gegebener Höchstvarianz, maximiere Rendite:
		$E(x) \to \max!$ unter der Nebenbedingung $V(x) \le V_0$.
\end{enumerate}
Portfolios, die die Optimierungsprobleme I und II lösen heißen auch \emph{effizient}.

\subsection{Computertomographie}

Es ergibt sich grob folgender Zusammenhang
\[
	b_i = \sum_{j=1}^n a_{ij} x_j
\]
für $i = 1, \dotsc, m$ und damit ein lineares Gleichungssystem $Ax = b$, $A \in \R^{m\times n}$.
$m$ ist dabei die Anzahl der zu messenden Strahlen und $n$ die Anzahl Pixel.

Im Allgemeinen ist das Gleichungssystem nicht exakt lösbar, deshalb streben wir eine bestmögliche Lösung an:
\[
	\|Ax - b\| \to \min!.
\]
Moderne Tomographieverfahren führen auf nicht-lineare Zusammenhänge, d.h. auf das Optimierungsproblem
\[
	\|F(x) - b\| \to \min!.
\]
In vielen inversen Problemen ist die Lösung, die am besten zu den Messdaten passt, unbrauchbar (daher \emph{Regularisierung}).



