\chapter{Globale Optimierung}



Bisher haben wir Algorithmen gefunden, die unter bestimmten Voraussetzungen global/lokal gegen lokale Minima (prä-)konvergierten.

Wie bestimmen wir das Minimum von
\[
	f(x) = \sin( 10 \pi (x + 0.2) ) \cos (30 (x + 0.4))
\]
auf $[0, 1]$?

Ideen:
\begin{itemize}
	\item
		Gradientenverfahren mit Startwert 1
	\item
		Analytisch ausrechnen
	\item
		Plotten und Angucken
\end{itemize}

\begin{alg}[Triviale globale Optimierung] \label{alg:13}
	\begin{algorithmic}
		\Require{$f: \R^n \to \R$ stetig}
		\State{Wähle abzählbare dichte Teilmenge $(q_m)_{m \in \N} \subset \R^n$}
		\State{$x_1 \gets q_1$}
		\For{$k \in \{1, 2, \dotsc \}$}
			\If{$f(q_{k+1}) < f(x_k)$}
				\State{$x_{k+1} \gets q_{k+1}$}
			\Else
				\State{$x_{k+1} \gets x_k$}
			\EndIf
		\EndFor
	\end{algorithmic}
\end{alg}

\begin{lem} \label{4.1}
	Sei $f: \R^n \to \R$ stetig.
	Die von \ref{alg:13} erzeugte Folge $(x_k)_{k\in\N}$ besitzt die Eigenschaften
	\begin{enumerate}[(a)]
		\item
			$f(x_{k+1}) \le f(x_k)$
		\item
			Jeder Häufungspunkt ist ein globales Minimum
	\end{enumerate}
	\begin{proof}
		\begin{enumerate}[(a)]
			\item
				trivial
			\item
				Sei $(x_k)_{k\in\N}$ eine konvergente Teilfolge mit $x = \lim_{K \ni k \to \infty} x_k$.
				Angenommen es existiert $y$ mit $f(y) < f(x)$.
				Dann existiert wegen der Stetigkeit von $f$ $\eps > 0$ mit $f(\eta) < f(x)$ für alle $\eta \in B_\eps(y)$.
				Es existiert also $m \in \N$ mit $q_m \in B_\eps(y)$, also $f(q_m) < f(x)$.
				Nach Konstruktion gilt aber $f(x_l) \le f(q_m)$ für alle $l \ge m$, also ein Widerspruch.

				Also $f(x) \le \inf_{l \in \N} f(q_l)$.
		\end{enumerate}
	\end{proof}
\end{lem}


Zur Implementierung einer abzählbaren dichten Teilmenge nutze baumartige Verfeinerung, z.B. für $[0,1]$

Die triviale globale Optimierung benötigt eine große Anzahl an Funktionsauswertungen, ist also unbrauchbar im $\R^n$ mit großem $n$ (“curse of dimensionality“).

Es kann keinen konvergenten globalen Optimierungsalgorithmus geben, der nicht $f$ auf einer abzählbaren dichten Teilmeng auswertet (siehe Orginalskript).

Ein Ausweg bieten die sogenannten \emph{Branch-and-Bound-Strategien}.


\subsection{Branch-and-Bound-Strategien}


Kann für ein Teilintervall ausgeschlossen werden, dass sich darin ein globales Minimium befindet, dann muss bei baumartiger Verfeinerung ein kompletter Ast nicht weiter betrachtet werden.
