% This work is licensed under the Creative Commons
% Attribution-NonCommercial-ShareAlike 3.0 Unported License. To view a copy of
% this license, visit http://creativecommons.org/licenses/by-nc-sa/3.0/ or send
% a letter to Creative Commons, 444 Castro Street, Suite 900, Mountain View,
% California, 94041, USA.

\documentclass{mycourse}

\newcommand{\udN}{\ensuremath{\;\text{u.d.N.}\,}}
\newcommand{\pred}{\text{pred}}
\newcommand{\ared}{\text{ared}}


\begin{document}

\title{Einführung in die Optimierung}

\maketitle

\tableofcontents


% fixme: erste Vorlesung

\coursetimestamp{16}{10}{2013}

Wir bezeichnen mit $x_i$ die in das $i$-te Produkt investierte Summe und mit $\my_i$ die erwartete Rendite.
Die erwartete Gesamtrendite ergibt sich dann als
\[
	\sum_{i=1}^n \my_i x_i = \my^T x.
\]
Sei $\Sigma \in \R^{n\times n}$ die Kovarianzmatrix der Renditen, dann ergibt sich die Gesamtvarianz als
\[
	V(x) = x^T \Sigma x.
\]
Optimierungsaufgaben wären dann
\begin{enumerate}[I.]
	\item
		Minimiere zu gegebener Mindestrendite die Varianz:
		$V(x) \to \min!$ unter der Nebenbedingung $E(x) \ge \my_0$.
	\item
		Zu gegebener Höchstvarianz, maximiere Rendite:
		$E(x) \to \max!$ unter der Nebenbedingung $V(x) \le V_0$.
\end{enumerate}
Portfolios, die die Optimierungsprobleme I und II lösen heißen auch \emph{effizient}.

\subsection{Computertomographie}

Es ergibt sich grob folgender Zusammenhang
\[
	b_i = \sum_{j=1}^n a_{ij} x_j
\]
für $i = 1, \dotsc, m$ und damit ein lineares Gleichungssystem $Ax = b$, $A \in \R^{m\times n}$.
$m$ ist dabei die Anzahl der zu messenden Strahlen und $n$ die Anzahl Pixel.

Im Allgemeinen ist das Gleichungssystem nicht exakt lösbar, deshalb streben wir eine bestmögliche Lösung an:
\[
	\|Ax - b\| \to \min!.
\]
Moderne Tomographieverfahren führen auf nicht-lineare Zusammenhänge, d.h. auf das Optimierungsproblem
\[
	\|F(x) - b\| \to \min!.
\]
In vielen inversen Problemen ist die Lösung, die am besten zu den Messdaten passt, unbrauchbar (daher \emph{Regularisierung}).



\chapter{Unrestringierte nichtlineare Optimierung}


Betrachte
\[
	\min_{x\in \R^n} f(x)
\]
mit (hinreichend glattem) $f : \R^n \to \R$.


\section{Grundlagen und Optimalitätsbedingungen}


$U \subset \R^n$ sei im Folgenden stets eine offene Menge.


\subsection{Grundlagen}

Für $x \in \R^n, A \in \R^{n\times n}$ ist
\[
	\|x\| := \sqrt{x^T x}, \qquad
	\|A\| := \max_{\|x\|=1} \|Ax\|.
\]
Für $F: U \to \R^m$, $F(x) = (f_1(x), \dotsc, f_m(x))^T$ ist
\[
	F'(x) = \begin{pmatrix}
		\pddx[x_1]{f_1}(x) & \cdots & \pddx[x_n]{f_1}(x) \\
		\vdots & \ddots & \vdots \\
		\pddx[x_1]{f_m}(x) & \cdots & \pddx[x_n]{f_m}(x) \\
	\end{pmatrix} \in \R^{m\times n}
\]
Für $f: U \to \R$ bezeichnet
\begin{align*}
	\nabla f(x)
	&= f'(x)^T
	= \begin{pmatrix}
		\pddx[x_1]{f} \\
		\vdots \\
		\pddx[x_n]{f} \\
	\end{pmatrix}, \\
	\nabla^2 f(x)
	&= \begin{pmatrix}
		\pddx[x_1^2]{f} & \cdots & \f{\partial^2 f}{\partial x_1 \partial x_n} \\
		\vdots & \ddots & \vdots \\
		\f{\partial^2 f}{\partial x_n \partial x_1} & \cdots & \f{\partial^2 f}{\partial x_n^2} \\
	\end{pmatrix} \in \R^{n\times n}
\end{align*}
den Gradienten $\nabla f$ und die Hessematrix $\nabla^2 f$ die Hesse-Matrix.
Die Bezeichnung $\nabla^2$ wird in der Literatur nicht einheitlich verwendet.

Für $d \in \R^n$ bezeichnen wir die \emph{Richtungsableitung} als
\[
	\pddx[d]{f}(x)
	= \lim_{t\to 0} \f {f(x+td) - f(x)}{t}
	= f'(x) d = \nabla f(x)^T d.
\]

\begin{df} \label{2.1}
	Zu $x_0,x_1 \in \R^n$ heißt
	\[
		[x_0, x_1] := \{ x_0 + t(x_1-x_0) : t \in [0,1] \}
	\]
	die \emph{Verbindungsstrecke} zwischen $x_0, x_1$.
\end{df}

\begin{st} \label{2.2}
	Sei $f: U \to \R$ stetig differenzierbar, $x_0, x_1 \in U$ und $[x_0, x_1] \subset U$.

	Dann ist die Funktion $g: t \mapsto f(x_t)$ (wobei $x_t := x_0 + t(x_1 - x_0) = (1-t) x_0 + tx_1$) auf einer Umgebung von $[0,1]$ stetig differenzierbar und
	\[
		g'(t) = \nabla f(x_t)^T (x_1 - x_0).
	\]
	Ist $f$ zweimal stetetig differenzierbar, so auch $g$ und es gilt
	\[
		g''(t) = (x_1 - x_0)^T \nabla^2 f(x_t) (x_1 - x_0).
	\]
	\begin{proof}
		Übungsaufgabe
	\end{proof}
\end{st}

\begin{st}[Linearer und quadratischer Taylor] \label{2.3}
	Sei $f : U \to \R$ stetig differenzierbar, $x\in U, \eps > 0$, so dass $B_\eps(x) \subset U$.
	Dann gilt
	\begin{enumerate}[(a)]
		\item
			Für alle $d \in B_\eps(0)$ existiert ein $s \in [0,1]$ mit
			\[
				f(x+d) = f(x) + \nabla f(x + sd)^T d.
			\]
			Außerdem gilt für alle $d \in B_\eps(0)$
			\[
				f(x+d) = f(x) + \nabla f(x)^T d + p(d)
			\]
			und das Restglied $p: B_\eps(0) \to \R$ erfüllt $p(d) = o(d)$, d.h.
			\[
				\lim_{d\to 0} \f {p(d)}{\|d\|} = 0.
			\]
		\item
			Ist $f: U \to \R$ zweimal stetig differenzierbar, dann existieren für alle $d \in B_\eps(0)$ ein $s \in [0,1]$mit
			\[
				f(x+d)
				= f(x) + \nabla f(x)^T d + \f 12 d^T \nabla^2 f(x+sd) d.
			\]
			Außerdem gilt für alle $d \in B_\eps(0)$
			\[
				f(x+d) = f(x) + \nabla f(x)^T d + \f 12 d^T \nabla^2 f(x) d + p(d)
			\]
			und $p: B_\eps(0) \to \R$ gilt $p = o(\|d\|^2)$, d.h.
			\[
				\lim_{d\to 0} \f{|p(d)|}{\|d\|^2} = 0.
			\]
	\end{enumerate}
	\begin{proof}
		Funktioniert mit eindimensionalem Taylor und \ref{2.2}.
	\end{proof}
\end{st}

\begin{lem} \label{2.4}
	Sei $F: U \to \R^m$ stetig differenzierbar, $x\in U$ und $\eps > 0$, sodass $B_\eps(x) \subset U$.

	Dann gilt für alle $d \in B_\eps(0)$
	\begin{align*}
		F(x+d)
		&= F(x) + \int_0^1 F'(x+td) d \dx[t] \\
		&= F(x) + \bigg( \int_0^1 F'(x+td) \dx[t] \bigg) d.
	\end{align*}
	Insbesondere gilt
	\[
		\| F(x+d) - F(x) \|
		\le \|d\| \sup_{t\in [0,1]} \|F'(x+td)\|.
	\]
	Diese Abschätzung wird oft auch \emph{mehrdimensionaler Mittelwertsatz} genannt.
\coursetimestamp{21}{10}{2013}

	Außerdem ist $F(x+d) = F(x) + F'(x)d + p(d)$ und $p: B_\eps(0) \to \R^m$ erfüllt $p(d) = o(d)$, d.h.
	$\lim_{d\to 0} \f {\|p(d)\|}{\|d\|} = 0$.

	\begin{proof}
		Da $U$ offen, existiert $\eps > 0$, so dass $[x- \eps d, x + d + \eps d] \subset U$.
		Definiere $f: (-\eps, 1 + \eps) \to \R^n$ durch
		\[
			f(t) := F(x + td) = F(g(t))
		\]
		mit $g:(-\eps, 1 + \eps) \to \R^n : t \mapsto x + td$.
		Es gilt
		\[
			f'(t)
			= F'(g(t)) g'(t)
			= F'(x + td) d.
		\]
		und damit
		\[
			F(x+d) - F(x)
			= f(1) - f(0)
			= \int_0^1 f'(t) \dx[t]
			= \int_0^1 F'(x + td) d \dx[t].
		\]

		Die zweite Behauptung ist die Definition der totalen Differenzierbarkeit.
	\end{proof}
\end{lem}

\begin{df} \label{2.5}
	Eine symmetrische Matrix $A = (a_{ij}) \in \R^{n\times m}$ heißt
	\begin{enumerate}[(a)]
		\item
			\emph{positiv semi-definit}, falls $x^TAx \ge 0$ für alle $x \in \R^n$,
		\item
			\emph{positiv definit}, falls $x^TAx > 0$ für alle $x \in \R^n \setminus \{0\}$.
	\end{enumerate}
	Analog definiert man \emph{negativ semi-definit} und \emph{negativ definit}.
\end{df}

\begin{st} \label{2.6}
	Zu einer symmetrischen Matrix $A \in \R^{n\times n}$ existiert eine Orthonormalbasis aus Eigenvektoren $v_1, \dotsc, v_n \in \R^n$ mit dazugehörigen Eigenwerten $\lambda_1, \dotsc, \lambda_n \in \R$, als
	\[
		v_j^T v_k
		= \delta_{jk}
		= \begin{cases}
			1 & j = k \\
			0 & \text{sonst}
		\end{cases},\qquad
		Av_k = \lambda_k v_k
	\]
	Es ist also
	\[
		A = \sum_{k=1}^n \lambda_k v_k v_k^T
	\]
	und mit $\lambda_{\text{min}}(A) := \min_{k=1,\dotsc,n} \lambda_k$, $\lambda_{\text{max}}(A) := \max_{k=1,\dotsc,n} \lambda_k$ gilt
	\[
		\lambda_{\text{min}}(A)
		= \min_{x\in\R^n} \f {x^T A x}{\|x\|^2}
		\le \f {x^T A x}{\|x\|^2}
		\le \max_{x\in\R^n} \f {x^TA x}{\|x\|^2}
		= \lambda_{\text{max}}(A)
	\]
	und
	\[
		\|A\|
		= \sup_{x\neq 0} \f{\|Ax\|}{\|x\|}
		= \max \{|\lambda_k| : k=1, \dotsc, n \}
	\]
	Insbesondere ist $A$ positiv semi-definit genau dann, wenn alle Eigenwerte $\ge 0$ und positiv definit genau dann, wenn alle Eigenwerte $>0$ sind,
	entsprechend negativ semi-definit, wenn alle Eigenwerte $\le 0$ und negativ semi-definit, wenn alle Eigenwerte $<0$ sind.
\end{st}

\begin{df} \label{2.7}
	Für eine invertierbare Matrix $A \in \R^{n\times n}$ heißt
	\[
		\kappa(A)
		:= \|A\| \|A^{-1}\|
	\]
	\emph{Kondition}.

	Für eine symmetrische, positiv definite Matrix $A \in \R^{n\times n}$ gilt offenbar
	\[
		\kappa(A)
		= \lambda_{\text{max}}(A) \lambda_{\text{max}}(A^{-1})
		= \f {\lambda_{\text{max}}(A)}{\lambda_{\text{min}}(A)}.
	\]
\end{df}

\begin{nt} \label{2.8}
	Die Kondition ist ein Maß für die Fehlerverstärkung durch $A^{-1}$ (d.h. durch Lösen eines LGS), denn
	\begin{align*}
		\dfrac {A^{-1}(b+\delta) - A^{-1}b}{\|A^{-1}b\|}
		&= \f {\|A^{-1}\delta\|}{\|A^{-1}b\|} \\
		&\le \|A^{-1}\| \|A\| \f {\|\delta\|}{\|A\|\|A^{-1}b\|} \\
		&\le \kappa(A) \f {\|\delta\|}{\|b\|}.
	\end{align*}
\end{nt}

\section{Optimalitätsbedingungen}

\begin{df} \label{2.9}
	Sei $X \subset \R^n$ und $f: X \to \R$.
	Ein Punkt $x \in X$ heißt
	\begin{enumerate}[(a)]
		\item
			\emph{globales Minimum von $f$ in $X$}, falls
			\[
				\forall y \in X: f(x) \le f(y),
			\]
		\item
			\emph{lokales Minimum von $f$}, falls
			\[
				\exists \eps > 0 \forall y \in X \cap B_\eps(x): f(x) \le f(y),
			\]
		\item
			\emph{striktes globales Minimum von $f$ in $X$}, falls
			\[
				\forall y \in X \setminus \{x\}: f(x) < f(y)
			\]
		\item
			\emph{striktes lokales Minimum von $f$}, falls
			\[
				\exists \eps > 0 \forall y \in (X \setminus \{x\}) \cap B_\eps(x): f(x) < f(y),
			\]
	\end{enumerate}
	Analog definiert man die entpsrechenden Maxima.
\end{df}

\begin{st}[Notwendige Optimalitätsbedingung 1. Ordnung] \label{2.10}
	Sei $U \subset \R^n$ offen, $f: U \to \R$ stetig differenzierbar.
	Ist $x \in U$ ein lokales Minimum von $f$, dann gilt $\nabla f(x) = 0$.

	Punkte mit $\nabla f(x) = 0$ heißen \emph{stationäre Punkte}
	\begin{proof}
		Für jedes $d \in \R^n$ gilt
		\[
			\nabla f(x)^T d
			= \lim_{t \to 0^+} \f {f(x+td) - f(x)}{t}
			\ge 0
		\]
		Für $d := - \nabla f(x)$ folgt $-\|\nabla f(x)\|^2 \ge 0$ und damit $\nabla f(x) = 0$.
	\end{proof}
\end{st}

\begin{st}[Notwendige Optimalitätsbedingung 2. Ordnung] \label{2.11}
	Sei $U \subset \R^n$ offen, $f: U \to \R$ zweimal stetig differenzierbar.
	Ist $x \in U$ ein lokales Minimum von $f$, dann ist $x$ ein stationärer Puunkt von $f$ und $\nabla^2 f$ ist positiv semi-definit, also
	\begin{enumerate}[(a)]
		\item
			$\nabla f(x) = 0$,
		\item
			$x^T \nabla^2 f(x) x \ge 0$ für alle $x \in \R^n$.
	\end{enumerate}
	\begin{proof}
		Teil (a) ergibt sich aus \ref{2.10}, zeige also (b):
		Sei $d \in \R^n$ und $\eps > 0$ so klein, dass $B_\eps(x) \subset U$ und $x$ globales Minimum in $B_\eps(x)$.
		Verwende \ref{2.3} mit $td$ anstelle von $d$:
		\[
			f(x+td)
			= f(x) + t \nabla f(x)^T d + \f {t^2}2 d^T \nabla^2 f(x) d + p(td
		\]
		für alle $td \in B_\eps(0)$.
		Also gilt für alle $|t| < \f \eps{\|d\|}$
		\[
			\nabla^2 f(x) d
			\ge - \f 2{t^2} p(td)
			\to 0
		\]
		für $t \to 0$.
	\end{proof}
\end{st}

\begin{st}[Hinreichende Optimalitätsbedingungen 2. Ordnung]
	Sei $U \subset \R^n$ offen, $f: U \to \R$ zweimal stetig differenzierbar.
	Ist $x \in U$ ein stationärer Punkt mit positiv definiter Hesse-Matrix, d.h.
	\begin{enumerate}[(a)]
		\item
			$\nabla f(x) = 0$,
		\item
			$d^T \nabla^2 f(x) d > 0$ für alle $d \in \R^n \setminus \{0\}$.
	\end{enumerate}
	Dann ist $x$ ein striktes lokales Minimum von $f$.
	\begin{proof}
		Nach \ref{2.6} ist mit (b) $\my := \lambda_{\text{min}} (\nabla^2 f(x)) > 0$ und
		\[
			d^T \nabla^2 f(x) d \ge \my \|d\|^2.
		\]
		Wie in \ref{2.11} ist
		\[
			f(x+d)
			= f(x) + \nabla f(x)^T d + \f 12 d^T \nabla^2 f(x) d + p(d)
		\]
		und $p(d) \le \f \my4 \|d\|^2$ für hinreichend kleine $d$.

		Also
		\[
			f(x+d)
			\ge f(x) + \f \my2 \|d\|^2 - \f \my4 \|d\|^2
			= f(x) + \f \my4 \|d\|^2
		\]
		und damit $f(x+d) > f(x)$ für alle hinreichend kleinen $d \neq 0$.
	\end{proof}
\end{st}

\begin{ex} \label{2.13}
	\begin{enumerate}[(a)]
		\item
			N1 ist nicht hinreichend.
			$f(x) = -x^2$ hat im kritischen Punkt ein Maximum,
			$f(x) = x^3$ hat in $x=0$ weder ein Maximum, noch ein Minimum.
			Zum Namen „Sattelpunkt“
			\[
				f(x_1,x_2) = x_1^2 - x_2^2
			\]
			bildet einen „Sattel“ in $(0,0)$.
		\item
			N2 ist nicht hinreichend, betrachte $f(x) = x^3$.
		\item
			H2 ist nicht notwendig, betrachte $f(x) = x^4$.
	\end{enumerate}
\end{ex}


\subsection{Konvexität}


\begin{df} \label{2.14}
	Eine Menge $X \subset \R^n$ heißt \emph{konvex}, wenn alle Verbindungsstrecken zweier Punkte darin enthalten sind, also für alle $x,y \in X, \lambda \in [0,1]$
	\[
		x_\lambda
		:= (1-\lambda)x + \lambda y \in X.
	\]
\end{df}

\begin{df} \label{2.15}
	Sei $X \subset \R^n$ konvex.
	$f: X \to \R$ heißt
	\begin{enumerate}[(a)]
		\item
			\emph{konvex}, falls für alle $x,y \in X, \lambda \in [0,1]$.
			\[
				f((1-\lambda)x + \lambda y)
				\le (1-\lambda) f(x) + \lambda f(y).
			\]
		\item
			\emph{strikt konvex}, falls für alle $x,y \in X, x \neq y, \lambda \in (0,1)$.
			\[
				f((1-\lambda)x + \lambda y)
				< (1-\lambda) f(x) + \lambda f(y).
			\]
		\item
			\emph{gleichmäßig konvex}, falls $\my > 0$ existiert, sodass für alle $x,y \in X, \lambda \in [0,1]$.
			\[
				f((1-\lambda)x + \lambda y) + \my \lambda(1-\lambda) \|y-x\|^2
				< (1-\lambda) f(x) + \lambda f(y).
			\]
			Zur Motivation betrachte auch \ref{2.17}.
	\end{enumerate}
\end{df}

\begin{st} \label{2.16}
	Sei $U \subset \R^n$ offen und $f: U \to \R$ stetig differenzierbar.
	$f$ ist auf einer konvexen Menge $X \subset U$
	\begin{enumerate}[(a)]
		\item
			genau dann konvex, wenn für alle $x,y \in X$
			\[
				\nabla f^T(x) (y-x)
				\le f(y) - f(x).
			\]
		\item
			genau dann strikt konvex, wenn für alle $x,y \in X, x \neq y$
			\[
				\nabla f^T(x) (y-x)
				< f(y) - f(x).
			\]
		\item
			genau dann gleichmäßig konvex, wenn ein $\my > 0$ existiert, sodass für alle $x,y \in X$.
			\[
				\nabla f(x)^T (y-x) + \my \|y-x\|
				\le f(y) - f(x)
			\]
	\end{enumerate}
	\begin{proof}
		\begin{enumerate}[(a)]
			\item
				\begin{seg}[„$\implies$“]
					Ist $f$ konvex, dann ist
					\[
						\nabla f(x)^T (y-x)
						= \lim_{t \to 0} \f {f(x+t(y-x)) - f(x)}t
						= \lim_{t \to 0} \f {f((1-t)x + ty) - f(x)}t
						\le \lim_{t \to 0} \f{(1-t)f(x) + tf(y) - f(x)}t
						= f(y) - f(x).
					\]
				\end{seg}
				\begin{seg}[„$\impliedby$“]
					Sei $\lambda \in [0,1]$ und $x,y \in X$.
					Dann gilt
					\begin{align*}
						f(x) - f(x_\lambda)
						&\ge \nabla f(x_\lambda)^T(x-x_\lambda) \\
						f(y) - f(x_\lambda)
						&\ge \nabla f(x_\lambda)^T(y-x_\lambda)
					\end{align*}
					Damit ist
					\begin{align*}
						(1-\lambda) f(x) + \lambda f(y)
						&= (1-\lambda)(f(x) - f(x_\lambda)) + \lambda (f(y) - f(x_\lambda)) + f(x_\lambda) \\
						&\ge (1-\lambda) \nabla f(x_\lambda)^T (x-x_\lambda) + \lambda \nabla f(x_\lambda)^T (y-x_\lambda) + f(x_\lambda) \\
						&= \nabla f(x_\lambda)^T \big( (1-\lambda) x + \lambda y - x_\lambda \big) + f(x_\lambda) \\
						&= f(x_\lambda)
					\end{align*}
				\end{seg}
			\item
				\begin{seg}[„$\implies$“]
					Sei $f$ strikt konvex, $x, y \in X, x\neq y$, setze
					\[
						x_{\f 12} := \f {f(x) + f(y)}2.
					\]
					Es gilt
					\[
						\nabla f(x)^T (y-x)
						= 2 \nabla f(x)^T (x_{\f 12} - x)
						\le 2 (f(x_{\f 12}) - f(x))
						< 2 (\f {f(x) + f(y)}2 - f(x))
						= f(y) - f(x).
					\]
				\end{seg}
				\begin{seg}[„$\impliedby$“]
					Analog zu (a) „$\impliedby$“
				\end{seg}
			\item
				\begin{seg}[„$\implies$“]
					Es gilt
					\[
						\nabla f(x)^T (y-x)
						= \lim_{t\to 0} \f {f(x_t) - f(x)}t
						\le \lim_{t\to 0} \f 1t \Big( (1-t)f(x) + tf(y) - \my t(1-t)\|y-x\|^2 - f(x) \Big)
						= f(y) - f(x) - \my \|y-x\|^2.
					\]
				\end{seg}
				\begin{seg}[„$\impliedby$“]
					Gehe vor, wie in (a) $\impliedby$:
					\begin{align*}
						(1-\lambda) f(x) + \lambda f(y)
						&= (1-\lambda) (f(x) - f(x_\lambda)) + \lambda (f(y) - f(x_\lambda)) + f(x_\lambda) \\
						&\ge (1-\lambda) \Big(\nabla f(x_\lambda)^T (x-x_\lambda) + \my \|x-x_\lambda\|^2 \Big) \\
							+ \lambda \Big( \nabla f(x_\lambda)^T (y-x_\lambda) + \my \|y-x_\lambda\|^2 \Big) + f(x_\lambda) \\
						&= f(x_\lambda) + (1-\lambda)\my \underbrace{\|x - x_\lambda\|^2}_{\lambda^2 \|y-x\|^2} + \lambda \my \underbrace{\|y-x_\lambda\|^2}_{(1-\lambda)^2\|y-x\|^2} \\
						&=  f(x_\lambda) + (1-\lambda)\lambda \my \|y-x\|^2.
					\end{align*}
				\end{seg}
		\end{enumerate}
	\end{proof}
\end{st}

\begin{st} \label{2.17}
	Sei $X \subset \R^n$ offen und kovex und $f: X \to \R^n$ zweimal stetig differenzierbar.
	$f$ ist auf $X$
	\begin{enumerate}[(a)]
		\item
			genau dann konvex, wenn $\nabla^2 f(x)$ für alle $x \in X$ positiv semidefinit ist, also für alle $x \in X, d \in \R^n$
			\[
				d^T \nabla^2 f(x) d \ge 0
			\]
		\item
			strikt konvex, wenn $\nabla^2 f(x)$ für alle $x \in X$ positiv semidefinit ist, also für alle $x \in X, d \in \R^n \setminus \{0\}$
			\[
				d^T \nabla^2 f(x) d > 0.
			\]
			Die Umkehrung gilt nicht.
		\item
			genau dann gleichmäßig konvex, wenn $\nabla^2 f(x)$ gleichmäßig positiv semidefinit ist, d.h. es existiert $\my > 0$, sodass für alle $x \in X, d \in \R^n \setminus \{0\}$
			\[
				d^T \nabla^2 f(x) d \ge \my \|d\|^2.
			\]
			Die Umkehrung gilt nicht.
	\end{enumerate}
	\begin{proof}
		\begin{enumerate}[(a)]
			\item
				\begin{seg}[„$\implies$“]
					Sei $f$ konvex und $d \in \R^n$.
					Für hinreichend kleine $t > 0$ gilt mit Taylor und $\ref{2.16}$
					\[
						\f {t^2}2 d^T \nabla^2 f(x) d + p(td)
						= f(x+td) - f(x) - t \nabla f(x)^T d
						\ge 0,
					\]
					also
					\[
						d^T \nabla^2 f(x) d
						\ge - \f {2 p(td)}{t^2} \to 0
					\]
					für $t \to 0$.
				\end{seg}
				\begin{seg}[$\impliedby$]
					Seien $x,y \in X$, dann gilt nach \ref{2.3}, dass $s \in [0,1]$ existiert mit
					\[
						f(y)
						= f(x) + \nabla f(x)^T d  + \f 12 (y-x)^T \nabla^2 f(x+s(y-x))(y-x)
						\ge f(x) + \nabla f(x)^T (y-x)
					\]
					und somit ist $f$ nach \ref{2.16} konvex.
				\end{seg}
			\item
				Analog zu $\impliedby$ in (a).
			\item
				\begin{seg}[„$\implies$“]
					Analog wie „$\implies$“ in (a) erhalten wir
					\[
						d^T \nabla^2 f(x) d
						\ge - \f {2 p(td)}{t^2} + \my \|d\|^2
						\to \my \|d\|^2.
					\]
				\end{seg}
				\begin{seg}[„$\impliedby$“]
					Analog wie in (a) existiert zu $x,y \in X$ ein $s \in [0,1]$ mit
					\begin{align*}
						f(x)
						&= f(x) + \nabla f(x)^T (y-x) + \f 12 (y-x)^T \nabla^2 f(x_s) (y-x) \\
						&\ge f(x) + \nabla f(x)^T (y-x) + \f 12 \my \|y-x\|^2,
					\end{align*}
					also ist $f$ nach \ref{2.16} gleichmäßig konvex.
				\end{seg}
		\end{enumerate}
	\end{proof}
\end{st}

\begin{ex} \label{2.18}
	In \ref{2.17} gilt die Rückrichtung in (b) im Allgemeinen nicht.
	Betrachte dazu $f(x) = x^4$, dann ist $f''(x)\big|_{x=0} = 0$, trotz Minimum in $x=0$.
\end{ex}

\coursetimestamp{28}{10}{2013}

\begin{st} \label{2.19}
	Sei $X \subset \R^n$ konvex, $f: X \to \R$ konvex.
	Dann gelten folgende Aussagen
	\begin{enumerate}[(a)]
		\item
			Jedes lokale Minimum von $f$ ist auch globales Minimum von $f$ von $x$.
		\item
			Ist $f$ strikt konvex, dann besitzt $f$ höchstens ein lokales Minimum.
			Nach (a) ist jedes lokale Minimum das eindeutige, strikte globale Minimum.
		\item
			Ist $f$ stetig differenzierbar auf $O \subset X$ offen, dann ist jeder stationäre Punkt in $X$ ein globales Minimum.
	\end{enumerate}
	\begin{proof}
		\begin{enumerate}[(a)]
			\item
				Sei $x \in X$ kein globales Minimum von $f$.
				Dann existiert $y \in X$ mit $f(y) < f(x)$.
				Mit der Konvexität folgt für alle $f \in (0,1]$
				\[
					f(x+t(y-x))
					\le (1-t)f(x) + t f(y)
					< f(x).
				\]
				mit entsprechend kleinem $t$ finden wir damit in jeder Umgebung von $x$ ein $\xi$ mit $f(\xi) < f(x)$, also ist $x$ kein lokales Minimum.
			\item
				Seien $x,y \in X$ zwei lokale Minima von $f$.
				Nach (a) sind $x,y$ globale Minima, also $f(x) = f(y)$.
				Angenommen $x \neq y$, dann wäre wegen der strikten Konvexität
				\[
					f(\f {x+y}2)
					< \f{f(x) + f(y)}2
					= f(x)
					= f(y),
				\]
				ein Widerspruch dazu, dass $x,y$ globale Minima waren.
			\item
				Ist $x \in X$ ein stationärer Punkt, dann ist nach \ref{2.16} für alle $y\in X$
				\[
					f(y) - f(x)
					\ge \nabla f(x)^T (y-x)
					= 0,
				\]
				also $f(y) \ge f(x)$ und somit $x$ globales Minimum.
		\end{enumerate}
	\end{proof}
\end{st}


\section{Das Gradientenverfahren}


In diesem Abschnitt sei $f: \R^n \to \R$ eine stetig differenzierbare Funktion.

\begin{conv*}
	Im folgenden schreiben wir für eine Folge von Vektoren in $\R^n$,
	\[
		x_0, x_1, x_2, \dotsc \in \R^n.
	\]
	teilweise auch
	\[
		x^{(0)} = \begin{pmatrix}
			x_1^{(0)} \\ \vdots \\ x_n^{(0)}
		\end{pmatrix},
		x^{(1)} = \begin{pmatrix}
			x_1^{(1)} \\ \vdots \\ x_n^{(1)}
		\end{pmatrix},
		\dotsc \in \R^n.
	\]
\end{conv*}

\subsection{Richtung des steilsten Abstiegs}

In der Umgebung eines Punktes $x_0 \in \R^n$ ist (gemäß Taylorapproximation)
\[
	f(x) \approx f(x_0) + \nabla f(x_0)^T (x-x_0).
\]
Nach Übungsaufgabe 3.1 %fixme: reference
ist das Minimum der linearen Approximation innerhalb einer Kugel um $x_0$
\[
	\min_{\|x-x_0\|=1} \big( f(x_0) + \nabla f(x_0)^T (x-x_0) \big)
	= f(x_0) - \nabla f(x_0)^T \f {\nabla f(x_0)}{\|\nabla f(x_0)\|}.
\]
In der linearen Approximation ist $x-x_0 = -s \nabla f(x_0)$ mit $s > 0$, also die Richtung des steilsten Abstiegs von $f$.

Ein naiver Ansatz für einen Algorithmus wäre folgender:
Beginne mit $x_0 uin \R^n$ und iteriere
\[
	x_{k+1} = x_k - s_k \nabla f(x_k),
\]
mit noch zu bestimmenden Schrittweiten $s_k > 0$.
Solche Verfahren werden \emph{Gradientenverfahren} genannt.

\subsection{Armijo-Schrittweitenregel}

In der linearen Approximation erwarten wir, dass für $s_k > 0$ und $d_k \in \R^n$ (hier z.B. $d_k = -\nabla f(x_k)$)
\[
	f(x_{k+1})
	= f(x_k + s_k d_k)
	\approx f(x_k) + s_k \nabla f(x_k)^T d_k.
\]
Die Armijo-Regel besagt:
vergleiche die tatsächlie Abnahme der Zielfunktion $f(x_k) - f(x_k + s_kd_k)$ mit der aus der linearen Approximation erwarteten Abnahme $-s_k \nabla f(x_k)^T d_k$.
Nur wenn die tatsächliche Abnahme einen voregegebenen Bruchteil (z.B. $\gamma = 1\%$) der erwarteten Abnahme erreicht, also wenn die sogenannte \emph{Armijo-Regel}
\[
	f(x_{k+1})
	\le f(x_k) + \gamma s_k \nabla f(x_k)^T d_k
\]
erfüllt ist, dann wird die Schrittweite akzeptiert, sonst verkürzt (z.B. auf $\beta s_k$ mit $\beta = \f 12 < 1$).

\begin{alg}
	\begin{algorithmic}
		\Require{$\beta \in (0,1), \gamma \in (0,1), x_k, d_k$}
		\State $s_k \gets \f 1{\beta}$
		\Repeat
			\State $s_k \gets \beta s_k$
			\State $x_{k+1} \gets x_k + s_k d_k$
		\Until $f(x_{k+1}) \le f(x_k) + \gamma s_k \nabla f(x_k)^T d_k$
	\end{algorithmic}
\end{alg}

\begin{df} \label{2.20}
	Ein Vektor $0 \neq d \in \R^n$ heißt \emph{Abstiegsrichtung} von $f$ im Punkt $x$, falls
	\[
		\nabla f(x)^T d < 0.
	\]
\end{df}

\begin{st} \label{2.21}
	Ist $d$ eine Abstiegsrichtung für $f$ im Punkt $x$, so existiert für jedes $\gamma \in (0,1)$ ein $t > 0$ mit
	\[
		f(x+sd)
		\le f(x) + \gamma s \nabla f(x)^T d
	\]
	für alle $s \in [0,t]$, d.h. die Armijo-Schrittweitenregel terminiert nach endlich vielen Schritten.
	\begin{proof}
		Für $s \to 0$ ist
		\[
			\f{f(x_k+sd)-f(x_k)}{s} - \gamma \nabla f(x)^T d
			\to (1 - \gamma) \underbrace{\nabla f(x_k)^T d}_{<0}
			< 0,
		\]
		also für hinreichend kleine $s>0$ ist
		\[
			\f{f(x_k+sd)-f(x_k)}{s} - \gamma \nabla f(x)^T d
			\le 0,
		\]
		d.h. die Armijo-Bedingung ist erfüllt.
	\end{proof}
\end{st}

\begin{ex} \label{2.22}
	Aus \ref{2.21} folgt insbesondere: in Richtung einer Abstiegsrichtung nimmt eine Funktion lokal ab.

	Die Umkehrung gilt hingegen nicht.
	Betrachte dazu $f(x) = -x^2$ an der Stelle $x = 0$: in Richtung $d=\pm 1$ nimmt die Funktion ab, aber beides sind keine Abstiegsrichtungen gemäß \ref{2.20}.
\end{ex}

\begin{nt} \label{2.23}
	Die Armijo-Regel ist ein spezieller, einfacher Fall für ein sogenanntes “Trust-Region”-Verfahren.
	In der aktuellen Iterierten wird $f$ dabei durch seine lineare Näherung
	\[
		f(x)
		\approx f(x_k) + \nabla f(x_k)^T (x-x_k)
	\]
	ersetzt.
	Wir vertrauen dieser Näherung jedoch nur für $x$ mit
	\[
		\|x - x_k\| \le \Delta_k,
	\]
	der sogenannten “Trust-Region”.
	Wir lösen also das restringierte Optimierungsproblem
	\[
		f(x_k) + \nabla f(x_k)^T (x-x_k) \to \min!
		\udN
		x \in B_{\Delta_k}(x_k)
	\]
	und erhalten $x_{k+1} = x_k - s_k \nabla f(x_k)$ ($s_k = \f {\Delta_k}{\|\nabla f(x_k)\|}$).
	Vergleiche anschließend die vorhergesagte Abnahme (\pred) mit tatsächlicher Abnahme (\ared)
	\begin{align*}
		\pred_k &= -s_k \nabla f(x_k) &
		\ared_k &= f(x_k) - f(x_{k+1})
	\end{align*}
	Falls $\f {\pred_k}{\ared_k} < \gamma$, so wird der Schritt mit verkleinerten $\Delta_k$ wiederholt.
\end{nt}

\coursetimestamp{30}{10}{2013}

% 2.2.3
\subsection{Konvergenz des Gradientenverfahrens}

Das Gradientenverfahren hat zwei Probleme:
Es könnte ein lokales Maximum gefunden werden, oder das Gradientenverfahren könnte nicht konvergieren ($f(x) = e^{-x}$).

\begin{st}[Präkonvergenz des Gradientenverfahrens] \label{2.24}
	Das Gradientenverfahren % fixme: algorithm reference
	terminiert entweder nach endlich vielen Schritten an einem stationären Punkt $x_k$, oder er erzeugt eine Folge $(x_k)_{k\in\N}$ mit
	\begin{enumerate}[(a)]
		\item
			$f(x_{k+1}) < f(x_k)$,
		\item
			jeder Haufungspunkt von $(x_k)$ ist eine stationärer Punkt von $f$.
	\end{enumerate}
	\begin{proof}
		Terminiert % fixme: reference algorithm
		nach endlich vielen Schritten an einem $x_k$, so gilt $\nabla f(x_k) = 0$.

		Der Algorithmus konvergiere nicht.
		\begin{enumerate}[a)]
			\item
				Nach der Armijo-Regel gilt wegen $\nabla f(x_k) \neq 0$
				\[
					f(x_{k+1})
					\le f(x_k) - \gamma s_k \nabla f(x_k)^T \nabla f(x_k)
					< f(x_k).
				\]
			\item
				Sei $x \in \R^n$ ein Häufungspunkt von $x_k$, d.h. es existiert eine unendlich Indexmenge $L \subset \N$, so dass
				\[
					\lim_{L \ni l \to \infty} x_l = x.
				\]
				Da $f$ stetig, gilt auch $\lim_{L \ni l \to \infty} f(x_l) = f(x)$.
				Wegen (a) ist $(f(x_k))_{k\in\N}$ streng monoton fallend und damit $f(x_k) \ge f(x)$ für alle $k \in \N$.
				Also konvergiert $(f(x_k))_{k\in\N}$ und $\lim_{k\to \infty} f(x_k) = f(x)$.

				Nach der Armijo-Regel gilt
				\[
					f(x_0) - f(x)
					= \sum_{k=0}^\infty (f(x_k) - f(x_{k+1}))
					\ge \sum_{k=0}^\infty \gamma s_k \| \nabla f(x_k) \|^2,
				\]
				also ist $\lim_{k\to\infty} s_k \|\nabla f(x_k)\|^2 = 0$ und insbesondere auch für die Teilfolge mit der Indexmenge $L$:
				\[
					\lim_{L \ni l \to \infty} s_ll \|\nabla f(x_l)\|^2 = 0.
				\]
				Also $\lim_{L \ni l \to \infty} s_l = 0$ oder $\lim_{L \ni l \to\infty} \nabla f(x_l) = \nabla f(x) = 0$ ($\nabla f$ stetig).
				Im letzteren Fall ist die Behauptung beweisen, sei also $\lim_{L \ni l \to \infty} s_l = 0$.

				In fast allen (o.B.d.A in allen) Schritten $l \in L$ hat durch die Armijo-Regel eine Schrittweitenverkürzung stattgefunden, d.h. $\f {s_l}{\beta}$ erfüllte die Armijo-Bedingung noch nicht, d.h. für alle $l \in L$ gilt
				\[
					f(x_l - \f {s_l}\beta \nabla f(x_l))
					> f(x_l) - \gamma \f {s_l}{\beta} \| \nabla f(x_l) \|^2.
				\]
				Nach \ref{2.3} gilt:
				Für alle $l \in L$ existiert $\sigma_l \in [0, \f {s_l}{\beta}]$, sodass
				\[
					f(x_l) - f(x_l - \f {s_l}\beta \nabla f(x_l))
					= \f {s_l}\beta \nabla f(x_l - \sigma_l \nabla f(x_l))^T \nabla f(x_l).
				\]
				Also
				\[
					-\gamma \f {s_l}\beta \|\nabla f(x_l)\|^2
					< f(x_l - \f{s_l}\beta \nabla f(x_l)) - f(x_l)
					= -\f {s_l}\beta \nabla f(\underbrace{x_l - \sigma_l \nabla f(x_l)}_{\to x})^T \nabla f(x_l)
				\]
				und somit für $\sigma_l \to 0$
				\[
					- \gamma \|\nabla f(x)\|^2 \le - \| \nabla f(x)\|^2.
				\]
				Aus $\gamma < 1$ folgt damit $\nabla f(x) = 0$.
		\end{enumerate}
	\end{proof}
\end{st}

\begin{lem} \label{2.25}
	Sei $(x_k)_{k\in\N} \subset \R^n, x \in \R^n$.
	Besitzt jede Teilfolge von $(x_k)_{k\in\N}$ wiederum eine (Teil-)Teilfolge, die gegen $x$ konvergiert, so konvergiert die ganze Folge $(x_k)_{k\in\N}$ gegen $x$.
	\begin{proof}
		Angenommen $(x_k)_{k\in\N}$ konvergiert nicht gegen $x$.
		Dann gibt es eine Umgebung von $x$, außerhalb derer unendlich viele Folgenglieder liegen.
		Diese bilden dann aber eine Teilfolge, die keine gegen $x$ konvergente Teilfolge besitzen kann.
	\end{proof}
\end{lem}

\begin{kor} \label{2.26}
	\begin{enumerate}[(a)]
		\item
			Sei $c > 0$, so dass die \emph{Niveaumenge}
			\[
				f^{-1}((-\infty, C))
				:= \{ x \in \R^n : f(x) \le c \}
			\]
			beschränkt ist und in ihr nur ein stationärer Punkt liegt.
			$N_c := f^{-1}((-\infty,c])$ ist abgeschlossen, also kompakt, also wird das Minimum auf $N_c$ angenommen, $f(x) \le f(y)$ für alle $y \in \R^n$ und somit $\nabla f(x) = 0$.
			%fixme: rewrite

			Dann ist dieser stationäre Punkt das globale Minimum von $f$ auf der Niveaumenge und das Gradientenverfahren konvergiert für jeden Startwert $x_0$ mit $f(x_0) \le c$ gegen dieses Minimum.
		\item
			Ist $f$ konvex, dann terminiert Algorithmus 2 % fixme: reference
			entweder nach endlich vielen Schritten an einem globalen Minimum, oder erzeugt eine Folge, deren Häufungspunkte globale Minima sind.
		\item
			Ist $f$ strikt konvex und besitzt ein Minimum und es existiert $C > 0$ mit $f^{-1}((-\infty,C])$ beschränkt, dann konvergiert die Folge aus Algorithmus 2 % fixme: reference
			für jedes $x_0$ mit $f(x_0) \le C$ gegen das globale Minimum.

			%fixme: Reicht auch nur die strikte konvexität?
	\end{enumerate}
\end{kor}



\end{document}
