\chapter{Restringierte Optimierung}



\section{Lineare Optimierung}


\subsection{Motivation}

Zur Schweinefütterung stehen zwei Möglichkeiten zur Verfügung:
Soja zu 1€ pro Einheit, enthält 2 Proteiene und 4 Fett, Kartoffeln zu 2€ pro Einheit, enthält 2 Proteine und 2 Fett.

Futter soll $\ge 10$ Proteine und $\le 12$ Fett besitzen.
Dies führt auf die Minimierungsaufgabe
\[
	x + 2y \to \min!
	\udN
	\begin{cases}
		2x + 2y &\ge 10 \\
		4x + 2y &\le 12 \\
		x &\ge 0 \\
		y &\ge 0
	\end{cases}.
\]

\coursetimestamp{11}{12}{2013}

Jede Ungleichung kann in die Form
\[
	\sum_{j} a_{ij} x_j \le b_i
\]
gebracht werden.
Jede Ungleichung kann in die Form
\begin{align*}
	\sum a_{ij} x_j + \xi &= b_i \\
	\xi &\ge 0
\end{align*}
gebracht werden.
Mit $x_j = x_j' - x_j'', x_j' \ge 0, x_j'' \ge 0$
kann für jede Variable $x_j \ge 0$ gefordert werden.

\begin{df} \label{3.1}
	Sei $A \in \R^{n\times n}, b\in \R^m, c \in \R^n$.
	Das Minimierungsproblem
	\[
		f(x) := c^T x \to \min!
		\quad\udN\quad
		Ax = b, x \ge 0
	\]
	heißt \emph{lineares Optimierungsproblem (engl. linear program) in Normalform}.
	$f$ heißt \emph{Zielfunktion} und
	\[
		\{x \in \R^n : Ax = b \land x \ge 0 \}
	\]
	\emph{zulässiger Bereich}.
\end{df}

\begin{conv*}
	Im Folgenden nutzen wir die Notation
	\[
		x = \begin{pmatrix}
			x_1 \\ \vdots \\ x_n
		\end{pmatrix} \in \R^n
	\]
	und setzen
	\[
		x \ge 0
		:\iff
		x_1, \dotsc, x_n \ge 0.
	\]
\end{conv*}

\begin{nt} \label{3.2}
	Der zulässige Bereich kann einelementig sein (dann ist dieses Element die Lösung) oder leer.

	Möglicherweise existiert auch für einen nicht-leeren zulässigen Bereich keine Lösung, z.B. $n=m=1, A = 0, b=0, c=-1$ (nach unten unbeschränkt).

	Ist der zulässige Bereich nicht-leer und beschränkt, dann existiert eine Lösung.
\end{nt}

\subsection{Geometrische Interpretation}

\begin{df} \label{3.3}
	\begin{enumerate}[(a)]
		\item
			Zu gegebenen Punkten $y^1, \dotsc, y^n \in \R^n$ heißt
			\[
				\sum_{i=1}^m \lambda_i y^i,
				\qquad
				\lambda_i \in [0,1], \sum_{i=1}^n \lambda_i = 1
			\]
			\emph{Konvexkombination}.
			Die Menge aller Konvexkombinationen
			\[
				S(y^1, \dotsc, y^m) := \Set{
					\sum_{i=1}^m \lambda_i y^i:
					\lambda_i \in [0,1], \sum_{i=1}^n \lambda_i = 1
				}
			\]
			heißt (der von $y^1,\dotsc, y^m$) aufgespannte Simplex.

			Damit ist $S(y^1, y^2) = [y^1, y^2]$ die Verbindungsstrecke aus \ref{2.1}.
		\item
			Für $\emptyset \neq K \subset \R^n$ nennen wir $x \in K$ eine \emph{Ecke von $K$}, falls
			\[
				\forall y^1, y^2 \in \R^n, x \in [y^1, y^2] \subset K : x = y^1 \lor x = y^2.
			\]
	\end{enumerate}
\end{df}

Betrachte stets $A \in \R^{m\times n}, b\in \R^m, c \in \R^n$ und
\[
	K:= \{ x \in \R^n : Ax = b, x \ge 0 \}.
\]
$K$ ist offenbar konvex, d.h. $\forall y^1, y^2 \in K : [y^1, y^2] \subset K$.

\begin{ex} \label{3.4}
	Es gilt $0 \in K \iff b = 0$.
	Für $0 \in K$ ist $x = 0$ Ecke von $K$, denn für alle $y^1, y^2 \in K$ mit
	\[
		0 = (1-\lambda) y^1 + \lambda y^2,
		\qquad \lambda \in [0,1]
	\]
	ist $y^1 = 0$ oder $y^2 = 0$.
\end{ex}

\begin{df} \label{3.5}
	Teilmengen $I \subset \{1, \dotsc, n\}$ nennen wir \emph{Indexmengen} und bezeichnen die Anzahl der Elemente in $I$ mit $|I|$.
	Für $I \neq \emptyset$ bilden wir für $A \in \R^{m\times n}, v \in \R^n$.
	\begin{align*}
		A_I &:= (a_{ij})_{\substack{i=1,\dotsc,m \\ j \in I}}
			\in \R^{m\times |I|}, \\
		v_I &:= (v_j)_{j\in I} \in \R^{|I|}
			\in \R^{|I|}.
	\end{align*}
	Jedem $x \in \R^n$ ordnen wir zu
	\[
		I_x = \Big\{ j \in \{1, \dotsc, n\} : x_j > 0 \Big\}
		\subset \{1, \dotsc, n\}
	\]
	Für $x \neq 0$ schreiben wir auch
	\[
		A_x := A_{I_x}
	\]
	und $v_x$ statt $v_{I_x}$.
	Es gilt offenbar für $0 \neq x \in K$
	\[
		b = Ax = A_x x_x.
	\]
	Es gilt auch
	\[
		A_x v_x = A v
	\]
	für alle $v \in \R^n$ mit $I_v \subset I_x$.
\end{df}

\begin{lem} \label{3.6}
	\begin{enumerate}[(a)]
		\item
			Ist $0 \neq x \in K$ keine Ecke von $K$, so existieren $y^1, y^2 \in K, y^1 \neq x, y^2 \neq x, y^1 \neq y^2$, so dass
			\[
				x \in [y^1, y^2], \quad I_{y^1}, I_{y^2} \subset I_x.
			\]
		\item
			$0 \neq x \in K$ ist genau dann eine Ecke von $K$, wenn $A_x$ injektiv ist, also $\rg(A_x) = |I_x|$.
	\end{enumerate}
	\begin{proof}
		\begin{enumerate}[(a)]
			\item
				Ist $x \in K$ keine Ecke, dann existieren $y^1, y^2 \in K, y^1 \neq y^2$ und für $\lambda \in (0,1)$
				\[
					x = (1-\lambda) y^1 + \lambda y^2.
				\]
				Für jedes $j$ gilt
				\[
					y_j^1 > 0 \implies x_j > 0 \qquad
					y_j^2 > 0 \implies x_j > 0
				\]
				also $I_{y^1}, I_{y^2} \subset I_x$.
			\item
				Außerdem ist
				\[
					A_x (y_x^1 - y_x^2)
					= Ay^1 - Ay^2
					= b - b = 0.
				\]
				Wegen $y^1 \neq y^2 \implies y_x^1 \neq  y_x^2$.
				Damit ist $A_x$ nicht injektiv, wenn $x$ keine Ecke ist.

				Sei $A_x$ nicht injektiv, dann existiert $0 \neq y_x \in \R^{|I_x|}$ mit $A_x y_x = 0$, den wir durch Nullen zu $y \in \R^n$ fortsetzen.
				Wähle $\eps > 0$ so klein, dass $\eps |y_j| < x_j$ für alle $j \in I_x$.
				Setze
				\[
					y^1 := x - \eps y,
					\quad
					y^2 := x + \eps y
				\]
				Offenbar ist $x \in [y^1, y^2]$ und $y^1 \neq x, y^2 \neq x$.
				Aus $Ay = A_x y_x = 0$ folgt
				\[
					Ay^1 = Ax = b = Ax = Ay^2.
				\]
				und wegen der Wahl von $\eps$ ist
				\[
					y^1, y^2 \ge 0.
				\]
		\end{enumerate}
	\end{proof}
\end{lem}
