\documentclass{mythesis}

\title{Der reell-algebraische Abbildungsgrad}
\author{Stephan Hilb}

\begin{document}

\maketitle

\begin{abstract}
    Die Windungszahl einer polynomiellen Kurve $\gamma: [0, 1] \to \C^*$ lässt sich mit dem halben Cauchy-Index der rationalen Funktion $\frac{\Re \gamma}{\Im \gamma}$ auf $[0,1]$ definieren, wodurch Überquerungen der rellen Achse orientiert mit $\pm \frac{1}{2}$ gewertet werden.

    In Folgenden werden wir einen algebraischen Algorithmus für die Berechnung des Cauchy-Index und damit der Windungszahl herleiten.

    Wir werden diesen Algorithmus in höheren Dimensionen verallgemeinert formulieren und verifizieren, dass wir damit die natürliche Verallgemeinerung der Windungszahl, den Abbildungsgrad, berechen.
\end{abstract}


\end{document}
