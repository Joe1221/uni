\documentclass{mythesis}

\usepackage{pgfplots}
\pgfplotsset{compat=1.11}

\DeclareDocumentCommand{\Ind}{}{\operatorname{Ind}}

\title{Der reell-algebraische Abbildungsgrad}
\author{Stephan Hilb}

\begin{document}

\maketitle

\begin{abstract}
    Die Windungszahl einer polynomiellen Kurve $\gamma: [0, 1] \to \C^*$ lässt sich als halben Cauchy-Index der rationalen Funktion $\frac{\Re \gamma}{\Im \gamma}$ auf $[0,1]$ definieren, was anschaulich gesprochen die Überquerungen der Kurve an der rellen Achse mit $\pm \frac{1}{2}$ zählt.
    Ein reell-algebraischer Algorithmus, basierend auf polynomieller Restbildung oder sogenannten Sturm-Ketten, erlaubt die explizite Berechnung dieses Cauchy-Index und damit auch der Windungszahl von $\gamma$.

    \fixme Wir werden diesen Algorithmus in höheren Dimensionen verallgemeinert formulieren und verifizieren, dass wir damit unter bestimmten Voraussetzungen die natürliche Verallgemeinerung der Windungszahl, den Abbildungsgrad, berechnen können.
\end{abstract}


\section{Der univariate Fall, Windungszahl und Cauchy-Index}

\subsection{Geometrische Motivation}

Betrachten wir eine geschlossene Kurve $\gamma: [a,b] \to \C^*$, wie sie in \ref{fig:closed_curve} zu sehen ist.
Um die Windungszahl von $\gamma$ zu bestimmen ist es anschaulich hinreichend, die (orientierten) Übergänge der Kurve an der positiven reellen Achse zu zählen.
Alternativ können wir die Übergänge über die gesamte reelle Achse zählen, wie in \ref{fig:closed_curve_counting} dargestellt und diese anschließend halbieren.
\begin{math}
    \deg(\gamma) = \frac{1}{2} \Ind_a^b(\Re \gamma : \Im \gamma)
\end{math}
Dies hat den Vorteil, dass wir die Zählung der Übergänge auf eine andere Art und Weise interpretieren können:
so zählen wir nun die Polstellen einer Funktion $r := \frac{\Re \gamma}{\Im \gamma}$ mit Wertungen wie in \ref{fig:poles_counting} dargestellt.

Damit wir unsere geometrisch motivierte Zählung formalisieren können, setzen wir $p := \Re \gamma, q := \Im \gamma$ als polynomiell voraus.
Auf diese Weise besitzt $q$ nur endlich viele Nullstellen und $r$ entsprechend nur endlich viele Polstellen.

Diese Zählung nennen Cauchy-Index und schreiben dafür $\Ind_a^b(p:q)$.
Wir geben nun eine exakte Definition in Anlehnung an \cite{eiserm_topology}.

\begin{definition}[Cauchy-Index] \label{thm:df:cauchy_index}
    Seien $p, q \in \R[X]$ reelle Polynome und $a, b, s \in \R$ mit $a \le b$.
    Für jede Nullstelle $s \in \R$ von $q$, d.h. $q(s) = 0$, sei $\epsilon > 0$ hinreichend klein, dass $p(t)q(t)$ jeweils auf $[s - \epsilon, s)$ und $(s, s + \epsilon]$ konstante Vorzeichen besitzt.

    Wir definieren nun
    \begin{math}
	\Ind_s^{\pm}(p:q) &:= \begin{cases}
	    \frac{1}{2} \sgn\big((pq)(s \pm \epsilon)\big) & q(s) = 0, \\
	    0 & \text{sonst},
	\end{cases}\\
	\Ind_s(p:q) &:= \Ind_s^+(p:q) - \Ind_s^-(p:q), \\
	\Ind_a^b(p:q) &:= \Ind_a^+(p:q) + \sum_{s \in (a,b)} \Ind_s(p:q) - \Ind_b^-(p:q).
    \end{math}
    $\Ind_a^b(p:q)$ nennen wir \emphdef[Cauchy-Index]{Cauchy-Index von $p$ und $q$ auf $[a,b]$}.
\end{definition}

\begin{note}
    \begin{itemize}
	\item
	    Die Summe $\sum_{s\in (a,b)} \Ind_s(p:q)$ ist wohldefiniert, da $\Ind_s(p:q)$ nur an den endlich vielen Nullstellen von $q$ in $(a,b)$ einen Beitrag leistet.
        \item
	    Wir stellen fest, dass die Definition für $\Ind_s(p:q)$ einen Pol genau so zählt, wie wir es uns in \ref{fig:poles_counting} vorgestellt haben, beispielswiese ergibt sich
	    \begin{math}
		\Ind_{-1}^1(1:X)
		&= \Ind_0(1:X)
		= \Ind_0^+(1:X) - \Ind_0^-(1:X)
		= \frac{1}{2} \big( 1 - (-1) \big)
		= 1, \\
		\Ind_{-1}^1(1:X^2)
		&= \Ind_0(1:X^2)
		= \Ind_0^+(1:X^2) - \Ind_0^-(1:X^2)
		= \frac{1}{2} ( 1 - 1)
		= 0.
	    \end{math}
	    Darüber hinaus beobachten wir
	    \begin{math}
		\Ind_{0}^{1}(1:X) &= \Ind_0^+(1:X) = \frac{1}{2}, \\
		\Ind_{-1}^{0}(1:X) &= \Ind_0^-(1:X) = \frac{1}{2}.
	    \end{math}
	    Dies liefert $\Ind_{-1}^0(1:X) + \Ind_{0}^1(1:X) = 1 = \Ind_{-1}^1(1:X)$ und legt nahe, dass sich der Cauchy-Index additiv unterteilen lässt.
    \end{itemize}
\end{note}

Außerdem 

\begin{tikzpicture}
    \begin{axis}[
	axis x line=middle,
	axis y line=middle,
	xmin=-2.5,xmax=2.5,
	ymin=-2,ymax=2
    ]
	\addplot+[no markers,smooth] coordinates
	    { (1,1) (0,2) (-1,1) (-2,-1) (2,-1.5) (1,1) };
    \end{axis}
\end{tikzpicture}


\begin{tikzpicture}
    \begin{axis}
	\addplot[samples=100,domain=-1:1]({2*x^2 - 1}, {x^3 - x});
    \end{axis}
\end{tikzpicture}








%\begin{tikzpicture}
%    \begin{axis}
%	\addplot3[surf,faceted color=black,color=red,colormap/greenyellow]({x},{y},{x^2});
%    \end{axis}
%\end{tikzpicture}












\end{document}
