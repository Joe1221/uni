\documentclass{mythesis}

\title{Der reell algebraische Abbildungsgrad}
\author{Stephan Hilb}

\DeclareDocumentCommand{\sgn}{}{\operatorname{sgn}}

\begin{document}

\maketitle

\section{Geometrische Motivation}

\subsection{Der Cauchy-Index}

\begin{itemize}
    \item
	\emph{Cauchy-Index 1 $[p:q]$} „zählt Polstellen von $r = \frac{p}{q}$ mit Vorzeichen“.
	Zunächst einfache Definition vermutlich über einseitige Grenzwerte (ohne $\frac{1}{2})$):
	\begin{math}
	    [p:q]_a &:= \begin{cases}
	        1 & \lim_{x \nearrow a} r(x) < 0 \land \lim_{x \searrow a} r(x) > 0 \\
	        -1 & \lim_{x \nearrow a} r(x) > 0 \land \lim_{x \searrow a} r(x) < 0 \\
		0 & \text{sonst}
	    \end{cases},\\
	    [p:q]_a^b &:= \sum_{x \in [a,b]} [p:q]_x
	\end{math}
    \item
	Eigenschaften des Cauchy-Index
    \item
	\emph{Stabile Auswertung $f\|_a$}:
	Schreibe in neuer Basis $f = \sum_{k=0}^n c_k (x-a)^k$, $m = \argmin_{k} \Set{c_k & k \neq 0}$ und setze
	\begin{math}
	    f\|_a := \begin{cases}
	        c_k & \text{$k$ gerade}, \\
		0 & \text{$k$ ungerade}
	    \end{cases}.
	\end{math}
    \item
	Eigenschaften der stabilen Auswertung:
	\begin{itemize}
	    \item
		falls $f(a) \neq 0$, dann $f\|_a = f(a)$,
	    \item
		Multiplikativität: $(fg)\|_a = f\|_a \cdot g\|_a$.
	\end{itemize}
    \item
	\emph{Cauchy-Index 2, alternative Definition} ($\frac{p}{q}$ gekürzt)
	\begin{math}
	    [p:q]_a^b = \sum_{\substack{x\in [a,b]\\ q(x) = 0}}\sgn(p\|_x \cdot q'\|_x).
	\end{math}
\end{itemize}

\subsection{Die Windungszahl}

\begin{itemize}
    \item
	Für polynomielle Wege $\gamma: [t_0, t_1] \to \C$, d.h. $\gamma(t) = p(t), p \in \C[X]$, definiere die Windungszahl
	\begin{math}
	    w(\gamma) := \frac{1}{2} [\Re p: \Im p]_{t_0}^{t_1}.
	\end{math}
    \item
	Geometrische Anschauung der Windungszahl anhand Definition 1 und 2 des Cauchy-Index:
	„Überquerungen der reellen Achse werden mit $\pm \frac{1}{2}$ gewertet“
    \item
	Eigenschaften der Wind
\end{itemize}

\subsection{Berechenbarkeit mittels Sturm-Ketten}

\begin{itemize}
    \item

\end{itemize}

\subsection{Eine anschauliche Verallgemeinerung}

\section{Der Index einer polynomiellen Funktion}

\subsection{Formulierung und Algorithmus}

\subsection{}

\section{Geometrische Charakterisierung des Abbildungsgrades}

\section{…}






\end{document}
