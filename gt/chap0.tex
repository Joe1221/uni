\chapter{???}

\Timestamp{2015-10-12}

\begin{conv}
    Standardmäßig sind Graphen $G(V, E)$ im Folgenden ungerichtet, ohne Mehrfachkanten und ohne Schleifen.
\end{conv}

In vielen Fällen (wie auch im Folgenden) stellt dies keine nennenswerte Einschränkung dar.

\begin{df}
    Der Grad eines Knotens ist
    \begin{math}
        d(x) = |\Set{y & xy \in E}|
    \end{math}
\end{df}

\begin{df}
    Ein \emphdef{Eulerkreis} ist ein Rundweg, der alle Kanten genau einmal durchläuft.
    Ein Eulerpfad von $a$ nach $b$ mit $a,b \in V$ ist ein Pfad von $a$ nach $b$, der alle Kanten genau einmal durchläuft.
\end{df}

\begin{ex}
    Haus vom Nikolaus
\end{ex}

\begin{df}
    Ein Graph $G = (V, E)$ ist zusammenhängend, wenn für $a,b \in V$ ein Pfad von $a$ nach $b$ existiert.
\end{df}


\begin{st}
    Sei $G = (V, E)$ zusammenhängend.
    Es gibt einen Eulerkreis in $G$ genau dann, wenn $\forall x \in V: d(x) \equiv 0 \bmod 2$.
    \begin{proof}
        Wähle ein $a \in V$ und start einen möglichst langen Pfad von $a$ nach $a$.
        Angenommen dies ist kein Eulerkreis, dann wurde eine ausgehende Kante von einem Knoten $b$ auf dem Pfad wurde nicht besucht.
        Von $b$ kann ein neuer Pfad gestartet werden.
        Kombiniert man beide, erhält man einen längeren Weg, also war der erste Pfad bereits ein Eulerkreis.

        Die andere Richtung ist trivial.
    \end{proof}
\end{st}

\begin{df}
    Ein Graph $G$ heißt \emphdef{plättbar}, wenn sich $G$ kreuzungsfrei in die Ebene zeichnen lässt.
    $G$ heißt \emphdef{planar}, falls er kreuzungsfrei in der Ebene eingebettet ist.
\end{df}

\begin{note}
    Zwei eingebettete Graphen können zwar abstrakt isomorph sein, jedoch nicht isotop zueinander.
\end{note}

\begin{st}[Eulerformel für planare Graphen]
    Sei $G(V, E)$ ein planarer Graph mit $z \in \N$ Zusammenhangskomponenten, $n := |V|$, $e := |E|$ und $f$ die Anzahl der Facetten.
    Dann gilt
    \begin{math}
        n - e + f = 1 + z.
    \end{math}
    \begin{proof}
        Für $e = 0$ ist $f = 1$ und die Aussage klar.
        Nehme eine Kante $xy$ hinzu und unterscheide folgende Fälle:
        \begin{enumerate}[1.]
            \item
                $xy$ verbindet 2 Kompontenten.
                In diesem Fall folgt $e \leadsto e + 1$, $f \leadsto f$ und $z \leadsto z - 1$.
            \item
                $xy$ verbindet $x$ und $y$ aus der selben Komponente.
                Wegen Planarität folgt $e \leadsto e + 1$, $f \leadsto f + 1$ und $z \leadsto z$.
        \end{enumerate}
    \end{proof}
\end{st}

\begin{kor}
    Für einen planaren, zusammenhängenden Graphen, dessen Facetten aus Polygonen mit mindestens $k$ Ecken bestehen, gilt
    \begin{math}
        (k - 2) e \le k(n - 2).
    \end{math}
    Insbesondere gilt für planare zusammenhängende Graphen
    \begin{enumerate}[i)]
        \item
            allgemein mit $k = 3$: $e \le 3n - 6$,
        \item
            für bipartite Graphen mit $k = 4$: $e \le 2n - 2$.
    \end{enumerate}
    \begin{proof}
        Aus einem Abzählargument erhalten wir $kf \le 2e$.
        Es folgt aus der Eulerformel
        \begin{math}
            2k = kn - ke + kf &\le kn - (k - 2)e
            \iff (k - 2)e &\le k(n - 2).
        \end{math}
    \end{proof}
\end{kor}

\begin{ex}
    \begin{itemize}
        \item
            $K_5$ ist nicht planar, denn
            \begin{math}
                10 \not\le 9 = 3\cdot 5 - 6.
            \end{math}
        \item
            $K_{3,3}$ ist nicht planar.
            $K_{3,3}$ erfüllt zwar
            \begin{math}
                9 \le 12 = 3 \cdot 6 - 6,
            \end{math}
            aber nicht
            \begin{math}
                9 \not\le 8 = 2 \cdot 4 - 4.
            \end{math}
    \end{itemize}
\end{ex}

\begin{st}
    Es existieren genau $5$ platonische Körper.
\end{st}

\begin{df}
    Ein \emphdef{Hamiltonkreis} in $G(V, E)$ ist ein Rundweg, der alle Knoten genau einmal besucht.
    \begin{note}
        Siehe auch das Traveling-Salesman-Problem.
    \end{note}
\end{df}

% Falsch
%\begin{st}
%    Falls es Knoten $x, y \in V$ mit $xy \not\in E$, aber $d(x) + d(y) \ge |V|$, dann gibt es einen Hamiltonkreis.
%\end{st}


\Timestamp{2015-10-15}

\begin{df}
    \begin{itemize}
        \item
            \emphdef{Adjazenzmatrix}: $A \in \Set{0,1}^{e \times e}$ mit
            \begin{math}
                A(i,y) = \begin{cases}
                    1 & \text{falls $ij \in E$}, \\
                    0 & \text{sonst}.
                \end{cases}
            \end{math}
            \begin{itemize}
                \item
                    $A$ ungerichtet genau dann, wenn $A$ symmetrisch ist
                \item
                    $A$ enthällt $0$-en auf der Diagonale genau dann, wenn keine Schlingen vorhanden sind.
            \end{itemize}
        \item
            \emphdef{bipartite} Graphen: $V = A \dunion B$ und alle Kanten von $A$ nach $B$.

            \emphdef{vollständige} bipartite Graphen: $K_{m,n}$ mit $|A| = m$, $|B| = n$.
        \item
            \emphdef{vollständige} Graphen: $K_n = (V, \binom{V}{2})$.
        \item
            \emphdef{Teilgraph}, \emphdef{Untergraph}: $(V', E')$ von $(V, E)$ mit $V' \subset V$ und $E' \subset E$.

            \emphdef{induzierte Untergraphen}: $(V', E')$ in $(V, E)$ mit $V' \subset V$ und $E' = \Set{xy & x,y \in V', xy \in E}$.
            Notation $G(V,E) \leadsto G(V')$.
        \item
            \emphdef{Unterteilungsgraph}: Setze auf Kanten zwischen Knoten.
        \item
            \emphdef{Kreise}: Wege mit selben Start- und Endpunkt, Länge $\ge 3$
            $C_n$: Kreis mit $n$ Knoten.
        \item
            $G(V, E)$ ist ein \emphdef{Wald} genau dann, wenn es keine einfachen Kreise gibt.
            Ein \emphdef{Baum} ist ein zusammenhängender Wald.
        \item
            Ein Polygon ist ein kreuzungsfreier Polygonzug.
            Ein Polygon $P$ heißt \emphdef{sternförmig}, wenn es eine offene Menge $K$ im Polygon gibt und das Innere von $P$ gesehen werden.
    \end{itemize}
\end{df}

% FIXME: Definition: Sternförmiges Polygon mit \epsilon-Umgebung

\begin{st}[Farý-Wagner]
    Jeder plättbare Graph hat eine planare Einbettung, bei der alle Kanten Strecken sind.
    Sogar: alle inneren Facetten sind sternförmig.
    \begin{proof}
        Ohne Einschränkung $|V| \ge 4$, $G$ zusammenhängend, $d(x) \ge 3$ (Knoten mit $d \le 2$ können zum Schluss eingefügt werden) und alle Facetten sind Dreiecke.

        Induktiv von außen nach innen Facetten hinzunehmen.
        Invariante: das Innere ist sternförmig und die bisher konstruierte Einbettung ist induzierter Untergraph des Anfangsgraphens.

        Konstruiere neue Facette, diese muss einen neuen Punkt im Inneren besitzen (Induktionsannahme), lege diesen ins Innere der $\epsilon$-Umgebung.
        Der Rest bleibt sternförmig.
    \end{proof}
\end{st}

\begin{st}
    Sei $G = (V, E)$ ein Graph, der folgender Bedingung genügt:
    \begin{math}
        \forall x,y \in V: xy \not\in E \implies d(x) + d(y) \ge |V| \ge 3.
    \end{math}
    Dann existiert ein Hamiltonkreis.
    \begin{proof}
        Induktion über $|\binom{V}{2}| - |E|$.
        Sei $n = |V|$, dann ist $E = \binom{V}{2}$ genau dann, wenn $G = K_n$.
        Dann gibt es Hamiltonkreise.
        Betrachte jetzt einen Graphen $G$ und $x,y \in V$ mit $xy \not\in E$.
        Definieren $G' = (V, E')$ mit $E' = E \cup \Set{xy}$.
        Die Induktionsvoraussetzung für $G'$ ist erfüllt und es existiert ein Hamiltonkreis.
        Definiere
        \begin{math}
            N(x) &:= \Set{i \in \Set{1, \dotsc, n} & xv_{i+1} \in E}
            \subset \Set{1, \dotsc, n-1}, \\
            N(y) &:= \Set{i & yv_i \in E}
            \subset \Set{2, \dotsc, n-1}.
        \end{math}
        Es existiert ein $i \in N(x) \cap N(y)$, da $d(x) = |N(x)|$, $d(y) = |N(y)|$.
    \end{proof}
\end{st}


\Timestamp{2015-11-02}

Ein Baum ist ein nicht-leerer zusammenhängender Wald.
Ein Knoten $x$ ist ein \emphdef{Blatt}, wenn $d(x) \le 1$.

\begin{lem}
    Jeder endliche Baum hat Blätter.
    \begin{proof}
        Wähle $x_0 \in V$ als Wurzel.
        Alle Knoten verfügen über einen eindeutigen Pfad zur Wurzel hin.
        Knoten mit maximaler Distanz zur Wurzel sind Blätter.
    \end{proof}
\end{lem}

\begin{kor}
    Sei $T = (V, E)$ ein Baum mit $|V| = n$ hat $|E| = n-1$ Kanten.
\end{kor}

\begin{st}[Cayley-Formel]
    Sei $K_n = (V, \binom{V}{2})$ ein vollständiger Graph mit $n \ge 2$ Knoten.
    Dann gilt
    \begin{math}
        \Set{T \subset \binom{V}{2} & \text{$(V, T)$ ist Spannbaum} } = n^{n-2}.
    \end{math}
    \begin{proof}
        Wir identifizieren dazu jeden Spannbaum mit einer Knotenfolge der Länge $n-2$, dem sogenannten \emphdef{Prüfer-Code}.
        Ohne Einschränkung $V = \Set{1, \dotsc, n}$ mit linearer Ordnung.

        $n = 2$ klar, leere Folge.
        Definiere $\In(T) := \Set{x \in V & d(x) \ge 2}$.
        Dann gilt $\In(T) \neq V$ (Baum hat Blätter).
        Da $V$ linear geordnet ist, gibt es ein kleinstes Blatt $b_1 \in V \setminus \In(T)$.
        Es existiert genau ein $p_1$ mit $b_1p_1 \in E$, $p_1 \in \In(T)$ (da $n \ge 3$).
        Setze $V' := V \setminus \Set{b_1}$, $E' := E \setminus \Set{b_1p_1}$.
        Induktiv $\Set{p_1, \dotsc, p_{n-2}} = \In(V, T)$.
        Dies definiert induktiv den Prüfercode $c(T) := (p_1, p_2, \dotsc, p_{n-2})$.

        Injektivität: Sei $c(T_1) = c(T_2)$, so haben $T_1$ und $T_2$ die gleichen Blätter.
        Somit wurde $b_1$ in beiden Prüfer-Codes gleich gewählt, sowie $b_1 p_1 \in T_1 \cap T_2$ und induktiv damit $T_1 = T_2$.

        Surjektivität: Konstruktion eines Baumes zum Prüfer-Code.
    \end{proof}
\end{st}

\begin{note}
    Dies ist hilfreich für randomisierte Bäume.
    Interessante Fragestellungen für randomisierte Bäume:
    \begin{itemize}
        \item
            Anzahl Blätter
        \item
            Maximaler Knotengrad
        \item
            Anzahl Kanten
    \end{itemize}
\end{note}


\Timestamp{2015-11-05}


\section{Färbbarkeit von Graphen}

\begin{df}
    Sei $G = (V, E)$, $E \subset \binom{V}{2}$.
    Eine $k$-Färbung ist eine Abbildung $c: V \to \Set{1, \dotsc, k}$ mit
    \begin{math}
        xy \in E \implies c(x) \neq c(y).
    \end{math}
\end{df}

$G$ besitzt eine 2-Färbung genau dann, wenn $G$ bipartit ist.
Wir können kürzeste Wege in $\LandauO(|E| + |V|\log |V|)$ lösen (Dijkstra).
Dijkstra mit Weglängen 0, 1 kann feststellen, ob ein Graph 2-färbbar ist, dies ist sogar in Linearzeit möglich.

Reduktion auf bekannte NP-vollständige Probleme (3-SAT, NAE-SAT, etc.)

\begin{st}
    3-Färbbarkeit ist NP-vollständig.
    \begin{proof}
        NAE-SAT: $F = C_1 \land \dotsb \land C_m$ mit Klauseln $C_j = \tilde x_{1j} \lor \tilde x_{2j} \lor \dotsb \lor \tilde x_{n,j}$, wobei $\tilde x \in \Set{x, \_x}$ und einer Belegung $\sigma: \Set{x_1, \dotsc, x_n} = \Var(F) \to \B = \Set{0,1}$, $\sigma(F) = 1 = (1-\sigma)(F)$.

        Reduktion $\text{3-SAT} \le \text{NAE-SAT}$.
        Sei $F \in \text{3-KNF}$ (3-Konjuktive Normalform, s.o.).
        Wähle neue Variable hinzu und ersetze jede Klausel $C_j = (\tilde x_{1j} \lor \tilde x_{2j} \lor \tilde x_{3j})$ durch
        \begin{math}
            C_j = (\tilde x_{1j} \lor \tilde x_{2j} \lor \tilde x_{3j} \lor y).
        \end{math}
        und $F' = (C_1' \land \dotsb \land C_m')$.
        Sei $\sigma(F) = 1$.
        Setze $\sigma(y) = 0$, dann gilt $F' \in \text{NAE-SAT}$.

        Sei umgekehrt $F' \in \text{NAE-SAT}$ mit $\sigma(F') = (1-\sigma)(F') = 1$.
        Unterscheide die Fälle $\sigma(y) = 1$ und $\sigma(y) = 0$.
        Im ersten Fall ist $(1-\sigma)(y) = 0$, sonst mit Symmetrie.
        Also ohne Einschränkung $\sigma(y) = 0$.
        Es folgt notwendigerweise $\sigma(F) = 1$.

        $\text{3-NAE-SAT} = \text{NAE-SAT} \cap \text{3-KNF}$.
        Von 4 Variablen auf 3 Variablen pro Klausel.
        Setze für $C_j = (\tilde x_{1j} \lor \tilde x_{2j} \lor \tilde x_{3j} \lor \tilde x_{4j})$
        \begin{math}
            C_j' = (\tilde x_{1j} \lor \tilde x_{2j} \lor y_j) \land (\_y_{j} \lor \tilde x_{3j} \lor \tilde x_{4j})
        \end{math}
        Durch geeignete Wahl von $y_j$ ist $C_j'$ erfüllt genau dann, wenn $C_j$ erfüllt war.

        Konstruiere Graphen: Dreiecke bestehend aus den 3-Klauseln, und verbundenen Paaren $x_i, \_{x_i}$ (Belegungsvariablen), verbunden zu einer Wurzel.
        Verbinde immer $x_i$ mit $\_{x_i}$.

        Ist der Graph 3-färbbar, so ist ohne Einschränkung die Wurzel blau und die Variablen definieren eine gültige Belegung.

        Andere Richtung: Für eine erfüllte Belegung färbe die Dreiecke: zwangsläufig eins rot, eins grün, färbe den anderen blau.
    \end{proof}
\end{st}

\begin{st}
    Jeder planare Graph ist 5-färbbar.
    \begin{proof}
        In jedem planaren Graphen existiert ein $x \in V$ mit Grad $d(x) \le 5$ (Widerspruch mit Euler-Formel).
        Betrachte $G'$ mit $V' = V \setminus \Set x$, wobei $d(x) \le 5$.
        Dieser ist per Induktion 5-färbbar.
        Es kommt kein $K_5$ vor. Skizze: Pentagram um $x$.
        Ohne Einschränkung $ab \not\in E$, identifiziere $a$ und $b$ und färbe per Induktion.
        Also kann $x$ die verbleibende Farbe erhalten.
    \end{proof}
\end{st}


\Timestamp{2015-11-09}

\section{Heiratssätze}

\begin{st}[Philip Hall, 1935]
    Seien $G = (A \sqcup B, E)$ ein bipartiter Graph mit $E \subset A \times B$ und $N(X) := \Set{b \in B & \exists a \in X: (a,b) \in E}$, $X \subset A$.
    Es gelte die „Heiratsbedingung“:
    \begin{math}
        \forall X \subset A \exists N(X) \subset B : |N(X)| \ge |X|.
    \end{math}
    Dann existiert ein (maximales) Matching $M$ mit $|M| = |A|$ (d.h. $M \subset E$ mit $vw \in M \implies vx \not\in M$).
    \begin{proof}
        Einfacher Algorithmus in $\LandauO(nm)$.
        Karp Hopcraft (1973) liefert die bisher beste Zeit $\LandauO(\sqrt{n} m)$.
    \end{proof}
\end{st}

\subsection{Stabile Heirat}

Gale-Shapley (Stabile Heirat), 1962.
Sei $n = |A| = |B|$.

Jedes $a \in A$ hat eine Präferenzliste $b_{a(1)} > b_{a(2)} > \dotsb > b_{a(n)}$ und jedes $b \in B$ entsprechend.
Eine Matching $M$ ist nicht stabil, wenn $\exists ab \in M, a' \in A, b' \in B$, sodass $a' > a$ für $b$ und $b' > b$ für $a$.

\begin{st}[Gale Shapley]
    Es gibt eine stabile Heirat und sie lässt sich nach maximal $n^2$ Schritten finden.
    \begin{proof}
        Jedes $b \in B$ fragt in der Reihenfolge einer Präferenz ein $a \in A$, ob Paarung möglich ist, es sei denn, $a$ wurde bereits gefragt, oder $a$ hat $b$ bereits verlassen.
        Jedes $a \in A$ paart sich nur dann neu, wenn der fragende $b$ höher in der Präferenzliste von $a$ steht.

        Im Laufe der Zeit steigen die $a \in A$ auf und die $b$ steigen ab.
        Der Algorithmus terminiert nach $n^2$ Schritten.

        Die resultierende Heirat ist stabil, betrachte dazu eine Heirat $a_i b$:
        Alle $a_j$ mit $j < i$ würde $b$ nicht akzeptieren, alle $a_j$ mit $j > i$ möchte $b$ nicht haben.

        Übung: Der Algorithmus ist optimal für $B$.
    \end{proof}
\end{st}

\begin{df}
    Ein Graph $G = (V, E)$ ist gerichtet, wenn $E \subset V \times V$.
    Wir betrachten $G = (V, E)$ ungerichtet als Spezialfall mit $E = E^{-1}$ und $(x,x) \in E$.

    Für $A, B \subset V$ ist eine Folge $(x_0, x_1, \dotsc, x_k)$ ein $AB$-Pfad, falls $x_0 \in A$, $x_k \in B$,
    $x_1, \dotsc, x_{k-1} \not\in A \cup B$, $x_1, \dotsc, x_{k-1} \not \in A \cup B$ und $(x_{i-1}, x_i) \in E$ für alle $1 \le i \le k$.

    Ein $AB$-Separator ist eine Menge $C \subset V$, sodass jeder $AB$-Pfad mindestens einen Knoten aus $C$ benutzt.
\end{df}

\begin{st}[Menger, 1929]
    Die maximale Zahl paarweiser Knoten-disjunkter $AB$-Pfade ist gleich der minimalen Größe eines $AB$-Separators.
    \begin{proof}
        Eine Richtung ist klar: Sei $C$ ein $AB$-Separator, dann existieren höchstens $|C|$ knotendisjunkte $AB$-Pfade.

        Für die Umkehrung: Induktion nach Anzahl Kanten $|E|$.
        Für $E = \emptyset$ ist $A \cap B$ ein minimaler $AB$-Separator und es gilbt $|A \cap B|$ Knoten-disjunkte Einpunktpfade.
        Sei jetzt $xy \in E$.
        Setze $G' = G - xy$ (Graph ohne Kante $xy$).
        Sei $k$ die maxmiale Zahl Knoten-disjunkter $AB$-Pfade in $G$.
        Sei jetzt $C$ ein minimaler $AB$-Separator in $G'$.
        Falls $|C| = k$, dann sind wir fertig.

\Timestamp{2015-11-12}
        Also gilt $|C| \le k-1$ (sogar $|C| = k-1$).
        Jetzt sind $S := C \cup \Set{x}$ und $T := C \cup \Set{y}$ $AB$-Separatoren in $G$.

        Sei $D$ ein $AS$-Separator in $G$, dieser ist insbesondere ein $AB$-Separator.
        Es folgt $|D| = k$, analog für $T$.
        Mit Induktion existieren $k$ knotendisjunkte Pfade von $A$ nach $S$ in $G'$, bzw. von $T$ nach $B$ in $G'$ und damit von $A$ nach $B$.
    \end{proof}
\end{st}


\paragraph{Graph-Parameter}


Sei $G = (V, E)$, $E \subset \binom{V}{2}$.
\begin{math}
    \alpha(G) := \max\Set{|I| & \text{$I \subset V$ unabhängig}},
\end{math}
wobei $I \subset V$ unabhängig ist, wenn $xy \in E \implies |I \cap \Set{x,y}| \le 1$.
\begin{math}
    \omega(G) := \max\Set{|C| & \text{$C \subset V$ ist Clique}},
\end{math}
wobei $C \subset V$ Clique ist, wenn $\forall x,y \in C : xy \in E$.

Es gilt für den komplementären Graphen
\begin{math}
    \alpha(G) = \omega(\_G).
\end{math}

Definiere
\begin{math}
    \kappa(G) &:= \min\Set{n \in \N & \text{$\exists C_1, \dotsc, C_n$ Clique mit $V = \bigcup_{i=1}^n C_i$}}, \\
    \chi(G) &:= \min\Set{n \in \N & \text{$\exists$ Färbung $c: V \to \Set{1, \dotsc, n}$}}.
\end{math}
Es gilt stets
\begin{math}
    \chi(G) &\ge \omega(G),\\
    \kappa(G) &\ge \alpha(G).
\end{math}

Für alle induzierte Untergraphen?
Wir nennen $G$ $\chi$-perfekt, wenn $\chi(G) = \omega(G)$.
Beispiele sind bipartite Graphen

Wir nennen $G$ $\alpha$-perfekt, wenn $\alpha(G) = \kappa(G)$.

Interessante Sätze sind:

\begin{st}
    Sei $G$ ein Graph und alle induzierten Teilgraphen sind $\chi$-perfekt genau dann, wenn alle induzierten Teilgraphen $\alpha$-perfekt sind.
\end{st}

\begin{st}
    Ein Graph ist $\chi$-, bzw. $\alpha$-perfekt (also alle induzierten Untergraphen $G'$ erfüllen $\alpha(G') = \kappa(G')$ und $\chi(G') = \omega(G')$) genau dann, wenn kein induzierter $C_k$ existiert mit $k$ ungerade und $k \ge 5$.
\end{st}

\begin{df}
    Ein \emphdef{Träger} $T$ (vertex cover) ist eine Teilmenge $T \subset V$ mit
    \begin{math}
        \forall xy \in E: \Set{x,y} \cap T \neq \emptyset.
    \end{math}
    Eine \emphdef{Paarung} $M$ (matching) ist eine Teilmenge $M \subset E$ mit
    \begin{math}
        xy \in M \land x'y' \in M \implies xy = x'y' \lor \Set{x,y} \cap \Set{x',y'} = \emptyset.
    \end{math}
\end{df}

Wir haben folgende Dualität:
\begin{lem}
    $T \subset V$ ist Träger genau dann, wenn $V \setminus T$ unabhängig ist.
\end{lem}

\begin{st}[König, 1931]
    Sei $G = (A \sqcup B, E)$ ein bipartiter Graph.
    Dann gilt
    \begin{math}
        \max\Set{|M| & \text{$M$ ist Matching}}
        = \min \Set{|T| & \text{$T$ ist Träger}}.
    \end{math}
    \begin{proof}
        Klar ist $\le$: jeder Träger muss für mindestens $|M|$ Knoten abdecken für ein beliebiges Matching.

        Für die umgekehrte Richtung: Betrachte den Beweis des Heiratssatzes.
        Skizze: Bobs/Alices, finde Verbesserungsweg, Widerspruch zu maximalem Matching.
    \end{proof}
\end{st}

\begin{ex}
    Übungen:
    \begin{itemize}
        \item
            Aus vorigem Satz folgt $\alpha$-Perfektheit für bipartite Graphen.
        \item
            Aus dem Heiratssatz folgt König.
            Umkehrung gilt auch.
        \item
            Menger impliziert Heiratssatz/König.
    \end{itemize}
\end{ex}


\Timestamp{2015-11-19}


\section{Fluss-Probleme}

%\begin{ex}
%    Fibonacci, nicht-terminierend?
%\end{ex}

\begin{df}
    Ein \emphdef{Flussnetzwerk} ist ein Graph $(V, E)$ mit $s, t \in V$, $s \neq t$ zusammen mit einer Funktion $c: V \times V \to \R_{\ge 0}$ und $E := \Set{(x,y) \in V \times V & c(x,y) > 0}$.

    Ein \emphdef{Fluss} ist eine Abbildung $f: V \times V \to \R$ mit $f(x,y) = -f(y,x)$ und den Eigenschaften
    \begin{enumerate}[i)]
        \item
            $\forall x,y: f(x,y) \le c(x,y)$,
        \item
            $\forall u \not\in \Set{s, t} : \sum_{v \in V} f(u,v) = \sum_{v \in V} f(v,u) = 0$.
    \end{enumerate}
    Der \emphdef{Wert} von $f$ ist $\|f\| := \sum_{v \in V} f(s,v)$.
\end{df}

\begin{df}
    Eine Partition $A \cup B = V$, $A \cap B = \emptyset$ heißt $s$-$t$-Schnitt, falls $s \in A$, $t \in B$.
    Wir setzen
    \begin{math}
        c(A,B) &= \sum_{(x,y) \in A \times B} c(x,y), \\
        f(A,B) &= \sum_{(x,y) \in A \times B} f(x,y).
    \end{math}
\end{df}

\begin{lem}
    Es gilt für jeden $s$-$t$-Schnitt $(A, B)$
    \begin{math}
        f(A, B) = \|f\|.
    \end{math}
    \begin{proof}
        Induktion nach $|A|$.
        Für $|A| = 1$ gilt $A = \Set{s}$ und die Aussage gilt gemäß Definiton.

        Betrachte nun $f(A \cup \Set{z}, B \setminus \Set{z})$ mit $z \in B$.
        Veränderung:
        \begin{math}
            \sum_{y \in B}f(z,y) - \sum_{x \in A}f(x,z)
            = \sum_{y \in B}f(z,y) + \sum_{x \in A}f(z,x)
            = \sum_{v \in V}f(z,v)
            = 0.
        \end{math}
    \end{proof}
\end{lem}

\begin{kor}
    Es gilt
    \begin{math}
        \|f\| = \sum_{v \in V} f(v,t).
    \end{math}
\end{kor}

\begin{df}
    Ein $s$-$t$-Schnitt $(A,B)$ heißt \emphdef{minimal}, falls $c(A,B)$ minimal ist.
\end{df}

Es gilt für jeden Fluss $f$ stets
\begin{math}
    \|f\| \le \min \Set{c(A,B) & \text{$(A,B)$ $s$-$t$-Schnitt}}.
\end{math}

\begin{df}
    Sei $f$ ein Fluss.
    Definiere $R_f := \Set{(x,y) & f(x,y) < c(x,y)}$, die Kanten des \emphdef{Residualgraphen}.
    \begin{note}
        Beachte
        \begin{math}
            (x,y) \in R_f
            \implies
            0 = f(x,y) + f(y,x)
            < c(x,y) + c(y,x),
        \end{math}
        $|R_f| \le m = 2|E|$.
    \end{note}
\end{df}

\begin{lem}
    $\|f\|$ ist maximal genau dann, wenn kein $s$-$t$-Weg in $R_f$ existiert.
    \begin{proof}
        Angenommen es gibt einen Pfad von $s$ nach $t$ in $R_f$, dann existiert ein Verbesserungsweg.

        Umgekehrt sei $A$ die Menge der Knoten zu denen ein Pfad von $s$ in $R_f$ existiert.
        Dann ist $(A, V \setminus A)$ ein $s$-$t$-Schnitt und $c(A,B) = \|f\|$, also $\|f\|$ maximal.
    \end{proof}
\end{lem}


\Timestamp{2015-11-23}


\subsection{Algorithmus von Dinitz}


Wir möchten einen maximalen Fluss in Zeit $\LandauO(n^2m)$ berechnen.
Setze $d_f(p,q)$ als die kürzeste Distanz in $R_f$ von $p$ nach $q$, falls ein Weg existiert und $\infty$ sonst.

Wir definieren $L_k \subset V$ als die Menge der Knoten mit Distanz $k$ von $s$.

Der Levelgraph hat nun die Kanten in $R_f$ von $L_k$ nach $L_{k+1}$.
Berechnung der Levelgraphen erfolgt mit einer Breitensuche in der Zeit $\LandauO(m)$.
Wir sind fertig, falls $t$ gefunden wird, o.E. $t \in L_k$, $d_f(s,t) < \infty$.

Idee: Durchlaufe eine Phase und nach dieser Phase gilt $d_f(s,t) > k$ für einen neuen Fluss $f'$.

\begin{algorithmic}
    \State{Starte Tiefensuche bei $s$, bis ein Knoten mit Ausgangsgrad $0$ gefunden wurde}
    \If{$p \neq t$}
        \State{Es gibt keinen Weg über $p$ von $s$ nach $t$ der Länge $k$.}
        \State{Entferne alle Eingansknoten von $p$. Also wird aus dem Levelgraphen eine Kante entfernt.}
    \Else
        \State{Verbesserung wurde gefunden.}
        \State{Verbessere Fluss und sättige eine Kante auf diesem Weg.}
        \State{Obwohl sich $R_f$ verändert, gibt es keine neuen kürzesten Wege von $s$ nach $t$}
    \EndIf
\end{algorithmic}
Nach höchstens Zeit $\LandauO(nm)$ gibt es bei $s$ keine ausgehenden Kanten mehr.
Invariante $d_f(s,t) \le k$, es existiert ein Weg in dem (modifizierten) Levelgraph.
Jetzt nach Abschluss der Phase gilt $d(s,t) \ge k + 1$.


\subsection{Lipton-Tarjan}

Sei $G = (V,E)$ ein plättbarer Graph.
Dann existiert eine disjunkte Zerlegung $V = A \cup B \cup S$ mit $S$ ist ein $A$-$B$-Separator mit $|A| < \frac{2}{3} n$, $|B| < \frac{2}{3}n$, $|S| \le 2 \sqrt{2n} \in \LandauO(\sqrt n)$.
Der Separator $S$ lässt sich in der Zeit $\LandauO(n)$ berechnen.
Anwendung: Berechnung maximal unabhängiger Mengen ist für planare Graphen NP-schwierig.
Dennoch folgt aus dem Satz von Lipton-Tarjan, dass sich eine maximal unabhängige Menge in der Zeit $2^{\LandauO(\sqrt n)}$ berechnen lässt.

Übung: $\forall \alpha < 1 : |A|,|B| < \alpha n \implies |S| \in \Omega(\sqrt n)$.

Beweis von Lipton Tarjan nach Naga Alson, Paul Segmour, Robin Thomas.
Sei $G = (V, E)$ ohne Einschränkung planar, trianguliert, zusammenhängend.
Setze $k := L \sqrt{2n}$, $n = |V|$.
Gesucht ist $S$ mit $|S| \le 2k$.

Betrachte einfachen Kreis $C$ mit $|C| \le 2k$.
Dann Zerlegung in $A(C)$ (außen) und $B(C)$ (innen) mit $V = A(C) \sqcup B(C) \sqcup C$.
Es soll gelten $|C| \le 2k$ und $|A(C)| < \frac{2}{3}n$ und $|C|$ minimal mit dieser Eigenschaft.

Einen solchen Kreis gibt es: Betrachte das äußere Dreieck, dann o.E. $n \ge 3$.
Dann $|C| = 3$, $|A(C)| = 0$.
Zu zeigen ist: $|B(C)| < \frac{2}{3}n$.
Mit Widerspruch: Sei also $|B(C)| \ge \frac{2}{3}n$.
Sei $D$ der von $B(C) \cup V(C$ induzierte Untergraph.
Betrachte für $x,y \in V(C)$ die Abstände $c(x,y)$ (Distanz in $C$), $d(x,y)$ (Distanz in $D$).
Also $c(x,y) \ge d(x,y)$.

Behauptung: $\forall x,y: c(x,y) = d(x,y)$.
Angenommen $\exists x,y : c(x,y) > d(x,y)$ mit $c(x,y)$ minimal.
$\pi$ sei der Pfad von $x$ nach $y$ in $D$ mit Länge $d(x,y)$.
Dann gilt $V(\pi) \cap V(C) = \Set{x,y}$.
Neue Kreise $C_1 = C_1' \cup \pi$, $C_2 = C_2' \cup \pi$, $|C_1|, |C_2| < |C|$.
Betrachet $B(C_1)$ und $B(C_2)$.
Dann $|B(C_1)| \ge |B(C_2)|$.
Wir wollen zeigen $|A(C_1)| < \frac{2}{3}n$.
Dann Widerspruch, da $|C|$ nicht minimal.
\begin{math}
    n - |A(C_1)|
    &= |B(C_1)| + |V(C_1)| \\
    &> \frac{1}{2} \big(|B(C_1)| + |B(C_2)| + |V(\pi)| - 2 \big) \\
    &= \frac{1}{2} |B(C)|
    \ge \frac{n}{3}
\end{math}
