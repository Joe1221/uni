\chapter{Cographen}

\Timestamp{2015-12-03}

\begin{df}
    Seien $G_1 = (V_1, E_1)$, $G_2 = (V_2, E_2)$ Graphen mit $V_1 \cap V_2 = \emptyset$.
    Wir definieren die disjunkte Vereinigung
    \begin{math}
        G_1 \sqcup G_2 := (V_1 \cup V_2, E_1 \cup E_2)
    \end{math}
    und das Komplexprodukt
    \begin{math}
        G_1 \ast G_2 := (V_1 \cup V_2, E_1 \cup E_2 \cup \Set{xy & x \in V_1, y \in V_2}).
    \end{math}
\end{df}

\begin{df}
    Alle Einpunktgraphen sind Cographen.
    Sind $G_1, G_2$ Cographen, so auch $G_1 \sqcup G_2$ und das Komplexprodukt $G_1 \ast G_2$.
\end{df}

\begin{st}
    Sei $G$ ein \emphdef{Cograph}, dann auch das Komplement $\_G$.
    \begin{proof}
        Falls $G = (\Set{\ast}, \emptyset)$, dann auch $\_G = (\Set{\ast}, \emptyset)$.
        Nun ist
        \begin{math}
            \_{G_1 \sqcup G_2} &= \_{G_1} \ast \_{G_2}, \\
            \_{G_1 \ast G_2} &= \_{G_1} \sqcup \_{G_2}.
        \end{math}
    \end{proof}
\end{st}

\begin{kor}
    Je zwei der Operationen $G_1 \sqcup G_2$, $G_1 \ast G_2$, $\_G$ definieren die Klasse der Cographen.
\end{kor}

\begin{ex}
    \begin{itemize}
        \item
            Alle Graphen $G = (V, E)$ mit $|V| \le 3$ sind Cographen.
        \item
            Sei $G$ ein Cograph und $|V| \ge 2$, dann ist $G$ zusammenhängend genau dann, wenn $\_G$ nicht zusammenhängend ist.
        \item
            Für $x, y \in V$ sei $d(x,y)$ die Distanz zwischen $x$ und $y$.
            Dann gilt für $m$-zusammenhängenden Cographen $d(x,y) \le 2$ für alle $x,y \in V$.
        \item
            $P_4$ (verbundene Reihe mit vier Knoten) ist selbstdual $P_4 = \_{P_4}$ und damit kein Cograph.

            $d(a,d) = 3$.
    \end{itemize}
\end{ex}

\Timestamp{2015-12-07}

\begin{df}
    $G$ heißt $P_4$-frei (auch $N$-frei) falls $P_4$ kein induzierter Untergraph von $G$ ist.
\end{df}

\begin{lem}
    $G$ ist genau dann $P_4$-frei, falls $\_G$ $P_4$-frei ist.
\end{lem}

\begin{kor}
    Jeder Cograph $G$ ist $P_4$-frei.
    \begin{proof}
        Mit vorigem Lemma genügt es disjunkte Vereinigungen zu betrachten: $P_4$ befindet sind in einer der beiden Komponenten, Aussage folgt durch Induktion.
    \end{proof}
\end{kor}

\begin{st}
    Seien $G$, $\_G$ zusammenhängend und $P_4$-frei.
    Dann ist $G$ unendlich oder $G = (\Set \ast, \emptyset)$ (Einpunktgraph).
    \begin{proof}
        Sei $|V| \ge 2$, $G = (V, E)$.
        $G$ ist zusammenhängend, wähle also $x_1, x_2 \in V$ mit $x_1x_2 \in E$.

        Induktiv sei eine Folge $(x_1, \dotsc, x_n$ konstruiert, sodass $\forall 1 \le i < j \le n$ aus $j$ gerade folgt, dass $x_ix_j \in E$ und aus $j$ ungerade folgt $x_ix_j \not\in E$.
        Diese Folge soll verlängert werden.

        Sei ohne Einschränkung $n$ gerade.
        $G$ ist zusammenhängend, also gilt $\forall x, g \in V: d(x,y) < \infty$.
        Angenommen es gäbe $x, z \in V$ mit $d(x,z) \ge 3$, dann auch $x,y \in V$ mit $d(x,y) = 3$.
        Dies bildet $P_4$ als induzierten Untergraphen, ein Widerspruch.
        Folglich muss $d(x,y) \le 2$ in $G$ und $\_G$ gelten.

        Es ist notwendigerweise $d(x_{n-1},x_n) = 2$.
        Als existiert $x_{n+1}$ mit $x_{n-1} x_{n+1} \not\in E$ und $x_n x_{n+1} \not\in E$.
        Falls $x_{n+1} = x_i$ für ein $i \le n$, so entsteht ein $P_4$: $G(\Set{x_{n-1}, x_n, x_i, x_{n+1}})$.
    \end{proof}
\end{st}

\begin{kor}
    $G$ ist $P_4$-frei genau dann, wenn $G$ ein Cograph ist.
\end{kor}


\section{Graphparameter}


Erinnerung:
\begin{math}
    \chi(G) &\ge \omega(G), \\
    \alpha(G) &\le \kappa(G).
\end{math}
Beispiel: $C_5$.


\begin{df}
    Ein Graph $G$ heißt $\chi$-perfekt (bzw. $\alpha$-perfekt), wenn für alle induzierten Untergraphen $G'$ von $G$ gilt $\chi(G') = \chi(G)$ (bzw. $\alpha(G') = \kappa(G')$).
\end{df}

\begin{st}
    $G$ ist $\chi$-perfekt genau dann, wenn $\_G$ $\alpha$-perfekt ist.
\end{st}

\begin{st}
    Bipartite Graphen und Cographen sind $\alpha$-perfekt.
\end{st}


\section{Chordale Graphen}

\begin{df}
    $G = (V, E)$ heißt \emphdef{chordal}, wenn jeder einfache Kreis mit mindesten $4$ Punkten eine Sehne hat.
\end{df}

\begin{ex}
    \begin{itemize}
        \item
            Wälder sind chordal.
        \item
            $K_n$ ist chordal.
        \item
            Es gibt triangulierte planare Graphen, die nicht chordal sind (Skizze: Sternmuster vergrößern und außen verbinden).
    \end{itemize}
\end{ex}

\begin{lem}
    Sei $G$ chordal und $G'$ ein induzierter Untegraph, dann ist $G'$ chordal.
\end{lem}

\begin{df}
    Sei $G = (V, E)$ ein Graph und $x \in V$.
    Dann heißt $x$ \emphdef{simplizial}, falls $N(x)$ eine Clique ist.
    Insbesondere ist dann $N(x) \sqcup \Set x$ auch eine Clique.

    Eine \emphdef{simpliziale Ordnung} ist eine Reihenfolge der Kanten $v_1, v_2, \dotsc, v_n$ sodass
    $\forall 1 \le i \le n$ $v_i$ simplizial in $G[v_i, \dotsc, v_n]$ (induzierter Untergraph von $v_i, \dotsc, v_n$) ist.
\end{df}

\begin{df}
    Für $a, b \in V$, $a \neq b \in E$ heißt $S \subset V$ ein $a$-$b$-Separator, falls $a$ und $b$ in verschiedenen Zusammenhangskomponenten von $G - S$ liegen (insbesondere $a,b \not\in S$).
    \begin{proof}
        Es existiert stets ein $a$-$b$-Separator (solange $ab \not\in E$).
    \end{proof}
\end{df}

\begin{st}
    Sei $G = (V, E)$ chordal und $a, b \in V$ mit $ab \not\in E$.
    Dann ist jeder minimale $a$-$b$-Separator $S$ eine Clique.
    \begin{proof}
        Seien $G_a$, $G_b$ die Komponenten von $a$, bzw $b$ in $G - S$.
        Falls $|S| \le 1$.
        Dann ist $S$ eine Clique.
        Seien also $x, y \in S$ mit $x \neq y$, zeige $xy \in E$.
        Da $S$ minimal gibt es Kanten von $x$ und $y$ nach $G_a$ und $G_b$.
        Wähle so einen einfachen Weg $x_1, \dotsc, x_k \in G_a$ mit $xx_1, x_ky \in E$.
        Analog wähle einen einfachen Weg $y_1, \dotsc, y_k \in G_a$ mit $xy_1, y_ky \in E$.
        Zusammen mit $x,y$ bildet dies einen einfachen Kreis ohne Sehne zwischen $x_i$ und $y_j$, folglich muss $xy \in E$.
    \end{proof}
\end{st}

\Timestamp{2015-12-10}

\begin{lem}
    Sei $G = (V, E)$ chordal und $G \neq K_n$, dann existiert $x,y \in V$ mit $xy \not\in E$ und $x,y$ sind simplizial in $G$.
    \begin{proof}
        Starte mit $a,b \in V$ und $ab \not\in E$, $a \neq b$, $S$ ein $a$-$b$-Separator, der eine Clique ist.
        Seien $G_a$ und $G_b$ die Komponenten von $a$ und $b$ in $G - S$.
        $G_a \cup S$ ist chordal.

        Falls $G_a \cup S$ vollständig, dann setze $x = a$.
        In diesem Fall ist $N(a) \subset G_a \cup S$ eine Clique.

        Andernfalls existieren nach Induktion zwei simpliziale Knoten $x_1, x_2 \in G_a \cup S$ mit $x \neq y$.
        Also ohne Einschränkung $x_1 \in S$ (da $S$ Clique), also $x_1 \in G_a$.
        Wie eben folgt: $N(x_1) \subset G_a \cup S$ ist eine Clique.

        Also gibt es $x \in G_a$ und $y \in G_b$ und beide simplizial in $G$.
        Insbesondere $xy \not\in E$.
    \end{proof}
\end{lem}

\begin{st}
    $G = (V, E)$ ist chordal genau dann, wenn eine simpliziale Ordnung auf $G$ existiert.
    \begin{proof}
        \begin{seg}{\ProofImplication}
            Falls $G$ vollständig ist, so ist jede Ordnung auf $G$ eine simpliziale Ordnung ($n!$ viele).
            Falls $G \neq K_n$, dann gibt es zwei simpliziale Knoten.
            Wähle einen als $x_1$.
            Betrachte dann $G - x_1$.
            Also gibt es insgesamt mindestens $2^n \le n!$ simpliziale Ordnungen ($P_n$ liefert untere Schranke).
        \end{seg}
        \begin{seg}{\ProofImplication*}
            Sei $C$ ein einfacher Kreis in $G$.
            Der Knoten in $C$ mit minimaler simplizialer Ordnung in $G$ bildet eine Sehne mit mindestens einem seiner Nachbarn.
        \end{seg}
    \end{proof}
\end{st}

\begin{st}
    Jeder bipartite Graph ist $\alpha$-perfekt.
    \begin{proof}
        Wenn $E = \emptyset$, dann ist $\chi(G) = 1 = \omega(G)$.
        Für $E \neq \emptyset$, $\chi(G) = 2 = \omega(G)$.

        $|M| = |T|$ für maximales Matching $M$ und minimalem Träger $T$.
        Sei $n := |V|$.
        \begin{math}
            \kappa(G) \le |M| + (n - 2 |M|)
            = n - |T|
            \le \alpha(G),
        \end{math}
        also $\alpha(G) = \kappa(G)$.
    \end{proof}
\end{st}

\begin{st}
    Jeder chordale Graph ist $\alpha$- und $\chi$-perfekt.
    \begin{proof}
        \begin{seg}{$\alpha$-Perfektheit}
            Sei $(x_1, \dotsc, x_n)$ eine simpliziale Ordnung von $G$.
            Wir konstruieren parallel eine unabhängige Menge $I$ und eine Cliquenüberdeckung: $I_1 = \Set{x_1}$ und $K_1 = \Set{x_1} \cup N(x_1)$.
            Verfahre für $I_k, K_k$ genauso rekursiv mit $G - K_{k-1}$.
        \end{seg}
        \begin{seg}{$\chi$-Perfektheit}
            Sei $x_1$ simplizial in $G$.
            Dann gilt für $G' = G - \Set{x_1}$: $\chi(G') = \omega(G')$.
            Falls $|N(x_1)| < \omega(G')$.
            Dann bleibt eine Farbe frei für $x_1$.
            Andernfalls gilt $|N(x_1)| = \omega(G')$.
            Dann gilt $\omega(G) = \omega(G') + 1$, also kann eine neue Farbe für $x_1$ verwendet werden.
            $\chi(G) \le \omega(G) \implies \chi(G) = \omega(G)$.
        \end{seg}
    \end{proof}
\end{st}


\section{Intervallgraphen}


\begin{df}
    Seien $I_1, \dotsc, I_n = [a_i, b_i] \subset \R$ reelle Intervalle.
    Definiere $G = (V, E)$ mit $V = \Set{I_i}_{i=1}^n$ und $I_iI_j \in E \iff I_i \cap I_j \neq \emptyset$.
    \begin{note}
        Ohne Einschränkung können wir stets $I_i = (a_i, b_i)$ mit $a_i < b_i$ wählen.

        Induzierte Untergraphen sind Intervallgraphen.
    \end{note}
\end{df}

\begin{lem}
    Intervallgraphen sind chordal.
    \begin{proof}
        Ordne Intervalle aufsteigend nach rechtem Endpunkt.
        Dies ist eine simpliziale Ordnung.
    \end{proof}
\end{lem}

\begin{df}
    Sei $G = (V, E)$ ungerichtet.
    Dann ist $F \subset E$ eine \emphdef{transitive Orientierung} (TRO) von $G$, falls $E = F \cup F^{-1}$, $F \cap F^{-1} = \emptyset$, $F^2 \subset F$.
    Existiert eine transitive Orientierung, so heißt $G$ \emphdef{transitiv orientierbar}.
\end{df}

\begin{st}
    Intervallgraphen sind perfekt.
    \begin{proof}
        Sei $G$ ein Intervallgraph, dann ist $\_G = (V, \_E)$ eine partielle Ordnung vermöge
        \begin{math}
            I_i = (a_i, b_i) \le I_j = (a_j, b_j) \iff I_i = I_j \lor b_i \le a_j.
        \end{math}
        Für Intervallgraphen lesen wir $E$ als Teilmenge $E \subset V \times V \setminus \Id_V$ mit
        \begin{math}
            E = E^{-1} = \Set{(y,x) \in V\times V & (x,y) \in E}.
        \end{math}
    \end{proof}
\end{st}

\begin{lem}
    Sei $G = (V,E)$ ein Intervallgraph.
    Dann ist $G$ chordal und $\_G$ ist transitiv orientierbar.
    \begin{proof}
        Einfache Übung.
    \end{proof}
\end{lem}

\begin{st}
    Sei $G$ chordal und transitiv orientierbar.
    Dann ist $G$ ein Intervallgraph.
    \begin{proof}[konstruktiv?]
        Induzierte Untergraphen eines transitiv orientierbaren Graphen sind transitiv orientierbar.

\Timestamp{2015-12-14}
        Sei $x \in V$ simplizial in $G$.

        $F \subset V \times V$ mit $F^2 \subset F$, $F \cap F^{-1} = \emptyset$, $F \cup F^{-1} = \_E = \binom{V}{2} \setminus E$.
        $G' = G - x$, $V' = V - \Set{x}$, $F' = F \cap V' \times V'$.
        $(V', F')$ ist transitiv orientierbar in $\_{G'}$.

        Per Induktion ist $G'$ ein Intervallgraph, d.h. es existieren zu $v \in V'$ Intervalle $I_v = (a_v, b_v)$.
        Ohne Einschränkung $\Set{a_v & v \in V'} \cap \Set{b_v & v \in V'} = \emptyset$.

        Zerlege $V' = Y \cup Z \cup U$, wobei
        \begin{math}
            Y &= \Set{y \in V' & (y,x) \in F}, \\
            Z &= \Set{z \in V' & (x,z) \in F}, \\
            U &= \Set{u \in V' & ux \in E},
        \end{math}
        Beachte $Y \times Z \subset F'$, da $F^2 \subset F$.
        Setze
        \begin{math}
            a_x
            &= \max \Set{b_y & y \in Y}
            < \min \Set{a_z & z \in Z}
            = b_x.
        \end{math}
        Setze $I_x = (a_x, b_x)$.
        Definiere neue Intervalle $I_u' = (a_u', b_u')$ für alle $u \in U$:
        \begin{math}
            a_u' &:= \min \Set{a_u, a_x}, &
            b_u' &:= \max \Set{b_u, b_x}.
        \end{math}
        Dann ist $I_u' \supset I_x$ und $u,v \in U \implies I_u' \cap I_v' \neq \emptyset$.

        Also ist $\Set{u} \cup N(u)$ eine Clique.
        Problem: neue Überschneidungen?
        Angenommen $I_u' \cap I_y \neq \emptyset$ ist neue Überschneidung für $y \in Y$ ($I_u' \cap I_z \neq \emptyset$ für $z \in Z$ analog).
        Dann ist $b_y < a_x$, weil $y \in Y$, also $b_y < a_u'$ (da neue Überschneidung).
        Also
        \begin{math}
            b_u < a_y < a_x < b_u'.
        \end{math}
        Somit $(u,y) \in F$ und $(y,x) \in F$, also $(u,x) \in F$, ein Widerspruch zu $u \not\in Y$.

        %Betrachte $G' = G - x$, dann ist $G'$ chordal und $\_{G'}$ transitiv orientierbar.
        %Per Induktion ist $V - \Set{x} = \Set{I_2, \dotsc, I_n}$ ein Intervallgraph.
        %$N(x) = \Set{I_{i_1}, \dotsc, I_{i_k}}$ bilden eine Clique.
        %%Wähle $I_x := \cap N(x) \neq \emptyset$ als neues Intervall.
    \end{proof}
\end{st}


\subsection{Permutationsgraphen}


\begin{df}
    Sei $V_n = \Set{1, \dotsc, n}$, $\pi \in \Perm(n)$.
    Sezte
    \begin{math}
        E_\pi = \Set{ij & (i-j)(\pi(i) - \pi(j)) < 0} \subset \binom{V}{2}.
    \end{math}
    Ein Graph $G = (V, E)$ heißt \emphdef{Permutationsgraph}, wenn er isomorph zu $G_\pi := (V_n, E_\pi)$ ist.
\end{df}

\begin{math}
    \pi &= \Matrix{1 & 2 & \hdots & n \\ \pi(1) & \pi(2) & \hdots & \pi(n)}. \\
    \_\pi &:= \Matrix{1 & 2 & \hdots & n \\ \pi(n) & \pi(n-1) & \hdots & \pi(1)}. \\
    \pi^{-1} &:= \Matrix{\pi(n) & \pi(n-1) & \hdots & \pi(1) \\ 1 & 2 & \hdots & n}.
\end{math}
Es gilt $\_{E_\pi} = E_\pi$.

\begin{st}
    Jeder Permutationsgraph $G_\pi$ ist transitiv orientierbar.
    \begin{proof}
        Setze
        \begin{math}
            F = \Set{(i,j) & i < j \land \pi(i) > \pi(j) }.
        \end{math}
    \end{proof}
\end{st}

\begin{st}
    $G = (V, E)$ ist Permutationsgraph genau dann, wenn $G$ und $\_G$ transitiv orientierbar sind.
    \begin{proof}
        \ProofImplication ist trivial, zeige \ProofImplication*

        Sei $F_1$ eine transitive Orientierung für $G$ und $F_2$ eine transitive Orientierung für $\_G$.
        $F_1 \cup F_2$ ist auch eine transitive Orientierung.
        Angenommen für $(x,y) \in F_1$, $(y, z) \in F_2$ wäre $(z,x) \in F_1 \cup F_2$.
        Für $(z,x) \in F_1$ ist $(z, y) \in F_1 \subset E$, ein Widerspruch, da $(y,z) \not\in E$.
        $(z,x) \in F_2$ ist auch unmöglich.

        Nun gilt $(x,z) \in E \cup \_E$.
        Falls $(x, z) \in E$, dann ist $(x, z) \in F_1 \subset F_1 \cup F_2$.

        $(V, F_1 \cup F_2)$ ist transitive Orientierung des vollständigen Graphen.
        Ohne Einschränkung sei $(V, F_1 \cup F_2)$ eine lineare Orientierung $(\Set{1, \dotsc, n}, <)$.

        Betrachte $F' = F_1^{-1} \cup F_2$ (ist lineare Ordnung, die zu $\pi^{-1}$ gehört).
        Dann gilt für $i < j$:
        \begin{math}
            (j,i) \in F_1^{-1} \cup F_2 \iff (j,i) \in F_1^{-1} \iff (i,j) \in F_1.
        \end{math}
        Also liefert $F'$ die Permutation $(\pi(i) - \pi(j))(i - j) < 0 \iff (i,j) \in F_1$.
        $G_\pi = (\Set{1, \dotsc, n}, F_1 \cup F_1^{-1})$.
    \end{proof}
\end{st}

\begin{st}
    Permutationsgraphen sind perfekt.
    \begin{proof}
        $\chi$-Perfektheit:

        Sei $G = (V, F \cup F^{-1})$ für eine transitive Orientierung $(V, F)$.
        Setze die minimalen Elemente:
        \begin{math}
            M := \Set{x \in V & \forall y \in V : (y,x) \not\in F}
        \end{math}
        $G' := G - M$ ist transitiv orientierbar.
        Also induktiv $\chi(G') = \omega(G')$ und $\alpha(G') = \kappa(G')$.

        Cliquen sind genau linear geordnete Mengen.
        Sei $K \subset G'$ Clique, dann existiert $x \in M$

        Also $\omega(G) = \omega(G') + 1$.
        Also färbe $x$ mit freier Farbe.
        Für alle $x,y \in M$ gilt $(x,y) \not\in E$, also $\chi(G) \le \omega(G) \implies \chi(G) = \omega(G)$.
    \end{proof}
\end{st}

