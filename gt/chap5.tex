\chapter{Cographen}

\Timestamp{2015-12-03}

\begin{df}
    Seien $G_1 = (V_1, E_1)$, $G_2 = (V_2, E_2)$ Graphen mit $V_1 \cap V_2 = \emptyset$.
    Wir definieren die disjunkte Vereinigung
    \begin{math}
        G_1 \sqcup G_2 := (V_1 \cup V_2, E_1 \cup E_2)
    \end{math}
    und das Komplexprodukt
    \begin{math}
        G_1 \ast G_2 := (V_1 \cup V_2, E_1 \cup E_2 \cup \Set{xy & x \in V_1, y \in V_2}).
    \end{math}
\end{df}

\begin{df}
    Alle Einpunktgraphen sind Cographen.
    Sind $G_1, G_2$ Cographen, so auch $G_1 \sqcup G_2$ und das Komplexprodukt $G_1 \ast G_2$.
\end{df}

\begin{st}
    Sei $G$ ein \emphdef{Cograph}, dann auch das Komplement $\_G$.
    \begin{proof}
        Falls $G = (\Set{\ast}, \emptyset)$, dann auch $\_G = (\Set{\ast}, \emptyset)$.
        Nun ist
        \begin{math}
            \_{G_1 \sqcup G_2} &= \_{G_1} \ast \_{G_2}, \\
            \_{G_1 \ast G_2} &= \_{G_1} \sqcup \_{G_2}.
        \end{math}
    \end{proof}
\end{st}

\begin{kor}
    Je zwei der Operationen $G_1 \sqcup G_2$, $G_1 \ast G_2$, $\_G$ definieren die Klasse der Cographen.
\end{kor}

\begin{ex}
    \begin{itemize}
        \item
            Alle Graphen $G = (V, E)$ mit $|V| \le 3$ sind Cographen.
        \item
            Sei $G$ ein Cograph und $|V| \ge 2$, dann ist $G$ zusammenhängend genau dann, wenn $\_G$ nicht zusammenhängend ist.
        \item
            Für $x, y \in V$ sei $d(x,y)$ die Distanz zwischen $x$ und $y$.
            Dann gilt für $m$-zusammenhängenden Cographen $d(x,y) \le 2$ für alle $x,y \in V$.
        \item
            $P_4$ (verbundene Reihe mit vier Knoten) ist selbstdual $P_4 = \_{P_4}$ und damit kein Cograph.

            $d(a,d) = 3$.
    \end{itemize}
\end{ex}

\Timestamp{2015-12-07}

\begin{df}
    $G$ heißt $P_4$-frei, ($N$-frei), falls $P_4$ kein induzierter Untergraph von $G$ ist.
\end{df}

\begin{lem}
    $G$ ist genau dann $P_4$-frei, falls $\_G$ $P_4$-frei ist.
\end{lem}

\begin{kor}
    Jeder Cograph $G$ ist $P_4$-frei.
    \begin{proof}
        Mit vorigem Lemma genügt es disjunkte Vereinigungen zu betrachten: $P_4$ befindet sind in einer der beiden Komponenten, Aussage folgt durch Induktion.
    \end{proof}
\end{kor}

\begin{st}
    Seien $G$, $\_G$ zusammenhängend und $P_4$-frei.
    Dann ist $G$ unendlich oder $G = (\Set \ast, \emptyset)$ (Einpunktgraph).
    \begin{proof}
        Sei $|V| \ge 2$, $G = (V, E)$.
        $G$ ist zusammenhängend, wähle also $x_1, x_2 \in V$ mit $x_1x_2 \in E$.

        Induktiv sei eine Folge $(x_1, \dotsc, x_n$ konstruiert, sodass $\forall 1 \le i < j \le n$ aus $j$ gerade folgt, dass $x_ix_j \in E$ und aus $j$ ungerade folgt $x_ix_j \not\in E$.
        Diese Folge soll verlängert werden.

        Sei ohne Einschränkung $n$ gerade.
        $G$ ist zusammenhängend, also gilt $\forall x, g \in V: d(x,y) < \infty$.
        Angenommen es gäbe $x, z \in V$ mit $d(x,z) \ge 3$, dann auch $x,y \in V$ mit $d(x,y) = 3$.
        Dies bildet $P_4$ als induzierten Untergraphen, ein Widerspruch.
        Folglich muss $d(x,y) \le 2$ in $G$ und $\_G$ gelten.

        Es ist notwendigerweise $d(x_{n-1},x_n) = 2$.
        Als existiert $x_{n+1}$ mit $x_{n-1} x_{n+1} \not\in E$ und $x_n x_{n+1} \not\in E$.
        Falls $x_{n+1} = x_i$ für ein $i \le n$, so entsteht ein $P_4$: $G(\Set{x_{n-1}, x_n, x_i, x_{n+1}})$.
    \end{proof}
\end{st}

\begin{kor}
    $G$ ist $P_4$-frei genau dann, wenn $G$ ein Cograph ist.
\end{kor}


\section{Graphparameter}


Erinnerung:
\begin{math}
    \chi(G) &\ge \omega(G), \\
    \alpha(G) &\le \kappa(G).
\end{math}
Beispiel: $C_5$.


\begin{df}
    Ein Graph $G$ heißt $\chi$-perfekt (bzw. $\alpha$-perfekt), wenn für alle induzierten Untergraphen $G'$ von $G$ gilt $\chi(G') = \chi(G)$ (bzw. $\alpha(G') = \kappa(G')$).
\end{df}

\begin{st}
    $G$ ist $\chi$-perfekt genau dann, wenn $\_G$ $\alpha$-perfekt ist.
\end{st}

\begin{st}
    Bipartite Graphen und Cographen sind $\alpha$-perfekt.
\end{st}


\section{Chordale Graphen}

\begin{df}
    $G = (V, E)$ heißt \emphdef{chordal}, wenn jeder einfache Kreis mit mindesten $4$ Punkten eine Sehne hat.
\end{df}

\begin{ex}
    \begin{itemize}
        \item
            Wälder sind chordal.
        \item
            $K_n$ ist chordal.
        \item
            Es gibt triangulierte planare Graphen, die nicht chordal sind (Skizze: Sternmuster vergrößern und außen verbinden).
    \end{itemize}
\end{ex}

\begin{lem}
    Sei $G$ chordal und $G'$ ein induzierter Untegraph, dann ist $G'$ chordal.
\end{lem}

\begin{df}
    Sei $G = (V, E)$ ein Graph und $x \in V$.
    Dann heißt $x$ \emphdef{simplizial}, falls $N(x)$ eine Clique ist.
    Insbesondere ist dann $N(x) \sqcup \Set x$ auch eine Clique.

    Eine \emphdef{simpliziale Ordnung} ist eine Reihenfolge der Kanten $v_1, v_2, \dotsc, v_n$ sodass
    $\forall 1 \le i \le n$ $v_i$ simplizial in $G[v_i, \dotsc, v_n]$ (induzierter Untergraph von $v_i, \dotsc, v_n$) ist.
\end{df}

\begin{df}
    Für $a, b \in V$, $a \neq b \in E$ heißt $S \subset V$ ein $a$-$b$-Separator, falls $a$ und $b$ in verschiedenen Zusammenhangskomponenten von $G - S$ liegen (insbesondere $a,b \not\in S$).
    \begin{proof}
        Es existiert stets ein $a$-$b$-Separator (solange $ab \not\in E$).
    \end{proof}
\end{df}

\begin{st}
    Sei $G = (V, E)$ chordal und $a, b \in V$ mit $ab \not\in E$.
    Dann ist jeder minimale $a$-$b$-Separator $S$ eine Clique.
    \begin{proof}
        Seien $G_a$, $G_b$ die Komponenten von $a$, bzw $b$ in $G - S$.
        Falls $|S| \le 1$.
        Dann ist $S$ eine Clique.
        Seien also $x, y \in S$ mit $x \neq y$, zeige $xy \in E$.
        Da $S$ minimal gibt es Kanten von $x$ und $y$ nach $G_a$ und $G_b$.
        Wähle so einen einfachen Weg $x_1, \dotsc, x_k \in G_a$ mit $xx_1, x_ky \in E$.
        Analog wähle einen einfachen Weg $y_1, \dotsc, y_k \in G_a$ mit $xy_1, y_ky \in E$.
        Zusammen mit $x,y$ bildet dies einen einfachen Kreis ohne Sehne zwischen $x_i$ und $y_j$, folglich muss $xy \in E$.
    \end{proof}
\end{st}




