\chapter{Cographen}

\Timestamp{2015-12-03}

\begin{df}
    Seien $G_1 = (V_1, E_1)$, $G_2 = (V_2, E_2)$ Graphen mit $V_1 \cap V_2 = \emptyset$.
    Wir definieren die Vereinigung
    \begin{math}
        G_1 \sqcup G_2 := (V_1 \cup V_2, E_1 \cup E_2)
    \end{math}
    und das Komplexprodukt
    \begin{math}
        G_1 \ast G_2 := (V_1 \cup V_2, E_1 \cup E_2 \cup \Set{xy & x \in V_1, y \in V_2}).
    \end{math}
\end{df}

\begin{df}
    Alle Einpunktgraphen sind Cographen.
    Sind $G_1, G_2$ Cographen, so auch $G_1 \sqcup G_2$ und das Komplexprodukt $G_1 \ast G_2$.
\end{df}

\begin{st}
    Sei $G$ ein \emphdef{Cograph}, dann auch das Komplement $\_G$.
    \begin{proof}
        Falls $G = (\Set{\ast}, \emptyset)$, dann auch $\_G = (\Set{\ast}, \emptyset)$.
        Nun ist
        \begin{math}
            \_{G_1 \sqcup G_2} &= \_{G_1} \ast \_{G_2}, \\
            \_{G_1 \ast G_2} &= \_{G_1} \sqcup \_{G_2}.
        \end{math}
    \end{proof}
\end{st}

\begin{kor}
    Je zwei der Operationen $G_1 \sqcup G_2$, $G_1 \ast G_2$, $\_G$ definieren die Klasse der Cographen.
\end{kor}

\begin{ex}
    \begin{itemize}
        \item
            Alle Graphen $G = (V, E)$ mit $|V| \le 3$ sind Cographen.
        \item
            Sei $G$ ein Cograph und $|V| \ge 2$, dann ist $G$ zusammenhängend genau dann, wenn $\_G$ nicht zusammenhängend ist.
        \item
            $P_4$ (verbundene Reihe mit vier Knoten) ist selbstdual und damit kein Cograph.
    \end{itemize}
\end{ex}
