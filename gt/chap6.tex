\chapter{Perfektheit}

\Timestamp{2016-01-21}

Claude Berge Vermutung 1960 (veröffentlich 1963)
\begin{itemize}
    \item
        Schwache Vermutung: $\chi$-perfekt genau dann, wenn $\alpha$-Perfekt
    \item
        Starke Vermutung: $G$ perfekt genau dann, wenn weder $G$ noch $\_G$ ein $C_n$, $n \ge 5$ als induzierten Untergraphen enthält.
\end{itemize}

László Lovász, 1972: Beweis der schwachen Vermutung (3 Seiten).

Maria Chudnovsky, Neil Robertson, Paul Seymaur, Robin Thomas, 2009: Beweis der starken Vermutung (180 Seiten).


\begin{st}[Lovász, 1972]
    Sei $G = (V, E)$ ein Graph.
    Dann sind äquivalent
    \begin{enumerate}[(i)]
        \item
            $G$ ist $\chi$-perfekt.
        \item
            $G$ ist $\alpha$-perfekt.
        \item
            Für alle induzierten Untergraphen $H$ von $G$ gilt:
            \begin{math}
                |H| \le \omega(H) \omega(\_H).
            \end{math}
    \end{enumerate}
    \begin{proof}
        Aussage (iii) ist invariant unter Komplement-Bildung, also lassen sich (i),(ii) gleichzeitig zeigen.
        Sei $G$ zunächst $\chi$-perfekt.
        Dann gilt für einen induzierten Untergraphen $H$ die Beziehung $\chi(H) = \omega(H)$ und $\omega(\_H) = \alpha(H)$.
        Nun ist
        \begin{math}
            |H| \le \chi(H) \alpha(H),
        \end{math}
        denn jede Farbklasse ($\chi$) enthält höchstens $\alpha(H)$ Elemente.

        Zeige nun (iii) $\implies$ (i) per Widerspruch.
        Betrachte $G = (V, E)$ minimal mit (iii) und $G$ ist nicht $\chi$-perfekt.
        Ohne Einschränkung ist für alle $y \in G$ $G - y$ $\chi$-perfekt
        Setze $\omega := \omega(G)$ und $\alpha := \alpha(G)$, sowie $A_0 := \Set{y_1, \dotsc, y_\alpha} \subset V$ die unabhängige Menge (Antikette).
        Für alle $y \in G$ ist $\omega(G - y) = \omega(G) = \omega$, da sonst eine Farbe frei für $y$ ($\omega(G-y) = \chi(G-y)$ wegen $\chi$-perfekt).
        Genauer gilt für jede unabhängige Menge $U \subset G$: $\omega(G - U) = \omega$.
        Für jedes $y_i \in A_0$ wähle eine Färbung $\chi_i$ von $G - y_i$ in $\omega$ Farben.
        Sei jetzt $W \subset V$ eine Clique von $G$ mit $|W| = \omega$.

        (*) Für $y_i \not\in w$ wird jede Farbklasse $\chi_i$ genau einmal getroffen.

        (**) Für $y_i \in W$ wird jede Farbklasse von $\chi_i$ (bis auf eine) genau einmal getroffen.
        Seien $A_{(i-1)\omega + 1}, \dotsc, A_{i\omega}$ die Farbklassen von $\chi_i$ insgesamt unabhängige Mengen $A_0$, $A_1$, $\dotsc$, $A_\omega$, $A_{\omega + 1}$, $\dotsc$, $A_{\alpha \omega}$.

        Sei $V = \Set{x_1, \dotsc, x_k}$.
        Definiere Matrizen $A \in \R^{(\alpha \omega + 1) \times n}$ mit
        \begin{math}
            (a_{ik}) = \begin{cases}
                0 & \text{für $x_k \not\in A_i$} \\
                1 & \text{für $x_k \in A_i$}
            \end{cases}.
        \end{math}
        Für $0 \le i \le \alpha\omega$ definiere $W_i \subset G - A_i$ und $|W_i| = \omega$ und $W_i$ ist Clique. 
        Setze $B = (b_{kj}) \in \R^{n \times (\alpha\omega + 1)}$ mit
        \begin{math}
            b_{kj} = \begin{cases}
                1 & \text{für $x_k \in W_j$} \\
                0 & \text{sonst}
            \end{cases}
        \end{math}
        Sei $C = A B \in \R^{(\alpha\omega + 1) \times (\alpha\omega + 1)}$,
        \begin{math}
            c_{ij} = \sum_{k=1}^n a_{ik} b_{kj} = | A_i \cap W_j | \in \Set{0,1}.
        \end{math}
        Für $i = j$ gilt $c_{ij} = 0$.
        Für $i \neq j$ gilt $|W_j \cap A_i| \ge 1$ (nach Lemma) und somit
        \begin{math}
            c_{ij} = \begin{cases}
                1 & \text{für $i \neq j$} \\
                0 & \text{sonst}
            \end{cases}.
        \end{math}
        Nun ist $\alpha\omega + 1 = \rg(C) \le \rg(A) \le n$.
    \end{proof}
\end{st}

\begin{lem}
    Sei $W \subset G$ Clique mit $|W| = w$.
    Dann existiert genau ein $i \in \Set{0, \dotsc, \alpha \omega}$ mit $W \cap A_i = \emptyset$.
    \begin{proof}
        \begin{enumerate}[1)]
            \item
                $W \cap A_0 = \emptyset$, dann wird jede Farbklass von $\chi_i$ genau einmal getroffen.
            \item
                $W \cap A_0 \neq \emptyset$, dann $W \cap A_0 = \Set{y_j}$.
                Für die Farbklassen, die zu $y_j$ gehören mit $i \neq j$ folgt die Behauptung mit (*)
                Für $y_i$ werden nach (**) alle Farbklassen zu $\chi_i$ bis auf eine getroffen.
        \end{enumerate}
    \end{proof}
\end{lem}


