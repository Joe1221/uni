\chapter{Ausgewählte Themen}


\section{Ramsey-Zahlen}


\Timestamp{2016-01-25}


\begin{st}
    Sei $G = (V,E)$ ein unendlicher Graph.
    Dann existiert ein unendlicher induzierter Untergraph $V'$, sodass $V'$ Clique oder unabhängig ist, d.h. $\alpha(G) + \omega(G) = \infty.$.

    Allgemeiner:
    Sei $G$ vollständig und $f: \binom{V}{2} \to \Set{0, \dotsc, c - 1}$ eine Färbung in $c$ Farben.
    Dann existiert ein monochromatischer unendlicher Untergraph.
    \begin{proof}
        Sei ohne Einschränkung $V = \N$.

        Wähle für den Knoten $i = 1$ eine Farbe $g(i)$, sodass für unendlich viele $j > i$ gilt $f(ij) = g(i)$, eliminiere die restlichen Knoten.
        Wiederhole für $i = 2, 3, \dotsc$ mit den jeweils verbleibenden Knoten.

        Unter den verbleibenden (unendlich vielen) Knoten wähle eine Farbe mit unendlich vielen zugehörigen Knoten.
        Diese bilden eine Clique.
    \end{proof}
\end{st}

\begin{df}
    Ein \emphdef{Hypergraph} $G = (V, E)$ besteht aus einer Menge $V$ von \emphdef{Knoten} und einer Menge $E \subset \binom{V}{k}$ von \emphdef{Hyperkanten} für ein $k \in \N$.

    Sei $C = \Set{0, \dotsc, c-1}$ eine Menge von $c$ Farben.
    Eine Färbung des vollständigen Hypergraphen $(V, \binom{V}{k})$ ist eine Abbildung $f: \binom{V}{k} \to C$.

    Eine Teilmenge $V' \subset V$ heißt \emphdef{monochromatisch}, falls $f(K) \in C$ konstant ist für alle $K \in \binom{V'}{k}$.
\end{df}

\begin{st}[Ramsey, 1930]
    Für alle $n, k, c \in \N$ existiert eine Zahl $R(n,k,c) \in \N$ minimal mit der folgenden Eigenschaft:
    % Für jeden Hypergraphen G = (V,E) mit $|V| \ge R(n,k,c)$ und jeder Färbung ...
    Ist $f: \binom{V}{k} \to \Set{0, \dotsc, c-1}$ eine Färbung für $|V| \ge R(n,k,c)$, dann existiert $V' \subset V$ mit $|V'| \ge n$ und $V'$ ist monochromatisch.
    \begin{proof}
        Induktionsanfang:
        \begin{math}
            R(n,1,c) = c(n-1) + 1.
        \end{math}
        Sei $r \in \N$ und $f: \binom{V}{k} \to C = \Set{0, \dotsc, c-1}$ mit $|V| \ge r$.
        Wir definieren induktiv Mengen $A_m, B_m \subset V$ mit den folgenden Eigenschaften:
        \begin{enumerate}[(i)]
            \item
                $\forall a \in A_m \forall b \in B_m: a < b$,
            \item
                $|A_m| = m$,
            \item
                Es gibt eine Färbung $g: \binom{A_m}{k-1} \to C$ mit der Eigenschaft, dass für alle $K \subset \binom{A_m}{k-1}$ gilt $f(K, b) = g(K)$ für alle $b \in B_m$.

                Für $K \in \binom{V}{k-1}$ und $b \in V$ mit $b > \max(K)$ bezeichne $K, b$ die Menge $K \cup \Set{b}$.
        \end{enumerate}
        Setze $A_0 :=  \emptyset$ und $B_0 := V$.

        Seien $A_m$ und $B_m$ bereits definiert.
        Wähle $a_{m+1} := \min B_m$ und setze $A_{m+1} = A_m \cup \Set{a_{m+1}}$.
        Wir wählen jetzt $B_{m+1} \subset B_m \setminus \Set{a_{m+1}}$ geeignet.

        Gesucht ist eine Erweiterung der Färbung zu $g: \binom{A_{m+1}}{k-1} \to C_m$ mit Eigenschaft (iii).
        $g(K)$ ist noch nicht definiert genau dann, wenn $a_{m+1} \in K$.
        Es müssen also noch insgesamt $\binom{m}{k-2}$ solcher $K$ gefärbt werden.

        Für $a_{m+1} \in K \in \binom{A_m}{k-1}$ wähle $s \in C$ mit $\Set{b \in B_m & f(K,b) = s}$ maximal.
        Dann färbe $g(K) = s$.

        Setze
        \begin{math}
            B_{m+1} = \Set{ b \in B_m \setminus \Set{a_{m+1}} & \forall a_{m+1} \in K \in \binom{A_{m+1}}{k-1} : f(K,b) = s_K }.
        \end{math}
        Dann gilt (i), (ii), (iii).

        Betrachte den Ausdünnungsprozess nacheinander für die $a_{m+1} \in K \in \binom{A_{m+1}}{k-1}$.
        Jedes Mal können wir $\frac{1}{|C|}$ Knoten weiter fortfahren.

        Wenn wir also $|B_{m+1}| \ge r$ garantieren wollen, müssen wir mit $|B_m| \ge c^{\binom{m}{k-2}} r$ starten.
        Für $R(n,k,c)$ müssen wir $A_m$ mit $|B_{m+1}| \ge 1$ erreichen mit $m = n - 1$.
        Also ist
        \begin{math}
            R(n,k,c) \le \prod_{m < R(n,k-1,c)} c^{\binom{m}{k-2}}
            = c^{\sum_{m < R(n,k-1,c)} \binom{m}{k-2}}
            = c^{\binom{R(n,k-1,c)}{k-1}}
        \end{math}
    \end{proof}
\end{st}

\begin{ex}
    $R(n) := R(n, 2, 2)$ ist die Situation für Graphen $(V, E)$.
    \begin{itemize}
        \item
            $R(1) = 0$
        \item
            $R(2) = 2$,
        \item
            $R(3) = 6$,
            denn $R(3) \neq 5$ wegen
            $\alpha(C_5) = 2 = \omega(C_5)$
\Timestamp{2016-01-28}
        \item
            $R(n,1,c) = c(n-1) + 1$
        \item
            $R(n,2,c) \le c^{c(n-1) + 1} \le c^{nc} = 2^{nc \log_2 c}$.
        \item
            $R(n) \le 4^n$.
    \end{itemize}
\end{ex}

\begin{st}[Erdös, 1947]
    $R(n) = R(n,2,2) \ge 2^{\frac{n}{2}}$ für $n \ge 2$.
    Insbesondere $R(n) \in 2^{\LandauO(n)}$.
    \begin{proof}
        Es genügt, dies für $n \ge 4$ zu zeigen, den $R(2) = 2$, $R(3) = 6$.
        Eine Färbung des vollständigen Graphen $K_m$ ist von der Form $\binom{K_m}{2} \to \Set{0,1}$.
        Es gibt jetzt $2^{\binom{m}{2} - \binom{n}{2} + 1}$ Färbungen, die eine feste Clique $K_n$ monochromatisch machen.
        Daher gibt es maximal $\binom{m}{n} 2^{\binom{m}{2} - \binom{n}{2} + 1}$ Färbungen, die \emph{irgendeine} Clique mit $n$ Knoten monochromatisch macht.

        Angenommen $R(n) = m < 2^{\frac{n}{2}}$.
        Betrachte
        \begin{math}
            \binom{m}{n} \dfrac{2^{\binom{m}{2} - \binom{n}{2} + 1}}{2^{\binom{m}{2}}}
            &= \binom{m}{n} 2^{-\binom{n}{2} + 1}
            \le \dfrac{m^n 2^{-\binom{n}{2} + 1}}{n!}
            \le \frac{m^n}{2^n} 2^{-\binom{n}{2} + 1} \\
            &< 2^{\frac{n^2}{2} - \binom{n}{2} + 1 - n}
            = 2^{\frac{n^2}{2} - \frac{n^2}{2} + \frac{n}{2} + 1 - n}
            = 2^{1 - \frac{n}{2}}
            < 1,
        \end{math}
        ein Widerspruch.
    \end{proof}
\end{st}


\section{Lipton Tarjan}

\Timestamp{2016-02-01}

\begin{st}
    Sei $G = (V, E, F)$ ein planarer Graph, $n = |V|$.
    Dann existiert eine Menge $S$ (Separator) und eine Partition $V = A \cup S \cup B$ mit $S$ separiert $A$ und $B$ und $|S| \le 4 \sqrt n$ und $|A|, |B| \le \frac{2}{3} n$.
    \begin{note}
        Mengen $A, S, B$, die obige Aussage erfüllen, können in Linearzeit $\LandauO(n)$ beantwortet werden (Euler: $|E| \le 3n - 6$).
    \end{note}
    \begin{proof}
        Der Beweis ist trivial, falls $n \le 4 \sqrt n$, setze dann $S := V$.
        Sei also $n > 16$.

        Bilde den Levelgraph:
        Wähle Startknoten $s \in V$ und setze
        \begin{math}
            L(t) := \Set{v \in V & d(s,v) = t}.
        \end{math}
        Also insbesondere $L(0) = \Set{s}$.
        Das maximale Level sei $L(l) \neq \emptyset$, sodass $L(l+1) = \emptyset$.
        \begin{math}
            \Big| \bigcup_{t<t_1} L(t) \Big|
            \le \frac{1}{2} n \ge
            \Big| \bigcup_{t>t_1} L(t) \Big|.
        \end{math}
        $L(t_1)$ ist ein Separator.
        Falls $|L(t_1)| \le 4 \sqrt n$, so sind wir fertig.
        Sei also $|L(t_1)| > 4 \sqrt n$.
        Definiere $t_0$ so, dass $t_0$ maximal, $t_0 < t_1$ und $|L(t_0)| \le \sqrt n$.
        Beachte $t_0$ existiert, denn $|L(0)| = 1$.
        Definiere $t_2$ so, dass $t_2$ minimal, $t_2 > t_1$ und $|L(t_2)| \le \sqrt n$.
        Beachte $t_2$ existiert, denn $L(l+1) = \emptyset$.

        Beachte $t_2 - t_0 < \sqrt n$. % ??
        Setze $C := \bigcup_{t<t_0} L(t)$, $D := \bigcup_{t_0 < t < t_2} L(t)$, $E := \bigcup_{t_2 < t} L(t)$, also
        \begin{math}
            V = C \dotcup L(t_0) \dotcup D \dotcup L(t_2) \dotcup E.
        \end{math}
        Wir sind fertig, falls $|D| \le \frac{2}{3} n$.
        Setze dann $S := L(t_0) \cup L(t_2)$ mit $|S| \le 2 \sqrt n$ und $A, B$ Übung.

        Sei also $|D| > \frac{2}{3}n$.
        Definiere $L = \Set{s} \cup D$, wobei $s$ mit allen Knoten aus Level $L(t_0 + 1)$ verbunden wird.

        Es genügt, den Satz für $L$ zu beweisen mit $|S| \le \sqrt 2\sqrt{n}$, $n = |V|$.
        Dann ist der richtige Separator für $G$: $S \cup L(t_0) \cup L(t_1)$.
        Für $L = A' \dotcup S \dotcup B'$ mit $|A'| \ge |B'|$ und ohne Einschränkung $|c| \ge |E|$ ist dann $A := A' \cup E$, $B := B' \cup C$.
        Ansatz: sei $\alpha$ so, dass $|D| = \alpha n$ mit $\alpha \ge \frac{2}{3}$.

        Ohne Einschränkung sei $L$ trianguliert, d.h. $L$ zusammenhängend und alle Facetten sind Dreiecke.
        Berechne einen Spannbaum $T$ von $L$ von rechts nach links:
        Für jeden Knoten $v \in L(t)$, $t > 0$ wähle genau einen Kante an einem $u \in L(t-1)$.

        $e \in E = E(L)$ heißt \emphdef{Brücke}, falls $e \in E \setminus T$.
        Für $u, v \in V$ sei $T[u,v]$ der eindeutige Pfad von $u$ nach $v$ in $T$.
        Wenn $e = (u,v)$ eine Brücke ist, dann ist $T[uv] \ast e$ ein Kreis der Länge $\le 2\sqrt n$.

        Zeige: Es gibt eine Brücke $e = (u,v)$ und $S = e \ast T[u,v]$ mit $A$ innen und $B$ außen und $|A|, |B| \le \frac{2}{3} |L|$.

        \begin{lem}
            Sei $G$ zusammenhängend, planar mit $n \ge 3$ Knoten und $T \subset E$ ein Spannbaum.
            Dann ist $E \setminus T = T^*$ ein Spannbaum im Dualgraphen $G^*$.
            \begin{proof}
                Es gilt $n - e + f = 2$ dank Euler.
                Dann ist $e = |E| = (n-1) + (f-1)$.
                Folglich ist $|T^*| = f - 1$ die richtige Anzahl Kanten im Dualgraphen, beachte: der Dualgraph hat $f$ Knoten.
                $T^*$ ist also Spannbaum, sowie es keinen Kreis gibt.
                Einen Kreis kann es in $T^*$ nicht geben, da er einen Knoten in $G$ isolieren würde.
            \end{proof}
        \end{lem}

        Sei nun $n = |L| = |D| + 1$.
        Berechne $T^*$.
        $T^*$ hat Blätter.
        Mache ein Blatt zur endgültigen Wurzel von $T^*$.
        Ohne Einsschränkung ist dieses Blatt die äußere Facette.
        Sei $e = (u,v)$ eine Brücke.
        Dann ist $c(e) := \ast e T[u,v]$ der Kreis.
        Definiere $I(E)$ als die Anzahl der inneren Knoten.
        Bei der Wurzel ist $I(r) = n - 3$.
        $c(e)$ sei eine Listendarstellung des Kreises $c(e) = (u, u_1, \dotsc, u_k, v)$ und $|c(e)|$ die Anzahl der Knoten.

        An der Wurzel gilt $I(r) + |c(r)| = n$.
        Berechne $(I(e), c(e), |c(e)|)$ von den Blätter zur Wurzel.

        \begin{enumerate}[1.]
            \item
                Fall 1 (Blatt)
                Hier ist $I(e) = 0$, $c(e) = (u,x,v)$.
                Suche jetzt von den Blättern zur Wurzel die erste Brücke mit $I(e) + |c(e)| \ge \frac{1}{3} n$.
            \item
                Fall 2 (kein Blatt und $u' \in T[u,v]$).
                Hier ist $I(e) = I(e')$, $c(e) = u \ast c(e')$, $c(e) = |c(e')| + 1$.
                Angenommen $I(e') + c(e') < \frac{1}{3} n$, dann ist
                \begin{math}
                    I(e) + c(e) < \frac{1}{3}n + 1 \le \frac{2}{3} n
                    \le \frac{2}{3} n.
                \end{math}
                Ok, falls $I(e) + |c(e)| \ge \frac{1}{2} n$ erstmalig.
            \item
                Fall 3 (kein Blatt und $u' \notin T[u,v]$).
                Hier ist $I(e) = I(e') + 1$ und $c(e) = (u,u_1, \dotsc, v)$ falls $(u', u, u_1, \dotsc, v) = c(e')$.
                Dann ist $|c(e)| = |c(e')| - 1$.

                Also ist $I(e) + |c(e)| = I(e') + |c(e')|$.
            \item
                Fall 3 (Dreieck mit keinen Spannbaumkanten)
                Schon bekannt sind
                \begin{math}
                    T[y,v] &= (y, \dotsc, x, v_1, \dotsc, v_k, v) = c(e'') \\
                    T[y,u] &= (y, \dotsc, x, u_1, \dotsc, u_l, u) = c(e')
                \end{math}
                mit $u_1 \neq v_1$.
                Dann ist
                \begin{math}
                    T[u,v] = c(e) = (v, v_k, \dotsc, v_1, x, u_1, \dotsc, u_k, u).
                \end{math}
                Damit gilt mit $p$ die Anzahl der Knoten $(y, \dotsc, x)$.
                \begin{math}
                    I(e) &= I(e') \cup I(e'') + |p| - 1,
                    |c(e)| &= |c(e')| + c(e'') - 2|p| + 1.
                \end{math}
                Falls $I(e') + |c(e')| \le \frac{1}{3}n$ und $I(e'') + |c(e'')| \le \frac{1}{3}n$, dann folgt $|I(e) + |c(e)|| \le \frac{2}{3} n$.
        \end{enumerate}
    \end{proof}
\end{st}

