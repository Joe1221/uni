\chapter{Färbbarkeit und Heiratssätze}


\section{Färbbarkeit}


\Timestamp{2015-11-05}


\begin{df}
    Sei $G = (V, E)$, $E \subset \binom{V}{2}$.
    Eine $k$-Färbung ist eine Abbildung $c: V \to \Set{1, \dotsc, k}$ mit
    \begin{math}
        xy \in E \implies c(x) \neq c(y).
    \end{math}
\end{df}

$G$ besitzt eine 2-Färbung genau dann, wenn $G$ bipartit ist.
Wir können kürzeste Wege in $\LandauO(|E| + |V|\log |V|)$ lösen (Dijkstra).
Dijkstra mit Weglängen 0, 1 kann feststellen, ob ein Graph 2-färbbar ist, dies ist sogar in Linearzeit möglich.

Reduktion auf bekannte NP-vollständige Probleme (3-SAT, NAE-SAT, etc.)

\begin{st}
    3-Färbbarkeit ist NP-vollständig.
    \begin{proof}
        NAE-SAT: $F = C_1 \land \dotsb \land C_m$ mit Klauseln $C_j = \tilde x_{1j} \lor \tilde x_{2j} \lor \dotsb \lor \tilde x_{n,j}$, wobei $\tilde x \in \Set{x, \_x}$ und einer Belegung $\sigma: \Set{x_1, \dotsc, x_n} = \Var(F) \to \B = \Set{0,1}$, $\sigma(F) = 1 = (1-\sigma)(F)$.

        Reduktion $\text{3-SAT} \le \text{NAE-SAT}$.
        Sei $F \in \text{3-KNF}$ (3-Konjuktive Normalform, s.o.).
        Wähle neue Variable hinzu und ersetze jede Klausel $C_j = (\tilde x_{1j} \lor \tilde x_{2j} \lor \tilde x_{3j})$ durch
        \begin{math}
            C_j = (\tilde x_{1j} \lor \tilde x_{2j} \lor \tilde x_{3j} \lor y).
        \end{math}
        und $F' = (C_1' \land \dotsb \land C_m')$.
        Sei $\sigma(F) = 1$.
        Setze $\sigma(y) = 0$, dann gilt $F' \in \text{NAE-SAT}$.

        Sei umgekehrt $F' \in \text{NAE-SAT}$ mit $\sigma(F') = (1-\sigma)(F') = 1$.
        Unterscheide die Fälle $\sigma(y) = 1$ und $\sigma(y) = 0$.
        Im ersten Fall ist $(1-\sigma)(y) = 0$, sonst mit Symmetrie.
        Also ohne Einschränkung $\sigma(y) = 0$.
        Es folgt notwendigerweise $\sigma(F) = 1$.

        $\text{3-NAE-SAT} = \text{NAE-SAT} \cap \text{3-KNF}$.
        Von 4 Variablen auf 3 Variablen pro Klausel.
        Setze für $C_j = (\tilde x_{1j} \lor \tilde x_{2j} \lor \tilde x_{3j} \lor \tilde x_{4j})$
        \begin{math}
            C_j' = (\tilde x_{1j} \lor \tilde x_{2j} \lor y_j) \land (\_y_{j} \lor \tilde x_{3j} \lor \tilde x_{4j})
        \end{math}
        Durch geeignete Wahl von $y_j$ ist $C_j'$ erfüllt genau dann, wenn $C_j$ erfüllt war.

        Konstruiere Graphen: Dreiecke bestehend aus den 3-Klauseln, und verbundenen Paaren $x_i, \_{x_i}$ (Belegungsvariablen), verbunden zu einer Wurzel.
        Verbinde immer $x_i$ mit $\_{x_i}$.

        Ist der Graph 3-färbbar, so ist ohne Einschränkung die Wurzel blau und die Variablen definieren eine gültige Belegung.

        Andere Richtung: Für eine erfüllte Belegung färbe die Dreiecke: zwangsläufig eins rot, eins grün, färbe den anderen blau.
    \end{proof}
\end{st}

\begin{st}
    Jeder planare Graph ist 5-färbbar.
    \begin{proof}
        In jedem planaren Graphen existiert ein $x \in V$ mit Grad $d(x) \le 5$ (Widerspruch mit Euler-Formel).
        Betrachte $G'$ mit $V' = V \setminus \Set x$, wobei $d(x) \le 5$.
        Dieser ist per Induktion 5-färbbar.
        Es kommt kein $K_5$ vor. Skizze: Pentagram um $x$.
        Ohne Einschränkung $ab \not\in E$, identifiziere $a$ und $b$ und färbe per Induktion.
        Also kann $x$ die verbleibende Farbe erhalten.
    \end{proof}
\end{st}


\section{Heiratssätze}

\Timestamp{2015-11-09}

\begin{st}[Philip Hall, 1935]
    Seien $G = (A \sqcup B, E)$ ein bipartiter Graph mit $E \subset A \times B$ und $N(X) := \Set{b \in B & \exists a \in X: (a,b) \in E}$, $X \subset A$.
    Es gelte die „Heiratsbedingung“:
    \begin{math}
        \forall X \subset A \exists N(X) \subset B : |N(X)| \ge |X|.
    \end{math}
    Dann existiert ein (maximales) Matching $M$ mit $|M| = |A|$ (d.h. $M \subset E$ mit $vw \in M \implies vx \not\in M$).
    \begin{proof}
        Einfacher Algorithmus in $\LandauO(nm)$.
        Karp Hopcraft (1973) liefert die bisher beste Zeit $\LandauO(\sqrt{n} m)$.
    \end{proof}
\end{st}

\section{Stabile Heirat}

Gale-Shapley (Stabile Heirat), 1962.
Sei $n = |A| = |B|$.

Jedes $a \in A$ hat eine Präferenzliste $b_{a(1)} > b_{a(2)} > \dotsb > b_{a(n)}$ und jedes $b \in B$ entsprechend.
Eine Matching $M$ ist nicht stabil, wenn $\exists ab \in M, a' \in A, b' \in B$, sodass $a' > a$ für $b$ und $b' > b$ für $a$.

\begin{st}[Gale Shapley]
    Es gibt eine stabile Heirat und sie lässt sich nach maximal $n^2$ Schritten finden.
    \begin{proof}
        Jedes $b \in B$ fragt in der Reihenfolge einer Präferenz ein $a \in A$, ob Paarung möglich ist, es sei denn, $a$ wurde bereits gefragt, oder $a$ hat $b$ bereits verlassen.
        Jedes $a \in A$ paart sich nur dann neu, wenn der fragende $b$ höher in der Präferenzliste von $a$ steht.

        Im Laufe der Zeit steigen die $a \in A$ auf und die $b$ steigen ab.
        Der Algorithmus terminiert nach $n^2$ Schritten.

        Die resultierende Heirat ist stabil, betrachte dazu eine Heirat $a_i b$:
        Alle $a_j$ mit $j < i$ würde $b$ nicht akzeptieren, alle $a_j$ mit $j > i$ möchte $b$ nicht haben.

        Übung: Der Algorithmus ist optimal für $B$.
    \end{proof}
\end{st}

\begin{df}
    Ein Graph $G = (V, E)$ ist gerichtet, wenn $E \subset V \times V$.
    Wir betrachten $G = (V, E)$ ungerichtet als Spezialfall mit $E = E^{-1}$ und $(x,x) \in E$.

    Für $A, B \subset V$ ist eine Folge $(x_0, x_1, \dotsc, x_k)$ ein \emphdef{$AB$-Pfad}, falls $x_0 \in A$, $x_k \in B$,
    $x_1, \dotsc, x_{k-1} \not\in A \cup B$ und $(x_{i-1}, x_i) \in E$ für alle $1 \le i \le k$.

    Ein \emphdef{$AB$-Separator} ist eine Menge $C \subset V$, sodass jeder $AB$-Pfad mindestens einen Knoten aus $C$ benutzt.
\end{df}

\begin{st}[Menger, 1929]
    Die maximale Zahl paarweiser Knoten-disjunkter $AB$-Pfade ist gleich der minimalen Größe eines $AB$-Separators.
    \begin{proof}
        Eine Richtung ist klar: Sei $C$ ein $AB$-Separator, dann existieren höchstens $|C|$ knotendisjunkte $AB$-Pfade.

        Für die Umkehrung: Induktion nach Anzahl Kanten $|E|$.
        Für $E = \emptyset$ ist $A \cap B$ ein minimaler $AB$-Separator und es gibt $|A \cap B|$ Knoten-disjunkte Einpunktpfade.
        Sei jetzt $xy \in E$.
        Setze $G' = G - xy$ (Graph ohne Kante $xy$).
        Sei $k$ die maxmiale Zahl Knoten-disjunkter $AB$-Pfade in $G$.
        Sei jetzt $C$ ein minimaler $AB$-Separator in $G'$.
        Falls $|C| = k$, dann sind wir fertig.

\Timestamp{2015-11-12}
        Also gilt $|C| \le k-1$ (sogar $|C| = k-1$).
        Jetzt sind $S := C \cup \Set{x}$ und $T := C \cup \Set{y}$ $AB$-Separatoren in $G$.

        Sei $D$ ein $AS$-Separator in $G$, dieser ist insbesondere ein $AB$-Separator.
        Es folgt $|D| = k$, analog für $T$.
        Mit Induktion existieren $k$ knotendisjunkte Pfade von $A$ nach $S$ in $G'$, bzw. von $T$ nach $B$ in $G'$ und damit von $A$ nach $B$.
    \end{proof}
\end{st}


\paragraph{Graph-Parameter}


Sei $G = (V, E)$, $E \subset \binom{V}{2}$.
\begin{math}
    \alpha(G) := \max\Set{|I| & \text{$I \subset V$ unabhängig}},
\end{math}
wobei $I \subset V$ unabhängig ist, wenn $xy \in E \implies |I \cap \Set{x,y}| \le 1$.
\begin{math}
    \omega(G) := \max\Set{|C| & \text{$C \subset V$ ist Clique}},
\end{math}
wobei $C \subset V$ Clique ist, wenn $\forall x,y \in C : xy \in E$.

Es gilt für den komplementären Graphen
\begin{math}
    \alpha(G) = \omega(\_G).
\end{math}

Definiere
\begin{math}
    \kappa(G) &:= \min\Set{n \in \N & \text{$\exists C_1, \dotsc, C_n$ Clique mit $V = \bigcup_{i=1}^n C_i$}}, \\
    \chi(G) &:= \min\Set{n \in \N & \text{$\exists$ Färbung $c: V \to \Set{1, \dotsc, n}$}}.
\end{math}
Es gilt stets
\begin{math}
    \chi(G) &\ge \omega(G),\\
    \kappa(G) &\ge \alpha(G).
\end{math}

Für alle induzierte Untergraphen?
Wir nennen $G$ $\chi$-perfekt, wenn $\chi(G) = \omega(G)$.
Beispiele sind bipartite Graphen

Wir nennen $G$ $\alpha$-perfekt, wenn $\alpha(G) = \kappa(G)$.

Interessante Sätze sind:

\begin{st}
    Sei $G$ ein Graph und alle induzierten Teilgraphen sind $\chi$-perfekt genau dann, wenn alle induzierten Teilgraphen $\alpha$-perfekt sind.
\end{st}

\begin{st}
    Ein Graph ist $\chi$-, bzw. $\alpha$-perfekt (also alle induzierten Untergraphen $G'$ erfüllen $\alpha(G') = \kappa(G')$ und $\chi(G') = \omega(G')$) genau dann, wenn kein induzierter $C_k$ existiert mit $k$ ungerade und $k \ge 5$.
\end{st}

\begin{df}
    Ein \emphdef{Träger} $T$ (vertex cover) ist eine Teilmenge $T \subset V$ mit
    \begin{math}
        \forall xy \in E: \Set{x,y} \cap T \neq \emptyset.
    \end{math}
    Eine \emphdef{Paarung} $M$ (matching) ist eine Teilmenge $M \subset E$ mit
    \begin{math}
        xy \in M \land x'y' \in M \implies xy = x'y' \lor \Set{x,y} \cap \Set{x',y'} = \emptyset.
    \end{math}
\end{df}

Wir haben folgende Dualität:
\begin{lem}
    $T \subset V$ ist Träger genau dann, wenn $V \setminus T$ unabhängig ist.
\end{lem}

\begin{st}[König, 1931]
    Sei $G = (A \sqcup B, E)$ ein bipartiter Graph.
    Dann gilt
    \begin{math}
        \max\Set{|M| & \text{$M$ ist Matching}}
        = \min \Set{|T| & \text{$T$ ist Träger}}.
    \end{math}
    \begin{proof}
        Klar ist $\le$: jeder Träger muss für mindestens $|M|$ Knoten abdecken für ein beliebiges Matching.

        Für die umgekehrte Richtung: Betrachte den Beweis des Heiratssatzes.
        Skizze: Bobs/Alices, finde Verbesserungsweg, Widerspruch zu maximalem Matching.
    \end{proof}
\end{st}

\begin{ex}
    Übungen:
    \begin{itemize}
        \item
            Aus vorigem Satz folgt $\alpha$-Perfektheit für bipartite Graphen.
        \item
            Aus dem Heiratssatz folgt König.
            Umkehrung gilt auch.
        \item
            Menger impliziert Heiratssatz/König.
    \end{itemize}
\end{ex}
