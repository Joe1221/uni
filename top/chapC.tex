%fixme: \scr U vs U und \scr T vs \tau

\chapter{Topologische Räume}

\coursetimestamp{28}{10}{2013}

\begin{df}
	Eine \emph{Topologie} auf einer Menge $X$ ist ein System $\scr T \subset \scr P(X)$ von Teilmengen von $X$ mit folgenden Eigenschaften
	\begin{enumerate}[O1)]
		\item
			$\emptyset, X \in \scr T$,
		\item
			$O_1, \dotsc, O_n \in \scr T \implies O_1 \cap \dotsb \cap O_n \in \scr T$,
		\item
			$O_i \in \scr T$ für $i \in I \implies \bigcup_{i\in I} O_i \in \scr T$.
	\end{enumerate}
	Das Paar $(X, \scr T)$ nennen wir dann \emph{topologischer Raum}.

	Eine Teilmenge $Y \subset X$ nennen wir \emph{offen}, wenn $Y \in \scr T$, und \emph{abgeschlossen}, wenn $X \setminus Y \in \scr T$.
\end{df}

\begin{ex}[Kleinste Beispiele]
	\begin{itemize}
		\item
			$X = \emptyset, \scr T = \{\emptyset\}$,
		\item
			$X = \{a\}, \scr T = \{ \emptyset, X \}$,
		\item
			Für $X = \{a,b\}$ gibt es vier Topologien:
			\begin{align*}
				&\{\emptyset, X\}, &
				&\{\emptyset, \{a\}, X\}, &
				&\{\emptyset, \{b\}, X \}, &
				&\{\emptyset, \{a\}, \{b\}, X\}.
			\end{align*}
		\item
			Die Anzahl möglicher Topologien auf einer Menge mit $n$ Elementen wird in \ref{tab:topcount} dargestellt.
	\end{itemize}
	\begin{table}[h]
		\centering
		\begin{tabular}{c|c}
			$n=|x|$ & $t_n = $ Anzahl der Topologie auf $X$ \\ \hline
			0 & 1 \\
			1 & 1 \\
			2 & 4 \\
			3 & 29 \\
			4 & 355 \\
			5 & 6942
		\end{tabular}
		\caption{Anzahl der Topologien auf einer Menge $X$ mit $n$ Elementen, asymptotisches Verhalten: $\log t_n \sim \f {n^2}4$}
		\label{tab:topcount}
	\end{table}
\end{ex}

\begin{ex}[diskrete Topologie]
	$\scr T = \scr P(X)$ auf $X$.
\end{ex}

\begin{ex}[indiskrete Topologie]
	$\scr T = \{\emptyset, X\}$ auf $X$.
\end{ex}

\begin{ex}[metrische Topologie] \label{ex:metric_topology}
	Sei $d: X \times X \to [0,\infty]$ eine Metrik auf einer Menge $X$.
	Dann ist die \emph{metrische Topologie} auf $(X,d)$ gegeben durch
	\begin{align*}
		\scr T_d
		&:= \big\{ O \subset X : \forall a \in O \exists \eps > 0 : B(a,\eps) \subset O \big\} \\
		&\;= \big\{ O \subset X : \forall a \in O \exists \eps > 0 : d(a,x) < \eps \implies x \in O \big\}.
	\end{align*}
	\begin{note}
		Die offenen Mengen hinsichtlich einer metrischen Topologie $\scr T_d$ auf $X$ entsprechen den offenen Mengen gemäß \ref{df:metric_space_terms} auf dem metrischen Raum $(X,d)$.
	\end{note}
\end{ex}

\begin{df}[metrisierbare Topologie]
	Eine Topologie $\scr T$ heißt \emph{metrisierbar}, wenn es eine Metrik $d$ gibt mit $\scr T = \scr T_d$.
\end{df}

\begin{ex}
	Die diskrete Topologie ist metrisierbar (dank der diskreten Metrik).

	Die indiskrete Topologie hingegen nicht (für $|x| \ge 2$).
	(Jede metrische Topologie trennt je zwei Punkte $a \neq b$, die indiskrete Topologie tut das nicht)
\end{ex}

\begin{df}
	Sind $\scr T_1 \subset \scr T_2 \subset \scr P(X)$ Topologien auf $X$, so nennen wir $\scr T_1$ \emph{gröber} als $\scr T_2$, bzw. $\scr T_2$ \emph{feiner} als $\scr T_1$.
\end{df}

\begin{ex}[Ordnungstopologie]
	Sei $(X,<)$ eine linear geordnete Menge, z.B. $(\R,<)$.
	Wir nennen $O \subset X$ offen, wenn zu jedem $x \in O$ ein offenes Intervall $I \subset X$ existiert mit $x \in I \subset O$.
	Das System $\scr T_<$ dieser offener Mengen heißt \emph{Ordnungstopologie} von $(X,<)$.
\end{ex}

\begin{ex}[Koendliche Topologie auf $X$]
	Setze
	\[
		\scr T
		:= \{ O \subset X : X \setminus O \text{ ist endlich} \} \cup \{\emptyset\}.
	\]
\end{ex}

\begin{ex}[Koabzählbare Topologie auf $X$]
	Setze
	\[
		\scr T
		:= \{ O \subset X : X \setminus O \text{ ist abzählbar} \} \cup \{ \emptyset \}.
	\]
\end{ex}

\begin{nt}
	Es gilt stets
	\[
		\scr T_{\text{indiskret}}
		\subset \scr T_{\text{koendlich}}
		\subset \scr T_{\text{koabzählbar}}
		\subset \scr T_{\text{diskret}}
	\]
	(nach links hin gröber, nach rechts hin feiner).
\end{nt}

\begin{df}
	Eine Abbildung $f: (X,\scr T_X) \to (Y, \scr T_Y)$ heißt \emph{stetig}, wenn aus $V \in \scr T_Y$ stets $f^{-1}(V) \in \scr T_X$ folgt.

	Hingegen heißt $f$ \emph{offen}, wenn aus $U \in \scr T_X$, stets $f(X) \in T_Y$ folgt und \emph{abgeschlossen}, wenn aus $X \setminus A \in \scr T_X$ stets $Y \setminus f(A) \in \scr T_Y$ folgt.
	\begin{note}
		Diese Eigenschaften sind voneinander unabhängig, d.h. man findet Beispiele, die jeweils unterschiedliche Eigenschaften abdecken, bzw. nicht abdecken:
		\begin{itemize}
			\item
				$\Id_\R: \R \to \R$ hat alle Eigenschaften,
			\item
				$f: \R \to \R, f(x) = x^2$ ist stetig, abgeschlossen, aber nichtoffen,
			\item
				$f: \R \to \R, f(x) = \arctan(x)$ ist stetig, offen, aber nicht abgeschlossen,
			\item
				$f: \R \to \R, f(x) = \f 1{1+x^2}$ ist stetig, aber weder offen, noch abgeschlossen,
			\item
				\dots
		\end{itemize}
	\end{note}
\end{df}

\begin{df}
	Eine Homöomorphismus $f: (X,\scr T_X) \to (Y, \scr T_Y)$ ist eine bijektive Abbildung $f: X \to Y$, sodass $f$ und $f^{-1}$ stetig sind.
\end{df}

\begin{nt}
	Aus der Stetigkeit von $f$ folgt im Allgemeinen nicht die Stetigkeit von $f^{-1}$.
	Betrachte dazu folgende Gegenbeispiele
	\begin{itemize}
		\item
			$f: [1,2[ \cup [3,4[ \to [1,3[$ definiert durch
			\[
				f(x) = \begin{cases}
					x & x \in [1,2[ \\
					x - 1 & x \in [3,4[
				\end{cases}.
			\]
			Anschaulich gesprochen „flickt“ $f$ beide Intervalle zusammen, während $f^{-1}$ sie wieder „zerreißt“.
			$f$ ist bijektiv, stetig, aber kein Homöomorphismus.
		\item
			$f : [0,1[ \to \S^1 \subset \C$ definiert durch
			\[
				f(t)
				= e^{2\pi i t}
				= (\cos (2\pi t), \sin (2\pi t))
			\]
			ist bijektiv, stetig, aber kein Homöomorphismus (es gibt sogar kein Homöomorphismus zwischen $[0,1[$ und $\S^1$).
	\end{itemize}
\end{nt}


\section{Umgebungen}


Sei im folgenden $(X, \scr T)$ ein topologischer Raum, $x \in X$.

\begin{df}
	Eine Menge $O$ heißt \emph{offene Umgebung} von $x$ in $(X,\scr T)$, wenn $x \in O \in \scr T$ gilt.
	Die Menge aller offenen Umgebungen von $x$ in $(X, \scr T)$ bezeichnen wir mit
	\[
		U_x^O := U_x^O(\scr T)
		:= \{ O \in \scr T : x \in O \}.
	\]

	Eine Menge $U$ heißt \emph{Umgebung} von $x$ in $(X, \scr T)$, wenn sie eine offene Umgebung von $x$ enthält.
	Die Menge aller Umgebungen von $x$ in $(X, \scr T)$ bezeichnen wir mit
	\[
		U_x := U_x(\scr T)
		:= \{ U \subset X : \exists O \in \scr T : x \in O \subset U \}
	\]
\end{df}

\begin{df}
	Eine Familie $\scr B_x \subset U_x$ heißt \emph{Umgebungsbasis} von $x$ in $(X, \scr T)$, wenn jede Umgebung $U$ von $x$ eine Umgebung $V \in \scr B_A$ enthält.

	Dann ist
	\[
		U_x = \{ U \subset X : \exists V \in \scr B_x : V \subset U \}.
	\]
\end{df}

\begin{ex}
	\begin{itemize}
		\item
			$U_x$ ist eine Umgebungsbasis, ebenso $U_x^O$.
		\item
			In jedem metrischen Raum $(X,d)$ bilden die Bälle $B(a,\eps)$ mit $\eps > 0$ eine Umgebungsbasis von $a$, ebenso $B(a, \f 1k)$ mit $k= 1, 2, \dotsc$,
			diese ist sogar abzählbar.
	\end{itemize}
\end{ex}

\begin{df}[Erstes Abzählbarkeitsaxiom]
	Ein topologischer Raum $(X, \scr T)$ erfüllt das \emph{erste Abzählbarkeitsaxiom} (1AA), wenn jeder Punkt $a \in X$ eine abzählbare Umgebungsbasis besitzt.
	\begin{note}
		Hat $a$ eine abzählbare Umgebungsbasis $(V_n)_{n\in\N}$, so auch eine offene Umgebungsbasis, etwa $(\mathring V_n)_{n\in\N}$.

		Durch Übergang zu $U_n := \mathring V_0 \cap \dotsb \cap \mathring V_n$ erhalten wir eine offene Basis $U_0 \supset U_1 \supset U_2 \supset \dotsb$.
	\end{note}
\end{df}

\begin{ex}
	\begin{itemize}
		\item
			Jeder metrische Raum erfüllt das 1AA.
		\item
			Auf einer überabzählbaren Menge $X$, etwa $X = \R$, erfüllen die koendliche und die koabzählbare Topologie nicht das 1AA.
			Diese können also nicht metrisiert werden.
	\end{itemize}
\end{ex}

\begin{df}
	Eine Folge $(x_n)_{n\in \N}$ in $X$ \emph{konvergiert} gegen $a \in X$ bezüglich $\scr T$, wenn jede Umgebung $U$ von $a$ in $(X, \scr T)$ fast alle Folgenglieder enthält:
	\[
		(x_n)_{n\in \N} \stack{\scr T}\to a
		\iff
		\forall U \in \scr U_a(\scr T) \exists m \in \N \forall n \ge m : x_n \in U
	\]
\end{df}

\begin{prop}
	Eine Folge $(x_n)$ in $X$ konvergiert gegen $a \in X$ genau dann, wenn
	\[
		\forall V \in \scr B_x \exists m \in \N \forall n \ge m : x_n \in V.
	\]
\end{prop}

Grenzwerte sind im Allgemeinen nicht eindeutig, betrachte dazu folgendes Beispiel

\begin{ex}
	Ist $(X, \{\emptyset, X\})$ ein indiskreter Raum, so konvergiert jede Folge gegen jeden beliebigen Punkt $a \in X$.
\end{ex}

\begin{df}
	Ein topologischer Raum $(X, \scr T)$ heißt \emph{hausdorffsch}, wenn zu je zwei Punkten $a \neq b$ in $X$ disjunkte Umgebungen existieren, d.h. $U \in \scr U_a, V \in \scr U_b$, so dass $U \cap V = \emptyset$.
\end{df}

\begin{ex}
	Jeder metrische Raum ist hausdorffsch.
\end{ex}

\begin{st}
	Ist $(X,\scr T)$ hausdorffsch, dann hat jede Folge in $X$ höchstens einen Grenzwert in $X$.

	Die Umkehrung gilt, wenn $(X, \scr T)$ das erste Abzählbarkeitsaxiom erfüllt.
	\begin{proof}
		Aus $x_n \to a$ und $y_n \to b$ folgt $a = b$ (mit $U \in \scr U_a, V \in \scr U_b, U \cap V = \emptyset$ folgt der Widerspruch).
	\end{proof}
\end{st}

\coursetimestamp{29}{10}{2013}

\begin{st}
	Jede Umgebungsbasis $\scr B_x$ von $x$ in $(X, \scr T)$ erfreut sich folgender Eigenschaften
	\begin{enumerate}[(UB1)]
		\item
			$\scr B_x \neq \emptyset$ und für $U \in \scr B_x$ gilt $x \in U \subset X$,
		\item
			Zu $U,V \in \scr B_x$ existiert $W \in \scr B_x$ mit $W \subset U \cap V$.
	\end{enumerate}
	Erfüllt umgekehrt $(\scr B_x)_{x\in X}$ die Bedingungen (UB1-2), dann ist
	\[
		\scr T := \big\{ O \subset X : \text{Zu jedem $x\in O$ existiert ein $U_x \in \scr B_x$ mit $U_x \subset O$} \big\}
	\]
	eine Topologie auf $X$ und
	$\scr B_x$ ist für jedes $x \in X$ eine Umgebungsbasis von $x$ im Raum $(X, \scr T)$.
	\begin{proof}
		$\scr T$ ist eine Topologie:
		\begin{enumerate}[(O1)]
			\item
				folgt aus (UB1).
			\item
				folgt aus (UB2):
				Für $U, V \in \scr T$ ist $W := U \cap V \in \scr T$ zu zeigen.
				Zu $x \in W = U \cap V$ existieren $U_x \in \scr B_x$ mit $U_x \subset U$ und $V_x \in \scr B_x$ mit $V_x \subset V$.
				Dann existieren $W_x \in \scr B_x$ mit $W_x \subset U_x \cap V_x$.
				Also
				\[
					x \in W_x \subset U_x \cap V_x \subset U \cap V = W,
				\]
				also $W \in \scr T$.
			\item
				gilt nach Konstruktion.
		\end{enumerate}
	\end{proof}
\end{st}

\begin{ex}
	\begin{enumerate}[1.)]
		\item
			Metrische Topologien sind vorgegeben durch
			\[
				\scr B_x := \{ B(x,\eps) : \eps > 0 \},
			\]
		\item
			Ordnungstopologien durch
			\[
				\scr B_x := \{ ]a,b[ : a < x < b \}.
			\]
	\end{enumerate}
\end{ex}

\section{Funktionenräume}

Betrachte Funktionenräume, speziell $f : \R \to \R$.
Sei im Folgenden stetz $f_n: \R \to \R$ eine Folge und $f: \R \to \R$.

\subsection{Punktweise Konvergenz}

Wir sagen, $f_n$ konvergiert punktweise gegen $f$, wenn
\[
	\forall x \in \R : f_n(x) \to f(x),
\]
also genau dann, wenn
\[
	\forall x \in \R \forall \eps > 0 \exists m \in \N \forall n \ge m : |f_n(x) - f(x)| < \eps.
\]
Wir definieren Umgebungen als
\[
	U(f; x, \eps)
	:= \{ g: \R \to \R : |g(x) - f(x)| < \eps \}.
\]
Diese erfüllen (UB1), aber nicht (UB2).
Für $J \subset \R$ endlich und $\eps \in \R_{>0}$ definieren wir
\[
	U(f; J, \eps)
	:= \{ g: \R \to \R : \forall x \in J : |g(x) - f(x)| < \eps \}.
\]
Diese erfüllen (UB1) und (UB2), denn
\[
	U(f; J, \eps) \cap U(f, J', \eps')
	\supset U(f; J \cup J', \min\{\eps, \eps'\}).
\]
Dies definiert eine Topologie $\scr T_{\text{pw}}$ wie im Satz. %fixme: reference

Es gilt $f_n \to f$ punktweise genau dann, wenn $f_n \to f$ bezüglich $\scr T_{\text{pw}}$.

\begin{st}
	Die Topologie $\scr T_{\text{pw}}$ ist nicht metrisierbar.
	Genauer: $\scr T_{\text{pw}}$ erfüllt das Hausdorff-Axiom, aber nicht das 1AA.
	\begin{proof}
		Angenommen $(U_n)_{n\in\N}$ sind Umgebungen der Nullfunktion.
		Zu jedem $n \in \N$ existiert $J_n \subset \R$ endlich und $\eps_n > 0$ mit $U(0; J_n, \eps_n) \subset U_n$.

		Die Menge $J = \bigcup_{k\in\N} J_n \subset \R$ ist abzählbar.
		Für $x_0 \in \R \setminus J$ enthält die Umgebung $U(0; x_0, 1)$ keine der Umgebungen $U_n$.
	\end{proof}
\end{st}

\subsection{Gleichmäßige Konvergenz} % fixme: \section?

Wir sagen, $f_n \to f$ konvergiere gleichmäßig, wenn
\[
	\forall \eps > 0 \exists m \in \N \forall n \ge m \underbrace{\forall x \in \R : |f_n(x) - f(x) | \le \eps}_{|f_n - f|_{\R} \le \eps},
\]
also genau dann, wenn
\[
	|f_n-f|_{\R} \to 0.
\]
Damit wird die gleichmäßige Konvergenz von der Supremumsnorm kontrolliert.
Setze
\[
	U(f,\eps)
	:= B(f,\eps)
	= \{ g : \R \to \R : \forall x \in \R : |g(x) - f(x)| < \eps \}.
\]
Dies erfüllt (UB1-2) und definiert so die (metrische) Topologie $\scr T_{\text{uni}} = \scr T_d$ bezüglich dier Supremumsmetrik
\[
	d(f,g)
	:= |f-g|_{\R}
	:= \sup_{x\in\R} |f(x) - g(x)|.
\]
Es ergibt sich allerdings ein Problem:
Potenzreihen wie $\exp(x) = \sum_{k=0}^\infty \f {x^k}{k!}$ oder $\sin(x) = \sum_{k=0}^\infty (-1)^k \f {x^{2k+1}}{(2k+1)!}$ oder $\cos (x) = \sum_{k=0}^\infty (-1)^k \f {x^{2k}}{(2k)!}$ konvergieren punktweise für jedes $x \in \R$, aber nicht gleichmäßig auf $\R$.

\subsection{Kompakte Konvergenz}

Wir sagen, $f_n \to f$ konvergiert \emph{kompakt} genau dann, wenn $f_n|_{[-r,r]} \to f|_{[-r,r]}$ gleichmäßig auf jedem kompakten Intervall $[-r,r]$ konvergiert, also genau dann, wenn
\[
	\forall r \in \R_{>0} \forall \eps \in \R_{>0} \exists m \in \N \forall n \ge n \underbrace{\forall x \in [-r,r] : |f_n(x) - f(x)| \le \eps}_{|f_n-f|_{[-r,r]\le \eps}}.
\]
\begin{ex}
	Die kompakte Konvergenz ist der richtige Konvergenzbegriff für Potenzreihen.
\end{ex}
Für $K = [-r,r]$ und $\eps > 0$ setzen wir
\[
	U(f; K, \eps)
	:= \{ g: \R \to \R : \forall x \in K : |f(x) - g(x)| < \eps \}.
\]
Dies erfüllt (UB1-2), denn
\[
	U(f;K,\eps) \cap U(f,K',\eps')
	\supset U(f; K \cup K', \min\{\eps, \eps'\}),
\]
und erzeugt damit die Topologie $\scr T_{\text{kpkt}}$.

\begin{st}
	Es gilt $f_n \to f$ kompakt genau dann, wenn $f_n \to f$ bezüglich $\scr T_{\text{kpkt}}$.
\end{st}

\begin{nt}
	Es gilt
	\[
		\scr T_{\text{pw}} \subsetneq \scr T_{\text{kpkt}} \subsetneq \scr T_{\text{uni}}.
	\]
\end{nt}

\begin{st}
	$\scr T_{\text{kpkt}}$ ist metrisierbar.
	\begin{proof}
		Idee: für $k \in \N$ betrachte $d_k(f,g) := |f-g|_{[-k,k]}$ und
		\[
			d(f,g) = \sum_{k=1}^\infty 2^{-k} d_k^* (f,g) \in [0,1],
		\]
		wobei $d_k^*$ die auf $[0,1]$ gestauchte Metrik ist.
		$d(f,g)$ ist ein Metrik auf $\R^\R$ (Übung) und induziert $\scr T_{\text{kpkt}}$.
	\end{proof}
\end{st}


% C3
\section{Inneres, Abschluss, Rand}


Sei $(X, \scr T)$ ein topologischer Raum und $M \subset X$.
Ein Punkt $x \in X$ heißt
\begin{itemize}
	\item
		\emph{innerer Punkt} von $M$ in $(X,\scr T)$, wenn
		\[
			M \in \scr U_x.
		\]
	\item
		\emph{äußerer Punkt} von $M$ in $(X,\scr T)$, wenn
		\[
			(X \setminus M) \in \scr U_x.
		\]
	\item
		\emph{Berührungspunkt} von $M$ in $(X, \scr T)$, wenn
		\[
			\forall U \in \scr U_x : U \cap M \neq \emptyset.
		\]
	\item
		\emph{Randpunkt} von $M$ in $(X, \scr T)$, wenn
		\[
			\forall U \in \scr U_x : U \cap M \neq \emptyset \neq U \cap (X \setminus M).
		\]
	\item
		\emph{Häufungspunkt} von $M$ in $(X, \scr T)$, wenn
		\[
			\forall U \in \scr U_x : (U \setminus \{x\}) \cap M \neq \emptyset.
		\]
	\item
		\emph{isolierer Punkt} von $M$ in $(X, \scr T)$, wenn
		\[
			\exists U \in \scr U_x : U \cap M = \{x\}.
		\]
\end{itemize}

\begin{df}
	Wir definieren das \emph{Innere} einer Menge $M \subset X$ als
	\begin{align*}
		\mathring M
		:= K_{\scr T}(M)
		&:= \{ x \in X : M \in U_x \} \\
		&\;= \bigcup \{ U \in \scr T : U \subset M \},
	\end{align*}
	den \emph{Abschluss} als
	\begin{align*}
		\_{M}
		:= H_{\scr T}(M)
		&:= \{ x \in X : \forall U \in \scr U_x : U \cap M \neq \emptyset \} \\
		&\;=  \bigcap \{ A \supset M : X \setminus A \in \scr T \}
	\end{align*}
	und den \emph{Rand} als
	\[
		\boundary M
		:= \_ M \setminus \mathring M,
	\]
	bzw. $\delta_{\scr T}(M) := H_{\scr T}(M) \setminus K_{\scr T}(M)$.
\end{df}

\begin{st}
	Sei $(X, \scr T)$ ein topologischer Raum, $A \subset X$ und $x \in X$.
	Wenn es eine Folge $(a_n)_{n\in\N}$ aus $A$ gibt mit $x_n \to x$ in $(X, \scr T)$, dann gilt $x \in \_A$.

	Die Umkehrung gilt, wenn das 1AA erfüllt, d.h. $x$ eine abzählbare Umgebungsbasis hat.
\end{st}

\begin{df}
	Eine Menge $M \subset X$ heißt \emph{dicht} in $(X, \scr T)$ wenn $\_M = X$ und \emph{diskret}, wenn jeder Punkt $x \in M$ isoliert ist.
\end{df}

\begin{ex}
	$\Q \subset \R$ ist dicht, $\Z \subset \R$ ist diskret.
\end{ex}


% C4
\section{Erzeugung von Topologien}


\begin{df}[Basis]
	Eine Teilmenge $\scr B \subset \scr T$ heißt \emph{Basis} der Topologie $\scr T$, wenn jede offene Menge $U \in \scr T$ Vereinigung von offenen Mengen aus $\scr B$ ist, also
	\[
		\scr T = \Big\{ \bigcup \scr S : \scr S \subset \scr B \Big\}.
	\]
\end{df}

\begin{ex}
	\begin{enumerate}[1)]
		\item
			Jede Topologie $\scr T$ hat eine Basis, z.B. $\scr B = \scr T$.
		\item
			Die übliche Topologie auf $\R$ hat als Basis $\scr B = \{ ]a,b[ : a < b \}$.
		\item
			Jeder metrische Raum $(X, \scr T_d)$ hat als Basis $\scr B = \{ B(a, \eps) : a \in X, \eps > 0\}$.
	\end{enumerate}
\end{ex}

