%fixme: \scr U vs U und \scr T vs \tau

\chapter{Topologische Räume}

\coursetimestamp{28}{10}{2013}

\begin{df}
	Eine \emph{Topologie} auf einer Menge $X$ ist ein System $\scr T \subset \scr P(X)$ von Teilmengen von $X$ mit folgenden Eigenschaften
	\begin{enumerate}[O1)]
		\item
			$\emptyset, X \in \scr T$,
		\item
			$O_1, \dotsc, O_n \in \scr T \implies O_1 \cap \dotsb \cap O_n \in \scr T$,
		\item
			$O_i \in \scr T$ für $i \in I \implies \bigcup_{i\in I} O_i \in \scr T$.
	\end{enumerate}
	Das Paar $(X, \scr T)$ nennen wir dann \emph{topologischer Raum}.

	Eine Teilmenge $Y \subset X$ nennen wir \emph{offen}, wenn $Y \in \scr T$, und \emph{abgeschlossen}, wenn $X \setminus Y \in \scr T$.
\end{df}

\begin{ex}[Kleinste Beispiele]
	\begin{itemize}
		\item
			$X = \emptyset, \scr T = \{\emptyset\}$,
		\item
			$X = \{a\}, \scr T = \{ \emptyset, X \}$,
		\item
			Für $X = \{a,b\}$ gibt es vier Topologien:
			\begin{align*}
				&\{\emptyset, X\}, &
				&\{\emptyset, \{a\}, X\}, &
				&\{\emptyset, \{b\}, X \}, &
				&\{\emptyset, \{a\}, \{b\}, X\}.
			\end{align*}
		\item
			Die Anzahl möglicher Topologien auf einer Menge mit $n$ Elementen wird in \ref{tab:topcount} dargestellt.
	\end{itemize}
	\begin{table}[h]
		\centering
		\begin{tabular}{c|c}
			$n=|x|$ & $t_n = $ Anzahl der Topologie auf $X$ \\ \hline
			0 & 1 \\
			1 & 1 \\
			2 & 4 \\
			3 & 29 \\
			4 & 355 \\
			5 & 6942
		\end{tabular}
		\caption{Anzahl der Topologien auf einer Menge $X$ mit $n$ Elementen, asymptotisches Verhalten: $\log t_n \sim \f {n^2}4$}
		\label{tab:topcount}
	\end{table}
\end{ex}

\begin{ex}[diskrete Topologie]
	$\scr T = \scr P(X)$ auf $X$.
\end{ex}

\begin{ex}[indiskrete Topologie]
	$\scr T = \{\emptyset, X\}$ auf $X$.
\end{ex}

\begin{ex}[metrische Topologie] \label{ex:metric_topology}
	Sei $d: X \times X \to [0,\infty]$ eine Metrik auf einer Menge $X$.
	Dann ist die \emph{metrische Topologie} auf $(X,d)$ gegeben durch
	\begin{align*}
		\scr T_d
		&:= \big\{ O \subset X : \forall a \in O \exists \eps > 0 : B(a,\eps) \subset O \big\} \\
		&\;= \big\{ O \subset X : \forall a \in O \exists \eps > 0 : d(a,x) < \eps \implies x \in O \big\}.
	\end{align*}
	\begin{note}
		Die offenen Mengen hinsichtlich einer metrischen Topologie $\scr T_d$ auf $X$ entsprechen den offenen Mengen gemäß \ref{df:metric_space_terms} auf dem metrischen Raum $(X,d)$.
	\end{note}
\end{ex}

\begin{df}[metrisierbare Topologie]
	Eine Topologie $\scr T$ heißt \emph{metrisierbar}, wenn es eine Metrik $d$ gibt mit $\scr T = \scr T_d$.
\end{df}

\begin{ex}
	Die diskrete Topologie ist metrisierbar (dank der diskreten Metrik).

	Die indiskrete Topologie hingegen nicht (für $|x| \ge 2$).
	(Jede metrische Topologie trennt je zwei Punkte $a \neq b$, die indiskrete Topologie tut das nicht)
\end{ex}

\begin{df}
	Sind $\scr T_1 \subset \scr T_2 \subset \scr P(X)$ Topologien auf $X$, so nennen wir $\scr T_1$ \emph{gröber} als $\scr T_2$, bzw. $\scr T_2$ \emph{feiner} als $\scr T_1$.
\end{df}

\begin{ex}[Ordnungstopologie]
	Sei $(X,<)$ eine linear geordnete Menge, z.B. $(\R,<)$.
	Wir nennen $O \subset X$ offen, wenn zu jedem $x \in O$ ein offenes Intervall $I \subset X$ existiert mit $x \in I \subset O$.
	Das System $\scr T_<$ dieser offener Mengen heißt \emph{Ordnungstopologie} von $(X,<)$.
\end{ex}

\begin{ex}[Koendliche Topologie auf $X$]
	Setze
	\[
		\scr T
		:= \{ O \subset X : X \setminus O \text{ ist endlich} \} \cup \{\emptyset\}.
	\]
\end{ex}

\begin{ex}[Koabzählbare Topologie auf $X$]
	Setze
	\[
		\scr T
		:= \{ O \subset X : X \setminus O \text{ ist abzählbar} \} \cup \{ \emptyset \}.
	\]
\end{ex}

\begin{nt}
	Es gilt stets
	\[
		\scr T_{\text{indiskret}}
		\subset \scr T_{\text{koendlich}}
		\subset \scr T_{\text{koabzählbar}}
		\subset \scr T_{\text{diskret}}
	\]
	(nach links hin gröber, nach rechts hin feiner).
\end{nt}

\begin{df}
	Eine Abbildung $f: (X,\scr T_X) \to (Y, \scr T_Y)$ heißt \emph{stetig}, wenn aus $V \in \scr T_Y$ stets $f^{-1}(V) \in \scr T_X$ folgt.

	Hingegen heißt $f$ \emph{offen}, wenn aus $U \in \scr T_X$, stets $f(X) \in T_Y$ folgt und \emph{abgeschlossen}, wenn aus $X \setminus A \in \scr T_X$ stets $Y \setminus f(A) \in \scr T_Y$ folgt.
	\begin{note}
		Diese Eigenschaften sind voneinander unabhängig, d.h. man findet Beispiele, die jeweils unterschiedliche Eigenschaften abdecken, bzw. nicht abdecken:
		\begin{itemize}
			\item
				$\Id_\R: \R \to \R$ hat alle Eigenschaften,
			\item
				$f: \R \to \R, f(x) = x^2$ ist stetig, abgeschlossen, aber nichtoffen,
			\item
				$f: \R \to \R, f(x) = \arctan(x)$ ist stetig, offen, aber nicht abgeschlossen,
			\item
				$f: \R \to \R, f(x) = \f 1{1+x^2}$ ist stetig, aber weder offen, noch abgeschlossen,
			\item
				\dots
		\end{itemize}
	\end{note}
\end{df}

\begin{df}
	Eine Homöomorphismus $f: (X,\scr T_X) \to (Y, \scr T_Y)$ ist eine bijektive Abbildung $f: X \to Y$, sodass $f$ und $f^{-1}$ stetig sind.
\end{df}

\begin{nt}
	Aus der Stetigkeit von $f$ folgt im Allgemeinen nicht die Stetigkeit von $f^{-1}$.
	Betrachte dazu folgende Gegenbeispiele
	\begin{itemize}
		\item
			$f: [1,2[ \cup [3,4[ \to [1,3[$ definiert durch
			\[
				f(x) = \begin{cases}
					x & x \in [1,2[ \\
					x - 1 & x \in [3,4[
				\end{cases}.
			\]
			Anschaulich gesprochen „flickt“ $f$ beide Intervalle zusammen, während $f^{-1}$ sie wieder „zerreißt“.
			$f$ ist bijektiv, stetig, aber kein Homöomorphismus.
		\item
			$f : [0,1[ \to \S^1 \subset \C$ definiert durch
			\[
				f(t)
				= e^{2\pi i t}
				= (\cos (2\pi t), \sin (2\pi t))
			\]
			ist bijektiv, stetig, aber kein Homöomorphismus (es gibt sogar kein Homöomorphismus zwischen $[0,1[$ und $\S^1$).
	\end{itemize}
\end{nt}


\section{Umgebungen}


Sei im folgenden $(X, \scr T)$ ein topologischer Raum, $x \in X$.

\begin{df}
	Eine Menge $O$ heißt \emph{offene Umgebung} von $x$ in $(X,\scr T)$, wenn $x \in O \in \scr T$ gilt.
	Die Menge aller offenen Umgebungen von $x$ in $(X, \scr T)$ bezeichnen wir mit
	\[
		\scr U_x^O := \scr U_x^O(\scr T)
		:= \{ O \in \scr T : x \in O \}.
	\]

	Eine Menge $U$ heißt \emph{Umgebung} von $x$ in $(X, \scr T)$, wenn sie eine offene Umgebung von $x$ enthält.
	Die Menge aller Umgebungen von $x$ in $(X, \scr T)$ bezeichnen wir mit
	\[
		\scr U_x := \scr U_x(\scr T)
		:= \big\{ U \subset X : \exists O \in \scr T : x \in O \subset U \big\}
	\]
\end{df}

\begin{df}
	Eine Familie $\scr B_x \subset \scr U_x$ heißt \emph{Umgebungsbasis} von $x$ in $(X, \scr T)$, wenn jede Umgebung $U$ von $x$ eine Umgebung $V \in \scr B_x$ enthält.

	$\scr U_x$ lässt sich dann darstellen durch
	\[
		\scr U_x = \big\{ U \subset X : \exists V \in \scr B_x : V \subset U \big\}.
	\]
\end{df}

\begin{ex}
	\begin{itemize}
		\item
			$\scr U_x$ ist eine Umgebungsbasis, ebenso $\scr U_x^O$.
		\item
			In jedem metrischen Raum $(X,d)$ bilden die Bälle $B(a,\eps)$ mit $\eps > 0$ eine Umgebungsbasis von $a$, ebenso $B(a, \f 1k)$ mit $k \in \N_{\ge 1}$ (diese ist sogar abzählbar).
	\end{itemize}
\end{ex}

\begin{df}[Erstes Abzählbarkeitsaxiom] \label{df:first_axiom_of_countability}
	Ein topologischer Raum $(X, \scr T)$ erfüllt das \emph{erste Abzählbarkeitsaxiom} (1AA), wenn jeder Punkt $a \in X$ eine abzählbare Umgebungsbasis besitzt.
	\begin{note}
		Hat $a$ eine abzählbare Umgebungsbasis $(V_n)_{n\in\N}$, so hat $a$ auch eine offene Umgebungsbasis, etwa $(\mathring V_n)_{n\in\N}$.
		Durch Übergang zu $U_n := \mathring V_0 \cap \dotsb \cap \mathring V_n$ erhalten wir sogar eine offene Basis der Form $U_0 \supset U_1 \supset U_2 \supset \dotsb$.
	\end{note}
\end{df}

\begin{ex}
	\begin{enumerate}[1.]
		\item
			Jeder metrische Raum erfüllt das erste Abzählbarkeitsaxiom.
		\item
			Auf einer überabzählbaren Menge $X$, etwa $X = \R$, erfüllen die koendliche und die koabzählbare Topologie das erste Abzählbarkeitsaxiom nicht.
			Wegen des vorigen Beispiels sind diese also nicht metrisierbar.
	\end{enumerate}
\end{ex}

\begin{df}[Konvergenz]
	Eine Folge $(x_n)_{n\in \N}$ in $X$ \emph{konvergiert} gegen $a \in X$ bezüglich $\scr T$, wenn jede Umgebung $U$ von $a$ in $(X, \scr T)$ fast alle Folgenglieder enthält, d.h.
	\[
		\forall U \in \scr U_a(\scr T) \exists m \in \N \forall n \ge m : x_n \in U.
	\]
	Wir schreiben dann
	\[
		(x_n)_{n\in \N} \xrightarrow{\scr T} a.
	\]
\end{df}

\begin{prop}
	Eine Folge $(x_n)$ in $X$ konvergiert gegen $a \in X$ genau dann, wenn
	\[
		\forall V \in \scr B_x \exists m \in \N \forall n \ge m : x_n \in V.
	\]
	% fixme: proof
\end{prop}

Grenzwerte sind im Allgemeinen nicht eindeutig, betrachte dazu folgendes Beispiel

\begin{ex}
	Ist $(X, \{\emptyset, X\})$ ein indiskreter Raum, so konvergiert jede Folge gegen jeden beliebigen Punkt $a \in X$.
\end{ex}

\begin{df}
	Ein topologischer Raum $(X, \scr T)$ heißt \emph{hausdorffsch}, wenn zu je zwei Punkten $a \neq b$ in $X$ disjunkte Umgebungen existieren, d.h. $U \in \scr U_a, V \in \scr U_b$, so dass $U \cap V = \emptyset$.
\end{df}

\begin{ex}
	Jeder metrische Raum ist hausdorffsch.
\end{ex}

\begin{st}
	Ist $(X,\scr T)$ hausdorffsch, dann hat jede Folge in $X$ höchstens einen Grenzwert in $X$.

	Die Umkehrung gilt, wenn $(X, \scr T)$ das erste Abzählbarkeitsaxiom erfüllt.
	\begin{proof}
		Zeige: aus $x_n \to a$ und $x_n \to b$ folgt $a = b$.
		Es sei $a \neq b$, dann existieren (da $(X, \scr T)$ hausdorffsch) $U \in \scr U_a, V \in \scr U_b$ mit $U \cap V = \emptyset$, ein Widerspruch zu $x_n \to a, x_n \to b$.

		Zeige nun die Umkehrung unter der Voraussetzung des ersten Abzählbarkeitsaxioms.
		Sei dazu $a, b \in X$ mit $a \neq b$ und seien $\scr B_a$ bestehend aus $U_0 \supset U_1 \supset \dotsb$ und $\scr B_b$ bestehend aus $V_0 \supset V_1 \supset \dotsb$ zwei absteigende, abzählbare Umgebungsbasen (siehe Bemerkung in \ref{df:first_axiom_of_countability}) von $a$, bzw. $b$.
		Wenn $i,j \in \N$ existieren mit $U_i \cap V_j = \emptyset$, so sind wir fertig.
		Angenommen jedoch, für alle $i,j \in \N$ ist $U_i \cap V_i \neq \emptyset$.
		Definiere dann die Folge $(z_i)_{i\in\N}$ durch
		\[
			z_i \in U_i \cap V_i \neq \emptyset.
		\]
		Für jedes $U \in \scr B_a$ existiert damit $m \in \N$, sodass $z_n \in U$ für $n \ge m$ und analog für $\scr B_b$.
		Also konvergiert $z_i \to a$ und $z_i \to b$.
		Da jede Folge nach Voraussetzung höchstens einen Grenzwert in $X$ besitzt, gilt $a = b$, ein Widerspruch.
		Somit ist die Aussage gezeigt.
	\end{proof}
\end{st}

\coursetimestamp{29}{10}{2013}

\begin{st}
	Jede Umgebungsbasis $\scr B_x$ von $x$ in $(X, \scr T)$ erfreut sich folgender Eigenschaften
	\begin{enumerate}[(UB1)]
		\item
			$\scr B_x \neq \emptyset$ und für $U \in \scr B_x$ gilt $x \in U \subset X$,
		\item
			Zu $U,V \in \scr B_x$ existiert $W \in \scr B_x$ mit $W \subset U \cap V$.
	\end{enumerate}
	Erfüllt umgekehrt $(\scr B_x)_{x\in X}$ die Bedingungen (UB1-2), dann ist
	\[
		\scr T := \big\{ O \subset X : \text{Zu jedem $x\in O$ existiert ein $U_x \in \scr B_x$ mit $U_x \subset O$} \big\}
	\]
	eine Topologie auf $X$ und
	$\scr B_x$ ist für jedes $x \in X$ eine Umgebungsbasis von $x$ im Raum $(X, \scr T)$.
	\begin{proof}
		$\scr T$ ist eine Topologie:
		\begin{enumerate}[(O1)]
			\item
				folgt aus (UB1).
			\item
				folgt aus (UB2):
				Für $U, V \in \scr T$ ist $W := U \cap V \in \scr T$ zu zeigen.
				Zu $x \in W = U \cap V$ existieren $U_x \in \scr B_x$ mit $U_x \subset U$ und $V_x \in \scr B_x$ mit $V_x \subset V$.
				Dann existieren $W_x \in \scr B_x$ mit $W_x \subset U_x \cap V_x$.
				Also
				\[
					x \in W_x \subset U_x \cap V_x \subset U \cap V = W,
				\]
				also $W \in \scr T$.
			\item
				gilt nach Konstruktion.
		\end{enumerate}
	\end{proof}
\end{st}

\begin{ex}
	\begin{enumerate}[1.)]
		\item
			Metrische Topologien sind vorgegeben durch
			\[
				\scr B_x := \{ B(x,\eps) : \eps > 0 \},
			\]
		\item
			Ordnungstopologien durch
			\[
				\scr B_x := \{ ]a,b[ : a < x < b \}.
			\]
	\end{enumerate}
\end{ex}

\section{Funktionenräume}

Wir betrachten jetzt Funktionenräume, speziell den Raum aller Funktionen $f: \R \to \R$.
Sei dazu im Folgenden stets $f_n: \R \to \R$ eine Funktionenfolge und $f: \R \to \R$ eine Funktion.

\subsection{Punktweise Konvergenz}

Bekanntermaßen konvergiert $f_n$ punktweise gegen $f$ genau dann, wenn
\[
	\forall x \in \R : f_n(x) \to f(x),
\]
oder genauer: wenn
\[
	\forall x \in \R \forall \eps > 0 \exists m \in \N \forall n \ge m : |f_n(x) - f(x)| < \eps.
\]
Ein naiver Ansatz, Umgebungen:call foreground()

Wir definieren Umgebungen als
\[
	U(f; x, \eps)
	:= \{ g: \R \to \R : |g(x) - f(x)| < \eps \}.
\]
Diese erfüllen (UB1), aber nicht (UB2).
Für $J \subset \R$ endlich und $\eps \in \R_{>0}$ definieren wir
\[
	U(f; J, \eps)
	:= \{ g: \R \to \R : \forall x \in J : |g(x) - f(x)| < \eps \}.
\]
Diese erfüllen (UB1) und (UB2), denn
\[
	U(f; J, \eps) \cap U(f, J', \eps')
	\supset U(f; J \cup J', \min\{\eps, \eps'\}).
\]
Dies definiert eine Topologie $\scr T_{\text{pw}}$ wie im Satz. %fixme: reference

Es gilt $f_n \to f$ punktweise genau dann, wenn $f_n \to f$ bezüglich $\scr T_{\text{pw}}$.

\begin{st}
	Die Topologie $\scr T_{\text{pw}}$ ist nicht metrisierbar.
	Genauer: $\scr T_{\text{pw}}$ erfüllt das Hausdorff-Axiom, aber nicht das 1AA.
	\begin{proof}
		Angenommen $(U_n)_{n\in\N}$ sind Umgebungen der Nullfunktion.
		Zu jedem $n \in \N$ existiert $J_n \subset \R$ endlich und $\eps_n > 0$ mit $U(0; J_n, \eps_n) \subset U_n$.

		Die Menge $J = \bigcup_{k\in\N} J_n \subset \R$ ist abzählbar.
		Für $x_0 \in \R \setminus J$ enthält die Umgebung $U(0; x_0, 1)$ keine der Umgebungen $U_n$.
	\end{proof}
\end{st}

\subsection{Gleichmäßige Konvergenz} % fixme: \section?

Wir sagen, $f_n \to f$ konvergiere gleichmäßig, wenn
\[
	\forall \eps > 0 \exists m \in \N \forall n \ge m \underbrace{\forall x \in \R : |f_n(x) - f(x) | \le \eps}_{|f_n - f|_{\R} \le \eps},
\]
also genau dann, wenn
\[
	|f_n-f|_{\R} \to 0.
\]
Damit wird die gleichmäßige Konvergenz von der Supremumsnorm kontrolliert.
Setze
\[
	U(f,\eps)
	:= B(f,\eps)
	= \{ g : \R \to \R : \forall x \in \R : |g(x) - f(x)| < \eps \}.
\]
Dies erfüllt (UB1-2) und definiert so die (metrische) Topologie $\scr T_{\text{uni}} = \scr T_d$ bezüglich dier Supremumsmetrik
\[
	d(f,g)
	:= |f-g|_{\R}
	:= \sup_{x\in\R} |f(x) - g(x)|.
\]
Es ergibt sich allerdings ein Problem:
Potenzreihen wie $\exp(x) = \sum_{k=0}^\infty \f {x^k}{k!}$ oder $\sin(x) = \sum_{k=0}^\infty (-1)^k \f {x^{2k+1}}{(2k+1)!}$ oder $\cos (x) = \sum_{k=0}^\infty (-1)^k \f {x^{2k}}{(2k)!}$ konvergieren punktweise für jedes $x \in \R$, aber nicht gleichmäßig auf $\R$.

\subsection{Kompakte Konvergenz}

Wir sagen, $f_n \to f$ konvergiert \emph{kompakt} genau dann, wenn $f_n|_{[-r,r]} \to f|_{[-r,r]}$ gleichmäßig auf jedem kompakten Intervall $[-r,r]$ konvergiert, also genau dann, wenn
\[
	\forall r \in \R_{>0} \forall \eps \in \R_{>0} \exists m \in \N \forall n \ge n \underbrace{\forall x \in [-r,r] : |f_n(x) - f(x)| \le \eps}_{|f_n-f|_{[-r,r]\le \eps}}.
\]
\begin{ex}
	Die kompakte Konvergenz ist der richtige Konvergenzbegriff für Potenzreihen.
\end{ex}
Für $K = [-r,r]$ und $\eps > 0$ setzen wir
\[
	U(f; K, \eps)
	:= \{ g: \R \to \R : \forall x \in K : |f(x) - g(x)| < \eps \}.
\]
Dies erfüllt (UB1-2), denn
\[
	U(f;K,\eps) \cap U(f,K',\eps')
	\supset U(f; K \cup K', \min\{\eps, \eps'\}),
\]
und erzeugt damit die Topologie $\scr T_{\text{kpkt}}$.

\begin{st}
	Es gilt $f_n \to f$ kompakt genau dann, wenn $f_n \to f$ bezüglich $\scr T_{\text{kpkt}}$.
\end{st}

\begin{nt}
	Es gilt
	\[
		\scr T_{\text{pw}} \subsetneq \scr T_{\text{kpkt}} \subsetneq \scr T_{\text{uni}}.
	\]
\end{nt}

\begin{st}
	$\scr T_{\text{kpkt}}$ ist metrisierbar.
	\begin{proof}
		Idee: für $k \in \N$ betrachte $d_k(f,g) := |f-g|_{[-k,k]}$ und
		\[
			d(f,g) = \sum_{k=1}^\infty 2^{-k} d_k^* (f,g) \in [0,1],
		\]
		wobei $d_k^*$ die auf $[0,1]$ gestauchte Metrik ist.
		$d(f,g)$ ist ein Metrik auf $\R^\R$ (Übung) und induziert $\scr T_{\text{kpkt}}$.
	\end{proof}
\end{st}


% C3
\section{Inneres, Abschluss, Rand}


Sei $(X, \scr T)$ ein topologischer Raum und $M \subset X$.
Ein Punkt $x \in X$ heißt
\begin{itemize}
	\item
		\emph{innerer Punkt} von $M$ in $(X,\scr T)$, wenn
		\[
			M \in \scr U_x.
		\]
	\item
		\emph{äußerer Punkt} von $M$ in $(X,\scr T)$, wenn
		\[
			(X \setminus M) \in \scr U_x.
		\]
	\item
		\emph{Berührungspunkt} von $M$ in $(X, \scr T)$, wenn
		\[
			\forall U \in \scr U_x : U \cap M \neq \emptyset.
		\]
	\item
		\emph{Randpunkt} von $M$ in $(X, \scr T)$, wenn
		\[
			\forall U \in \scr U_x : U \cap M \neq \emptyset \neq U \cap (X \setminus M).
		\]
	\item
		\emph{Häufungspunkt} von $M$ in $(X, \scr T)$, wenn
		\[
			\forall U \in \scr U_x : (U \setminus \{x\}) \cap M \neq \emptyset.
		\]
	\item
		\emph{isolierer Punkt} von $M$ in $(X, \scr T)$, wenn
		\[
			\exists U \in \scr U_x : U \cap M = \{x\}.
		\]
\end{itemize}

\begin{df}
	Wir definieren das \emph{Innere} einer Menge $M \subset X$ als
	\begin{align*}
		\mathring M
		:= K_{\scr T}(M)
		&:= \{ x \in X : M \in U_x \} \\
		&\;= \bigcup \{ U \in \scr T : U \subset M \},
	\end{align*}
	den \emph{Abschluss} als
	\begin{align*}
		\_{M}
		:= H_{\scr T}(M)
		&:= \{ x \in X : \forall U \in \scr U_x : U \cap M \neq \emptyset \} \\
		&\;=  \bigcap \{ A \supset M : X \setminus A \in \scr T \}
	\end{align*}
	und den \emph{Rand} als
	\[
		\boundary M
		:= \_ M \setminus \mathring M,
	\]
	bzw. $\delta_{\scr T}(M) := H_{\scr T}(M) \setminus K_{\scr T}(M)$.
\end{df}

\begin{st}
	Sei $(X, \scr T)$ ein topologischer Raum, $A \subset X$ und $x \in X$.
	Wenn es eine Folge $(a_n)_{n\in\N}$ aus $A$ gibt mit $x_n \to x$ in $(X, \scr T)$, dann gilt $x \in \_A$.

	Die Umkehrung gilt, wenn das 1AA erfüllt, d.h. $x$ eine abzählbare Umgebungsbasis hat.
\end{st}

\begin{df}
	Eine Menge $M \subset X$ heißt \emph{dicht} in $(X, \scr T)$ wenn $\_M = X$ und \emph{diskret}, wenn jeder Punkt $x \in M$ isoliert ist.
\end{df}

\begin{ex}
	$\Q \subset \R$ ist dicht, $\Z \subset \R$ ist diskret.
\end{ex}


% C4
\section{Erzeugung von Topologien}


\begin{df}[Basis]
	Eine Teilmenge $\scr B \subset \scr T$ heißt \emph{Basis} der Topologie $\scr T$, wenn jede offene Menge $U \in \scr T$ Vereinigung von offenen Mengen aus $\scr B$ ist, also
	\[
		\scr T = \Big\{ \bigcup \scr S : \scr S \subset \scr B \Big\}.
	\]
\end{df}

\begin{ex}
	\begin{enumerate}[1)]
		\item
			Jede Topologie $\scr T$ hat eine Basis, z.B. $\scr B = \scr T$.
		\item
			Die übliche Topologie auf $\R$ hat als Basis $\scr B = \{ ]a,b[ : a < b \}$.
		\item
			Jeder metrische Raum $(X, \scr T_d)$ hat als Basis $\scr B = \{ B(a, \eps) : a \in X, \eps > 0 \}$.
	\end{enumerate}
\end{ex}

\coursetimestamp{04}{11}{2013}

\begin{prop}
	Sei $(X, \scr T)$ ein topologischer Raum und $\scr B \subset \scr T$.
	Dann sind folgende Aussagen äquivalent
	\begin{enumerate}[1)]
		\item
			$\scr B$ ist Basis von $\scr T$, d.h.
			\[
				\forall U \in \scr T \exists \scr S \subset \scr B : U = \bigcup \scr S.
			\]
		\item
			\[
				\forall U \in \scr T : U = \bigcup \{ B \in \scr B : B \subset U \}.
			\]
		\item
			\[
				\forall U \in \scr T \forall x \in U \exists B \in \scr B : x \in B \subset U
			\]
	\end{enumerate}
	Demnach ist $\scr B$ genau dann eine Basis von $\scr T$, wenn $\scr B_x := \{ B \in \scr B : x \in B \}$ eine Umgebungsbasis für jedes $x \in X$ ist.
	% fixme: proof
\end{prop}

\begin{st}
	Jede Basis $\scr B$ einer Topologie $\scr T$ auf $X$ erfreut sich folgender Eigenschaften
	\begin{enumerate}[(B1)]
		\item
			$X = \bigcup B$,
		\item
			Für $U, V \in \scr B$ existiert $\scr S \subset \scr B$ mit
			\[
				U \cap V = \bigcup S.
			\]
	\end{enumerate}
	Erfüllt umgekehrt $\scr B \subset \scr P(X)$ die Bedingungen (B1-2), so ist
	\[
		\scr T = \Big\{ \bigcup \scr S : \scr S \subset \scr B \Big\}
	\]
	eine Topologie auf $X$ mit Basis $\scr B$.
	% fixme: proof
\end{st}

\begin{st}
	Sei $\scr S \subset \scr P(X)$.
	Setze
	\begin{align*}
		\scr B &:= \big\{ S_1 \cap \dotsc \cap S_n : n \in \N, S_1, \dotsc, S_n \in \scr S \big\} \\
		\scr T &:= \Big\{ \bigcup \scr B' : \scr B' \subset \scr B \Big\}.
	\end{align*}
	Dann ist $\scr T$ die gröbste (kleinste) Topologie, die $\scr S$ enthält und $\scr B$ ist eine Basis von $\scr T$.
	% fixme: proof
\end{st}

\begin{ex}
	% fixme: reference
	\begin{itemize}
		\item
			Ist $\scr S$ selbst eine Topologie, dann ist $\scr S = \scr B = \scr T$.
		\item
			Für $\scr S = \emptyset$ ist $\scr B = \{X\}$ (leerer Schnitt ist $X$), $\scr T = \{\emptyset, X \}$.
		\item
			Aus
			\[
				\scr S = \{ ]a,+\infty[, ]-\infty, b[ : a,b \in \R \}
			\]
			erhalten wir
			\begin{align*}
				\scr B &= \scr S \cup \{\R \}  \cup \{ ]a,b[ : a < b \in \R \}, \\
				\scr T &= \text{Ordnungstopologie (übliche Topologie) auf $\R$}.
			\end{align*}
	\end{itemize}
\end{ex}

\begin{conv}
	Wir nennen $\scr S$ ein \emph{Erzeugendensystem} (oder \emph{Subbasis}) von $\scr T$.
\end{conv}

\begin{df}[Zweites Abzählbarkeitsaxiom]
	Ein topologischer Raum $(X, \scr T)$ erfüllt das \emph{zweite Abzählbarkeitsaxiomm} (2AA), wenn $\scr  T$ eine abzählbare Basis erlaubt.
\end{df}

\begin{ex}
	Betrachte $X = \R$ mit der üblichen Topologie $\scr T$ und $\scr B = \{ ]a,b[ : a,b \in \R \}$.
	Eine abzählbare Basis wäre
	\[
		\scr B = \{ ]a,b[ : a,b \in \Q \} \subset \scr B.
	\]
\end{ex}

\begin{ex}
	Betrachte $X = \R^n$ mit der euklidischen Topologie $\scr T$ und Basis
	\[
		\scr B = \Big\{ ]a_1,b_1[ \times \dotsb \times ]a_n,b_n[ : a_1 < b_1, \dotsc, a_n < b_n \text{ in $\R$} \Big\}.
	\]
	Werden $a_i, b_i$ rational gewählt, so erhält man eine abzählbare Basis für $\R^n$.
\end{ex}

\begin{st}
	Ist $\scr B$ eine Basis der Topologie $\scr T$ auf $X$, dann ist
	\[
		\Phi
		: \scr T \to \scr P (\scr B)
		: \Phi(U) := \{ B \in \scr B : B \subset U \}
	\]
	injektiv.
	Ist $\scr B$ abzählbar, so folgt
	\[
		\card(\scr T) \le \card(\R^n) = \card(\R).
	\]
	Speziell für die euklidische Topologie auf $\R^n$ $(n \ge 1)$ gilt $\card(\scr T) = \card(\R)$, dank der Injektion $\R \to \scr T : r \mapsto B(0,r)$.
	\begin{proof}
		$\Phi$ ist injektiv, da $U = \bigcup \Phi(U)$.
	\end{proof}
\end{st}

\begin{lem}
	% fixme: formulierung
	Zu jeder diskreten Teilmenge $A \subset X$ existiert eine Injektion $A \injto \scr B$, also $\card (A) \le \card(\scr B)$.
	\begin{proof}
		Zu $a \in A$ existiert eine Umgebung $U_a \in \scr T$ mit $U_a \cap A = \{a\}$.
		Somit existiert $B_a \in \scr B$ mit $a \in B_a \subset U_a$.
		Die Zuordnung $a \mapsto B_a$ ist injektiv, denn $B_a \cap A = \{a\}$.
	\end{proof}
\end{lem}

\begin{st}
	Sei $\scr C_b(\R, \R)$ der Vektorraum aller stetigen und beschränkten Funktionen $f: \R \to \R$ mit der Supremumsnorm $|f| = \sup \{|f(x)| : x \in \R \}$.

	Dieser erfüllt das erste Abzählbarkeitsaxiom, aber nicht das zweite Abzählbarkeitsaxiom.
	\begin{proof}
		Sei $A = \{0, 1\}^\Z$ % fixme: überabzählbar?.
		Jede Folge $a \in A$ definiert $f_a : \R \to \R$ durch
		\[
			f_a(x)
			:= (1-t)a_k + t a_{k+1},
			\qquad x = k+t, k \in \Z, t \in [0,1]
		\]
		(lineare Splineinterpolation).
		Die Zuordnung $A \to \scr C_b (\R, \R) : a \mapsto f_a$ ist injektiv, denn $a = f_a|_\Z$.
		Die Menge $\{f_a : a \in A\}$ ist also überabzählbar und diskret, denn $|f_a - f_b| = 1$ für $a\neq b$.
	\end{proof}
\end{st}

\begin{proof}
	Erlaubt $\scr T$ eine abzählbare Basis $\scr B$, dann auch eine abzählbare dichte Teilmenge $A \subset X$.
	\begin{proof}
		Zu $B \in \scr B$ mit $B \neq \emptyset$ wählen wir $a_B \in B$.
		Die Menge $A = \{a_B : B \in \scr B\}$ ist abzählbar und dicht:
		Zu jeder offenen Menge $U \neq \emptyset$ existiert $B \in \scr B, \scr B \neq \emptyset$ mit $B \subset U$, also $a_B \in B \subset U$.
	\end{proof}
\end{proof}

\begin{df}
	Ein topologischer Raum $(X, \scr T)$ heißt \emph{separabel}, wenn eine abzählbare dichte Teilmenge $A \subset X$ existiert.
\end{df}

\begin{st}
	Sei $(X,d)$ ein metrischer Raum und $A \subset X$ eine abzählbare dichte Teilmenge, dann gilt das zweite Abzählbarkeitsaxiom.
	\begin{proof}
		Folgende Basis erzeugt den metrischen Raum $(X,d)$:
		\[
			\scr B = \{ B(a, \f 1k) : a \in A, k \in \N_{\ge 1} \}.
		\]
		Für $U \in \scr T$ zeigen wir $U = \bigcup \{ B(a, \f 1k \in \scr B : B(a, \f 1k) \subset U \}$.
		„$\supset$“ ist klar, zeige „$\subset$“:
		Sei $k \in \N_{\ge 1}$ mit $B(x, \f 2k) \subset U$.
		Da $A$ dicht ist, existiert $a \in A \cap B(x, \f 1k)$.
		Es gilt
		\[
			x \in \scr B(a, \f 1k) \subset B(x, \f 2k) \subset U.
		\]
	\end{proof}
\end{st}

\begin{ex}
	$\R^n$ hat als Basis $\scr B = \{ B(a, \f 1k) : a \in \Q^n, k \in \N_{\ge 1} \}$
\end{ex}

\begin{st}
	Sei $\Omega$ eine Menge, $1 \le p \le \infty$.
	Dann ist $\ell^p(\Omega)$ genau dann separabel, wenn $\Omega$ abzählbar ist.

	Insbesondere sind $\R^n = \ell^2(\{1, \dotsc, n\})$, sowie $\ell^2(\N)$ und $\ell^2(\Z)$ separabel und erfüllen das zweite Abzählbarkeitsaxiom.
\end{st}


\section{Teilräume und Quotientenräume}


Seien $f: X \to Y$ und $V_i \subset X, i \in I$, dann gilt
\begin{align*}
	f^{-1}(Y \setminus V) &= X \setminus f^{-1}(V) \\
	f^{-1}\Big(\bigcap_{i\in I} V_i\Big) &= \bigcap_{i\in I} f^{-1}(V_i)
\end{align*}
\begin{df}
	Seien $X \stack f\to Y \stack g\to Z$ Abbildungen.
	\begin{enumerate}[(1)]
		\item
			Zurückziehen (pullback):
			Jede Topologie $\scr T_Z$ auf $Z$ induziert eine Topologie $g^*\scr T_Z$ auf $Y$ durch
			\[
				g^* \scr T_Z
				:= \{ g^{-1}(Wr : W \in \scr T_Z \}.
			\]
			Diese heißt \emph{initiale Topologie} bezüglich $g$ und ist die gröbste Topologie, für die $g$ stetig ist.
		\item
			Vorschieben (push forward):
			Jede Topologie $\scr T_X$ auf $X$ induziert eine Topologie $f_* \scr T_X$ auf $Y$ durch
			\[
				f_* \scr T_X = \{ V \subset Y : f^{-1}(V) \in \scr T_X \}.
			\]
			Diese heißt \emph{finale Topologie} bezüglich $f$ und ist die feinste Topologige auf $Y$, für die $f$ stetig ist.
	\end{enumerate}
\end{df}

\begin{ex}
	Sei $f: X \to Y$ konstant.
	\begin{itemize}
		\item
			$f^* \scr T_Y = \{ \emptyset, X \}$ ist die indiskrete Topologie,
		\item
			$f^* \scr T_X = \scr P(Y)$ ist die diskrete Topologie.
	\end{itemize}
\end{ex}

\begin{ex}
	Betrachte
	\begin{align*}
		s&: \R^m \to \R^n, (x_1,\dotsc, x_m) \mapsto (x_1, \dotsc, x_m, 0, \dotsc, 0), \\
		p&: \R^n \to \R^m, (x_1,\dotsc, x_n) \mapsto (x_1, \dotsc, x_m).
	\end{align*}
	Dann ist $s^* \scr T_{\R^n} = \scr T_{\R^m}$ und $p_* \scr T_{\R^n} = \scr T_{\R^m}$.
\end{ex}

\subsection{Teilraumtopologie}

\begin{df}
	Sei $(X, \scr T_X)$ ein topologischer Raum und $A \subset X$.
	Wir setzen
	\[
		\scr T_A = \{ A \cap U : U \in \scr T_X \}
		= \{ \jota_A^{-1}(U) : U \in \scr T_X \}
		= \jota_A^* \scr T_X
	\]
	(pullback bezüglich $\jota_A : A \to X, a \mapsto a$).
	Dies nennen wir \emph{Teilraumtopologie} auf $A$ und $(A, \scr T_A)$ einen Teilraum von $(X, \scr T_X)$.
\end{df}

\begin{ex}
	\begin{itemize}
		\item
			Die Teilraumtopologie von $\R$ in $\C$ ist die übliche Topologie auf $\R$.
		\item
			Ist $(X, d)$ ein metrischer Raum, $A \subset X, d_A = d|_{A\times A}$, so gilt $(A, \scr T_A) = (A, \scr T_{d_A})$.
	\end{itemize}
\end{ex}

\begin{nt}
	Eine offene Menge in $A$ liegt auch in $X$, ist aber im Allgemeinen nicht offen in $X$.

	$]0,1[$ ist offen in $\R$, aber nicht offen in $\C$.
\end{nt}

\begin{st}
	\begin{enumerate}[(1)]
		\item
			Die Teilraumtopologie $\scr T_A$ auf $A$ ist die gröbste, für die $\jota_A : A \to X$ stetig ist.
		\item
			Genau dann ist $f: Y \to A$ stetig, wenn $\jota_A \circ f: Y \to X$ stetig ist.
	\end{enumerate}
\end{st}

\begin{df}
	Seien $(X,\scr T_X)$ und $(Y, \scr T_Y)$ topologische Räume.
	Eine Abbildung $f: (X, \scr T_X) \to (Y, \scr T_Y)$ heißt \emph{Einbettung}, wenn $f$ ein Homöomorphismus von $X$ auf das Bild $f(X) \subset Y$ induziert.
\end{df}

\begin{ex}
	$\R$ kann in $\C$ eingebettet werden.
\end{ex}

\coursetimestamp{05}{11}{2013}

\subsection{Einbettungen}

\begin{ex}
	$\R^m \injto \R^n, (x_1, \dotsc, x_n) \mapsto (x_1, \dotsc, x_m, 0, \dotsc, 0)$ ist eine Einbettung.

	$p: [0,1[ \to \C, p(t) = e^{2\pi i t}$ ist stetig und injektiv, aber keine Einbettung.
\end{ex}

\subsection{Überdeckungen}

\begin{df}
	Eine Überdeckung $X = \bigcup_{i\in X} X_i$ heißt \emph{punktweise endlich}, wenn für jeden Punkt $x \in X$ die Menge $I_x := \{ i \in I : x \in X_i \}$ endlich ist.
	Die Überdeckung heißt \emph{lokal endlich}, wenn jeder Punkt $x \in X$ eine Umgebung $U$ in $X$ besitzt, für die die Menge $I_U := \{ i \in I : X_i \cap U \neq \emptyset \}$ endlich ist.
\end{df}

\begin{ex}
	\begin{itemize}
		\item
			Die Überdeckung
			\[
				\R = \bigcup_{n \in \N} ]-n, n[
			\]
			ist weder punktweise, noch lokal.
		\item
			Die Überdeckung
			\[
				\R = \bigcup_{x \in \R} \{x\}
			\]
			ist punktweise, aber nicht lokal endlich.
		\item
			Die Überdeckung
			\[
				\R = \bigcup_{n\in \Z} ]n-1, n+ 1[
			\]
			ist sowohl punktweise, als auch lokal endlich.
	\end{itemize}
\end{ex}

\begin{st} % fixme: label
	Sei $X = \bigcup_{i\in I} X_i$ eine offene Überdeckung, oder eine lokal endliche Überdeckung.

	Eine Teilmenge $V \subset X$ ist genau dann offen (abgeschlossen), wenn $V \cap X_i$ in jedem Teilraum $X_i$ offen (bzw. abgeschlossen) ist.
	\begin{proof}
		Sei $V_i := V \cap X_i$, es gilt dann $V = \bigcup_{i\in I} V_i$.
		Seien alle $X_i$ offen.
		Da $V_i$ in $X_i$ offen ist, existiert $\tilde V_i$ in $X$ offen mit $V_i = \tilde V_i \cap X$.
		Da $X_i$ in $X$ offen ist, ist auch $V_i$ offen in $X$.
		Damit ist auch $V = \bigcup_{i\in I} V_i$ offen in $X$.

		Abgeschlossenheit genauso durch Komplementbildung.
		Für lokal-endliche abgeschlossene Überdeckungen analog (Übung). % fixme: ref.
	\end{proof}
\end{st}

\subsection{Verklebung stetiger Abbildungen}

Betrachte $f: \R \to \R$ definiert durch
\[
	f(x) = \begin{cases}
		1 & x \in \Q \\
		0 & x \in \R \setminus \Q
	\end{cases}.
\]
$f$ ist unstetig, obwohl aus zwei stetigen Funktionen verklebt.
Dagegen ist die Funktion $f: \Q \to \Q$ mit
\[
	f(x) = \begin{cases}
		0 & x^2 > 2 \\
		1 & x^2 < 2
	\end{cases}.
\]
stetig.
$h : \R \to \R$ mit
\[
	h(x) = \begin{cases}
		0 & x < 0 \\
		1 & x \ge 0
	\end{cases}
\]
ist unstetig.
$k : \R \to \R$ mit
\[
	k(x) = \begin{cases}
		\sqrt[3]{x} & x \ge 0 \\
		-\sqrt[3]{-x} & x \le 0
	\end{cases}
\]
ist stetig.

\begin{st}
	Sei $(X, \scr T)$ ein topologischer Raum, $X = \bigcup_{i \in I} X_i$ und $f_i: X \to Y$ stetig mit $f_i|_{X_i \cap X_j} = f_j|_{X_i \cap X_j}$ für alle $i,j \in I$.

	Dann ist $f = \bigcup_{i \in I} f_i : X \to Y$ die einzige Abbildung mit $f|_{X_i} = f_i$ für alle $i \in I$.

	Ist die Überdeckung $(X_i)_{i\in I}$ offen oder lokal-endliche abgeschlossen, so ist $f$ stetig.
	\begin{proof}
		Sei $V \subset Y$ offen.
		Da $f_i : X_i \to Y$ stetig, ist $f_i^{-1}(V) \subset X_i$ offen in $X_i$.
		Damit ist $U = f^{-1}(V)$ offen in $X$, denn $U \cap X_i = f_i^{-1}(V)$ ist offen in jedem $X_i$.
		% fixme: ref
	\end{proof}
\end{st}

\subsection{Quotiententopologie}

\begin{df}
	Sei $(X, \scr T)$ ein topologischer Raum und $R \subset X \times X$ eine Äquivalenzrelation auf $X$.
	Sei $Q := X / R$ die Quotentenmenge und $q: X \to Q$ die Quotentenabbildung.
	Die \emph{Quotiententopologie} auf $Q$ ist dann gegeben durch
	\[
		\scr T_Q
		:= \{ U \subset Q : q^{-1}(U) \subset \scr T \}
		= q_* \scr T.
	\]
	Wir nennen dann $(Q, \scr T_Q)$ den \emph{Quotientenraum} von $(X, \scr T)$ bezüglich $R$.
\end{df}

\begin{ex}
	Betrachte $X = [-1,1]^2 \subset \R^2$ mit $f: [-1,1]^2 \to \R^3$ definiert durch
	\[
		f(s, t) = \begin{pmatrix}
			3 \cos(\pi s) \\
			3 \sin(\pi s) \\
			t
		\end{pmatrix}.
	\]
	$f$ bildet $X$ auf einen Zylinder im $\R^3$ ab.
	Definiere jetzt $g: [-1,1]^2 \to \R^3$ mit
	\[
		g(s, t) = \begin{pmatrix}
			(3 + t \sin(\f{\pi s}2)) \cos (\pi s) \\
			(3 + t \sin(\f{\pi s}2)) \sin (\pi s) \\
			t \cos(\f {\pi s}2)
		\end{pmatrix}
	\]
	Dies verklebt linke und rechte Seite von $X$ gegenläufig zu einem Möbiusband im $\R^3$.
\end{ex}

\begin{st}[Eigenschaften der finalen Topologie]
	\begin{enumerate}[(1)]
		\item
			Die Quotententopologie ist die feinste, für die $q$ stetig ist.
		\item
			$g: (Q, \scr T_Q) \to (Y, \scr T_y)$ ist genau dann stetig, wenn $f = g \circ q: X \to Y$ stetig ist.
	\end{enumerate}
\end{st}

\begin{df}
	Eine Abbildung $f: (X, \scr T_X) \to (Y, \scr T_y)$ heißt \emph{Identifizierung}, wenn sie einen Homöomorphismus auf dem Quotientenraum $X / R_f$ nach $Y$ induziert.
	Dabei ist $R_f = \{ (x,x') \in X \times X : f(x) = f(x') \}$.
	\begin{note}
		Jede Identifizierung muss surjektiv sein.
	\end{note}
\end{df}

\begin{ex}
	Die Abbildung $p: \R^n \to \R^m, (x_1, \dotsc, x_n) \mapsto (x_1, \dotsc, x_m)$ ist eine Identifizierung.
\end{ex}

\begin{ex}
	$p: [0,1[ \to \S^1, p(t) = e^{2\pi i t}$ ist stetig und surjektiv, sogar bijektiv, aber keine Identifizierung.
\end{ex}

\begin{st}[Kanonische Faktorisierung]
	Jede stetig Abbildung $f: X \to Y$ faktorisiert gemäß $f = \jota \circ \_f \circ g$ in die (surjektive) Quotientenabbildung $q: X \surto X / R_f$, die induzierte stetige Bijektion $\_f: X / R_f \to f(X)$ und die Inklusion $\jota: f(X) \injto y$.
\end{st}

\begin{ex}[Universelle Überlagerung der Kreislinie]
	Betrachte auf $\S^1 = \{ (x,y) \in \R^2 : x^2 + y^2 = 1 \}$ die Abbildung $p: \R \to \S^1$, definiert durch
	\[
		p(t) = e^{2\pi i t} = \Big( \cos(2\pi t), \sin(2\pi t) \Big).
	\]
	Es gilt $p(t) = p(t') \iff t - t' \in \Z$.

	$p$ ist eine Identifizierung, induziert also einen Homöomorphismus $\_p : \R / \Z \to \S^1$.
	\begin{proof}
		Es gibt zwei Arten, dies zu beweisen.
		\begin{seg}[Implizite Methode]
			$\R / \Z = q(\R) = q([0,1])$ ist kompakt, $\S^1$ ist hausdorffsch, also ist jede stetige Bijektion $\_p: \R / \Z \to \S^1$ ein Homöomorphismus (Kanonische Faktorisierung und abgeschlossene Menge wird auf abgeschlossene Meneg abgebildet).
		\end{seg}
		\begin{seg}[Explizite Methode]
			Für
			\begin{align*}
				U_1 &= \{ (x,y) \in \S^1 : x > 0 \},&
				V_1 &= ] - \f 14, \f 14 [, \\
				U_2 &= \{ (x,y) \in \S^1 : y > 0 \},&
				V_2 &= ] 0, \f 12 [, \\
				U_3 &= \{ (x,y) \in \S^1 : x < 0 \},&
				V_3 &= ] \f 14, \f 34 [, \\
				U_4 &= \{ (x,y) \in \S^1 : y < 0 \},&
				V_4 &= ] \f 12, 1 [, \\
				U_5 &= U_1,&
				V_5 &= V_1 + 1
			\end{align*}
			konstruieren wir $s_k: \S^1 \supset U_k \to V_k \subset \R$ durch
			\begin{align*}
				s_1(x,y) &= \f 1{2\pi} \arctan(\f yx), \\
				s_2(x,y) &= \f 1{2\pi} \arctan(\f xy), \\
				\vdots\; &= \quad \vdots \\
				s_5(x,y) &= s_1(x,y) + 1.
			\end{align*}
			$s_k : U_k \homto V_k$ ist ein Homöomorphismus mit Inversen $p|_{V_k} : V_k \homto U_k$.
			Leider lassen sich die $s_k: U_k \to \R$ nicht verkleben, aber im Quotentenraum geht es.

			%fixme: q: \R \to \R / \Z, s_k: U_k \to \R, p: \R \to \S^1, \_{s_k}: U_k \to \R / \Z, \_p: \R / \Z \to \S^1, \_s: \S^1 \to \R / \Z

			Wir erhalten $\_{s_k}: U_k \to \R / \Z$ und diese verkleben wir zu $\_s: \S^1 \to \R / \Z$.
			Nach Konstruktion gilt
			\[
				\_s \circ \_p = \Id_{\S^1}, \quad
				\_p \circ \_s = \Id_{\R / \Z}.
			\]
		\end{seg}
	\end{proof}
\end{ex}

