%fixme: \scr U vs U und \scr T vs \tau

\chapter{Topologische Räume}

\coursetimestamp{28}{10}{2013}

\begin{df}
	Eine \emph{Topologie} auf einer Menge $X$ ist ein System $\scr T \subset \scr P(X)$ von Teilmengen von $X$ mit folgenden Eigenschaften
	\begin{enumerate}[O1)]
		\item
			$\emptyset, X \in \scr T$,
		\item
			$O_1, \dotsc, O_n \in \scr T \implies O_1 \cap \dotsb \cap O_n \in \scr T$,
		\item
			$O_i \in \scr T$ für $i \in I \implies \bigcup_{i\in I} O_i \in \scr T$.
	\end{enumerate}
	Das Paar $(X, \scr T)$ nennen wir dann \emph{topologischer Raum}.
	Eine Teilmenge $Y \subset X$ nennen wir \emph{offen}, wenn $Y \in \scr T$, und \emph{abgeschlossen}, wenn $X \setminus Y \in \scr T$.
\end{df}

\begin{ex}[Kleinste Beispiele]
	\begin{itemize}
		\item
			$X = \emptyset, \scr T = \{\emptyset\}$,
		\item
			$X = \{a\}, \scr T = \{ \emptyset, X \}$,
		\item
			Für $x = \{a,b\}$ gibt es vier Topologien
			\begin{enumerate}[1.]
				\item
					$\{\emptyset, X\}$
				\item
					$\{\emptyset, \{a\}, X\}$
				\item
					$\{\emptyset, \{b\}, X \}$
				\item
					$\{\emptyset, \{a\}, \{b\}, X\}$
			\end{enumerate}
	\end{itemize}
	\begin{table}
		\centering
		\caption{Anzahl der Topologien auf einer Menge $X$ mit $n$ Elementen, asymptotisches Verhalten: $\log t_n \sim \f {n^2}4$}
		\begin{tabular}{c|c}
			$n=|x|$ & $t_n = $ Anzahl der Topologie auf $X$ \\ \hline
			0 & 1 \\
			1 & 1 \\
			2 & 4 \\
			3 & 29 \\
			4 & 355 \\
			5 & 6942
		\end{tabular}
	\end{table}
\end{ex}

\begin{ex}[diskrete Topologie]
	$\scr T = \scr P(X)$ auf $X$.
\end{ex}

\begin{ex}[indiskrete Topologie]
	$\scr T = \{\emptyset, X\}$ auf $X$.
\end{ex}

\begin{ex}[metrische Topologie] \label{ex:metric_topology}
	Die metrische Topologie zu einer (Halb-)Metrik $d: X \times X \to [0,\infty]$ ist gegeben durch
	\begin{align*}
		\scr T_d
		&= \{ O \subset X : \forall a \in O \exists \eps > 0 : B(a,\eps) \subset O \} \\
		&= \{ O \subset X : \forall a \in O \exists \eps > 0 : d(a,x) < \eps \implies x \in O \}.
	\end{align*}
\end{ex}

\begin{nt}
	% fixme: reference
	Die offenen Mengen einer metrischen Topologie entsprechen den offenen Mengen aus dem letzten Kapitel.
\end{nt}

\begin{df}[metrisierbare Topologie]
	Eine Topologie $\scr T$ heißt \emph{metrisierbar}, wenn es eine Metrik $d$ gibt mit $\scr T = \scr T_d$.
\end{df}

\begin{ex}
	Die diskrete Topologie ist metrisierbar (dank der diskreten Metrik).

	Die indiskrete Topologie hingegen nicht (für $|x| \ge 2$).
	(Jede metrische Topologie trennt je zwei Punkte $a \neq b$, die indiskrete Topologie tut das nicht)
\end{ex}

\begin{df}
	Sind $\scr T_1 \subset \scr T_2 \subset \scr P(X)$ Topologien auf $X$, so nennen wir $\scr T_1$ \emph{gröber} als $\scr T_2$, bzw. $\scr T_2$ \emph{feiner} als $\scr T_1$.
\end{df}

\begin{ex}
	Sei $(X,<)$ eine linear geordnete Menge, z.B. $(\R,<)$.
	Wir nennen $O \subset X$ offen, wenn zu jedem $x \in O$ ein offenes Intervall $I \subset X$ existiert mit $x \in I \subset O$.
	Das System $\scr T_<$ dieser offener Mengen heißt \emph{Ordnungstopologie} von $(X,<)$.
\end{ex}

\begin{ex}[Koendliche Topologie auf $X$]
	Setze
	\[
		\scr T
		:= \{ O \subset X : X \setminus O \text{ ist endlich} \} \cup \{\emptyset\}.
	\]
\end{ex}

\begin{ex}[Koabzählbare Topologie auf $X$]
	Setze
	\[
		\scr T
		:= \{ O \subset X : X \setminus O \text{ ist abzählbar} \} \cup \{ \emptyset \}.
	\]
\end{ex}

\begin{nt}
	Es gilt stets
	\[
		\scr T_{\text{indiskret}}
		\subset \scr T_{\text{koendlich}}
		\subset \scr T_{\text{koabzählbar}}
		\subset \scr T_{\text{diskret}}
	\]
	(nach links hin gröber, nach rechts hin gröber).
\end{nt}

\begin{df}
	Eine Abbildung $f: (X,\scr T_X) \to (Y, \scr T_Y)$ heißt \emph{stetig}, wenn aus $V \in \scr T_Y$ stets $f^{-1}(V) \in \scr T_X$ folgt.

	Hingegen heißt $f$ \emph{offen}, wenn aus $U \in \scr T_X$, stets $f(X) \in T_Y$ folgt und \emph{abgeschlossen}, wenn aus $X \setminus A \in \scr T_X$ stets $Y \setminus f(A) \in \scr T_Y$ folgt.
	\begin{nt}
		Diese Eigenschaften sind voneinander unabhängig:
		\begin{itemize}
			\item
				$\Id_\R: \R \to \R$ hat alle Eigenschaften,
			\item
				$f: \R \to \R, f(x) = x^2$ ist stetig, abgeschlossen, aber nicht offen,
			\item
				$f: \R \to \R, f(x) = \arctan(x)$ ist stetig, offen, aber nicht abgeschlossen,
			\item
				$f: \R \to \R, f(x) = \f 1{1+x^2}$ ist stetig, aber weder offen, noch abgeschlossen,
			\item
				\dots
		\end{itemize}
	\end{nt}
\end{df}

\begin{df}
	Eine Homöomorphismus $f: (X,\scr T_X) \to (Y, \scr T_Y)$ ist eine bijektive Abbildung $f: X \to Y$, sodass $f$ und $f^{-1}$ stetig sind.
\end{df}

\begin{nt}
	Aus der Stetigkeit von $f$ folgt im Allgemeinen nicht die Stetigkeit von $f^{-1}$.
	Betrachte dazu $f: [1,2[ \cup [3,4] \to [1,3[$ mit
	\[
		f(x) = \begin{cases}
			x & x \in [1,2[ \\
			x - 1 & x \in [3,4[
		\end{cases}.
	\]
	$f$ ist bijektiv, stetig, aber kein Homöomorphismus.
	Oder: $f : [0,1[ \to \S^1 \subset \C$ definiert durch
	\[
		f(t)
		= e^{2\pi i t}
		= (\cos (2\pi t), \sin (2\pi t))
	\]
	ist bijektiv, stetig, aber kein Homöomorphismus (es gibt sogar kein Homöomorphismus zwischen $[0,1[$ und $\S^1$).
\end{nt}


\section{Umgebungen}


Sei im folgenden $(X, \scr T)$ ein topologischer Raum, $x \in X$.

\begin{df}
	Eine Menge $\scr O$ heißt \emph{offene Umgebung} von $x$ in $(X,\scr T)$, wenn $x \in O \in \scr T$ gilt.
	\[
		U_x^O = U_x^O(\scr T)
		:= \{ O \in \scr T : x \in O \}
	\]

	Eine Menge $U$ heißt \emph{Umgebung} von $x$ in $(X, \scr T)$, wenn sie eine offene Umgebung von $x$ enthält
	\[
		U_x = U_x(\scr T)
		:= \{ U \subset X : \exists O \in \scr T : x \in O \subset U \}
	\]
\end{df}

\begin{df}
	Eine Familie $\scr B_x \subset U_x$ heißt \emph{Umgebungsbasis} von $x$ in $(X, \scr T)$, wenn jede Umgebung $U$ von $x$ eine Umgebung $V \in \scr B_A$ enthält.

	Dann ist
	\[
		U_x = \{ U \subset X : \exists V \in \scr B_x : V \subset U \}.
	\]
\end{df}

\begin{ex}
	\begin{itemize}
		\item
			$U_x$ ist eine Umgebungsbasis, ebenso $U_x^O$.
		\item
			In jedem metrischen Raum $(X,d)$ bilden die Bälle $B(a,\eps)$ mit $\eps > 0$ eine Umgebungsbasis von $a$, ebenso $B(a, \f 1k)$ mit $k= 1, 2, \dotsc$,
			diese ist sogar abzählbar.
	\end{itemize}
\end{ex}

\begin{df}
	Ein topologischer Raum $(X, \scr T)$ erfüllt das \emph{erste Abzählbarkeitsaxiom} (1AA), wenn jeder Punkt $a \in X$ eine abzählbare Umgebungsbasis besitzt.
	\begin{nt*}
		Hat $a$ eine abzählbare Umgebungsbasis $(V_n)_{n\in\N}$, so auch eine offene Umgebungsbasis, etwa $(\mathring V_n)_{n\in\N}$.

		Durch Übergang zu $U_n := \mathring V_0 \cap \dotsb \cap \mathring V_n$ erhalten wir eine offene Basis $U_0 \supset U_1 \supset U_2 \supset \dotsb$.
	\end{nt*}
\end{df}

\begin{ex}
	\begin{itemize}
		\item
			Jeder metrische Raum erfüllt das 1AA.
		\item
			Auf einer überabzählbaren Menge $X$, etwa $X = \R$, erfüllen die koendliche und die koabzählbare Topologie nicht das 1AA.
			Diese können also nicht metrisiert werden.
	\end{itemize}
\end{ex}

\begin{df}
	Eine Folge $(x_n)_{n\in \N}$ in $X$ \emph{konvergiert} gegen $a \in X$ bezüglich $\scr T$, wenn jede Umgebung $U$ von $a$ in $(X, \scr T)$ fast alle Folgenglieder enthält:
	\[
		(x_n)_{n\in \N} \stack{\scr T}\to a
		\iff
		\forall U \in \scr U_a(\scr T) \exists m \in \N \forall n \ge m : x_n \in U
	\]
\end{df}

\begin{prop}
	Eine Folge $(x_n)$ in $X$ konvergiert gegen $a \in X$ genau dann, wenn
	\[
		\forall V uin \scr B_x \exists m \in \N \forall n \ge m : x_n \in V.
	\]
\end{prop}

Grenzwerte sind im Allgemeinen nicht eindeutig, betrachte dazu folgendes Beispiel

\begin{ex}
	Ist $(X, \{\emptyset, X\})$ ein indiskreter Raum, so konvergiert jede Folge gegen jeden beliebigen Punkt $a \in X$.
\end{ex}

\begin{df}
	Ein topologischer Raum $(X, \scr T)$ heißt \emph{hausdorffsch}, wenn zu je zwei Punkten $a \neq b$ in $X$ disjunkte Umgebungen existieren, d.h. $U \in \scr U_a, V \in \scr U_b$, so dass $U \cap V = \emptyset$.
\end{df}

\begin{ex}
	Jeder metrische Raum ist hausdorffsch.
\end{ex}

\begin{st}
	Ist $(X,\scr T)$ hausdorffsch, dann hat jede Folge in $X$ höchstens einen Grenzwert in $X$.

	Die Umkehrung gilt, wenn $(X, \scr T)$ das erste Abzählbarkeitsaxiom erfüllt.
	\begin{proof}
		Aus $x_n \to a$ und $y_n \to b$ folgt $a = b$ (mit $U \in \scr U_a, V \in \scr U_b, U \cap V = \emptyset$ folgt der Widerspruch).
	\end{proof}
\end{st}
