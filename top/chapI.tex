\chapter{Klassifikation der Flächen}



\section{Mannigfaltigkeiten}


\begin{df}
	Ein Raum $M$ heißt \emph{lokal euklidisch} der Dimension $n$, wenn zu jedem $x \in M$ eine offene Umgebung $U \subset M$ existiert und ein Homöomorphismus $h: U \homeomorphicto \R^n$
\end{df}

\begin{ex}
	\begin{enumerate}[1)]
		\item
			Jede offene Menge $\M \subset \R^n$ ist lokal euklidisch der Dimension $n$.
		\item
			$\S^n \subset \R^{n+1}$ ist lokal euklidisch der Dimension $n$.
		\item
			$\D^n \subset \R^n$ ist nicht lokal euklidisch (jede Umgebung für Punkte auf dem Rand sind nicht homöomorph zu $\R^n$)
		\item
			Die Gerade mit doppeltem Ursprung ist lokal-euklidisch, aber nicht hausdorffsch.
	\end{enumerate}
\end{ex}

Setze
\begin{align*}
	\R_{\ge 0}^n &:= \{ x \in \R^n : x_1 \ge 0 \} \\
	\boundary \R_{\ge 0}^n &:= \{ x \in \R^n : x_1 = 0 \} \\
	\Int \R_{\ge 0}^n &:= \{ x \in \R^n : x_1 > 0 \}
\end{align*}

\begin{df}
	Sei $M$ ein topologischer Raum, sodass
	\begin{enumerate}[1)]
		\item
			$M$ ist hausdorffsch,
		\item
			Jede Komponente von $M$ hat abzählbare Basis der Topologie (zweites Abzählbarkeitsaxiom).
	\end{enumerate}
	Wir nennen $M$ eine \emph{$n$-Mannigfaltigkeit}, wenn gilt
	\begin{enumerate}[1),resume]
		\item
			Zu $x \in M$ existiert eine offene Umgebung $U \subset M$ und $h: U \homeomorphicto V$ auf eine offene Menge $v \subset \R_{\ge 0}^n$.
	\end{enumerate}
	\begin{align*}
		\Int M &:= \{ x \in M : \exists h: h(x) \in \Int \R_{\ge 0}^n \} \\
		\boundary M &:= \{ x \in M : \exists h: h(x) \in \boundary \R_{\ge 0}^n \}
	\end{align*}
	Ist $M$ kompakt und $\partial M = \emptyset$, so heißt $M$ \emph{geschlossen}.
	Ist $M$ nicht kompakt und $\partial M = \emptyset$, so heißt $M$ \emph{offen}.
\end{df}

\begin{st}
	Ist $M \neq \emptyset$ sowohl $m$-, als auch $n$-Mannigfaltigkeit, so gilt $m = n$.
	\begin{proof}
		Folgt aus Invarianz der Dimension.
	\end{proof}
\end{st}

\begin{df}
	Ein \emph{Atlas} von $M$ ist eine Familie von \emph{Karten}
	\[
		\scr A = (M \supset U_i \xrightarrow[\homeomorphic]{h_i} V_i \subset \R_{\ge 0}^n)_{i\in I}
	\]
	mit $M = \bigcup_{i \in I} U_i$ und $U_i, V_i$ offen.

	Zu $i,j \in I$ sei $U_{ij} := U_i \cap U_j, V_{ij} = h_i(U_{ij}), h_{ij} : V_i \supset V_{ij} \homeomorphicto V_{ji} \subset V_j$ mit $h_{ij} = h_j \circ h_i^{-1}|_{V_{ij}}$.

	Man nennt $h_{ij}$ den \emph{Kartenwechsel} von $i$ nach $j$.
	Ein Atlas $\scr A$ heißt
	\begin{itemize}
		\item
			\emph{orientiert}, wenn alle $h_{ij}$ orientierungserhaltend sind,
		\item
			\emph{$\scr C^r$-glatt}, wenn alle $h_{ij}$ $\scr C^r$-glatt sind.
	\end{itemize}
\end{df}


% §I2
\section{Projektive Räume}


Bisher haben wir meist die Sphären als Teilräume betrachtet:
\[
	\S^n = \{ x \in \R^{n+1} : x_0^2 + \dotsb + x_n^2 = 1 \} \subset \R^{n+1}.
\]
Wir können Sphären aber auch als Quotienten betrachten:
\[
	(\R^{n+1} \setminus \{0\}) / \R_{>0} \homeomorphic \S^n.
\]
(d.h jeweils zwei Vektoren in $\R^{n+1} \setminus \{0\}$ werden miteinander identifiziert, wenn sie voneinander mit einem positiven Faktor linear abhängig sind)
Aber warum und wie funktioniert das?
\[
	\begin{tikzcd}[column sep=large,row sep=large]
		\R^{n+1} \setminus \{0\} \arrow{r}{x \mapsto (\f x{|x|}, |x|)} \arrow{d}{q} \arrow{dr}{x \mapsto \f x{|x|}} &
		\S^n \times \R_{>0} \arrow{l}{rv \mapsfrom (v,r)} \arrow{dl} \arrow{d}{pr_1} \\
		(\R^{n+1} \setminus \{0\}) / \R_{>0} \arrow{r}{\homeomorphic} &
		\S^n \arrow{l}{}
	\end{tikzcd}
\]

\subsection{Reell-projektive Räume}

Setze
\begin{align*}
	\R\P^n &:= (\R^{n+1} \setminus \{0\}) / (\R \setminus \{0\}) \\
	\C\P^n &:= (\C^{n+1} \setminus \{0\}) / (\C \setminus \{0\})
\end{align*}

Sei $q: X := \R^{n+1} \setminus \{0\} \to \R\P^n$ die Quotientenabbildung.
Zu $x = (x_0, \dotsc, x_n) \in X$ schreiben wir $q(x) = [x_0 : \dotsc : x_n]$.
Dies nennen wir \emph{homogene Koordinaten}, denn es gilt
\[
	[x_0 : \dotsc : x_n ] = [\lambda x_0 : \dotsc : \lambda x_n]
\]
für alle $\lambda \in \R \setminus \{0\}$.

\begin{nt}
	Es gilt $\R\P^n \homeomorphic \S^n / \{\pm 1\}$.
	Insbesondere ist damit $\R\P^n$ kompakt.

	$\R\P^n$ ist eine kompakte $n$-Mannigfaltigkeit ohne Rand.
	\begin{proof}
		\[
			\begin{tikzcd}[column sep=large,row sep=large]
				\R^{n+1} \setminus \{0\} \arrow{r}{x\mapsto \f x{|x|}} \arrow{d}{q} \arrow{dr}{} &
				\S^n \times \R_{>0} \arrow{l}{\jota} \arrow{dl} \arrow{d}{p} \\
				\R\P^n \arrow{r}{\homeomorphic} &
				\S^n \setminus \{\pm 1\} \arrow{l}
			\end{tikzcd}
		\]
	\end{proof}
\end{nt}


\section{Klassifikation geschlossener Flächen}


\begin{ex}
	\begin{itemize}
		\item
			$F_0^+ = \S^2$
		\item
			$F_1^+ = \S^1 \times \S^1$
		\item
			$F_2^+$
		\item
			$F_3^+$
	\end{itemize}
	Dies sind zusammenhängende, kompakte $2$-Mannigfaltigkeiten ohne Rand.
	Sie sind zudem orientierbar.
	Es gibt zudem eine Familie nicht-orientierbarer, geschlossener Flächen: $F_g^- := F_g^+ / \{\pm 1\}$
	\begin{itemize}
		\item
			$F_0^- = \S^2 / \{\pm 1\}$ ist die reell-projektive Ebene,
		\item
			$F_1^- = (\S^1 \times \S^1)_{\pm 1}$ ist die kleinsche Flasche.
	\end{itemize}
\end{ex}



