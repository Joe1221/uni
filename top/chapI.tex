\chapter{Klassifikation der Flächen}



\section{Mannigfaltigkeiten}


\begin{df}
	Ein Raum $M$ heißt \emph{lokal euklidisch} der Dimension $n$, wenn zu jedem $x \in M$ eine offene Umgebung $U \subset M$ existiert und ein Homöomorphismus $h: U \homeomorphicto \R^n$
\end{df}

\begin{ex}
	\begin{enumerate}[1)]
		\item
			Jede offene Menge $\M \subset \R^n$ ist lokal euklidisch der Dimension $n$.
		\item
			$\S^n \subset \R^{n+1}$ ist lokal euklidisch der Dimension $n$.
		\item
			$\D^n \subset \R^n$ ist nicht lokal euklidisch (jede Umgebung für Punkte auf dem Rand sind nicht homöomorph zu $\R^n$)
		\item
			Die Gerade mit doppeltem Ursprung ist lokal-euklidisch, aber nicht hausdorffsch.
	\end{enumerate}
\end{ex}

Setze
\begin{align*}
	\R_{\ge 0}^n &:= \{ x \in \R^n : x_1 \ge 0 \} \\
	\boundary \R_{\ge 0}^n &:= \{ x \in \R^n : x_1 = 0 \} \\
	\Int \R_{\ge 0}^n &:= \{ x \in \R^n : x_1 > 0 \}
\end{align*}

\begin{df}
	Sei $M$ ein topologischer Raum, sodass
	\begin{enumerate}[1)]
		\item
			$M$ ist hausdorffsch,
		\item
			Jede Komponente von $M$ hat abzählbare Basis der Topologie (zweites Abzählbarkeitsaxiom).
	\end{enumerate}
	Wir nennen $M$ eine \emph{$n$-Mannigfaltigkeit}, wenn gilt
	\begin{enumerate}[1),resume]
		\item
			Zu $x \in M$ existiert eine offene Umgebung $U \subset M$ und $h: U \homeomorphicto V$ auf eine offene Menge $v \subset \R_{\ge 0}^n$.
	\end{enumerate}
	\begin{align*}
		\Int M &:= \{ x \in M : \exists h: h(x) \in \Int \R_{\ge 0}^n \} \\
		\boundary M &:= \{ x \in M : \exists h: h(x) \in \boundary \R_{\ge 0}^n \}
	\end{align*}
	Ist $M$ kompakt und $\partial M = \emptyset$, so heißt $M$ \emph{geschlossen}.
	Ist $M$ nicht kompakt und $\partial M = \emptyset$, so heißt $M$ \emph{offen}.
\end{df}
