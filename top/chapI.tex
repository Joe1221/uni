\chapter{Klassifikation der Flächen}



\section{Mannigfaltigkeiten} \label{sec:manifolds}


\begin{df}
	Ein Raum $M$ heißt \emph{lokal euklidisch} der Dimension $n$, wenn zu jedem $x \in M$ eine offene Umgebung $U \subset M$ existiert und ein Homöomorphismus $h: U \homeomorphicto \R^n$
\end{df}

\begin{ex}
	\begin{enumerate}[1)]
		\item
			Jede offene Menge $M \subset \R^n$ ist lokal euklidisch der Dimension $n$.
		\item
			$\S^n \subset \R^{n+1}$ ist lokal euklidisch der Dimension $n$.
		\item
			$\D^n \subset \R^n$ ist nicht lokal euklidisch (jede Umgebung für Punkte auf dem Rand sind nicht homöomorph zu $\R^n$)
		\item
			Die Gerade mit doppeltem Ursprung ist lokal-euklidisch, aber nicht hausdorffsch.
	\end{enumerate}
\end{ex}

Wir vereinbaren für $n \ge 1$:
\begin{align*}
	\R_{\ge 0}^n &:= \{ x \in \R^n : x_1 \ge 0 \}, \\
	\boundary \R_{\ge 0}^n &:= \{ x \in \R^n : x_1 = 0 \}, \\
	\Int \R_{\ge 0}^n &:= \{ x \in \R^n : x_1 > 0 \}.
\end{align*}
Im Fall $n = 0$ setzen wir dagegen $\R_{\ge 0}^0 := \{ 0 \}$, $\boundary \R_{\ge 0}^0 := \emptyset$, $\Int \R_{\ge 0}^n := \{ 0 \}$.

\begin{df}
	Sei $M$ ein topologischer Raum, sodass
	\begin{enumerate}[1)]
		\item
			$M$ hausdorffsch und
		\item
			jede Komponente von $M$ eine abzählbare Basis der Topologie hat (zweites Abzählbarkeitsaxiom).
	\end{enumerate}
	Wir nennen $M$ eine \emph{$n$-Mannigfaltigkeit}, wenn gilt
	\begin{enumerate}[1),resume]
		\item
			Zu $x \in M$ existiert eine offene Umgebung $U \subset M$ und $h: U \homeomorphicto V$ auf eine offene Menge $V \subset \R_{\ge 0}^n$.
	\end{enumerate}
	\begin{align*}
		\Int M &:= \{ x \in M : \exists h: h(x) \in \Int \R_{\ge 0}^n \} \\
		\boundary M &:= \{ x \in M : \exists h: h(x) \in \boundary \R_{\ge 0}^n \}
	\end{align*}
	Ist $M$ kompakt und $\partial M = \emptyset$, so heißt $M$ \emph{geschlossen}.
	Ist $M$ nicht kompakt und $\partial M = \emptyset$, so heißt $M$ \emph{offen}.
\end{df}

\begin{st}
	Ist $M \neq \emptyset$ sowohl $m$-, als auch $n$-Mannigfaltigkeit, so gilt $m = n$.
	\begin{proof}
		Folgt aus Invarianz der Dimension.
	\end{proof}
\end{st}

\begin{df}
	Ein \emph{Atlas} von $M$ ist eine Familie von \emph{Karten}
	\[
		\scr A = (M \supset U_i \xrightarrow[\homeomorphic]{h_i} V_i \subset \R_{\ge 0}^n)_{i\in I}
	\]
	mit $M = \bigcup_{i \in I} U_i$ und $U_i, V_i$ offen.

	Zu $i,j \in I$ sei $U_{ij} := U_i \cap U_j, V_{ij} = h_i(U_{ij}), h_{ij} : V_i \supset V_{ij} \homeomorphicto V_{ji} \subset V_j$ mit $h_{ij} = h_j \circ h_i^{-1}|_{V_{ij}}$.

	Man nennt $h_{ij}$ den \emph{Kartenwechsel} von $i$ nach $j$.
	Ein Atlas $\scr A$ heißt
	\begin{itemize}
		\item
			\emph{orientiert}, wenn alle $h_{ij}$ orientierungserhaltend sind,
		\item
			\emph{$\scr C^r$-glatt}, wenn alle $h_{ij}$ $\scr C^r$-glatt sind.
	\end{itemize}
\end{df}


% §I2
\section{Projektive Räume}


Bisher haben wir meist die Sphären als Teilräume betrachtet:
\[
	\S^n = \{ x \in \R^{n+1} : x_0^2 + \dotsb + x_n^2 = 1 \} \subset \R^{n+1}.
\]
Wir können Sphären aber auch als Quotienten betrachten:
\[
	(\R^{n+1} \setminus \{0\}) / \R_{>0} \homeomorphic \S^n.
\]
(d.h jeweils zwei Vektoren in $\R^{n+1} \setminus \{0\}$ werden miteinander identifiziert, wenn sie voneinander mit einem positiven Faktor linear abhängig sind).
Aber warum und wie funktioniert das?
\[
	\begin{tikzcd}[column sep=large,row sep=large]
		\R^{n+1} \setminus \{0\} \arrow{r}{x \mapsto (\f x{|x|}, |x|)} \arrow{d}{q} \arrow{dr}{x \mapsto \f x{|x|}} &
		\S^n \times \R_{>0} \arrow{l}{rv \mapsfrom (v,r)} \arrow{dl} \arrow{d}{pr_1} \\
		(\R^{n+1} \setminus \{0\}) / \R_{>0} \arrow{r}{\homeomorphic} &
		\S^n \arrow{l}{}
	\end{tikzcd}
\]

\subsection{Reell-projektive Räume}

Setze
\begin{align*}
	\R\P^n &:= (\R^{n+1} \setminus \{0\}) / (\R \setminus \{0\}) \\
	\C\P^n &:= (\C^{n+1} \setminus \{0\}) / (\C \setminus \{0\})
\end{align*}

Sei $q: X := \R^{n+1} \setminus \{0\} \to \R\P^n$ die Quotientenabbildung.
Zu $x = (x_0, \dotsc, x_n) \in X$ schreiben wir $q(x) = [x_0 : \dotsc : x_n]$.
Dies nennen wir \emph{homogene Koordinaten}, denn es gilt
\[
	[x_0 : \dotsc : x_n ] = [\lambda x_0 : \dotsc : \lambda x_n]
\]
für alle $\lambda \in \R \setminus \{0\}$.

\begin{nt}
	Es gilt $\R\P^n \homeomorphic \S^n / \{\pm 1\}$.
	Insbesondere ist damit $\R\P^n$ kompakt.

	$\R\P^n$ ist eine kompakte $n$-Mannigfaltigkeit ohne Rand.
	\begin{proof}
		\[
			\begin{tikzcd}[column sep=large,row sep=large]
				\R^{n+1} \setminus \{0\} \arrow{r}{x\mapsto \f x{|x|}} \arrow{d}{q} \arrow{dr}{} &
				\S^n \times \R_{>0} \arrow{l}{\jota} \arrow{dl} \arrow{d}{p} \\
				\R\P^n \arrow{r}{\homeomorphic} &
				\S^n \setminus \{\pm 1\} \arrow{l}
			\end{tikzcd}
		\]
	\end{proof}
\end{nt}


\section{Klassifikation geschlossener Flächen}

\paragraph{Modellflächen}

\begin{ex}
	\begin{itemize}
		\item
			$F_0^+ = \S^2$
		\item
			$F_1^+ = \S^1 \times \S^1$
		\item
			$F_2^+$
		\item
			$F_3^+$
	\end{itemize}
	Dies sind zusammenhängende, kompakte $2$-Mannigfaltigkeiten ohne Rand.
	Sie sind zudem orientierbar (dafür steht das $+$).
	Es gibt zudem eine Familie nicht-orientierbarer, geschlossener Flächen: $F_g^- := F_g^+ / \{\pm 1\}$
	\begin{itemize}
		\item
			$F_0^- = \S^2 / \{\pm 1\}$ ist die reell-projektive Ebene,
		\item
			$F_1^- = (\S^1 \times \S^1)_{\pm 1}$ ist die Kleinsche Flasche.
	\end{itemize}
\end{ex}

\coursetimestamp{20}{01}{2014}

% geschlossene Flächen: kompakt, ohne Rand

\paragraph{Triangulierte Modellflächen} 

Setze
\begin{align*}
	Q_0 &:= [-2, 2] \times [-2 \times 2] \\
	Q_1 &:= Q_0 \setminus ([-1,1] \times [-1,1])\\
	Q_g &:= \bigcup_{k=1}^g (Q_1 - 2 - 2g + 4k) \qquad g \ge 2
\end{align*}
%fixme: drawing
und
\[
	H_g := Q_g \times [-1,+1] \subset \R^3
\]
$H_g$ nennen wir \emph{Henkelkörper vom Geschlecht $g$}.

\begin{st}
	Für $g \in \N$ ist $F_g^+ := \boundary H_g$ eine zusammenhängende, geschlossene, orientierbare Fläche.
	Es gilt $\chi(F_g^+) = 2 - 2g$.
	Auch $F_g^- = F_g^+ / \{\pm 1\}$ ist eine zusammenhängende, geschlossene, nicht-orientierbare Fläche.
	Es gilt $\chi(F_g^-) = 1 - g$.
\end{st}

\begin{st}
	Sei $K$ ein Simplizialkomplex.
	Die topologische Realisierung $|K|$ ist genau dann eine Fläche, wenn der Stern jeder Ecke eine der folgenden Formen hat:

	Die Ecke liegt im Inneren mit sternförmig anliegenden Dreiecken, oder sie liegt auf dem Rand mit sternförmig anliegenden Dreiecken.
	%fixme: drawing
	\begin{proof}
		Übung
	\end{proof}
\end{st}

\paragraph{Polygonmodelle}

Sei $n \in \N_{\ge 1}$, $\gamma_k: [0,1] \to \S^1 = \boundary \D^2$, $\gamma_k(t) = e^{\f {2\pi i}n (k-1 + t)}$ für $k = 1, \dotsc, n$.
Sei $w = w_1 \dotsc w_n$ ein Wort  in $a^{\pm 1}, b^{\pm 1}, c^{\pm 1}, \dotsc$

Für $w_k = w_l$, identifiziere $\gamma_k(t) \sim \gamma_l(t)$, für $w_k = w_k^{-1}$ identifiziere $\gamma_k(t) \sim \gamma_l(1-t)$.

\begin{df}
	$\D^2 / \<w\> := \D^2 / \sim$
\end{df}

\begin{ex}
	\begin{itemize}
		\item
			$\D^2 / \<aa^{-1}\> \homeomorphic \S^2$
		\item
			$\D^2 / \<aa\> \homeomorphic \R\P^2$
		\item
			$\D^2 / \<aba^{-1}b^{-1}\> \homeomorphic \S^1 \times \S^1$
		\item
			$\D^2 / \<abab^{-1}\> \homeomorphic (\S^1 \times \S^1) / \{\pm 1\}$ (Kleinsche Flasche)
	\end{itemize}
\end{ex}

\begin{st}
	$\D^2 / \<w\>$ ist genau dann eine geschlossene Fläche, wenn jeder Buchstabe in $w$ genau zweimal vorkommt.

	Tritt ein Buchstabe in $w$ zweimal mit gleichem Exponenten auf, so ist $\D^2 / \<w\>$ nicht-orientierbar, andernfalls orientierbar.
\end{st}

\begin{ex}[Kanonische Beispiele]
	\begin{enumerate}[(1)]
		\item
			$\D^2 / \<a_1b_1a_1^{-1}b_1^{-1} \dotsc a_gb_ga_g^{-1}b_g^{-1} \> \homeomorphic F_g^+$,
			$\chi(F_g^+) = 2 - 2g$.
		\item
			$\D^2 / \<c_0c_0 \dotsc c_gc_g\> \homeomorphic F_g^-$,
			$\chi(F_g^-) = 1 - g$.
	\end{enumerate}
\end{ex}

\begin{lem}
	Jede zusammenhängende, geschlossene, triangulierte Fläche $F$ ist homöomorph zu $\D^2 / \<w\>$ mit einem geeigneten Wort $w$.
\end{lem}

\begin{lem}
	\begin{itemize}
		\item
			Es gilt
			\[
				\D^2 / \<\dotsc a^\eps \dotsc a^\delta \dotsc \>
				= \D^2 / \<\dotsc b^\eps \dotsc b^\delta \dotsc \>,
			\]
			solange $a, b$ sonst nicht vorkommen.
		\item
			$\D^2 / \<w_1 w_2 \dotsc w_n\> \homeomorphic \D^2 / \<w_2 \dotsc w_n w_1\>$
		\item
			Es gilt
			\[
				\D^2 / \<\dotsc abb^{-1}c \dotsc \> = \D^2 / \< \dotsc ac \dotsc \>
			\]
			solange $b$ sonst nicht vorkommt.
	\end{itemize}
\end{lem}

\begin{lem}
	Jede Kreuzhaube $\dotsc c \dotsc c \dotsc$ können wir gruppieren zu $\dotsc cc \dotsc$.
\end{lem}

\begin{lem}
	Es gilt $\D^2 / \<\dotsc xcc \dotsc \> = \D^2 / \< \dotsc ccx \dotsc \>$.
\end{lem}

\begin{lem}
	Jeder Henkel $\dotsc a  \dotsc b \dotsc a^{-1} \dotsc b^{-1} \dotsc$ kann gruppiert werden zu $\dotsc aba^{-1}b^{-1} \dotsc$.
\end{lem}

\begin{lem}
	$\D^2 / \<\dotsc aba^{-1}b^{-1}x\dotsc \> =  \D^2 / \< \dotsc x aba^{-1}b^{-1} \dotsc \>$.
\end{lem}

\begin{lem}
	$\D^2 / \<\dotsc ccaba^{-1}b^{-1}\dotsc \> =  \D^2 / \< \dotsc cc aabb \dotsc \>$.
\end{lem}

\begin{st}
	Mit obigen Umformungen kann jedes Flächenwort $w$ überführt werden in $w = c_1c_1 \dotsc c_k c_k a_1b_1a_1^{-1}b^{-1} \dotsc a_l b_l a_l^{-1} b_l^{-1}$.

	Für $k \ge 1$ kann dies weiter vereinfacht werden zu $w = c_1c_1 \dotsc c_{k'} c_{k'}$ mit $k' = k + 2l$.
\end{st}

\coursetimestamp{21}{01}{2014}

Flächen ohne Rand: $F_g^+$.
Flächen mit $r \ge 1$ Randkomponenten: $F_{g,r}^+, F_{g,r}^-$

\begin{st}[Klassifikation kompakter Flächen]
	Jede zusammenhängende kompakte Fläche $F$ (eventuelle mit Rand) ist homöomorph zu genau einem der obigen Modelle $F_{g,r}^\eps$.

	Genauer gilt mit $r$ Randkomponenten und Euler-Charakteristik $\chi$: ist $F$ orientierbar, so gilt $\eps = +, \chi(F) = 2 - 2g - r$, andernfalls gilt $\eps = -$ und $\chi(F) = 1 - g - r$.
	\begin{proof}
		Analog zum geschlossenen Fall.
	\end{proof}
\end{st}

Es gilt $F_g^+ \injto \R^3$ und $\F_{g,r}^{\pm} \injto \R^3$ für $r \ge 1$, aber $F_g^- \not\injto \R^3$. Warum?

\begin{st}
	Ist $S \subset \R^3$ eine zusammenhängende, kompakte Fläche.
	Ist $S$ berandet, so ist $\R^3 \setminus S$ zusammenhängend.
	Ist $S$ geschlossen, so gilt $\R^3 \setminus S = A \dunion B$ mit $A, B$ offen, zusammenhängend, $A$ unbeschränkt, $B$ beschränkt, $\boundary A = \boundary B = S$, also $\_A = A \cup S$, $\_B = B \cup S$.
	In diesem Fall ist $S$ orientierbar.
	\begin{proof}[im polyedralen Fall]
		Analog zum Satz von Jordan für $\S^1 \homeomorphic C \subset \R^2$.
		Dann kann $S$ orientiert werden durch den von $B$ nach außen zeigenden Normalenvektor.
	\end{proof}
\end{st}
