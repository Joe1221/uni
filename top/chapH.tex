\chapter{Abbildungsgrad und Topologie des \texorpdfstring{$\R^n$}{R\textasciicircum n}}



\section{Dei Umlaufzahl}


Wir wollen jeder stetigen Abbildung $\gamma: [0,1] \to \C^* := \C \setminus \{0\}$ mit $\gamma(0) = \gamma(1)$ eine Umlaufzahl zuordnen, später sogar allgemein $\S^n \to \R^n \setminus \{0\}$.

% fixme: picture

\begin{st}
	Sei $V = \R^d$ (oder ein normierter $\R$-Vektorraum).
	Für jede offene Menge $X \subset V$ gilt:
	jeder Weg $\gamma: [0,1] \to X$ ist homotop in $X$ relativ zu $\{0, 1\}$ zu einer polygonalen Abbildung $|w|: [0,1] \to X$.

	Je zwei polygonale Wege sind genau dann homotop in $X$ relativ der Endpunkte $\{0,1\}$, wenn sie polygonal homotop sind.
	Dabei heißen polygonale Wege $w, w'$ \emph{polygonal homotop in $X$}, wenn sie sich durch folgende elementare Homotopien ineinander überführen lassen:
	\[
		v_0 \dotsc v_{k-1} v_k v_{k+1} \dotsc v_n \approx v_0 \dotsc v_{k-1} v_{k+1} \dotsc v_n
	\]
	mit $[v_{k+1}, v_k, v_{k+1}] \subset X$.
	% fixme: picture
	\begin{proof}
		Wie in der simpliziale Approximation.
	\end{proof}
\end{st}

\begin{df}
	Für jeden polygonalen Weg $w = v_0 v_1 \dotsc v_n$ in $\C^*$ definieren wir
	\[
		\deg(w) := \f 1{2\pi} \sum_{k=1}^n \angle (v_{k-1}, v_k).
	\]
\end{df}

\begin{ex}
	Für $\Delta = [v_0, v_1, v_2]$ und $w = v_0 v_1 v_2 v_0$ gilt
	\begin{align*}
		0 \not\in \Delta & \implies \deg(w) = 0, \\
		0 \in \Int\Delta & \implies \deg(w) = \pm 1.
	\end{align*}
\end{ex}

\begin{lem}
	Für $w = v_0 \dotsc v_n$ mit $v_0 = v_n$ gilt $\deg(w) \in \Z$.
	\begin{proof}
		Anschaulich: Winkel werden addiert und summieren sich zu Vielfachen von $2\pi$.
	\end{proof}
\end{lem}

\begin{lem}
	Aus $w \approx w'$ folgt $\deg(w) = \deg(w')$.
\end{lem}

\begin{lem}
	Jeder Polygonzug $w= v_0 v_1 \dotsc v_n$ ist kürzbar bis alle Winkel $\angle (v_{k-1}, v_k)$ gleiches Vorzeichen haben.
	\begin{proof}
		Anschaulich klar.
		% fixme: picture
	\end{proof}
\end{lem}

\begin{st}
	Seien $w = v_0 v_1 \dotsc v_n$ und $w' = v_0' v_1' \dotsc v_{n'}'$ Polygonzüge mit $v_0 = v_0'$ und $v_n = v_{n'}'$.
	Genau dass gilt $w \approx w'$, wenn $\deg(w) = \deg(w')$.
	\begin{proof}
		Siehe Skizze: Unterteile, dann ist $w \approx w'$.
		% fixme: picture
	\end{proof}
\end{st}

