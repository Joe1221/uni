\chapter{Abbildungsgrad und Topologie des \texorpdfstring{$\R^n$}{R\textasciicircum n}}



\section{Die Umlaufzahl}


Wir wollen jeder stetigen Abbildung $\gamma: [0,1] \to \C^* := \C \setminus \{0\}$ mit $\gamma(0) = \gamma(1)$ eine Umlaufzahl zuordnen, später sogar allgemein $\S^n \to \R^n \setminus \{0\}$.

% fixme: picture

\begin{st}
	Sei $V = \R^d$ (oder ein normierter $\R$-Vektorraum).
	Für jede offene Menge $X \subset V$ gilt:
	jeder Weg $\gamma: [0,1] \to X$ ist homotop in $X$ relativ zu $\{0, 1\}$ zu einer polygonalen Abbildung $|w|: [0,1] \to X$.

	Je zwei polygonale Wege sind genau dann homotop in $X$ relativ der Endpunkte $\{0,1\}$, wenn sie polygonal homotop sind.
	Dabei heißen polygonale Wege $w, w'$ \emph{polygonal homotop in $X$}, wenn sie sich durch folgende elementare Homotopien ineinander überführen lassen:
	\[
		v_0 \dotsc v_{k-1} v_k v_{k+1} \dotsc v_n \approx v_0 \dotsc v_{k-1} v_{k+1} \dotsc v_n
	\]
	mit $[v_{k+1}, v_k, v_{k+1}] \subset X$.
	% fixme: picture
	\begin{proof}
		Wie in der simpliziale Approximation.
	\end{proof}
\end{st}

\begin{df}
	Für jeden polygonalen Weg $w = v_0 v_1 \dotsc v_n$ in $\C^*$ definieren wir
	\[
		\deg(w) := \f 1{2\pi} \sum_{k=1}^n \angle (v_{k-1}, v_k).
	\]
\end{df}

\begin{ex}
	Für $\Delta = [v_0, v_1, v_2]$ und $w = v_0 v_1 v_2 v_0$ gilt
	\begin{align*}
		0 \not\in \Delta & \implies \deg(w) = 0, \\
		0 \in \Int\Delta & \implies \deg(w) = \pm 1.
	\end{align*}
\end{ex}

\begin{lem}
	Für $w = v_0 \dotsc v_n$ mit $v_0 = v_n$ gilt $\deg(w) \in \Z$.
	\begin{proof}
		Anschaulich: Winkel werden addiert und summieren sich zu Vielfachen von $2\pi$.
	\end{proof}
\end{lem}

\begin{lem}
	Aus $w \approx w'$ folgt $\deg(w) = \deg(w')$.
\end{lem}

\begin{lem}
	Jeder Polygonzug $w= v_0 v_1 \dotsc v_n$ ist kürzbar bis alle Winkel $\angle (v_{k-1}, v_k)$ gleiches Vorzeichen haben.
	\begin{proof}
		Anschaulich klar.
		% fixme: picture
	\end{proof}
\end{lem}

\begin{st}
	Seien $w = v_0 v_1 \dotsc v_n$ und $w' = v_0' v_1' \dotsc v_{n'}'$ Polygonzüge mit $v_0 = v_0'$ und $v_n = v_{n'}'$.
	Genau dass gilt $w \approx w'$, wenn $\deg(w) = \deg(w')$.
	\begin{proof}
		Siehe Skizze: Unterteile, dann ist $w \approx w'$.
		% fixme: picture
	\end{proof}
\end{st}

\coursetimestamp{17}{12}{2013}

Alternative Betrachtung der Umlaufzahl
\[
	\eta(u, v) := \begin{cases}
		0 & [u,v] \cap \R_{<0} = \emptyset \\
		\f 12 \big( \sign \Im(u) - \sign \Im(v) \big) & \text{sonst}
	\end{cases}.
\]
dann ergibt sich $\deg(v_0 \dotsc v_n) = \sum_{k=1}^n \eta(v_{k-1}, v_k)$.

\begin{nt}
	Für $\gamma: [0,1] \to \C^*$ stetig und stückweise $\scr C^1$ mit $\gamma(0) = \gamma(1)$ gilt
	\[
		\deg(\gamma) = \f 1{2\pi i} \int_{t=0}^1 \f {\gamma'(t)}{\gamma(t)} \dx[t].
	\]
	Reell betrachten wir das Vektorfeld $f: \R^2 \setminus \{0\} \to \R^2 \to \{0\}$ mit
	\[
		f(x,y) := \f 1{x^2 + y^2} (-y, x).
	\]
	Es gilt $\rot(f) = 0$, aber $f$ besitzt kein Potential.
	Es gilt
	\[
		\deg(\gamma) = \f 1{2\pi} \int_{t=0}^1 f(\gamma(t)) \gamma'(t) \dx[t]
	\]
\end{nt}


\section{Der Abbildungsgrad}


Für $k \in \Z$ haben wir $\phi_k: \S^1 \to \S^1, \phi(z) = z^k$.
Reell betrachtet gilt
\[
	\phi_k(\cos t, \sin t) = (\cos kt, \sin kt).
\]
Allgemeiner definieren wir $\phi_k: \R^{n+1} \to \R^{n+1}$ durch
\[
	\phi_k(r \cos t, r \sin t, x_3, \dotsc, x_{n+1})
	:= (r\cos kt, r\sin kt, x_3, \dotsc, x_{n+1}).
\]

\begin{st}[Brouwer-Hopf]
	Die Abbildung $\Z \to [\S^n, \S^n], k \mapsto [\phi_k]$ (d.h. von $\Z$ zur Homotopieklasse aller Abbildungen von $\S^n$ nach $\S^n$) ist bijektiv.
	Es gibt also eine inverse Bijektion
	\[
		\deg: [\S^n, \S^n] \to \Z,
		\deg(\phi_k) = k.
	\]
	\begin{proof}[Beweis-Idee]
		Simpliziale Approximation und Umläufe zählen für $n = 1$.
	\end{proof}
\end{st}

\begin{ex}
	Für jede Matrix $A \in \GL_{n+1}(\R)$ definiere $f_A: \S^n \to \S^n$ durch
	$f_A(x) := \f {Ax}{|Ax|}$ (stetig).
	Für
	\[
		E_\pm = \diag(1, \pm 1, 1, \dotsc, 1)
	\]
	ist $f_{E_\pm} = \phi_{\pm 1}$.
	Also ist $\deg(f_{E_{\pm 1}})  = \pm 1 = \sign \det (E_\pm)$.
\end{ex}

\begin{prop}
	Für $A \in \GL_{n+1} (\R)$ gilt $\deg(f_A) = \sign \det(A)$.
	\begin{proof}
		Es gilt $\pi_0(\GL_{n+1}(\R)) = \{ [E_+], [E_-] \}$.
		Jeder Weg $\gamma: [0,1] \to \GL_{n+1}(\R)$ von $A$ nach $E_\pm$ definiert eine Homotopie von $f_A$ nach $f_{E_\pm}$ vermöge $H(t, x) = \f {\gamma(t) x}{|\gamma(t) x |}$.
		Also gilt $\deg(f_A) = \deg(f_{E_\pm}) = \sign \det A$.
	\end{proof}
\end{prop}

\begin{kor}
	$\deg(f\circ g) = \deg(f) \deg(g)$.
	\begin{proof}
		Es gilt $f \homotopic \phi_k, g \homotopic \phi_l$, also $f \circ g \homotopic \phi_k \circ \phi_l = \phi_{kl}$.
	\end{proof}
\end{kor}

\begin{kor}
	Ist $f: \S^n \to \S^n$ ein Homöomorphismus, so gilt $\deg(f) = \pm 1$.
\end{kor}

\begin{kor} \label{sphere_non-contractible}
	Es gilt $\S^n \not\homotopic *$.
	\begin{proof}
		Es gilt $[S^n, *] = \{*\}$, $[\S^n, \S^n] \isomorphic \Z$.
		Es existiert keine Bijektion zwischen $\{*\}$ und $\Z$, also ist $* \not\homotopic \S^n$.
	\end{proof}
\end{kor}

\begin{kor}
	Es gilt $\S^n \subset \D^{n+1}$ ist kein Retrakt.
	\begin{proof}
		Sei $\jota: \S^n \injto \D^{n+1}$ die Inklusion.
		Angenommen es gäbe $r: \D^{n+1} \to \S^n$ mit $r \circ \jota = \Id_{\S^n}$.
		\[
			\begin{tikzcd}[column sep=small]
				~& \D^{n+1} \arrow{dr}{r} &\\
				\S^n \arrow{rr}{\Id \homotopic *}\arrow{ur}{\jota\homotopic *} & & \S^n
			\end{tikzcd}
			\!\!\xRightarrow{[\S^n, -]}\!\!\!\!
			\begin{tikzcd}[column sep=tiny]
				~& {[\S^n, \D^n] = \{*\}} \arrow{dr}{r_*} & \\
				{\Z \homeomorphic [\S^n, \S^n]} \arrow{ur}{\jota_*} \arrow{rr}{\Id_*} & & {[\S^n, \S^n] \homeomorphic \Z}
			\end{tikzcd}
			% fixme: display issue
		\]
		ein Widerspruch zu $\S^n \not\homotopic *$, bzw. zu $\Z \not\homeomorphic \{*\}$.
	\end{proof}
\end{kor}

\begin{st}
	Sei $A \subset \R^n$ kompakt und $\mathring A \neq \emptyset$.
	Dann existiert keine Retraktion $r: A \to \boundary A$.
	\begin{proof}
		Angenommen $r: A \to \boundary A$ wäre eine Retraktion, also stetig und $r|_{\boundary A} = \Id_{\boundary A}$.
		Wir können \oBdA annehmen, dass $0 \in \mathring A \subset A \subset \B^n$.
		Betrachte $g: \D^n \to \S^{n-1}$ mit
		\[
			g(x) := \begin{cases}
				\f x{|x|} & x \in \D^n \setminus \mathring A \\
				\f {r(x)}{|r(x)|} & x \in A
			\end{cases}.
		\]
		$g$ ist wohldefiniert, stetig und erfüllt $g|_{\S^{n-1}} = \Id_{\S^{n-1}}$, ein Widerspruch zu $\S^{n+1} \subset \D^n$ ist kein Retrakt.
	\end{proof}
\end{st}

\begin{st}[Brouwerscher Fixpunktsatz]
	Jede stetige Abbildung $f: \D^n \to \D^n$ besitzt (mindestens) einen Fixpunkt, d.h. es existiert $a \in \D^n$ mit $f(a) = a$.
	\begin{proof}
		Siehe Skizze
		% fixme picture
	\end{proof}
\end{st}


\section{Satz vom Igel}


\begin{df}
	Ein \emph{tangentiales Vektorfeld} auf $\S^n$ ist eine Abbildung $v: \S^n \to \R^{n+1}$ mit $\<x, v(x)\> = 0$.
\end{df}

\begin{st}[Der Satz vom gekämmten Igel]
	\begin{enumerate}[(1)]
		\item
			Ist $n$ ungerade, so existiert ein nirgends verschwindendes Vektorfeld auf $\S^n$
		\item
			Für $n$ gerade, hat jedes tangentiale Vektorfeld mindestens eine Nullstelle.
	\end{enumerate}
	\begin{proof}
		\begin{enumerate}[(1)]
			\item
				Sei $n = 2m - 1$ und betrachte
				$v: \R^{n+1} \to \R^{n+1}$ definiert durch
				\[
					v(x_1, x_2, \dotsc, x_{2m-1}, x_{2m})
					:= (-x_2, x_1, \dotsc, -x_{2m}, x_{2m-1}).
				\]
				Dies definiert $v: \S^n \to \R^{n+1}$ mit $\<x, v(x)\> = 0$.
			\item
				Zu $v: \S^n \to \R^{n+1}$ betrachte
				$H: [0,1] \times \S^n \to \S^n$ definiert durch
				\[
					H_t(x) := \f {x+ tv(x)}{|x+tv(x)|}.
				\]
				Wenn $v$ nirgends verschwindet, dann ist $H_1(x) \neq x$ für alle $x$.
				Wir betrachten $K: [0,1] \times \S^n \to \S^n$, definiert durch
				\[
					K_t(x) := \f {(1-t) H_1(x) - tx}{|(1-t) H_1(x) - tx|}.
				\]
				$K_t$ ist wohldefiniert wegen $H_1(x) \neq x$ und stetig.
				Es gilt $K_0 = H_1, K_1 = -\Id$.
				Dann gilt $\Id_{\S^n} \stack{H}\homotopic H_1 \stack{K}\homotopic -\Id_{\S^n}$.
				Es gilt aber
				\begin{align*}
					\deg(\Id_{\S^n}) &= 1, &
					\deg(-\Id_{\S^n}) &= (-1)^{n+1} = - 1.
				\end{align*}
				Ein Widerspruch zur Homotopieinvarianz des Abbildungsgrad.
		\end{enumerate}
	\end{proof}
\end{st}


\section{Der Satz von Borsuk-Ulam}

Wir beobachten: ist $k$ ungerade, so gilt $\phi_k(-x) = -\phi_k(x)$.
Gilt nun auch die Umkehrung?

\begin{st}[Borsuk-Ulam]
	Sei $f: \S^n \to \S^n$ stetig mit $f(-x) = -f(x)$.
	Dann ist $\deg(f)$ ungerade.
	\begin{proof}
		Folgt
	\end{proof}
\end{st}

\coursetimestamp{07}{01}{2014}

\begin{kor}
	Für $m > n$ existiert keine stetige Abbildung $f: \S^m \to \S^n$ mit $f(-x) = -f(x)$.
	\begin{proof}
		Angenommen $f: \S^m \to \S^n$ sei stetig und ungerade.
		Auch die Standardeinbettung $\jota: \S^n \injto \S^m$ ist ungerade.
		Somit ist auch $g = \jota \circ f: \S^m \to \S^m$ ungerade.
		Wegen $\deg(g)$ ungerade und damit insbesondere $\deg(g) \neq 0$ gilt $g \not\homotopic *$.
		Andererseits ist $\jota$ zusammenziehbar und damit auch $g = \jota \circ f \homotopic * \circ f = *$, ein Widerspruch.
	\end{proof}
\end{kor}

\begin{kor}
	Ist $f: \S^n \to \R^n$ stetig, so existiert $x \in \S^n$ mit $f(-x) = f(x)$.
	\begin{proof}
		Der Fall $n = 0$ ist trivial ($\S^0 = \{\pm 1\}, \R^0 = \{0\}$).
		Der Fall $n = 1$ ist leicht mit dem Zwischenwertsatz (betrachte $g(x) = f(x) - f(-x)$).

		Angenommen $f(x) \neq f(-x)$ für alle $x \in \S^n$.
		Dann wäre $g(x) = \f {f(x) - f(-x)}{|f(x) - f(-x)|}$ eine stetig ungerade Abbildung $g: \S^n \to \S^{n-1}$, ein Widerspruch zum letzten Korollar.
	\end{proof}
\end{kor}

\begin{st}
	Seien $A_1, \dotsc, A_n \subset \R^n$ kompakt (es reicht auch messbar und beschränkt).
	Dann existiert mindestens eine affine Hyperebene $H \subset \R^n$ sodass $\my(A_k \cap H_+) = \my(A_k \cap H_-)$ für alle $k = 1, \dotsc, n$.
	\begin{proof}
		Setze $\my_k(B) := \my(A_k \cap B)$ und nutze den folgenden Satz.
	\end{proof}
\end{st}

\begin{st}
	Seien $\my_1, \dotsc, \my_n$ endliche Borel-Maße, wobei jede affine Hyperebene $H$ Maß 0 hat.
	Dann existiert $H \subset \R^n$ sodass $\my_k(H_+) = \my_k(H_-)$ für alle $k$.
	\begin{proof}
		Jeder Vektor $v = (v_0, \dotsc, v_n) \in \S^n$ definiert Halbräume
		\begin{align*}
			H_+(v) &:= \{x \in \R^n : v_1x_1 + \dotsb + v_n x_n > v_0 \}, \\
			H_-(v) &:= \{x \in \R^n : v_1x_1 + \dotsb + v_n x_n < v_0 \}.
		\end{align*}
		Hierbei gilt $H_+(-v) = H_-(v)$.
		Für $v = (1, 0, \dotsc, 0)$ gilt $H_+(v) = \emptyset, H_-(v) = \R^n$.
		Wir definieren $f_k: \S^n \to \R$ durch $f_k(v) = \my_k(H_+(v))$ und erhalten $f: \S^n \to \R^n, f = (f_1, \dotsc, f_n)$.
		Dies Abbildung ist stetig (Übung).
		Dann existiert $v \in \S^n$ mit $f(v) = f(-v)$ nach letztem Korollar, also $\my_k(H_+(v)) = \my_k(H_-(v))$ für alle $k = 1, \dotsc, n$.
	\end{proof}
\end{st}


\section{Invarianz der Dimension, des Randes und der Orientierung}


\begin{st}
	Für $m \neq n$ gilt $\S^m \not\homotopic \S^n$ und insbesondere $\S^m \not\homeomorphic \S^n$.
	\begin{proof}
		Sei \oBdA $m > n$.
		Angenommen $\S^m \homotopic \S^n$, d.h.
		\[
			\begin{tikzcd}
				\S^m \arrow{r}{f} \arrow{d}[swap]{\Id_{\S^m}} & \S^n \arrow{d}{\Id_{\S^n}} \arrow{dl}[swap]{g} \\
				\S^m \arrow{r}{f} & \S^n
			\end{tikzcd}.
		\]
		Dann gilt $g \homotopic *$ nach \ref{spheres_function_null-homotopic}, also auch $\Id_{\S^m} \homotopic g \circ f \homotopic *$, ein Widerspruch zu $\S^m$ nicht zusammenziehbar (siehe \ref{sphere_non-contractible}).
	\end{proof}
\end{st}

\begin{kor}
	Für $m \neq n$ gilt $\R^m \not\homeomorphic \R^n$.
	\begin{proof}
		Angenommen $f$ wäre eine Homöomorphismus $\R^m \homeomorphicto \R^n$.
		Im Fall $m = 0$ folgt sofort $n = 0$, sei also $m, n \ge 1$.
		Sei $x \in \R^m, y = f(x) \in \R^n$.
		Dann erhalten wir
		\[
			\S^{m-1} \homotopic \R^m \setminus \{x\} \xrightarrow[\homeomorphic]{f|_{\R^m \setminus \{x\}}^{\R^n \setminus \{y\}}} \R^n \setminus \{y\} \homotopic \S^{n-1}
		\]
		also $\S^{m-1} \homotopic \S^{n-1}$ und somit $m = n$.

		Alternativ lässt dies auch mit der Alexandroff-Kompaktifizierung beweisen: Widerspruch mittels des fortgesetzten Homömomorphismus $\hat f: \S^m \homeomorphicto \S^n$.
	\end{proof}
\end{kor}

\begin{st}
	Sei $0 \le m < n$.
	Sei $U \subset \R^n$ offen, $x \in U$.
	\begin{enumerate}[(1)]
		\item
			In jede Umgebung $U$ von $x$ in $U$ existieren stetige Abbildungen $f: \S^{n-1} \to U' \setminus \{0\}$, die nicht zusammenziehbar sind.
			Kurz: $[\S^{n-1}, U' \setminus \{0\}] \neq \{*\}$.
		\item
			Hingegen existieren Umgebungen $U'$ von $x$ in $U$, sodass $[\S^{m-1}, U' \setminus \{0\}] = \{*\}$.
	\end{enumerate}
	Hierdurch wird $n$ allein durch topologische Eigenschaften des Raumes $U$ festgelegt.
	\begin{proof}
		Es existiert $r \in \R_{>0}$ mit $B(x, 2r) \subset U$.
		Wir erhalten $f = f_{x,r} : \S^{n-1} \injto U \setminus \{x\}$ mit $f(s) = x + rs$.
		Wäre $f$ nullhomotop in $U$, so auch die Komposition
		\[
			\S^{n-1} \stack{f}\injto U \setminus \{x\} \xrightarrow{\jota} \R^n \setminus \{x\} \stack{\homeomorphic}\to \R^n \setminus \{0\} \stack{\homotopic}\to \S^{n-1}
		\]
		Das steht im Widerspruch zu $\deg(\Id_{\S^{n-1}}) = 1$.
		Hingegen ist jede stetig Abbildung $f: \S^{m-1} \to B(x, r) \setminus \{0\}$ zusammenziehbar, denn
		\begin{align*}
			\S^{m-1} \stack{f}\to B(x,r) \setminus \{x\} & \xleftarrow{\homeomorphic} \R^n \setminus \{0\} \homotopic \S^{n-1} \\
			x + r \f {v}{1+|v|} & \mapsfrom v
		\end{align*}
	\end{proof}
\end{st}

\begin{st}[Invarianz der Dimension]
	Sei $U \subset \R^p$ offen und $V \subset \R^q$ offen und nicht-leer.
	Wenn ein Homeomorphismus $f: U \stack{\homeomorphic}\to V$ existiert, so gilt $p = q$.
\end{st}

\coursetimestamp{13}{01}{2014}

\section{Jordan-Schoenflies}

\begin{st}[Jordan-Schoenflies]
	Sei $C \subset \R^2$ eine Jordan-Kurve, d.h. $C \homeomorphic \S^1$.
	\begin{enumerate}[1)]
		\item
			Jordan-Zerlegung: $\R^2 \setminus C = A \dunion B$ mit $A, B$ offen und wegzusammenhängend, $B$ beschränkt und $A$ unbeschränkt.
		\item
			$\boundary A = \boundary B = C$, $\_A = A \cup C$, $\_B = B\cup C$.
		\item
			Schoenfließ: $B \homeomorphic \B^2$, $\_B \homeomorphic \D^2$.
		\item
			Verschärfung: $\exists h: \R^2 \homeomorphicto \R^2: h(C) = \S^1$.
	\end{enumerate}
	\begin{proof}[für polygonale Kurven $\S^1 \homeomorphic C \subset \R^2$]
		Sei $C = [v_0, v_1] \cup [v_1, v_2] \cup \dotsb \cup [v_{n-1}, v_n]$ mit $v_n = v_0$ und $[v_{i-1}, v_i[ \cap [v_{j-1} v, v_j[ = \emptyset$ für alle $i \neq j$.

		1) folgt aus den Lemmas, 2) Übung.
	\end{proof}
\end{st}

\begin{lem}
	$\R^2 \setminus C$ hat mindestens zwei Komponenten.
	\begin{proof}
		Es gibt $f: \R^2 \setminus C \to \{0,1\}$ stetig und surjektiv:
		\[
			f(x) = \deg(C - x) \mod 2.
		\]
		$f$ ist lokal konstant und damit insbesondere stetig.
		$f$ ist surjektiv, beide Werte werden angenommen.
	\end{proof}
\end{lem}

\begin{lem}
	$\R^2 \setminus C$ hat höchstens zwei Komponenten.
	\begin{proof}
		Wähle
		\[
			\eps < \delta(C) := \f 12 \min \Set{ d([v_{i-1}, v_i], [v_{j-1}, v_j]]) | [v_{i-1}, v_i] \cap [v_{j-1}, v_j] = \emptyset }.
		\]
		Sei $U := \{ x \in \R^2 : d(x, C) \le \eps^2 \}$.
		Es gilt $U \homeomorphic C \times [-1, 1]$ (Zylinder), oder $U \homeomorphic$ Möbius-Band, wobei das Möbius-Band jedoch nicht in Frage kommt (letztes Lemma).

		Wähle $a, b \in \R^2$ links und rechts von der Kurve in $U$.
		Jeder Punkt $x \in \R^2$ lässt sich jetzt entlang von $U$ ohne $C$ zu schneiden entweder zu $a$ oder $b$ verbinden.
	\end{proof}
\end{lem}

\begin{lem}
	$\_B = B \cup C$ lässt sich triangulieren.
	\begin{proof}
		Zunächst bilden die Halbebenenschnitte entlang der Kanten eine polytopale Unterteilung und diese lässt sich durch baryzentrische Unterteilung in eine simpliziale Unterteilung überführen.
	\end{proof}
\end{lem}

\begin{df}
	Sei $T$ eine Triangulierung von $\_B$.
	Ein Dreieck $\triangle = [a,b,c] \in T$ heißt \emph{einklappbar}, wenn $\triangle \cap C = [a,b]$ oder $\triangle \cap C = [a,b] \cup [b,c]$.
\end{df}

\begin{lem}
	Zu jedem einklappbarem Dreieck $\triangle$ existiert $h: \R^2 \homeomorphicto \R^2$ mit $h(\_B) = \_{B \setminus \triangle}$.
\end{lem}

\begin{lem}
	Hat $T$ mindestens $2$ Dreiecke, so mindestens $2$ einklappbare.
	\begin{proof}
		Zeige per Induktion über $n = $ die Anzahl der Dreiecke.
		Für $n = 2$ ist die Aussage klar.
		Sei $n \ge 3$.
		Jede Kante von $C$ liegt in einem Dreieck $\triangle = [a,b,c] \in T$.
		Sei $\triangle$ \oBdA nicht einklappbar.
		Teile auf in linke und rechte Seite und wende Induktionsvoraussetzung auf die kleineren beiden Teile an, so findet man jeweils ein einklappbares Dreieck $\triangle'$, bzw. $\triangle''$, die nicht $\triangle$ sind.
		Diese sind auch einklappbar im ursprünglichen $\_B$.
	\end{proof}
\end{lem}

\coursetimestamp{14}{01}{2014}

\begin{st}[Jordan-Brouwer-Trennungssatz]
	Für $\S^{n-1} \homeomorphic S \subset \R^n$ hat das Komplement $\R^n \setminus S$ genau zwei Komponenten:
	\[
		\R^n \setminus S
		= A \dunion B
	\]
	mit $A, B$ offen, zusammenhängend, $B$ beschränkt, $A$ unbeschränkt.
\end{st}

\begin{st}[Alexander]
	Ist $A$ kompakt, so haben für je zwei Einbettungen $f, g: A \injto \R^n$ die Komplemente $\R^n \setminus f(A)$ und $\R^n \setminus g(A)$ gleich viele Komponenten.
\end{st}

\begin{st}[Invarianz des Gebiets]
	Sei $U \subset \R^n$ offen und $f: U \to \R^n$ injektiv und stetig.
	Dann ist die Bildmenge $V = f(U)$ offen in $\R^n$ und $f: U \to V$ ist ein Homöomorphismus.
	\begin{proof}
		Für $n = 1$ ist dies leicht zu zeigen mit dem Zwischenwertsatz.

		Wir zeigen, dass $f$ offen ist in jedem Punkt $x \in U$.
		Sei $r > 0$ mit $B(x, 2r) \subset U$ und $S := \boundary B(x,r) \homeomorphic \S^{n-1}$.
		Dann trennt $C := f(S) \homeomorphic \S^{n-1}$ den Raum $\R^n$ in zwei Komponenten $\R^n \setminus C = A \dunion B$ wie in Jordan-Brouwer-Trennungssatz % fixme: ref
		Auch $B' = f(B(x,r))$ ist zusammenhängend, ebenso $A' = \R^n \setminus f(\_B(x,r))$ (Satz von Alexander).
		Also $\R^n \setminus C = A' \dunion B'$.
		Daraus folgt $A = A'$ und $B = B'$.
		Insbesondere ist $f(B(x,r))$ in $\R^n$ offen.
	\end{proof}
\end{st}
