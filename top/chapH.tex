\chapter{Abbildungsgrad und Topologie des \texorpdfstring{$\R^n$}{R\textasciicircum n}}



\section{Die Umlaufzahl}


Wir wollen jeder stetigen Abbildung $\gamma: [0,1] \to \C^* := \C \setminus \{0\}$ mit $\gamma(0) = \gamma(1)$ eine Umlaufzahl zuordnen, später sogar allgemein $\S^n \to \R^n \setminus \{0\}$.

% fixme: picture

\begin{st}
	Sei $V = \R^d$ (oder ein normierter $\R$-Vektorraum).
	Für jede offene Menge $X \subset V$ gilt:
	jeder Weg $\gamma: [0,1] \to X$ ist homotop in $X$ relativ zu $\{0, 1\}$ zu einer polygonalen Abbildung $|w|: [0,1] \to X$.

	Je zwei polygonale Wege sind genau dann homotop in $X$ relativ der Endpunkte $\{0,1\}$, wenn sie polygonal homotop sind.
	Dabei heißen polygonale Wege $w, w'$ \emph{polygonal homotop in $X$}, wenn sie sich durch folgende elementare Homotopien ineinander überführen lassen:
	\[
		v_0 \dotsc v_{k-1} v_k v_{k+1} \dotsc v_n \approx v_0 \dotsc v_{k-1} v_{k+1} \dotsc v_n
	\]
	mit $[v_{k+1}, v_k, v_{k+1}] \subset X$.
	% fixme: picture
	\begin{proof}
		Wie in der simpliziale Approximation.
	\end{proof}
\end{st}

\begin{df}
	Für jeden polygonalen Weg $w = v_0 v_1 \dotsc v_n$ in $\C^*$ definieren wir
	\[
		\deg(w) := \f 1{2\pi} \sum_{k=1}^n \angle (v_{k-1}, v_k).
	\]
\end{df}

\begin{ex}
	Für $\Delta = [v_0, v_1, v_2]$ und $w = v_0 v_1 v_2 v_0$ gilt
	\begin{align*}
		0 \not\in \Delta & \implies \deg(w) = 0, \\
		0 \in \Int\Delta & \implies \deg(w) = \pm 1.
	\end{align*}
\end{ex}

\begin{lem}
	Für $w = v_0 \dotsc v_n$ mit $v_0 = v_n$ gilt $\deg(w) \in \Z$.
	\begin{proof}
		Anschaulich: Winkel werden addiert und summieren sich zu Vielfachen von $2\pi$.
	\end{proof}
\end{lem}

\begin{lem}
	Aus $w \approx w'$ folgt $\deg(w) = \deg(w')$.
\end{lem}

\begin{lem}
	Jeder Polygonzug $w= v_0 v_1 \dotsc v_n$ ist kürzbar bis alle Winkel $\angle (v_{k-1}, v_k)$ gleiches Vorzeichen haben.
	\begin{proof}
		Anschaulich klar.
		% fixme: picture
	\end{proof}
\end{lem}

\begin{st}
	Seien $w = v_0 v_1 \dotsc v_n$ und $w' = v_0' v_1' \dotsc v_{n'}'$ Polygonzüge mit $v_0 = v_0'$ und $v_n = v_{n'}'$.
	Genau dass gilt $w \approx w'$, wenn $\deg(w) = \deg(w')$.
	\begin{proof}
		Siehe Skizze: Unterteile, dann ist $w \approx w'$.
		% fixme: picture
	\end{proof}
\end{st}

\coursetimestamp{17}{12}{2013}

Alternative Betrachtung der Umlaufzahl
\[
	\eta(u, v) := \begin{cases}
		0 & [u,v] \cap \R_{<0} = \emptyset \\
		\f 12 \big( \sign \Im(u) - \sign \Im(v) \big) & \text{sonst}
	\end{cases}.
\]
dann ergibt sich $\deg(v_0 \dotsc v_n) = \sum_{k=1}^n \eta(v_{k-1}, v_k)$.

\begin{nt}
	Für $\gamma: [0,1] \to \C^*$ stetig und stückweise $\scr C^1$ mit $\gamma(0) = \gamma(1)$ gilt
	\[
		\deg(\gamma) = \f 1{2\pi i} \int_{t=0}^1 \f {\gamma'(t)}{\gamma(t)} \dx[t].
	\]
	Reell betrachten wir das Vektorfeld $f: \R^2 \setminus \{0\} \to \R^2 \to \{0\}$ mit
	\[
		f(x,y) := \f 1{x^2 + y^2} (-y, x).
	\]
	Es gilt $\rot(f) = 0$, aber $f$ besitzt kein Potential.
	Es gilt
	\[
		\deg(\gamma) = \f 1{2\pi} \int_{t=0}^1 f(\gamma(t)) \gamma'(t) \dx[t]
	\]
\end{nt}


\section{Der Abbildungsgrad}


Für $K \in \Z$ haben wir $\phi_k: \S^1 \to \S^1, \phi(z) = z^k$.
Reell betrachtet gilt
\[
	\phi_k(\cos t, \sin t) = (\cos kt, \sin kt).
\]
Allgemeiner definieren wir $\phi_k: \R^{n+1} \to \R^{n+1}$ durch
\[
	\phi_k(r \cos t, r \sin t, x_3, \dotsc, x_{n+1})
	:= (r\cos kt, r\sin kt, x_3, \dotsc, x_{n+1}).
\]

\begin{st}[Brouwer-Hopf]
	Die Abbildung $\Z \to [\S^n, \S^n], k \mapsto [\phi_k]$ (d.h. von $\Z$ zur Homotopieklasse aller Abbildungen von $\S^n$ nach $\S^n$ ist bijektiv.
	Es gibt also eine inverse Bijektion
	\[
		\deg: [\S^n, \S^n] \to \Z,
		\deg(\phi_k) = k.
	\]
	\begin{proof}[Beweis-Idee]
		Simpliziale Approximation und Umläufe zählen für $n = 1$.
	\end{proof}
\end{st}

\begin{ex}
	Für jede Matrix $A \in \GL_{n+1}(\R)$ definiere $f_A: \S^n \to \S^n$ durch
	$f_A(x) := \f {Ax}{|Ax|}$ (stetig).
	Für
	\[
		E_\pm = \diag(1, \pm 1, 1, \dotsc, 1)
	\]
	ist $f_{E_\pm} = \phi_{\pm 1}$.
	Also ist $\deg(f_{E_{\pm 1}})  = \pm 1 = \sign \det (E_\pm)$.
\end{ex}

\begin{prop}
	Für $A \in \GL_{n+1} (\R)$ gilt $\deg(f_A) = \sign \det(A)$.
	\begin{proof}
		Es gilt $\pi_0(\GL_{n+1}(\R)) = \{ [E_+], [E_-] \}$.
		Jeder Weg $\gamma: [0,1] \to \GL_{n+1}(\R)$ von $A$ nach $E_\pm$ definiert eine Homotopie von $f_A$ nach $f_{E_\pm}$ vermöge $H(t, x) = \f {\gamma(t) x}{|\gamma(t) x |}$.
		Also gilt $\deg(f_A) = \deg(f_{E_\pm}) = \sign \det A$.
	\end{proof}
\end{prop}

\begin{kor}
	$\deg(f\circ g) = \deg(f) \deg(g)$.
	\begin{proof}
		Es gilt $f \homotopic \phi_k, g \homotopic \phi_l$, also $f \circ g \homotopic \phi_k \circ \phi_l = \phi_{kl}$.
	\end{proof}
\end{kor}

\begin{kor}
	Ist $f: \S^n \to \S^n$ ein Homöomorphismus, so gilt $\deg(f) = \pm 1$.
\end{kor}

\begin{kor}
	Es gilt $\S^n \not\homotopic *$.
	\begin{proof}
		Es gilt $[S^n, *] = \{*\}$, $[\S^n, \S^n] \isomorphic \Z$.
		Es existiert keine Bijektion zwischen $\{*\}$ und $\Z$, also ist $* \not\homotopic \S^n$.
	\end{proof}
\end{kor}

\begin{kor}
	Es gilt $\S^n \subset \D^{n+1}$ ist kein Retrakt.
	\begin{proof}
		Sei $\jota: \S^n \injto \D^{n+1}$ die Inklusion.
		Angenommen es gäbe $r: \D^{n+1} \to \S^n$ mit $r \circ \jota = \Id_{\S^n}$.
		\[
			\begin{tikzcd}[column sep=small]
				~& D^{n+1} \arrow{dr}{r} &\\
				\S^n \arrow{rr}{\Id \homotopic *}\arrow{ur}{\jota\homotopic *} & & \S^n
			\end{tikzcd}
			\!\!\xRightarrow{[\S^n, -]}\!\!\!\!
			\begin{tikzcd}[column sep=tiny]
				~& {[\S^n, \D^n] = \{*\}} \arrow{dr}{r_*} & \\
				{\Z \homeomorphic [\S^n, \S^n]} \arrow{ur}{\jota_*} \arrow{rr}{\Id_*} & & {[\S^n, \S^n] \homeomorphic \Z}
			\end{tikzcd}
			% fixme: display issue
		\]
		ein Widerspruch zu $\S^n \not\homotopic *$, bzw. zu $\Z \not\homeomorphic \{*\}$.
	\end{proof}
\end{kor}

\begin{st}
	Sei $A \subset \R^n$ kompakt und $\mathring A \neq \emptyset$.
	Dann existiert keine Retraktion $r: A \to \boundary A$.
	\begin{proof}
		Angenommen $r: A \to \boundary A$ wäre eine Retraktion, also stetig und $r|_{\boundary A} = \Id_{\boundary A}$.
		Wir können \oBdA annehmen, dass $0 \in \mathring A \subset A \subset \B^n$.
		Betrachte $g: \D^n \to \S^{n-1}$ mit
		\[
			g(x) := \begin{cases}
				\f x{|x|} & x \in \D^n \setminus \mathring A \\
				\f {r(x)}{|r(x)|} & x \in A
			\end{cases}.
		\]
		$g$ ist wohldefiniert, stetig und erfüllt $g|_{\S^{n-1}} = \Id_{\S^{n-1}}$, ein Widerspruch zu $\S^{n+1} \subset \D^n$ ist kein Retrakt.
	\end{proof}
\end{st}

\begin{st}[Brouwerscher Fixpunktsatz]
	Jede stetige Abbildung $f: \D^n \to \D^n$ besitzt (mindestens) einen Fixpunkt, d.h. es existiert $a \in \D^n$ mit $f(a) = a$.
	\begin{proof}
		Siehe Skizze
		% fixme picture
	\end{proof}
\end{st}


\section{Satz vom Igel}


\begin{df}
	Ein \emph{tangentiales Vektorfeld} auf $\S^n$ ist eine Abbildung $v: \S^n \to \R^{n+1}$ mit $\<x, v(x)\> = 0$.
\end{df}

\begin{st}[Der Satz vom gekämmten Igel]
	\begin{enumerate}[(1)]
		\item
			Ist $n$ ungerade, so existiert ein nirgends verschwindendes Vektorfeld auf $\S^n$
		\item
			Für $n$ gerade, hat jedes tangentiale Vektorfeld mindestens eine Nullstelle.
	\end{enumerate}
	\begin{proof}
		\begin{enumerate}[(1)]
			\item
				Sei $n = 2m - 1$ und betrachte
				$v: \R^{n+1} \to \R^{n+1}$ definiert durch
				\[
					v(x_1, x_2, \dotsc, x_{2m-1}, x_{2m})
					:= (-x_2, x_1, \dotsc, -x_{2m}, x_{2m-1}).
				\]
				Dies definiert $v: \S^n \to \R^{n+1}$ mit $\<x, v(x)\> = 0$.
			\item
				Zu $v: \S^n \to \R^{n+1}$ betrachte
				$H: [0,1] \times \S^n \to \S^n$ definiert durch
				\[
					H_t(x) := \f {x+ tv(x)}{|x+tv(x)|}.
				\]
				Wenn $v$ nirgends verschwindet, dann ist $H_1(x) \neq x$ für alle $x$.
				Wir betrachten $K: [0,1] \times \S^n \to \S^n$, definiert durch
				\[
					K_t(x) := \f {(1-t) H_1(x) - tx}{|(1-t) H_1(x) - tx|}.
				\]
				$K_t$ ist wohldefiniert wegen $H_1(x) \neq x$ und stetig.
				Es gilt $K_0 = H_1, K_1 = -\Id$.
				Dann gilt $\Id_{\S^n} \stack{H}\homotopic H_1 \stack{K}\homotopic -\Id_{\S^n}$.
				Es gilt aber
				\begin{align*}
					\deg(\Id_{\S^n}) &= 1, &
					\deg(-\Id_{\S^n}) &= (-1)^{n+1} = - 1.
				\end{align*}
				Ein Widerspruch zur Homotopieinvarianz des Abbildungsgrad.
		\end{enumerate}
	\end{proof}
\end{st}


\section{Der Satz von Borsuk-Ulam}


\begin{st}
	Sei $f: \S^n \to \S^n$ mit $f(-x) = -f(x)$.
	Dann ist $\deg(f)$ ungerade.
	\begin{proof}
		Folgt
	\end{proof}
\end{st}

