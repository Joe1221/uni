\chapter{Die Sprache der Kategorien}

\coursetimestamp{03}{12}{2013}


\section{Kategorien}


\begin{df}[Katogorie]
	Eine \emph{(Links)Kategorie} $\Cat C = (\Ob, \Mor, \circ)$ hat folgende Bestandteile:
	\begin{enumerate}[a)]
		\item
			eine Klasse $\Ob$ von \emph{Objekten} $A, B, C, D, \dotsc \in \Ob$,
		\item
			zu $A, B \in \Ob$ eine Klasse $\Mor(A,B)$ von \emph{Morphismen von $A$ nach $B$},
		\item
			zu $A, B, C \in \Ob$ eine \emph{Verknüpfung}
			\[
				\circ: \Mor(B, C) \times \Mor(A, B) \to \Mor(A, C).
			\]
	\end{enumerate}
	wobei folgende Bedingungen erfüllt sein müssen:
	\begin{enumerate}[1)]
		\item
			Zu jedem $B \in \Ob$ existiert $\Id_B \in \Mor(B, B)$, sodass für alle $A, C \in \Ob, f \in \Mor(A, B), g \in \Mor(B, C)$ gilt
			\begin{align*}
				\Id_B \circ f &= f, &
				g \circ \Id_B &= g.
			\end{align*}
		\item
			Es gilt Assoziativität, d.h. für alle $A, B, C, D \in \Ob, f \in \Mor(A, B), g \in \Mor(B, C), h \in \Mor(C, D)$ gilt
			\[
				h \circ (g \circ f) = (h \circ g) \circ f.
			\]
	\end{enumerate}
	\begin{note}
		Die Verknüpfung ist derart gewählt, dass nach links hin verknüpft wird (wie bei der Komposition von Funktionen auch).
		Kategorien mit Verknüpfung nach rechts hin heißen \emph{Rechtskategorien}.
	\end{note}
\end{df}

\begin{ex}
	\begin{itemize}
		\item
			$\Cat{Top}$: (topologische Räume, stetige Abbildungen, Komposition)
		\item
			\begin{itemize}
				\item
					$\Cat{Set}$: (Mengen, Abbildungen, Komposition).
				\item
					$\Cat{Set_{inj}}$: (Mengen, injektive Abbildungen, Komposition).
				\item
					$\Cat{Set_{sur}}$: (Mengen, surjektive Abbildungen, Komposition).
			\end{itemize}
		\item
			$\Cat{FinSet}$: (endliche Mengen, Abbildungen, Komposition)
		\item
			$(X, \le, \text{Transitivität})$
		\item
			$\Cat{Grp}$: (Gruppen, Gruppenhomomorphismen, Komposition)
		\item
			$\Cat{FinGrp}$: (endliche Gruppen, Gruppenhomomorphismen, Komposition)
		\item
			$\Cat{Ab}$: (abelsche Gruppen, Gruppenhomomorphismen, Komposition)
		\item
			$\Cat{Vec_K}$: ($K$-Vektorräume, $K$-lineare Abbildungen, Komposition)
		\item
			$\Cat{Mat_K}$: ($\N$, Matrizen über $K$, Multiplikation)
			\[
				\Mat_K(q,r) \times \Mat(p,q) \to \Mat_K(p,r)
			\]
	\end{itemize}
\end{ex}

\paragraph{Kommutative Diagramme}

Ein Diagramm
\[
	\begin{tikzcd}[column sep=small]
		& B \arrow{dr}{g} & \\
		A \arrow{ur}{f} \arrow{rr}{h} & & C
	\end{tikzcd}
\]
kommutiert, falls $h = g \circ f$.

\begin{ex}
	\begin{itemize}
		\item
			\[
				\begin{tikzcd}[row sep=tiny]
					& B \arrow{dr}{g} \arrow{dd}{\Id_B} & \\
					A \arrow{ur}{f} \arrow{dr}[below]{f} & & C \\
					& B \arrow{ur}[below]{g} &
				\end{tikzcd}
			\]
		\item
			\[
				\begin{tikzcd}[column sep=large]
					~& B \arrow{dr}{h\circ g} & \\
					A \arrow{ur}{f} \arrow{rr}[xshift=-7mm]{(h\circ g) \circ f}\arrow{rr}[xshift=-7mm,swap]{h\circ (g \circ f)} \arrow{dr}[swap]{g\circ f} & & D \\
					& C \arrow[leftarrow,crossing over]{uu}[yshift=5mm,swap]{g} \arrow{ur}[swap]{h} &
				\end{tikzcd}
			\]
		\item
			$g$ ist Linksinverse von $f$; $g$ ist Rechtsinverse von $f$; $f$ und $g$ sind zueinander invers:
			\[
				\begin{tikzcd}
					X \arrow{d}[swap]{\Id_X} \arrow{r}{f} & Y \arrow{dl}{g} \\
					X &
				\end{tikzcd}
				\!
				\begin{tikzcd}
					~& Y \arrow{d}{\Id_Y} \arrow{dl}[swap]{g} \\
					X \arrow{r}{f} & Y
				\end{tikzcd}
				\qquad
				\begin{tikzcd}
					X \arrow{r}{f} \arrow{d}[left]{\Id_X} & Y \arrow{d}{\Id_Y} \arrow{dl}{g} \\
					X \arrow{r}[below]{f} & Y
				\end{tikzcd}
			\]
	\end{itemize}
\end{ex}

\begin{df}
	\begin{itemize}
		\item
			Zwei Morphismen $f: X \to Y$ und $g: Y \to X$ in $\Cat{C}$ heißen \emph{zueinander invers}, wenn $g \circ f = \Id_X$ und $f \circ g = \Id_Y$ gilt.
		\item
			Wir nennen $f: X \to Y$ in $\Cat{C}$ invertierbar, oder einen $\Cat{C}$-Isomorphismus, geschrieben $f: X \isoto Y$, wenn es zu $f$ einen inversen Morphismus $g: Y \to X$ in $\Cat{C}$ gibt.
		\item
			Zwei Objekte $X, Y \in \Cat{C}$ heißen \emph{isomorph}, geschrieben $X \isomorphic Y$, wenn ein Isomorphismus $f: X \to Y$ existiert.
	\end{itemize}
\end{df}

\begin{ex}
	\begin{itemize}
		\item
			In $\Cat{Set}$ sind Bijektionen Isomorphismen.
		\item
			In $(X, \le, \text{Transitivität})$ bedeutet Isomorphie Gleichheit.
		\item
			In $\Cat{Vec_K}$ $K$-lineare Isomorphismen.
		\item
			In $\Cat{Mat_K}$ beudetet Isomorphie $m = n$ in $\N$.
		\item
			In $\Cat{Grp}$ Gruppenisomorphismen.
		\item
			In $\Cat{Top}$ Homöomorphismen.
	\end{itemize}
\end{ex}


\section{Funktoren}


\begin{df}
	Seien $\Cat{C}, \Cat{D}$ Kategorien.
	Ein \emph{kovarianter Funktor} $F: \Cat{C} \to \Cat{D}$ ordnet jedem Objekt $X$ in $\Cat{C}$ ein Objekt $F(X)$ in $\Cat{D}$ zu und jedem Morphismus $f: X \to Y$ in $\Cat{C}$ einen Morphismus $F(f) : F(X) \to F(Y)$ in $\Cat{D}$, sodass $F(\Id_X) = \Id_{F(X)}$ und $F(g\circ f) = F(g) \circ F(f)$.
	\[
		\begin{tikzcd}[column sep=small]
			~ & Y \arrow{dr}{g} & \\
			X \arrow{ur}{f} \arrow{rr}[swap]{h} & & Z
		\end{tikzcd}
		\mapsto
		\begin{tikzcd}[column sep=tiny]
			~ & F(Y) \arrow{dr}{F(g)} & \\
			F(X) \arrow{ur}{F(f)} \arrow{rr}[swap]{F(h)} & & F(Z)
		\end{tikzcd}
	\]
\end{df}

\begin{ex}
	\begin{itemize}
		\item
			\emph{kontravarianter Funktor}:
			\begin{align*}
				\Hom_K(\argdot, K): \Vec_K &\to \Vec \\
				V &\mapsto V^* = \Hom_K(V,K) \\
				(f: V \to W) &\mapsto (f^*: W^* \to V^*, f^*(h) = h \circ f)
			\end{align*}
		\item
			\emph{kovarianter Funktor}:
			\begin{align*}
				\Hom_K(X, \argdot) : \Vec_K &\to \Vec_K \\
				V & \mapsto \Hom_K(X, V) \\
				(f: V \to W) & \mapsto (f_*: \Hom(X, V) \to \Hom(X, V), f_*(h) = f \circ h)
			\end{align*}
	\end{itemize}
\end{ex}

\begin{df}
	Ein \emph{kontravarianter Funktor} $G: \Cat{C} \to \Cat{D}$ ordnet jedem Objekt $X$ in $\Cat{C}$ ein Objekt $G(X)$ in $\Cat{D}$ zu und jedem Morphismus $f: X \to Y$ in $\Cat{C}$ einen Morphismus $G(f) : G(Y) \to G(X)$ in $\Cat{D}$, sodass $G(\Id_X) = \Id_{G(X)}$ und $G(g\circ f) = G(f) \circ G(g)$
	\[
		\begin{tikzcd}[column sep=small]
			~ & Y \arrow{dr}{g} & \\
			X \arrow{ur}{f} \arrow{rr}{h} & & Z
		\end{tikzcd}
		\mapsto
		\begin{tikzcd}[column sep=tiny]
			~ & G(Y) \arrow[leftarrow]{dr}{G(g)} & \\
			G(X) \arrow[leftarrow]{ur}{G(f)} \arrow[leftarrow]{rr}{G(h)} & & G(Z)
		\end{tikzcd}
	\]
\end{df}

\begin{ex}
	\begin{itemize}
		\item
			Zusammenhangskomponenten
			\begin{align*}
				\scr Z: \Cat{Top} &\to \Cat{Set} \\
				X &\mapsto \scr Z(X) \\
				(f:X \to Y) &\mapsto (\scr Z(f): \scr Z(X) \to \scr Z(Y))
			\end{align*}
		\item
			Wegzusammenhangskomponenten
			\begin{align*}
				\pi_0: \Cat{Top} &\to \Cat{Set} \\
				X & \mapsto \pi_0(X) \\
				(f: X \to Y) &\mapsto (\pi_0(f): \pi_0(X) \to \pi_0(Y))
			\end{align*}
		\item
			„Vergiss“-Funktor (die Topologie wird „vergessen“)
			\begin{align*}
				V: \Cat{Top} &\to \Cat{Set} \\
				(X,\scr T) & \mapsto X \\
				(f: (X,\scr T_X) \to (Y, \scr T_Y)) &\mapsto (f: X \to Y)
			\end{align*}
		\item
			Potenzmenge (kovariant)
			\begin{align*}
				\scr P_*: \Cat{Set} & \to \Cat{Set} \\
				X, & \mapsto \scr P(X) \\
				(f: X \to Y) &\mapsto (f_*: \scr P(X) \to \scr P(y), f_*(A) := f(A))
			\end{align*}
		\item
			Potenzmenge (kontravariant)
			\begin{align*}
				\scr P^*: \Cat{Set} & \to \Cat{Set} \\
				X & \mapsto \scr P(X) \\
				(f: X \to Y) &\mapsto (f^*: \scr P(Y) \to \scr P(X), f^*(B) := f^{-1}(B))
			\end{align*}
	\end{itemize}
\end{ex}


\section{Natürliche Transformationen}


\begin{df}
	Seien $F, G: \Cat{C} \to \Cat{D}$ Funktoren.
	Eine \emph{(natürliche) Transformation} $t: F \to G$ ordnet jedem Objekt $X$ in $\Cat{C}$ einen Morphismus $t(X): F(X) \to G(X)$ in $\Cat{D}$ zu, sodass für jeden Morphismus $f: X \to Y$ in $\Cat{C}$ die Gleichung $t(Y) \circ F(f) = G(f) \circ t(X)$ gilt.
	\[
		\begin{tikzcd}
			X \arrow{d}{f} & F(X) \arrow{d}{F(f)} \arrow{r}{t(X)} & G(X) \arrow{d}{G(f)} \\
			Y & F(Y) \arrow{r}{t(Y)} & G(Y)
		\end{tikzcd}
	\]
	Gilt hierbei $t(X) : F(X) \isoto G(X)$ für alle $X$, so heißt $t$ \emph{(natürliche) Äquivalenz} von $F$ und $G$.
\end{df}

\begin{ex}
	\[
		\begin{tikzcd}
			~ & \pi_0(X) \arrow{dd}[yshift=5mm]{t(X)} \arrow{rr}{\pi_0} & & \pi_0(Y) \arrow{dd}{t(Y)} \\
			X \arrow{ur}{a\mapsto [a]_X} \arrow{dr}[swap]{a\mapsto \scr Z_X(a)} \arrow[crossing over]{rr}[swap,xshift=5mm]{f} & & Y \arrow{ru}{b\mapsto[b]_Y} \arrow{dr}[swap]{b\mapsto \scr Z_Y(b)} & \\
			& \scr Z(X) \arrow{rr}[below]{\scr Z(f)} && \scr Z(Y)
		\end{tikzcd}
%		\begin{tikzcd}[row sep=small,column sep=small]
%			X \arrow{dr}{f} \arrow{dd}[left]{g\circ f} \arrow{rr} &  & \pi_0(X) \arrow{dr}{\pi_0(f)} & \\
%			& Y \arrow{dl}{g} \arrow{rr} &  & \pi_0(Y) \arrow{dl}{\pi_0(g)}\\
%			Z \arrow{rr}  & & \pi_0(Z) \arrow[leftarrow, crossing over]{uu}[yshift=5mm]{\pi_0(g\circ f)} &
%		\end{tikzcd}.
	\]
	mit
	\begin{align*}
		t(X): \pi_0(X) &\to \scr Z(X) \\
		[a]_X &\mapsto \scr Z_X(a)
	\end{align*}
	\begin{note}
		Lokale wegzusammenhängende Räume bilden eine Unterkategorie von $\Cat{Top}$.
		Hierauf sind die Funktoren $\pi_0$ und $\scr Z$ natürlich äquivalent.
	\end{note}
\end{ex}
