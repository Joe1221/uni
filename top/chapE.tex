\chapter{Zusammenhang und Homotopie}

\coursetimestamp{26}{11}{2013}

\section{Zusammenhang}

\begin{df} \label{df:intermediate_value_property}
	Sei $f: X \to \R$ eine stetige Funktion.
	Wir definieren die \emph{Zwischenwerteigenschaft} als
	\[
		\forall y \in \R, a,b \in X: f(a) \le y \le f(b) \implies \exists x \in X : f(x) = y.
	\]
\end{df}

\begin{st}
	Sei $(X, \scr T)$ ein topologischer Raum.
	Dann sind folgende Aussagen äquivalent:
	\begin{enumerate}[(1)]
		\item
			Jede stetige Funktion $f: X \to \R$ hat die Zwischenwerteigenschaft
		\item
			Für $f: X \to \R$ stetig ist $f(X) \subset \R$ ein Intervall.
		\item
			Jede stetige Funktion $f: X \to \{0,1\} \subset \R$ ist konstant.
		\item
			Jede stetige Funktion $f: X \to Y$ in einen diskreten Raum $Y$ ist konstant.
		\item
			Für jede offene Zerlegung $X = A \dunion B$ gilt $A = \emptyset$ oder $B = \emptyset$.
	\end{enumerate}
	$(X, \scr T)$ mit diesen Eigenschaften nennen wir \emph{zusammenhängend}.
	\begin{proof}
		\begin{segnb}[„$(1)\implies(2)\implies(3)$“]
			Leicht nachvollziehbar.
		\end{segnb}
		\begin{segnb}[„$(3)\implies(4)$“]
			Sei $x\in X, y := f(x)$ und $g: Y \to \{0,1\} \subset \R, g(y) := 1, g|_{Y\setminus \{y\}} := 0$.
			$g$ ist stetig, da $Y$ diskret, also konstant $g = 1$ nach Voraussetzung.
			Damit ist auch $f(X) \subset g^{-1}(1) = \{y\}$, also $f$ konstant.
		\end{segnb}
		\begin{segnb}[„$(4)\implies(5)$“]
			Sei $f : X \to \{0,1\}, f|_A := 0, f|_B := 1$.
			$f$ ist stetig (alle möglichen Urbilder sind offen in $X$), also ist $f$ nach Voraussetzung konstant und damit $A = f^{-1}(\{0\}) = \emptyset$ oder $B = f^{-1}(\{1\}) = \emptyset$.
		\end{segnb}
		\begin{segnb}[„$(5)\implies(1)$“]
			Sei $f: X \to \R$ stetig.
			Angenommen es existieren $y\in \R$ und $a,b \in X$ mit $f(a) \le y \le f(b)$, aber $y \not\in f(X)$.
			Dann wäre $X = A \dunion B$ mit $A = f^{-1}(\R_{<y}) \ni a, B = f^{-1}(\R_{>y}) \ni b$ eine offene Zerlegung von $X$, ein Widerspruch zur Voraussetzung.
		\end{segnb}
	\end{proof}
\end{st}

\begin{ex}
	\begin{itemize}
			%fixme: diskret, indiskret
		\item
			Jedes Intervall $I \subset \R$ ist zusammenhängend (nach Zwischenwertsatz, \ref{st:intermediate_value_theorem}).
		\item
			$\R \setminus \{a\}$ ist nicht zusammenhängend, denn $\R \setminus \{a\} = \R_{<a} \dunion \R_{>a}$ ist offene Zerlegung.
		\item
			$\Q$ ist nicht zusammenhängend (für $\xi \in \R \setminus \Q$ ist $\Q = \Q_{<\xi} \dunion \Q_{>\xi}$ offene Zerlegung).
	\end{itemize}
\end{ex}

\begin{st}
	Sei $(X, <)$ linear geordnet.
	Jedes Intervall $I \subset X$ ist zusammenhängend genau dann, wenn $X$ vollständig geordnet und zu allen $a, b \in X, a < b$ ein Zwischenwert $c \in X, a < c < b$ existiert.
	\begin{proof}
		Wie in $\R$, bzw. $\Q$.
	\end{proof}
\end{st}

\begin{ex}
	\begin{itemize}
		\item
			$(\R, <)$ ist vollständig geordnet, also jedes Intervall $I \subset \R$ zusammenhängend.
		\item
			Das Intervall $[0,1]_\Q$ ist nicht zusammenhängend, denn für $\xi \in [0,1] \setminus [0,1]_\Q$ ist
			\[
				[0,1]_\Q =
				\Set{ x \in \Q | 0 \le x < \xi }
				\dunion
				\Set{ x \in \Q | \xi < x \le 1 }
			\]
			eine offene Zerlegung.
			Also ist $(\Q, <)$ nicht vollständig geordnet.
	\end{itemize}
\end{ex}

\begin{st}
	Sei $f: X \to Y$ stetig.
	Falls $X$ zusammenhängend ist, dann auch $f(X)$.
	\begin{proof}
		Ist $f(X) = A \dunion B$ offene Zerlegung, so auch $X = f^{-1}(A) \dunion f^{-1}(B)$, also $A = \emptyset$ oder $B = \emptyset$.
	\end{proof}
\end{st}

\begin{lem}
	Sei $X$ ein topologischer Raum und $A_i \subset X$ für $i \in I$ zusammenhängend.
	Für jedes Paar $i,j \in I$ existiere eine Kette $i = i_0 , i_1, \dotsc, i_n = j$ in $I$ mit $A_{i_{k-1}} \cap A_{i_k} \neq \emptyset$ für $k = 1, \dotsc, n$.

	Dann ist $A = \bigcup_{i\in I} A_i$ zusammenhängend.
	\begin{proof}
		Sei $f: A \to \{0,1\}$ stetig.
		Dann ist $f_i := f|_{A_i}: A_i \to \{0,1\}$ stetig, also $f_i$ konstant, kurz $f_i = c_i$.
		Für $i = i_0, \dotsc, i_n = j$ wie oben gilt dann $c_i = \dotsb = c_j$.
	\end{proof}
\end{lem}

\begin{lem}
	Seien $X_1, \dotsc, X_n$ zusammenhängend, so auch $X = X_1 \times \dotsb \times X_n$.
	\begin{proof}
		Sei $a \in X$ und $f: X \to \{0,1\}$ stetig.
		Zu $b \in X$ betrachte
		\begin{align*}
			A_i &= \{ b_1 \} \times \dotsb \times \{b_{i-1}\} \times X_i \times \{a_{i+1}\} \times \dotsb \times \{a_n\} \\
			&\homeomorphic X_i.
		\end{align*}
		Es gilt $a \in A_1$, $b \in A_n$ und $A_{i-1} \cap A_i \ni (b_1, \dotsc, b_{i-1}, a_i, \dotsc, a_n)$.
		Nach dem Lemma ist $A_1 \cup \dotsb \cup A_n$ zusammenhängend, also $f(b) = f(a) = \const$.
	\end{proof}
\end{lem}

\begin{lem}
	Sei $(X, \scr T)$ ein topologischer Raum, $A \subset X$ zusammenhängend und $A \subset B \subset \_A$.
	Dann ist auch $B$ zusammenhängend.
	\begin{proof}
		Sei $f: B \to \{0,1\}$ stetig, dann ist $f|_A = c$ konstant.
		Die Abbildungen $f, c: B \to \{0,1\}$ sind stetig und stimmen auf $A$ überein.
		$A$ ist dicht in $B$ und $\{0,1\}$ ist hausdorffsch.
		Also gilt $f=c$.
	\end{proof}
\end{lem}

\begin{st}
	Sei $(X_i)_{i\in I}$ eine Familie topologischer Räume mit $X_i \neq \emptyset$.
	Der Produktraum $X = \prod_{i\in I} X_i$ ist genau dann zusammenhängend, wenn jedes $X_i$ zusammenhängend ist.
	\begin{proof}
		\begin{segnb}[„$\implies$“]
			Klar, da $p_i: X \to X_i$ stetig und surjektiv ist.
		\end{segnb}
		\begin{segnb}[„$\implies$“]
			Sei $a \in X$.
			Für $J \subset I$ endlich sei
			\begin{align*}
				A_J &:= \prod_{i\in I} A_i, &
				A_j &:= \begin{cases}
					X_j & j \in J \\
					\{a_j\} & j \in I \setminus J
				\end{cases}.
			\end{align*}
			$A_j$ ist zusammenhängend nach obigem Lemma.
			Damit ist auch $A = \bigcup_{J} A_J$ zusammenhängend, denn $A_J \cap A_j \ni \{a\}$.

			Es gilt außerdem $\_A = X$:
			Jede offen Menge $U \subset X$ enthält $\prod_{i \in I} U_i$, wobei $U_i \subset X_i$ offen und $U_i = X_i$ für $i \in I \setminus J$ außerhalb einer endlichen Menge.
			Ist $U$ nicht-leer, dann gilt $U_i \neq \emptyset$ für alle $i \in I$.
			Demnach existiert $b \in U$ mit $b_j \in U_j$ für alle $j \in J$ und $b_i = a_i$ für alle $i \in I \setminus j$.
			Damit gilt $b \in A_J$, also $b \in U \cap A$.
		\end{segnb}
	\end{proof}
\end{st}

%fixme \eqq ===
\begin{df}
	Sei $(X, \scr T)$ ein topologischer Raum.
	Für $x,y \in X$ definieren wir $x \eqq y$ durch die Bedingung, dass $x,y$ in einer zusammenhängenden Teilmenge von $X$ liegen.
	Dies ist eine Äquivalenzrelation.
	Die Äquivalenzklasse $\scr Z(x)$ von $x$ heißt \emph{(Zusammenhangs-)Komponente} von $x$ in $X$.
	Die definiert die Zerlegung
	\[
		\scr Z(X) := \{ \scr Z(x) : x \in X \}.
	\]
\end{df}

\begin{ex}
	\begin{itemize}
		\item
			$X$ ist zusammenhängend genau dann, wenn $\scr Z(X) = \{X\}$.
		\item
			$\scr Z(\R \setminus \{0\}) = \{ \R_{<0}, \R_{>0} \}$.
		\item
			$\scr Z(\Q) = \{ \{x\} : x \in \Q \}$.
	\end{itemize}
\end{ex}

\begin{st}
	\begin{enumerate}[(1)]
		\item
			$\scr Z(x)$ ist der größte zusammenhängende Teilraum von $X$, der $x$ enthält.
		\item
			Jede Komponente $\scr Z(x)$ ist abgeschlossen.
		\item
			Ist die Zerlegung $\scr Z(X)$ endlich, so ist jede Komponente offen und $X = \bigdunion \scr Z(X)$ ist eine Summentopologie.
	\end{enumerate}
\end{st}

\begin{st}
	\begin{enumerate}[(1)]
		\item
			Ist $f: X \to Y$ stetig, so gilt $f(\scr Z(x)) \subset \scr Z(f(x))$.
		\item
			Wir erhalten hieraus $\scr Z(f): \scr Z(X) \to \scr Z(Y): \scr Z(x) \mapsto \scr Z(f(x))$.
		\item
			Es gilt $\scr Z(\Id_X) = \Id_{\scr Z(X)}$ und $\scr Z(g\circ f) = \scr Z(g) \circ \scr Z(f)$.
	\end{enumerate}
	\begin{proof}
		(1) und (2) sind klar nach den vorigen Ergebnissen, zeige noch (3).
		Es gilt
		\begin{align*}
			\scr Z(g\circ f) (\scr Z(x))
			&= \scr Z(g(f(\scr Z(x)))) \\
			&= \scr Z(g) (\scr Z(f(\scr Z(x)))) \\
			&= \scr Z(g)(\scr Z(f)(\scr Z(x))) \\
			&= (\scr Z(g) \circ \scr Z(f)) (\scr Z(x)).
		\end{align*}
	\end{proof}
\end{st}


\section{Wegzusammenhang}

Analog zum Weg in metrischen Räumen definieren wir Wege in topologischen Räumen.

\begin{df}
	Ein \emph{Weg} im topologischen Raum $(X, \scr T)$ ist eine stetige Abbildung $\gamma: [0,1] \to X$.
	Dabei heißt $\gamma(0) = a$ \emph{Anfangspunkt} und $\gamma(1) = b$ \emph{Endpunkt}.

	Der Raum $X$ heißt \emph{wegzusammenhängend}, wenn zu jedem Paar $a,b \in X$ ein Weg von $a$ nach $b$ in $X$ existiert (d.h. $\gamma:[0,1] \to X$ stetig mit $\gamma(0) = a, \gamma(1) = b$).

	Wir definieren die Menge aller Wege $P(X)$ und die Menge aller Wege von $a$ nach $b$, $P(X, a, b)$ als
	\begin{align*}
		P(X) &= \scr C ([0,1], X), \\
		P(X, a, b) &= \{ \gamma : [0,1] \to X \text{ stetig } : \gamma(0) = a, \gamma(1) = b \}.
	\end{align*}
	Außerdem folgende Abbildungen, bzw. Operatoren:
	\begin{enumerate}[1)]
		\item
			$X \injto P(X), a \mapsto \const_{[0,1]}^a$,
		\item
			$\_\  : P(X, a, b) \to P(X, b, a), \_\gamma(t) := \gamma(1-t)$,
			% fixme: \_\ doesn't look good
		\item
			$\ast: P(X, a, b) \times P(X, b, c) \to P(X, a, c)$ durch
			\[
				(\gamma_1 \ast \gamma_2)(t) := \begin{cases}
					\gamma_1 (2t) & 0 \le t \le \f 12 \\
					\gamma_2 (2t - 1) & \f 12 \le t \le 1
				\end{cases}.
			\]
	\end{enumerate}
\end{df}

\begin{df}
	Wir nennen $x,y \in X$ \emph{verbindbar} in $X$ (durch einen Weg in $X$), wenn ein Weg $\gamma: [0,1] \to X$ von $\gamma(0) = x$ nach $\gamma(1) = y$ existiert.
	Dies ist eine Äquivalenzrelation.

	Die Äquivalenzklasse $[x]$ von $x$ heißt \emph{Weg(-zusammenhangs)komponente}.
	Dies definiert die Zerlegung
	\[
		\pi_0 (X) := \Set{ [x] | x \in X }.
	\]
	$X$ heißt \emph{wegzusammenhängend}, wenn $\pi_0(X) = \{X\}$.
\end{df}

\begin{ex}
	\begin{itemize}
		\item
			Jedes Intervall $I \subset \R$ ist wegzusammenhängend.
		\item
			Jede sternförmige Menge $X \subset \R^n$ ist wegzusammenhängend.
		\item
			$\S^n \subset \R^{n+1}$ ist wegzusammenhängend für $n \ge 1$.
	\end{itemize}
\end{ex}

\begin{st}
	Wegzusammenhang impliziert Zusammenhang, aber nicht umgekehrt.
	\begin{proof}
		Wie im metrischen Fall.
		Gegenbeispiel in \ref{ex:topologists_sine_curve}.
	\end{proof}
\end{st}

\begin{ex}[Sinuskurve des Topologen] \label{ex:topologists_sine_curve}
	Sei
	\begin{align*}
		A &:= \Set{ \big(x, \sin( \f {\pi}x)\big) | x \in (0,1] }, \\
		B &:= \Set{ 0 } \times [-1, 1].
	\end{align*}
	Dann ist $C := A \cup B = \_A$ zusammenhängend, aber nicht wegzusammenhängend.
\end{ex}

\coursetimestamp{02}{12}{2013}

\begin{st}
	Ist $f: X \to Y$ stetig und $X$ wegzusammenhängend, so auch $f(X)$.
\end{st}

\begin{st}
	Seien $(X_i, \scr T_i)$ mit $X_i \neq \emptyset$ topologische Räume.
	Genau dann ist $\prod_{i \in I} X_i$ wegzusammenhängend, wenn jeder Raum $X_i$ dies ist.
	\begin{proof}
		Folgt mit universeller Abbildungseigenschaft und dem letzten Satz.
	\end{proof}
\end{st}

\begin{st}
	\begin{enumerate}[1)]
		\item
			Ist $f: X \to Y$ stetig, so gilt
			\[
				f([a]_X) \subset [f(a)]_Y.
			\]
		\item
			Wir erhalten mit 1) die Abbildung \\ $\pi_0(f): \pi_0(X) \to \pi_0(Y): [a]_X \mapsto [f(a)]_Y$.
		\item
			Es gilt $\pi_0(\Id_X) = \Id_{\pi_0(X)}$ und $\pi_0(g \circ f) = \pi_0(g) \circ \pi_0(f)$.
			Das folgende Diagramm kommutiert:
			\[
				\begin{tikzcd}[row sep=small]
					X \arrow{dr}{f} \arrow{dd}[left]{g\circ f} \arrow{rr} &  & \pi_0(X) \arrow{dr}{\pi_0(f)} & \\
					& Y \arrow{dl}{g} \arrow{rr} &  & \pi_0(Y) \arrow{dl}{\pi_0(g)}\\
					Z \arrow{rr}  & & \pi_0(Z) \arrow[leftarrow, crossing over]{uu}[yshift=5mm]{\pi_0(g\circ f)} &
				\end{tikzcd}.
			\]
	\end{enumerate}
	\begin{proof}
		\begin{enumerate}[1),start=3]
			\item
				Es gilt
				\begin{align*}
					\pi_0(g\circ f)([a]_X)
					&= [(g\circ f)(a)]_Z
					= [g(f(a))]_Z
					= \pi_0(g)([f(a)]_Y) \\
					&= \pi_0(g)(\pi_0(f)([a]_X))
					= (\pi_0(g)\circ \pi_0(f))([a]_X).
				\end{align*}
		\end{enumerate}
	\end{proof}
\end{st}


\section{Lokaler Zusammenhang}


\begin{df}
	Ein topologischer Raum $(X, \scr T)$ heißt \emph{lokal (weg-)zusammenhängend} in $a \in X$, wenn jede offene Umgebung von $a$ in $X$ eine (weg-)zusammenhängende offene Umgebung enthält.
\end{df}

\begin{ex}
	\begin{itemize}
		\item
			$\R^n$ ist lokal wegzusammenhängend, denn $B(a,\eps)$ ist sternförmig und somit wegzusammenhängend.
			Ebenso jeder topologische Vektorraum (ausgeglichene Umgebung ist sternförmig).
		\item
			Ist $X$ lokal wegzusammenhängend, so auch jede offene Teilmenge $U \subset X$.
	\end{itemize}
\end{ex}

\begin{ex}
	\begin{itemize}
		\item
			Die „Sinuskurve des Topologen“ (siehe \ref{ex:topologists_sine_curve}) ist zusammenhängend, aber nicht lokal zusammenhängend.
		\item
			Der „rationale Kamm“
			\[
				X = ([0,1] \times \{0\}) \cup ([0,1]_\Q \times [0,1])
			\]
			ist wegzusammenhängend, aber nicht lokal wegzusammenhängend.
		\item
			$X = [0,1] \cup [2,3]$ ist lokal (weg-)zusammenhängend, aber nicht global (weg-)zusammenhängend.
	\end{itemize}
\end{ex}

Zu jedem Raum $X$ haben wir die Zerlegungen $X = \bigdunion \scr Z(X)$ und $X = \bigdunion \pi_0(X)$.

\begin{st}
	Ist $X$ lokal zusammenhängend, so ist die Zerlegung $X = \bigdunion \scr Z(X)$ offen.

	Ist $X$ lokal wegzusammenhängend, so ist die Zerlegung $X = \bigdunion \pi_0(X)$ offen.
	In diesem Fall gilt $\pi_0(X) = \scr Z(X)$.
	\begin{proof}
		Leicht nachzuvollziehen.
	\end{proof}
\end{st}


\section{Homotopie stetiger Abbildungen}


\begin{df}
	Eine \emph{Homotopie} ist eine stetige Abbildung $H: [0,1] \times X \to Y$.
	Für jedes $t \in [0,1]$ ist dann $H_t: X \to Y$ mit $H_t(x) := H(t,x)$ stetig.

	Zwei stetige Abbildungen $f,g : X \to Y$ heißen \emph{homotop} in $Y$, wenn es eine Homotopie $H: [0,1] \times X \to Y$ mit $H_0 = f$ und $H_1 = g$ gibt.
	Wir schreiben dann $H: f \homotopic g$, oder $f\homotopic g$

	Ist $f: X \to  Y$ homotop zu $\const_X^{\ast}: X \to \{\ast\} \subset Y$, so nennen wir $f$ \emph{nullhomotop}, Kurz $f \homotopic \ast$.

	Der Raum $X$ heißt \emph{zusammenziehbar}, wenn $\exists * \in Y: \Id_X \homotopic \ast$.
\end{df}

\begin{ex} \label{st:starlike_contractible}
	Sei $X \subset \R^n$ sternförmig bezüglich $a \in X$ (z.B. $X = \R^n, a = 0$).
	Dann ist $X$ zusammenziehbar mittels
	\begin{align*}
		H: [0,1] \times X &\to X, \\
		H(t,x) &= (1-t)x + ta.
	\end{align*}
	$H$ ist wohldefiniert und stetig für alle $(t,x) \in [0,1] \times X$.
	Es gilt $H_0 = \Id_X, H_1 = \const_X^{a}$.
\end{ex}

\begin{st} \label{st:sphere_non_antipodal_homotopic}
	Seien $f,g : X \to \S^n$ stetig und nirgends antipodal, d.h. $f(x) \neq -g(x)$ für alle $x \in X$.
	Dann sind $f$ und $g$ homotop in $\S^n$ mittels
	\[
		H(t,x) := \dfrac {(1-t)f(x) - tg(x)}{|(1-t)f(x) - tg(x)|}.
	\]
	\begin{proof}
		klar
	\end{proof}
\end{st}

\begin{kor}
	Ist $f: X \to \S^n$ stetig, aber nicht surjektiv, so gilt $f \homotopic *$.
	\begin{proof}
		Es gibt zwei Möglichkeiten, dies zu beweisen:
		\begin{enumerate}[1.]
			\item
				Wähle $a \in \S^n \setminus f(X)$ und $g = \const_X^{-a}$, dann ist $f \homotopic g$ gemäß \ref{st:sphere_non_antipodal_homotopic}.
			\item
				Mittels stereographischer Projektion.
				Für $a \in \S^n \setminus f(X)$ ist $\S^n \setminus \{a\} \homeomorphic \R^n$ und $\R^n \homotopic *$ wie in \ref{st:starlike_contractible}.
		\end{enumerate}
	\end{proof}
\end{kor}

\begin{df}
	Die Homotopie lässt folgende Eigenschaften zu
	\begin{enumerate}[(1)]
		\item
			Zu $f: X \to Y$ haben wir $H: f \homotopic f$ mit $H(t,x) = f(x)$.
		\item
			Zu $H: f \homotopic g$ definieren wir $\_H: g \homotopic f$ mit $H(t,x) := H(1-t,x)$.
		\item
			Zu $H: f \homotopic g$ und $K: g\homotopic h$ definieren wir $(H \ast K) : f \homotopic h$ durch
			\[
				(H \ast K)(t,x) = \begin{cases}
					H(2t, x) & 0 \le t \le \f 12 \\
					K(2t-1), x) & \f 12 \le t \le 1
				\end{cases}.
			\]
	\end{enumerate}
	Insbesondere ist damit Homotopie eine Äquivalenzrelation.
\end{df}

\begin{df}
	Die Menge der Homotopieklassen von $\scr C(X, Y)$ bezeichnen wir mit
	\[
		[X,Y] := \scr C(X, Y) / \homotopic.
	\]
\end{df}

%fixme : anmerkung zur notation
\begin{st}
	Aus $H: f_0 \homotopic f_1 : X \to Y$ und $K: g_0 \homotopic g_1: Y \to Z$ folgt $L: g_0 \circ f_0 \homotopic g_1 \circ f_1$ mittels
	\[
		L(t, x) = K(t, H(t,x)).
	\]
	Wir schreiben auch $L_t = K_t \circ H_t$.
\end{st}

\begin{kor}
	Wir haben eine Kategorie $\Cat{Toph}$:
	\begin{itemize}
		\item
			Objekte sind topologische Räume $X, Y, Z, \dotsc $,
		\item
			Morphismen sind Homotopieklassen $[f]$ von stetigen Abbildungen $f: X \to Y$,
		\item
			Die Komposition ist $[g] \circ [f] = [g\circ f]$.
	\end{itemize}
\end{kor}

\begin{st}
	Der Funktor $\pi_0: \Cat{Top} \to \Cat{Set}$ ist homotopie-invariant, d.h. aus $f \homotopic g$ folgt $\pi_0(f) = \pi_0(g)$.
	\[
		\begin{tikzcd}[row sep=tiny]
			\Cat{Top} \arrow{dr}{\pi_0} \arrow{dd}[swap]{q} & \\
			& \Cat{Set} \\
			\Cat{Toph} \arrow{ur}[swap]{\pi_0} & \\
		\end{tikzcd}.
	\]
\end{st}

\begin{df}
	Zwei Räume $X, Y$ heißen \emph{homotopie-äquivalent}, geschrieben $X \homotopic Y$, wenn es stetige Abbildungen $f: X \to Y$ und $g: Y \to X$ gibt mit $g \circ f \homotopic \Id_X$ und $f \circ g \homotopic \Id_Y$.
\end{df}

\begin{ex}
	\begin{enumerate}[1)]
		\item
			Falls $X \homeomorphic Y$, dann auch $X \homotopic Y$.
		\item
			Aus $X \homotopic Y$ folgt nicht $X \homeomorphic Y$, z.b. $\R \not \homeomorphic \R^2$, aber $\R \homotopic \R^2 \homotopic \ast$.
	\end{enumerate}
\end{ex}

\begin{ex}
	Sei $X = \S^n$ und $Y = \R^{n+1} \setminus \{0\}$.
	Dann sind $X$ und $Y$ homotopie-äquivalent.
	\begin{proof}
		Sei $f: X \injto Y$ die Inklusion und $g: Y \to X, g(y) = \f y{|y|}$.
		Offenbar ist $g \circ f = \Id_X$, zeige noch $f \circ g \homotopic \Id_X$:
		\begin{enumerate}[i)]
			\item
				$f \circ g$ ist \emph{Retraktion}, d.h. $(f \circ g)|_X = \Id_X$,
			\item
				$H: [0,1] \times Y \to Y$ mit
				\[
					H(t,y) = (1-t)y + t (f \circ g)(y)
					\in Y
				\]
				ist ein \emph{Deformationsretraktion}, d.h. stetig mit $H_0 = \Id_Y, H_1(Y) \subset X$ und $H_1|_X = \Id_X$.
			\item
				$H$ ist eine \emph{starke Deformationsretraktion}, d.h. zusätzlich gilt $H_t|_X = \Id_X$ für alle $t \in [0,1]$.
		\end{enumerate}
		Also ist $f \circ g \homotopic \Id_X$.
	\end{proof}
\end{ex}



