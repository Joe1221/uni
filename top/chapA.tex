% Alphabetical
\chapter{Was ist und was soll die Topologie?}

Eine vorläufig Antwort ist:
Topologie ist qualitative Geometrie.

\begin{ex}
	% fixme: images
	Ränder von Dreiecken, Quadraten und Kreisen sind \emph{homöomorph}.
	Flächen von Dreiecken, Quadraten und Kreisen sind \emph{homöomorph}.
	Die Ränder sind jedoch nicht zu den Flächen homöomorph.
\end{ex}


\section{Zentrales Beispiel: Flächen}

% fixme: img: sphere, torus, …

\begin{df}
	Eine (berandete) \emph{Fläche} ist ein metrischer Raum, der lokal homöomorph ist zu
	\[
		\R^2_+ := \{ (x,y) \in \R^2 \suchthat x \ge 0 \}
	\]
\end{df}

\begin{ex}
	Zu $g \in \N, r \in \N_{\ge 1}$ betrachten wir
	\begin{align*}
		F_{g,r}^+ := % fixme: img
		F_{g,r}^- := % fixme: img
	\end{align*}
	% fixme: img: F_{0,1}^+, F_{0,2}^+, F_{0,1}^-

	% fixme: Definition für r=0
	Flächen ohne Rand: $F_g^+ := F_{g,0}^+$, $F_g^- := F_g^+ / \{\pm 1\}$.

	Diese Beispiele $F_{g,r}^{\pm}$ sind kompakte und zusammenhängende Flächen.

	Es stellen sich folgende Fragen
	\begin{enumerate}[1)]
		\item
			Ist unsere Liste vollständig?
		\item
			Ist unsere Liste redundanzfrei?
	\end{enumerate}
\end{ex}

\begin{ex}
	% fixme: img
	Sind diese Beispiele homöomorph zu $F_{g,r}^{\pm}$?
\end{ex}

\begin{st}[Klassifikation der kompakten Flächen]
	Jede kompakte, zusammenhängende Fläche $F$ ist homöomorph zu genau einem der Modelle $F_{g,r}^\pm$.
\end{st}

Wir werden in dieser Vorlesung wie folgt vorgehen:

\begin{enumerate}[1.]
	\item
		Analytische Topologie
	\item
		Geometrische Topologie
	\item
		Algebraische Topologie
\end{enumerate}
