\chapter{Überlagerungen}


\coursetimestamp{03}{02}{2014}



\section{Überlagerungen}


\begin{df}
	Seien $X, \tilde X$ topologische Räume, $p: \tilde X \to X$ surjektiv.
	Eine offene Menge $U \subset X$ wird durch $p$ \emph{trivial überlagert}, wenn $\tilde U = p^{-1} (U) = \dunion_{i\in I} \tilde U_i$ disjunkte Vereinigung von offenen Mengen $\tilde U_i \subset \tilde X_i$ ist, sodass für jedes $i \in I$ die Einschränkung $p_i = p|_{\tilde U_i}^U \to U_i$ ein Homöomorphismus ist.
	\[
		%fixme: Pfannkuchen-Zeichnung
		\begin{tikzcd}
			\tilde X \arrow{d}{p} \\
			X
		\end{tikzcd}
	\]
	Die Abbildung $p$ heißt \emph{Überlagerung}, wenn jeder Punkt $x \in X$ eine offene Umgebung $U \subset X$ besitzt, die von $p$ trivial überlagert wird.

	Sie heißt \emph{$n$-blättrig}, wenn $|p^{-1}(x)| = n$ für alle $x \in X$.
	Sie heißt \emph{trivial}, wenn ganz $X$ trivial überlagert wird.
	\begin{note}
		$p$ ist insbesondere stetig.
		Die Anzahl der Blätter ist für jede Zusammenhangskomponente konstant.
	\end{note}
\end{df}

\begin{ex}
	\begin{itemize}
		\item
			$\id_X: X \to X$ ist eine Überlagerung, ebenso jeder Homöomorphismus $p: \tilde X \homeomorphicto X$.
		\item
			$q: X \times F \to X$ mit $q(x, y) = x$ und diskreter Faser $F = q^{-1}(x)$ für alle $x \in X$.
		\item
			$p: \tilde X = [-1, 1] \to X = [0, 1], p(x) = x^2$.
			Dies ist keine Überlagerung, denn die Blätterzahl ist nicht konstant.
		\item
			$p^*: \tilde X = [-1, 1] \setminus \{0\} \to X = [0, 1] \setminus \{0\}, p^*(x) = x^2$ ist eine Überlagerung.
		\item
			$p: \D^2 \to \D^2$ mit $p(z) = z^2$ ist keine Überlagerung, denn die Blätterzahl ist nicht konstant.
			Hingegen ist $p: \D^2 \setminus \{0\} \to \D^2 \setminus \{0\}, p(z) = z^2$ ist eine nicht-triviale Überlagerung.
			%fixme: riemannblatt? zeichnung
		\item
			$\exp: (\C, +) \to (\C^*, \cdot), \exp(z) = \sum_{k=0}^\infty \f {z^k}{k!}$.
			Es gilt $\ker(\exp) = 2\pi i \Z$.

			$p: (\R, +) \to (\S^1, 1), p(t) = e^{it} = (\cos t, \sin t)$.
			Es gilt $\ker(p) = 2\pi \Z$.
			\[
				\begin{tikzcd}
					\C \arrow{d}{\exp} & \R \times \R \arrow{l}[swap]{x+iy \mapsfrom (x,y)} \arrow{d}{(x,y)\mapsto(e^x, e^{iy})}\\
					\C^* & \R_{>0} \times \S^1 \arrow{l}{rs \mapsfrom (r,s)}
				\end{tikzcd}
				\begin{tikzcd}
					\C \arrow{d}{\exp} \arrow{dr} & \\
					\C^* & \C / 2\pi i \Z \arrow{l}{\homeomorphic}
				\end{tikzcd}
				\begin{tikzcd}
					\R \arrow{d}{p} \arrow{dr}{q} & \\
					\S^1 & \R / 2\pi \Z \arrow{l}{\homeomorphic}
				\end{tikzcd}
			\]
		\item
			Komplexe Wurzeln
			\[
				\begin{tikzcd}[row sep=small]
					\C \arrow{dd}[swap]{u \mapsto e^u} \arrow{dr}{u \mapsto w = e^{\f un}} & \\
					& \C^* \arrow{dl}{p_n: w \mapsto w^n} \\
					\C^* &
				\end{tikzcd}
			\]
		\item
			Geometrische Beispiele
			\begin{itemize}
				\item
					$q: \S^n \surto \R\P^n = \S^n / \{\pm\}$
				\item
					$q: F_g^+ \surto F_g^- = F_g^+ / \{\pm\}$
				\item
					Graph, $Y \stack{p}\to X$, $X = Y / \{\pm\}$ (zweiblättrig)
					$\chi(X) = -1, \chi(Y) = -2$
					% fixme: drawing

					Graph, $Z \stack{q}\to X$ (dreiblättrig, aber $Z$)
					$\chi(Y) = -3$
					% fixme: drawing

					Ebenso für Flächen.
			\end{itemize}
		\item
			Sei $p: \C \to \C$ eine Polynomabbildung, $p(z) = z^n + a_1 z^{n-1} + \dotsb + a_n$ mit $a_1, \dotsc, a_n \in \C$, $n \ge 1$.
			Dann ist $p$ surjektiv und außerhalb der kritischen Werte eine $n$-blättrige Überlagerung.
	\end{itemize}
\end{ex}

\begin{df}
	Sei $G$ eine Gruppe und $X$ eine Menge.
	Eine \emph{Linksoperation} von $G$ auf $X$ ist eine Abbildung $\phi: G \times X \to X$, geschrieben $(g, x) \mapsto g \cdot x = gx$, sodass $1 x = x, g(hx) = (gh) \cdot x$.

	Die definiert $x \sim y$ durch $\exists g \in G: y = gx$.
	Für $x \in X$ heißt $Gx := \{gx : g \in G\}$ die \emph{Bahn} von $x$ unter $G$.
	Der \emph{Bahnenraum} ist $G \\ X := \{ Gx : x \in X \}$, kurz
	\[
		G \stack{\phi}\to X \stack{q}\surto G \\ X.
	\]
\end{df}

\begin{ex}
	\begin{enumerate}[1)]
		\item
			$(\R, +)$ operiert auf $\C$ durch $\phi: \R \times \C \to \C, (t, z) \mapsto e^{2\pi i t} \cdot z$.
		\item
			$(\R, +)$ operiert auf $\C$ durch $\psi: \R \times \C \to \C, (t, z) \mapsto t + z$.
		\item
			$T = 2\pi i \Z$ operiert auf $\C$ durch $T \times \C \to \C, (t, z) \mapsto t + z$.
		\item
			$(\Z^2, +)$ operiert auf $\C$ durch $\Z^2 \times \C \to \C, (t, z) \mapsto t + z$.
	\end{enumerate}
\end{ex}

\begin{df}
	Die Operation $\phi: G \times X \to X$ heißt \emph{stetig}, wenn für jedes $g \in G$ die Abbildung $\phi_g: X \to X, x \mapsto gx$ stetig ist (also mit $\phi_{g^{-1}} = \phi_{g}^{-1}$ ein Homömmorphismus).

	Die Operation $\phi$ heißt \emph{frei}, wenn $gx \neq x$ für alle $g \in G \setminus \{1\}$ und alle $x \in X$.
	Die Operation $\phi$ heißt \emph{frei diskontinuierlich}, wenn jeder Punkt $x \in X$ eine Umgebung $U \subset X$ besitzt, sodass $U \cap g U = \emptyset$ für alle $g \in G \setminus \{1\}$.
\end{df}


% G \stack{\phi}\to (Bogen) \tilde X
\begin{st}
	Sei $G \stack{\phi}\to X \stack{q}\to X := G \\ \tilde X$ eine freie diskontinuierliche Operattion.
	Dann ist $q$ eine Überlagerung.
	\begin{proof}
		Klar aus Definition.
	\end{proof}
\end{st}

\coursetimestamp{04}{02}{2014}

\section{Hochhebung von Wegen}


Das Hochebungsproblem sucht $\tilde f$ in dem folgenden Diagramm:
\[
	\begin{tikzcd}~
		& (\tilde X, \tilde x_0) \arrow{d}{p} \\
		(W, w_0) \arrow{ur}{\tilde f} \arrow{r}{f} & (X, x_0)
	\end{tikzcd}
\]

\begin{st}
	Sei $p: (\tilde X, \tilde x_0) \to (X, x_0)$ eine Überlagerung.
	Zu jeder stetigen Abbildung $f: ([0,1]^n, 0) \to (X, x_0)$ existiert genau eine Hochhebung $\tilde f: ([0,1]^n, 0) \to (\tilde X, \tilde x_0)$ stetig mit $p \circ \tilde f = f$.
	\begin{proof}
		Folgende Skizze mit Kompaktheit und Lebesgue-Zahl.
		%fixme: drawing
		Für beliebiges $n$ per Induktion.
	\end{proof}
\end{st}

Anwendung auf Wege und Homotopien
% fixme: drawing

\begin{st}
	Jede Überlagerung $p: (\tilde X, \tilde x) \to (X, x)$ induziert Bijektionen
	\[
		\begin{tikzcd}
			P(\tilde X, \tilde x) \arrow{d}{P_\#} \arrow{r}{\quot} & \Pi(\tilde X,\tilde x) \arrow{d}{P_\#} \\
			P(X, x) \arrow{u}{P_b} \arrow{r}{\quot} & \Pi(X, x) \arrow{u}{P_b}
		\end{tikzcd}
	\]
	mit $P_\#(\tilde \alpha) = p \circ \tilde \alpha, P_b(\alpha) = \tilde \alpha$ (Hochebung von Wegen), bzw. $P_\#([\tilde \alpha]) = [p\circ \tilde \alpha], P_b([\alpha]) = [\tilde \alpha]$ (Hochhebung von Homotopien).
\end{st}

\begin{kor}
	Der Gruppenhomomorphismus $p_\#: \pi_1(\tilde X, \tilde x) \to \pi_1 (X, x)$ ist injektiv.
\end{kor}

\begin{ex}
	\[
		\pi_1(K, a) \isomorphic \<s, t | -\>
		\pi_1(\tilde K, \tilde a) \isomorphic \<s_1, s_2, t_1 | -\>
	\]
	\begin{align*}
		p_\#: \pi_1(\tilde K, \tilde a) &\to \pi_1(K, a) \\
		s_1 &\mapsto s \\
		s_2 &\mapsto t^{-1}st \\
		t_1 &\mapsto t^2 \\
	\end{align*}
	Das heißt, die freie Gruppe $\<s,t,| -\>$ von Rang $2$ hat als freie Untergruppe $\<s, t^{-1}st, t^2\>$ vom Rang $3$.
\end{ex}

\begin{st}
	Sei $p: \tilde X \to X$ eine Überlagerung.
	Zu jedem Weg $\alpha: ([0,1], 0) \to (X, x)$ und $\tilde x \in p^{-1}(x)$ existiert genau eine Hochhebung $\tilde \alpha: ([0,1], 0) \to (\tilde X, \tilde x)$.
	Wir setzen $\tilde x \cdot [\alpha] := \tilde \alpha (1)$.
	Dies ist wohldefiniert und erfüllt $\tilde x \cdot [1_x] = \tilde x$, sowie $(\tilde x \cdot [\alpha]) \cdot [\beta] = \tilde x \cdot ([\alpha] \cdot [\beta])$.
\end{st}

\begin{ex}
	Betrachte $p: \R \to \S^1, p(t) = e^{2\pi i t}$.
	Hebe $\alpha: ([0,1], 0) \to (\S^1, x)$ hoch zu $\tilde \alpha: ([0,1], 0) \to (\R, \tilde x)$ mit $p \circ \tilde \alpha = \alpha$, d.h. $\alpha(t) = e^{2\pi i t \tilde \alpha(t)}$.

	$\tilde \alpha(t) - \tilde \alpha(0)$ misst die Winkeländerung, $\tilde x \cdot [\alpha] = \tilde x + \deg(\alpha)$.
\end{ex}

\begin{kor}
	$G := \pi_1(X, x_0)$ operiert auf $F = p^{-1}(x_0)$ durch $F \times G \to F$ mit $(\tilde x, [\alpha]) \mapsto \tilde x \cdot [\alpha]$.
\end{kor}

\begin{st}
	Für $p: (\R, 0) \to (\S^1, 1)$ mit $p(t) = e^{2\pi i t}$ erhalten wir $\deg: \pi_1 (\S^1, 1) \homeomorphicto \Z$ durch $0 \cdot [\alpha] = \deg([\alpha])$.
\end{st}

\begin{st}
	Für $q: \S^2 \to \R\P^2, q(x) = \{\pm x\}$ erhalten wir $h: \pi_1(\R\P^2, x_0) \homeomorphicto \{\pi 1\} \homeomorphic \Z / 2$ durch $\tilde x_0 \cdot [\alpha] = h([\alpha]) \cdot \tilde x_0$.
\end{st}

\begin{st}
	Für jede Überlagerung $G \stack{\phi}\to \tilde X \stack q\to X$ gilt:
	\begin{enumerate}[1)]
		\item
			die Operation von $G$ kommutiert mit dem Fasertransport durch $\Pi(X)$:
			\[
				(g\cdot \tilde x) \cdot [\alpha] = g \cdot (\tilde x \cdot [\alpha]).
			\]
		\item
			für jeden Basispunkt $\tilde x_0 \in x$ und $x_0 = q(\tilde x_0)$ haben wir einen Gruppenhomomorphismus $h: \pi_1(X, x_0) \to G$ mit $\tilde x_0 \cdot [\alpha] = h([\alpha]) \tilde x_0$.
		\item
			Ist $\tilde X$ wegzusammenhängend, so ist $h$ surjektiv.
			\[
				\im(h) = \Set{ g \in G | \text{$\tilde x_0$ und $g \tilde x_0$ sind in $\tilde X$ verbindbar} }
			\]
		\item
			Ist $\tilde X$ einfach wegzusammenhängend, so ist $h$ bijektiv.
			\[
				\ker(h) = q_\#\big(\pi_1(\tilde X, \tilde x_0) \big)
			\]
	\end{enumerate}
\end{st}

Ist also $\tilde X$ wegzusammenhängend, sohaben wir die \emph{kurze exakte Sequenz}, d.h.
\[
	1 \to \pi_1 (\tilde X, \tilde x_0)
	\xrightarrow{q_\#} \pi_1 (X, x_0)
	\xrightarrow{h} G
	\to 1
\]
mit $q_\#$ injektiv, $h$ surjektiv, $\ker(h) = \im(q_\#)$.

\begin{ex}
	$\Z \xrightarrow{\phi} (\R, 0) \stack q\to (\S^1, 1), q(t) = e^{2\pi i t}$
	\[
		\begin{tikzcd}
			\pi_1(\R, 0) \arrow[inj]{r}{q_\#} \arrow{d} & \pi_1(\S^1, 1) \arrow[sur]{r}{h} \arrow{d}{\homeomorphic} & \Z \\
			0 & \Z & \Z
		\end{tikzcd}
	\]
\end{ex}

\begin{ex}
	$\{\pm 1\} \xrightarrow{\S^n} \stack q\surto \R\P^n$
	Für $n \ge 2$:
	\[
		\begin{tikzcd}
			\pi_1(\S^n, \tilde x_0) \arrow{r}{q_\#} \arrow{d}{\homeomorphic} & \pi_1 (\R\P^n, x_0) \arrow{r}{h, \isomorphic} \arrow{d}{\homeomorphic} &  \{\pm 1\} \arrow{d}{\isomorphic} \\
			0 \arrow[inj]{r} &  \Z / 2 & \Z /2
		\end{tikzcd}
	\]
	Für $n = 1$:
	\[
		\begin{tikzcd}
			\pi_1(\S^1, \tilde x_0) \arrow{r}{q_\#} \arrow{d}{\homeomorphic} & \pi_1 (\R\P^1, x_0) \arrow{r}{h, \isomorphic} \arrow{d}{\homeomorphic} &  \{\pm 1\} \arrow{d}{\isomorphic} \\
			\Z \arrow[inj]{r}{\cdot 2} &  \Z / 2 \arrow[sur]{r} & \Z /2
		\end{tikzcd}
	\]
\end{ex}

\begin{ex}
	$\Z \to \R \stack \rho\to \SO_2, \rho(t) = \begin{psmallmatrix} \cos(2\pi t) & -\sin(2\pi t) \\ \sin(2\pi t) & \cos(2\pi t)\end{psmallmatrix}$
	\[
		\begin{tikzcd}
			\pi_1(\R, 0) \arrow{r}{\rho_\#} \arrow{d}{\homeomorphic} & \pi_1 (\SO_2, 1) \arrow{r}{h, \isomorphic} \arrow{d}{\homeomorphic} &  \{\pm 1\} \arrow{d}{\isomorphic} \\
			0 \arrow[inj]{r} &  \Z  & \Z
		\end{tikzcd}
	\]
\end{ex}

\begin{ex}
	$\{\pm 1\} \to \S^3 \to \SO_3$
	\[
		\begin{tikzcd}
			\pi_1(\S^3, 1) \arrow{r}{\rho_\#} \arrow{d}{\homeomorphic} & \pi_1 (\SO_3, 1) \arrow{r}{h, \isomorphic} \arrow{d}{\homeomorphic} &  \{\pm 1\} \arrow{d}{\isomorphic} \\
			0 \arrow[inj]{r} &  \Z / 2 & \Z / 2
		\end{tikzcd}
	\]
\end{ex}


\section{Galois-Korrespondenz}


\begin{ex}
	$\deg: \pi_1(\S^1, 1) \isomorphicto \Z$.
	Untergruppen von $\Z$: $\{0\}$ und $n\Z$ für $n \in \N_{\ge 1}$.
	\begin{align*}
		p_n: (\S^1, 1) &\to (\S^1, 1), z \mapsto z^n,& \im(p_n)_\# &= n \Z \\
		p_0: (\R, 0) &\to (\S^1, 1), t \mapsto e^{2\pi i t},& \im(p_n)_\# &= 0
	\end{align*}
\end{ex}

\begin{st}
	Zu $X$ zusammenhängend und lokal einfach zusammenhängend existiert eine universelle Überlagerung $G \to \tilde X \stack q\to  X$, d.h. $\pi_0(\tilde X) = \{ \tilde X \}$ und $\pi_1 (\tilde X, \tilde x_0) = \{1\}$, also $h: \pi_1(X, x_0) \isomorphicto G$.

	Jede zusammenhängende Überlagerung $p: (Y, y_0) \to (X, x_0)$ definiert eine Untergruppe $\im(p_\#) < G$.

	Zu jeder Untergruppe $U < G$ existiert $p: (Y, y_0) \to (X, x_0)$ mit $\im(p_\#) = U$.
	Diese Überlagerung ist eindeutig bis auf Isomorphie.
\end{st}
