%\part{Geometrische Topologie}

\chapter{Simpliziale Komplexe}
\coursetimestamp{09}{12}{2013}

\section{Simplizialkomplexe}

\begin{df}
	Sei $V$ ein $\R$-Vektorraum und $v_0, v_1, \dotsc, v_n \in V$.
	Wir nennen $v_0, v_1, \dotsc, v_n$ \emph{affin unabhängig}, wenn $v_1 - v_0, v_2 - v_0, \dotsc, v_n - v_0$ linear unabhängig sind.
	\begin{note}
		Die Wahl von $v_0$ ist willkürlich und unerheblich, äquivalent wäre: $v_0 - v_k, \dotsc, v_{k-1} - v_k, v_{k+m} - v_k, \dotsc, v_n - v_k$ linear unabhängig für beliebiges $k \in \N$.
	\end{note}
\end{df}

\begin{prop} \label{prop:affine_independent}
	Sei $V$ ein $\R$-Vektorraum.
	$v_0, v_1, \dotsc, v_n \in V$ sind affin unabhängig genau dann, wenn aus $\sum_{i=0}^n t_i v_i = \sum_{i=0}^n t_i' v_i$ und $\sum_{i=0}^n t_i = \sum_{i=0}^n t_i'$ für alle $i \in I$ folgt $t_i = t_i'$.
	\begin{proof}
		\begin{segnb}[„$\implies$“]
			Es gilt
			\[
				0
				= \sum_{i=0}^n (t_i' - t_i) v_i - \overbrace{\sum_{i=0}^n (t_i' - t_i) v_0}^{= 0}
				= \sum_{i=1}^n (t_i' - t_i) (v_i - v_0),
			\]
			wegen der affinen Unabhängigkeit ist $\{v_i - v_0\}_{i=1}^n$ linear unabhängig, also $t_i' = t_i$ für alle $i \in \{1, \dotsc, n\}$ und somit dank $\sum_{i=0}^n t_i = \sum_{i=0}^n t_i'$ auch $t_0' = t_0$.
		\end{segnb}
		\begin{segnb}[„$\implies$“]
			Sei $\sum_{i=1}^n \lambda_i (v_i - v_0) = 0$.
			Setze $t_i = 0$ für $i \in \{0, \dotsc, n\}, t_0' := -\lambda_1 - \lambda_2 - \dotsb - \lambda_n$ und $t_i := \lambda_i$ für $i \in \{1, \dotsc, n\}$.
			Dann ist
			\begin{align*}
				\sum_{i=0}^n t_i = 0 &= \sum_{i=0}^n t_i' \\
				\sum_{i=0}^n t_i v_i = 0 &= \sum_{i=1}^n \lambda_i (v_i - v_0) = \sum_{i=0}^n t_i' v_i
			\end{align*}
			also $t_i' = t_i = 0$ und somit auch $\lambda_i = 0$.
		\end{segnb}
	\end{proof}
\end{prop}

\begin{df}
	Für $n \in \N$ ist der \emph{$n$-dimensionale Standardsimplex} der Raum
	\[
		\Delta^n := \Set{ (t_0, \dotsc, t_n) \in \R^{n+1} | t_0, \dotsc, t_n \ge 0, t_0 + \dotsb + t_n = 1 }.
	\]
\end{df}

\begin{conv}
	Im Folgenden sei $V$ stets ein $\R$-Vektorraum und $v_0, \dotsc, v_n \in V$ affin unabhängig.
\end{conv}

\begin{df}
	Seien $v_0, \dotsc, v_n \in V$ affin unabhängig.
	Ihre Konvexe Hülle ist der \emph{affine $n$-Simplex}
	\[
		\Delta = [v_0, v_1, \dotsc, v_n]
		= \Set{ t_0v_0 + t_1v_1 + \dotsb + t_nv_n | (t_0, \dotsc, t_n) \in \Delta^n }.
	\]
	%fixme: ungeschickte notation mit \Delta (dimension? v_i?)
	Im Kontext von $\Delta$ nennen wir $t_0, t_1, \dotsc, t_n$ auch \emph{baryzentrische Koordinaten}.
	\begin{note}
		Nach \ref{prop:affine_independent} sind die baryzentrischen Koordinaten eindeutig.
	\end{note}
\end{df}

\begin{ex}
	\begin{itemize}
		\item
			$\Delta^n = [e_0, \dotsc, e_n] \subset \R^{n+1}$
		\item
			$[0,e_1,\dotsc,e_n] = \Set{ (t_1, \dotsc, t_n) \in \R^n | t_1, \dotsc, t_n \ge 0, t_1 + \dotsb + t_n \le 1 }$
	\end{itemize}
\end{ex}

\begin{nt}
	Es gilt $h: \Delta^n \homto \Delta$ mit
	\[
		h(t_0, \dotsc, t_n) = t_0 v_0 + \dotsb + t_n v_n.
	\]
	\begin{itemize}
		\item
			$\vert: [v_0, \dotsc, v_n] \mapsto \{v_0, \dotsc, v_n\}$
		\item
			$\dim: [v_0, \dotsc, v_n] \mapsto n$
	\end{itemize}
\end{nt}

\begin{df}
	Für $S \subset \{v_0, \dotsc, v_n\}$ heißt $[S]$ eine \emph{Seite} von $\Delta$, kurz $[S] \le \Delta$.

	Eine Seite $[S] \le \Delta$ mit $[S] \neq \Delta$ heißt \emph{echt}, kurz $[S] < \Delta$.

	Der \emph{Rand} von $\Delta$ ist $\boundary \Delta := \bigcup_{[S] < \Delta} [S]$. \\
	Das \emph{Innere} ist $\Delta := \Delta \setminus \boundary \Delta$.

	\begin{note}
		Rand und Inneres sind nur dann gleichbedeutend mit den bisherigen topologischen Definitionen, wenn wir sie bezüglich des affinen Vektorraums des Simplex betrachten.
		% fixme: besser formulieren
	\end{note}
\end{df}


\subsection{Affine Simplizialkomplexe}


% fixme: Zerlegung eines Quadrats in verschiedene Simplizes, auch kompliziertere Gebilde (z.b. Octaeder)

\begin{df}
	Ein \emph{(affiner) Simplizialkomplex} in $V$ ist eine Menge $\scr K$ von affinen Simplizes $\Delta \subset V$, für die gilt
	\begin{enumerate}[1), start=0]
		\item
			$\emptyset \in \scr K$,
		\item
			$\Delta' \le \Delta \in \scr K \implies \Delta' \in \scr K$,
		\item
			$\Delta_1, \Delta_2 \in \scr K \implies \Delta_1 \cap \Delta_2 \le \Delta_1, \Delta_2$.
	\end{enumerate}
	Wir vereinbaren
	\[
		\dim \scr K := \sup \Set{ \dim \delta | \Delta \in \scr K }.
	\]
	Die \emph{Eckenmenge}
	\[
		\Omega(\scr K) = \bigcup \Set{ \vert \Delta | \Delta \in \scr K },
	\]
	die \emph{Polyeder}
	\[
		|\scr K| := \bigcup \scr K := \bigcup \Set{ \Delta | \Delta \in \scr K }.
	\]
	Jedes Simplex $\Delta \in \scr K$ versehen wir mit seiner euklidischen Topologie.
	Das Polyeder $|\scr K|$ versehen wir mit seiner \emph{simplizialen Topologie}:
	eine Menge $U \subset |\scr K|$ sei offen, wenn $U \cap \Delta$ offen in $\Delta$ ist für jeden Simplex $\Delta \in \scr K$.
\end{df}

\begin{ex}[Gegenbeispiele]
	% fixme: ergänzen
\end{ex}

\begin{ex}
	\begin{itemize}
		\item
			$0$-dim: $\scr K = \{ \emptyset, \{a\} : a \in \Omega \}$, $|\scr K| = \Omega$ mit diskreter Topologie.
		\item
			$1$-dimensional: simpliziale Graphen.
		\item
			simpliziale Sinuskurve des Topologen, $B$ ist offen.
	\end{itemize}
\end{ex}

\begin{nt}
	Ist $\scr K$ in $V$ (lokal-)endlich, so stimmen die simpliziale Topologie auf $|\scr K|$ und die Teilraumtopologie auf $|\scr K| \subset V$ überein.

	Im Allgemeinen gilt das nicht, siehe obiges Beispiel.
\end{nt}


\subsection{Kombinatorische Simplizialkomplexe}

%fixme: drawing
\[
	K := \Set{ \emptyset, \{a\}, \{b\}, \{c\}, \{d\}, \{a,b\}, \{b,c\}, \{a,c\}, \{c,d\}, \{a,b,d\} }.
\]

\begin{df}
	Ein \emph{kombinatorischer Simplizialkomplex} ist ein System $K$ endlicher Mengen mit $\emptyset \in K$ und $T \subset S \in K \implies T \in K$.

	Wir vereinbaren $\dim S = |S| - 1$ und $\dim K := \sup \Set{ \dim S | S \in K }$.
	Die Eckenmenge ist
	\[
		\Omega(K) := \bigcup K := \bigcup_{S \in K} S.
	\]
\end{df}

\begin{df}
	Eine \emph{Darstellung} $f: K \to V$ ist eine Abbildung $f: \Omega(K) \to V$, so dass
	\begin{enumerate}[1)]
		\item
			für $S \in K$ ist $f(S) \subset V$ affin unabhängig,
		\item
			für $S, T \in K$ gilt $[f(S)] \cap [f(T)] = [f(S \cap T)]$.
	\end{enumerate}
	Damit ist $\scr K := K_f := \Set{ [f(S)] | S \in K }$ ein affiner Simplizialkomplex.
	Das Polyeder $|K|_f := |\scr K|$ heißt \emph{topologische Realisierung von $K$ mittels $f$}.
\end{df}

\begin{df}
	Im $\R$-Vektorraum $\R^\Omega$ ist die kanonische Basis $(e_s)_{s \in \Omega}$ gegeben durch $e_s: \Omega \to \R$ mit $e_s(s) = 1$ und $e_s(s') = 0$ für $s' \neq s$.
	Die Abbildung $f: \Omega \to \R^\Omega: s \mapsto e_s$ heißt \emph{kanonische Darstellung} von $K$.
	%fixme: prüfe: ist das eine darstellung?
	Der Komplex $\scr K = \Set{ [f(s)] | s \in K}$ heißt \emph{kanonischer affiner Simplizialkomplex zu $K$} und $|K| := |\scr K|$ heißt \emph{kanonische Realisierung} von $K$.
\end{df}

\begin{nt}
	Alternativ lässt sich die kanonische Realisierung auch angeben als
	\[
		|K| = \Set{ x: \Omega(K) \to [0,1] | \supp(x) \in K, \sum_{s \in \Omega} x(s) = 1 }.
	\]
\end{nt}

\begin{ex}
	Zu $D^n := \scr P({0, 1, \dotsc, n})$ ist die kanonische Realisierung der Standardsimplex $\Delta^n = [e_1, \dotsc, e_n] \subset \R^{n+1}$ zusammen mit all seinen Seiten.
\end{ex}

\begin{nt}
	Es exstiert $h: |K| \homto |K|_f$ mit
	%fixme: ref zur darstellung aus letzter bemerkung
	\[
		h(x) = \sum_{s\in \Omega} x(s) f(s)
	\]
	%fixme: proof
\end{nt}

\begin{df}
	Seien $K, L$ Simplizes.
	Eine simpliziale Abbildung $f: K \to L$ ist eine Abbildung $f: \Omega(K) \to \Omega(L)$ mit $S \in K \implies f(S) \in L$.
	Dies definiert eine Abbildung $|f|: |K| \to |L|$ durch affine Fortsetzung
	\[
		|f|:
		\sum_{s \in \Omega(K)} x(s) e_s
		\mapsto
		\sum_{s \in \Omega(K)} x(s) \underbrace{f(e_s)}_{= e_{f(s)}}.
	\]
\end{df}

\begin{nt}
	Wir haben folgenden Zusammenhang zwischen kombinatorische Simplizialkomplexe, affine Simplizialkomplexe und Topologie:
	% fixme: drawing
	Wir werden daher im folgenden nur allgemein von Simplizialkomplexen sprechen.
\end{nt}


\section{Triangulierung topologischer Räume}


\begin{df}
	Sei $X$ ein topologischer Raum.
	Eine \emph{Triangulierung} $(K, h)$ von $X$ ist ein Simplizialkomplex $K$ und ein Homöomorphismus $h: |K| \homto X$.
\end{df}

%fixme: drawing, tirangulierung von sphere und torus

\begin{ex}[Platonische Körper]
	%Platonische Körper sind konvexe Hüllen von endlich vielen Punkten.

	Tetraeder, Würfel, Octaeder, etc. sind \emph{Platonische Körper}.

	Dies sind \emph{Polytope}, bzw. \emph{polytopale Komplexe}.
	% fixme: definition of polytopaler Komplex
	% Simpliziale Komplexe \subset Polytopale Komplexe

	Hier sind baryzentrische Koordinaten nicht mehr unbedingt eindeutig, wie bei den Simplizes!
\end{ex}

\subsubsection{Kegelkonstruktion}

%fixme: drawing

\subsubsection{Zentrische Unterteilung}

%fixme: drawing

\coursetimestamp{10}{12}{2013}

Ein Oktaeder ist gegeben durch $[\pm e_1, \pm e_2, \pm e_3]$.
% fixme: drawing
Dies ist kein Simplex, aber kann in acht Simplizes aufgeteilt werden, indem der Ursprung hinzugenommen wird.

\begin{df}
	Sei $\scr K$ ein polytopaler Komplex.
	Eine \emph{Unterteilung} von $\scr K$ ist ein Komplex $\scr K'$ mit $|\scr K'| = |\scr K|$, sodass jedes Polytop $P \in \scr K$ Vereinigung von endlich vielen $P' \in \scr K'$ ist.
	\begin{note}
		(Lokale) Endlichkeit und Dimension bleibt erhalten.
	\end{note}
\end{df}

\begin{df}
	Sei $\scr L \subset \scr K$ ein Teilkomplex.
	Eine \emph{Unterteilung von $\scr K$ relativ zu $\scr L$} ist eine Unterteilung $\scr K'$ mit $\scr L \subset \scr K'$ (d.h. $P \in \scr L$ werden nicht unterteilt).
\end{df}

\begin{st}
	Sei $\scr K$ ein polytopaler Komplex und $\scr L \subset \scr K$ ein simplizialer Teilkomplex, etwa $\scr L = \{\emptyset\}$.
	Dann wird $\scr K$ trianguliert durch zentrische Unterteilung $\scr K'$ aller Polytope $P \in \scr K \setminus \scr L$, wobei wir zu jedem $P$ ein Zentrum $x_p \in \Int p$ willkürlich vorgeben können.
	\begin{proof}
		Induktion über die Skelette $\scr K_{\le 0}' := \scr K_{\le 0}$.
		Sei $n \ge 1$ und $\scr K'_{<n}$ eine simpliziale Unterteilung von $\scr K_{<n}$.
		Wir setzen dann
		\[
			\scr K_{\le n}'
			:= \scr K_{<n}' \cup
			\Set{ [v_0,\dotsc, v_k, x_p] |
				\begin{aligned}
					P \in \scr K \setminus L, \dim P = n, \\
					[v_0, \dotsc, v_k] \in \scr K'_{<n},
					[v_0, \dotsc, v_k] \subset \boundary P
				\end{aligned}}.
		\]
	\end{proof}
\end{st}

\begin{ex}
	\begin{itemize}
		\item
			Baryzentrische Unterteilung
		\item
			Zentrische Unterteilung mit beliebigen Zentren und $\scr L = \scr K_{\le 1}$
	\end{itemize}
\end{ex}

\begin{df}
	Sind $x_1, \dotsc, x_n$ die Eckpunkte des Polytops $P$, so heißt
	\[
		\beta(P) := \f 1n x_1 + \dotsb + \f 1n x_n
	\]
	\emph{das Baryzentrum von $P$}.
	Wählt man im Satz % fixme ref
	$x_P := \beta(P)$ für alle $P \in \scr K \setminus L$, so nennen wir $\beta(\scr K, \scr L) := (\scr K', \scr L)$ die \emph{baryzentrische Unterteilung} von $\scr K$ relativ zu $\scr L$.
\end{df}

\subsection{Produkte von Komplexen}

Produkte von Simplizes sind im Allgemeinen keine Simplizes ($\Delta^1 \times \Delta^1$ ergibt ein Quadrat, kein Dreieck).

\begin{st}
	Sind $\scr K \in V, \scr L \in W$ polytopale Komplexe, so auch
	\[
		\scr K \times \scr L = \Set{ P \times Q | P \in \scr K, Q \in \scr L }
	\]
	ein polytopaler Komplex in $V \times W$.
	Die baryzentrische Unterteilung $\scr M = (\scr K \times \scr L)'$ ist ein simplizialer Komplex und liefert eine Triangulierung $| \scr M| = | \scr K \times \scr L|$.

	Sind $\scr K, \scr L$ (lokal-)endlich, so auch $\scr K \times \scr L$ und $\scr M$.
	Die polytopale Topologie auf $|\scr K \times \scr L|$ stimmt mit der Produkttopologie auf $|\scr K| \times |\scr L|$ überein.
\end{st}

\subsection{Euler-Charakteristik}

\begin{df}[Euler-Charakteristik]
	Sei $\scr K$ ein endlicher polytopaler Komplex.
	Setze
	\[
		f_d = \Big| \Set{ P \in \scr K | \dim P = d } \Big|.
	\]
	Die \emph{Euler-Charakteristik} ist gegeben durch
	\[
		\chi(\scr K)
		:= \sum_{\emptyset \neq P \in \scr K}
		(-1)^{\dim P}.
	\]
\end{df}

\begin{ex}
	Der $n$-Simplex $\Delta^n \homeomorphic \D^n$ und sein Rand $\boundary \Delta^n \homeomorphic \S^{n-1}$ werden trianguliert durch
	\[
		D^n
		= \scr P({0,1, \dotsc, n})
	\]
	bzw.
	\[
		S^{n-1}
		= D^n \setminus \{ \{0,1,\dotsc,n\} \}.
	\]
	Es gilt
	\begin{align*}
		\chi(D^0) &= 1 &
		\phi(S^{-1}) &= 0 \\
		\chi(D^1) &= 2-1 = 1 &
		\phi(S^{0}) &= 2 \\
		\chi(D^2) &= 3 - 3 + 1 = 1 &
		\phi(S^{1}) &= 3 - 3 = 0 \\
		\chi(D^3) &= 4 - 6 + 4 - 1 = 1 &
		\phi(S^{2}) &= 4 - 6 + 4 = 2 \\
	\end{align*}
\end{ex}

\begin{prop}
	Es gilt
	\begin{align*}
		\chi(D^n) &= 1 &
		\chi(S^n) &= 1 + (-1)^n
	\end{align*}
	\begin{proof}
		Übung
	\end{proof}
\end{prop}

\begin{ex}
	Triangulierungen für $\S^2$:
	\begin{description}
		\item
			$\chi(\delta \text{Tetraeder}) = 4 - 6 + 4 = 2$
		\item
			$\chi(\delta \text{Oktaeder}) = 6 - 12 + 8 = 2$
		\item
			$\chi(\delta \text{Hexaeder}) = 8 - 12 + 6 = 2$
		\item
			$\chi(\delta \text{Dodekaeder}) = 20 - 30 + 12= 2$
		\item
			$\chi(\delta \text{Ikosaeder}) = 12 - 30 + 20= 2$
	\end{description}
\end{ex}

\begin{st}[Satz von Euler]
	Jede Triangulierung (oder polytopale Unterteilung) von $\scr S^2$ erfüllt $\chi(\scr K) = 2$.
\end{st}

\begin{st}[Euler-Poincaré]
	Seien $\scr K, \scr L$ endliche polytopale Komplexe.
	Aus $|\scr K| \homeomorphic |\scr L$ folgt $\chi(\scr K) = \chi(\scr L)$.
	\begin{note}
		Mit Hilfe diesen Satzes lässt sich die Euler-Charakteristik eines Polytops als diejenige einer beliebigen Triangulierung definieren.
	\end{note}
\end{st}


\section{Simpliziale Approximation}


\begin{df}
	Sei $K$ ein (kombinatorischer) Simplizialkomplex mit kanonischer Realisierung $|K| \subset \R^{\Omega}$.
	Definiere die \emph{simpliziale Metrik} auf der Realisierung als
	\[
		d(x,y) :=
		\max \Set{ |x(s) - y(s)| : s \in \Omega }
	\]
\end{df}

\begin{st}
	Die simpliziale Topologie auf $|K|$ ist feiner als die metrische Topologie.
	Bei stimmen überein, wenn $K$ lokal-endlich ist.
	\begin{proof}
		Leicht nachzuvollziehen
	\end{proof}
\end{st}

\begin{df}
	Zu $a \in \Omega(K)$ betrachten wir den \emph{offenen Stern}
	\begin{align*}
		\mathring \stern_K(a)
		&:= \bigcup \Set{ \Int |S| : a \in S \in K } \\
		&= \Set{ x \in |K| : x(a) > 0 }
		= B(a, 1)
	\end{align*}
	und den \emph{abgeschlossenen Stern}
	\begin{align*}
		\_\stern_K(a)
		&:= \bigcup \Set{ |S| : a \in S \in K }.
	\end{align*}
\end{df}

\begin{st}
	Jede Ecke $a \in \Omega(K)$ ist starker Deformationsretrakt von $\_\stern_K(a) \subset |K|$, ebenso von $B(a, \eps)$ für $0 < \eps \le 1$.
	Jeder Punkt $a \in |K|$ ist starker Deformationsretrakt einer offenen Umgebung $a \in U \subset |K|$.
	\begin{proof}
		Wähle $H: [0,1] \times \_\stern_K(a) \to \_\stern_K(a)$ durch
		\[
			H(t, x) = ta + (1-t) x.
		\]
		Dies ist wohldefiniert, da $\_\stern_K(a)$ sternförmig und stetig, da stetig auf jedem Simplex.
		Es gilt $H_0 = \Id$ und $H_1 = \const_{\_\stern_K(a)}^a$, sowei $H(t,a) = a$ für alle $t \in [0,1]$.

		Allgemein für $a \in |K|$ wählen wir eine zentrische Zerlegung mit $a$ als Ecke.
	\end{proof}
\end{st}

\begin{lem}
	Jeder $n$-Simplex $\Delta = [v_0, \dotsc, v_n]$ realisiert seinen Durchmesser auf den Ecken:
	\[
		\diam \Delta := \max \Set{ d(v_i, v_j) | i,j = 0, \dotsc, n }.
	\]
	Nach baryzentrischer Unterteilung gilt für jeden Teilsimplex $\Delta' \subset \Delta$ die Ungleichung
	\[
		\diam \Delta' \le \f n{n-1} \diam \Delta.
	\]
\end{lem}

\begin{st}
	Sei $K$ ein Simplizialkomplex kanonisch realisiert durch $\scr K$ in $\R^\Omega$.
	Sei $|\scr K| = \bigcup_{i\in I} U_i$ eine Überdeckung durch offene Mengen $U_i \subset |\scr K|$.

	Dann existiert eine simpliziale Unterteilung $\scr K'$ des Polyeders $|\scr K| = |\scr K'|$, sodass für jede Ecke $a \in \Omega(\scr K')$ der abgeschlossene Stern $\_\stern_{\scr K'}(a)$ ganz in einer Menge $U_i$ enthalten ist.
\end{st}
