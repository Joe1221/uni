%\part{Geometrische Topologie}

\chapter{Simpliziale Komplexe}
\coursetimestamp{09}{12}{2013}

\section{Simplizialkomplexe}

\subsection{Simplizes}

\begin{df}
	Sei $V$ ein $\R$-Vektorraum und $v_0, v_1, \dotsc, v_n \in V$.
	Wir nennen $v_0, v_1, \dotsc, v_n$ \emph{affin unabhängig}, wenn $v_1 - v_0, v_2 - v_0, \dotsc, v_n - v_0$ linear unabhängig sind.
	\begin{note}
		Die Wahl von $v_0$ ist willkürlich und unerheblich, äquivalent wäre: $v_0 - v_k, \dotsc, v_{k-1} - v_k, v_{k+m} - v_k, \dotsc, v_n - v_k$ linear unabhängig für beliebiges $k \in \N$.
	\end{note}
\end{df}

\begin{prop} \label{prop:affine_independent}
	Sei $V$ ein $\R$-Vektorraum.
	$v_0, v_1, \dotsc, v_n \in V$ sind affin unabhängig genau dann, wenn aus $\sum_{i=0}^n t_i v_i = \sum_{i=0}^n t_i' v_i$ und $\sum_{i=0}^n t_i = \sum_{i=0}^n t_i'$ für alle $i \in I$ folgt $t_i = t_i'$.
	\begin{proof}
		\begin{segnb}[„$\implies$“]
			Es gilt
			\[
				0
				= \sum_{i=0}^n (t_i' - t_i) v_i - \overbrace{\sum_{i=0}^n (t_i' - t_i) v_0}^{= 0}
				= \sum_{i=1}^n (t_i' - t_i) (v_i - v_0),
			\]
			wegen der affinen Unabhängigkeit ist $\{v_i - v_0\}_{i=1}^n$ linear unabhängig, also $t_i' = t_i$ für alle $i \in \{1, \dotsc, n\}$ und somit dank $\sum_{i=0}^n t_i = \sum_{i=0}^n t_i'$ auch $t_0' = t_0$.
		\end{segnb}
		\begin{segnb}[„$\implies$“]
			Sei $\sum_{i=1}^n \lambda_i (v_i - v_0) = 0$.
			Setze $t_i = 0$ für $i \in \{0, \dotsc, n\}, t_0' := -\lambda_1 - \lambda_2 - \dotsb - \lambda_n$ und $t_i := \lambda_i$ für $i \in \{1, \dotsc, n\}$.
			Dann ist
			\begin{align*}
				\sum_{i=0}^n t_i = 0 &= \sum_{i=0}^n t_i' \\
				\sum_{i=0}^n t_i v_i = 0 &= \sum_{i=1}^n \lambda_i (v_i - v_0) = \sum_{i=0}^n t_i' v_i
			\end{align*}
			also $t_i' = t_i = 0$ und somit auch $\lambda_i = 0$.
		\end{segnb}
	\end{proof}
\end{prop}

\begin{df}[Standardsimplex]
	Für $n \in \N_0$ ist der \emph{$n$-dimensionale Standardsimplex} definiert als der Raum
	\[
		\Delta^n := \Set{ (t_0, \dotsc, t_n) \in \R^{n+1} | t_0, \dotsc, t_n \ge 0, t_0 + \dotsb + t_n = 1 }.
	\]
	Wir definieren außerdem $\Delta^{-1} := \emptyset$.
\end{df}

\begin{conv}
	Im Folgenden sei $V$ stets ein $\R$-Vektorraum und $v_0, \dotsc, v_n \in V$ affin unabhängig.
\end{conv}

\begin{df}[affiner Simplex]
	Die konvexe Hülle von $v_0, \dotsc, v_n$ ist der \emph{affine $n$-Simplex}
	\begin{align*}
		\Delta
		&:= [v_0, v_1, \dotsc, v_n] \\
		&:= \Set{ t_0v_0 + t_1v_1 + \dotsb + t_nv_n | (t_0, \dotsc, t_n) \in \Delta^n }.
	\end{align*}
	%fixme: ungeschickte notation mit \Delta (dimension? v_i?)
	Im Kontext von $\Delta$ nennen wir $t_0, t_1, \dotsc, t_n$ auch \emph{baryzentrische Koordinaten}.

	Der affine $(-1)$-Simplex sei $[\emptyset] := \emptyset$.

	Außerdem definieren wir die Abbildungen $\vert: \Delta \to \scr P(V), \dim: \Delta \to \N_0$ durch
	\begin{align*}
		\vert: [v_0, \dotsc, v_n] &\mapsto \{v_0, \dotsc, v_n\}, \\
		\dim: [v_0, \dotsc, v_n] &\mapsto n.
	\end{align*}
	\begin{note}
		Nach \ref{prop:affine_independent} sind die baryzentrischen Koordinaten eindeutig.
	\end{note}
\end{df}

\begin{conv}
	Soweit nicht anders angemerkt sei $\Delta := [v_0, v_1, \dotsc, v_n]$ stets der von $v_0, \dotsc, v_n$ aufgespannte affine Simplex.

	Das Attribut „affin“ wird bei Simplizes oft weggelassen.
\end{conv}

\begin{ex}
	\begin{itemize}
		\item
			Der $n$-dimensionale Standardsimplex ist ein affiner $n$-Simplex:
			\[
				\Delta^n = [e_0, \dotsc, e_n] \subset \R^{n+1}
			\]
		\item
			$\Delta^n \subset \R^{n+1}$ ist gemäß \ref{nt:standard_simplex_affine_simplex_homeomorphism} homöomorph zu
			\begin{align*}
				[0,e_1,\dotsc,e_n]
				&= \Set{ (t_1, \dotsc, t_n) \in \R^n \;\Big|\; t_1, \dotsc, t_n \ge 0, t_1 + \dotsb + t_n \le 1 } \\
				&\subset \R^n.
			\end{align*}
	\end{itemize}
\end{ex}

\begin{nt} \label{nt:standard_simplex_affine_simplex_homeomorphism}
	Der Standardsimplex $\Delta^n$ und der affine $n$-Simplex $\Delta$ sind homöomorph vermöge $h: \Delta^n \homto \Delta$,
	\[
		h(t_0, \dotsc, t_n) = t_0 v_0 + \dotsb + t_n v_n.
	\]
	\begin{proof}
		Surjektivität folgt direkt aus der Definition, Injektivität aus der Eindeutigkeit der baryzentrischen Koordinaten und die Homöomorphie damit mittels \ref{st:euclidean_linear_space_topological_space_linear_function}.
	\end{proof}
\end{nt}

\begin{df}
	\begin{enumerate}[1)]
		\item
			Für $S \subset \vert(\Delta)$ heißt $[S]$ eine \emph{Seite} von $\Delta$, kurz $[S] \le \Delta$.

			Eine Seite $[S] \le \Delta$ heißt \emph{echt}, kurz $[S] < \Delta$, falls $[S] \neq \Delta$.
		\item
			Der \emph{Rand} von $\Delta$ ist $\boundary \Delta := \bigcup_{[S] < \Delta} [S]$, das \emph{Innere} ist $\Int \Delta := \Delta \setminus \boundary \Delta$.
	\end{enumerate}
	\begin{note}
		Rand und Inneres sind nur dann gleichbedeutend mit den bisherigen topologischen Definitionen, wenn wir die Topologie des affinen Vektorraums zugrundelegen und nicht die des gesamten Vektorraums.

		So ist beispielsweise für $\Delta := [0,e_1] \subset \R^2$ der Rand des Simplex
		\[
			\boundary \Delta
			= \big\{(0,0), (1,0)\big\}
			= \boundary_{\scr T_\Delta} \Delta,
		\]
		während $\boundary_{\scr T_V} \Delta = \Delta$.
	\end{note}
\end{df}


\subsection{Affine Simplizialkomplexe}

Im Folgenden wollen wir komplexere Gebilde, bestehend aus Simplizes, betrachten und beschreiben.
Diese sogenannten Simplizialkomplexe werden im Wesentlichen durch die Simplizes beschrieben aus denen sie bestehen, inklusive Simplizes niederer Dimensionen, welche beispielsweise gemeinsame Kanten oder Ecken beschreiben.

% fixme: Zerlegung eines Quadrats in verschiedene Simplizes, auch kompliziertere Gebilde (z.b. Octaeder)

\begin{df}[affiner Simplizialkomplex]
	Ein \emph{(affiner) Simplizialkomplex} $\scr K$ in $V$ ist eine Menge von affinen Simplizes $\Delta \subset V$, mit den Eigenschaften
	\begin{enumerate}[1), start=0]
		\item
			$\emptyset \in \scr K$,
		\item
			$\Delta' \le \Delta \in \scr K \implies \Delta' \in \scr K$,
		\item
			$\Delta_1, \Delta_2 \in \scr K \implies \Delta_1 \cap \Delta_2 \le \Delta_1, \Delta_2$.
	\end{enumerate}
	Wir vereinbaren die \emph{Dimension}
	\begin{align*}
		\dim \scr K &:= \sup \Set{ \dim \Delta | \Delta \in \scr K },
	\intertext{
		die \emph{Eckenmenge}
	}
		\Omega(\scr K) &:= \bigcup \Set{ \vert \Delta | \Delta \in \scr K },
	\intertext{
		und den \emph{Polyeder}
	}
		|\scr K| &:= \bigcup \scr K = \bigcup \Set{ \Delta | \Delta \in \scr K }.
	\end{align*}
	Jedes Simplex $\Delta \in \scr K$ versehen wir mit seiner euklidischen Topologie.

	Das Polyeder $|\scr K|$ versehen wir mit seiner \emph{simplizialen Topologie}:
	eine Menge $U \subset |\scr K|$ sei offen, wenn für jeden Simplex $\Delta \in \scr K$ die Menge $U \cap \Delta$ offen in $\Delta$ ist.
\end{df}

\begin{ex}[Gegenbeispiele]
	% fixme: ergänzen
\end{ex}

\begin{ex}
	\begin{itemize}
		\item
			Für eine Eckenmenge $\Omega$ ist
			\[
				\scr K = \{ \emptyset \} \cup \big\{ \{a\} : a \in \Omega \big\}
			\]
			ein $0$-dimensionaler Simplizialkomplex.
			Es gilt $|\scr K| = \Omega$ und die simpliziale Topologie stimmt mit der diskreten überein.
		\item
			Ein Simplizialkomplex $\scr K$ mit $\dim \scr K \le 1$ heißt \emph{simplizialer Graph}.
			Siehe für Details \coursehref{Blatt09.pdf}{Übungsblatt 9, Teil 3}.
		\item
			Die simpliziale Sinuskurve des Topologen ($A$ bestehend aus abzählbar vielen geraden Strecken entlang der Sinuskurve, $B$ als die vertikale Strecke).
			Hier sind $A$ und $B$ disjunkt und beide offen (in \ref{ex:topologists_sine_curve} war $B$ nicht offen in $A \cup B$).
	\end{itemize}
\end{ex}

\begin{nt}
	Ist $\scr K$ in $V$ (lokal-)endlich, so stimmen die simpliziale Topologie auf $|\scr K|$ und die Teilraumtopologie auf $|\scr K| \subset V$ überein.

	Im Allgemeinen gilt das nicht, siehe obiges Beispiel.
\end{nt}


\subsection{Kombinatorische Simplizialkomplexe}

Es liegt nahe, dass ein Simplizialkomplex auch beschrieben werden könnte durch Angabe der Eckpunkte und der Information, welche $k$ Eckpunkte zu affinen $(k-1)$-Simplizes verbunden werden sollen.
Beispielsweise
\[
	K := \Big\{ \emptyset, \{a\}, \{b\}, \{c\}, \{d\}, \{a,b\}, \{b,c\}, \{a,c\}, \{c,d\}, \{a,b,d\} \Big\}.
\]
%fixme: drawing
Auf diese Art gegebene Simplizialkomplexe nennen wir kombinatorische Simplizialkomplexe.

\begin{df}[kombinatorischer Simplizialkomplex]
	Ein \emph{kombinatorischer Simplizialkomplex} ist ein System $K$ endlicher Mengen mit den Eigenschaften
	\begin{enumerate}[1),start=0]
		\item
			$\emptyset \in K$,
		\item
			$T \subset S \in K \implies T \in K$.
	\end{enumerate}
	Wir vereinbaren für $S \in K$ die Dimension $\dim S = |S| - 1$ und für $K$: $\dim K := \sup \Set{ \dim S | S \in K }$, sowie die Eckenmenge
	\[
		\Omega(K) := \bigcup K := \bigcup_{S \in K} S.
	\]
\end{df}

\begin{conv}
	Im Folgenden sei $K$ soweit nicht anders festgelegt stets ein kombinatorischer Simplizialkomplex mit Eckenmenge $\Omega = \Omega(K)$.
\end{conv}

\begin{df}[Darstellung, Realisierung]
	Eine \emph{Darstellung} $f: K \to V$ ist eine Abbildung $f: \Omega(K) \to V$, so dass
	\begin{enumerate}[1)]
		\item
			für $S \in K$ ist $f(S) \subset V$ affin unabhängig,
		\item
			für $S, T \in K$ gilt $[f(S)] \cap [f(T)] = [f(S \cap T)]$.
	\end{enumerate}
	Damit ist $\scr K := K_f := \Set{ [f(S)] | S \in K }$ ein affiner Simplizialkomplex.
	Das Polyeder $|K|_f := |\scr K|$ heißt \emph{topologische Realisierung von $K$ mittels $f$}.
\end{df}

\begin{df}[kanonische Realisierung]
	Sei im $\R$-Vektorraum $\R^\Omega$ die kanonische Basis $(e_s)_{s \in \Omega}$ gegeben durch $e_s: \Omega \to \R$ mit $e_s(s) := 1$ und $e_s(s') := 0$ für $s' \neq s$.

	Die Abbildung $f: \Omega \to \R^\Omega: s \mapsto e_s$ heißt \emph{kanonische Darstellung} von $K$.
	Der Komplex $\scr K := \Set{ [f(s)] | s \in K}$ heißt \emph{kanonischer affiner Simplizialkomplex zu $K$} und $|K| := |\scr K|$ heißt \emph{kanonische Realisierung} von $K$.
	%fixme: Aussagen beweisen?
	\begin{note}
		Wir können die kanonische Realisierung auch folgendermaßen charakterisieren:
		\[
			|K| = \Set{ x: \Omega \to [0,1] | \supp(x) \in K, \sum_{s \in \Omega} x(s) = 1 }.
		\]
		% fixme: warum ist dies äquivalent?
		Dies erlaubt die Beschreibung eines Punktes $x \in |K|$ durch eindeutige baryzentrische Koordinaten.
	\end{note}
\end{df}


\begin{ex}
	Zu $D^n := \scr P(\{0, 1, \dotsc, n\})$ ist die kanonische Realisierung das Standardsimplex $\Delta^n = [e_0, \dotsc, e_n] \subset \R^{n+1}$ zusammen mit all seinen Seiten.
\end{ex}

\begin{nt}
	Für jede beliebige Darstellung $f$ von $K$ exstiert ein Homöomorphismus $h: |K| \homto |K|_f$ zwischen der kanonischen Realisierung und der Realisierung durch $f$, definiert durch
	\[
		h(x) = \sum_{s\in \Omega} x(s) f(s)
	\]
	\begin{proof}
		Funktioniert wie in \ref{nt:standard_simplex_affine_simplex_homeomorphism}.
	\end{proof}
\end{nt}

\begin{df}
	Seien $K, L$ Simplizes.
	Eine \emph{simpliziale Abbildung} $f: K \to L$ ist eine Abbildung $f: \Omega(K) \to \Omega(L)$ mit $S \in K \implies f(S) \in L$.

	Dies definiert außerdem eine Abbildung $\real f = |f|: |K| \to |L|$ durch affine Fortsetzung
	\[
		|f|:
		\sum_{s \in \Omega(K)} x(s) e_s
		\mapsto
		\sum_{s \in \Omega(K)} x(s) \underbrace{f(e_s)}_{= e_{f(s)}}.
	\]
	% fixme: etwas unklar
\end{df}

\begin{nt}
	Kombinatorische Simplizialkomplexe mit ihren simplizialen Abbildungen bilden die Kategorie $\Cat{SComp}$, ebenso wie affine Simplizialkomplexe mit ihren simplizialen Abbildungen die Kategorie $\Cat{AComp}$ bilden.
	Die Funktoren $\mathrm{real}$ für die Realisierung und $\vert$ als Abbildung auf die Eckpunkte bilden folgenden Zusammenhang.
	\[
		\begin{tikzcd}
			\Cat{SComp} \arrow{rr}{\mathrm{real}} & & \Cat{AComp} \arrow{ll}{\vert}
		\end{tikzcd}
	\]
	% fixme: proof, real & vert -> Id
	%Wir werden daher im folgenden nur allgemein von Simplizialkomplexen sprechen.
\end{nt}


\section{Triangulierung topologischer Räume}


\begin{df}
	Sei $X$ ein topologischer Raum.
	Eine \emph{Triangulierung} $(K, h)$ von $X$ ist ein Simplizialkomplex $K$ und ein Homöomorphismus $h: |K| \homto X$.
\end{df}

%fixme: drawing, tirangulierung von sphere und torus

\begin{ex}[Platonische Körper]
	%Platonische Körper sind konvexe Hüllen von endlich vielen Punkten.

	Tetraeder, Würfel, Octaeder, etc. sind \emph{Platonische Körper}.

	Dies sind \emph{Polytope}, bzw. \emph{polytopale Komplexe}.
	% fixme: definition of polytopaler Komplex
	% Simpliziale Komplexe \subset Polytopale Komplexe

	Hier sind baryzentrische Koordinaten nicht mehr unbedingt eindeutig, wie bei den Simplizes!
\end{ex}

\subsubsection{Kegelkonstruktion}

%fixme: drawing

\subsubsection{Zentrische Unterteilung}

%fixme: drawing

\coursetimestamp{10}{12}{2013}

Ein Oktaeder ist gegeben durch $[\pm e_1, \pm e_2, \pm e_3]$.
% fixme: drawing
Dies ist kein Simplex, aber kann in acht Simplizes aufgeteilt werden, indem der Ursprung hinzugenommen wird.

\begin{df}
	Sei $\scr K$ ein polytopaler Komplex.
	Eine \emph{Unterteilung} von $\scr K$ ist ein Komplex $\scr K'$ mit $|\scr K'| = |\scr K|$, sodass jedes Polytop $P \in \scr K$ Vereinigung von endlich vielen $P' \in \scr K'$ ist.
	\begin{note}
		(Lokale) Endlichkeit und Dimension bleibt erhalten.
	\end{note}
\end{df}

\begin{df}
	Sei $\scr L \subset \scr K$ ein Teilkomplex.
	Eine \emph{Unterteilung von $\scr K$ relativ zu $\scr L$} ist eine Unterteilung $\scr K'$ mit $\scr L \subset \scr K'$ (d.h. $P \in \scr L$ werden nicht unterteilt).
\end{df}

\begin{st}
	Sei $\scr K$ ein polytopaler Komplex und $\scr L \subset \scr K$ ein simplizialer Teilkomplex, etwa $\scr L = \{\emptyset\}$.
	Dann wird $\scr K$ trianguliert durch zentrische Unterteilung $\scr K'$ aller Polytope $P \in \scr K \setminus \scr L$, wobei wir zu jedem $P$ ein Zentrum $x_p \in \Int p$ willkürlich vorgeben können.
	\begin{proof}
		Induktion über die Skelette $\scr K_{\le 0}' := \scr K_{\le 0}$.
		Sei $n \ge 1$ und $\scr K'_{<n}$ eine simpliziale Unterteilung von $\scr K_{<n}$.
		Wir setzen dann
		\[
			\scr K_{\le n}'
			:= \scr K_{<n}' \cup
			\Set{ [v_0,\dotsc, v_k, x_p] |
				\begin{aligned}
					P \in \scr K \setminus L, \dim P = n, \\
					[v_0, \dotsc, v_k] \in \scr K'_{<n},
					[v_0, \dotsc, v_k] \subset \boundary P
				\end{aligned}}.
		\]
	\end{proof}
\end{st}

\begin{ex}
	\begin{itemize}
		\item
			Baryzentrische Unterteilung
		\item
			Zentrische Unterteilung mit beliebigen Zentren und $\scr L = \scr K_{\le 1}$
	\end{itemize}
\end{ex}

\begin{df}
	Sind $x_1, \dotsc, x_n$ die Eckpunkte des Polytops $P$, so heißt
	\[
		\beta(P) := \f 1n x_1 + \dotsb + \f 1n x_n
	\]
	\emph{das Baryzentrum von $P$}.
	Wählt man im Satz % fixme ref
	$x_P := \beta(P)$ für alle $P \in \scr K \setminus L$, so nennen wir $\beta(\scr K, \scr L) := (\scr K', \scr L)$ die \emph{baryzentrische Unterteilung} von $\scr K$ relativ zu $\scr L$.
\end{df}

\subsection{Produkte von Komplexen}

Produkte von Simplizes sind im Allgemeinen keine Simplizes ($\Delta^1 \times \Delta^1$ ergibt ein Quadrat, kein Dreieck).

\begin{st}
	Sind $\scr K \in V, \scr L \in W$ polytopale Komplexe, so auch
	\[
		\scr K \times \scr L = \Set{ P \times Q | P \in \scr K, Q \in \scr L }
	\]
	ein polytopaler Komplex in $V \times W$.
	Die baryzentrische Unterteilung $\scr M = (\scr K \times \scr L)'$ ist ein simplizialer Komplex und liefert eine Triangulierung $| \scr M| = | \scr K \times \scr L|$.

	Sind $\scr K, \scr L$ (lokal-)endlich, so auch $\scr K \times \scr L$ und $\scr M$.
	Die polytopale Topologie auf $|\scr K \times \scr L|$ stimmt mit der Produkttopologie auf $|\scr K| \times |\scr L|$ überein.
\end{st}

\subsection{Euler-Charakteristik}

\begin{df}[Euler-Charakteristik]
	Sei $\scr K$ ein endlicher polytopaler Komplex.
	Setze
	\[
		f_d = \Big| \Set{ P \in \scr K | \dim P = d } \Big|.
	\]
	Die \emph{Euler-Charakteristik} ist gegeben durch
	\[
		\chi(\scr K)
		:= \sum_{\emptyset \neq P \in \scr K}
		(-1)^{\dim P}.
	\]
\end{df}

\begin{ex}
	Der $n$-Simplex $\Delta^n \homeomorphic \D^n$ und sein Rand $\boundary \Delta^n \homeomorphic \S^{n-1}$ werden trianguliert durch
	\[
		D^n
		= \scr P({0,1, \dotsc, n})
	\]
	bzw.
	\[
		S^{n-1}
		= D^n \setminus \{ \{0,1,\dotsc,n\} \}.
	\]
	Es gilt
	\begin{align*}
		\chi(D^0) &= 1 &
		\phi(S^{-1}) &= 0 \\
		\chi(D^1) &= 2-1 = 1 &
		\phi(S^{0}) &= 2 \\
		\chi(D^2) &= 3 - 3 + 1 = 1 &
		\phi(S^{1}) &= 3 - 3 = 0 \\
		\chi(D^3) &= 4 - 6 + 4 - 1 = 1 &
		\phi(S^{2}) &= 4 - 6 + 4 = 2 \\
	\end{align*}
\end{ex}

\begin{prop}
	Es gilt
	\begin{align*}
		\chi(D^n) &= 1 &
		\chi(S^n) &= 1 + (-1)^n
	\end{align*}
	\begin{proof}
		Übung
	\end{proof}
\end{prop}

\begin{ex}
	Triangulierungen für $\S^2$:
	\begin{description}
		\item
			$\chi(\delta \text{Tetraeder}) = 4 - 6 + 4 = 2$
		\item
			$\chi(\delta \text{Oktaeder}) = 6 - 12 + 8 = 2$
		\item
			$\chi(\delta \text{Hexaeder}) = 8 - 12 + 6 = 2$
		\item
			$\chi(\delta \text{Dodekaeder}) = 20 - 30 + 12= 2$
		\item
			$\chi(\delta \text{Ikosaeder}) = 12 - 30 + 20= 2$
	\end{description}
\end{ex}

\begin{st}[Satz von Euler]
	Jede Triangulierung (oder polytopale Unterteilung) von $\scr S^2$ erfüllt $\chi(\scr K) = 2$.
\end{st}

\begin{st}[Euler-Poincaré]
	Seien $\scr K, \scr L$ endliche polytopale Komplexe.
	Aus $|\scr K| \homeomorphic |\scr L$ folgt $\chi(\scr K) = \chi(\scr L)$.
	\begin{note}
		Mit Hilfe diesen Satzes lässt sich die Euler-Charakteristik eines Polytops als diejenige einer beliebigen Triangulierung definieren.
	\end{note}
\end{st}


\section{Simpliziale Approximation}


\begin{df}
	Sei $K$ ein (kombinatorischer) Simplizialkomplex mit kanonischer Realisierung $|K| \subset \R^{\Omega}$.
	Definiere die \emph{simpliziale Metrik} auf der Realisierung als
	\[
		d(x,y) :=
		\max \Set{ |x(s) - y(s)| : s \in \Omega }
	\]
\end{df}

\begin{st}
	Die simpliziale Topologie auf $|K|$ ist feiner als die metrische Topologie.
	Bei stimmen überein, wenn $K$ lokal-endlich ist.
	\begin{proof}
		Leicht nachzuvollziehen
	\end{proof}
\end{st}

\begin{df}
	Zu $a \in \Omega(K)$ betrachten wir den \emph{offenen Stern}
	\begin{align*}
		\mathring \star_K(a)
		&:= \bigcup \Set{ \Int |S| : a \in S \in K } \\
		&= \Set{ x \in |K| : x(a) > 0 }
		= B(a, 1)
	\end{align*}
	und den \emph{abgeschlossenen Stern}
	\begin{align*}
		\_\star_K(a)
		&:= \bigcup \Set{ |S| : a \in S \in K }.
	\end{align*}
\end{df}

\begin{st}
	Jede Ecke $a \in \Omega(K)$ ist starker Deformationsretrakt von $\_\star_K(a) \subset |K|$, ebenso von $B(a, \eps)$ für $0 < \eps \le 1$.
	Jeder Punkt $a \in |K|$ ist starker Deformationsretrakt einer offenen Umgebung $a \in U \subset |K|$.
	\begin{proof}
		Wähle $H: [0,1] \times \_\star_K(a) \to \_\star_K(a)$ durch
		\[
			H(t, x) = ta + (1-t) x.
		\]
		Dies ist wohldefiniert, da $\_\star_K(a)$ sternförmig und stetig, da stetig auf jedem Simplex.
		Es gilt $H_0 = \Id$ und $H_1 = \const_{\_\star_K(a)}^a$, sowei $H(t,a) = a$ für alle $t \in [0,1]$.

		Allgemein für $a \in |K|$ wählen wir eine zentrische Zerlegung mit $a$ als Ecke.
	\end{proof}
\end{st}

\begin{lem}
	Jeder $n$-Simplex $\Delta = [v_0, \dotsc, v_n]$ realisiert seinen Durchmesser auf den Ecken:
	\[
		\diam \Delta := \max \Set{ d(v_i, v_j) | i,j = 0, \dotsc, n }.
	\]
	Nach baryzentrischer Unterteilung gilt für jeden Teilsimplex $\Delta' \subset \Delta$ die Ungleichung
	\[
		\diam \Delta' \le \f n{n-1} \diam \Delta.
	\]
\end{lem}

\begin{st}
	Sei $K$ ein Simplizialkomplex kanonisch realisiert durch $\scr K$ in $\R^\Omega$.
	Sei $|\scr K| = \bigcup_{i\in I} U_i$ eine Überdeckung durch offene Mengen $U_i \subset |\scr K|$.

	Dann existiert eine simpliziale Unterteilung $\scr K'$ des Polyeders $|\scr K| = |\scr K'|$, sodass für jede Ecke $a \in \Omega(\scr K')$ der abgeschlossene Stern $\_\star_{\scr K'}(a)$ ganz in einer Menge $U_i$ enthalten ist.
\coursetimestamp{16}{12}{2013}
	\begin{proof}
		Wir zeigen den Satz für endliches $K$.
		Dann ist $|\scr K|$ kompakt und die simpliziale Topologie wird von der simplizialen Metrik induziert.
		Es existiert zu $(U_i)_{i\in I}$ eine Lebesgue-Zahl $\lambda > 0$, d.h. für $X \subset |\scr K|$ mit $\diam X < \lambda$ existiert $i \in I$ mit $X \subset U_i$.
		Sei $n = \dim \scr K$ und $q = \f n{n+1} < 1$.
		Wir wählen $m \in \N$ so, dass $q^m < \f \lambda 2$.
		In der $m$-fachen baryzentrischen Unterteilung $\scr K' = \beta^m \scr K$ hat jeder Simplex $\Delta \in \scr K$ Durchmesser $\diam(\Delta) < \f \lambda 2$, also jeder abgeschlossene Stern Durchmesser $< \lambda$.
	\end{proof}
\end{st}

\begin{nt}
	Seien $\scr K, \scr L$ simpliziale Komplexe und $\phi: K \to L$ die simpliziale Abbildung mit affiner Fortsetzung $|\phi|: |K| \to |L|$.
	Es gilt
	\begin{align*}
		|\phi|(\mathring \star(a, K)) &\subset \mathring \star (\phi(a), L), \\
		|\phi|(\_\star(a,K)) &\subset \_\star(\phi(a), L).
	\end{align*}
\end{nt}

\begin{df}
	Sei $f: |K| \to |L|$ stetig.
	Eine \emph{simpliziale Approximation von $f$} ist eine Abbildung $\phi: \Omega(K) \to \Omega(L)$ mit
	\[
		f(\mathring \star(a, K)) \subset \mathring(\phi(a), L)
	\]
	für alle $a \in \Omega(K)$.
\end{df}

\begin{lem}
	\begin{enumerate}[(1)]
		\item
			$\phi$ ist simplizial, d.h. für $S \in K \implies \phi(S) \in L$.
		\item
			Für jedes $x \in |K|$ liegen $f(x)$ und $|\phi|(x)$ in einem gemeinsamen Simplex von $|L|$.
		\item
			Die Abbildungen $f$ und $|\phi|$ sind homotop vermöge
			\[
				H(t, x) = (1 - t) f(x) + t |\phi|(x).
			\]
	\end{enumerate}
	\begin{proof}
		Übung
	\end{proof}
\end{lem}

\begin{st}
	Jede stetige Abbildung $f: |K| \to |L|$ ist homotop zu einer Abbildung $g: |K| = |K'| \to |L|$, die simplizial ist auf einer hinreichend feinen Unterteilung $K'$ von $K$.
	\begin{proof}
		$|L| = \bigcup_{b\in \Omega(L)} \mathring\star(b,L)$ ist eine offen Überdeckung von $|L|$ und damit
		\[
			|K| = \bigcup_{b\in\Omega(L)} f^{-1}(\mathring\star(b,L))
		\]
		ebenfalls.
		Es existiert eine Unterteilung $K'$, sodass für jede Ecke $a \in \Omega(K')$ ein $\phi(a) \in \Omega(L)$ existiert mit $\_\star(a, K') \subset f^{-1}(\star(\phi(a),L))$.
		Also
		\[
			f(\mathring \star(a,K'))
			\subset f(\_\star(a,K'))
			\subset \mathring\star(\phi(a),L).
		\]
	\end{proof}
\end{st}

\begin{st}
	Für $m < n$ ist jede stetige Abbildung $f: \S^m \to \S^n$ zusammenziehbar.
	\begin{proof}
		Wir triangulieren $|S^m| \homto \S^m$ und $|S^n| \homto \S^n$.
		Es genügt also zu zeigen: Jede stetige Abbildung $f: |S^m| \to |S^n|$ ist zusammenziehbar.
		Zu einer hinreichend feinen Unterteilung $K$ von $S^m$ existiert eine simpliziale Approximation $\phi: K \to S^n$ mit $f \homto |\phi|: |S^m| = |K'| \to |S^n|$.
		Wegen $\dim K = \dim S^m = m < n = \dim S^n$ ist $|\phi|$ nicht surjektiv, somit $|\phi| \homto *$.
		Erinnerung:
		\[
			|S^n| \setminus \{q\} \homeomorphic \S^n \setminus \{p\} \homeomorphic \R^n \homotopic *.
		\]
	\end{proof}
\end{st}
