% D
\chapter{Kompaktheit}



\section{Grundlagen}


\begin{df}[Kompakter topologischer Raum] \label{df:compact_space}
	Ein topologischer Raum $(X, \scr T)$ heißt \emph{kompakt}, wenn zu $X = \bigcup_{i\in I} U_i$ mit $U_i \in \scr T$ endlich viele Indizes $i_1, \dotsc, i_n \in I$ existieren mit $X = \bigcup_{k=1}^n U_{i_k}$.
\end{df}

\begin{ex}
	\begin{itemize}
		\item
			Jeder Raum mit indiskreter Topologie ist kompakt.
		\item
			$\R^n = \bigcup_{n\in\N} B(0,n)$ enthält keine endliche Teilüberdeckung, $\R^n$ ist also mit euklidischer Topologie nicht kompakt.
		\item
			Für einen diskreten Raum $(X, \scr T)$ gilt:
			\[
				(X, \scr T) \text{ kompakt }
				\iff
				\text{ $X$ endlich}.
			\]
	\end{itemize}
\end{ex}

\begin{df}[Kompakte Teilmenge/Kompakter Teilraum] \label{df:compact_subspace}
	\begin{enumerate}[(1)]
		\item
			$A \subset X$ heißt \emph{kompakt in $(X, \scr T_X)$}, wenn jede offene Überdeckung von $A$, $\bigcup_{i\in I} U_i \supset A$, mit $U_i \in \scr T_X$ eine endliche Teilüberdeckung enthält.
		\item
			$A \subset X$ heißt \emph{relativ kompakt in $(X, \scr T_X)$}, wenn $\_A$ in $(X, \scr T_X)$ kompakt ist.
	\end{enumerate}
\end{df}

\begin{nt}
	Sei $(A, \scr T_A)$ ein Teilraum von $(X, \scr T)$.
	$K \subset A$ ist kompakt in $(X, \scr T)$ genau dann, wenn $K$ kompakt in $(A, \scr T_A)$ ist.

	Insbesondere ist der Teilraum $(A, \scr T_A)$ genau dann kompakt (gemäß \ref{df:compact_space}), wenn $A \subset X$ in $(X, \scr T_X)$ kompakt ist (gemäß \ref{df:compact_subspace}).
	\begin{proof}
		\begin{segnb}[„$\implies$“]
			Sei $\bigcup_{i\in I} U_i \supset K, U_i \in \scr T_A$ eine offene Überdeckung von $K$ in $(A, \scr T_A)$.
			Nach Definition des Teilraums existieren $V_i \in \scr T : U_i = V_i \cap A$ für alle $i \in I$.
			\[
				\bigcup_{i\in I} V_i
				\supset A \cap \bigcup_{i\in I} V_i
				= \bigcup_{i\in I} \underbrace{(V_i \cap A)}_{= U_i}
				\supset K
			\]
			ist eine offene Überdeckung von $K$ in $(X, \scr T_X)$ mit endlicher Teilüberdeckung $\bigcup_{k=1}^n V_{i_k} \supset K$.
			Also ist
			\[
				\bigcup_{k=1}^n \underbrace{U_{i_k}}_{= V_{i_k} \cap A}
				= A \cap \bigcup_{k=1}^n V_{i_k}
				\supset A \cap K
				= K
			\]
			eine endliche Teilüberdeckung von $\bigcup_{i\in I} U_i \supset K$.
		\end{segnb}
		\begin{segnb}[„$\impliedby$“]
			Sei $\bigcup_{i\in I} U_i \supset K, U_i \in \scr T$ eine offene Überdeckung von $K$ in $(X, \scr T)$ und $V_i := A \cap U_i \in \scr T_A$ für $i \in I$.
			\[
				\bigcup_{i\in I} \underbrace{V_i}_{=A \cap U_i}
				= A \cap \bigcup_{i\in I} U_i
				\supset A \cap K
				= K
			\]
			ist eine offene Überdeckung von $K$ in $(A, \scr T_A)$ mit endlicher Teilüberdeckung $\bigcup_{k=1}^n V_{i_k} \supset K$.
			Also ist
			\[
				\bigcup_{k=1}^n U_{i_k}
				\supset A \cap \bigcup_{k=1}^n U_{i_k}
				= \bigcup_{k=1}^n \overbrace{V_{i_k}}^{=A \cap U_{i_k}}
				\supset K
			\]
			eine endliche Teilüberdeckung von $\bigcup_{i\in I} U_i \supset K$.
		\end{segnb}
	\end{proof}
\end{nt}

\begin{ex}
	\begin{itemize}
		\item
			Jede endliche Menge $A \subset X$ ist kompakt.
		\item
			$\Z \subset \R$ ist nicht kompakt, ebensowenig $\Q \subset \R$.
		\item
			$]0,1] \subset \R$ ist nicht kompakt, aber relativ kompakt in $\R$, denn $\_{]0,1]} = [0,1]$ (Abschluss in $\R$).
	\end{itemize}
\end{ex}

\begin{st}[Kompakte Ordnungstopologie, Charakterisierung]
	Sei $(X, <)$ ein geordnete Menge.
	Genau dann ist jedes Intervall $[a,b]$ in $X$ kompakt bezüglich der Ordnungstopologie, wenn $(X, <)$ vollständig geordnet ist (d.h. $(X,<)$ erfüllt das Supremums-Axiom).
\end{st}

\begin{ex}
	\begin{itemize}
		\item
			$(\R, <)$ ist vollständig geordnet und jedes Intervall $[a,b] \subset \R$ ist kompakt.
		\item
			$(\Q, <)$ ist nicht vollständig geordnet und jedes Intervall $[a,b]_\Q$ mit $a<b$ ist nicht kompakt.

			Sei $\xi \in [0,1] \setminus \Q$, z.B. $\xi = \f{\sqrt 2}2$.
			Dann bilden
			\begin{align*}
				S_n &= [0, \xi - \f 1n[_{\Q},&
				T_n &= ]\xi + \f 1n, 1]_{\Q}
			\end{align*}
			eine offene Überdeckung $[0,1]_{\Q} = \bigcup_{n\in\N} S_n \cup \bigcup_{n\in\N} T_n$ ohne endliche Teilüberdeckung.
	\end{itemize}
\end{ex}

\begin{lem} \label{lem:hausdorff_compact_subspace_neighbourhood}
	Sei $(X, \scr T)$ hausdorffsch, $A \subset X$ kompakt und $b \in X \setminus A$.
	Dann existieren offene Umgebungen $U$ von $A$ in $X$ und $V$ von $b$ in $X$ mit $U\cap V = \emptyset$.
	\begin{proof}
		Zu $a \in A$ existieren $a \in U_a \in \scr T$ und $b \in V_a \in \scr T$ mit $U_a \cap V_a = \emptyset$.
		Da $A$ kompakt, gilt $A \subset U_{a_1} \cup \dotsb \cup U_{a_n} =: U$ und $V := V_{a_1} \cap \dotsb \cap V_{a_1} \ni b$.
	\end{proof}
\end{lem}

\begin{st}
	Sind $A, B$ kompakt mit $A \cap B = \emptyset$ in einem Hausdorff-Raum $(X, \scr T)$, dann existieren $A \subset U \in \scr T$ und $B \subset V \in \scr T$ mit $U \cap V = \emptyset$.
	\begin{proof}
		Analog zu oben.
	\end{proof}
\end{st}

\coursetimestamp{18}{11}{2013}

\begin{st}
	In jedem kompakten Raum $(X, \scr T)$ ist jede abgeschlossene Teilmenge $A \subset X$ kompakt.
	\begin{proof}
		Sei $A \subset \bigcup_{i\in I} U_i$ mit $U_i \in \scr T$, dann ist
		\[
			X
			= (X \setminus A) \cup \bigcup_{i\in I} U_i
			= (X \setminus A) \cup U_{i_1} \cup \dotsb \cup U_{i_n},
		\]
		also $A \subset U_{i_1} \cup \dotsb \cup U_{i_n}$.
	\end{proof}
\end{st}

\begin{st}
	In einem Hausdorff-Raum $(X, \scr T)$ ist jede kompakte Teilmenge $A \subset X$ abgeschlossen.
	\begin{proof}
		Nach \ref{lem:hausdorff_compact_subspace_neighbourhood} findet sich zu jedem $b \in X \setminus A$ eine offene Umgebung in $X \setminus A$.
	\end{proof}
\end{st}

\subsection{Kompaktheit unter Abbildungen}

\begin{st}
	Ist $X$ kompakt, $f: X \to Y$ stetig, dann ist $f(X) \subset Y$ kompakt.
	\begin{proof}
		Sei $f(X) \subset \bigcup_{i\in I} V_i$ mit $V_i \in \scr T_y$.
		Da $f$ stetig ist
		\[
			U_i = f^{-1}(V_i) \in \scr T_X
		\]
		und $X = \bigcup_{i\in I} U_i = \bigcup_{k=1}^n U_{i_k}$
		Aus $f(U_i) \subset V_i$ folgt
		\[
			f(X)
			\subset f(U_{i_1}) \cup \dotsb \cup f(U_{i_n})
			\subset V_{i_1} \cup \dotsb \cup V_{i_n}.
		\]
	\end{proof}
\end{st}

\begin{st}
	Ist $X$ kompakt, $f: X \to \R$ stetig, so existieren $a, b \in X$ mit
	\[
		f(a) \le f(x) \le f(b)
	\]
	für alle $x \in X$.
	\begin{proof}
		Leicht nachvollziehbar.
	\end{proof}
\end{st}

\begin{st}
	Sei $X$ kompakt, $Y$ hausdorffsch.
	Dann ist jede stetige Abbildung $f: X \to Y$ abgeschlossen und induziert in der kanonischen Faktorisierung ein Homöomorphismus $\_f$.
	\[
		\begin{tikzcd}
			X \arrow{r}{f} \arrow[sur]{d}{q} & Y \\
			X / R_f \arrow{r}{\_f} & f(X) \arrow[inj]{u}{\jota}
		\end{tikzcd}
	\]
	\begin{proof}
		Sei $A \subset X$ abgeschlossen.
		Da $X$ kompakt, ist auch $A$ kompakt und damit wegen Stetigkeit von $f$ auch $f(A)$.
		Da $Y$ hausdorffsch, ist also auch $f(A)$ abgeschlossen.
	\end{proof}
\end{st}

\begin{ex}
	Wir wissen $\R / \Z \homto \scr S^1$.
	Die Abbildung $p: \R \to \S^1$ mit $p(t) = e^{2\pi i t}$ ist stetig, surjektiv und $p(x) = p(x') \iff x-x' \in \Z$.
	In der kanonischen Faktorisierung erhalten wir
	\[
		\begin{tikzcd}
			\R \arrow{d}{q} \arrow{r}{p} & \S^1 \\
			\R / \Z \arrow{r}{\_p} & \S^1 \arrow{u}
		\end{tikzcd}.
	\]
	Da $\R / \Z = q([0,1])$ kompakt ist und $\S^1 \subset \R^2$ hausdorffsch, ist $\_p$ ein Homöomorphismus.
\end{ex}

\begin{ex}
	$\D^n / \S^{n-1} \homeomorphic \S^n$
\end{ex}

\subsection{Summen}

\begin{st}
	Sei $(X_\lambda)_{\lambda\in\Lambda}$ eine Familie topologischer Räume mit $X_\lambda \neq \emptyset$.
	Die Summe $X = \bigdunion_{\lambda \in \Lambda} X_\lambda$ ist genau dann kompakt, wenn alle $X_\lambda$ kompakt sind und $\Lambda$ endlich ist.
	\begin{proof}
		\begin{segnb}[„$\impliedby$“]
			klar
		\end{segnb}
		\begin{segnb}[„$\implies$“]
			Angenommen $X_\lambda$ offen, dann ist $X = \bigcup_{\lambda\in\Lambda} X_\lambda$ offene Überdeckung ohne echte Teilüberdeckung.
			Also ist $X_\lambda$ abgeschlossen in $X$, also kompakt.
			% fixme: prüfen
		\end{segnb}
	\end{proof}
\end{st}

\subsection{Charakterisierung durch Filter}

\begin{st}
	Ein topologischer Raum $(X, \scr T)$ ist genau dann kompakt, wenn jeder Ultrafilter $\scr F$ auf $X$ konvergiert, also ein $x \in X$ existiert mit $\scr F \supset \scr U_x$.
	\begin{proof}
		\begin{segnb}[„$\implies$“]
			Sei $X$ kompakt und $\scr F$ ein Ultrafilter.
			Angenommen $\scr F$ konvergiert nicht, d.h. jeder Punkt $x \in X$ hat eine offene Umgebung $x \in U_x \in \scr T$ mit $U_x \not\in \scr F$.
			Da $X$ kompakt, existiert eine Überdeckung $X = U_{x_1} \cup \dotsb \cup U_{x_n}$ mit $U_{x_1}, \dotsc, U_{x_n} \not\in \scr F$.
			Da $\scr F$ ein Ultrafilter ist, gilt $X \setminus U_{x_1}, \dotsc, X \setminus U_{x_n} \in \scr F$.
			Nach (F2) ist
			\[
				(X \setminus U_{x_1}) \cap \dotsc \cap (X \setminus U_{x_n})
				= X \setminus (U_{x_1} \cup \dotsc \cup U_{x_n})
				= \emptyset \in \scr F,
			\]
			ein Widerspruch zu (F1).
		\end{segnb}
		\begin{segnb}[„$\impliedby$“]
			Angenommen $X$ ist nicht kompakt, d.h. es existiert $X = \bigcup_{i\in I} U_i$, $U_i \in \scr T$ ohne endliche Teilüberdeckung.
			Dann ist
			\[
				\scr E
				= \Set{ X \setminus \bigcup_{i\in J} U_i | J \subset I \text{ endlich} }
			\]
			eine Filterbasis.
			Diese erzeugt einen Filter und dieser liegt in einem Ultrafilter $\scr F$ auf $X$.
			Nach Voraussetzung konvergiert $\scr F$ gegen ein $x \in X$, d.h. $\scr F \supset \scr U_x$.
			Wegen $X = \bigcup_{i\in I} U_i$ existiert $j \in I$ mit $x \in U_j \in \scr T$, also $U_j \in \scr U_x \subset \scr F$.
			Andererseits gilt $X \setminus U_j \in \scr E \subset \scr F$.
			Nach (F2) ist also $U_j \cap (U \setminus U_j) = \emptyset \in \scr F$, ein Widerspruch zu (F1).
		\end{segnb}
	\end{proof}
\end{st}

\subsection{Produkte}

\begin{st}[Tychonoff, 1930]
	Seien $(X_i, \scr T_i)_{i\in I}$ nicht-leere topologische Räume.
	Der Produktraum $(X, \scr T) = \prod_{i\in I} (X_i, \scr T_i)$ ist kompakt genau dann, wenn jeder Raum $(X_i, \scr T_i)$ kompakt ist.
	\begin{proof}
		\begin{segnb}[„$\implies$“]
			$p_i: X \to X_i$ ist stetig und surjektiv, also ist $X_i$ kompakt.
		\end{segnb}
		\begin{segnb}[„$\impliedby$“]
			Sei $\scr F$ ein Ultrafilter auf $X$.
			Dann ist $p_i(\scr F)$ ein Ultrafilter auf $X_i$ (Übungsaufgabe). % fixme: ref
			Da $X_i$ kompakt, existiert $x_i \in X_i$ mit $p_i(\scr F) \to x_i$.
			Dann konvergiert $\scr F$ auf $X$ gegen $x = (x_i)_{i\in I}$.
		\end{segnb}
	\end{proof}
\end{st}

\begin{ex}
	\begin{itemize}
		\item
			$\{0,1\}^\N$ ist kompakt.
		\item
			$[a_1,b_1] \times \dotsb \times [a_n,b_n] \subset \R^n$ ist kompakt.
		\item
			$\prod_{k\in\N} [a_k,b_k] \subset \R^\N$ ist kompakt.
		\item
			$\prod_{x\in\R} [a_x,b_x] \subset \R^\R$ ist kompakt.
	\end{itemize}
\end{ex}

\begin{ex}[Hilbert-Würfel]
	Der Hilbert-Würfel $[0,1]^\N$ ist kompakt und metrisierbar.
	Ebenso in $\ell^2(\N)$ ist der Hilbert-Quader
	\[
		H = \Set{ f: \N \to \R | \le f(k) \le \f 1{k+1} }
		= [0,1] \times [0,\f 12] \times [0, \f 13] \times \dotsb
	\]
	Es gilt $H \homeomorphic [0,1]^\N$.
\end{ex}

\begin{st}[verwandte Kompaktheitsbegriffe]
	Für topologische Räume gilt
	\begin{itemize}
		\item
			Kompaktheit impliziert abzählbare Kompaktheit
		\item
			abzählbare Kompaktheit impliziert Kompaktheit unter Voraussetzung des zweiten Abzählbarkeitsaxioms
		\item
			Folgenkompaktheit impliziert abzählbare Kompaktheit
		\item
			abzählbare Kompaktheit impliziert Folgenkompaktheit unter Voraussetzung des ersten Abzählbarkeitsaxioms.
		\item
			Abzählbare Kompaktheit impliziert Pseudokompaktheit
	\end{itemize}
	\begin{proof}
		Wie im metrischen Fall.
		% fixme: ref
	\end{proof}
\end{st}


\section{Lokale Kompaktheit}


\begin{df}
	Ein Raum $(X, \scr T)$ heißt \emph{lokal-kompakt}, wenn jede Umgebung eines Punktes eine kompakte Umgebung enthält, d.h.
	\[
		\forall x \in X \forall U \in \scr U_x \exists K \in \scr U_x : \text{$K \subset U$ und $K$ kompakt}.
	\]
\end{df}

\begin{ex}
	\begin{itemize}
		\item
			$\R^n$ ist lokal-kompakt.
		\item
			Kein Punkt $x\in \Q$ besitzt eine kompakte Umgebung, insbesondere ist also $\Q$ nicht lokal-kompakt.
	\end{itemize}
\end{ex}

\begin{st} \label{st:hausdorff_compact_neighbourhoods_locally_compact_space}
	Sei $(X, \scr T)$ ein Hausdorff-Raum und zu jedem Punkt $x \in X$ existiere eine kompakte Umgebung.
	Dann ist $(X, \scr T)$ lokal-kompakt.
	\begin{proof}
		Sei $x \in X$ und $K \in \scr U_x$ kompakt.
		Zu jeder offenen Umgebung $U \in \scr U_x$ müssen wir eine kompakte Umgebung $K_U \in \scr U_x$ mit $K_U \subset U$ finden.

		Setze
		\[
			A := K \setminus U
		\]
		($A$ kompakt) und seien $V, W$ offene Umgebungen von $x$, bzw. $A$ mit $V \cap W = \emptyset$ (Hausdorff-Eigenschaft).
		Setze
		\[
			K_U := K \setminus W
		\]
		$K_U$ ist abgeschlossen und kompakt und zudem eine Umgebung von $x$, denn $x \in K \cap V \subset K \setminus W = K_U$.
		Weiterhin gilt $K_U \subset U$.
	\end{proof}
\end{st}

\begin{kor}
	Jeder kompakte Hausdorff-Raum ist lokal-kompakt.
	\begin{proof}
		Der Raum ist kompakte Umgebung eines jeden seiner Punkte, also folgt die Aussage direkt mit \ref{st:hausdorff_compact_neighbourhoods_locally_compact_space}.
	\end{proof}
\end{kor}

\begin{lem}
	Ist $(X, \scr T)$ lokal-kompakt, so auch jeder offene und jeder abgeschlossene Teilraum.
	\begin{proof}
		Für einen Teilraum $(O, \scr T_O)$ mit beliebigem $x \in O \in \scr T$ und Umgebung $x \in U \in \scr T_O$ existiert $V \in \scr T$ mit $U = V \cap O \in \scr T$.
		Also existiert eine kompakte Teilmenge $K \subset U$, welche auch kompakt in $(O, \scr T_O)$ ist.
	\end{proof}
\end{lem}

\begin{st}
	Sei $(X, \scr T)$ ein topologischer Raum, $U \subset X$ offen und $A \subset X$ abgeschlossen.
	\begin{itemize}
		\item
			Ist $(X, \scr T)$ lokal-kompakt und $Y = U \cap A$ mit Teilraumtopologie $\scr T_Y$, dann ist $Y$ lokal-kompakt.
		\item
			Ist $(X, \scr T)$ hausdorffsch, so gilt auch die Umkehrung.
		\item
			Ist $Y \subset X$ lokal-kompakter Teilraum, so existiert eine Zerlegung $Y = U \cap A$ mit $U, (X \setminus A) \in \scr T$.
	\end{itemize}
\end{st}

\begin{df}
	Sei $(X, \scr T)$ ein topologischer Raum.
	Eine Menge $U \subset X$ heißt \emph{$K$-offen}, wenn für jedes Kompaktum $K \subset X$ die Schnittmenge $U \cap K$ offen in $(K, \scr T_K)$ ist.

	Dies definiert die \emph{kompakt erzeugte Topologie}
	\[
		K \scr T :=
		\Set{ U \subset X | U \text{ ist $K$-offen} }.
	\]
\end{df}

\begin{lem}
	Es gilt
	\[
		\scr T \subset K \scr T = K(K \scr T)
	\]
	und die kompakten Teilräume sind dieselben.
\end{lem}

\begin{st}
	Jeder lokal-kompakte Raum $(X, \scr T)$ ist kompakt erzeugt, d.h. $K \scr T = \scr T$.
	\begin{proof}
		Wir zeigen $K \scr T \subset \scr T$.
		Sei $U \in K \scr T$, d.h. $U \subset X$ mit $U \cap K$ offen in jedem Kompaktum $K \subset X$.
		Wir zeigen $U \in \scr T$.
		Zu $x \in U$ existiert eine kompakte Umgebung $K$ in $X$, also $U \cap K$ offen in $K$, d.h. es existiert $V \in \scr T$ mit $U \cap K = V \cap K$.
		Dies ist eine Umgebung von $x$ in $X$, da $V$ und $K$ solche sind.
		Also ist auch $U \supset V \cap K$ ein Umgebung von $x \in X$.
		Damit ist $U$ offen in $X$, also $U \in \scr T$.
	\end{proof}
\end{st}

\coursetimestamp{19}{11}{2013}

\begin{st}
	Jeder metrische Raum ist kompakt erzeugt.

	Sogar jeder topologischer Raum mit erstem Abzählbarkeitsaxiom ist kompakt erzeugt.
\end{st}

\begin{st}
	Ein Hausdorff-Raum ist genau dann kompakt erzeugt, wenn er Quotient eines lokal-kompakten Raumes ist.
\end{st}

\subsection{Topologische Vektorräume}

\begin{df}
	Ein topologischer $\R$-Vektorraum $(X, \scr T, +, \cdot)$ ist ein $\R$-Vektorraum mit einer Hausdorff-Topologie $\scr T$ auf $X$, sodass Addition $+: X\times X \to X$ und Skalarmultiplikation $\cdot: \R \times X \to X$ stetig sind.
\end{df}

\begin{ex}
	\begin{itemize}
		\item
			$(\R^n, \scr T_{\R^n}, +, \cdot)$,
		\item
			Jeder normierte $\R$-Vektorraum $(X,|\argdot|)$,
		\item
			Auf $\R^{(\N)}, \scr C_c(\R,\R), \scr C([0,1],\R)$ definieren die $p$-Normen mit $1 \le p \le \infty$ überabzählbar viele Vektorraumtopologien.
		\item
			$\R^\R$ mit der Topologie der punktweisen, gleichmäßigen oder kompakten Konvergenz
	\end{itemize}
\end{ex}

\begin{lem}
	\begin{enumerate}[(1)]
		\item
			Jede Translation $\tau_v: X \to X: x \mapsto x + v$ ist ein Homöomorphismus.
		\item
			Jede Streckung $\my_a: X \to X: x \mapsto a x$ mit $a \in \R \setminus \{0\}$ ist ein Homöomorphismus.
		\item
			Jede Umgebung $U$ von $0$ ist absorbierend, d.h. $X = \bigcup_{n\in\N} n U$.
		\item
			Jede Umgebung $U$ von $0$ enthält eine ausgeglichene, offene Umgebung $V \subset U$, d.h. $[-1,1] V = V$.
	\end{enumerate}
	\begin{proof}
		\begin{enumerate}[(1),start=3]
			\item
				Für $x \in X$ ist $f: \R \to X$ mit $f(a) = ax$ stetig als Komposition
				\[
					\begin{tikzcd}
						f: \R \arrow[hom]{r} &
						\R \times \{x\} \arrow[inj]{r} &
						\R \times X \arrow{r} &
						X
					\end{tikzcd}.
				\]
				Also ist $n^{-1}x \to 0$ für $n \to \infty$, d.h. es gibt $m \in \N$ sodass $n^{-1}x \in U$ für alle $n \ge m$.
				Somit $x \in nU$ für $n \ge m$.
			\item
				Die Skalarmultiplikation mit $a$, $m: \R \times X \to X, m(a,x) = ax$ ist stetig (Argument wie oben).
				Für $0 \in U \in \scr T$ ist $m^{-1}(U) \subset \R \times X$ offen.
				Wegen $m(0,0) = 0$ gilt $(0,0) \in m^{-1}(U)$.
				Es existiert $\eps \in \R_{>0}$ und $0 \in W \in \scr T$ sodass $]-\eps, \eps[ \times W \subset m^{-1}(U)$.
				Damit ist $V = \bigcup_{0 < |a| < \eps} a W \subset U$ offen und ausgeglichen.
		\end{enumerate}
	\end{proof}
\end{lem}

\begin{st}
	Sei $(\R^n, \scr T_{\R^n}, +, \cdot)$ der euklidische Vektorraum und $(X, \scr T, +, \cdot)$ ein beliebiger topologischer $\R$-Vektorraum.
	Dann gelten folgende Aussagen
	\begin{enumerate}[1)]
		\item
			Jede lineare Abbildung $f: \R^n \to X$ ist stetig.
		\item
			Ist $f$ injektiv, so ist $f$ auch abgeschlossen.
		\item
			Ist $f$ bijektiv, so ist sie ein Homöomorphismus.
	\end{enumerate}
	\begin{proof}
		\begin{enumerate}[(1)]
			\item
				Sei $e_1, \dotsc, e_n \in \R^n$ die Standardbasis und $v_i = f(e_i), i=1, \dotsc, n$.
				Dank $\R$-Linearität gilt
				\[
					f(a) = a_1 v_1 + \dotsc + a_n v_n.
				\]
				$f$ ist also stetig, denn
				\begin{align*}
					f:
					\R^n &\homto (\R \times \{v_1\}) \times \dotsb \times (\R \times \{v_n\}) \\
					&\quad\injto (\R \times X) \times \dotsb \times (\R \times X) \\
					&\quad\stack{\cdot^n}\to X \times \dotsb \times X
					\stack{+^n}\to X
				\end{align*}
				ist als Komposition stetig.
			\item
				In $\R^n$ sei $K_0 = \_B(0,1), K_r = \_B(0,2^r) \setminus B(0,2^{r-1})$.
				Die Familie $(K_r)_{r\in\N}$ von Kreisringen ist kompakt und abgeschlossen und lokal-endlich.
				Die Bilder $(f(K_r))_{r\in\N}$ sind dies ebenfalls:

				Da $f$ injektiv ist, folgt aus $0 \not\in K_1$ auch $0 = f(0) \not\in f(K_1)$, d.h. es existiert eine Umgebung $U$ von $0$ mit $U \cap K_1 = \emptyset$.
				$U$ enthält eine ausgeglichene Umgebung $V$ von $0$.
				Dann ist $2^{-v}V \subset V$ disjunkt von $K_1$, also ist $V \cap 2^r K_1 = \emptyset$, d.h. $V \cap K_r = \emptyset$ für $r \ge 1$.
				Jeder Punkt $x$ liegt in einer offenen Menge $2^k V$, diese schneidet nur endlich viele $K_r$.

				Für $A \subset \R^n$ abgeschlossen ist $A \cap K_r$ kompakt, also $f(A \cap K_r)$ kompakt, also abgeschlossen.
				Damit ist $f(A) = \bigcup_{r\in\N} f(A \cap K_r)$ abgeschlossen, da lokal-endliche Vereinigung.
			\item
				klar
		\end{enumerate}
	\end{proof}
\end{st}

\begin{kor}
	Die euklidische Topologie auf $\R^n$ ist die einzige Vektorraumtopologie auf $(\R^n, +, \cdot)$.

	Auf jeden endliche-dimensionalen $\R$-Vektorraum existiert genau eine Vektorraumtopologie.
	\begin{note}
		Auf $\Q^n$ stimmt dies nicht.
		So ist $\Q^2 \subset \R^2$ und $\Q[\sqrt 2] \subset \R$ $\Q$-linear isomorph mit
		\[
			f: \Q^2 \to \Q[\sqrt 2], f(a,b) = a + b \sqrt{2},
		\]
		aber die Teilraumtopologien sind verschieden.
		$f$ ist stetig, aber kein Homöomorphismus:
		Zu $n \in \N$ existieren $a_n, b_n \in \Z$ mit $(a_n, b_n) \neq (0,0)$ und $|a_n + b_n \sqrt{2}| < \f 1n$.
		Damit ist $a_n + b_n \sqrt{2} \to 0$, aber $(a_n, b_n) \not\to (0,0)$.
	\end{note}
\end{kor}

\begin{lem}
	Sei $\scr B$ eine Umgebungsbasis von $O$ in $X$.
	Für $A \subset X$ gilt $\_A = \cap_{U \in \scr B} A + U$
	\begin{proof}
		Für $x \in X$ ist $\scr B_x = \{ x - U : U \in \scr B\}$ eine Umgebungsbasis.
		Genau dann gilt $x \in \_A$, wenn $(x-U) \cap A \neq \emptyset$ für alle $U \in \scr B$.
		Das bedeutet, für jedes $U \in \scr B$ existiert $a \in (x-U) \cap A$, also $x \in a + U$.
	\end{proof}
\end{lem}

\begin{st}
	Ein topologischer $\R$-Vektorraum $(X, \scr T, +, \cdot)$ ist genau dann lokal-kompakt, wenn $\dim_{\R} X < \infty$.
	\begin{proof}
		\begin{segnb}[„$\impliedby$“]
			Klar mit vorigem Satz
		\end{segnb}
		\begin{segnb}[„$\implies$“]
			Es existiert eine offene Umgebung $0 \in V \in \scr T$ um $0$ mit $\_V$ kompakt.
			Für $U \in \scr U_0$ gilt $X = \bigcup_{n\in\N} 2^n U$ (absorbierend), also
			\[
				V \subset \_V \subset \bigcup_{k\in\N}2^k U
			\]
			und damit $V \subset \_V \subset 2^n U$ für $n$ groß genug, d.h. $2^{-n} V \subset U$, also ist $\{2^{-n} V : n\in \N\}$ eine Umgebungsbasis von $O$.
			Mit $V$ ist auch $x + \f 12 V$ offen für jedes $x \in X$, also $\_V \subset X = \bigcup_{x\in X} x + \f 12 V$.
			Da $\_V$ kompakt, ist damit
			\[
				\_V = (x_1 + \f 12 V) + \dotsb + (x_m + \f 12 V).
			\]
			Sei $Y = \R x_1 + \dotsb + \R x_m$.
			Aus $V \subset Y + \f 12 V$ und $\f 12 Y = Y$ folgt $\f 12 V \subset Y + \f 12 V$, also $V \subset Y + \f 12 V \subset Y + Y + \f 14 V \subset Y + \f 12 V$.
			Per Induktion gilt $V \subset Y + 2^{-n}V$, somit auch
			\[
				V
				\subset \bigcap_{n\in\N} Y + 2^{-n} V
				= \_Y
				= Y.
			\]
			Es folgt $X = \bigcup_{n\in\N} nV \subset Y$, also ist
			\[
				\dim_{\R} X = \dim_{\R} Y \le m < \infty.
			\]
		\end{segnb}
	\end{proof}
\end{st}


\section{Alexandroff-Kompaktifizierung}


\begin{df}
	Eine \emph{Kompaktifizierung} eines topologischen Raumes $(X, \scr T)$ ist ein Paar $(Y, \kappa)$ bestehend aus einem kompakten Hausdorff-Raum $Y$ und einer Einbettung $\kappa: X \injto Y$ mit $\_{\kappa(X)} = Y$.
\end{df}

\begin{ex}
	\begin{itemize}
		\item
			$]0,1] \injto [0,1]$,
		\item
			$]-1,1[ \injto [-1,1]$,
		\item
			$\R \injto \_\R = \R \cup \{\pm\infty\}$,
		\item
			$\B^n \injto \D^n$,
		\item
			die stereographische Projektion $\kappa: \R^n \homto \S^n \setminus \{p\} \injto \S^n$.
	\end{itemize}
\end{ex}

\begin{st}[Alexandroff]
	Sei $(X, \scr T)$ ein topologischer Raum und $\infty \not\in X$.
	Auf $\hat X := X \cup \{\infty\}$ definieren wir die \emph{Alexandroff-Topologie}
	\[
		\hat{\scr T} :=
		\scr T \cup \Set{ \hat X \setminus K | K \subset X \text{ kompakt und abgeschlossen} }.
	\]
	Dies ist eine Topologie auf $\hat X$.
	Der Raum $(\hat X, \hat{\scr T})$ ist kompakt und hierin ist $(X, \scr T)$ ein offener Teilraum.
	Die Topologie $\hat{\scr T}$ ist die feinste mit dieser Eigenschaft.

	Genau dann ist $(\hat X, \hat{\scr T})$ hausdorffsch, wenn $(X, \scr T)$ hausdorffsch und lokal-kompakt ist.
\end{st}


\section{Die Kompakt-Offen-Topologie}


\coursetimestamp{25}{11}{2013}
% D5 ( D4: kompakt-offen-Topologie )
\section{Trennungsaxiome und Metrisierbarkeit}


Mit „Trennen zweier Mengen $A, B$“ meinen wir in topologischen Räumen das Finden von zwei disjunkten (offenen) Umgebungen um $A$, bzw. $B$.

Je nach Topologie ist das Trennen unterschiedlicher Mengen unter verschieden starken Voraussetzungen möglich, oder auch nicht.

\begin{df}[Trennungsaxiome] \label{df:separation_axioms}
	Für einen topologischen Raum $(X, \scr T)$ definieren wir folgende \emph{Trennungseigenschaften}:
	\begin{enumerate}[label=(\SepAxiom{\arabic*}),start=0,leftmargin=3.5em,series=sepaxioms]
		\item
			Zu $a, b \in X, a \neq b$ hat einer eine Umgebung, die den anderen nicht enthält.
		\item
			Zu $a, b \in X, a \neq b$ haben beide jeweils eine Umgebung, die den anderen nicht enthält.
		\item
			Zu $a, b \in X, a \neq b$ existieren disjunkte Umgebungen (Hausdorff"=Eigenschaft).
		\item
			Zu $A \subset X$ abgeschlossen und $b \in X \setminus A$ existieren disjunkte Umgebungen.
		\item
			Zu $A, B \subset X$ abgeschlossen und disjunkt existieren disjunkte Umgebungen.
		\item
			Zu $A, B \subset X$ mit $\_A \cap B = A \cap \_B = \emptyset$ existieren disjunkte Umgebungen.
	\end{enumerate}
	\begin{enumerate}[label=(\SepAxiom{\arabic*\sfrac12}),resume*=sepaxioms,start=2]
		\item
			Zu $a, b \in X, a \neq b$ existiert $f: X \to [0,1]$ stetig mit $f(a) = 0, f(b) = 1$.
		\item
			Zu $A \subset X$ abgeschlossen und $b \in X \setminus A$ existiert $f: X \to [0,1]$ stetig mit $f|_A = 0, f(b) = 1$.
		\item
			Zu $A, B \subset X$ abgeschlossen und disjunkt existiert $f: X \to [0,1]$ stetig mit $f|_A = 0, f|_B = 1$.
		\item
			Zu $A, B \subset X$ abgeschlossen und disjunkt existiert $f: X \to [0,1]$ stetig mit $f^{-1}(0) = A, f^{-1}(1) = B$.
	\end{enumerate}
\end{df}

\begin{nt}
	Jeder metrische Raum $(X, \scr T)$ erfüllt alle Trennungseigenschaften aus \ref{df:separation_axioms}.
	Beliebige disjunkte Mengen lassen sich in $(X, \scr T)$ im Allgemeinen jedoch nicht trennen.
	\begin{proof}
		Wähle im Fall von \SepAxiom5 explizit als Umgebungen
		\begin{align*}
			U &:= \Set{ x \in X | d(x, A) < d(x, B) } \supset A \\
			V &:= \Set{ x \in X | d(x, B) < d(x, A) } \supset B.
		\end{align*}
		\SepAxiom1, \SepAxiom2, \SepAxiom3 und \SepAxiom4 folgen auf ähnliche Weise.

		Im Fall von \SepAxiom{5\sfrac12} lässt sich explizit $f: X \to [0,1]$ wählen als
		\[
			f(x)
			:= \f {d(x,A)}{d(x,A) + d(x,B)}.
		\]
		\SepAxiom{2\sfrac12}, \SepAxiom{3\sfrac12} und \SepAxiom{4\sfrac12} folgen wieder auf ähnliche Weise.

		Beliebige disjunkte Mengen lassen sich nicht trennen:
		in $\R$ lassen sich $\Q$ und $\R \setminus \Q$ beispielsweise nicht trennen.
	\end{proof}
\end{nt}

Für die Trennungseigenschaften in topologischen Räumen gelten die Zusammenhänge wie in \ref{fig:separation_axioms} dargestellt.

\begin{figure}[h]
	\centering
	\begin{tikzpicture} [
			axiom/.style={rectangle,draw=black!50},
			node distance=14mm,
			minimum size=7mm,
			inner sep=1mm,
			implies/.style={double,-implies,double equal sign distance, shorten >=2mm, shorten <=2mm},
			label/.style={font=\scriptsize,auto,swap},
			bend angle=15,
		]
		\node[axiom] (T0) {\SepAxiom0};
		\node[axiom] (T1) [right=of T0] {\SepAxiom1};
		\node[axiom] (T2) [right=of T1] {\SepAxiom2};
		\node[axiom] (T3) [right=of T2] {\SepAxiom3};
		\node[axiom] (T4) [right=of T3] {\SepAxiom4};
		\node[axiom] (T5) [right=of T4] {\SepAxiom5};
		\node[axiom] (T2h) [below=of T2] {\SepAxiom{2\sfrac12}};
		\node[axiom] (T3h) [below=of T3] {\SepAxiom{3\sfrac12}};
		\node[axiom] (T4h) [below=of T4] {\SepAxiom{4\sfrac12}};
		\node[axiom] (T5h) [below=of T5] {\SepAxiom{5\sfrac12}};
		\draw[implies] (T1) to node[label] {} (T0);
		\draw[implies] (T2) to[bend right] node[label] {} (T1);
		\draw[implies] (T1) to[bend right] node[label] {$X$ endlich} (T2);
		\draw[implies] (T3) to[bend right] node[label] {\SepAxiom1} (T2);
		\draw[implies] (T2) to[bend right] node[label] {Kompaktheit} (T3);
		\draw[implies] (T4) to[bend right] node[label] {\SepAxiom1} (T3);
		\draw[implies] (T3) to[bend right] node[label,align=center] {Kompaktheit \\ $\lor$ 2AA \\ $\lor$ Lindelöf} (T4);
		\draw[implies] (T5) to node[label] {} (T4);
		\draw[implies] (T3h) to node[label] {\SepAxiom1} (T2h);
		\draw[implies] (T4h) to node[label] {\SepAxiom1} (T3h);
		\draw[implies] (T5h) to node[label] {} (T4h);
		\draw[implies] (T2h) to node[label] {} (T2);
		\draw[implies] (T3h) to node[label] {} (T3);
		\draw[implies] (T4h) to[bend left] node[label] {} (T4);
		\draw[implies] (T4) to[bend left] node[label] {} (T4h);
		\draw[implies] (T5h) to node[label] {} (T5);
	\end{tikzpicture}
	\caption{Zusammenhänge zwischen Trennungseigenschaften}
	\label{fig:separation_axioms}
\end{figure}

\begin{df}
	Ein topologischer Raum $(X, \scr T)$ heißt
	\begin{enumerate}[1)]
		\item
			\emph{hausdorffsch}, wenn $\SepAxiom2$ erfüllt ist,
		\item
			\emph{regulär}, wenn \SepAxiom1 und \SepAxiom3 erfüllt sind,
		\item
			\emph{vollständig regulär}, wenn \SepAxiom1 und \SepAxiom{3\sfrac 12} erfüllt sind,
		\item
			\emph{normal}, wenn \SepAxiom1 und \SepAxiom4 erfüllt sind,
		\item
			\emph{vollständig normal}, wenn \SepAxiom1 und \SepAxiom5 erfüllt sind,
		\item
			\emph{perfekt normal}, wenn \SepAxiom1 und \SepAxiom{5\sfrac 12} erfüllt sind.
	\end{enumerate}
\end{df}

\begin{lem}[Tychonoff]
	2AA und \SepAxiom3 impliziert \SepAxiom4.
\end{lem}

\begin{st}[Urysohn]
	Sei $(X, \scr T)$ ein \SepAxiom4-Raum.
	Dann gilt \SepAxiom{4\sfrac 12}.
	\begin{note}
		\SepAxiom4 besagt insbesondere:
		Zu $A \subset O$ mit $A$ abgeschlossen und $O$ offen existiert $U$ offen mit $A \subset U \subset  \_{U} \subset O$.
	\end{note}
	\begin{proof}
		Sei $A, B \subset X$ abgeschlossen, $A \cap B = \emptyset$.
		Setze $\_{A_0} = A_0 := B$, $A_1 := X \setminus A$.
		$A_1$ ist offene Umgebung von $\_{A_0}$.
		Sei $f_0: X \to [0,1]$ definiert durch $f_0|_{A_1} = 0, f_0|_{A} = 1$.
		Nach \SepAxiom4 existiert $A_{\f 12} \in \scr T$ mit
		\[
			B = \_{A_0} \subset A_{\f 12} \subset \_{A_{\f 12}} \subset A_1
		\]
		Sei $f: X \to [0,1], f|_{A_{\f 12}} = 0, f|_{A_1 \setminus A_{\f 12}} = \f 12, f|_A = 1$.

		Per Induktion erhalten wir $A_{\f {k}{2^n}} \in \scr T$ mit
		\[
			B = \_{A_{\f 0{2^n}}} \subset A_{\f 1{2^n}} \subset \_{A_{\f 1{2^n}}}
			\subset \dotsb \subset
			\_{A_{\f{2^n - 1}{2^n}}}
			\subset A_{\f {2^n}{2^n}}
			= X \setminus A.
		\]
		Wir definieren $f_n : X \to [0,1]$ durch $f|_{A_{\f 1{2^n}}} = 0$,
		\[
			f\Big|_{A_{\f {k+1}{2^n}} \setminus A_{\f {k}{2^n}}}
			= \f k{2^n}
		\]
		und $f|_A = 1$.
		Für jedes $x \in X$ ist $f_n(x) \in [0,1]$ monoton wachsend in $n$.
		Also $f_n(x) \to f(x) \in [0,1]$, $f_n$ konvergiert gleichmäßig gegen $f: X \to [0,1]$.

		Für $n \in \N$ wird $X$ überdeckt durch
		\begin{align*}
			U_n^0 &= A_{\f 1{2^n}} \\
			U_n^1 &= A_{\f 2{2^n}} \setminus A_{\f 0{2^n}} \\
			\vdots \quad &= \qquad \vdots \\
			U_n^{2^n-1} &= A_{\f {2^n}{2^n}} \setminus A_{\f {2^n-2}{2^n}} \\
			U_n^{2^n} &= X \setminus \_{A_{\f {2^n-1}{2^n}}}.
		\end{align*}
		Auf $U_n^k$ schwankt $f_n$ um höchstens $2^{-n}$ und $f$ um höchstens $2\cdot 2^{-n}$.
		Zu jedem $x \in X$ und $\eps > 0$ wählen wir $n \in \N$ so dass $2^{-n+1} < \eps$.
		Es gilt $x \in U_n^k$ für ein $k$.
		Dann ist
		\[
			f(U_n^k) \subset ]f(x) - \eps, f(x) + \eps[.
		\]
		Das bedeutet, $f$ ist stetig.
		% fixme: prüfen
	\end{proof}
\end{st}

\begin{lem}
	Sei $A \subset X$ abgeschlossen und $\phi: A \to [-s, s]$ stetig.
	Dann existiert $\Phi: X \to [-s, s]$ stetig mit
	\[
		|\phi(a) - \Phi(a)| \le \f 23 s
	\]
	für alle $a \in A$.
	\begin{proof}
		Setze $M := \phi^{-1}([-s, - \f s3]), N := \phi^{-1}([\f s3, s])$.
		$M, N$ sind ebgeschlossen.
		Nach Urysohn existiert $\Phi: X \to [-\f s3, \f s3]$ mit $\Phi|_M = - \f s3, \Phi|_N = \f s3$.
	\end{proof}
\end{lem}

\begin{st}[Tietze]
	Sei $(X, \scr T)$ ein \SepAxiom4-Raum.
	Ist $A \subset X$ abgeschlossen und $f: A \to [a,b]$ stetig.
	Dann existiert $F: X \to [a,b]$ stetig mit $F|_A = f$.
	\begin{proof}
		Sei \oBdA $[a,b] = [-1,1]$.
		Zu $f: A \to [-1,1]$ existiert $\Phi_0 : X \to [-\f 13, \f 13]$ wie im Lemma.
		Der Fehler auf $A$ ist $\phi_0 = f - \Phi_0|_A : A \to [-\f 23, \f 23]$.
		Zu $\phi_0$ existiert $\Phi_1: X \to [-\f 13 \cdot \f 23, \f 13 \cdot \f 23]$ wie im Lemma. % fixme: ref

		Per Induktion für $n \in \N$ ergibt sich der verbleibende Fehler als
		\[
			\phi_n = f - \Big( \Phi_0 + \Phi_1 + \dotsb + \Phi_n \Big)\Big|_A
			: A \to [-(\f 23)^{n+1}, (\f 23)^{n+1}.
		\]
		Dank Lemma existiert $\Phi_{n+1}: X \to [-\f 13(\f 23)^{n+1}, \f 13 (\f 23)^{n+1}]$.
		Damit konvergiert $F_n = \sum_{k=0}^n \Phi_n$ gleichmäßig auf $X$, denn
		\[
			\sum_{k=0}^n \|\Phi_k\|
			\le \sum_{k=0}^\infty \f 13 (\f 23)^k
			= 1.
		\]
		Die Grenzfunktion $F: X \to \R$ ist also stetig als glechimäßiger Grenzwert stetiger Funktionen und erfüllt
		\[
			\|F\| \le \sum_{k=0}^\infty \|\Phi_k\| = 1,
		\]
		also $F(X) \subset [-1, 1]$.
		Wegen $\phi_n \to 0$ gilt $f - F_n|_A \to 0$, also $F|_A = f$.
	\end{proof}
\end{st}

\begin{kor}
	Sei $(X, \scr T)$ ein \SepAxiom4-Raum.
	Ist $A \subset X$ abgeschlossen, $f: A \to \R^n$ stetig, so existiert $F: X \to \R^n$ stetig mit $F|_A = f$.
	\begin{note}
		Der Zielraum $\R^n$ ist wesentlich.
		Sei $Y = \R \setminus \{0\}, X = [-1,1], A = \{-1, 1\}, f: A \to Y, f(\pm 1) := \pm 1$.
		Hier existiert nach Zwischenwertsatz keine stetige Fortsetzung.
	\end{note}
\end{kor}

\begin{st}[Metrisierbarkeitssatz von Urysohn, 1924]
	Sei $(X, \scr T)$ ein topologischer Raum, der dem zweiten Abzählbarkeitsaxiom genügt.
	Dann sind äquivalent
	\begin{enumerate}[1)]
		\item
			$X$ ist metrisierbar,
		\item
			$X$ ist regulär, d.h. \SepAxiom1 und \SepAxiom3 ist erfüllt,
		\item
			$X$ ist normal, d.h. \SepAxiom1 und \SepAxiom4 ist erfüllt,
		\item
			$X$ ist homöomorph zu einem Teilraum von $[0,1]^\N$ (Hilbertwürfel).
	\end{enumerate}
	\begin{proof}
		\begin{segnb}[„(1)$\implies$(2)“]
			klar
		\end{segnb}
		\begin{segnb}[„(2)$\implies$(3)“]
			Folgt mit Lemma von Tychonoff. % fixme: ref
		\end{segnb}
		\begin{segnb}[„(3)$\implies$(4)“]
			Sei $\scr B$ eine abzählbare Basis.
			Sei $I$ die abzählbare Menge aller Paare $i = (U,V)$ mit $U,V \in \scr B$ und $\_U \subset V$.

			Dank Urysohn existiert $f_i: X \to [0,1]$ mit $f_i|_{\_U} = 0, f_i|_{X \setminus V} = 1$.
			Wir erhalten hieraus $h: X \to [0,1]^I, h(x) = (f_i(x))_{i \in I}$.
			Dies ist eine Einbettung.
		\end{segnb}
		\begin{segnb}[„(4)$\implies$(1)“]
			$[0,1]^\N$ ist metrisierbar, etwa durch
			\[
				d(x,y) = \sum_{k=0}^\infty 2^{-k-1} |x_i - y_i|,
			\]
			siehe Übungsaufgabe. % fixme: ref
		\end{segnb}
	\end{proof}
\end{st}
