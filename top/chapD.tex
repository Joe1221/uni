% D
\chapter{Kompaktheit}



\begin{df}
	Ein topologischer Raum $(X, \scr T)$ heißt \emph{kompakt}, wenn zu $X = \bigcup_{i\in I} U_i$ mit $U_i \in \scr T$ Indizes $i_1, \dotsc, i_n \in I$ existieren mit $X = \bigcup_{k=1}^n U_{i_k}$.
\end{df}

\begin{ex}
	\begin{itemize}
		\item
			$\R^n = \bigcup_{n\in\N} B(0,n)$ enthält keine endlichen Teilüberdeckung, $\R^n$ ist also nicht kompakt.
		\item
			Für einen diskreten Raum $(X, \scr T)$ gilt:
			\[
				(X, \scr T) \text{ kompakt }
				\iff
				\text{ $X$ endlich}.
			\]
	\end{itemize}
\end{ex}

\begin{df}
	\begin{enumerate}[(1)]
		\item
			$A \subset X$ heißt \emph{kompakt in $(X, \scr T_X)$}, wenn für jede offene Überdeckung $A \subset \bigcup_{i\in I} U_i$ mit $U_i \in \scr T_X$ eine endliche Teilüberdeckung enthält.
		\item
			$A \subset X$ heißt \emph{relativ kompakt in $(X, \scr T_X)$}, wenn $\_A$ in $(X,, \scr T_X)$ kompakt ist.
	\end{enumerate}
\end{df}

\begin{nt}
	Nach Definition der Teilraumtopologie % fixme: ref
	bedeutet $V \in \scr T_A$, dass $U \in \scr T_X$ existiert mit $V = U \cap A$.

	Daher ist die Definition für $A$ kompakt in $(X, \scr T_X)$ äquivalent dazu, dass der Teilraum $(A, \scr T_A)$ kompakt ist.
\end{nt}

\begin{ex}
	\begin{itemize}
		\item
			Jede endliche Menge $A \subset X$ ist kompakt.
		\item
			$\Z \subset \R$ ist nicht kompakt, ebensowenig $\Q \subset \R$.
		\item
			$]0,1] \subset \R$ ist nicht kompakt, aber relativ kompakt in $\R$, denn $\_{]0,1]} = [0,1]$ (Abschluss in $\R$).
	\end{itemize}
\end{ex}

\begin{st}
	Sei $(X, <)$ ein geordnete Menge.
	Genau dann ist jedes Intervall $[a,b]$ in $X$ kompakt bezüglich der Ordnungstopologie, wenn $(X, <)$ vollständig geordnet ist (d.h. erfüllt das Supremums-Axiom).
\end{st}

\begin{ex}
	\begin{itemize}
		\item
			$(\R, <)$ ist vollständig geordnet und jedes Intervall $[a,b] \subset \R$ ist kompakt.
		\item
			$(\Q, <)$ ist nicht vollständig geordnet und jedes Intervall $[a,b]_\Q$ mit $a<b$ ist nicht kompakt.

			Sei $\xi \in [0,1] \setminus \Q$, z.B. $\xi = \f{\sqrt 2}2$.
			Dann bilden
			\begin{align*}
				S_n &= [0, \xi - \f 1n[_{\Q},&
				T_n &= ]\xi + \f 1n, 1]_{\Q}
			\end{align*}
			eine offene Überdeckung $[0,1]_{\Q} = \bigcup_{n\in\N} S_n \cup \bigcup_{n\in\N} T_n$ ohne endliche Teilüberdeckung.
	\end{itemize}
\end{ex}

\begin{lem}
	Sei $(X, \scr T)$ hausdorffsch, $A \subset X$ kompakt und $b \in X \setminus A$.

	Dann existieren offene Umegebungen $U$ von $A$ in $X$ und $V$ von $b$ in $X$ mit $U\cap V = \emptyset$.
	\begin{proof}
		Zu $a \in A$ existieren $a \in U_a \in \scr T$ und $b \in V_a \in \scr T$ mit $U_a \cap V_a = \emptyset$.
		Da $A$ kompakt, gilt $A \subset U_{a_1} \cup \dotsb \cup U_{a_n} =: U$ und $V := V_{a_1} \cap \dotsb \cap V_{a_1} \ni b$.
	\end{proof}
\end{lem}

\begin{st}
	Sind $A, B$ kompakt mit $A \cap B = \emptyset$ in einem Hausdorff-Raum $(X, \scr T)$, dann existieren $A \subset U \in \scr T$ und $B \subset V \in \scr T$ mit $U \cap V = \emptyset$.
	\begin{proof}
		Analog zu oben.
	\end{proof}
\end{st}

\coursetimestamp{18}{11}{2013}

\begin{st}
	In jedem kompakten Raum $(X, \scr T)$ ist jede abgeschlossene Teilmenge $A \subset X$ kompakt.
	\begin{proof}
		Sei $A \subset \bigcup_{i\in I} U_i$ mit $U_i \in \scr T$, dann ist
		\[
			X
			= (X \setminus A) \cup \bigcup_{i\in I} U_i
			= (X \setminus A) \cup U_{i_1} \cup \dotsb \cup U_{i_n},
		\]
		also $A \subset U_{i_1} \cup \dotsb \cup U_{i_n}$.
	\end{proof}
\end{st}

\begin{st}
	In jedem Hausdorff-Raum $(X, \scr T)$ ist jede kompakte Teilmenge $A \subset X$ abgeschlossen.
	\begin{proof}
		Siehe Lemma
		% fixme: ref
	\end{proof}
\end{st}

\begin{st}
	Ist $X$ kompakt, $f: X \to Y$ stetig, dann ist $f(X) \subset Y$ kompakt.
	\begin{proof}
		Sei $f(X) \subset \bigcup_{i\in I} V_i$ mit $V_i \in \scr T_y$.
		Da $f$ stetig ist
		\[
			U_i = f^{-1}(V_i) \in \scr T_X
		\]
		und $X = \bigcup_{i\in I} U_i = \bigcup_{k=1}^n U_{i_k}$
		Aus $f(U_i) \subset V_i$ folgt
		\[
			f(X)
			\subset f(U_{i_1}) \cup \dotsb \cup f(U_{i_n})
			\subset V_{i_1} \cup \dotsb \cup f(V_{i_n}).
		\]
	\end{proof}
\end{st}

\begin{st}
	Ist $X$ kompakt, $f: X \to \R$ stetig, so existieren $a, b \in X$ mit
	\[
		f(a) \le f(x) \le f(b)
	\]
	für alle $x \in X$.
	\begin{proof}
		Leicht nachvollziehbar.
	\end{proof}
\end{st}

\begin{st}
	Sei $X$ kompakt, $Y$ hausdorffsch.
	Dann ist jede stetige Abbildung $f: X \to Y$ abgeschlossen und induziert in der kanonischen Faktorisierung ein Homöomorphismus $\_f$.
	\[
		\begin{tikzcd}
			X \arrow{r}{f} \arrow[sur]{d}{q} & Y \\
			X / R_f \arrow{r}{\_f} & f(X) \arrow[inj]{u}{\jota}
		\end{tikzcd}
	\]
	\begin{proof}
		Sei $A \subset X$ abgeschlossen und $X$ kompakt, ist $A$ kompakt und damit wegen Stetigkeit von $f$ auch $f(A)$.
		Da $Y$ hausdorffsch, ist also auch $f(A)$ abgeschlossen.
	\end{proof}
\end{st}

\begin{ex}
	Wir wissen $\R / \Z \homto \scr S^1$.
	Die Abbildung $p: \R \to \S^1$ mit $p(t) = e^{2\pi i t}$ ist stetig, surjektiv und $p(x) = p(x') \iff x-x' \in \Z$.
	In der kanonischen Faktorisierung erhalten wir
	\[
		\begin{tikzcd}
			\R \arrow{d}{q} \arrow{r}{p} & \S^1 \\
			\R / \Z \arrow{r}{\_p} & \S^1 \arrow{u}
		\end{tikzcd}.
	\]
	Da $\R / \Z = q([0,1])$ kompakt ist und $\S^1 \subset \R^2$ hausdorffsch, ist $\_p$ ein Homöomorphismus.
\end{ex}

\begin{ex}
	$\D^n // \S^{n-1} \homto \S^n$
\end{ex}

\begin{st}
	Sei $(X_\lambda)_{\lambda\in\Lambda}$ eine Familie topologischer Räume mit $X_\lambda \neq \emptyset$.
	Die Summe $X = \bigdunion_{\lambda \in \Lambda} X_\lambda$ ist genau dann kompakt, wenn alle $X_\lambda$ kompakt sind und $\Lambda$ endlich ist.
	\begin{proof}
		\begin{segnb}[„$\impliedby$“]
			klar
		\end{segnb}
		\begin{segnb}[„$\implies$“]
			Angenommen $X_\lambda$ offen, dann ist $X = \bigcup_{\lambda\in\Lambda} X_\lambda$ offene Überdeckung ohne echte Teilüberdeckung.
			Also ist $X_\lambda$ abgeschlossen in $X$, also kompakt.
			% fixme: prüfen
		\end{segnb}
	\end{proof}
\end{st}

\begin{st}
	Ein topologischer Raum $(X, \scr T)$ ist genau dann kompakt, wenn jeder Ultrafilter $\scr F$ auf $X$ konvergiert, also ein $x \in X$ existiert mit $\scr F \supset \scr U_x$.
	\begin{proof}
		\begin{segnb}[„$\implies$“]
			Sei $X$ kompakt und $\scr F$ ein Ultrafilter.
			Angenommen $\scr F$ konvergiert nicht, d.h. jeder Punkt $x \in X$ hat eine offene Umgebung $x \in U_x \in \scr T$ mit $U_x \not\in \scr F$.
			Da $X$ kompakt, existiert eine Überdeckung $X = U_{x_1} \cup \dotsb \cup U_{x_n}$ mit $U_{x_1}, \dotsc, U_{x_n} \not\in \scr F$.
			Da $\scr F$ ein Ultrafilter ist, gilt $X \setminus U_{x_1}, \dotsc, X \setminus U_{x_n} \in \scr F$.
			Nach (F2) ist
			\[
				(X \setminus U_{x_1}) \cap \dotsc \cap (X \setminus U_{x_n})
				= X \setminus (U_{x_1} \cup \dotsc \cup U_{x_n})
				= \emptyset \in \scr F,
			\]
			ein Widerspruch zu (F1).
		\end{segnb}
		\begin{segnb}[„$\impliedby$“]
			Angenommen $X$ ist nicht kompakt, d.h. es existiert $X = \bigcup_{i\in I} U_i$, $U_i \in \scr T$ ohne endliche Teilüberdeckung.
			Dann ist
			\[
				\scr E
				= \Set{ X \setminus \bigcup_{i\in J} U_i | J \subset I \text{ endlich} }
			\]
			eine Filterbasis.
			Dieser erzeugt einen Filter und dieser liegt in einem Ultrafilter $\scr F$ auf $X$.
			Nach Voraussetzung konvergiert $\scr F$ gegen ein $x \in X$, d.h. $\scr F \supset \scr U_x$.
			Wegen $X = \bigcup_{i\in I} U_i$ existiert $j \in I$ mit $x \in U_j \in \scr T$, also $U_j \in \scr U_x \subset \scr F$.
			Andererseits gilt $X \setminus U_j \in \scr E \subset \scr F$.
			Nach (F2) ist also $U_j \cap (U \setminus U_j) = \emptyset \in \scr F$, ein Widerspruch zu (F1).
		\end{segnb}
	\end{proof}
\end{st}

\begin{st}[Tychonoff, 1930]
	Seien $(X, \scr T)_{i\in I}$ nicht-leere topologische Räume.
	Der Produktraum $(X, \scr T) = \prod_{i\in I} (X_i, \scr T_i)$ ist kompakt genau dann, wenn jeder Raum $(X_i, \scr T_i)$ kompakt ist.
	\begin{proof}
		\begin{segnb}[„$\implies$“]
			$p_i: X \to X_i$ ist stetig und surjektiv, also ist $X_i$ kompakt.
		\end{segnb}
		\begin{segnb}[„$\impliedby$“]
			Sei $\scr F$ ein Ultrafilter auf $X$.
			Dann ist $p_i(\scr F)$ ein Ultrafilter auf $X_i$ (Übungsaufgabe). % fixme: ref
			Da $X_i$ kompakt, existiert $x_i \in X_i$ mit $p_i(\scr F) \to x_i$.
			Dann konvergiert $\scr F$ auf $X$ gegen $x = (x_i)_{i\in I}$.
		\end{segnb}
	\end{proof}
\end{st}

\begin{ex}
	\begin{itemize}
		\item
			$\{0,1\}^\N$ ist kompakt.
		\item
			$[a_1,b_1] \times \dotsb \times [a_n,b_n] \subset \R^n$ ist kompakt.
		\item
			$\prod_{k\in\N} [a_k,b_k] \subset \R^\N$ ist kompakt.
		\item
			$\prod_{x\in\R} [a_x,b_x] \subset \R^\R$ ist kompakt.
	\end{itemize}
\end{ex}

\begin{ex}[Hilbert-Würfel]
	Der Hilbert-Würfel $[0,1]^\N$ ist kompakt und metrisierbar.
	Ebenso in $\ell^2(\N)$ ist der Hilbert-Quader
	\[
		H = \Set{ f: \N \to \R | \le f(k) \le \f 1{k+1} }
		= [0,1] \times [0,\f 12] \times [0, \f 13] \times \dotsb
	\]
	Es gilt $H \homeomorphic [0,1]^\N$.
\end{ex}

\begin{st}[verwandte Kompaktheitsbegriffe]
	Für topologische Räume gilt
	\begin{itemize}
		\item
			Kompaktheit impliziert abzählbare Kompaktheit
		\item
			abzählbare Kompaktheit impliziert Kompaktheit unter Voraussetzung des zweiten Abzählbarkeitsaxioms
		\item
			Folgenkompaktheit impliziert abzählbare Kompaktheit
		\item
			abzählbare Kompaktheit impliziert Folgenkompaktheit unter Voraussetzung des ersten Abzählbarkeitsaxioms.
		\item
			Abzählbare Kompaktheit impliziert Pseudokompaktheit
	\end{itemize}
	\begin{proof}
		Wie im metrischen Fall.
		% fixme: ref
	\end{proof}
\end{st}


\section{Lokale Kompaktheit}


\begin{df}
	Ein Raum $(X, \scr T)$ heißt \emph{lokal-kompakt}, wenn jede Umgebung eine Punktes eine kompakte Umgebung enthält.
\end{df}

\begin{ex}
	\begin{itemize}
		\item
			$\R^n$ ist lokal-kompakt.
		\item
			Kein Punkt $x\in \Q$ besitzt eine kompakte Umgebung, insbesondere ist also $\Q$ nicht lokal-kompakt.
	\end{itemize}
\end{ex}

\begin{st}
	Sei $(X, \scr T)$ ein Hausdorff-Raum und zu jedem Punkt $x \in X$ existiere eine kompakte Umgebung.
	Dann ist $(X, \scr T)$ lokal-kompakt.
	\begin{proof}
		Sei $x \in X$ und $K \in \scr U_x$ kompakt.
		Zu jeder offenen Umgebung $U \in \scr U_x$ müssen wir eine kompakte Umgebung $K_U \in \scr U_x$ mit $K_U \subset U$ finden.

		Setze
		\[
			A := U \setminus K
		\]
		(kompakt) und seien $V, W$ offen Umgebungen von $x$, bzw. $A$ mit $V \cap W = \emptyset$.
		Setze
		\[
			K_U := K \setminus W
		\]
		$K_U$ ist abgeschlossen und kompakt und zudem eine Umgebung von $x$ ($x \in K \cap V \subset K \setminus W = K_U$) mit $K_U \subset U$.
	\end{proof}
\end{st}

\begin{kor}
	Jeder kompakte Hausdorff-Raum ist lokal-kompakt.
\end{kor}

\begin{lem}
	Ist $X$ lokal-kompakt, so auch jeder offene und jeder abgechlossene Teilraum.
\end{lem}

\begin{st}
	Ist $X$ lokal-kompakt und $Y = U \cap A$ mit $U, (X \setminus A) \in \scr T$, dann ist $Y$ lokal-kompakt.

	Ist $X$ hausdorffsch, so gilt auch die Umkehrung.

	Ist $Y \subset X$ lokal-kompakt, so existiert eine Zerlegung $Y = U \cap A$ mit $U, (X \setminus A) \in \scr T$.
\end{st}

\begin{df}
	Sei $(X, \scr T)$ ein topologischer Raum.
	Eine Menge $U \subset X$ heißt $K$-offen, wenn für jedes Kompaktum $K \subset X$ die Schnittmenge $U \cap K$ offen in $(K, \scr T_K)$ ist.

	Dies definiert die \emph{kompakt erzeugte Topologie}
	\[
		K \scr T :=
		\Set{ U \subset X | U \text{ ist $K$-offen } }.
	\]
\end{df}

\begin{lem}
	Es gilt
	\[
		\scr T \subset K \scr T = K(K \scr T)
	\]
	und die kompakten Teilräume sind dieselben.
\end{lem}

\begin{st}
	Jeder lokal-kompakte Raum $(X, \scr T)$ ist kompakt erzeugt, d.h. $K \scr T = \scr T$.
	\begin{proof}
		Wir zeigen $K \scr T \subset \scr T$.
		Sei $U \in K \scr T$, d.h. $U \subset X$ mit $U \cap K$ offen in jedem Kompaktum $K \subset X$.
		Wir zeigen $U \in \scr T$.
		Zu $x \in U$ existiert eine kompakte Umgebung $K$ in $X$, also $U \cap K$ offen in $K$, d.h. es existiert $V \in \scr T$ mit $U \cap K = V \cap K$.
		Dies ist eine Umgebung von $x$ in $X$, da $V$ und $K$ solche sind.
		Also ist auch $U \supset V \cap K$ ein Umgebung von $x \in X$.
		Damit ist $U$ offen in $X$, also $U \in \scr T$.
	\end{proof}
\end{st}
