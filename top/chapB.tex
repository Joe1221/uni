\chapter{Metrische Räume}

% paragraph B1
\section{Reelle Zahlen}

\begin{st}[Existenz und Eindeutigkeit der reellen Zahlen]
	Es existiert ein vollständiger, geordneter Körper $(\R,+,\cdot,<)$.

	Je zwei solcher Körper sind isomorph mittels eines eindeutig bestimmten Körperhomomorphismus.
\end{st}


\section{Skalarprodukt und Norm}

\begin{df}
	Sei $\K \in \{\R, \C\}$.
	Auf $V := \K^n$ definieren wir das \emph{Skalarprodukt}
	\[
		\<\argdot, \argdot\> : V \times V \to \K
	\]
	durch
	\[
		\<(x_1,\dotsc, x_n), (y_1,\dotsc, y_n)\> := \_{x_1}y_1 + \dotsb + \_{x_n}y_n.
	\]
	Es gelten
	\begin{enumerate}[({S}0)]
		\item
			$\<x,x\>  \in \R_{\ge 0}$
		\item
			$\<x,x\> > 0$ für $x \neq 0$
		\item
			$\<y,x\> = \_{\<x,y\>}$
		\item
			$\<x,\lambda y + \my z\> = \lambda \<x,y\> + \my\<x,z\>$.
	\end{enumerate}
	Jede Abbildung $V\times V \to \K$ mit (S0-3) ist ein Skalarprodukt.
\end{df}

\begin{ex}
	Sei $\Omega$ eine Menge, $\K^{\Omega} := \{f : \Omega \to \K\}$,
	\[
		\K^{(\Omega)} = \{ f: \Omega \to \K : f \text{ hat endlichen Träger} \}.
	\]
	Dann ist
	\[
		\<f,g\> := \sum_{x\in \Omega} \_{f(x)}g(x)
	\]
	ein Skalarprodukt
\end{ex}

\begin{ex}
	Auf $V = C([a,b], \C)$ ist
	\[
		\<f,g\> = \f 1{b-a} \int_{x-a}^b \_{f(x)} g(x) \dx
	\]
	ein Skalarprodukt.
\end{ex}


