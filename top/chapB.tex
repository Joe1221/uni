\chapter{Metrische Räume}

% paragraph B1
\section{Reelle Zahlen}

\begin{st}[Existenz und Eindeutigkeit der reellen Zahlen]
	Es existiert ein vollständiger, geordneter Körper $(\R,+,\cdot,<)$.

	Je zwei solcher Körper sind isomorph mittels eines eindeutig bestimmten Körperhomomorphismus.
\end{st}


\section{Skalarprodukt und Norm}

\begin{df}
	Sei $\K \in \{\R, \C\}$.
	Auf $V := \K^n$ definieren wir das \emph{Skalarprodukt}
	\[
		\<\argdot, \argdot\> : V \times V \to \K
	\]
	durch
	\[
		\<(x_1,\dotsc, x_n), (y_1,\dotsc, y_n)\> := \_{x_1}y_1 + \dotsb + \_{x_n}y_n.
	\]
	Es gelten
	\begin{enumerate}[({S}0)]
		\item
			$\<x,x\>  \in \R_{\ge 0}$
		\item
			$\<x,x\> > 0$ für $x \neq 0$
		\item
			$\<y,x\> = \_{\<x,y\>}$
		\item
			$\<x,\lambda y + \my z\> = \lambda \<x,y\> + \my\<x,z\>$.
	\end{enumerate}
	Jede Abbildung $V\times V \to \K$ mit (S0-3) ist ein Skalarprodukt.
\end{df}

\begin{ex}
	Sei $\Omega$ eine Menge, $\K^{\Omega} := \{f : \Omega \to \K\}$,
	\[
		\K^{(\Omega)} = \{ f: \Omega \to \K : f \text{ hat endlichen Träger} \}.
	\]
	Dann ist
	\[
		\<f,g\> := \sum_{x\in \Omega} \_{f(x)}g(x)
	\]
	ein Skalarprodukt
\end{ex}

\begin{ex}
	Auf $V = C([a,b], \C)$ ist
	\[
		\<f,g\> = \f 1{b-a} \int_{x-a}^b \_{f(x)} g(x) \dx
	\]
	ein Skalarprodukt.
\end{ex}

\begin{st}
	Sei $V$ ein $K$-Vektorraum und $\<\argdot, \argdot\>$ ein Skalarprodukt.
	Dann gilt die Cauchy-Schwartz-Ungleichung (CSU):
	\[
		|\<u,v\>|^2 \le \<u,u\> \<v,v\>.
	\]
	Hieraus folgt für die Norm $|u| := \sqrt{\<u,u\>}$
	\begin{enumerate}[({N}0)]
		\item
			$|0| = 0$
		\item
			$|v| > 0$ für  $v\neq 0$
		\item
			$|\lambda v| = |\lambda| |v|$ für $v \in V, \lambda \in K$.
		\item
			$|u+v| \le |u| + |v|$ (Dreiecksungleichung)
	\end{enumerate}
\end{st}

\begin{df}
	Eine \emph{Norm} auf $V$ ist eine Abbildung $|\argdot| : V \to \R_{\ge 0}$, die (N0-3) erfüllt.
\end{df}

\begin{ex}
	Auf $\R^n$:
	\begin{itemize}
		\item
			\emph{Taxinorm} oder $\ell^1$-Norm:
			\[
				|x|_1 = |x_1| + \dotsb + |x_n|.
			\]
		\item
			\emph{Euklidische Norm} oder $\ell^2$-Norm:
			\[
				|x|_2 = \sqrt{|x_1|^2 + \dotsb + |x_n|^2}.
			\]
		\item
			\emph{Supremumsnorm} oder $\ell^\infty$-Norm:
			\[
				|x|_\infty := \sup \{ |x_1|, \dotsc, |x_n| \}
			\]
		\item
			Allgemeine $\ell^p$-Norm ($1\le p < \infty$):
			\[
				|x|_p := \Big( |x_1|^p + \dotsb + |x_n|^p \Big)^{\f 1p}
			\]
			% fixme: Skizze zu $B_p := \{x\in \R^2 : |x|_p \le 1\}$
	\end{itemize}
	\begin{note}
		Auf $\R^n$ gilt
		\[
			|x|_\infty
			\le |x|_2
			\le |x|_1
			\le n |x|_\infty
		\]
	\end{note}
\end{ex}

\begin{ex}
	Sei $\K = \R, \C$ und $\Omega$ ein Menge.
	Wir schreiben
	\begin{align*}
		\Abb(\Omega, \K) = \K^\Omega &:= \{f : \Omega \to \K\} \\
		\|f\|_{\infty} := |f|_\Omega &:= \sup \{ |f(x)| : x \in \Omega \} \\
		\|f\|_{p} &:= \bigg( \sum_{x\in\Omega} |f(x)|^p \bigg)^{\f 1p} \\
		\ell^p(\Omega) &:= \{ f : \Omega \to \K : \|f\|_p < \infty \}
	\end{align*}
\end{ex}

\begin{ex}
	Sei $\Omega \subset \R^n$ messbar, $f : \Omega \to \R$ messbar,
	\begin{align*}
		\|f\|_p &:= \bigg( \int_{x\in \Omega} |f(x)|^p \dx \bigg)^{\f 1p} \\
		\scr L^p(\Omega) &:= \{ f: \Omega \to \K : \|f\|_p < \infty \}
	\end{align*}
	Aus $\|f\|_p = 0$ folgt nur $f=0$ fast überall.
	Setze
	\begin{align*}
		N &:= \{ f : \Omega \to \K : f(x) = 0 \text{ für fast alle $x\in\Omega$} \\
		L^p &:= \ell^p / N
	\end{align*}
	Auf $L^p(\Omega)$ ist $\|\argdot\|_p$ tatsächlich eine Norm.
\end{ex}

\begin{ex}
	Für Matrizen $A \in \K^{m\times n}$ setze
	\[
		|A| := \bigg( \sum_{i=1}^m \sum_{j=1}^n |a_{ij}|^2 \bigg)^{\f 12}.
	\]
	Dies ist ein Norm (Euklidische Norm auf $\K^{mn}$) und erfüllt zudem
	\[
		|A B | \le |A| |B|,
		\qquad A \in \K^{m\times n}, \B \in \K^{n\times m}.
	\]
	Insbesondere $|Av| \le |A| |v|$ für $v\in \K^n (= \K^{n\times 1})$.
\end{ex}

\begin{ex}
	$\K^{n\times n}$ ist eine Algebra über $\K$ mit Norm $|A|$ wie oben.
\end{ex}

\begin{df}
	Eine \emph{normierte $\K$-Algebra} ist eine Algebra $(A_i)$ über $\K$ mit einer Norm $|\argdot|: A \to \R_{\ge 0}$, sodass $|uv| \le |u||v|$ für alle $u,v \in A$ gilt.
\end{df}

\begin{ex}
	Sei $\K = \R, \C$.
	Beispiele für normierte Algebren sind
	\begin{itemize}
		\item
			$\K^{n\times n}$
		\item
			$\ell^\infty(\Omega, \K)$
	\end{itemize}
\end{ex}


\section{Metrische Räume}


\begin{ex}
	Betrachte den Euklidischen Abstand auf $\R^n$
	\[
		d(x,y) = |x-y|_2 = \sqrt{ (x_1 - y_2)^2  + \dotsb + (x_n - y_n)^2 }.
	\]
	Es gelten folgende Eigenschaften
	\begin{enumerate}[{(M)}0]
		\item
			$d(x,x) = 0$
		\item
			$d(x,y) > 0$ für $x\neq y$
		\item
			$d(x,y) = d(y,x)$
		\item
			$d(x,z) \le d(x,y) + d(y,z)$
	\end{enumerate}
\end{ex}

\begin{df}
	Sei $X$ eine Menge.
	Eine \emph{Metrik} auf $X$ ist eine Abbildung $d : X \times X \to [0, \infty]$, die (M0-3) erfüllt.

	Das Paar $(X,d)$ heißt dann \emph{metrischer Raum}.
\end{df}

\begin{ex}
	$X = \R^n$ mit $d$ der euklidischen Metrik.
\end{ex}

\begin{ex}
	Auf jeder Menge $X$ haben wir die \emph{diskrete Metrik}
	\[
		d(x,y) = \begin{cases}
			0 & x=y \\
			1 & x\neq y
		\end{cases}.
	\]
\end{ex}

\begin{ex}
	Die \emph{französische Eisenbahnmetrik} $d: \R^n \times \R^n \to \R_{\ge 0}$
	\[
		d(x,y) := \begin{cases}
			|x-y|_2 & \R x = \R y \\
			|x|_2 + |y|_2 & \R x \neq \R y
		\end{cases}
	\]
\end{ex}

\begin{ex}[Teilräume]
	Ist $(X,d)$ ein metrischer Raum und $A \subset X$, dann ist $d_A := d\big|_{A\times A} : A \times A \to [0,\infty]$ eine Metrik auf $A$.
\end{ex}

\begin{ex}[Produkträume]
	Seien $(X,d_i)$ mit $i \in I$ metrische Räume.
	Auf $X = \prod_{i\in I} X_i$ (dem Produktraum) erhalten wir die Supremumsmetrik
	\[
		d(x,y) := \sup \{ d_i(x_i,y_i) : i \in I \}.
	\]
\end{ex}

\begin{ex}[Abbildungsräume]
	Ist $(Y,d_y)$ ein metrischer Raum, $X$ eine Menge, dann trägt $Y^X = \Abb(X,Y)$ die Metrik
	\[
		d(f,g) := \sup \{ d_y(f(x),g(x)) : x \in X \}
	\]
	für $f,g : X \to Y$.
\end{ex}

\begin{df}
	Eine \emph{isometrische Einbettung} $f:(X,d_X) \to (Y,d_Y)$ ist eine Abbildung $f: X\to Y$ mit $d(f(a), f(b)) = d_X(a,b)$ für alle $a,b \in X$.
	Ist $f$ bijektiv, so heißt $f$ eine \emph{Isometrie}.
\end{df}

\begin{ex}
	Verschiebungen, Drehungen, Spiegelungen auf $\R^n$.
\end{ex}

\begin{ex}
	Sei $m \le n$, dann ist	$f: \R^n \to \R^n$ mit
	\[
		f(x_1, \dotsc, x_m) = (x_1, \dotsc, x_m, 0, \dotsc, 0)^T
	\]
	eine isometrischen Einbettung.
\end{ex}

\begin{ex}
	Die Räume $\ell^2(\Z, \C)$ und $L^2([0,2\pi], \C)$ sind isometrisch dank Fourier-Analyse/Synthese.
\end{ex}

Seien $(X,d_X), (Y,d_Y)$ metrische Räume.
Jede Funktion $f : X \to Y$ erfüllt
\[
	l d(a,b) \le d(f(a), f(b)) \le L d(a,b)
\]
für alle $a,b \in X$ mit den Konstanten $l=0, L = \infty$.

\begin{df}
	Wir nennen $f$ \emph{Lipschitz-stetig} wenn
	\[
		l d(a,b) \le d(f(a), f(b)) \le L d(a,b)
	\]
	für ein $L$ mit $0 \le L < \infty$ gilt.

	Ist zusätzlich $0 < l \le L < \infty$, so heißt $f$ \emph{bi-Lipschitz-stetig}.
\end{df}

\begin{ex}
	\begin{itemize}
		\item
			$f$ ist Isometrie genau dann, wenn $l=L=1$ genügt.
		\item
			$f$ ist konstant genau dann, wenn $L=0$ genügt.
	\end{itemize}
\end{ex}

\begin{ex}
	\begin{itemize}
		\item
			$f(x) = x^2$, $L=2, l=0$.
		\item
			$g(x) = \sqrt{x}$, $L=\infty, l=\f 12$.
	\end{itemize}
\end{ex}

\begin{df}
	In einem metrischen Raum $(X,d)$ ist der \emph{offene Ball} um $a \in X$ mit Radius $r\in [0,\infty]$ die Menge
	\[
		B(a,r) := B_{(X,d)}(a, r) := \{ x \in X : d(a,x) < r \}.
	\]
	Der \emph{abgeschlossene Ball} ist
	\[
		\_{B}(a,r) := \_{B}_{(X,d)}(a, r) := \{ x \in X : d(a,x) \le r \}.
	\]
\end{df}

\begin{df}
	Eine Menge $U \subset X$ heißt \emph{Umgebung} von $a$ in $(X,d)$, wenn es $\eps \in \R_{> 0}$ gibt, sodass $B(a,\eps) \subset U$.

	Die Menge $U$ heißt \emph{offen} in $(X,d)$, wenn sie Umgebung jedes Punktes $a \in U$ ist.

	Eine Menge $A \subset X$ heißt \emph{abgeschlossen} in $(X,d)$, wenn $X \setminus A$ offen ist in $(X,d)$.

	Die Familie $\_{S_{X,d}}$ aller offenen Mengen heißt die \emph{Topologie} des Raumes $(X,d)$.
\end{df}

\begin{prop}
	Jeder offene Ball ist offen.
	Jeder abgeschlossene Ball ist abgeschlossen.

	Es gelten
	\begin{enumerate}[({O}1)]
		\item
			$\emptyset, X$ sind offen.
		\item
			$U_1,\dotsc,U_n$ offen $\implies U_1 \cap \dotsb \cap U_n$ offen.
		\item
			$U_i$ offen ($i\in I$) $\implies \bigcup_{i\in I} U_i$ offen.
	\end{enumerate}
	\begin{enumerate}[({A}1)]
		\item
			$\emptyset, X$ sind abgeschlossen.
		\item
			$A_1,\dotsc,A_n$ abgeschlossen $\implies A_1 \cup \dotsb \cup A_n$ offen.
		\item
			$A_i$ abgeschlossen ($i\in I$) $\implies \bigcap_{i\in I} A_i$ offen.
	\end{enumerate}
\end{prop}

\begin{df}
	Zu $M \subset X$ definieren wir das \emph{Innere} von $M$ in $(X,d)$ durch
	\begin{align*}
		\mathring M &:= \{ x \in X : M \text{ ist Umgebung von $x$ in $(X,d)$} \} \\
		&= \{ x \in X : \exists \eps > 0 B(x,\eps) \subset M \} \\
		&= \bigcup \{ U \subset M : U \text{ ist offen in $(X,d)$}.
		% fixme: prüfen
	\end{align*}

	Der \emph{Abschluss} (oder abgeschlossene Hülle) von $M$ in $(X,d)$ ist
	\begin{align*}
		\_{M} &:= \{ x \in X : \text{ Jede Umgebung von $x$ in $(X,d)$ schneidet $M$} \} \\
		&= \{ x \in X : \forall \eps > 0 : B(x,\eps) \cap M \neq \emptyset \} \\
		&= \bigcap \{ A \supset M : \text{$A$ ist abgschlossen in $(X,d)$} \}
		% fixme: prüfen
	\end{align*}

	Der \emph{Rand} $\boundary M$ von $M$ in $(X,d)$ ist
	\[
		\boundary M := \_{M} \setminus \mathring M
	\]
\end{df}

\begin{ex}
	Im $\R^n$ betrachten wir
	\begin{align*}
		\D^n := \{ x \in \R^n : x_1^2 + \dotsb + x_n^2 \le 1 \}, \\
		\B^n := \{ x \in \R^n : x_1^2 + \dotsb + x_n^2 < 1 \}, \\
		\S^{n-1} := \{ x \in \R^n : x_1^2 + \dotsb + x_n^2 = 1 \}.
	\end{align*}
\end{ex}

\coursetimestamp{21}{10}{2013}

\begin{nt}
	Wir verwenden für ein Mengensystem $\scr A$ die Notation für die Vereinigung über alle enthaltenen Mengen
	\[
		\bigcup \scr A
		= \bigcup_{A \in \scr A}
		= \bigcup_{i\in I} A_i.
	\]
\end{nt}


\section{Konvergenz und Stetigkeit}

Sei im Folgenden $(X,d)$ stets ein metrischer Raum.

\begin{df}
	Eine Folge $(x_n)_{n\in \N}$ in $X$ \emph{konvergiert} gegen $a \in X$, wenn $d(x_n, a) \to 0$, d.h.
	\[
		\forall \eps > 0 \exists m \in \N \forall n \ge m : d(x_n, a) < \eps,
	\]
	oder mit anderen Worten: Jede Umgebung von $a$ enthält fast alle Folgenglieder.
\end{df}

\begin{df}
	Eine Funktion $f: X \to Y$ heißt stetig in einem Punkt $a \in X$, wenn
	\[
		\forall \eps > 0 \exists \delta > 0 : \forall x \in X : d_X(x,a) < \delta \implies d_Y(f(x), f(a)) < \eps.
	\]

	Die Funktion $f: X \to Y$ heißt \emph{stetig} (auf ganz $X$), wenn
	\[
		\forall a \in X \forall \eps > 0 \exists \delta > 0 \forall x \in X: d_X(x,a) < \delta \implies d_Y(f(x), f(a)) < \eps
	\]
	und \emph{gleichmäßig stetig}, wenn
	\[
		\forall \eps > 0 \exists \delta > 0 \forall x,a \in X: d_X(x,a) < \delta \implies d_Y(f(x), f(a)) < \eps
	\]

	Die Menge aller stetigen Abbildung $f: X \to Y$ bezeichnen wir mit $C(X,Y)$.

	Ist $f$ bijektiv und sowohl $f$, als auch $f^{-1}$ stetig, so heißt $f$ \emph{Homöomorphismus}.
\end{df}

\begin{st}
	Für $f: X \to Y$ sind äquivalent
	\begin{enumerate}[1)]
		\item
			$f$ ist stetig,
		\item
			$f$ ist folgenstetig, d.h. aus $x_n \to a$ in $X$ folgt $f(x_n) \to f(a)$ in $Y$,
		\item
			Für $V \subset Y$ offen ist $f^{-1}(V) \subset X$ offen,
		\item
			Für $V \subset Y$ abgeschlossen ist $f^{-1}(V) \subset X$ abgeschlossen.
	\end{enumerate}
	\begin{proof}
		Übungsaufgabe % fixme: reference
	\end{proof}
\end{st}

\begin{df}
	Eine Folge $f_n: X \to Y$ \emph{konvergiert punktweise} gegen $f: X \to Y$, wenn für alle $x \in X$ gilt $f_n(x) \to f(x)$, d.h.
	\[
		\forall x \in X \forall \eps > 0 \exists m \in \N \forall n \ge m : d_Y(f_n(x), f(x)) \le \eps.
	\]
	Hingegen konvergiert $f_n: X \to Y$ \emph{gleichmäßig} gegen $f: X \to Y$, wenn
	\[
		\forall \eps > 0 \exists m \in \N \forall n \ge m \underbrace{\forall x \in X : d_Y(f_n(x), f(x)) \le \eps}_{\iff d (f_n, f) \le \eps},
	\]
	wobei
	\[
		d(f,g)
		:= \sup \{ d_Y(f(x), g(x)) : x \in X \}.
	\]
\end{df}

\begin{ex}
	Sei $f_n : [0,1] \to \R, f_n(x) := x^n$.
	Dann konvergiert $(f_n)$ punktweise gegen $f:[0,1] \to \R$ mit $f(x) = 0$ für $0 \le x < 1$ und $f(x) = 1$ für $x=1$.

	Die Konvergenz ist nicht gleichmäßig, denn
	\[
		|f_n - f|_{[0,1]} = 1
	\]
\end{ex}

\begin{ex}
	Die Polynome $f_n(x) = \sum_{k=0}^n \f{(-1)^k}{2k!} x^{2k}$ konvergieren in jedem $x \in \R$ gegen $f(x) = \cos(x)$.

	Die Konvergenz ist jedoch nicht gleichmäßig, denn es gilt
	\[
		|f_n - f|_{\R} = \infty.
	\]
	Wir haben aber immerhin gleichmäßige Konvergenz auf jedem Kompaktum $[-r,r] \subset \R$.
\end{ex}

\begin{st}
	Konvergiert $f_n \to f$ gleichmäßig und sind alle $f_n$ stetig, so auch $f$.
	\begin{proof}
		Übungsaufgabe % fixme: reference
	\end{proof}
\end{st}

\begin{st}[Zwischenwertsatz]
	Jede stetige Funktion $f: [a,b] \to \R$ hat die Zwischenwerteigenschaft (ZWE): zu $y \in \R$ mit $f(a) \le y \le f(b)$ existiert $x \in \R$ mit $x \in [a,b]$ mit $f(x) = y$.
	\begin{nt}
		Die Vollständigkeit von $\R$ ist hier wesentlich.
		Betrachte $f: \Q \to \Q$ mit $f(x) = x^2 - 2$ hat nicht die ZWE, denn $f(1) = - 1 < 0$, $f(z) = 2 > 0$, aber es gibt kein $x \in \Q$ mit $f(x) = 0$.
	\end{nt}
\end{st}

\begin{st}
	Für jeden metrischen Raum $(X,d)$ sind folgende Aussagen äquivalent:
	\begin{enumerate}[(1)]
		\item
			Jede stetige Funktion $f: X \to \R$ hat die ZWE,
		\item
			Für jede stetige Funktion $f: X \to \R$ ist $f(X) \subset \R$ ein Intervall,
		\item
			Jede stetige Funktion $f: X \to \{0,1\} \subset \R$ ist konstant.
		\item
			Für jede offene Zerlegung $X = A \dunion B$ gilt $A = \emptyset$ oder $B = \emptyset$.
	\end{enumerate}
	In diesem Fall nennen wir $(X,d)$ \em{zusammenhängend}.
	\begin{proof}
		$(1) \implies (2) \implies (3)$ sind klar.
		\begin{seg}[$(3) \implies (4)$]
			Ist $X = A \dunion B$ eine offene Zerlegung, so ist \oBdA $f: X \to \{0, 1\}$ mit $f(A) = \{0\}$ und $f(B) = \{1\}$ stetig.
		\end{seg}
		\begin{seg}[$(4) \implies (1)$]
			Angenommen $f: X \to \R$ stetig und zu $a,b \in X$ existiert $y \in \R$ mit $f(a) < y < f(b)$ aber $y \not\in f(X)$.
			Dann wäre $A = f^{-1}(\R_{<Y})$ und $B = f^{-1}(\R_{>0})$ eine offene Zerlegung von $X$.
			Wegen $a \in A$ und $b \in B$ widerspricht dies $(4)$.
		\end{seg}
	\end{proof}
\end{st}

\begin{ex}
	Jedes Intervall $I \subset \R$ ist zusammenhängend (ZWS).
	Hingegene ist $\R \setminus \{a\}$ nicht zusammenhängend ($A = \R_{<a}, B=\R_{>a}$ sind nichtleere offene Zerlegungen).
\end{ex}

\begin{st}
	Ist $f: X \to Y$ stetig und $X$ zusammenhängend, so ist auch $f(X) \subset Y$ zusammenhängend.
	\begin{proof}
		Ist $f(X) = A \dunion B$ eine offene Zerlegung, dann auch $X = f^{-1}(A) \dunion f^{-1}(B)$.
	\end{proof}
\end{st}

\begin{df}
	Ein \emph{Weg} im Raum $X$ ist eine stetige Abbildung $\gamma: [0,1] \to X$.
	Dabei heißt $\gamma(0)$ \emph{Anfangspunkt} und $\gamma(1)$ \emph{Endpunkt}.

	Der Raum $X$ heißt \emph{wegzusammenhängend}, wenn zu jedem Paar $a,b \in X$ ein Weg von $a$ nach $b$ in $X$ existiert (d.h. $\gamma:[0,1] \to X$ stetig mit $\gamma(0) = a, \gamma(1) = b$).
\end{df}

\begin{ex}
	$\R^n$ ist wegzusammenhängend.

	$\R^n \setminus \{x_0\}$ ist wegzusammenhängend für $n \ge 2$

	$\S^n$ ist wegzusammenhängend für $n \ge 1$.
\end{ex}

\begin{st}
	Jeder wegzusammenhängende Raum ist zusammenhängend (i.A. aber nicht umgekehrt).
	\begin{proof}
		Angenommen $X = A \dunion B$ sei offene Zerlegung mit $a \in A$ und $b \in B$.
		Dann existiert ein Weg $\gamma: [0,1] \to X$ mit $\gamma(0) = a$ und $\gamma(1) = b$.
		Also ist $[0,1] = f^{-1}(A) \dunion f^{-1}(B)$ eine offene Zerlegung mit $0 \in f^{-1}(A)$ und $1 \in f^{-1}(B)$, ein Widerspruch.

		Ein Gegenbeispiel
		\[
			X := \{ (x,\sin(\f 1x) : x \in \R>0) \cup ( \{0\} \times [0,1] )
		\]
		ist zusammenhängend, aber nicht wegzusammenhängend.
	\end{proof}
\end{st}

\begin{df}
	Zwei Metriken $d,e : X \times X \to [0, \infty]$ heißen (topologisch) \emph{äquivalent}, wenn sie die selben offenen Mengen definieren.
	Wir nennen $d$ \emph{feiner} als $e$ (oder $e$ ist gröber als $d$), wenn jeder $\eps$-Ball von $e$ einen $\delta$-Ball von $d$ enthält, d.h.
	\[
		\forall a \in X \forall \eps > 0 \exists \delta > 0 : B_{(X,d)}(a, \delta) \subset B_{(X,e)}(a, \eps).
	\]
\end{df}

\begin{prop}
	Äquivalent sind
	\begin{enumerate}[(1)]
		\item
			$d$ ist feiner als $e$
		\item
			Jede Umgebung bezüglich $e$ ist auch Umgebung bezüglich $d$.
		\item
			Jede offene Menge bezüglich $e$ ist auch offene Menge bezüglich $d$.
		\item
			Die Identität $\Id: (X,d) \to (X,e), x \mapsto x$ ist stetig.
		\item
			Gilt $x_n \to a$ bezüglich $d$, dann auch $x_n \to a$ bezüglich $e$.
		\item
			Ist $f: X \to Y$ stetig bezüglich $e$, dann auch bezüglich $d$.
		\item
			Ist $f: Y \to X$ stetig bezüglich $d$, dann auch bezüglich $e$.
	\end{enumerate}
	Seien $|\argdot|$ und $\|\argdot\|$ zwei Normen auf einem $K$-Vektorraum $V$.
	$|\argdot|$ ist genau dann feiner als $\|\argdot\|$, wenn ein $L \in \R_{>0}$ existiert mit $\|x\| \le L |x|$ für alle $x \in V$.
	$|\argdot|$ und $\|\argdot\|$ sind genau dann äquivalent, wenn $l,L \in \R_{>0}$ existiert, sodass
	\[
		l|x| \le \|x\| \le L|x|
	\]
	für alle $x \in X$ gilt.
\end{prop}

\begin{ex}
	Sei $1 \le p < q \le \infty$.
	\begin{enumerate}[(1)]
		\item
			Auf $\R^n$ sind die $\ell^p$-Norm und die $\ell^q$-Norm äquivalent.
		\item
			Auf $\R^(\N)$ ist die $\ell^p$-Norm echt feiner als die $\ell^q$-Norm.

			Betrachte dazu
			\[
				f_n(k) = \begin{cases}
					n^{-\f 1p} & 1 \le k \le n \\
					0 & k > n
				\end{cases}.
			\]
			Es gilt
			\begin{align*}
				|f_n|_p &= 1, \\
				|f_n|_q &= n^{\f 1q-\f 1p} \to 0 \qquad (n\to \infty).
			\end{align*}
		\item
			Auf $C([a,b], \R)$ ist die $L^q$-Norm echt feiner als die $L^p$-Norm.
		\item
			Auf $C_C(\R,\R)$ (stetige Funktionen mit kompaktem Träger) sind $L^q$- und $L^p$-Norm unvergleichbar.
	\end{enumerate}
\end{ex}

\begin{ex}
	Ist $d: X \times X \to [0,\infty]$ eine Metrik, so auch $d': X \times X \to [0,1]$ (gestauchte Metrik) mit
	\[
		d'(x,y) = \f {d(x,y)}{1 + d(x,y)}.
	\]
	Hieraus lässt sich $d$ rekonstruieren durch
	\[
		d(x,y) = \f {d'(x,y)}{1-d'(x,y)}.
	\]
	Diese beiden sind äquivalent, d.h. sie definieren die selben Topologien.
	Ebenso die gekappte Metrik
	\[
		d^*(x,y) := \min \{d(x,y), 1 \}.
	\]
\end{ex}


Für $A, B \subset X$ und $a,b \in X$ definieren wir
\begin{align*}
	d(a,B)
		&:= \inf \{ d(a,b) : b \in B \}, \\
	d(A,B)
		&:= \inf \{ d(a,b) : a \in A, b \in B \}.
\end{align*}

\coursetimestamp{22}{10}{2013}

\section{Vollständige metrische Räume}


\begin{df}
	Eine Folge $(x_n)_{n\in \N}$ in einem metrischen Raum $(X,d)$ heißt \emph{Cauchy-Folge}, wenn
	\[
		\forall \eps > 0 \exists n \in \N, \forall p,q \ge n : d(x_p,x_q) \le \eps.
	\]
	Wir nennen $(X,d)$ \emph{vollständig}, wenn jede Cauchy-Folge in $(X,d)$ einen Grenzwert in $X$ besitzt.
\end{df}

\begin{nt}
	Jede konvergente Folge ist eine Cauchy-Folge, aber im Allgemeinen nicht umgekehrt.
\end{nt}

\begin{ex}
	In $X = ]0,1]$ ist $x_n = 2^{-n}$ eine Cauchy-Folge, aber nicht konvergent.

	In $\Q$ ist $x_n = \sum_{k=0}^n \f 1{k!}$ eine Cauchy-Folge, aber nicht konvergent.
\end{ex}

\begin{st}
	$\R$ ist vollständig (bezüglich der üblichen euklidischen Metrik).

	Ebenso $\C$ und $\R^n, \C^n$ für $n = 1, 2, \dotsc$.
\end{st}

\begin{st}
	Seien $(X_i,d_i)$ vollständige metrische Räume für $i \in I$.
	Dann ist auch $X = \prod_{i\in I} X_i$ vollständig bezüglich der Supremums-Metrik
	\[
		d(x,y) := \sup \{ d_i(x_i, y_i) : i \in I \}.
	\]
	\begin{proof}
		% fixme: bezeichnung x_i, x_n
		Sei $(x_n)_{n\in \N}$ eine Cauchy-Folge in $(X,d)$.
		Dann ist für $i \in I$ auch $(x_{n,i})_{n\in \N}$ eine Cauchy-Folge in $(X_i,d_i)$.
		Dank Vollständigkeit existiert $x_i \in X_i$ mit $x_{n,i} \to x_i$.
		Definiere $ x= (x_i)_{i\in I} \in X$.

		Zu  $\eps > 0$ existiert $m \in \N$ sodass für $n,k > m$ gilt
		\[
			d_i(x_{n,i}, x_{k,i}) \le \eps
		\]
		Für $k \to \infty$ erhalten wir
		\[
			d_i(x_{n,i},x_i) \le \eps
		\]
		für $n \ge m$ und $i \in I$.
		Also $d(x_n, x) \le \eps$ für $n \ge m$ und somit $x_n \to x$.
	\end{proof}
\end{st}

\begin{lem}
	Sei $(X,d)$ vollständig und $A \subset X$.
	$(A,d_A)$ ist vollständig genau dann, wenn $A$ in $X$ abgeschlossen ist.
	\begin{proof}
		Übung
	\end{proof}
\end{lem}

\begin{st}
	Ist $(Y,d_Y)$ vollständig, so auch $\Abb(X,Y) = Y^X$ mit der Supremumsmetrik.
	Hierin ist $C(X,Y)$ abgeschlossen und somit ebenfalls vollständig.
\end{st}

\begin{ex}
	$C([0,1] \to \R)$ sind vollständig bezüglich der Supremums-Metrik.
	% fixme: wie funktioniert der beweis?
\end{ex}

Das Prinzip der Vollständigkeit hat in der Analysis viele Anwendungen: Banachscher Fixpunktsatz, Satz von Picard-Linelöf, Potenzreihen (in $\R, \C$ oder Banach-Algebren).


\section{Kompaktheit}


\begin{df}
	Ein metrischer Raum $(X,d)$ heißt \emph{kompakt}, wenn jede offene Überdeckung eine endliche Teilüberdeckung enthält, d.h. zu jeder Überdeckung $X = \bigcup_{i \in I} U_i$ mit $U_i \subset X$ offen existieren $i_1, \dotsc, i_n$ für die $X = \bigcup_{i=1}^n U_i$ gilt.
\end{df}

\begin{df}
	% fixme: diskreter metrischer Raum
\end{df}

\begin{ex}
	Für $(X,d)$ diskret ist $(X,d)$ kompakt genau dann, wenn $X$ endlich ist.

	$\R^n$ ist nicht kompakt.
	Betrachte
	\[
		\R^n = \bigcup_{n\in \N} B(0,n).
	\]
	Hier existiert keine endliche Teilüberdeckung.
\end{ex}

\begin{df}
	Wir nennen $\delta \in \R_{>0}$ \emph{Lebesgue-Zahl} einer Überdeckung $X = \bigcup_{i\in I} U_i$,  wenn zu jedem $x \in X$ ein Index $i$ existiert, sodass $B(x,\delta) \bigcup U_i$ gilt.
\end{df}

\begin{ex}
	Ist $(X,d)$ diskret, dann ist $\delta \in ]0,1]$ Lebesgue-Zahl zu jeder Überdeckung, denn $B(x,\delta) = \{x\}$.

	Betrachte
	\[
		\R = \bigcup_{n \in \Z} ]n-1, n+1[
	\]
	Hier ist jede Zahl $\delta \in ]0,\f 12]$ Lebesgue-Zahl.

	Betrachte
	\[
		\R = \bigcup_{n\in\N} ] \log n, \log(n+2) [,
	\]
	wobei $\log 0 = -\infty$.
	Hierzu existiert keine Lebesgue-Zahl, denn
	\[
		|\log (n+2) - \log(n)| = \log (\f {n+2}n) \to 0
	\]
	für $n \to \infty$.
\end{ex}

\begin{df}
	Ein metrischer Raum $(X,d)$ heißt \emph{totalbeschränkt}, wenn zu jedem $\eps > 0$ eine endliche Familie $a_1, \dotsc, a_n \in X$ existiert mit $X = B(a_n,\eps) \cup \dotsc \cup B(a_n,\eps)$.
\end{df}

\begin{nt}
	Jeder totalbeschränkte Raum ist auch beschränkt, umgekehrt im Allgemeinen jedoch nicht.
	\begin{proof}
		Es gilt
		\[
			d(x,a_1) < \eps + \max \{d(a_1, a_k) : k = 2,\dotsc, n \}.
		\]

		In $\ell^\infty(\N, \R)$ ist $\_B(0,1)$ beschränkt, aber nicht totalbeschränkt.
		Für die kanonische Basis $(e_n)_{n\in \N}$ gilt
		\[
			|e_n - e_m|_\infty = 1
			\qquad n \neq m.
		\]
		Für $\eps = \f 12$ enthält $B(a_i,\eps)$ also höchsten ein $e_n$.
	\end{proof}
\end{nt}

\begin{prop}
	In $\R^n$ gilt: jede beschränkte Menge $A \subset \R^n$ ist total beschränkt.
	\begin{proof}
		Übung
		% beginne mit würfel
	\end{proof}
\end{prop}

\begin{st}[Charakterisierung kompakter metrischer Räume]
	Für jeden metrischen Raum $(X,d)$ sind äquivalent
	\begin{enumerate}[(1)]
		\item
			$(X,d)$ ist kompakt
		\item
			Abzählbare Kompaktheit, d.h. jede abzählbare Überdeckung enthält eine endliche Teilüberdeckung.
		\item
			Häufungspunkte: jede Folge $(x_n)_{n\in \N}$ in $X$ hat einen Häufungspunkt $a \in X$.
		\item
			Folgenkompaktheit: jede Folge $(x_n)_{n\in \N}$ in $X$ hat eine konvergente Teilfolge.
		\item
			Pseudokompaktheit: jede stetige Funktio $f: X \to \R$ nimmt Minimum und Maximum an, d.h. es gibt $a,b \in X$ mit $f(a) \le f(x) \le f(b)$ für alle $x \in X$.
		\item
			Lebesgue-Kompaktheit: $(X,d)$ ist totalbeschränkt und jede offene Überdeckung erlaubt eine Lebesgue-Zahl.
		\item
			Heine-Borel-Lebesgue-Kompaktheit: $(X,d)$ ist vollständig und totalbeschränkt.
	\end{enumerate}
	\begin{proof}
		$(1) \implies (2)$ ist trivial.
		\begin{seg}[$(2) \implies (3)$]
			Zu $(x_n)_{n\in\N}$ sei $E_n = \{x_k : k \ge n\}$ und $A_n = \_{E_n}$ abgeschlossen und $U_n = X \setminus A_n$ offen.
			Wir haben $E_0 \supset E_1 \supset \dotsb \supsetneq \emptyset$, daher ist $A_0 \supset A_1 \supset \dotsb \supsetneq \emptyset$ und $U_0 \subset U_1 \subset \dotsb \subsetneq X$.

			Jedes $a \in A := \bigcap_{n\in \N} A_n$ ist ein Häufungspunkt von $(x_n)_{n\in\N}$.
			Wäre $A = \emptyset$, dann wäre $\bigcup_{n \in \N} U_n = X$ eine offene Überdeckung ohne endliche Teilüberdeckung, ein Widerspruch.
		\end{seg}
		\begin{seg}[$(3) \implies (4)$]
			Zu $(x_n)_{n\in N}$ in $X$ existiert ein Häufungspunkt $a \in X$.
			Sei $n_0 \in \N$.
			Zu $k \in \N_{\ge 1}$ enthält $B(a, 2^{-k})$ unendlich viele $x_n$, also existiert $n_k > n_{k-1}$ mit $x_{n_k} \in B(a,2^{-k})$.
			Dann ist $x_{n_k} \to a$.
		\end{seg}
		\begin{seg}[$(4) \implies (5)$]
			Es existiert eine Folge $(x_n)$ in $X$ mit $f(x_n) \to \sup f$.
			Eine konvergente Teilfolge $x_{n_k} \to b$ liefert $f(x_{n_k}) \to \sup f$ und $f(x_{n_k}) \to f(b)$ dank Stetigkeit, also $f(b) = \sup f$.
		\end{seg}
		\begin{seg}[$(5) \implies (6)$]
			Sei $X = \bigcup_{i \in I} U_i$ offene Überdeckung, $A_i = X \setminus U_i$.
			Dann ist $f: X \to \R$ mit
			\[
				f(x) = \sup \{ d(x, A_i) : i \in I \}
			\]
			stetig.
			Dank $(5)$ existiert $x_0 \in X$ mit $f(x_0) = \inf f$.
			Für jedes $x \in X$ existiert $i \in I$ mit $x \in U_i$, also $x \not\in A_i$.
			Da $A_i$ abgeschlossen ist, folgt $d(x,A_i) > 0$, also $f(x) > 0$.
			Insbesondere gilt $\inf f = f(x_0) > 0$.
			Jedes $\delta \in ]0, \inf f[$ ist eine Lebesgue-Zahl.

			Angenommen für ein $\eps \in \R_{>0}$ und jede endliche Familie $a_1, \dotsc, a_n \in X$ gilt $B(a_1,\eps) \cup \dotsb \cup B(a_n,\eps) \subsetneq X$
			Per Induktion existiert $a_1, a_2, \dotsc$ mit $d(a_m,a_n) \ge \eps$ für alle $m \neq n$.
			Die Funktion $f_n : X \to [0,1]$ mit
			\[
				f_n(x) = \max \{ 0, 1 - \f 3\eps d(x,a_n) \}
			\]
			ist stetig und hat Träger in $\_B(a_n, \f \eps3)$ (jeweils disjunkt).
			Also ist
			\[
				f(x) = \sum_{n\in \N} n f_n
			\]
			stetig, aber unbeschränkt.
		\end{seg}
		\begin{seg}[$(6) \implies (7)$]
			Sei $(x_n)_{n \in \N}$ eine Cauchy-Folge in $(X,d)$.
			Angenommen zu jedem $a \in X$ existiert $\eps > 0$ sodass $U_a = B(a, \eps)$ nur endlich viele $x_n$ enthält.
			Zu
			\[
				X = \bigcup_{a \in X} U_a
			\]
			existiert eine Lebesgue-Zahl $\delta > 0$ nach (6).
			Wegen Totalbeschränktheit (6) gilt $X = B(a_1, \delta) \cup \dotsc \cup B(a_n,\delta)$.
			Da $B(a_k,\delta) \subset U_{a(k)}$ nur endlich viele $x_n$ enthält, erreichen wir einen Widerspruch.
			D.h. es gibt mindestens einen Häufungspunkt $a \in x$.
			Da $(x_n)$ Cauchy-Folge ist, folgt $x_n \to a$.
		\end{seg}
		\begin{seg}[$(7) \implies (4) \implies (3) \implies (2) \implies (1)$]
			siehe Vorlesungsnotizen.
			%fixme: reference or copy
		\end{seg}
	\end{proof}
\end{st}

