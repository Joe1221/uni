\chapter{Die Fundamentalgruppe \texorpdfstring{$\pi_1(X, x_0)$}{π₁(X, x₀)}}



Bisher hatten wir $\pi_0: \Cat{Top} \to \Cat{Set}$, unser Ziel ist $\pi_1: \Cat{Top_*} \to \Cat{Grp}$.


\section{Das Fundamentalgruppoid $\pi(X)$}


\begin{df}
	Setze
	\begin{align*}
		P(X) &:= \scr C([0,1], X) = \{ \gamma : [0,1] \to X \text{ stetig} \}, \\
		P(X,a,b) &:= \scr C([0,1], X) = \{ \gamma : [0,1] \to X \text{ stetig mit $\gamma(0) = a$, $\gamma(1) = b$} \}
	\end{align*}
	und
	\begin{enumerate}[(1)]
		\item
			$X \injto P(X)$ vermöge $a \mapsto 1_a, 1_a(t) = a$,
		\item
			$\_{\ }:P(X,a,b) \to P(X,b,a), \_\gamma(t) = \gamma(1-t)$,
		\item
			$*: P(X,a,b) \times P(X,b,c) \to P(X,a,c)$ mit
			\[
				(\gamma_1 * \gamma_2)(t) = \begin{cases}
					\gamma_1(2t) & 0 \le t \le \f 12 \\
					\gamma_2(2t-1) & \f 12 \le t \le 1
				\end{cases}.
			\]
	\end{enumerate}
\end{df}

\begin{df}
	Zwei Wege $\alpha, \beta \in P(X,a,b)$ heißen \emph{äquivalent}, geschrieben $a \sim b$, wenn es eine stetige Abbildung $H: [0,1] \times [0,1] \to X$ mit $H(0,t) = \alpha(t)$ und $H(1,t) = \beta(t)$ für alle $t \in [0,1]$ und $H(s,0) = a, H(s, 1) = b$ für alle $s \in [0,1]$.

	Wir setzen $\Pi(X,a,b) := P(X,a,b) / \sim$.
\end{df}



\begin{prop}
	Die Äquivalenz von Wegen „$\sim$“ ist eine Äquivalenzrelation.

	Aus $H: \alpha \sim \beta$ folgt $\_H: \_\alpha \sim \_\beta$, also
	\[
		\_{\ }: \Pi(X, a, b) \to \Pi(X, b, a), \_{[\alpha]} := [\_\alpha].
	\]

	Aus $H: \alpha \sim \alpha'$ und $K: \beta \sim \beta'$ folgt $H*K: \alpha * \beta \sim \alpha'*\beta'$, also
	\[
		*: \Pi(X, a, b) \times \Pi(X, b, c) \to \Pi(X, a, c), [\alpha] * [\beta] := [\alpha * \beta].
	\]
	\begin{proof}
		Reflexivität, Symmetrie, Transitivität klar.

		Inversion mit $\_H(s,t) = H(s,1-t)$.
		Verknüpfung ähnlich wie oben.
	\end{proof}
\end{prop}

\begin{st}
	Jeder topologische Raum $X$ definiert eine Kategorie $\Pi(X)$:
	\begin{enumerate}[(a)]
		\item
			Objekte sind die Punkte $a,b,c \in X$,
		\item
			Morphismen sind $[\alpha] \in \Pi(X,a,b)$,
		\item
			Die Verknüpfung $*$ ist die obige.
	\end{enumerate}
	Diese Verknüpfung erfüllt also
	\begin{enumerate}[(1)]
		\item
			Identität: Für $\alpha \in P(X,a,b)$ gilt $1_a * \alpha \sim \alpha \sim \alpha * 1_b$
			also $[1_a]*[\alpha] = [\alpha] = [\alpha] * [1_b]$,
		\item
			Assoziativität: Für $\alpha \in P(X,a,b)$ und $\beta \in P(X,b,c)$ und $\gamma \in P(X,c,d)$ gilt $(\alpha*\beta)*\gamma \sim \alpha*(\beta*\gamma)$, also $([\alpha]*[\beta])*[\gamma] = [\alpha] *([\beta]*[\gamma])$.
	\end{enumerate}
	Außerdem gilt noch
	\begin{enumerate}[(1),resume]
		\item
			Inverse: Für $\alpha \in P(X,a,b)$ gilt $\alpha * \_\alpha \sim 1_a$, $\_\alpha * \alpha \sim 1_b$, also $[\alpha]*[\_\alpha] = [1_b]$ und $[\_\alpha] * [\alpha] = [1_b]$.
	\end{enumerate}
	\begin{proof}
		\begin{enumerate}[(1)]
			\item
				Es gilt $H: 1_a * \alpha \sim \alpha$ mit
				\[
					H(s,t) = \begin{cases}
						a & 0 \le t \le \f{1-s}2 \\
						\alpha (\f{2t-1+s}{1+s}) & \f {1-s}2 \le t \le 1
					\end{cases}.
				\]
				Analog $\alpha * 1_b \sim \alpha$.
			\item
				Setze
				\[
					H(s,t) = \begin{cases}
						\alpha(\f{4t}{1+s}) & 0 \le t \le \f{1+s}4 \\
						\beta(4t - 1-s) & \f{1+s}4 \le t \le \f{2+s}4 \\
						\gamma(\f{4t-2-s}{2-s}) & \f{2+s}4 \le t \le 1
					\end{cases}.
				\]
			\item
				Setze
				\[
					H(s,t) = \begin{cases}
						\alpha(2t) & 0 \le t \le \f{1-s}{2} \\
						\alpha(1-s) & \f{1-s}2 \le t \le \f{1+s}2 \\
						\alpha(2-2t)  & \f{1+s}2 \le t \le 1
					\end{cases}.
				\]
		\end{enumerate}
	\end{proof}
\end{st}




