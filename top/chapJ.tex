\chapter{Die Fundamentalgruppe \texorpdfstring{$\pi_1(X, x_0)$}{π₁(X, x₀)}}



Bisher hatten wir $\pi_0: \Cat{Top} \to \Cat{Set}$, unser Ziel ist $\pi_1: \Cat{Top_*} \to \Cat{Grp}$.


\section{Das Fundamentalgruppoid $\pi(X)$}


\begin{df}
	Setze
	\begin{align*}
		P(X) &:= \scr C([0,1], X) = \{ \gamma : [0,1] \to X \text{ stetig} \}, \\
		P(X,a,b) &:= \scr C([0,1], X) = \{ \gamma : [0,1] \to X \text{ stetig mit $\gamma(0) = a$, $\gamma(1) = b$} \}
	\end{align*}
	und
	\begin{enumerate}[(1)]
		\item
			$X \injto P(X)$ vermöge $a \mapsto 1_a, 1_a(t) = a$,
		\item
			$\_{\ }:P(X,a,b) \to P(X,b,a), \_\gamma(t) = \gamma(1-t)$,
		\item
			$*: P(X,a,b) \times P(X,b,c) \to P(X,a,c)$ mit
			\[
				(\gamma_1 * \gamma_2)(t) = \begin{cases}
					\gamma_1(2t) & 0 \le t \le \f 12 \\
					\gamma_2(2t-1) & \f 12 \le t \le 1
				\end{cases}.
			\]
	\end{enumerate}
\end{df}

\begin{df}
	Zwei Wege $\alpha, \beta \in P(X,a,b)$ heißen \emph{äquivalent}, geschrieben $a \sim b$, wenn es eine stetige Abbildung $H: [0,1] \times [0,1] \to X$ mit $H(0,t) = \alpha(t)$ und $H(1,t) = \beta(t)$ für alle $t \in [0,1]$ und $H(s,0) = a, H(s, 1) = b$ für alle $s \in [0,1]$.

	Wir setzen $\Pi(X,a,b) := P(X,a,b) / \sim$.
\end{df}



\begin{prop}
	Die Äquivalenz von Wegen „$\sim$“ ist eine Äquivalenzrelation.

	Aus $H: \alpha \sim \beta$ folgt $\_H: \_\alpha \sim \_\beta$, also
	\[
		\_{\ }: \Pi(X, a, b) \to \Pi(X, b, a), \_{[\alpha]} := [\_\alpha].
	\]

	Aus $H: \alpha \sim \alpha'$ und $K: \beta \sim \beta'$ folgt $H*K: \alpha * \beta \sim \alpha'*\beta'$, also
	\[
		*: \Pi(X, a, b) \times \Pi(X, b, c) \to \Pi(X, a, c), [\alpha] * [\beta] := [\alpha * \beta].
	\]
	\begin{proof}
		Reflexivität, Symmetrie, Transitivität klar.

		Inversion mit $\_H(s,t) = H(s,1-t)$.
		Verknüpfung ähnlich wie oben.
	\end{proof}
\end{prop}

\begin{st}
	Jeder topologische Raum $X$ definiert eine Kategorie $\Pi(X)$:
	\begin{enumerate}[(a)]
		\item
			Objekte sind die Punkte $a,b,c \in X$,
		\item
			Morphismen sind $[\alpha] \in \Pi(X,a,b)$,
		\item
			Die Verknüpfung $*$ ist die obige.
	\end{enumerate}
	Diese Verknüpfung erfüllt also
	\begin{enumerate}[(1)]
		\item
			Identität: Für $\alpha \in P(X,a,b)$ gilt $1_a * \alpha \sim \alpha \sim \alpha * 1_b$
			also $[1_a]*[\alpha] = [\alpha] = [\alpha] * [1_b]$,
		\item
			Assoziativität: Für $\alpha \in P(X,a,b)$ und $\beta \in P(X,b,c)$ und $\gamma \in P(X,c,d)$ gilt $(\alpha*\beta)*\gamma \sim \alpha*(\beta*\gamma)$, also $([\alpha]*[\beta])*[\gamma] = [\alpha] *([\beta]*[\gamma])$.
	\end{enumerate}
	Außerdem gilt noch
	\begin{enumerate}[(1),resume]
		\item
			Inverse: Für $\alpha \in P(X,a,b)$ gilt $\alpha * \_\alpha \sim 1_a$, $\_\alpha * \alpha \sim 1_b$, also $[\alpha]*[\_\alpha] = [1_b]$ und $[\_\alpha] * [\alpha] = [1_b]$.
	\end{enumerate}
	\begin{proof}
		\begin{enumerate}[(1)]
			\item
				Es gilt $H: 1_a * \alpha \sim \alpha$ mit
				\[
					H(s,t) = \begin{cases}
						a & 0 \le t \le \f{1-s}2 \\
						\alpha (\f{2t-1+s}{1+s}) & \f {1-s}2 \le t \le 1
					\end{cases}.
				\]
				Analog $\alpha * 1_b \sim \alpha$.
			\item
				Setze
				\[
					H(s,t) = \begin{cases}
						\alpha(\f{4t}{1+s}) & 0 \le t \le \f{1+s}4 \\
						\beta(4t - 1-s) & \f{1+s}4 \le t \le \f{2+s}4 \\
						\gamma(\f{4t-2-s}{2-s}) & \f{2+s}4 \le t \le 1
					\end{cases}.
				\]
			\item
				Setze
				\[
					H(s,t) = \begin{cases}
						\alpha(2t) & 0 \le t \le \f{1-s}{2} \\
						\alpha(1-s) & \f{1-s}2 \le t \le \f{1+s}2 \\
						\alpha(2-2t)  & \f{1+s}2 \le t \le 1
					\end{cases}.
				\]
		\end{enumerate}
	\end{proof}
\end{st}

\coursetimestamp{27}{01}{2014}


\section{Die Fundamentalgruppe eines punktierten Raumes $(X,x_0)$}


\begin{df}
	Wir setzen
	\[
		\pi_1(X, x_0)
		:= \Pi(X, x_0, x_0)
	\]
	mit $[\alpha] \cdot [\beta] := [\alpha * \beta]$.

	$(\pi_1(X, x_0), \cdot)$ bildet eine Gruppe.
\end{df}

\begin{ex}
	Betrachte
	\[
		\deg: \big(\pi_1(\C \setminus \{0\}, 1), \cdot\big) \homeomorphicto (\Z, +).
	\]
	Dies ist ein Gruppenisomorphismus.
\end{ex}

\paragraph{Verschieben des Fußpunktes}

\begin{prop}
	Jeder Weg $\gamma \in P(X, x_0, x_1)$ induziert einen Gruppenisomorphismus
	\begin{align*}
		h_\gamma: \pi_1(X,x_0) &\isomorphicto \pi_1(X, x_1),& [\alpha] &\mapsto [\_\gamma * \alpha * \gamma], \\
		h_{\_\gamma}: \pi_1(X,x_1) &\isomorphicto \pi_0(X, x_1),& [\beta] &\mapsto [\gamma * \beta * \_\gamma].
	\end{align*}
	Ist $\gamma' \in P(X, x_0, x_1)$ ein weiterer Weg, so gilt
	\[
		h_{\gamma'} = c^{-1} \cdot h_\gamma \cdot c
	\]
	mit $c = [\_\gamma * \gamma'] \in \pi_1(X, x_1)$.
	\begin{proof}
		Direkt nachzurechnen.
	\end{proof}
\end{prop}

\begin{df}
	$X$ heißt \emph{einfach zusammenhängend}, wenn $\pi_0(X) = \{X\}$ und $\pi_1(X, x_0) = \{1\}$.
\end{df}

\begin{ex}
	$\R^n$ ist einfach-zusammenhängend, ebenso $X \subset \R^n$ sternförmig.
\end{ex}

\begin{ex}
	$\S^n$ ist wegzusammenhängend für $n \ge 1$ und einfach zusammenhängend für $n \ge 2$.
	\begin{proof}
		Funktioniert mit stereographischer Projektion und simplizialer Approximation.
	\end{proof}
\end{ex}

\paragraph{Funktorialität}

\begin{lem}
	Jede stetige Abbildung $f: (X, x_0) \to (Y, y_0)$ induziert einen Gruppenhomomorphismus $f_\# := \pi_1(f) : \pi_1(X, x_0) \to \pi_1(Y, y_0)$ mit $f_\#([\alpha]) = [f \circ \alpha]$.
	\begin{proof}
		Nachrechnen.
	\end{proof}
\end{lem}

\begin{st}
	Wir erhalten einen (kovarianten Funktor) $\pi_1: \Cat{Top_*} \to \Cat{Grp}$.
	\[
		\begin{tikzcd}[column sep=small]
			~ & (Y,y_0) \arrow{dr}{g} & \\
			(X,x_0) \arrow{ur}{f} \arrow{rr}[swap]{h=g \circ f} & & (Z, z_0)
		\end{tikzcd}
		\xrightarrow{\pi}
		\begin{tikzcd}[column sep=tiny]
			~ & \pi_1(Y, y_0) \arrow{dr}{\pi_1(g)} & \\
			\pi_1(X,x_0) \arrow{ur}{\pi_1(f)} \arrow{rr}[swap]{\pi_1(h)} & & \pi_1(Z,z_0)
		\end{tikzcd}
	\]
	\begin{proof}
		Nachrechnen.
	\end{proof}
\end{st}


\begin{kor}
	Angenommen $f: (X, x_0) \to (Y, y_0)$ ist ein Homöomorphismus, d.h. stetig und es existiert $g: (Y, y_0) \to (X, x_0)$ stetig mit $g \circ f = \id_{(X,x_0)}$ und $f\circ g = \id_{(Y,y_0)}$.
	Dann ist $f_\# : \pi_1(X, x_0) \to \pi_1(Y,y_0)$ ein Isomorphismus, denn
	\[
		g_\# \circ f_\# = (g \circ f)_\# = (\id_{(X, x_0)})_\# = \id_{\pi_1(X, x_0)}
	\]
	und $f_\# \circ g_\# = \id_{\pi_1(Y,y_0)}$.
\end{kor}

\begin{st}
	Für $f, g: (X, x_0) \to (Y, y_0)$ impliziert jede Homotompie $H: f \homotopic g$ relativ zu $x_0$
	\[
		\pi_1(f) = \pi(g) : \pi_1(X, x_0) \to \pi_1(Y, y_0).
	\]
	\begin{proof}
		Nachrechnen.
	\end{proof}
\end{st}

\begin{kor}
	Sind $f: (X, x_0) \to (Y, y_0)$ und $g: (Y, y_0) \to (X, x_0)$ mit $g \circ f \homotopic \id_{(X, x_0)}$ relativ zu $x_0$ und $f \circ g \homotopic \id_{(Y, y_0)}$ relativ zu $y_0$, dann erfüllt $f_\#: \pi_1(X, x_0) \to \pi_1(Y, y_0)$ und $g_\# : \pi_1(Y, y_0) \to \pi_1(X, x_0)$
	\begin{align*}
		f_\# \circ g_\# &= \id_{\pi_1(Y, y_0)}, & g_\# \circ f_\# = \id_{\pi_1(X, x_0)}.
	\end{align*}
\end{kor}


\section{Polygonale Fundamentalgruppe}


\begin{df}
	Sei $V$ ein $\R$-Vektorraum und $x_0 \in X \subset V$.
	\[
		\pi_1^{\mathrm{pl}} := \{ \text{Polygonzüge in $(X, x_0)$} \} / \text{polygonale Homotopie}.
	\]
\end{df}

\begin{st}
	Für $X \subset \R^d$ offen und $x_0 \in x$ haben wir
	\begin{align*}
		\phi: \pi_1^{\mathrm{pl}}(X, x_0) &\to \pi_1(X, x_0), & \phi([w]) &= [|w|],
	\end{align*}
	wobei mit $|w|$ die topologische Realisierung gemeint ist.
	\begin{proof}
		Funktioniert mit Polygonaler Approximation von Wegen und Homotopien.
	\end{proof}
\end{st}

\begin{st}
	Wir erhalten so $\deg: \pi_1(\C^*, 1) \isomorphicto \Z$.
\end{st}

\begin{st}
	Für $X = \C \setminus \{0, -1, -2, \dotsc, 1 - n\}$ ist $\pi_1(X, 1)$ frei vom Rand $n$, d.h. es existiert $\psi: (X, 1) \isomorphicto \<s_1, \dotsc, n_n |  - \>$.
	\begin{note}
		Für $n = 0, X = \C$ ist $\pi_1(X, 1) = \{1\}$.
		Für $n = 1, X = \C^*$ ist $\pi_1(X, 1) \isomorphic (\Z, +) = (\<s_1| - \>, \cdot)$.
	\end{note}
	\begin{proof}
		Ähnliche Idee wie bei der Umlaufzahl: zähle Übergänge auf der negativen reellen Achse zwischen den Segmenten im jeweiligen Erzeuger $s_1, \dotsc, s_n$.
		% fixme: drawing

		Für $[u,v] \subset X$, setze
		\[
			\eta(u,v) = \begin{cases}
				1 & [u,v] \cap \R_{<0} = \emptyset \\
				s_k^\eps & [u,v] \cap S_k \neq \emptyset
			\end{cases}
		\]
		wobei $\eps := \f 12 [\sign \Im u - \sign \Im v]$.
		Setze jetzt
		\[
			\eta(v_0 v_1 v_2 \dotsc v_{n-1} v_n) := \eta(v_0 v_1) \cdot \eta(v_1 v_2) \cdot \dotsb \cdot \eta(v_{n-1} v_n)
		\]
		und $\psi: \pi_1^{\mathrm{pl}}(X, 1) \to \<s_1, \dotsc, s_n | - \>$ mit $\psi([w]) = \eta(w)$.

		Dies ist ein Gruppenhomomorphismus (surjektiv, injektiv: nachrechnen und Induktion).
	\end{proof}
\end{st}

\begin{ex}
	Sei $X = \C \setminus \{0, 1\}$.

	Es gilt $s_1s_2^{-1} \not\homotopic s_2^{-1} s_1$.
\end{ex}


\section{Simpliziale Fundamentalgruppe}


\begin{df}
	Sei $K$ ein Simplizialkomplex mit Eckenmenge $\Omega$ und $x_0 \in \Omega$.
	Sei $|K| \subset \R^{(\Omega)}$ die topologische Realisierung und
	\[
		\pi_1(K, x_0) := \dfrac{ \{ \text{Kantenzüge in $(|K|, x_0)$ mit Eckenmenge $\Omega$} \} }
		{ \text{polygonaler Homotopie} }.
	\]
\end{df}

\begin{st}
	Wir haben $\phi: \pi_1(K, x_0) \isomorphicto \pi_1(|K|, x_0)$ mit $\phi([w]) = [|w|]$.
	\begin{proof}
		Funktioniert mit simplizialer Approximation.
	\end{proof}
\end{st}

\begin{ex}
	Betrachte $\S^1 \homeomorphic |K|$ mit
	\[
		K = \Big\{ \{a\}, \{b\}, \{c\}, \{a, b\}, \{b, c\}, \{c, a\} \Big\}.
	\]
	Wege auf $\S^1$ können wir also als Wörter in $\Omega$ darstellen.
\end{ex}

\begin{st}
	Für jeden Baum $T$ gilt $\pi_1(T, x_0) = \{1\}$.
	\begin{proof}
		$|T| \homotopic *$.
	\end{proof}
\end{st}

\begin{st}
	Sei $K$ ein zusammenhängender Graph, $T \subset K$ ein Spannbaum von $K$ und $x_0$ ein Ecke.

	Dann ist $\pi_1(|K|, x_0)$ frei über $|K \setminus T|$ Erzeugern, genauer existiert ein Isomorphismus
	\[
		\psi: \pi_1(|K|, x_0) \to F := \< S | R \>
	\]
	mit Erzeugern
	\[
		S = \{ s_{ab} | \{a,b\} \in K \setminus T \}
	\]
	und Relationen
	\[
		R = \big\{ s_{ab} s_{ba} | \{a,b\} \in K \setminus T \} \big\}.
	\]
	\begin{proof}
		Ähnlich wie für $\C \setminus \{0, -1, \dotsc, 1-n \}$.
	\end{proof}
\end{st}

\coursetimestamp{28}{01}{2014}

\subsection{Anwendung auf berandete Flächen}

Für einen passenden Graphen $K$ ist $|K|$ zu $F_{g,r}^+$ homotopie-äquivalent.
Es gilt
\[
	\<a_1, b_1, \dotsc, a_g, b_g, d_2, \dotsc, d_r | - \>
	\homotopic
	\pi_1(|K|,x_0)
	\homotopic
	\pi_1(F_{g,r}^+, x_0),
\]
wobei $a, b, d$ jeweils Kanten im Henkel sind (durch Wegnehmen entsteht ein Baum).

Genauso
\[
	\pi_1(F_{g,r}^-)
	\homotopic
	\<c_0, \dotsc, c_g, d_2, \dotsc, d_r | - \>.
\]

\begin{st}
	Sei $K$ ein zusammenhängender Simplizialkomplex, $T \subset K$ ein Spannbaum und $x_0$ eine Ecke.
	Dann haben wir $\psi: \pi_1(K, x_0) \isomorphicto G = \<S | R \>$ mit Erzeugern $S = \{ s_{ab} | \{a,b\} \in K\}$ und Relationen
	\[
		R = \{ s_{ab} | \{a,b\} \in T \}
		\cup \Big\{ s_{ab} s_{ba}, a_{ab}s_{bc}s_{ca} | \{a,b,c\} \in K \Big\}.
	\]
	\begin{proof}
		Wir wohldefinieren $\psi: \pi_1(K, x_0) \to G$ durch
		\[
			\pi([v_0v_1 \dotsc v_{n-1} v_n]) = s_{v_0v_1} s_{v_1v_2} \dotsb s_{v_{n-1}}s_{v_n}
		\]
		und $\phi: G \to \pi_1(K, x_0)$ durch
		\[
			\phi(s_{ab}) = \underbrace{x_0 \dotsb a}_{\text{in $T$}} * ab * \underbrace{b \dotsb x_0}_{\text{in $T$}}.
		\]
		Wir prüfen $\phi \circ \psi = \id$ und $\psi \circ \phi = \id$.
	\end{proof}
\end{st}

\begin{st}
	Es existieren Gruppenisomorphismen
	\begin{align*}
		\pi_1(F_g^+, x_0)
		&\isomorphicto \< a_1 b_1, \dotsc, a_g b_g | a_1b_1a_1^{-1}b_1^{-1} \dotsc a_gb_g a_g^{-1}b_g^{-1} \>, \\
		\pi_1(F_g^-, x_0)
		&\isomorphicto \< c_0, c_1, \dotsc, c_g | c_0c_0 c_1c_1 \dotsb c_g c_g \>.
	\end{align*}
	\begin{proof}
		Betrachte $F_{g}^{\pm} = \D^2 / \<w\>$ mit $w = a_1b_1a_1^{-1}b_1^{-1} \dotsb a_gb_g a_g^{-1}b_g^{-1}$, bzw. $w = c_0 c_0 c_1 c_1 \dotsb c_g c_g$.
		Wir triangulieren $\S^1 \subset \D^2$, sowie $\S^1 / \<w\> \subset \D^2 \<w\>$.
		Sei $L \subset K$ dieser Komplex.
		$\pi_1(|L|, x_0) \homotopic \<a_1, b_1, \dotsc, a_g, b_g | - \>$.
		Wähle ein Dreieck $\triangle \in K$, setze $K' := K \setminus \{ \triangle \}$.
		Wir haben eine Homotopie-Äquivalenz zwischen $|L|$ und $|K'|$.

		Der Kantenweg $w' = x_0 \dotsb a * abc * c \dotsb x_0$ in $K'$ entspricht $w$ in $L$, genauer $r_\#([w']) = [w]$.
		\[
			\<a_1, b_1, \dotsc, a_g, b_g | - \>
			\isomorphicto \pi_1(|L|, x_0)
			\isomorphicto \pi_1(|K'|, x_0)
			\xrightarrow{\text{inc}_\#} \pi_1(|K|, x_0),
		\]
		wobei $\text{inc}_\#$ Quotient mit $w$ als einzige Relation.
	\end{proof}
\end{st}

\begin{ex}
	\begin{itemize}
		\item
			$F_0^- = \R\P^2$ und $\pi_1(\R\P^2, x_0) \homotopic \Z / 2$.
		\item
			$F_1^+ = \S^1 \times \S^1$ (Torus).
			$\pi_1(F_1^+, x_0) = \<a,b | aba^{-1}b^{-1}\>$, d.h. $aba^{-1}b^{-1} = 1$, also $ab = ba$.
			Somit ist $\pi_1(F_1^+, x_0) \isomorphic \Z \times \Z$.
	\end{itemize}
\end{ex}

\subsection{Abelsch-Machung}

Es gilt
\begin{align*}
	\pi_1(F_g^+)_{ab}
	&= \left\< a_1, b_1, \dotsc, a_g, b_g \bigg| \begin{aligned}
		[a_1,b_1] = \dotsb = [a_g, b_g] &= 1, \\
		[a_i, a_j] = [a_i, b_j] = [b_i, b_j] &= 1
	\end{aligned}
	\right\> \\
	&= \Big\< a_1, b_1, \dotsc, a_g, b_g \Big| [a_i, a_j] = [a_i, b_j] = [b_i, b_j] = 1 \Big\> \\
	&\isomorphic \Z^{2g}.
\end{align*}
und
\begin{align*}
	\pi_1(F_g^-)_{ab}
	&= \<c_0, c_1, \dotsc, c_g | c_0^2 c_1^2 \dotsc c_g^2 = 1, [c_i, c_j] = 1 \> \\
	&\isomorphic / \< 2c_0 + 2c_1 + \dotsb + 2c_g = 0\> \\
	&\isomorphic \Z / 2 \times \Z^g.
\end{align*}

\begin{st}
	Die Gruppen $\pi_1(F_g^{\pm}, x_0)$ sind untereinander nicht isomorph, d.h. aus $\pi_1(F_g^\eps) \isomorphic \pi_1(F_h^\delta)$ folgt $(g,\eps) = (h,\delta)$.
\end{st}

\begin{kor}
	Aus $F_g^\eps \homeomorphic F_h^\delta$ folgt $(g,\eps) = (h,\delta)$.
	\begin{note}
		Hieraus folgt insbesondere, dass $\chi(F_g^\eps)$ eine topologische Invariante ist.
	\end{note}
\end{kor}

\begin{kor}
	Aus $F_g^\eps \homotopic F_h^\delta$ folgt $(g,\eps) = (h,\delta)$.
\end{kor}

\subsection{Projektive Räume}

Wir kennen $\R\P^n = \S^n / \{\pm 1\}$ und $\pi_1(\S^1, 1) \isomorphic \Z / 2$.

\begin{align*}
	\R\P^1 &= \S^1 / \{\pm 1\} \homeomorphic \S^1, & \pi_1(\S^1, 1) & \isomorphic \Z; \\
	\R\P^2 &= \S^2 / \{\pm 1\}, & \pi_1(\R\P^2, x_0) &\isomorphic \Z / 2; \\
	\R\P^3&, & \pi_1(\R\P^3, x_0) &\isomorphic \Z / 2; \\
	\vdots\;& & \vdots\quad & \qquad \vdots
\end{align*}
Für orthogonale Gruppen
\begin{align*}
	\SO_2 \R &\homeomorphic \S^1, & \pi_1(\SO_2 \R, 1) &\isomorphic \Z; \\
	\SO_3 \R &\homeomorphic \R\P^3, & \pi_1(\SO_3 \R, 1) &\isomorphic \Z / 2.
\end{align*}








