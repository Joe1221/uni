%Für $f \in \C[x]$ ist $f = c(x-z_1)\dotsb(x-z_n)$ bis auf $c\in \C^*$ durch die Nullstellen $z_1, \dotsc, z_n$ eindeutig bestimmt.
%
%Für $f_i \in k[x_1, \dotsc, x_n]$ betrachten wir die (affine) algebraische Varietät
%\begin{math}
%    Z(f_1, \dotsc, f_n) = \Set{x \in k^n & f_i(x) = 0},
%\end{math}
%wobei $f_i: k^n \to k$.
%
%Wir nutzen dazu die Methoden der kommutativen Algebra.
%\begin{ex}
%    \begin{enumerate}[1.]
%        \item
%            Teilbarkeitsrelation durch Ideale
%        \item
%            Für $f \in k[x_1, \dotsc, x_n]$ betrachte $T := Z(f) = \Set{x \in k^n & f(x) = 0}$.
%            Setze
%            \begin{math}
%                I(T) = \Set{g \in k[x_1, \dotsc, x_n] & \forall x \in T: g(x) = 0}.
%            \end{math}
%            Für $g,h \in I(T)$ ist auch $g + h \in I(T)$.
%            Für $p \in k[x_1, \dotsc, x_n]$ und $g \in I(T)$ ist $pg \in I(T)$.
%            $I(T)$ bildet ein Ideal.
%    \end{enumerate}
%\end{ex}
%

\chapter{Ringe und Ideale}

\Timestamp{2015-10-22}

\begin{df}
    $\frak a \in A$ ist ein \emphdef{Ideal}, genau dann, wenn
    \begin{enumerate}[i)]
        \item
            $a, b \in \frak a \implies a + b \in \frak a$,
        \item
            $x \in A, x \in \frak a \implies x a \in \frak a$.
    \end{enumerate}
    Insbesondere definiert die Multiplikation auf $A$ eine Abbildung $\argdot: A \times \frak a \to \frak a$.
    \begin{note}
        Ist $B \subset A$ Unterring, so haben wir $\argdot: B \times B \to B$, aber nicht $\argdot: A \times B \to B$.
    \end{note}
\end{df}

\begin{ex}
    \begin{itemize}
        \item
            $(0) = \Set{0} \subset A$.
        \item
            $\frak a = A$,
        \item
            Für $a \in A$ ist
            \begin{math}
                (a) = \Set{x a & x \in A},
            \end{math}
            das von $a$ erzeugte Ideal.
    \end{itemize}
\end{ex}

\begin{df}
    $\frak m \neq A$ heißt \emphdef{maximal}, wenn
    \begin{math}
        \forall \frak a \subset A: \frak m \subset \frak a \implies \frak m = \frak a \lor \frak a = A.
    \end{math}
    $\frak p \subsetneq A$ heißt \emphdef{prim}, wenn
    \begin{math}
        ab \in \frak p \implies a \in \frak p \lor b \in \frak p.
    \end{math}
\end{df}

\begin{ex}
    Sei $k$ ein Körper, $A := k[x_1, \dotsc, x_n]$.
    \begin{enumerate}[i)]
        \item
            Für $A = \Z$ sind die Primideale genau
            \begin{math}
                \Set{(p) & p \text{ prim}}
            \end{math}
        \item
            Ist $f \in A$ irreduzibel, dann ist $(f)$ ein Primideal.
        \item
            Jedes maximale Ideal $\frak p$ ist prim.
            Ist $A$ ein Hauptidealring (z.B. $\Z$, oder $k[x]$), dann gilt auch die Umkehrung falls $\frak p \neq 0$.
    \end{enumerate}
\end{ex}

\begin{st}
    \begin{enumerate}[i)]
        \item
            $\frak p \subset A$ ist prim genau dann, wenn $A / \frak p$ nullteilerfrei
        \item
            $\frak m \subset A$ ist maximal genau dann, wenn $A / \frak m$ ein Körper ist.
    \end{enumerate}
\end{st}

\begin{nt}
    Das Zornsche Lemma liefert: Maximale Ideale existieren in jedem Ring.
    
    Genauso: Ist $f \in A$ eine \emphdef{multiplikative Menge} (d.h. $1 \in S$, $a,b \in S \implies ab \in S$) so gilt:
    Aus $\frak a \in A \ S$ Ideal folgt $\exists \frak p$ Ideal mit $\frak a \subset \frak p$ und $\frak p$ ist prim).
\end{nt}

\begin{nt}
    \begin{math}
        k[x_1, \dotsc, x_n] = \Set{\sum_{\text{endl}} b_{i_1, \dotsc, i_n} x_i^{i_1} \dotsb x_n^{i_n}, n \in \N, b_{i_1, \dotsc, i_n} \in k},
    \end{math}
    Polynome in $n$ Unbekannten vs. polynomielle Funktion in $n$ Unbekannten.
    Eine Unterscheidung ist wichtig, betrachte z.B. $x^2 + x \in \Z_2[x]$.
    Falls $k$ unendlich viele Elemente hat, so sind beide Begriffe gleichwertig, insbesondere wenn $k$ algebraisch abgeschlossen ist.
\end{nt}

\begin{ex}
    Für $a \in k^n$ betrachte die Evaluation in $a$ $\ev_a: k[x_1, \dotsc, x_n] \ni f \mapsto f(a) \in k$.

    Für einen Ring-Morphismus $f: A \to B$ gilt $\im f \isomorphic A / \ker f$.
    $\ker f \subset A$ ist stets ein Ideal, $\im f \subset B$ ist nur ein Unterring von $B$.

    Hier $\im \ev_a = k = k[x_1, \dotsc, x_n] / \ker \ev_a$.
    $\ker \ev_a = \frak m$ ist maximal.

    Wir behaupten, dass $\frak m = (x_1 - a_1, \dotsc, x_n - a_n) \subset k[x_1, \dotsc, x_n]$.
    \begin{proof}
        \begin{segnb}{$\supset$}
            $f_i(x_1, \dotsc, x_n) = x_i - a_i$, also $f_i(a_1, \dotsc, a_n) = f_i(a) = (x_i - a_i)(a_i) = 0$.
        \end{segnb}
        \begin{segnb}{$\subset$}
            Sei $(a_1, \dotsc, a_n) = (0, \dotsc, 0)$ (sonst betrachte $y_i = x_i - a_i$).
            Sei $f \in \ker \ev_a$, $f = \sum a_{i_1, \dotsc i_n} x_1^{i_1} \dotsb x_n^{i_n} = x_1 g(x_1, \dotsc, x_n) + f_2(x_2, \dotsc, x_n)$ dann ist
            \begin{math}
                0 = f(a) = \underbrace{a_1}_{=0} g_1(a_1, \dotsc, a_n) + f_2(a_2, \dotsc, a_n),
            \end{math}
            also $f_2(a_2, \dotsc, a_n) = 0$.
            Induktiv folgt
            \begin{math}
                f(x) = x_1 g_1 + \dotsb + x_n g_n \in (x_1, \dotsc, x_n),
            \end{math}
            also $\ker \ev_0 \subset (x_1, \dotsc, x_n)$.
        \end{segnb}
    \end{proof}

    Demnach ist
    \begin{math}
        I(\Set{a}) &:= \Set{ f \in k[x_1, \dotsc, x_n] & f(a) = 0 } \\
        &= \ker \ev_a = (x_1 - a_1, \dotsc, x_n - a_n)
    \end{math}
    maximales Ideal.

    \begin{note}
        Hilberts Nullstellensatz besagt: Für algebraisch abgeschlossenes $k$ gilt: jedes maximale Ideal in $k[x_1, \dotsc, x_n]$ ist von dieser Form.
        Es ergibt sich eine geometrische Korrespondenz: Punkte in $k^n$ entsprechen maximalen Idealen in $k[x_1, \dotsc, x_n]$.
    \end{note}
\end{ex}

\Timestamp{2015-10-13}

Sei $A = k[x_1, \dotsc, x_n]$, $X \subset k^n$.
\begin{math}
    I(X) = \Set{f \in k[x_1, \dotsc, x_n] & \forall x \in X : f(x) = 0}
\end{math}
ist ein Ideal.

Später behandeln wir als Analogon zu Körpererweiterungen auch Ringerweiterungen.

\begin{df}
    $k \subset K$ heißt \emphdef[Körpererweiterung!endlich]{endliche} Körpererweiterung, wennn $\dim_k K < \infty$ ($K$ als $k$-Vektorraum betrachtet).

    $k \subset K$ heißt \emphdef[Körpererweiterung!algebraisch]{algebraische} Körpererweiterung, wenn $\forall \alpha \in K \exists f \in k[x]: f(\alpha) = 0$.
\end{df}

\begin{ex}
    \begin{enumerate}[i)]
        \item
            $\R \subset \C$ ist endlich und algebraisch,
        \item
            $k \subset \_k$ ist algebraisch
    \end{enumerate}
\end{ex}

\begin{nt}
    Eine endliche Körpererweiterung ist eine algebraische Körpererweiterung.
    \begin{proof}
        Sei $\alpha \in K \setminus k$, dann existiert ein minimales $n \in \N$ mit
        \begin{math}
            \Set{1, \alpha, \dotsc, \alpha^{n-1}, \alpha^n} \subset K
        \end{math}
        linear abhängig über $k$.
        Es existiert also eine Darstellung
        \begin{math}
            \alpha^n =  \sum_{i=0}^{n-1} x_i \lambda^i
        \end{math}
        und dammit $f(\alpha) = 0$ mit $f := x^n + \sum_{i=0}^{n-1} \lambda_i x^i$.
    \end{proof}
\end{nt}

\begin{df}
    Sei $k[\alpha]$ der Ring, der von $k$ und $\alpha$ in $k$ erzeugt wird, d.h.
    \begin{math}
        k[\alpha] = \Set{ \sum_{i=0}^m \mu_i \alpha^i & m \in \N, \mu_i \in k } \subset K
    \end{math}
    In $k[\alpha]$ gilt $f(\alpha) = \alpha^n + \sum \lambda_i \alpha^i = 0$.
\end{df}

\begin{nt}
    $k[\alpha]$ ist ein Körper.
    \begin{proof}
        Betrachte $\ev_a: k[x] \to k[\alpha]$, $g \mapsto g(\alpha)$.
        $\ev_\alpha$ ist surjektiv, also ist $k[x] / \ker \ev_\alpha \isomorphic \im \ev_\alpha = k[\alpha]$.
        $\ker \ev_\alpha$ ist eine Ideal und $k[x]$ ist euklidischer Ring, insbesondere ist $k[x]$ ein Hauptidealring.
        Es folgt $\ker \ev_\alpha = (p)$ und $\ker \ev_a$ Primideal.
        In Hauptidealen sind Primideale ungleich 0 maximale Ideale.
        $\ker \ev_\alpha$ ist also maximales Ideal  und $k[x] / \ker \ev_\alpha \isomorphic k[\alpha] = k(\alpha)$ ein Körper.
        ($k(x) = \Quot k[x]$).
    \end{proof}
    \begin{note}
        \begin{enumerate}[i)]
            \item
                Sei $(p)$ ein Primideal, also $p$ ein irreduzibles Polynom, o.E. normiert, d.h.
                \begin{math}
                    p(x) = x^d + \sum_{i=0}^{d-1} \lambda_i x^i
                \end{math}
                Minimalpolyonm von $\alpha$.
                Insbesondere $p(\alpha) = \alpha^d + \sum \lambda_i \alpha^i = 0$ und $d = \dim_k k(\alpha)$.
            \item
                Die Umkehrung ist falsch!
                Betrachte $\Q \subset \Q(\sqrt[n]{3}) \subset \_\Q$.
                Minimalpolynom $p = x^n - 3 \in \Q[x]$ ist irreduzibel nach Eisenstein.
        \end{enumerate}
    \end{note}
\end{nt}

\begin{ex}[Übung]
    Betrachte $k \subset K$ algebraische Körpererweiterung, $a \in K^n$.
    Zeigen Sie, dass $\ker \ev_aa = (x_1 - a_1, \dotsc, x_n - a_n) \cap k[x_1, \dotsc, x_n]$ (Schnitt in $K[x_1, \dotsc, x_n]$.
\end{ex}


\section{Lokale Ringe}


\begin{df}
    $A$ heißt \emphdef{lokal}, wenn er genau ein maximales Ideal $\frk m$ enthält.
\end{df}

\begin{ex}
    Körper $(k, (0))$
\end{ex}

\begin{st}
    Für einen Ring $A$ sind äquivalent
    \begin{enumerate}[i)]
        \item
            $A$ ist lokaler Ring mit maximalem Ideal $\frk m$.
        \item
            Alle Nicht-Einheiten bilden ein Ideal $\frk m$.
        \item
            Es existiert ein Ideal $\frk m \neq A$, sodass aus $x \in A \setminus \frk m$ folgt, dass $x$ eine Einheit ist.
        \item
            Es existiert ein maximales Ideal $\frk m \subset A$ mit $1 + \frk m = \Set{1 + x & x \in \frk m} \subset A^*$.
    \end{enumerate}
    \begin{proof}
        \begin{seg}{\ProofImplication)[1][2]}
            Sei $(A, \frk m)$ lokal, dann ist $A = A^* \sqcup \frk m$ disjunkte Zerlegung.
            $\frk m$ ist also die Menge der Nichteinheiten und ein Ideal.
        \end{seg}
        \begin{seg}{\ProofImplication)[2][1]}
            Sei $\frk a$ ein maximales Ideal, dann ist $\frk a \subset A \setminus A^* = \frk m$ (sonst $\frk a = A$).
            $\frk m$ ist maximal.
            Wäre $\tilde{\frk m}$ ein weiteres maximales Ideal, dann $\tilde{\frk m} \subset A \setminus A^* = \frk m$, d.h. $\tilde{\frk m} \subset \frk m$, also $\tilde{\frk m} = \frk m$ nach Maximalität.
            Also ist $(A, \frk m)$ lokal.
        \end{seg}
        \begin{seg}{\ProofImplication)[2][3], \ProofImplication)[3][2]}
            Reformulierung.
        \end{seg}
        \begin{seg}{\ProofImplication)[1][4]}
            Sei $(A, \frk m)$ lokal, d.h. $A = A^* \sqcup \frk m$.
            Also $1 + \frk m \subset A^*$, denn $(1 + \frk m) \cap \frk m = \emptyset$ (sonst $1 \in \frk m$).
        \end{seg}
        \begin{seg}{\ProofImplication)[4][3]}
            Sei $x \in A \setminus \frk m$ wie in (iv).
            $\frk m$ ist maximal, also ist das von $\frk m$ und $x$ erzeugte Ideal $(\frk m, x) = A$.
            Folglich existiert $y \in A$ und $m \in \frk m$ mit $A \ni 1 = m + xy$ und $yx = 1 + (-m)$ mit $-m \in \frk m$.
            Gemäß Voraussetzung ist $yx \in A^*$, also $x \in A^*$.
        \end{seg}
    \end{proof}
\end{st}

\begin{ex}
    \begin{enumerate}[i)]
        \item
            Angenommen, wir fragen nach Teilbarkeit durch $p \in \Z$ prim.
            $5 \divs n$ in $\Z$ genau dann wenn $5 \divs n$ in $\Z[\frac 12, \frac 13, \frac 17] = A$.
            $A$ ist ein Ring in $\Q$, der durch $\Z$ und $\frac{1}{2}, \frac{1}{3}, \frac{1}{7}$ erzeugt wird.

            Besser
            \begin{math}
                \Z_{(5)} = \Set{\frac{p}{q} \in \Q & 5 \ndivs q} \subset \Q.
            \end{math}
            Dann ist $5 \ndivs n$ in $\Z$ genau dann, wennn $\frac{n}{m} \in \Z_{(5)}$ ist eine Einheit.
        \item
            Die Nichteinheiten in $\Z_{(5)}$ sind $5\Z_{(5)} = \Set{5 \frac{p}{q} & \frac{p}{q} \in \Z_{(5)}}$, ein Ideal.
            $(\Z_{(5)}, 5\Z_{(5)})$ ist lokaler Ring.
        \item
            $k[x]$ an Stelle von $\Z$, $x \in k[x]$ irreduzibel.
            \begin{math}
                k[x]_{(x)} &= \Set{\frac{f}{g} \in k(x) &  x \ndivs g} \\
                &= \Set{\frac{f}{g} \in k(x) & g(0) \neq 0}.
            \end{math}
            Mit
            \begin{math}
                \frk m = \Set{ \frac{f}{g} \in k(x) & f(0) = 0}.
            \end{math}
    \end{enumerate}
\end{ex}
