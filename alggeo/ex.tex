\begin{ex}
    Berechne $\Spec B[y]$ für einen Hauptidealring $B$.
    \begin{proof}
        Setze $A = B[y]$ und $K = \Quot B$.
        Erinnerung: $f \in K[y]$ läst sich als $f = a f_0$ schreiben mit $a \in K$ und $f_0 \in B[y]$ primitiv (d.h. die Koeffizienten von $f_0$ haben keinen gemeinsamen Faktor).
        Lemma von Gauß: Das Produkt zweier primitiver Polynome bleibt primitiv.

        Die Primideale von $A = B[y]$ sind wie folgt:
        \begin{itemize}
            \item
                $(0)$
            \item
                $(q)$ mit $q \in A$ prim
            \item
                $\frk m = (p,g)$, wobei $p \in B \subset A$ prim in $B$ und $g \in A = B[y]$ mit $\_g \in (B / (p))[y]$ irreduzibel.
        \end{itemize}
        Denn:
        Ist $\frk p \in \Spec A$ ein Hauptideal, dann sind wir fertig.
        Also nehme an, dass $f_1, f_2 \in \frk p$ ohne gemeinsamen Faktor in $A = B[y]$ existieren.

        \begin{enumerate}[1.]
            \item
                Zeige: $f_1, f_2$ haben keinen gemeinsamen Faktor in $K[y]$.

                Falls nicht, d.h. $f_i h g_i$ mit $h, g_1, g_2 \in K[y]$ und $\deg h > 0$.
                Dann betrachte $h = a h_0$, $g_i = b_i \gamma_i$ mit $a, b_i \in K$ und $h_0, \gamma_i \in B[y]$ primitiv.
                Folglich ist $h_0 \gamma_i$ primitiv, also ist
                \begin{math}
                    f_i = hg_i = a b_i \underbrace{h_0 \gamma_i}_{\in A = B[y]} \in A = B[y],
                \end{math}
                also $ab_i \in B$, d.h. $h_0 \divs f_i$ in $A$, ein Widerspruch.
            \item
                Zeige: Sei $\frk a = (f_1, f_2)$, dann folgt $\frk a \cap B \neq (0)$.

                $K[y]$ ist ein HIR und $\gcd(f_1, f_2) = 1$ nach Schritt 1.
                Es existieren $g_1, g_2 \in K[y]$ mit $g_1 f_1 + g_2 f_2 = 1$.
                Sei $b \in B$ gemeinsamer Nenner der Koeffizienten, dann $\frk a \ni bgf_1 + bg_2f_2 = b \in B$, also $0 \neq b \in \frk a \cap B$.
            \item
                Da $\frk p \in \Spec A$, folgt ${\frk p}^c = \frk p \cap B$ ist prim und nichttrivial, d.h. ($B$ HIR) ${\frk p}^c = p \cap B = (p)$ mit $p \in B$ prim. 
                $(p)$ ist maximal in $B$ (da $B$ HIR).
                Also ist $k_p := B / (p)$ ein Körper.
                Betrachte dann die Reduktionsabbildung
                \begin{math}
                    A = B[y] \xto{\rho} k_p[y] \to 0,
                \end{math}
                gegeben durch Reduktion der Koeffizienten modulo $p$ in $B$.
                $\ker \rho = (p)^e$ (bezüglich $B \injto A$) ist das von $p$ in $A$ erzeugte Ideal, $\ker p \subset \frk p$.

                Es existiert ein maximalen Ideal in $k_p[y] \isomorphic A / (p)^e = \im \rho / \ker \rho$ mit $\rho^{-1}(\frk m) = \frk p$.
                $k_p[y]$ Hauptideal, also $\frk m = (\_g)$ mit $\_g$ irreduzibel, $g \in A$.
                Also
                \begin{math}
                    \frk p = \rho^{-1}(\frk m) = {\frk m}^c = (p,g)
                \end{math}
                wie behauptet.
        \end{enumerate}
    \end{proof}
    \begin{note}
        \begin{math}
            A / \frk p
            \isomorphic (A / (p)^e / (\frk p / (p)^e
            \isomorphic k_p[y] / (\_g)
        \end{math}
        ist algebraische, sogar endliche Körpererweiterung von $k_p$.

        Ist $B =k[x]$ und $k$ algebraische abgeschlossen, dann ist $k_p = k[x] /(p) = k$.
        Also $\frk m = (p, g)$ mit $p \in k[x]$ irreduzibel, d.h. $p = x - a$, $a \in k$, $g \in k[x,y]$, $\_g \in (k[x]/(p))[y] \isomorphic k[y]$ irreduzibel, d.h. $\_g = y - b$.

        $\frk m = (x-a, y-b) \subset B[y] = k[x,y]$ entspricht dem Punkt $(a,b) \in k^2$.

        Beweis aus Reid, Commutative Algebra.
    \end{note}
\end{ex}
