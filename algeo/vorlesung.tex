\documentclass{mycourse}

\renewcommand{\O}{\mathcal{O}}

\title{Algorithmische Geometrie}

\begin{document}

\maketitle
\tableofcontents

\chapter{?}

\section{Range-Tree in höheren Dimensionen}

Mit einem 2-4-Baum können Suchanfragen über Intervalle in $\O(\log n + k)$ beantwortet werden (wobei $n$ die Anzahl der Elemente und $k$ die Anzahl der zurückgegebenen Elemente ist).

Wir wollen soeine Suche auf mehrere Dimensionen verallgemeinern.

\begin{seg}[1. Idee]
	Baue jeweils Suchstruktur über jede Dimension auf (z.B. mit einem 2-4-Baum).
	Schneide anschließend die beiden Ergebnismengen $E_1$ und $E_2$, wobei $E_1$ die Menge der Punkte mit $x$-Koordinate in $[l,r]$ und $E_2$ die Menge Punkte mit $y$-Koordinate in $[n,o]$ bezeichnet.

	Allerding kostet das $\O(\log n + k_1 + k_2)$ und $k_1 + k_2$ kann $\approx n$ sein, obwohl die endgültige Ausgabe leer ist.
\end{seg}

\begin{seg}[Bessere Strategie für $\R^2$]
	Baue Suchstruktur (z.B. 2-4-Baum) über $x$-Koordinate auf.
	Speichere explizit für jeden inneren Knoten die Menge der Punkte an den Blättern seines Teilbaumes in einer Suchstruktur über die $y$-Koordinate.

	Der Platzverbrauch beträgt $\O(n \log g)$.

	Eine Anfrage an $[l,r] \times [n,o]$ bestimmt zunächst die Suchpfade für $l$ und $r$ in der Primärstruktur und befragt die Sekundärstrukturen der Knoten, welchen „nach innen“ hängen.

	Der Zeitaufwand für eine Anfrage beträgt $\O(\log n) + \O(\log n \log n) + \O(k) = \O(\log^2 n + k)$
\end{seg}


\end{document}
