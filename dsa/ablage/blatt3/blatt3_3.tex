\documentclass{scrartcl}
\usepackage{mathe-blatt}
\blattdsa

\begin{document}
Die schlechteste Laufzeit ergibt sich, wenn bei jedem Quicksort-Aufruf für das Pivotelement das kleinste oder das größte Element des Arrays gewählt wird.
Für ein Array der Größe $n$ ergibt sich nach der Aufteilung nämlich ein leeres Array und eins der Größe $n-1$, auf das erneut Quicksort aufgerufen wird.
Für die Laufzeit ergibt sich dann:
\[
\mathcal O\left(\sum_{k=1}^nk\right) = \mathcal O(n^2)
\]
Setze $j=\min\{i,n\}$ und betrachte folgende Anordnung:
\[
(n-i+2), (n-i+3), \dotsc, n, \underbrace{1}_{j\text{-te Stelle}}, 2, \dotsc, (n-i+1)
\]
Beim ersten Aufruf wird das $\min\{i,n\}$-te Element als Pivotelment gewählt, also in diesem Fall die $1$, was das kleinste Element im Array ist.
Das entstehende Arrray der Größe $n-1$, auf das wiederum Quicksort aufgerufen wird, sieht so aus:
\[
(n-i+2), (n-i+3), \dotsc, n, \underbrace{2}_{j\text{-te Stelle}}, 3, \dotsc, (n-i+1)
\]
Dabei ist $2$ wiederum das kleinste Element und an $j$-ter Stelle.

Das setzt sich fort, bis das Array Länge $i-1$ und folgende Struktur hat:
\[
(n-i+2), (n-i+3), \dotsc, n
\]
Jetzt wird stets das letzte Element als Pivotelement gewählt, weil $i$ kleiner als die Größe des Arrays ist.
Das letzte Element ist aber ab diesem Schritt stets das Größte in dieser Anordnung.

Damit ist sichergestellt, dass die Laufzeit $\mathcal O(n^2)$ beträgt.


\end{document}

