\documentclass{scrartcl}
\usepackage{dot2texi}

\usepackage{tikz}
\usetikzlibrary{shapes,arrows}

\usepackage{mathe-vorlesung}

\newcommand{\stehlen}{\texttt{stehlen}}
\newcommand{\amort}{\texttt{amort}}
\newcommand{\cost}{\texttt{cost}}
\newcommand{\entfernen}{\texttt{entfernen}}

\author{Fabian Hartkopf, Frank Heyen, Stephan Hilb}
\title{Problem 4}

\begin{document}
\maketitle
Wir zeigen zunächst, dass die amortisierten Kosten einer \stehlen-Operation konstant sind:

Die Potentialfunktion ist (wenn $n_i$ die Anzahl der Knoten mit $i$ Kindern bezeichnet) definiert als
\[
	\phi(z) = 0\cdot n_3 + 1\cdot n_2 + 2\cdot n_1 + 2\cdot n_4 + 4\cdot n_5
\]
Bei der \stehlen-Operation wird einem Geschwisterknoten $b$ (der vor dem Stehlen mindestens drei Kinder hatte) ein Kind gestohlen und dem Knoten $a$ (das nur ein Kind besitzt) hinzugefügt.

Für die restlichen Knoten ändert sich die Anzahl der Kinder nicht.
Also reicht es, die Veränderung der Potentialfunktion für diese beiden Knoten zu betrachten:

Knoten $a$ hatte ein Kind und erhält eines dazu, also sinkt die Potentialfunktion für diesen Knoten um genau $1$ ($\Delta\phi_a = -1$).
Vom Knoten $b$ wissen wir nicht genau, wie viele Kinder er hatte, aber wenn er eines dazu bekommt, kann die Potentialfunktion für diesen Knoten maximal um $2$ zunehmen ($\Delta\phi_b \le 2$).

Also erhält man für den Zustand nach dem Stehlen
\[
	\phi(z+1) = \phi(z) + \Delta\phi_a +\Delta\phi_b \le \phi(z) - 1 + 2 = \phi(z) + 1
\]
Mit anderen Worten: das Potential nimmt beim Stehlen höchstens um 1 zu.

Für die amortisierten Kosten des Stehlens ergibt sich damit
\[
	\amort(\stehlen) = \phi(z+1) - \phi(z) + \cost(\stehlen) \le 1 + \const
\]
also ist $\amort(\stehlen)$ konstant.

Für die $\entfernen$-Operation wird angenommen, dass der Baum nach dem Entfernen durch eine $\stehlen$-Operation balanciert werden kann und keine Verschmelz-Operation nötig ist.

Beim Entfernen nimmt die Potentialfunktion höchstens um 1 zu (z.B. von 3 Kindknoten auf 2, oder von 2 auf 1).
Außerdem muss noch eine \stehlen-Operation durchgeführt werden, bei der (wie oben gezeigt), die Potenzialfunktion ebenfalls höchstens um 1 zunimmt.

Damit ergibt sich für die amortisierten Kosten der \entfernen-Operation:
\begin{align*}
	\amort(\entfernen) &= \phi(z+1) - \phi(z) + \cost(\entfernen)\\
					   &\le 1 + 1 + \const = 2 +\const
\end{align*}
Also ebenfalls konstante Kosten.











\end{document}


