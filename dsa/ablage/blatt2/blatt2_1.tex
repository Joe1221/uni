\documentclass{scrartcl}
\usepackage{mathe-blatt}
\blattdsa

\begin{document}

\setcounter{aufgabe}{0}

\begin{enumerate}
\item
Sei $|f(n)|\le c_1|g(n)|$ für alle $n\ge {n_0}_1$ und $|h(n)|\le c_2|i(n)|$ für alle $n\ge {n_0}_2$.
\begin{enumerate}
\item
\begin{align*}
|f(n)+h(n)|
&\le |f(n)|+|h(n)|
\le c_1|g(n)|+c_2|i(n)| \qquad \forall n\ge \max\left\{{n_0}_1, {n_0}_2\right\}\\
&\le (c_1+c_2)|\max\{g(n),i(n)\}| \\
&\implies f(n)+h(n)\in \mathcal O(\max\{g(n),i(n)\})
\end{align*}
\item
\begin{align*}
|f(n)\cdot h(n)|
&\le |f(n)|\cdot|h(n)|
\le c_1|g(n)|c_2|i(n)| \qquad \forall n\ge \max\left\{{n_0}_1, {n_0}_2\right\}\\
&= (c_1c_2)|g(n),i(n)| \\
&\implies f(n)\cdot h(n)\in \mathcal O(g(n),i(n))
\end{align*}
\end{enumerate}

\item
Sei $|f(n)|\le c_1|g(n)|$ für alle $n\ge {n_0}_1$ und $|g(n)|\le c_2|h(n)|$ für alle $n\ge {n_0}_2$.
Dann gilt:
\begin{align*}
|f(n)|
&\le c_1|g(n)|
\le c_1c_2|h(n)| \qquad \forall n\ge \max\left\{{n_0}_1, {n_0}_2\right\}\\
&\implies f\in \mathcal O(h)
\end{align*}

\item
\begin{enumerate}
\item
Sei $f\in\mathcal O(g)$, mit $|f(n)|\le c|g(n)|$ für $n\ge n_0$, dann ist
\[
\limsup_{n\to\infty}\left|\frac{f(n)}{g(n)}\right|=c<\infty
\]
Sei umgekehrt $S=\limsup_{n\to\infty}\left|\frac{f()}{g(n)}\right|<\infty$ gegeben,
dann existiert ein $N$, sodass für alle $n\ge N$ gilt:
\[
\sup_{m\ge n}\left|\frac{f(m)}{g(m)}\right|
\le S
\implies |f(n)|\le S|g(n)|
\implies f\in\mathcal O(g)
\]
\item
Sei $f\in o(g)$, dann gibt es für alle $c>0$ ein $n_0$, sodass für alle $n\ge n_0$ gilt:
\begin{align*}
|f(n)|\le c|g(n)|
\iff \left|\frac{f(n)}{g(n)}\right| \le c
\end{align*}
Dies entspricht gerade der Grenzwert-Definition für $\lim_{n\to\infty}\left|\frac{f(n)}{g(n)}\right|=0$.
Sei umgekehrt $\lim_{n\to\infty}\left|\frac{f(n)}{g(n)}\right|=0$, dann gibt es für alle $\varepsilon>0$ ein $N$,
sodass für $n\ge N$ gilt:
\[
\left|\frac{f(n)}{g(n)}\right| \le \varepsilon
\iff |f(n)|\le \varepsilon|g(n)|
\implies f\in o(g)
\]
\end{enumerate}

\item
\begin{align*}
a^n\in o(b^n)
\qquad &\iff
\lim_{n\to\infty}\frac{f(n)}{g(n)}
=\lim_{n\to\infty}\frac{a^n}{b^n}
=\lim_{n\to\infty}\left(\frac ab\right)^n
=0\\
&\iff \frac ab<1
\iff a<b
\end{align*}

\item
\begin{lem*}
Es gilt $f\in\omega(g)$ genau dann, wenn $\lim_{n\to\infty}\frac{f(n)}{g(n)}=\infty$.
Der Beweis erfolgt analog zu 3b)
\end{lem*}
\begin{enumerate}
\item
\[
f(n) = \left|\sum_{i=0}^ka_in^k\right|
= n^k\left|\sum_{i=0}^ka_in^{i-k}\right|
= n^k\left|a_k+\sum_{i=1}^ka_in^{i-k}\right|
\]
Da jedoch $\lim_{n\to\infty}\sum_{i=1}^ka_in^{i-k}=0$, existiert $N$ und $c$, sodass für alle $n\ge N$ gilt:
\[
f(n)=n^k\left|a_k+\sum_{i=1}^ka_in^{i-k}\right|\le cn^k
\]
und damit $f\in\mathcal O(n^k)$.
\begin{align*}
f(n) = \left|\sum_{i=0}^ka_in^k\right|
\ge c\left|n^k\right|\\
\qquad \iff \left|\sum_{i=0}^ka_in^{i-k}\right| \ge c
\end{align*}
Da $\lim_{n\to\infty}\sum_{i=0}^ka_in^{i-k}=a_k$, existiert $N$, sodass für alle $\varepsilon>0$ und $n\ge N$ gilt:
\begin{align*}
\left|\sum_{i=0}^ka_in^{i-k}\right|
\ge a_k +\varepsilon =: c\\
\qquad \implies 
f(n) = \left|\sum_{i=0}^ka_in^i\right|
\ge c\left|n^k\right|
\end{align*}
und damit $f\in\Omega(n^k)$.

\item
Da $l>k$:
\[
\lim_{n\to\infty}\frac{f(n)}{n^l}
=\lim_{n\to\infty}\frac{\sum_{i=0}^ka_in^i}{n^l}
=\lim_{n\to\infty}\frac{\sum_{i=0}^ka_in^{i-l}}1 =0
\]
und damit $f\in o(n^l)$
\item
Beweis erfolgt analog zu 5b) mit $f\in\omega(g) \iff \lim_{n\to\infty}\frac{f(n)}{g(n)}=\infty$.
\item
Da $f$ von Grad $k$ ist, ist $f^{(k)}(n)$ beschränkt.
Also gilt:
\[
\lim_{n\to\infty}\frac{f(n)}{a^n}=
\lim_{n\to\infty}\frac{f^{(k)}(n)}{\ln^ka\cdot a^n}
=0
\]
und damit $f\in o(a^n)$
\item
Beweis erfolgt analog zu 5d).
\end{enumerate}
\end{enumerate}


\end{document}
