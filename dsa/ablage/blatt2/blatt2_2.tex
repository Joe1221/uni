\documentclass{scrartcl}
\usepackage{mathe-blatt}

\blattdsa


\begin{document}

\begin{enumerate}
\item
Zeige durch Kontraposition. Angenommen $f\in\Omega(g)$, also
\[
\exists c>0,n_0:n\ge n_0 \implies |f(n)| \ge c|g(n)|
\]
dann gilt ($g(n)\not\to 0$, da sonst $f\not\in o(g)$ trivial):
\[
|f(n)| \ge c|g(n)| > \frac c2|g(n)|
\]
was jedoch der Vorraussetzung
\[
f\in o(g) \iff |f(n)|\le c|g(n)| \qquad \forall c>0
\]
widerspricht.

\item
Gegenbeispiel:
\[
f(n)=n^2\sin(\frac \pi2n), \qquad g(n)=n
\]
Es gilt $f\not\in\Omega(g)$, da die geraden Folgenglieder von $f$ gleich $0<c|g(n)|$.

Außerdem aber auch: $f\not\in o(g)$, denn die ungeraden Folgenglieder von f wachsen asymptotisch schneller als $g$.

\item
Gegenbeispiel:
\[
f(n)=n\sin{\frac \pi2n}, \qquad g(n)=n
\]
Es ist $f\in\mathcal O(g)$ für $c=1,n_0=1$ und $f\not\in\Omega(g)$ (für beliebiges $c$ und $n_0$ wähle $n=2n_0$ und damit $|f(2n_0)|=0<c|g(n)|$).

Außerdem ist aber $f\not\in o(g)$, denn für $c=\frac 12$ und $n=2k+1$ ist $|f(n)|>\frac 12|g(n)$.

\item
Die Hinrichtung wurde in 1. gezeigt.
Zeige die Rückrichtung durch Kontraposition: sei $f\not\in o(g)$, dann existiert, weil $f$ und $g$ Polynome sind, folgender Grenzwert:
\[
\lim_{n\to\infty}\left|\frac{f(n)}{g(n)}\right|>0
\]
und damit auch ein $c$, so dass für hinreichend große $n$ gilt:
\[
|f(x)|\ge c|g(n)|
\]
Womit $f\in\Omega(g)$ wäre.

\end{enumerate}


\end{document}
