\documentclass{scrartcl}

\usepackage[utf8x]{inputenc}
\usepackage{ngerman}

\usepackage{bbm}
\usepackage{algpseudocode}
\usepackage{verbatim}
\usepackage{amsmath}
\usepackage{amssymb}
\usepackage{amsthm}
\usepackage{enumerate}

\author{Fabian Hartkopf, Frank Heyen, Stephan Hilb}
\title{Problem 4}

\begin{document}

\maketitle

Wir zeigen induktiv, dass mindestens $\lceil \frac {3n}2 \rceil - 2$ Vergleiche nötig sind, um in einer Menge von Elementen mit Sicherheit das Größte und Kleinste zu finden.
\\[1em]

Man überzeuge sich leicht, dass die minimale Anzahl nötiger Vergleiche für eine Menge mit einem Element $0$ und für eine Menge mit zwei Elementen $1$ beträgt.
Das deckt sich auch mit der Formel:
\begin{align*}	
	\left\lceil \frac {3\cdot1}2 \right\rceil - 2 &= 2 -2 = 0\\
   \left\lceil \frac{3\cdot 2}{2} \right\rceil -2 &= 3 - 2 = 1
\end{align*}
Damit haben wir den Induktionsanfang für eine gerade und eine ungerade Zahl gezeigt.

Unsere Induktionsannahme sei, dass wir für eine Menge mit $n$ Elementen mindestens
\[
	\left\lceil \frac {3n}2 \right\rceil -2
\]
Vergleiche benötigen, um Größtes und Kleinstes Element zu bestimmen.
\\[1em]

Haben wir jetzt eine Menge von $n+2$ Elementen gegeben, so muss jeder Algorithmus zur Lösung des Problems zu Beginn zwei Elemente wählen und miteinander vergleichen (!).
Wir nennen diese beiden Elemente $a$ und $b$. 
Weiterhin sei ohne Beschränkung der Allgemeinheit $a<b$.
Die Anzahl möglicher Elemente, die noch kleinstes \emph{oder} größtes Element sein könnten ist jetzt $n$ ($a$ könnte höchstens kleinstes Element sein, aber nicht größtes, analog $b$).

Der Algorithmus muss also von diesen verbleibenden $n$ Elementen das kleinste Element $c$ und das größte Element $d$ ermitteln.
Nach Induktionsannahme wird dazu mindestens $\lceil \frac {3n}2\rceil -2$ Vergleiche benötigt.

Jetzt gibt es zwei Kandidaten $a$ und $c$ für das kleinste Element und zwei Kandidaten $b$ und $d$ für das größte Element der Menge mit $n+2$ Elementen.
Man überzeugt sich leicht, dass für diese Ermittlung mindestens $2$ Vergleiche von nöten sind (einer zwischen $a$ und $c$ und einer zwischen $b$ und $d$).
Es ergibt sich also insgesamt für die Mindestanzahl nötiger Vergleiche:
\begin{align*}
	1 + \left\lceil \frac {3n}2 \right\rceil -2 + 2 &= \left\lceil \frac{3n}2 + 3\right\rceil -2\\
																																	 &= \left\lceil\frac{3(n+2)}2\right\rceil -2
\end{align*}
Damit ist der Induktionsschritt für $n+2$ elmentige Mengen gezeigt.
Da der Induktionsanfang mit einer geraden und einer ungeraden Anzahl ausgeführt wurde, gilt die Behauptung für alle $n\ge 1$.






\end{document}
