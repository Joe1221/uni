\chapter{Einführung}

\Timestamp{2014-04-13}

\section{Zöpfe}

\subsection{Zöpfe}

Elementare Zöpfe: $s_i$ (Stränge $i, i+1$ vertauschen).
Folgende Relationen sind erfüllt:
\begin{math}
    s_i s_i^{-1} &= 1 \\
    s_i s_{i+1} s_i &= s_{i+1} s_i s_{i+1} \\
    s_is_j &= s_j s_i \quad \text{für $|i-j| \ge 2$}
\end{math}
Diese Relationen bilden die Zopfgruppe $B_n$.
Wir beobachten $B_1 = 1$, $B_2 = \Z$.

Drall $v: (B_n, *) \to (\Z, +)$, orientiertes Zählen von Kreuzungen.

Der Drall ist eine Invariante, genügt aber nicht zur Klassifikation.
$B_n$ für $n \ge 3$ ist nicht kommutativ.

\subsection{Dirac-Zöpfe}

Ein Ende offen, neue Relation:
\begin{math}
    s_1s_2 \dotsb s_{n-1}s_{n-1} \dotsb s_2 s_1 = 1.
\end{math}
Diese bilden die Dirac-Gruppe $B_n'$.

\section{Knoten}

Reidemeister-Züge ändern das Diagramm, aber nicht den Knoten.
Diese elementaren Bewegungen reichen bereits aus.

