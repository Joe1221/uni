\chapter{Einführung}

\Timestamp{2014-04-13}

\section{Zöpfe}

\subsection{Zöpfe}

Elementare Zöpfe: $s_i$ (Stränge $i, i+1$ vertauschen).
Folgende Relationen sind erfüllt:
\begin{math}
    s_i s_i^{-1} &= 1 \\
    s_i s_{i+1} s_i &= s_{i+1} s_i s_{i+1} \\
    s_is_j &= s_j s_i \quad \text{für $|i-j| \ge 2$}
\end{math}
Diese Relationen bilden die Zopfgruppe $B_n$.
Wir beobachten $B_1 = 1$, $B_2 = \Z$.

Drall $v: (B_n, *) \to (\Z, +)$, orientiertes Zählen von Kreuzungen.

Der Drall ist eine Invariante, genügt aber nicht zur Klassifikation.
$B_n$ für $n \ge 3$ ist nicht kommutativ.

\subsection{Dirac-Zöpfe}

Ein Ende offen, neue Relation:
\begin{math}
    s_1s_2 \dotsb s_{n-1}s_{n-1} \dotsb s_2 s_1 = 1.
\end{math}
Diese bilden die Dirac-Gruppe $B_n'$.

\section{Knoten}

Reidemeister-Züge ändern das Diagramm, aber nicht den Knoten.
Diese elementaren Bewegungen reichen bereits aus.

\Timestamp{2015-04-15}

Bilden Knoten auch eine Gruppe?
Es existiert nicht unbedingt ein inverses Element (Dreifärbung).
Die Knoten bilden ein Monoid.

Die Anzahl der Dreifärbungen $\col(A)$ bilden eine Invariante unter Reidemeisterzüge (nach Fox).
$\col(\argdot)$ ist mulitplikativ.


\section{Verschlingungen}

Mit/Ohne Orientierung.
Analog wie oben, definiere den Drall $v: \arrow{\scr D} \to \Z$.
Der Drall ist keine Invariante unter R1-Zügen, aber unter R2-, R3-Zügen.
Definiere die Verschlingungszahl als
\begin{math}
    \lk(D) = \frac{1}{2} \sum_{\text{$c$ gemischt}} v(c).
\end{math}
Diese ist eine invariante unter Reidemeister-Züge.

Für $\lk(L) \neq 0$ lässt sich $L$ nicht in Kompenenten trennen.
Die Umkehrung gilt nicht!

\section{Reduzierte und Alternierende Diagramme}

Für solche Diagramme lassen sich keine Reidemeister-Züge ausführen (ohne die Eigenschaft zu verletzen).
Sie bilden eine Art „lokaler Minima“ für die Kreuzungszahl.

Tait-Vermutungen:
\begin{enumerate}[1.]
    \item
        Reduzierte, alternierende Diagramme besitzen minimale Kreuzungszahl
    \item
        Reduzierte, alternierende Diagramme des selben Knotens besitzen den selben Drall
    \item
        Je zwei reduzierte, alternierende Diagramme des selben Knotens lassen sich durch „Flypes“ ineinander überführen.
\end{enumerate}
