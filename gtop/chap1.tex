% A
\chapter{Der Satz von Reidemeister und erste Invarianten}

\Timestamp{2015-04-15}

Unser Ziel ist folgender Satz:
\begin{st}
    Jeder Knotentyp in $\R^3$ lässt sich durch ein Knotendiagramm im $\R^2$ darstellen.
    Je zwei solcher Diagramme stellen genau dann den selben Knotentyp dar, wenn sie sich durch Reidemeisterzüge $R1, R2, R3$ ineinander überführen lassen.
\end{st}
Dazu sind zunächst die Begriffe \emph{Knotentyp} und \emph{Knotendiagramm} zu klären.

% A1
\section{Polygonale Knoten}

Ein Polygonzug $P = (v_0, v_1, \dotsc, v_n)$ im $\R^d$ ist eine endliche Folge von Punkten $v_0, \dotsc, v_n \in \R^d$. 
Gibt man eine Unterteilung $0 = t_0 < t_1 < \dotsb < t_n = 1$ vor, so definiert dies $\gamma_P: [0,1] \to \R^d$ durch affine Interpolation:
\begin{math}
    \gamma_P(t) = \frac{t_k - t}{t_k - t_{k-1}} v_{k-1} + \frac{t-t_{k-1}}{t_k -t_{k-1}} v_k
\end{math}
für $t \in [t_{k-1}, t_k]$.
$P$ heißt \emphdef{einfach}, wenn für alle $i < j$
\begin{math}
    [v_{i-1}, v_i] \cap [v_{j-1}, v_j]
    = \begin{cases}
        \Set{v_i} & \text{für $j = i + 1$} \\
        \emptyset & \text{für $j \ge i + 2$}
    \end{cases},
\end{math}
d.h. genau dann, wenn $\gamma_p$ injektiv ist.
$P$ heißt \emphdef{geschlossen}, wenn $v_0 = v_n$.
Wir erhalten dann $\_\gamma_P: \S^1 \to \R^d$ durch $\_\gamma(e^{2\pi i t} = \gamma_P(t)$.

$P$ heißt \emphdef{einfach geschlossen} (oder \emphdef{doppelpunktfrei}), wenn
\begin{math}
    (v_{i-1}, v_i] \cap ( v_{j-1}, v_j] = \emptyset
\end{math}
für $1 \le i < j \le n$, d.h. genau dann, wenn $\_\gamma_p$ injektiv ist.

\begin{df}
    Ein (polygonaler) Knoten in $\R^d$ ist ein einfach geschlossener Polygonzug $P = (v_0, v_1, \dotsc, v_n)$ in $\R^d$.
    Hierzu gehört die Parametrisierung $\_\gamma_p: \S^1 \injto \R^d$ mit Bildmenge $K = |P| := [v_0, v_1] \cup \dotsb \cup [v_{n-1},v_n] \subset \R^d$.
\end{df}

Wir betrachten Züge
\begin{math}
    P = (v_0, \dotsc, v_k, v_{k+1}, \dotsc, v_n)
    \leftrightarrow
    P' = (v_0, \dotsc, v_k, v, v_{k+1}, \dotsc, v_n).
\end{math}
Für die Betrachtung von Knoten ist eine Einschränkung der Bewegungen nützlich.

Wir setzen $\Delta := [v_k, v, v_{k+1}]$ und fordern
\begin{math}
    |P| \cap \Delta = [v_k, v_{k+1}]
    \quad \land \quad
    |P'| \cap \Delta = [v_k, v] \cup [v, v_{k+n}].
\end{math}
Dies nennen wir einen \emphdef[Delta-Zug]{$\Delta$-Zug}


\begin{df}
    Seien $P$ und $P'$ polygonale Knoten in $\R^d$.
    Eine \emphdef{polygonale Isotopie} von $P$ und $P'$ ist eine Folge von $\Delta$-Zügen:
    \begin{math}
        P = P_0 \leftrightarrow P_1 \leftrightarrow \dotsb \leftrightarrow P_l = P'.
    \end{math}
    Wir nennen $P, P'$ dann \emphdef{polygonal isotop} (oder $\Delta$-äquivalent).

    Die Äquivalenzklasse $[P]$ heißt der \emph{Knotentyp} von $P$.
\end{df}
