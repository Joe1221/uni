% A
\chapter{Der Satz von Reidemeister und erste Invarianten}

\Timestamp{2015-04-15}

Unser Ziel ist folgender Satz:
\begin{st}
    Jeder Knotentyp in $\R^3$ lässt sich durch ein Knotendiagramm im $\R^2$ darstellen.
    Je zwei solcher Diagramme stellen genau dann den selben Knotentyp dar, wenn sie sich durch Reidemeisterzüge $R1, R2, R3$ ineinander überführen lassen.
\end{st}
Dazu sind zunächst die Begriffe \emph{Knotentyp} und \emph{Knotendiagramm} zu klären.

% A1
\section{Polygonale Knoten}

Ein Polygonzug $P = (v_0, v_1, \dotsc, v_n)$ im $\R^d$ ist eine endliche Folge von Punkten $v_0, \dotsc, v_n \in \R^d$. 
Gibt man eine Unterteilung $0 = t_0 < t_1 < \dotsb < t_n = 1$ vor, so definiert dies $\gamma_P: [0,1] \to \R^d$ durch affine Interpolation:
\begin{math}
    \gamma_P(t) = \frac{t_k - t}{t_k - t_{k-1}} v_{k-1} + \frac{t-t_{k-1}}{t_k -t_{k-1}} v_k
\end{math}
für $t \in [t_{k-1}, t_k]$.
$P$ heißt \emphdef{einfach}, wenn für alle $i < j$
\begin{math}
    [v_{i-1}, v_i] \cap [v_{j-1}, v_j]
    = \begin{cases}
        \Set{v_i} & \text{für $j = i + 1$} \\
        \emptyset & \text{für $j \ge i + 2$}
    \end{cases},
\end{math}
d.h. genau dann, wenn $\gamma_p$ injektiv ist.
$P$ heißt \emphdef{geschlossen}, wenn $v_0 = v_n$.
Wir erhalten dann $\_\gamma_P: \S^1 \to \R^d$ durch $\_\gamma(e^{2\pi i t} = \gamma_P(t)$.

$P$ heißt \emphdef{einfach geschlossen} (oder \emphdef{doppelpunktfrei}), wenn
\begin{math}
    (v_{i-1}, v_i] \cap ( v_{j-1}, v_j] = \emptyset
\end{math}
für $1 \le i < j \le n$, d.h. genau dann, wenn $\_\gamma_p$ injektiv ist.

\begin{df}
    Ein (polygonaler) Knoten in $\R^d$ ist ein einfach geschlossener Polygonzug $P = (v_0, v_1, \dotsc, v_n)$ in $\R^d$.
    Hierzu gehört die Parametrisierung $\_\gamma_p: \S^1 \injto \R^d$ mit Bildmenge $K = |P| := [v_0, v_1] \cup \dotsb \cup [v_{n-1},v_n] \subset \R^d$.
\end{df}

Wir betrachten Züge
\begin{math}
    P = (v_0, \dotsc, v_k, v_{k+1}, \dotsc, v_n)
    \leftrightarrow
    P' = (v_0, \dotsc, v_k, v, v_{k+1}, \dotsc, v_n).
\end{math}
Für die Betrachtung von Knoten ist eine Einschränkung der Bewegungen nützlich.

Wir setzen $\Delta := [v_k, v, v_{k+1}]$ und fordern
\begin{math}
    |P| \cap \Delta = [v_k, v_{k+1}]
    \quad \land \quad
    |P'| \cap \Delta = [v_k, v] \cup [v, v_{k+n}].
\end{math}
Dies nennen wir einen \emphdef[Delta-Zug]{$\Delta$-Zug}


\begin{df}
    Seien $P$ und $P'$ polygonale Knoten in $\R^d$.
    Eine \emphdef{polygonale Isotopie} von $P$ und $P'$ ist eine Folge von $\Delta$-Zügen:
    \begin{math}
        P = P_0 \leftrightarrow P_1 \leftrightarrow \dotsb \leftrightarrow P_l = P'.
    \end{math}
    Wir nennen $P, P'$ dann \emphdef{polygonal isotop} (oder $\Delta$-äquivalent).

    Die Äquivalenzklasse $[P]$ heißt der \emph{Knotentyp} von $P$.
\end{df}


\Timestamp{2015-04-20}

\begin{ex}
    Man bestimme alle Knotentypen in $\R^2$ und anschließend in $\R^d$ für $d > 4$.
\end{ex}

\begin{df}
    Ein Knoten $P = (v_1, \dotsc, v_n)$ in $\R^d$ heißt \emphdef{trivial}, wenn $P$ isotop ist zu einem Dreieck, d.h. zu $P' = (v_1,v_2,v_3)$.
\end{df}

\begin{prop}
    Jeder Knoten in $\R^2$ ist trivial.
    \begin{proof}
        Das ist die Aussage des (polygonalen) Satzes von Schönflies.
    \end{proof}
\end{prop}

\begin{note}
    In $\R^2$ gibt es \emph{zwei} Äquivalenzklassen trivialer Knoten (Orientierung).
    In $\R^d$ mit $d \ge 3$ sind alle trivialen Knoten äquivalent.
    In $\R^d$ mit $d \ge 4$ sind alle Knoten trivial.
\end{note}

\begin{ex}
    Ein (polygonaler) Knoten $P$ in $\R^3$ ist trivial genau dann, wenn er eine Scheibe berandet, also eine eingebettete simpliziale Fläche $S \subset \R^3$, sodass $\Boundary S = |P|$ und $S \homeomorphic \D^2$.
    \begin{proof}
        Hinrichtung leicht, Rückrichtung wie Schönflies.
    \end{proof}
\end{ex}

\subsection{Beweglichkeit von Knoten}

\begin{st}
    Sei $P = (v_1, \dotsc, v_n)$ in $\R^d$ ein polygonaler Knoten.
    Dann existiert ein $\epsilon > 0$, sodasss jede Folge $P' = (v_1', \dotsc, v_n')$ in $\R^d$ mit $|v_i - v_i'| < \epsilon$ einen Knoten definiert und $P' \sim P$ gilt.
    \begin{enumerate}[(1)]
        \item
            $P \sim P + v$ für $v \in \R^d$,
        \item
            $P \sim \rho(P)$ für jede Drehung $\rho$,
        \item
            $P \sim \lambda P$ für $\lambda \in \R_{>0}$.
            Für $\lambda < 0$ gilt dies im Allgemeinen nicht.
    \end{enumerate}
    \begin{proof}
        Wähle
        \begin{math}
            \epsilon := \frac{1}{2} \min \Set{\Set{d(v_i, v_j) & |i-j| \ge 1 } \cup \Set{d([v_{i-1},v_i], [v_{j-1},v_j]) & |i-j| \ge 2}}.
        \end{math}
        \begin{enumerate}[(1)]
            \item
                Wähle $n \in \N$ so groß, dass $|\frac{v}{n}| < \epsilon$.
                Betrachte $\tau: \R^3 \to \R^3, x \mapsto x + \frac{v}{n}$.
                Nun ist $P \sim \tau(P) \sim \dotsb \sim \tau^n(P) = P + v$.
            \item
                Analog.
            \item
                Ähnlich.
        \end{enumerate}
    \end{proof}
\end{st}

% todo: Für affine Abbildungen

\begin{kor}
    Die Menge aller Knotentypen im $\R^d$ ist (höchstens) abzählbar.
    \begin{proof}
        Wir betrachten wieder polygonale Knoten mit Ecken in $\Q^d$.
        Dies sind abzählbar viele und repräsentieren alle Knotentypen dank Beweglichkeit.
    \end{proof}
\end{kor}

\begin{note}
    \begin{enumerate}[(1)]
        \item
            Es gilt $P(v_1, v_2, \dotsc, v_n) \sim P' = (v_2, \dotsc, v_n, v_1)$.
        \item
            Aus $|P| = |P'|$ in $\R^d$ mit gleicher Orientierung folgt $P \sim P'$.
    \end{enumerate}
\end{note}


% A2
\section{Knotendiagramme und der Satz von Reidemeister}


Betrachte die Projektion $\pi: \R^3 \to \R^2, (x,y,z) \mapsto (x,y)$.
Das Bild eines Knotens unter $\pi$ mit Kreuzungsinformationen wollen wir Knotendiagramm nennen.

\begin{df}
    Ein Polygonzug $P = (v_1, \dotsc, v_n)$ in $\R^d$ heißt \emphdef{regulär}, wenn
    \begin{math}
        [v_{i-1},v_i] \cap [v_{j-1},v_j] \cap [v_{k-1}, v_{k}] = \emptyset
    \end{math}
    für alle $1 \le i < j < k \le n$ gilt.
\end{df}

\begin{note}
    Durch Einfügen von Zwischenpunkten können wir annehmen:
    Jede Kante $[v_{i-1}, v_i]$ schneidet neben $[v_{i-2}, v_{i-1}]$ und $[v_{i},v_{i+1}]$ höchstens eine weitere Kante $[v_{j-1},v_j]$ mit $|i-j| \ge 2$.
    Außerdem schneiden die Nachbarn jeweils nur ihre eigenen Nachbarn.

    Wenn sich zwei Kanten kreuzen, so wählen wir eine als „drüber“, die andere als „drunter“.
    Der reguläre Polygonzug mit diser Zusatzinfo heißt \emphdef{Diagramm}.
\end{note}

\begin{lem}
    Zu jedem Knotendiagramm $D$ in $\R^2$ existiert eine Hochhebung zu einem Knoten $P$ in $\R^3$ mit $\pi(P) = D$ und in jeder Kreuzung hat die drüberlaufende Kante höhere $z$-Koordinate als die drunterlaufende.
    Je zwei solcher Hochhebungen $P, P'$ zu $D$ sind isotop (d.h. $\Delta$-äquivalent).
\end{lem}

In diesem Sinne stellt jedes Knotendiagramm einen Knotentyp dar.


\subsection{Reidemeister-Züge für polygonale Diagramme im \texorpdfstring{$\R^2$}{ℝ²}}


Im wesentlichen drei Reidemeisterzüge $R1$ bis $R3$ ($R0$ unwesentlich).

\begin{lem}
    Zu jedem Knotendiagramm $D = (v_1, \dotsc, v_n)$ in $\R^2$ existiert $\epsilon > 0$ sodass jedes $D' = (v_1', \dotsc, v_n')$ in $\R^2$ ebenfalls ein Knoten diagramm ist und $D' \stack{R0}\sim D$ gilt.
    \begin{enumerate}[(1)]
        \item
            $D \sim D + v$,
        \item
            $D \sim \rho(D)$,
        \item
            $D \sim \lambda D$ wie zuvor.
    \end{enumerate}
    \begin{proof}
        Ähnlich wie zuvor.
    \end{proof}
\end{lem}

\begin{lem}
    Seien $D, D'$ zwei Knotendiagramme in $\R^2$ und $P, P'$ in $\R^3$ Hochhebungen.
    Aus $D \stack{R}\sim D'$ folgt $P \stack{\Delta}\sim P'$.
\end{lem}

\begin{df}
    Zu $P$ in $\R^3$ heißt eine Projektionsrichtung $v \in \S^2$ \emphdef{regulär}, wenn $\pi: \R^2 \to \<v\>^\orth \homeomorphic \R^2$ den Knoten $P$ regulär projiziert, d.h. $D = \pi(P)$ ist regulär, andernfalls \emphdef{singulär}.
\end{df}

\begin{lem}
    Die Menge der singulären Richtungen zu $P$ besteht aus endlich vielen Punkten und Kurven auf der Sphäre $\S^2$.
    Ihr Komplement, die Menge aller regulären Richtungen, ist daher offen und dicht.
\Timestamp{2015-04-22}
    \begin{proof}
        Drei Fälle singulärer Richtungen $v \in \S^2$:
        \begin{enumerate}[(1)]
            \item
                Eine Kante $[v_{i-1},v_i]$ wird auf einen Punkt projiziert.
                Dies ist genau dann der Fall, wenn $v \in \R(v_i - v_{i-1}) \cap \S^2 = \Set{\pm u}$.
            \item
                Ein Punkt $v_i$ und eine Kante $[v_{j-1}, v_j]$ fallen zusammen.
                Dies ist genau dann der Fall, wenn $v \in (\R(v_{j-1} - v_i) + \R(v_j - v_i)) \cap \S^2 = \texp{Großkreis}$.
            \item
                Drei Kanten schneiden sich in einem Punkt.
                Ohne Einschränkung seien die Geraden $G_1, G_2, G_3$ paarweise windschief (sonst siehe Fall 2).
                Zu $p_1 \in G_1$ existiert höchstens ein $p_2 \in G_2$, sodass $\_{p_1 p_2} \cap G_3 = \Set{p_3}$.
                Somit parametrisiert $p_1 \in G_1$ eine Geradenschaar, deren Richtungen eine Kurve auf $\S^2$ (ausrechnen liefert eine Quadrik).
        \end{enumerate}
        In allen drei Fällen hat die Ausnahmemenge in $\S^2$ ein offenes, dichtes Komplement.
        Der endliche Schnitt dieser Komplemente ist offen und dicht (Übung).
    \end{proof}
\end{lem}

\begin{st}[Reidemeister, 1926]
    Zwei Knotendiagramme $D, D'$ in $\R^2$ stellen genau dann den selben Knotentyp in $\R^3$ dar, wenn sie Reidemeister-äquivalent sind.
    \begin{proof}
        Seien $P, P'$ in $\R^3$ Hochhebungen von $D$, bzw. $D'$.
        Falls $D \stack{R}\sim D'$, dann ist auch $P \stack{\Delta}\sim P'$.

        Sei nun umkehrt $P \stack{\Delta}\sim P'$, d.h.
        %\begin{math}
        %    % fixme: \Delta oben, R? unten, \pi nach unten
        %    \begin{tikzcd}
        %        P = P_0 \arrow{r}{g} & P_1 & P_2 & \dots &  P_l = P' \\
        %        D = D_0 & D_1 & D_2 & \dots &  D',
        %    \end{tikzcd}
        %\end{math}
        für $\pi: \R^3 \to \R^2$.
        Es existiert eine beliebig kleine Drehung $\rho: \R^3 \to \R^3$, sodass $\pi$ alle $\rho(P_i)$ regulär projiziert.
        Wir wählen $\rho$ so klein, dass $\pi(P_0)$ zu $\pi(\rho P_0)$ $R0$-äquivalent sind.
        % fixme: tikzcd

        Der Vorteil dieser Drehung ist, dass jeder $\Delta$-Zug regulär projiziert wird.
        Wir betrachten nun elementare auftretende Fälle (Skizze).

        Nach hinreichend feiner Unterteilung in kleine Dreiecke erhalten wir eine Folge elementarer Fälle.
    \end{proof}
\end{st}

\begin{ex}
    Dreifärbungen nach Fox.
    Es gilt
    \begin{math}
        |\Col_3(\texp{Trivialer Knoten})| &= 3, &
        |\Col_3(\texp{Kleeblattschlinge})| &= 3,
    \end{math}
    Die Dreifärbung von Knotendiagrammen ist $R$-invariant und also auch eine Invariante unter den Knotentypen.
\end{ex}

\begin{kor}
    In $\R^3$ existieren abzählbar unendlich viele Knotentypen.
\end{kor}


\subsection{Verallgemeinerung zu \texorpdfstring{$p$}{p}-Färbungen}

Sei $p \ge 3$ ungerade (später: $p$ prim).

Wir Betrachten Färbungen von $D$, d.h. $f: \Set{\text{Bögen von } D} \to \Z / p$ mit folgender Kreuzungsbedingung:
% Kreuzung, ac drunter, bd drüber
\begin{math}
    b = d
    \quad\land\quad a - b + c - d = 0
\end{math}
Sei $\Col_p(D)$ die Menge dieser Färbungen.

\begin{st}
    Jeder $R$-Zug $D \leftrightarrow D'$ induziert eine Bijektion $\Col_p(D) \leftrightarrow \Col_p(D')$.
    \begin{proof}
        Man betrachte die drei elementaren Reidemeister-Züge.
    \end{proof}
\end{st}

\begin{note}
    Für $p = 3$ erhalten wir die einfache Regel:
    An jeder Kreuzung entweder 1 oder 3 Farben.
\end{note}

\begin{ex}
    Ist der Achterknoten äquivalent zum trivialen Knoten?
    Es gilt
    \begin{math}
        \Col_3(\texp{Achterknoten}) &= 3, &
        \Col_5(\texp{Achterknoten}) &= 25,
    \end{math}
\end{ex}

\begin{prop}
    Sei $p$ prim.
    Bezüglich der punktweisen Operationen ist $\Col_p(D)$ ein $\F_p$-Vektorraum.
    Für $n = \dim \Col_p(D)$ gilt $\Col_p(D) \isomorphic \F_p$ und $|\Col_p(D)| = p^n$.
\end{prop}

\begin{df}
    Für jedes Diagramm $D$ in $\R^2$ sei $\Cr(D)$ die Anzahl der Kreuzungen.

    Sei $P$ ein polygonaler Knoten und $[P]$ ein Knotentype.
    Wir setzen die Kreuzungszahl $\Cr([P]) := \min \Cr(D)$, minimiert über alle darstellende Diagramme $D$ zu $[P]$.
\end{df}

\begin{ex}
    Alle Knoten mit $\le 2$ Kreuzungen sind äquivalent zum trivialen Knoten.
    Es gilt $\Cr(\texp{trivialer Knoten}) = 0$ und $\Cr(\texp{Kleeblattschlinge}) = 3$.
\end{ex}

\subsection{Topologische und glatte Knoten}

\begin{df}
    Ein \emphdef{topologischer Knoten} ist eine Einbettung $\kappa: \S^1 \injto \R^3$.
    Zwei Knoten $\kappa_0, \kappa_1$ heißen \emphdef{isotop}, wenn es eine Isotopie gibt, also $H: [0,1] \times \S^1 \to \R^3$ stetig, sodass $h_t: \S^1 \to \R^3$, $h_t(s) = h(t,s)$ eine Einbettung ist.
\end{df}

\begin{note}
    $\kappa$ kann wild aussehen: betrachte den Kleeblattknoten hintereindandergereiht, jeweils skaliert mit Faktor $q^k$, $q < 1$.
\end{note}


