% A
\chapter{Der Satz von Reidemeister und erste Invarianten}

\Timestamp{2015-04-15}

Unser Ziel in diesem Kapitel ist folgender Satz:
\begin{st}
    Jeder Knotentyp in $\R^3$ lässt sich durch ein Knotendiagramm im $\R^2$ darstellen.
    Je zwei solcher Diagramme stellen genau dann den selben Knotentyp dar, wenn sie sich durch Reidemeisterzüge $\Reid1, \Reid2, \Reid3$ ineinander überführen lassen.
\end{st}
Dazu sind zunächst die Begriffe \emph{Knotentyp} und \emph{Knotendiagramm} zu klären.

% A1
\section{Polygonale Knoten}

\begin{df}
    Ein \emphdef{Polygonzug} $P = (v_0, v_1, \dotsc, v_n)$ im $\R^d$ ist eine endliche Folge von Punkten $v_0, \dotsc, v_n \in \R^d$.

    Die Wahl einer Unterteilung $0 = t_0 < t_1 < \dotsb < t_n = 1$ definiert eine Parametrisierung $\gamma_P: [0,1] \to \R^d$ durch affine Interpolation:
    \begin{math}
        \gamma_P(t) &= \frac{t_k - t}{t_k - t_{k-1}} v_{k-1} + \frac{t-t_{k-1}}{t_k -t_{k-1}} v_k,
        && t \in [t_{k-1}, t_k].
    \end{math}

    $P$ heißt \emphdef{einfach}, wenn für alle $i < j$
    \begin{math}
        [v_{i-1}, v_i] \cap [v_{j-1}, v_j]
        = \begin{cases}
            \Set{v_i} & \text{für $j = i + 1$} \\
            \emptyset & \text{für $j \ge i + 2$}
        \end{cases},
    \end{math}
    d.h. genau dann, wenn $\gamma_p$ injektiv ist.

    $P$ heißt \emphdef{geschlossen}, wenn $v_0 = v_n$ ist.
    Wir erhalten dann die $\_\gamma_P: \S^1 \to \R^d$ durch $\_\gamma(e^{2\pi i t}) = \gamma_P(t)$.

    $P$ heißt \emphdef{einfach geschlossen} (oder \emphdef{doppelpunktfrei}), wenn $v_0 = v_n$ und
    \begin{math}
        (v_{i-1}, v_i] \cap ( v_{j-1}, v_j] = \emptyset
    \end{math}
    für $1 \le i < j \le n$, d.h. genau dann, wenn $\_\gamma_p$ injektiv ist.
\end{df}

\begin{df}
    Ein \emphdef[Knoten!polygonal]{polygonaler Knoten} in $\R^d$ ist ein einfach geschlossener Polygonzug $P = (v_0, v_1, \dotsc, v_n)$ in $\R^d$.
    Hierzu gehört eine \emphdef[Knoten!Parametrisierung]{Parametrisierung} $\_\gamma_P: \S^1 \xto[injective] \R^d$ mit \emphdef[Knoten!Bildmenge]{Bildmenge} $K := |P| := [v_0, v_1] \cup \dotsb \cup [v_{n-1},v_n] \subset \R^d$.

    \begin{note}
        Wir schreiben für polygonale Knoten auch $P = (v_1, \dotsc, v_n)$.
        In diesem Fall ist $v_0 = v_n$ implizit im Polygonzug enthalten.
    \end{note}
\end{df}

Wir betrachten nun Bewegungen eines Polygonzugs.

\begin{df}
    Für zwei Polygonzüge
    \begin{math}
        P &= (v_0, \dotsc, v_k, v_{k+1}, \dotsc, v_n) \\
        P' &= (v_0, \dotsc, v_k, v, v_{k+1}, \dotsc, v_n).
    \end{math}
    setzen wir $\Delta := [v_k, v, v_{k+1}]$ und fordern
    \begin{math}
        |P| \cap \Delta = [v_k, v_{k+1}]
        \quad \land \quad
        |P'| \cap \Delta = [v_k, v] \cup [v, v_{k+1}].
    \end{math}
    Die Modifikation von $P$ zu $P'$ (oder umgekehrt) nennen wir \emphdef[Delta-Zug]{$\Delta$-Zug}.
\end{df}

\begin{df}
    Seien $P$ und $P'$ polygonale Knoten in $\R^d$.
    Eine \emphdef{polygonale Isotopie} von $P$ und $P'$ ist gegeben durch eine endliche Folge von $\Delta$-Zügen:
    \begin{math}
        P = P_0 \leftrightarrow P_1 \leftrightarrow \dotsb \leftrightarrow P_l = P'.
    \end{math}
    Existiert eine solche polygonale Isotopie, so nennen wir $P$ und $P'$ \emphdef{polygonal isotop} oder \emphdef[Delta-äquivalent]{$\Delta$-äquivalent} und schreiben $P \sim P'$.

    Die Äquivalenzklasse $[P]$ heißt \emph{Knotentyp} von $P$.
\end{df}


\Timestamp{2015-04-20}

\begin{df}
    Ein Knoten $P = (v_1, \dotsc, v_n)$ in $\R^d$ heißt \emphdef{trivial}, wenn $P$ isotop zu einem Knoten $P' = (v_1',v_2',v_3')$ (d.h. einem Dreieck) ist.
\end{df}

\begin{prop}
    Jeder Knoten in $\R^2$ ist trivial.
    \begin{proof}
        Das ist die Aussage des (polygonalen) Satzes von Schönflies.
    \end{proof}
\end{prop}

\begin{nt}
    \begin{itemize}
        \item
            In $\R^2$ gibt es genau \emph{zwei} Äquivalenzklassen trivialer Knoten (gegeben durch entsprechende Orientierung).
        \item
            In $\R^d$ mit $d \ge 3$ sind alle trivialen Knoten äquivalent.
        \item
            In $\R^d$ mit $d \ge 4$ sind alle Knoten trivial.
    \end{itemize}
\end{nt}

\begin{ex}
    Ein (polygonaler) Knoten $P$ in $\R^3$ ist trivial genau dann, wenn er eine Scheibe berandet (d.h. es existiert eine eingebettete simpliziale Fläche $S \subset \R^3$, sodass $\Boundary S = |P|$ und $S \homeomorphic \D^2$).
    \begin{proof}
        Hinrichtung leicht, Rückrichtung wie Schönflies.
    \end{proof}
\end{ex}

\subsection{Beweglichkeit von Knoten}

\begin{st}
    Sei $P = (v_1, \dotsc, v_n)$ in $\R^d$ ein polygonaler Knoten.
    Dann existiert $\epsilon > 0$, sodass jede Folge $P' = (v_1', \dotsc, v_n')$ in $\R^d$ mit $|v_i - v_i'| < \epsilon$ einen Knoten definiert mit $P' \sim P$.

    Außerdem gilt
    \begin{enumerate}[(1)]
        \item
            $P \sim P + v$ für $v \in \R^d$,
        \item
            $P \sim \rho(P)$ für jede Drehung $\rho$,
        \item
            $P \sim \lambda P$ für $\lambda \in \R_{>0}$.
    \end{enumerate}
    \begin{note}
        $P \sim \lambda P$ für $\lambda < 0$ gilt im Allgemeinen nicht.
    \end{note}
    \begin{proof}
        Wähle
        \begin{math}
            \epsilon := \frac{1}{2} \min \Set{\Big. \Set{\big. d(v_i, v_j) & |i-j| \ge 1 } \cup \Set{\big. d([v_{i-1},v_i], [v_{j-1},v_j]) & |i-j| \ge 2}},
        \end{math}
        wobei die Indizes und der Index-Abstand $|i-j|$ zyklisch zu verstehen sind ($v_0 = v_n$).
        \begin{enumerate}[(1)]
            \item
                Wähle $n \in \N$ so groß, dass $|\frac{v}{n}| < \epsilon$.
                Betrachte $\tau: \R^3 \to \R^3, x \mapsto x + \frac{v}{n}$.
                Nun ist $P \sim \tau(P) \sim \dotsb \sim \tau^n(P) = P + v$.
            \item
                Analog.
            \item
                Ähnlich.
        \end{enumerate}
    \end{proof}
\end{st}

% todo: Beweglichkeit für orientierungserhaltende affine Abbildungen?

\begin{kor}
    Die Menge aller Knotentypen im $\R^d$ ist (höchstens) abzählbar.
    \begin{proof}
        Wir betrachten wieder polygonale Knoten mit Ecken in $\Q^d$.
        Dies sind abzählbar viele und repräsentieren alle Knotentypen dank Beweglichkeit.
    \end{proof}
\end{kor}

\begin{nt}
    \begin{enumerate}[(1)]
        \item
            Es gilt $P(v_1, v_2, \dotsc, v_n) \sim P' = (v_2, \dotsc, v_n, v_1)$.
        \item
            Aus $|P| = |P'|$ in $\R^d$ mit gleicher Orientierung folgt $P \sim P'$.
    \end{enumerate}
\end{nt}


% A2
\section{Knotendiagramme und der Satz von Reidemeister}


Betrachte die Projektion $\pi: \R^3 \to \R^2, (x,y,z) \mapsto (x,y)$.
Das Bild eines Knotens im $\R^3$ unter der Projektion $\pi$ zusammen mit Kreuzungsinformationen wollen wir Knotendiagramm nennen.

\begin{df}
    Ein Polygonzug $P = (v_1, \dotsc, v_n)$ in $\R^d$ heißt \emphdef{regulär}, wenn
    \begin{math}
        [v_{i-1},v_i] \cap [v_{j-1},v_j] \cap [v_{k-1}, v_{k}] = \emptyset
    \end{math}
    für alle $1 \le i < j < k \le n$ gilt.
\end{df}

\begin{df}
    Sei $P \subset \R^2$ ein regulärer Polygonzug.
    An jeder Kreuzung zweier Kanten lege man fest, welche oben und welche unten verläuft.
    $P$ zusammen mit dieser Kreuzungsinformation nennen wir \emphdef{Knotendiagramm}.
    \begin{note}
        Durch Einfügen von Zwischenpunkten können wir (und werden oft implizit) annehmen:
        \begin{itemize}
            \item
                Jede Kante $[v_{i-1}, v_i]$ schneidet neben $[v_{i-2}, v_{i-1}]$ und $[v_{i},v_{i+1}]$ höchstens eine weitere Kante $[v_{j-1},v_j]$ mit $|i-j| \ge 2$.
            \item
                Die Nachbarn $[v_{i-2}, v_{i-1}]$, $[v_i, v_{i+1}]$ einer jeden Kante $[v_{i-1}, v_i]$ schneiden jeweils nur ihre eigenen Nachbarn.
        \end{itemize}
    \end{note}
\end{df}

\begin{lem}
    Zu jedem Knotendiagramm $D$ existiert eine Hochhebung zu einem Knoten $P$ in $\R^3$ mit $\pi(P) = D$ und in jeder Kreuzung hat die darüberliegende Kante höhere $z$-Koordinate als die darunterliegende.

    Je zwei solcher Hochhebungen $P, P'$ zu $D$ sind isotop.

    Wir nennen $[P] = [P']$ den zu $D$ zugehörigen Knotentyp und sagen: „$D$ stellt $[P]$ dar“.
    \begin{proof}
        Handwaving
    \end{proof}
\end{lem}

Jedes Knotendiagramm stellt also genau einen Knotentyp dar.
Es stellt sich jetzt umgekehrt die Frage, wie sich die Diagramme zu zwei isotopen Knoten voneinander unterscheiden.

\subsection{Reidemeister-Züge für polygonale Diagramme im \texorpdfstring{$\R^2$}{ℝ²}}

\begin{df}
    Wir definieren auf Knotendiagrammen folgende lokale Züge (Modifikationen):
    \begin{enumerate}[(R1),start=0,leftmargin=3em]
        \item deformieren,
        \item Schleife bilden,
        \item Kante überqueren,
        \item Kreuzung überqueren.
    \end{enumerate}
    Wir nennen diese \emphdef{Reidemeister-Züge}.
    Zwei Knotendiagramme $D_1, D_2$, die sich durch eine endliche Anzahl von Reidemeisterzügen voneinander unterscheiden nennen wir R-äquivalent und schreiben $D_1 \sim D_2$.
    Unterscheiden sie sich nur durch R0-Züge, so sprechen wir von R0-Äquivalenz und schreiben $D_1 \stack{R0}\sim D_2$.
\end{df}

\begin{lem}
    Zu jedem Knotendiagramm $D = (v_1, \dotsc, v_n)$ in $\R^2$ existiert $\epsilon > 0$ sodass jedes $D' = (v_1', \dotsc, v_n')$ in $\R^2$ mit $|v_i - v_i'| < \epsilon$ ebenfalls ein Knotendiagramm ist und $D' \stack{R0}\sim D$ gilt.

    Damit gilt außerdem
    \begin{enumerate}[(1)]
        \item
            $D \stack{R0}\sim D + v$,
        \item
            $D \stack{R0}\sim \rho(D)$,
        \item
            $D \stack{R0}\sim \lambda D$ wie zuvor.
    \end{enumerate}
    \begin{proof}
        Ähnlich wie die Aussage für Knoten.
    \end{proof}
\end{lem}

\begin{lem}
    Seien $D, D'$ zwei Knotendiagramme in $\R^2$ und $P, P'$ in $\R^3$ Hochhebungen.
    Aus $D \sim D'$ folgt $P \sim P'$.
    \begin{proof}
        Die Reidemeisterzüge lassen sich in den Hochhebungen durch $\Delta$-Züge bewerkstelligen.
    \end{proof}
\end{lem}

\begin{df}
    Zu einem Knoten $P$ in $\R^3$ heißt eine Projektionsrichtung $v \in \S^2$ \emphdef{regulär}, wenn $\pi: \R^3 \to \<v\>^\orth \homeomorphic \R^2$ den Knoten $P$ regulär projiziert, d.h. $D = \pi(P)$ ist regulär, andernfalls nennen wir $v$ \emphdef{singulär}.
\end{df}

\begin{lem}
    Die Menge der singulären Richtungen zu $P$ besteht aus endlich vielen Punkten und Kurven auf der Sphäre $\S^2$.
    Ihr Komplement, die Menge aller regulären Richtungen, ist daher offen und dicht.
\Timestamp{2015-04-22}
    \begin{proof}
        Drei Fälle singulärer Richtungen $v \in \S^2$:
        \begin{enumerate}[(1)]
            \item
                Eine Kante $[v_{i-1},v_i]$ wird auf einen Punkt projiziert.
                Dies ist genau dann der Fall, wenn $v \in \R(v_i - v_{i-1}) \cap \S^2 = \Set{\pm u}$.
            \item
                Ein Punkt $v_i$ und eine Kante $[v_{j-1}, v_j]$ fallen zusammen.
                Dies ist genau dann der Fall, wenn $v \in (\R(v_{j-1} - v_i) + \R(v_j - v_i)) \cap \S^2 = \texp{Großkreis}$.
            \item
                Drei Kanten schneiden sich in einem Punkt.
                Ohne Einschränkung seien die Geraden $G_1, G_2, G_3$ paarweise windschief (sonst siehe obige Fälle).
                Zu $p_1 \in G_1$ existiert höchstens ein $p_2 \in G_2$, sodass $\_{p_1 p_2} \cap G_3 = \Set{p_3}$.
                Somit parametrisiert $p_1 \in G_1$ eine Geradenschaar, deren Richtungen eine Kurve auf $\S^2$ ergeben (explizites ausrechnen liefert eine Quadrik).
        \end{enumerate}
        In allen drei Fällen hat die Ausnahmemenge in $\S^2$ ein offenes, dichtes Komplement.
        Der endliche Schnitt dieser Komplemente ist offen und dicht.
    \end{proof}
\end{lem}

\begin{st}[Reidemeister, 1926]
    Zwei Knotendiagramme $D, D'$ in $\R^2$ stellen genau dann den selben Knotentyp in $\R^3$ dar, wenn sie Reidemeister-äquivalent sind.
    \begin{proof}
        Die erste Implikation wurde bereits gezeigt (jedes Diagramm stellt genau einen Knotentyp dar).

        Sei für die zweite Implikation also $P \sim P'$, d.h.
        \begin{math}
            \begin{tikzcd}
                P = P_0 \ar[r,mapsto,"\Delta"] & P_1 \ar[r,mapsto,"\Delta"] & P_2 \ar[r,mapsto,"\Delta"] & \dots \ar[r,mapsto,"\Delta"] & P_l = P'
            \end{tikzcd}
        \end{math}
        für $\pi: \R^3 \to \R^2$.
        Es existiert eine beliebig kleine Drehung $\rho: \R^3 \to \R^3$, sodass $\pi$ alle $\rho(P_i)$ regulär projiziert.
        Wir wählen $\rho$ so klein, dass $\pi(P_0) \stack{R0}\sim \pi(\rho(P_0))$ und $\pi(P_l) \stack{R0}\sim \pi(\rho(P_l))$.
        \begin{math}
            \begin{tikzcd}
                P = P_0 \ar[r,mapsto,"\Delta"] \ar[d,"\pi"] &
                \rho(P_0) \ar[r,mapsto,"\Delta"] \ar[d,"\pi"] &
                \rho(P_1) \ar[r,mapsto,"\Delta"] \ar[d,"\pi"] &
                \dots \ar[r,mapsto,"\Delta"] &
                \rho(P_l) \ar[r,mapsto,"\Delta"] \ar[d,"\pi"] &
                P_l = P' \ar[d,"\pi"] \\
                %
                D \ar[r,mapsto,"R0"] &
                \hat D_0 \ar[r,mapsto,"R?"] &
                \hat D_1 \ar[r,mapsto,"R?"] &
                \dots \ar[r,mapsto,"R?"] &
                \hat D_l \ar[r,mapsto,"R0"] &
                D',
            \end{tikzcd}
        \end{math}
        Der Vorteil dieser Drehung ist, dass jeder $\Delta$-Zug regulär projiziert wird.
        Nach hinreichend feiner Unterteilung der $\Delta$-Züge in kleine Dreiecke erhalten wir eine Folge elementarer Fälle.
        Betrachtet man nun die auftretenden elementaren Fälle, so stellt man fest, dass alle durch Reidemeister-Züge bewerkstelligt werden können (Skizze!).
    \end{proof}
\end{st}

\begin{ex}
    Dreifärbungen nach Fox.
    Es gilt
    \begin{math}
        |\Col_3(\Knot{0_1})| &= 3, &
        |\Col_3(\Knot{3_1})| &= 9.
    \end{math}
    Die Dreifärbung von Knotendiagrammen ist $R$-invariant und folglich eine Invariante unter den Knotentypen (man betrachte dazu die einzelnen Reidemeister-Züge).

    Insbesondere ist die Kleeblattschlinge nicht trivial.
\end{ex}

\begin{kor}
    In $\R^3$ existieren abzählbar unendlich viele Knotentypen.
    \begin{proof}
        Für $n \in \N$ betrachte die verbundene Summe von $n$ Kleeblattschlingen.
        Man erhält einen Knoten mit $3^{n+1}$ verschiedenen Dreifärbungen.
    \end{proof}
\end{kor}


\subsection{\texorpdfstring{$p$}{p}-Färbungen und Kreuzungszahl}

Sei $p \ge 3$ ungerade (später: $p$ prim).

Wir Betrachten Färbungen von $D$, d.h. $f: \Set{\text{Bögen von } D} \to \Z / p$ mit folgender Kreuzungsbedingung:
% Kreuzung, ac drunter, bd drüber
\begin{math}
    b = d
    \quad\land\quad
    a - b + c - d = 0.
\end{math}
Sei $\Col_p(D)$ die Menge dieser Färbungen.

\begin{st}
    Jede R-Äquivalenz $D \sim D'$ induziert eine Bijektion $\Col_p(D) \leftrightarrow \Col_p(D')$.
    \begin{proof}
        Man betrachte die drei elementaren Reidemeister-Züge.
    \end{proof}
\end{st}

\begin{note}
    Für $p = 3$ erhalten wir die einfache Regel, die wir aus den Dreifärbungen nach Fox bereits kennen:
    An jeder Kreuzung dürfen entweder 1 oder 3 Farben zusammenkommen.
\end{note}

\begin{ex}
    Ist der Achterknoten $\Knot{4_1}$ äquivalent zum trivialen Knoten?
    Es gilt
    \begin{math}
        \Col_3(\Knot{0_1}) &= 3, &
        \Col_5(\Knot{0_1}) &= 5, \\
        \Col_3(\Knot{4_1}) &= 3, &
        \Col_5(\Knot{4_1}) &= 25.
    \end{math}
    Mit der 3-Färbung ist noch keine Aussage möglich, erst die 5-Färbung liefert die Antwort.
\end{ex}

\begin{prop}
    Sei $p$ prim, also $\F_p$ ein Körper.
    Bezüglich der punktweisen Operationen ist $\Col_p(D)$ ein $\F_p$-Vektorraum.
    Für $n = \dim \Col_p(D)$ gilt $\Col_p(D) \isomorphic \F_p^n$ und $|\Col_p(D)| = p^n$.
\end{prop}

\begin{df}
    Für jedes Diagramm $D$ in $\R^2$ definierien wir die \emphdef{Kreuzungszahl} $\Cr(D)$ als die Anzahl der Kreuzungen in $D$.

    Sei $P$ ein polygonaler Knoten und $[P]$ ein Knotentyp.
    Die \emphdef{Kreuzungszahl} des Knotentyps $[P]$ sei $\Cr([P]) := \min_D \Cr(D)$, minimiert über alle darstellende Diagramme $D$ zu $[P]$.
\end{df}

\begin{ex}
    Alle Knoten mit $\le 2$ Kreuzungen sind äquivalent zum trivialen Knoten.
    Es gilt $\Cr(\Knot{0_1}) = 0$ und $\Cr(\Knot{3_1}) = 3$.
\end{ex}

\Timestamp{2015-04-27}

% A3
\section{Knoten und Verschlingungen, Schlingel und Zöpfe}

\begin{df}
    Ein \emphdef{topologischer Knoten} ist eine Einbettung $\_\gamma: \S^1 \injto \R^d$.
    Vermöge $\gamma(t) = \_\gamma(e^{2\pi i t})$ ist dies äquivalent zu stetigem $\gamma: \R \to \R^d$ mit $\gamma(s) = \gamma(t) \iff s - t \in \Z$.

    Wir nennen diesen Knoten \emphdef[glatter Knoten]{glatt}, wenn $\gamma \in C^1(\R, \R^d)$ und $\gamma'(t) \neq 0$ für alle $t \in \R$.
    Entsprechend definieren wir \emphdef[stückweise glatter Knoten]{stückweise glatt}.
\end{df}

\begin{df}
    Eine stetige Abbildung $h: [0,1] \times \S^1 \to \R^d$ heißt \emphdef{Homotopie}.
    Wir definieren $h_t: \S^1 \to \R^d$ durch $h_t(s) = h(t,s)$.
    Wir nennen $h$ eine \emphdef{Isotopie}, wennn $h_t$ eine Einbettung ist für jedes $t \in [0,1]$.
    Wir nennen $H: [0,1] \times \R^d \to \R^d$ eine \emphdef{Homöotopie} (oder \emphdef{ambiente Isotopie}), wenn $H_t$ ein Homöomorphismus ist für alle $t \in [0,1]$.

    Entsprechend definieren wir \emphdef[glatte Isotopie]{glatte Isotopien} und \emphdef[Diffeotopie]{Diffeotopien}.
\end{df}

\begin{note}
    Topologische Knoten können wild aussehen: betrachte die Kleeblattschlinge hintereinandergereiht, jeweils skaliert mit Faktor $q^k$, $q < 1$.
\end{note}

\paragraph{Alternative Definitionen für Knotentypen}

\[
\begin{tikzcd}
    \frac{\Set{\text{polyg. Diagramme}}}{\text{R-Züge}} \ar[r,leftrightarrow,"\cong"]& \frac{\Set{\text{polyg. Knoten}}}{\text{polyg. Isotopie}} \ar[d,leftrightarrow,"\cong"] \\
    & \frac{\Set{\text{stw. glatte Knoten}}}{\text{amb. Isotopie}} \ar[r,leftrightarrow,"\cong"] & \frac{\Set{\text{zahme Knoten}}}{\text{amb. Isotopie}} \ar[r,"\subsetneq"] & \frac{\Set{\text{topol. Knoten}}}{\text{amb. Isotopie}}\\
    \frac{\Set{\text{glatte Diagramme}}}{\text{R-Züge}} \ar[r,leftrightarrow,"\cong"] & \frac{\Set{\text{glatte Knoten}}}{\text{glatte Isotopie}} \ar[u,leftrightarrow,"\cong"]
\end{tikzcd}
\]


\begin{enumerate}[1)]
    \item
        Es existieren wilde Knoten (topologisch, aber nicht zahm).
        Übung: Es gibt überabzählbar viele wilde Knotentypen.
    \item
        Isotopie ist nicht der richtige Äquivalenzbegriff: Knoten können sich unter Isotopie auflösen (siehe Übung).
\end{enumerate}

\begin{st}
    Für topologische Knoten $\_\gamma: \S^1 \injto \R^3$ sind äquivalent:
    \begin{enumerate}[(1)]
        \item
            $\_\gamma$ ist ambient isotop zu einem polygonalen Knoten, d.h. es existiert eine Homöotopie $H: [0,1] \times \R^3 \to \R^3$ mit $H_0 = \id_{\R^3}$ und $H_1 \circ \_\gamma$ polygonal ist.
        \item
            $\_\gamma$ erlaubt eine Tubenumgebung $F: \S^1 \times \B^2 \injto \R^3$ mit $\_\gamma(s) = F(s,0)$.
        \item
            $\_\gamma$ ist lokal flach, d.h. lokal homöomorph zu $\R^1 \injto \R^3, x \mapsto (x,0,0)$.
    \end{enumerate}
    \begin{proof}
        $(1) \implies (2) \implies (3)$ ist klar.
        $(3) \implies (1)$ ist mühsam und soll hier nicht ausgeführt werden.
    \end{proof}
\end{st}

\begin{st}
    Für polygonale Knoten $K = |P|$ und $K' = |P'|$ sind äquivalent:
    \begin{enumerate}[(a)]
        \item
            $P$ und $P'$ sind polygonal isotop ($\Delta$-Züge).
        \item
            Es existiert eine Homöotopie $H_t: [0,1] \times \R^3 \to \R^3$ mit $H_0 = \id_{\R^3}$ und $H_1(K) = K'$ (ambiente Isotopie).
        \item
            Es existiert ein orientierungserhaltener Homöomoorphismus $h: \R^3 \homto \R^3$ mit $h(K) = K'$.
    \end{enumerate}
    \begin{proof}
        (a) $\implies$ (b) ist leicht, sogar mit einer stückweise affinen Homöotopie.
        (b) $\implies$ (c) ist trivial mit $H_1$ (orientierungserhaltend, da von $\id_{\R^3}$ ausgegangen).
        (c) $\implies$ (a) ist mühsam, siehe Burde-Zieschang. % fixme: ref
    \end{proof}
\end{st}

\begin{st}
    Für topologische Knoten $\_\gamma: \S^1 \to \R^3$ sind äquivalent:
    \begin{enumerate}[(1)]
        \item
            $\_\gamma$ ist glatt.
        \item
            $\_\gamma$ erlaubt eine glatte Tubenumgebung.
        \item
            $\_\gamma$ ist lokal diffeomorph zu $\R^1 \injto \R^3, x \mapsto (x,0,0)$.
            $\_\gamma(\S^1) \subset \R^3$ ist eine glatte Untermannigfaltigkeit und $\_\gamma: \S^1 \to \_\gamma(\S^1)$ ein Diffeomorphismus glatter Mannigfaltigkeiten.
    \end{enumerate}
\end{st}

\begin{st}[Isotopie-Einbettungssatz von Thom 1957]
    Seien $M, N$ glatte Mannigfaltigkeiten und $h: [0,1] \times M \to N$ eine glatte Isotopie, die außerhalb eines Kompaktums $M_0 \subset M$ nichts bewegt.
    Dann kann man $h$ in eine Diffeotopie $H: [0,1] \times N \to N$ einbetten, die außerhalb eines Kompaktums $N_0 \subset N$ nichts bewegt.
    Das bedeutet $H_0 = \id_N$ und $h_t = H_t \circ h_0$ für alle $t \in [0,1]$.
    \begin{proof}
        Mittels Differentialtopologie, siehe Bröcker-Jänich oder Hirsch. % fixme: ref
    \end{proof}
\end{st}

Es lassen sich auch Knoten betrachten, die nicht in $\R^3$, sondern auf einer anderen $3$-Mannigfaltigkeit $M$, etwa in $\S^3$ oder $\S^1 \times \D^2$ (Torus), eingebettet sind.

\begin{ex}
    Sei $F$ eine Fläche (z.B. kompakt, zusammenhängend, $F \homeomorphic F_{g,r}^\pm$).
    Wähle $M := F \times [-1,1]$.
\end{ex}

\begin{df}
    Wir nutzen die Notation
    \begin{math}
        \scr K(M) = \frac{\Set{\text{zahme/glatte/polygonale Knoten $\S^1 \injto M$}}}{\text{ambiente Isotopie}}.
    \end{math}
\end{df}

Übung: $\R^3 \injto \S^3$ induziert $\scr K(\R^3) \homto \scr K(\S^3)$, aber $\R^2 \injto \S^2$ induziert $\scr K(\R^2) \to \scr K(\S^2)$ surjektiv, nicht injektiv.

\paragraph{Verschlingungen}

\begin{ex}
    \begin{itemize}
        \item
            $\Link{0_1^2}$: Triviale Verschlingung, getrennte Kreise.
        \item
            $\Link{2_1^2}$: Hopf-Verschlingung $H_+, H_-$.
        \item
            $\Link{5_1^2}$: Whitehead-Verschlingung $W$.
        \item
            $\Link{6_2^3}$: Borromäische Ringe $B$.
    \end{itemize}
\end{ex}

Die Behandlung von Verschlingungen erfolgt analog zu den Knoten.
Inbesondere gilt der Satz von Reidemeister für Verschlingungen.


\begin{df}
    Wir definieren die Verschlingungszahl $\lk(L_i, L_j)$ für eine Verschlingung $L = L_1 \cup \dotsb \cup L_n$.
    % fixme: \eps(p) definieren!
    \begin{math}
        \lk(L_i,L_j) = \frac{1}{2} \sum_{p\in D_i \cap D_j} \eps(p).
    \end{math}
    \begin{note}
        Die Definition ist R-invariant auf dem Diagramm und daher von der Wahl des Diagramms unabhängig.
        Somit ist die Verschlingunszahl eine Eigenschaft von $L$.
    \end{note}
\end{df}

\begin{ex}
    Die Hopf-Verschlingung ist nicht trennbar, denn es gilt
    \begin{math}
        \lk(H_-) &= -1, &
        \lk(H_+) &= +1,
    \end{math}
    während $\lk(\Link{0_1^2}) = 0$.
\end{ex}

Übung: Kann $W$ getrennt werden? und $B$? (Dreifärbungen)

\subsection{Schlingel}

Sei $F := \R^2$, $M = F \times [0,1]$.
Wir betrachten nun nicht mehr geschlossene Knoten in $M$, sondern lassen auch offene Knoten (mit Endpunkten auf $F \times \Set{0,1}$) zu.

Der Rest bleibt wie zuvor: polygonale Einbettungen und Isotopie.
Geeignet formuliert gilt auch hier dier Satz von Reidemeister.
Vorteil: verbundene Summe (aneinanderhängen).


\Timestamp{2015-04-29}

Seien $n, m \in \N$ und $A$ die Menge der Ein-, Ausgänge mit $|A| = 2m$.
Betrachte zahme Einbettungen
\begin{math}
    f: ([0,1] \times \Set{1, \dotsc, m}) \cup (\S^1 \times \Set{m+1, \dotsc, m+n}) \injto M
\end{math}
($m$ Stränge, $n$ geschlossene Knoten).
$T := \im(f) \subset M$ ist eine orientierte Untermannigfaltigkeit.
Es soll gelten $\Boundary T = T \cap \Boundary M = A$.

Sei $F$ nun eine spezielle Fläche, z.B. $\R^2, \D^2, F_{g,r}^\pm$ und setze $M := F \times [0,1]$ mit $A = A_0 \times \Set{0} \cup A_1 \times \Set{1}$ und $A_0, A_1 \subset F$.

Wir setzen
\begin{math}
    \scr T_F(A_0,A_1) := \frac{\Set{\text{Schlingel in $(M, A)$}}}{\text{ambiente Isotopie}}.
\end{math}
Dies bildet eine Kategorie bezüglich der Verknüpfung $\#: \scr T_F(A_0,A_1) \times \scr T_F(A_1, A_2) \to \scr T_F(A_0, A_2)$.
Genau:
\begin{math}
    \Phi: M \sqcup M = M \times \Set{1,2} &\to M \\
    (x,t,1) & \mapsto (x, \frac{t}{2}) \\
    (x,t,2) & \mapsto (x, \frac{t+1}{2})
\end{math}
Für zwei Schlingel $T_1 \subset (M, A_0 \cup A_1), T_2 \subset (M, A_1 \cup A_2)$ setze
\begin{math}
    T := T_1 \# T_2 :=  \Phi(T_1 \times \Set{1} \cup T_2 \times \Set{2})
    \subset (M, A_0, A_1).
\end{math}

\begin{note}
    Die Verknüpfung ist wohldefiniert modulo Isotopie von $(M, \Boundary M)$ und wir erhalten eine Kategorie: Objekte sind $A \subset F$, Morphismen sind Schlingeltypen mit der Verknüpfung $\#$.

    Werden orientierte Schlingel betrachtet, so muss die Orientierung bei der Verklebung beachtet werden.
\end{note}

Sei nun speziell $F = \R^2$, $p,q \in \N$ mit $p + q = 2m$ und
\begin{math}
    A_0 &= \Set{(1,0), (2,0), \dotsc, (p,0)}, \\
    A_1 &= \Set{(1,0), (2,0), \dotsc, (q,0)}
\end{math}
Setze $\scr T(p,q) := \scr T_{\R^2}(A_0, A_1)$.
Für $p = q = m$ definieren wir einen \emphdef[reiner Schlingel]{reinen Schlingel}:
es besteht aus $m$ Intervallen, jedes läuft von $(i,0)$ nach $(i,1)$, keine geschlossenen Knoten, d.h. $n = 0$.

Die Schlingel $\scr T(m,m)$ bilden einen Monoid, die reinen Schlingel $\scr T_{\text{rein}}(m,m)$ bilden einen Untermonoid.
Gibt es invertierbare Element in diesem Monoid? Ja: Zöpfe.

\subsection{Zöpfe}

\begin{df}
    Ein Zopf mit $n$ Strängen ist ein strikt monotoner (bzgl. Zeitachse) Schlingel in $\R^2 \times [0,1]$ von $A_0$ nach $A_1$ ($A_0 = A_1 = \Set{(1,0), \dotsc, (n,0)}$).
\end{df}

\begin{note}
    Jeder Zopf ist invertierbar in $\scr T(n,n)$ vermöge der Spiegelung $\Psi: F \times [0,1] \to F \times [0,1]: (x,t) \mapsto (x, 1-t)$.
\end{note}

\begin{note}
    Für Zöpfe gibt es zwei Arten von Isotopien: Isotopien in der Menge der Zöpfe und Isotopien von Zopf zu Zopf in der Menge der Schlingel.
    Diese sind sogar äquivalent (später)
\end{note}

\begin{st}[Artin, 1925]
    Zöpfe in $\R^2 \times [0,1]$ wie oben bilden eine Gruppe $B_n$.
    Diese wird erzeugt von den \emphdef[elementare Zöpfe]{elementaren Zöpfen}: Verdrillung zweier benachbarter Stränge $s_i, s_i^{-1}$.
    Neben
    \begin{enumerate}[i)]
        \item[0)]
            $s_i s_i^{-1} = s_i^{-1} s_i = 1$ (R2)
    \end{enumerate}
    gelten die elementaren Relationen
    \begin{enumerate}[i),resume]
        \item
            $s_i s_{i+1} s_i = s_{i+1} s_i s_{i+1}$ für $i = 1, \dotsc, n-2$ (R3),
        \item
            $s_i s_j = s_j s_i$ für $|i-j| \ge 2$ (R0).
    \end{enumerate}
    Die elementaren Relationen erzeugen alle Relationen, d.h.
    \begin{math}
        B_n = \Gen{ s_1, \dotsc s_{n-1} & \begin{aligned} s_is_js_i &= s_js_is_j \text{ für $|i-j| = 1$}, \\ s_i s_j &= s_j s_i \text{ für $|i-j| \ge 2$} \end{aligned} }.
    \end{math}
    \begin{proof}
        Wir verfahren wie im Beweis des Satzes von Reidemeister für Knoten, statt Knotendiagramme haben wir nun Elemente aus $B_n$.
        \begin{enumerate}[i)]
            \item
                polygonale Zöpfe, polygonale Isotopie zwischen Zöpfen
            \item
                reguläre Projektion (dann erzeugen $s_1, \dotsc, s_{n-1}$ die Gruppe),
            \item
                $\Delta$-Züge entsprechen R0, R2, R3 (hier: R1 nicht möglich), also erzeugen die elementaren Relationen alle Relationen.
        \end{enumerate}
        Übung: Wiederholung und Anpassung.
    \end{proof}
\end{st}

\begin{note}
    Wir haben einen Gruppenhomomorphismus von $B_n$ auf die symmetrische Gruppe $S_n$ (alle Permutationen von $n$ Elementen).
    Dazu bilde man $s_i$ auf die elementare Vertauschung $(i, i+1)$ ab.
    Es gilt
    \begin{math}
        S_n = \Gen{ \sigma_1, \dotsc, \sigma_{n-1}, \sigma_i = (i,i+1)
            & \begin{aligned} \sigma_i \sigma_j \sigma_i &= \sigma_j \sigma_i \sigma_j \text{ für $|i-j| = 1$}, \\
                \sigma_i \sigma_j &= \sigma_j \sigma_i  \text{ für $|i-j| \ge 2$}, \\
                \sigma_i^2 &= 1.
            \end{aligned}
        }
    \end{math}
    Übung!
\end{note}

Sei $P_n$ die reine Zopfgruppe auf $n$ Strängen, $B_n$ die Zopfgruppe und $S_n$ die symmetrische Gruppe.
Artin liefert
\begin{math}
    P_n \injto B_n \stack{\pi_n} \surto S_n.
\end{math}
Dies ist eine kurze exakte Sequenz.
Wir können Stränge hinzufügen und vergessen:
\begin{math}
    \iota_n: B_{n-1} &\injto B_n, \\
    r: P_n &\to P_{n-1}
\end{math}
Dies sind Gruppenhomomorphismen.
Es gilt $\tau_{n-1} \circ \iota_{n-1} = \id_{P_{n-1}}$.
Der Kern von $r_n: P_n \to P_{n-1}$ wird frei erzeugt von $a_{1,n}, \dotsc, a_{n-1,n}$:
$a_{ij}$ für $i < j$ geht über die nächsten $j - i$ Stränge, unter $j$ und über $i+1, \dotsc, j-1$.
Es gilt
\begin{math}
    F_{n-1} := \ker(r_n) = \pi_1(\R^2 \setminus \Set{n-1 \text{ Punkte}}, *).
\end{math}
Wir haben
\begin{math}
    F_{n-1} \injto P_n \surto P_{n-1}
\end{math}
spaltend, d.h. $i_{n-1}: P_{n-1} \to P_n$.

% semidirect ><|
Anders formuliert $P_n = F_{n-1} \semidirect P_{n-1}$.

\begin{ex}
    $P_1 = \Set{1}$, $P_2 = \<s_1^2\> = F_1 \isomorphic \Z$, $P_3 = F_2 \semidirect F_1$ und
    \begin{math}
        P_n = F_{n-1} \semidirect (F_{n-2} \semidirect ( \dotso ( F_2 \semidirect F_1) \dotso )).
    \end{math}
\end{ex}

Übung: es gilt
\begin{math}
    Z(S_n) = \begin{cases}
        S_2 \isomorphic \Z / 2 & \text{für $n = 2$}, \\
        1 & \text{für $n \neq 2$}.
    \end{cases}\quad
    Z(F_n) = \begin{cases}
        F_1 \isomorphic \Z & \text{für $n=1$}, \\
        1 & \text{für $n \neq 1$}.
    \end{cases}
\end{math}
außerdem:
Sei $h: G \surto H$ ein surjektiver Gruppenhomomorphismus.
Dann gilt $h(Z(G)) \subset Z(H)$.

\begin{st}
    Im Zentrum von $B_n$ liegt der reine Zopf
    \begin{math}
        z_n = (s_1s_2 \dotso s_{n-1})^n
    \end{math}
    (eine volle Drehung).
    Dieser erzeugt das Zentrum, genauer:
    \begin{math}
        Z(P_n) &= \<z_n\> \isomorphic \Z, \\
        Z(B_n) &= \begin{cases}
            B_2 \isomorphic \Z & \text{für $n = 2$}, \\
            \<z_n\> \isomorphic \Z & \text{für $n \neq 2$}.
        \end{cases}
    \end{math}
    \begin{proof}
        Es gilt $z_n s_i = s_i z_n$ für $i = 1, \dotsc, n-1$ (entweder nachrechnen, oder argumentieren, da $z_n$ volle Drehung).
        Also ist $z_n \in Z(B_n)$.
        \begin{enumerate}[(1)]
            \item
                Zeige $Z(P_n) = \<z_n\>$.
                $\supset$ ist klar.
                Es verbleibt $\subset$.
                Für $n=1$ gilt $P_1 = \Set 1$ und $Z(P_1) = \Set{1} = \<z_1\>$.
                Für $n=2$ gilt $P_1 = \<s_1^2\> = \Z$ und $Z(P_2) = \<s_1^2\> = \<z_2\>$.

                Zeige nun $Z(P_n) \subset \<z_n\>$ für $n \ge 3$ per Induktion.
                Sei hierzu $z \in Z(P_n)$.
                Unter $r_n: P_n \to P_{n-1}$ gilt $r_n(z_n) = z_{n-1}$.
                Wir haben $r_1(z) \in Z(P_{n-1}) = \<z_{n-1}\>$, also $r_n(z) = z_{n-1}^k$ für $k \in \Z$.
                Setze $z' := z z_n^{-k}$ und erhalte $r_n(z') = 1$.
                Für $z' \in \ker(r_n) = F_{n-1}$ und $z' \in Z(P_n)$ gilt $z' \in \Z(F_{n-1}) = \Set{1}$, d.h. $z' = 1$ und $z = z_n^k$.
            \item
                $Z(B_2) = B_2$ ist klar.
                Für $n \ge 3$ betrachte $\pi: B_n \surto S_n$.
                Es folgt
                \begin{math}
                    \pi(Z(B_n)) \subset Z(S_n) = \Set{1},
                \end{math}
                also $Z(S_n) \subset P_n$ und $Z(B_n) \subset Z(P_n) = \<z_n\>$.
        \end{enumerate}
    \end{proof}
\end{st}

\begin{math}
    B_n &= \Gen{s_1, \dotsc, s_n & \begin{aligned}
        s_i s_j &= s_j s_i && \text{ für $|i-j| \ge 2$}, \\
        s_is_js_i &= s_j s_i s_j && \text{ für $|i-j| = 1$}.
    \end{aligned}} \\
    (B_n)_{\text{ab}} &= \Gen{s_1, \dotsc, s_n & \begin{aligned}
        s_i s_j &= s_j s_i && \text{ für alle $i,j$}, \\
        s_is_js_i &= s_j s_i s_j && \text{ für $|i-j| = 1$}.
    \end{aligned}} \isomorphic \<s_1\> \isomorphic \Z, \\
    S_n &= \Gen{t_1,\dotsc, t_n & \begin{aligned}
        t_i t_j &= t_j t_i && \text{ für $|i-j| \ge 2$}, \\
        t_it_j t_i &= t_j t_i t_j && \text{ für $|i-j| = 2$}, \\
        t_i^2 &= 1.
    \end{aligned}} \\
    (S_n)_{\text{ab}} &\isomorphic \Z / 2.
\end{math}
Die Abelschmachung von $S_n$ ist der einzige Gruppenhomomorphismus $S_n \to \Z /2$ und entspricht der Signatur $\sign: S_n \surto \Z / 2$ einer Permutation.

\begin{kor}
    Die Zopfgruppen $B_n$ sind nicht untereinander isomorph, denn in der Abelschmachung $\alpha_n: B_n \to \Z$ gilt $\alpha_n(Z(B_n)) =  n(n-1) \Z$.
\end{kor}


\subsection{Verbundene Summe von Knoten}

Bilde den Abschluss $\cl: \scr T(n,n) \to \scr L = \scr L(\R^3)$, welcher Schlingel auf Verschlingungen abbildet, indem die Ein-/Ausgänge entsprechend miteinander verbunden werden.

$\cl$ ist wohldefiniert modulo Isotopie.
\begin{st}
    Speziell ist $\cl: \scr T_{\text{rein}}(1,1) \to \scr K = \scr K(\R^3)$ ist Bijektion.
    \begin{proof}
        Stelle Knoten im $\R^3$ polygonal dar, dann als Diagramme, markiere einen Punkt.
        Schneide an diesem Punkt auf und Verbinde beide Enden zum Ein-, bzw. Ausgang.
        Dies bildet einen Schlingel und zeigt damit die Surjektivität.
        Genauer haben wir eine Umkehrabbildung $\op: \scr K \to \scr T_{\text{rein}}(1,1)$.

        Konstruktion $\scr D^\cdot \to \scr T_{\text{rein}}(1,1)$, $D \mapsto T$ wie oben.
        R-Züge auf $D$ weg von Markierung sind $\Delta$-Züge für $T$.
        Sonst: Verschieben der Markierung über Kreuzungen möglich (Invarianz unter $\op$).
        Damit ist $\op: \scr K \to \scr T_{\text{rein}}(1,1)$ wohldefiniert.
        Wir sehen $\cl \circ \op = \id_{\scr K}$ und $\op \circ \cl = \id_{\scr T}$.
    \end{proof}
\end{st}

Damit erklären wir die verbundene Summe:
\begin{math}
    \scr K \times \scr K &\stack{\#}\to \scr K, \\
    \scr T_r(1,1) \times \scr T_r(1,1) &\stack{\#}\to \scr T_r(1,1)
\end{math}
Die Definition ist unabhängig von der Wahl des Punktes.


\Timestamp{2015-05-04}

\begin{note}
    Es gilt $\Cr([K] \# [L]) \le \Cr([K]) + \Cr([L])$.
    Ein offenes Problem ist die Frage nach Gleichheit, könnte die verbundene Summe auch mit weniger Kreuzungen auskommen?
\end{note}


\subsection{Brückenzahl}

Betrachte in einem Knotendiagramm Minima und Maxima, $h: \R^2 \times [0,1] \to [0,1]$.
Wir definieren $\br: \scr K \to \N$ durch
\begin{math}
    \br([K]) = \min_{K' \sim K} |\Set{\text{Minima von $K'$}}|.
\end{math}
Etwas allgemeiner: $\br: \scr T(1,1) \to \N$
\begin{math}
    \br([T]) = \min_{T' \sim T} |\Set{\text{Minima im Inneren von $T'$}}|.
\end{math}
\begin{note}
    Es gilt $\br([\cl(T)]) = 1 + \br(T)$.
\end{note}

\begin{ex}
    Es gilt
    \begin{math}
        \br(\texp{trivialer Knoten}) &= 1, &
        \br(\texp{Kleeblattschlinge}) &= 2.
    \end{math}
    Für die Kleeblattschlinge ist $\le$ klar, dank der Zeichnung.
    $\ge$ ist klar dank folgendere Bemerkung:
\end{ex}

\begin{note}
    Für $[K] \in \scr K$ gilt $br([K]) = 1 \iff K \sim \texp{trivialer Knoten}$.
    \begin{proof}
        $\impliedby$ ist klar.
        Für $\implies$ stelle $K$ durch ein Diagramm $D$ mit $\Cr(D) = 1$ dar.
        Dann ist $D \sim \texp{trivialer Knoten}$.
    \end{proof}
\end{note}

\begin{st}
    Für $[K] \in \scr K$ gilt $1 \le \dim( \Col_p(K)) \le \br(K)$.
    \begin{proof}
        Der Knoten habe $b$ Minima und Maxima.
        Der Teil zwischen Minima und Maxima bildet einen Zopf.
        Wähle Eingangsfarben aus $\F_p^b$.
        Diese propagieren sich nach rechts.
        Am Ende liefert jedes Maximum eine Gleichung.
        Die Anzahl Lösungen des Gleichungssystems sind höchstens $b$ und mindestens $1$ (wegen der trivialen Färbung).
    \end{proof}
\end{st}

\begin{ex}
    Für die $n$-fache verbunde Summe der Kleeblattschlinge $K$ gilt
    \begin{math}
        \br(K) = n + 1.
    \end{math}
    \begin{proof}
        $\le$ ist klar dank der Zeichnung.
        $\ge$ gilt dank $\dim \Col_p = n + 1$.
    \end{proof}
\end{ex}

\begin{st}[Schubert 195x]
    Es gilt
    \begin{math}
        \br(K \# L) = \br(K) + \br(L) - 1.
    \end{math}
    \begin{note}
        Schöner ist für offene Knoten
        \begin{math}
            \br(K \# L) = \br(K) + \br(L).
        \end{math}
    \end{note}
    \begin{proof}
        $\le$ ist klar.
        $\ge$ ist mühsam (hier nicht ausgeführt).
    \end{proof}
\end{st}

\begin{df}
    Ein Knotentyp $K \in \scr K$ heißt \emphdef{prim}, wenn aus $K = A \# B$ folgt: $A \sim \texp{trivialer Knoten}$ oder $B \sim \texp{trivialer Knoten}$.
\end{df}

\begin{kor}
    Jeder Knoten ist eine verbundene Summe von Primknoten.
    \begin{proof}
        Induktion über $n = \br(K)$.
        Für $n = 1$ ist $K$ der triviale Knoten, also leere Summe.
        Für $n = 2$ ist der Knoten prim.
        Sei also $n \ge 3$.
        \begin{enumerate}[i)]
            \item
                Ist $K$ prim, so sind wir fertig.
            \item
                Ist $K = A \# B$ und $A \not\sim \texp{trivialer Knoten} \not\sim B$, dann ist
                \begin{math}
                    \underbrace{\br(K) -1}_{=n} = \underbrace{\br(A) - 1}_{\ge 1} + \underbrace{\br(B) - 1}_{\ge 1}.
                \end{math}
                Es folgt $\br(A), \br(B) < \br(K) = n$.
                Per Induktion sind $A$ und $B$ verbundene Summe aus Primknoten und somit auch $K$.
        \end{enumerate}
    \end{proof}
    \begin{note}
        Es gilt sogar die Eindeutigkeit dieser Zerlegung, siehe später. % fixme: ref
    \end{note}
\end{kor}

\subsection{Entknotungszahl}


Für Knoten $K, K'$ definieren wir die Entknotungszahl
\begin{math}
    u(K,K') &:= \min |\Set{\text{Kreuzungswechsel von $K$ nach $K'$}}|, \\
    u(K) &:= u(K, \texp{trivialer Knoten}).
\end{math}

\begin{ex}
    \begin{itemize}
        \item
            Es gilt $u(K,K') = 0$ genau dann, wenn $K \sim K'$.
        \item
            Die Entkreuzungszahl der Kleeblattschlinge ist $1$.
        \item
            Die Entkreuzungszahl der Verbundenen Summe zweier Kleeblattschlingen ist $2$.
            \begin{proof}
                $\le$ ist klar.
                $\ge$ mit folgendem Satz: $f = 3$, also $u = 2$.
            \end{proof}
    \end{itemize}
\end{ex}

\begin{st}
    Sei $f: \scr K \to \N$, $f(K) = \dim \Col_p(F)$.
    Bei jeder Entknotungsoperation von $K_+$ zu $K_-$ gilt
    \begin{math}
        |f(K_+) - f(K_-)| \le 1.
    \end{math}
    \begin{proof}
        Betrachte drei Diagramme, die sich nur an einer Stelle unterscheiden.
        $D_+, D_0, D_-$ (jeweils Kreuzungen, $D_+$ drüber, $D_-$ drunter, $D_0$ beliebig).
        Es gelten die Relationen für die Stränge
        \begin{math}
            D_+&: & a - b +c - d &= 0,  & a &= c,\\
            D_0&: & a - b + c - d &= 0, \\
            D_-&: & a - b + c - d &= 0, & b &= d,
        \end{math}
        Es gilt $\Col_p(D_+) \subset \Col_p(D_0) \supset_p(D_-)$.
        Die Untervektorräume haben jeweils Kodimension 1, folglich unterscheiden sich beide Untervektorräume in der Dimension um höchstens 1.
    \end{proof}
\end{st}

\begin{kor}
    Die $n$-fach verbundene Summe von Kleeblattschlingen besitzt Entknotungszahl $n$.
\end{kor}

\begin{note}
    Es gilt $u(K \# L) \le u(K) + u(L)$, die Umkehrung ist eine offene Frage.
\end{note}

Wir hatten $\Cr, u, \br$ auf Additivität hin geprüft. Die Brückenzahl $\br$ war additiv für Knoten, aber nicht für Tangles:
betrachte das Gegenbeispiel (siehe Skizze):
Es gilt $\br(A) = 1$ ($\le$ ist klar, $\ge$: zeige $A$ ist kein Zopf) und $\br(B) = 1$ ($\le$ ist klar, $\ge$: zeige $B$ ist kein Zopf).
Bei der Verbundenen Summe löscht sich ein Min-Max-Paar aus.




