% Kap C
\chapter{Primfaktorzerlegung von Knoten}

\Timestamp{2015-06-01}

% §C1
\section[Der Satz von Alexander-Schönflies]{Der Satz von Alexander-Schönflies für glatte Einbettungen \texorpdfstring{$\S^2 \injto \R^3$}{𝕊² ↪ ℝ³}}

Bewegung von Bällen.
Betrachte Einbettungen $f, g: \D^n \injto \R^n$.

Sind $f, g$ isotop?

\begin{st}[Milnor, Plais, Cerf, ≈1960]
    Sei $M = \R^n$ oder allgemeinere eine zusammenhängende, glatte $n$-Mannigfaltigkeit ohne Rand.
    Seien $f,g : \D^k \injto M$ glatte Einbettung, $0 \le k \le n$,
    Falls $k = n \ge 1$ fordern wir zusätzlich, dass entweder $M$ nicht orientierbar ist, oder, dass $f, g$ das selbe Orientierungsverhalten haben.

    Dann existiert eine glatte Isotopie von $f$ nach $g$, d.h. es existiert $H: [0,1] \times \D^k \to M$ glatt, $H_0 = f$, $H_1 = g$ und $H_t$ eine glatte Einbettung für alle $t \in [0,1]$.
    \begin{note}[Zusatz: Thom Isotopiefortsetzungssatz]
        Es existiert $\tilde H: [0,1] \times M \to M$ glatt, $H_0 = \id_M$, $H_1 \circ f = g$ und für alle $t \in [0,1]$ ein Diffeomorphismus $H_t: M \to M$.
    \end{note}
    \begin{proof}
        Siehe Hirsch, Thm. 8.3.1
    \end{proof}
\end{st}

\begin{ex}
    Ein glatter Knoten $\kappa: \S^1 \injto \R^3$ ist genau dann glatt isotop zum trivialen Knoten $\S^1 \subset \R^2 \subset \R^3$, wenn $\kappa$ einen glatten $2$-Ball berandet, d.h. es existiert $f: \D^2 \injto \R^3$ glatt mit $f|_{\S^1} = \kappa$.
    \begin{proof}
        \begin{seg}{\ProofImplication}
            $\S^1$ berandet $\D^2$, die Isotopie von $\S^1$ nach $\kappa$ überführt dies zu einer glatten Scheibe.
        \end{seg}
        \begin{seg}{\ProofImplication*}
            Mit obigem Satz.
        \end{seg}
    \end{proof}
    \begin{note}
        Übung: selbiges gilt im polygonalen Fall.
    \end{note}
\end{ex}

\begin{nt}
    Jede Einbettung $f: \S^0 = \Set{\pm 1} \injto \R^1$ zerlegt $\R^1$ in drei Komponenten
    \begin{math}
        \R \setminus f(\S^0) = A \sqcup B \sqcup C
    \end{math}
    mit $B$ beschränkt und $B \homeomorphic \B^1$, $\_B \homeomorphic \D^1$.
\end{nt}

\begin{st}[Jordan-Schönflies]
    Jede topologische/polygonale/glatte Einbettung $f: \S^1 \injto \R^2$ zerlegt $\R^2$ in zwei Komponenten, $\R^2 \setminus f(\S^1) = A \sqcup B$ mit $A, B$ offen und zusammenhängend, $A$ unbeschränkt, $B$ beschränkt.
    Es existiert ein Homöomorphismus/polygonaler Homoömorphismus/Diffeomorphismus $h: \R^2 \to \R^2$ mit $h \circ f(\S^1) = \S^1$.
    \begin{proof}
        Der topologische Fall ist sehr aufwändig und wird ausgelassen, für den polygonalen Fall siehe Vorlesung Topologie.
        Zeige die aussage für den glatten Fall:

        Sei $f:\S^1 \injto \R^2$ glatte Einbettung, zeige $f$ berandet eine glatte Kreisscheibe, d.h. $F: \D^2 \injto \R^2$, $F|_{\S^1} = f$.

        Betrachte die Höhenfunktion $h: \R^2 \to \R$, $h(x,y) = y$.
        Wir nennen $x \in \S^1$ \emph{regulär} (1), wenn $(h \circ f)'(x) \neq 0$ gilt und \emph{kritisch} (2), wenn $(h \circ f)'(x) = 0$.
        Wir unterscheiden \emphdef{regulär kritisch} (2a), wenn lokal $(h \circ f)(x) \sim \pm x^2$ und \emph{singulär kritisch} (2b), wenn $(h \circ f)'(x) = (h \circ f)'' = 0$.

        Wir nennen $h \circ f: \S^1 \to \R$ eine \emphdef{Morse-Funktion}, wenn nur die Fälle (1) und (2a) auftreten.

        Damit gilt: Jeder kritische Punkt ist Minimum oder Maximum, diese liegen isoliert.
        Da $\S^1$ kompakt ist, existieren nur endlich viele kritische Punkte.
        Wir können sie auf verschiedene Höhen annehmen.
        \begin{st}[Morse]
            „Nach beliebig kleiner Isotopie ist $f$ von dieser Form“ (Morse-Einbettungen liegen dicht).
        \end{st}

        Sei $t \in \R$ ein regulärer Wert.
        Dann schneidet $\R \times \Set{t}$ die Einbettung $f$ transversal in endlich vielen Punkten.
        Es existiert ein Nachbarpaar (spezifizieren!) solcher Schnittpunkte mit Durchstößen in entgegengesetzter Richtung.
        Führe Chirurgie aus: Schneide und Verbinde so, dass die Durchstöße verschwinden.

        Per Induktion nach der Anzahl der Übergänge erhalten wir schließlich eine Kollektion von glatt eingebetteten Kreislinien.

        Behauptung: Jede Kreislinie in $\R \times (t,\infty)$ berandet eine glatte Kreisscheibe.
        \begin{enumerate}[1),start=0]
            \item
                Für $t > \max(h \circ f)$ ist dies trivial.
            \item
                Gilt dies für $t$ und enthält $[t', t]$ keine kritische Werte, so gilt dies auch für $t'$.
            \item
                Für die Passage eines Maximums betrachte die Chirurgie: eine Kreisscheibe entsteht.
            \item
                Für die Passage eines Minimums betrachte die Chirurgie: eine Kreisscheibe verschwindet.
                Beachte mehrere Fälle: Paare können verschieden verbunden sein bei Chirurgie (einzelne Kreisscheibe, zwei Kreisscheiben, verschachtelte Kreisscheiben).
        \end{enumerate}
        Per Induktion erreichen wir schließlich $t < \min(h \circ f)$.
    \end{proof}
\end{st}

\begin{st}[Alexander-Schönflies, ≈1920]
    Jede glatte (!) Einbettung $f: \S^2 \injto \R^3$ berandet einen glatten $3$-Ball $F: \D^3 \injto \R^3$, $F|_{\S^2} = f$.
    \begin{proof}
        Nach beliebig kleiner Isotopie von $f$ ist $h \circ f: \S^2 \to \R$ eine Morse-Funktion, genauer: $h: \R^3 \to \R$, $h(x,y,z) = z$.
        Jeder kritische Punkt ist entweder Maximum (lokal: $z = -x^2-y^2$) oder Minimum (lokal: $z = x^2 + y^2$) oder Sattelpunkt (lokal $z = x^2 - y^2$).

        Sei $t \in \R$ ein regulärer Wert, $S := \im f$.
        Wir führen Chirurgie entlang der Ebene $E_t: \R^2 \times \Set t$ durch und erhalten $S_t$.
        $E_t$ schneidet $f$ in endlich vielen glatten Kreislinien (Satz über implizite Funktion lokal, Klassifikation der $1$-Mannigfaltigkeiten global).
        Jede dieser Kreislinien berandet eine Scheibe (siehe voriger Satz).
        Wähle eine „innerste Kreisscheibe“ und führe Chirurgie aus.
        Nach endlich vielen Schritten sind alle Kreislinien damit wegoperiert.

\Timestamp{2015-06-03}
        Behauptung: In $\R^2 \times (t,\infty)$ besteht $S_t$ aus glatten $2$-Sphären und jede berandet einen glatten $3$-Ball.
        \begin{itemize}
            \item
                Für $t > \max(h\circ f)$ ist dies trivial.
            \item
                Gilt die Aussage für $t$ und enthält $[t', t]$ keine kritischen Werte, so gilt sie auch für $t'$.
            \item
                \begin{itemize}
                    \item
                        Passage eines Maximums.
                    \item
                        Passage eines Minimums, zwei Fälle: Ball innen, Ball außen.
                    \item
                        Passage eines Sattelpunktes.
                        Diskussion aller Fälle:
                        \begin{itemize}
                            \item
                                Zwei Kreise, separate Bälle. Ok.
                            \item
                                Zwei Kreise, ein Ball. Unmöglich, da $S \subset \R^3$ Sphäre.
                            \item
                                Ein verbundener Kreis, ineinandergestülpte Hose. Ok.
                            \item
                                Zwei weitere Fälle, entstehend durch Spiegelung der vorigen. Ok.
                        \end{itemize}
                        Dies sind alle möglichen Fälle (Warum?).
                        Schließlich gilt für $t < \min(h\circ f)$: Die Glatte Sphäre $S \subset \R^3$ berandet einen glatten Ball.
                \end{itemize}
        \end{itemize}
        Gilt
    \end{proof}
\end{st}

Die Aussage für $\S^0 \injto \R^1$ war banal, $\S^1 \injto \R^2$ ist Jordan-Schönflies und $\S^2 \injto \R^3$ ist Alexander-Schönflies.
$\S^3 \injto \R^4$ ist ein offenes Problem: die sogenannte glatte Schönfließ-Vermutung.
$\S^{n-1} \injto \R^n$ für $n \ge 5$ gilt dies dank eines $h$-Borlismus.


% §C2
\section{Seifert-Flächen und Knotengeschlecht}

Berandet jeder Knoten eine kompakte Fläche? ist diese orientiert?

Der triviale Knoten berandet eine Kreisfläche, aber auch ein Möbiusband (und viele weitere).
Der Kleeblatt-Knoten berandet ein Möbiusband mit 1,5 Windungen (nicht orientierbar).
Berandet der Kleeblatt-Knoten auch eine orientierbare Fläche? Ja!

\begin{st}[Seifert 1934, Pontryagin 1930]
    Zu jeder orientierten Verschlingung $K \subset \R^3$ existiert eine kompakte, orientierte, zusammenhängende Fläche $S \subset \R^3$ mit $\Boundary S = K$ (mitsamt Orientierung).
    \begin{proof}[Seifert-Algorithmus]
        Stelle $K$ als Diagramm in $\R^2$ dar.
        Löse jede Kreuzung auf (entsprechend Orientierung neu verbinden).
        Es bleiben Kreislinien in $\R^2$.
        Jede berandet eine Kreisscheibe.
        Verklebe diese nun je nach Kreuzung.
        Wir erhalten eine kompakte, orientierte Fläche $S$ mit $\Boundary S = K$.
        Ist $S$ noch nicht zusammenhängend, so klebe Zylinder an.
    \end{proof}
\end{st}

Erinnerung: Flächenklassifikation:
\begin{math}
    S \homeomorphic F_{g,r}^+
\end{math}
für $g \in \N$, $r \in \N_{\ge 1}$.

\begin{df}
    Das Geschlecht eines Knotens $K$ ist
    \begin{math}
        g(K) := \min \Set{g(S) & \text{$S$ ist kompakt, orientiert, zusammenhängend, $\Boundary S = K$}}
    \end{math}
\end{df}

\begin{note}
    \begin{itemize}
        \item
            $K$ ist trivial genau dann, wenn $g(K) = 0$ ist.
        \item
            Kleeblattknoten: $g(3_1) = 1$.
        \item
            Achterknoten: $g(4_1) = 1$.
    \end{itemize}
\end{note}

\Timestamp{2015-06-08}

\subsection{Additivität}

\begin{st}
    Für alle Knoten $K_1, K_2$ gilt $g(K_1 \# K_2) = g(K_1) + g(K_2)$.
    \begin{proof}
        Sei $K_i = \Boundary S_i$ mit $g(K_i) = g(S_i)$ (minimales Geschlecht).
        Dann ist $S = S_1 \#_{\Boundary} S_2$ eine Seifert-Fläche für $K = K_1 \# K_2$.
        Damit gilt
        \begin{math}
            g(K_1 \# K_2) \le g(S) = g(S_1) + g(S_2) = g(K_1) = g(K_2),
        \end{math}
        womit „$\le$“ gezeigt ist.

        Sei nun $S$ eine Seifert-Fläche zu $\Boundary S = K = K_1 \# K_2$ mit minimalem Geschlecht, d.h. $g(S) = g(K)$.
        Sei $D \subset \R^3$ ein Ball, der $K$ in die Faktoren $K_1$ und $K_2$ teilt.
        Ohne Einschränkung sei $E = \Boundary D$ glatt.

        Nach Isotopie von $S$ können wir Transversalität von $S$ zu $E$ annehmen.
        Somit besteht $S \cap E$ aus $n$ disjunkten Kreislinien und einem Segment von $A$ nach $B$.
        \begin{seg}[$n = 0$]
            Wir können $S$ aufschneiden.
            Es ist dann $S = S_1 \#_{\Boundary} S_2$ und $g(K) = (S) = g(S_1) + g(S_2) \ge g(K_1) + g(K_2)$.
        \end{seg}
        \begin{seg}{$n \ge 1$}
            Wähle einen innersten Kreis, dieser berandet eine Scheibe $\D^2 \xto[homeomorphic] F \subset E$ mit $F \cap S = \boundary F$.
            Betreibe nun Chirurgie an $F$ und erhalte aus $S$ so $S'$.
            Es gilt $\chi(S') = \chi(S) + 2$.
            Wäre $S'$ zusammenhängend, so $g(S') = g(S) - 1$ (Erinnerung für zusammenhängende Flächen: $\chi(F_g^+) = 2 - 2g$).
            Wegen $\Boundary S' = \Boundary S = K$ ist dies nicht möglich.
            Also ist $S'$ nicht zusammenhängend, $S = S_1 \sqcup S_2$.
            Da $K$ zusammenhängend ist, folgt $\Boundary S_1 = K$, $\Boundary S_2 = \emptyset$ (notwendigerweise gilt $\S_2 \homeomorphic \S^2$).
            Wir können $S_2$ löschen und erhalten $S_1$ mit $\Boundary S_1 = K$, $g(S_1) = g(S)$ und $S_1 \cap E$ hat eine Kreiskomponente weniger.
            Induktiv erreichen wir den Fall $n = 0$.
        \end{seg}
    \end{proof}
\end{st}

\begin{kor}
    Aus $A \# B \sim \KnotTriv$ folgt $A \sim \KnotTriv$, $B \sim \KnotTriv$.
\end{kor}

\begin{df}
    Ein Knotentyp $K$ heißt \emphdef{prim}, wenn aus $K = A \# B$ stets entweder $A = \KnotTriv$ oder $B = \KnotTriv$ folgt.
\end{df}

\begin{prop}
    Knoten vom Geschlecht 1 sind prim.
    \begin{proof}
        Sei $K = A \# B$ mit $g(K) = 1$, dann ist $g(A) + g(B) = 1$ und somit $g(A) = 1$, $g(B) = 0$ oder umgekehrt.
    \end{proof}
\end{prop}

\begin{ex}
    $3_1$ und $4_1$ sind prim.
\end{ex}

\begin{prop}
    Es gibt unendlich viele Primknotentypen, z.B. die Twistknoten $T_n$, $n \in \Z \setminus \Set 0$.
    \begin{proof}
        Es gilt $g(T_n) \le 1$ gemäß Skizze.
        Für $n \neq 0$ gilt $T_n \neq 0$ (mit Färbungszahl oder später Alexander-Polynom).
        Zudem $T_m \neq T_n$ für $m \neq n$.
    \end{proof}
\end{prop}

\begin{st}
    Jeder Knotentyp $K$ ist eine verbundene Summe aus Primknotentypen, d.h. es existieren $P_1, \dotsc P_n$ prim mit $K = P_1 \# \dotsb \# P_n$.
    \begin{proof}
        Induktion über $g(K)$: für $g(K) = 0$ ist $K$ trivial und die Aussage gilt für  $n = 0$, für $g(K) = 1$ ist $K = P_1$ prim.
        Sei nun $g(K) \ge 2$.
        Betrachte zwei Fälle: falls $K$ prim, so ist $K = P_1$ ein verbundene Summe aus Primknoten.
        Falls $K$ nicht prim ist, so existieren $A, B$ nicht-trivial mit $K = A \# B$, $g(K) = g(A) + g(B)$, $g(A), g(B) \ge 1$, also $g(A), g(B) < g(K)$.
        Gemäß Induktionvoraussetzung ist $A = P_1 \# \dotsb \# P_m$, $B = P_{m+1} \# \dotsb \# P_n$ und somit $K = P_1 \# \dotsb \# P_n$.
    \end{proof}
\end{st}

Ziel: \emph{eindeutige} Primfaktorzerlegung analog zu $(\N_{\ge 1}, \cdot, 1)$.
Gegenbeispiele:
\begin{itemize}
    \item
        $\Z[i\sqrt{5}] \subset \C$ (Übung).
    \item
        Oder betrachte
        \begin{math}
            \R[x] \supset \Set{f: \R \to \R, f'(0) = 0} = \R[x^2, x^3]
        \end{math}
        Hier ist $x^6 = x^2 x^2 x^2 = x^3 x^3$.
    \item
        Betrachte $(M, \cdot, 1)$ mit
        \begin{math}
            M = \N \setminus \Set{13}.
        \end{math}
        Nun ist ${13}^6 = {13}^2 {13}^2 {13}^2 = {13}^3 {13}^3$.
\end{itemize}
Betrachte einen langen Knoten $K$ (1-Schlingel).
Sei $S = S_1 \cup \dotsb \cup S_n$, $n \ge 1$ eine Familie disjunkter glatter 2-Sphären $S_i \subset \R^3$, die den Knoten transvers in je höchstens zwei Punkten schneiden.
Jede Sphäre $S_i$ berandet $D_i \homeomorphic \D^3$.
Jeder Ball $D_i$ beinhaltet einen Knoten $K_i$ (oder nicht).
Enthält $D_i$ weitere Bälle, so sind diese zur Definition von $K_i$ vorher auszuscheiden und durch den trivialen zu ersetzen.

\begin{prop}
    Es gilt $K = K_1 \# \dotsb \# K_n$.
\end{prop}

\begin{df}
    Nach Löschung aller trivialen und leeren Beiträge erhalten wir $K = P_1 \# \dotsb \# \# P_m$ ($m \le n$) mit $P_i$ nicht-trivial.
    Wir sagen, dass $S$ den Knoten $K$ in $P_1, \dotsc, P_m$ zerlegt.
\end{df}

\begin{lem}
    Sind $P_1, \dotsc, P_m$ prim und $S^* \supset S$ eine Verfeinerung (durch Hinzufügen weiterer 2-Sphären), dann zerlegt $S^*$ den Knoten $K$ in dieselben Primknoten $P_1, \dotsc, P_m$.
    \begin{proof}
        Wir finden $P_i = P_i' \# P_i''$, damit klar.
    \end{proof}
\end{lem}

\begin{st}
    Sei $K$ ein Knotentyp und $P_1, \dotsc, P_m$, $Q_1, \dotsc, Q_m$ prim und
    \begin{math}
        K = P_1 \# \dotsb \# P_2 = Q_1 \# \dotsb \# Q_n.
    \end{math}
    Dann gilt $m = n$ und nach Umordnung $P_1 = Q_1$, \dots, $P_n = Q_n$.
    \begin{proof}
        $S$ zerlege $K$ in $P_1, \dotsc, P_m$, $S'$ zerlege $K$ in $Q_1, \dotsc, Q_n$.
        \begin{seg}[$S \cap S' = \emptyset$]
            Wenn $S \cap S' = \emptyset$, dann betrachte $S^* = S \cup S'$.
            Wegen $S^* \supset S, S'$, zerlegt $S^*$ in $P_1, \dotsc, P_n$, bzw. $Q_1, \dotsc, Q_n$ (Eindeutigkeit bis auf Umordnung).
        \end{seg}
        \begin{seg}[$S \cap S' \neq \emptyset$]
            Nach Isotopie von $S'$ schneiden sich $S$ und $S'$ transvers.
            $S \cap S'$ ist eine Familien disjunkter glatter Kreislinien.

        \end{seg}
    \end{proof}
\end{st}

