% Kapitel D
\chapter{Seifert-Form, Signator und Alexander-Polynom}

Ziel: Ausgehend von einem Knoten $K$ konstruiren wir eine Seifert-Fläche $S \subset \R^3$, $\Boundary S = K$.
Daraus gewinnen wir die Seifert-Form $\Theta_s: H_1 S \times H_1 S \to \Z$ ($H_1 = \pi_1(S)_{\text{ab}} \isomorphic \Z^{2g}$ nach Flächenklassifikation) und das Alexander-Polynom
\begin{math}
    \Delta(K)
    := \det(q^{-1} \Theta - q \Theta^t) \in \Z[q^{\pm 1}].
\end{math}
Dis ist unabhängig von den gewählten Fläche $S$ und daher eine Invariante des Knotentyps von $K$. 


% §D1
\section{Äquivalenz von Seifert-Flächen}

\begin{note}
    Sei $K$ ein Knoten.
    \begin{itemize}
        \item
            Es gibt zu $K$ unendlich viele Seifert-Flächen (Anhängen von Henkel, Erhöhung des Geschlechts).
        \item
            Auch Seifert-Flächen minimalen Geschlechts nicht nicht eindeutig (d.h. zueinander isotop), siehe Skizze.
            \begin{math}
                \Boundary S_1 &= A \cup B = \Boundary S_2,
                S_0 \cap S_2 &= A \cap B = S_1 \cap S_0
            \end{math}
            $S_0 \cup S_1$ und $S_0 \cup S_2$ sind Seifert-Flächen zu $K$.
            Es gilt $S_0 \cup S_1 \homeomorphic S_0 \cup S_2 \homeomorphic F_{1,1}^+$, also gleiches Geschlecht $1$.

            Alle Seifert-Flächen $S$ zu $K$ mit minimalem Geschlecht $g(S) = g(K)$ sind untereinander homöomorph (nach Flächenklassifikation $\homeomorphic F^+_{g,1}$), aber im allgemeinen sind ihre Einbettungen im $\R^3$ nicht isotop.
    \end{itemize}
\end{note}


\subsection{Chirurgie auf Seifert-Flächen}


Die folgende Chirurgie (eingebette im $\R^3$) ändert die Fläche $S$ zu $S'$, nicht aber ihren Rand $\Boundary S = \Boundary S' = K$.
Bild: Aufschneiden einer Säule.

\begin{st}
    Je zwei Seifert-Flächen $S, S'$ zum selben Knoten $K$ lassen sich ineinander überführen durch eine endliche Folge solcher Chirurgieren und Isotopie.
    Wir sprechen von \emphdef{$S$-Äquivalenz}.
    \begin{proof}
        Betrachte zunächst nur die Seifert-Fläche, die aus dem Seifert-Algorithmus entstehen.
        $D \mapsto S$ Kanonische Seifert-Fläche zum Diagramm $D$.
        Betrachte den Effekt von $R$-Zügen:
        \begin{enumerate}[R1]
            \item
                …
            \item
                …
            \item
                Übung
        \end{enumerate}
    \end{proof}
\end{st}





