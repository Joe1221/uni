% Kapitel D
\chapter{Seifert-Form, Signatur und Alexander-Polynom}

Ziel: Ausgehend von einem Knoten $K$ konstruiren wir eine Seifert-Fläche $S \subset \R^3$, $\Boundary S = K$.
Daraus gewinnen wir die Seifert-Form $\Theta_s: H_1 S \times H_1 S \to \Z$ ($H_1 = \pi_1(S)_{\text{ab}} \isomorphic \Z^{2g}$ nach Flächenklassifikation) und das Alexander-Polynom
\begin{math}
    \Delta(K)
    := \det(q^{-1} \Theta - q \Theta^t) \in \Z[q^{\pm 1}].
\end{math}
Dis ist unabhängig von den gewählten Fläche $S$ und daher eine Invariante des Knotentyps von $K$. 


% §D1
\section{Äquivalenz von Seifert-Flächen}

\begin{note}
    Sei $K$ ein Knoten.
    \begin{itemize}
        \item
            Es gibt zu $K$ unendlich viele Seifert-Flächen (Anhängen von Henkel, Erhöhung des Geschlechts).
        \item
            Auch Seifert-Flächen minimalen Geschlechts nicht nicht eindeutig (d.h. zueinander isotop), siehe Skizze.
            \begin{math}
                \Boundary S_1 &= A \cup B = \Boundary S_2,
                S_0 \cap S_2 &= A \cap B = S_1 \cap S_0
            \end{math}
            $S_0 \cup S_1$ und $S_0 \cup S_2$ sind Seifert-Flächen zu $K$.
            Es gilt $S_0 \cup S_1 \homeomorphic S_0 \cup S_2 \homeomorphic F_{1,1}^+$, also gleiches Geschlecht $1$.

            Alle Seifert-Flächen $S$ zu $K$ mit minimalem Geschlecht $g(S) = g(K)$ sind untereinander homöomorph (nach Flächenklassifikation $\homeomorphic F^+_{g,1}$), aber im allgemeinen sind ihre Einbettungen im $\R^3$ nicht isotop.
    \end{itemize}
\end{note}


\subsection{Chirurgie auf Seifert-Flächen}


Die folgende Chirurgie (eingebette im $\R^3$) ändert die Fläche $S$ zu $S'$, nicht aber ihren Rand $\Boundary S = \Boundary S' = K$.
Bild: Aufschneiden einer Säule.

\begin{st}
    Je zwei Seifert-Flächen $S, S'$ zum selben Knoten $K$ lassen sich ineinander überführen durch eine endliche Folge solcher Chirurgieren und Isotopie.
    Wir sprechen von \emphdef{$S$-Äquivalenz}.
    \begin{proof}
        Betrachte zunächst nur die Seifert-Fläche, die aus dem Seifert-Algorithmus entstehen.
        $D \mapsto S$ Kanonische Seifert-Fläche zum Diagramm $D$.
        Betrachte den Effekt von $R$-Zügen:
        \begin{enumerate}[R1]
            \item
                …
            \item
                …
            \item
                Übung
        \end{enumerate}
    \end{proof}
\end{st}

\Timestamp{2015-06-15}

\begin{lem}
    Jede Seifert-Fläche zu $K$ ist S-äquivalent zu einer kanonischen S-Fläche (d.h. aus dem Algorithmus angewendet auf ein Diagramm zu K entstanden).
    \begin{proof}
        Wir nutzen die Flächenklassifikation: $S \homeomorphic F_{g,1}^+$.
        Die Einbettung in $\R^3$ können wir modulo Isotopie darstellen als Rechteck mit $2g$ Bändern (die unter Wahrung der Orientierung verschlungen sind).
        Wir betrachten $D = \Boundary S$ als Knotendiagramm und wenden den Seifert-Algorithmus an.
        Die so entstehende Fläche $S'$ ist S-äquivalent zur ursprünglichen Fläche $S$.
        Wir betrachten dazu (zwei verschiedenartige) Kreuzungen zweier Bänder und die Verdrillung eines Bandes, jeweils: $S \leadsto D \leadsto D' \leadsto S'$ und $S \sim S'$ durch S-Äquivalenz.
    \end{proof}
\end{lem}

\begin{lem}
    Je zwei kanonische Seifert-Flächen zu $K$ sind $S$-äquivalent.
    \begin{proof}
        Betrachte R-Zügen von $D$ nach $D'$.
    \end{proof}
\end{lem}


% §D2
\section{Seifert-Form}


Wir betrachten einen Knoten $K \subset \R^3$ und wählen eine zugehörige Seifert-Fläche $S \subset \R^3$, also $K = \Boundary S$.
Wir bestimmen nun $\Theta_S: H_1(S) \times H_1(S) \to \Z$.

Erinnerung:
\begin{math}
    \pi_1(S, *)
    = \pi_1(F_{g,1}^+)
    = \Gen{a_1, b_1, \dotsc, a_g, b_g & -}
    \isomorphic F_{2g}
\end{math}
Wir setzen die erste Homologiegruppe als
\begin{math}
    H_1(S) := \pi_1(S)_{\text{ab}} \isomorphic \Z^{2g}.
\end{math}
In der Abelsch-Machung können wir den Fußpunkt „vergessen“.
Alternativ: Betrachtung im Henkelmodell, Visualisierung der Abelsch-Machung im Henkelmodell.

Wir wollen messen, wie $S$ in $\R^3$ „verschlungen“ ist.

\begin{prop}
    Jede Seifert-Fläche $S \subset \R^3$ lässt sich aufdicken zu $[-1,1] \times S \xto[injective] \R^3$, mit $\Set{0} \times S \xto{\text{proj.}} S$ durch $(0,x) \mapsto x$.
    Die Orientierung legen wir so fest, dass die positive Normale auf $S$ „von $-1$ nach $1$ zeigt“.
    Dies ist eindeutig bis auf Isotopie bei fester Fläche $S$.

    Jede Kurve $\alpha$ auf $S$ definiert Kurven $\alpha^\uparrow = \Set{1} \times \alpha$, und $\alpha^\downarrow = \Set{-1} \times \alpha$.

    Wir definieren $\Theta_S: H_1 S \times H_1 S \to \Z$ durch
    \begin{math}
        (\alpha, \beta) \mapsto \lk(\alpha^\downarrow, \beta^\uparrow).
    \end{math}
    Dies ist eine bilineare Abbildung.
    Nach Wahl einer Basis $a_1, b_1, \dotsc, a_g, b_g \in H_1 S$ (wie oben) entspricht dies einer Matrix $\Z^{2g \times 2g}$.
\end{prop}

\begin{ex}
    \begin{itemize}
        \item
            Triviales Beispiel:
            Für $S = F_{g,1}^+$ in der Standard-Einbettung
            \begin{math}
                \lk(a^\uparrow, b^\downarrow) = \lk(b^\downarrow, a^\uparrow) &= 0, \\
                \lk(a^\downarrow, b^\downarrow) = \lk(b^\uparrow, a^\downarrow) &= 1.
            \end{math}
            Wir erhalten für $\Theta_S \in \Z^{2g}$ eine Blockmatrix mit Blöcken
            \begin{math}
                \Matrix*{0 & 1 \\ 0 & 0} \in \Z^{2 \times 2}.
            \end{math}
        \item
            Twist-Knoten $T_k$ mit $2k$ Kreuzungen, entspricht $k$ vollen Drehungen der Seifert-Fläche $S$, $\Boundary S = T_k$.
            \begin{math}
                \lk(a^\downarrow, b^\uparrow) &= 0, \\
                \lk(b^\downarrow, a^\uparrow) &= -1, \\
                \lk(a^\downarrow, a^\uparrow) &= -k, \\
                \lk(b^\downarrow, b^\uparrow) &= -1,
            \end{math}
    \end{itemize}
\end{ex}

Es stellt sich die Frage, ob $\det(\Theta_S)$ invariant ist.
Dies gilt leider nicht.
Für folgende Ausdrücke jedoch schon:
\begin{itemize}
    \item
        $|\det(\Theta_S + \Theta_S^T)|$ (Determinante),
    \item
        $\sgn(\Theta_S + \Theta_S^T)$ (Signatur),
    \item
        $\det(q^{-1}\Theta_S - q\Theta_S^T) \in \Z[q^{\pm 1}]$ (Alexander-Polynom).
\end{itemize}


\Timestamp{2015-06-17}

Auswirkung der S-Äquivalenz auf die Seifert-Form (Anfügen eines Zylinders von $S$ zu $S'$)
\begin{math}
    H_1 S \xto[injective] H_1 S'
    = H_1 S \oplus \<a\> \oplus \<b\>,
\end{math}
denn
\begin{math}
    \pi_1(S,*) &= \Gen{a_1, b_1, \dotsc, a_g, b_g & - } \\
    \pi_1(S',*) &= \Gen{a_1, b_1, \dotsc, a_g, b_g, a, b & - }
\end{math}

Es ergibt sich
\begin{math}
    \Theta_{S'} =
    \Matrix{
        &  &  & & 0 & * \\
        & \Theta_S & & \vdots & \vdots \\
        & & & 0 & * \\
        0 & \hdots & 0 & 0 & 0 \\
        * & \hdots & * & 1 & * \\
    }
\end{math}

\begin{df}
    Zwei Matrizen $M \in \Z^{2g \times 2g}$ und $M' \in \Z^{2g' \times 2g'}$ heißen \emphdef{S-äquivalent}, wenn sie sich durhc eine endliche Folge solcher Operationen unterscheiden:
    \begin{enumerate}[1)]
        \item
            Basiswechsel: $M' = T^t M T$ mit $T \in \GL_{2n} \Z$,
        \item
            Stabilisierung:
            \begin{math}
                M' =
                \Matrix{
                    &  &  &  0 & * \\
                    & M & & \vdots & \vdots \\
                    & & & 0 & * \\
                    0 & \hdots & 0 & 0 & 0 \\
                    * & \hdots & * & 1 & 0 \\
                }.
            \end{math}
            (oder umgekehrt).
    \end{enumerate}
\end{df}

\begin{prop}
    $|\det(\Theta_S + \Theta_S^t)|$ ist unabhängig von der Wahl der Basis von $H_1 S$ und von der Wahl der Seifert-Fläche $S$ zu einem Knoten $K$.
    Somit ist $|\det(K)| := |\det(\Theta_S + \Theta_S^t)|$ eine Invariante des Knotens.
    \begin{proof}
        Stelle $\Theta_S: H_1 S \times H_1 S \to \Z$ dar durch eine Matrix $M \in \Z^{2g \times 2g}$, hierzu Basiswahl.
        \begin{enumerate}[1)]
            \item
                Basiswechsel: $M' = T^t M T$ mit $T \in \GL_{2g} \Z$.
                Es gilt
                \begin{math}
                    M' + {M'}^t
                    = T^t M T + T^t M^t T
                    = T^t (M + M^t) T
                \end{math}
                und für die Determinante
                \begin{math}
                    \det(M' + {M'}^t)
                    = \det(T^t) \det(M + M^t) \det(T)
                    = \det(M + M^t),
                \end{math}
                da $\det(T) = \det(T^t) = \pm 1$.
            \item
                Stabilisierung:
                \begin{math}
                    M' =
                    \Matrix{
                        &  &  &  0 & * \\
                        & M & & \vdots & \vdots \\
                        & & & 0 & * \\
                        0 & \hdots & 0 & 0 & 0 \\
                        * & \hdots & * & 1 & 0 \\
                    }.
                \end{math}
                also
                \begin{math}
                    M' + {M'}^t =
                    \Matrix{
                        &  &  &  0 & * \\
                        & M + M^t & & \vdots & \vdots \\
                        & & & 0 & * \\
                        0 & \hdots & 0 & 0 & 1 \\
                        * & \hdots & * & 1 & 0 \\
                    }.
                \end{math}
                und $\det(M' + (M')^t) = - \det(M + M^t)$.

        \end{enumerate}
    \end{proof}
    \begin{note}
        Wir retten das Vorzeichen:
        \begin{math}
            \det(K) &= (-1)^{g(S)} \det(\Theta_S + \Theta_S^t) \\
            &= (-i)^{2g(S)} \det(\Theta_s + \Theta_s^t) \\
            &= \det(-i\Theta_s - i \Theta_s^t).
        \end{math}
    \end{note}
\end{prop}

\begin{ex}
    \begin{itemize}
        \item
            Twistknoten: $T_k$.
            Seifert-Form:
            \begin{math}
                \Theta_k &= \Matrix{
                    -k & 0 \\
                    -1 & -1
                }, &&
                \Theta_k + \Theta_k^t &= \Matrix{
                    -2k & - 1 \\
                    -1 & -2
                }.
            \end{math}
            Es gilt
            \begin{math}
                |\det(T_k)| &= |4k - 1|, \\
                \det(T_k) &= (-1)^1 (4k-1) = 1 - 4k.
            \end{math}
            Dies unterscheidet bereits alle Twistknoten.
            Anwendung: Für $k \neq 0$ ist $T_k$ prim.
            \begin{enumerate}[1.]
                \item
                    $T_0 = 0$ trivial, $\det 0 = 1$
                \item
                    $T_1 = 3_1$ Kleeblattknoten, $\det 3_1 = -3$,
                \item
                    $T_{-1} = 4_1$ Achterknoten, $\det 4_1 = 5$,
            \end{enumerate}
    \end{itemize}
\end{ex}

\begin{df}
    Die Signatur einer symmetrischen Matrix ist
    \begin{math}
        \sgn(M) := \sum_{k=1}^{2g} \sgn(\lambda_k)
    \end{math}
    für Eigenwerte $\lambda_k$.
\end{df}

\begin{prop}
    Die Signatur von $\Theta_S + \Theta_S^t$ ist eine Invariante des Knotens $K$, geschrieben $\sgn(K) := \sgn(\Theta_S + \Theta_S^t)$.
    \begin{proof}
        \begin{enumerate}[1)]
            \item
                Basiswechsel: $M' = T^t M T$ mit $T \in \GL_{2g} \Z$.

                Trägheitssatz von Sylvester (Signatur bleibt bei beliebiger Diagonalisierung erhalten).
            \item
                Stabilisierung:
                \begin{math}
                    M' + {M'}^t =
                    \Matrix{
                        &  &   & 0 & 0 \\
                        & M + M^t & & \vdots & \vdots \\
                        & & & 0 & 0 \\
                        0 & \hdots & 0 & 0 & 1 \\
                        0 & \hdots & 0 & 1 & 0 \\
                    }.
                \end{math}
                Durch Diagonalisierung
                \begin{math}
                    \Matrix{
                        \lambda_1 &  &  & 0 & 0 \\
                        & \ddots & & \vdots & \vdots \\
                        & & \lambda_{2g} & 0 & 0 \\
                        0 & \hdots & 0 & 1 & 0 \\
                        0 & \hdots & 0 & 0 & -1 \\
                    }.
                \end{math}
        \end{enumerate}
    \end{proof}
\end{prop}


\begin{ex}
    \begin{math}
        \Theta_k + \Theta_k^t = \Matrix{-2k & -1 \\ -1 & -2}.
    \end{math}
    Charakteristisches Polynom:
    \begin{math}
        (x + 2k)(x + 2) - 1
        &= x^2 + (2k+2) x + (4k - 1) \\
    \end{math}
    Für $k = 0$:
    \begin{math}
        = (x - (\sqrt 2 + 1))(x - (\sqrt 2 - 1)).
    \end{math}
    für $k = 1$:
    \begin{math}
        x^2 + 6x + 3
    \end{math}
    zwei Eigenwerte kleiner Null, also $\sgn(T_1) = -2$.

    Für $k = -1$:
    \begin{math}
        x^2 - 5,
    \end{math}
    also $\sgn(T_{-1}) = 0$.

    Für $k = -2$:
    \begin{math}
        x^2 - 2x - 9,
    \end{math}
    also $\sgn(T_{-2}) = 0$.

    Allgemein:
    \begin{math}
        \sgn(T_k) = \begin{cases}
            0 & \text{für $k \le 0$}, \\
            -2 & \text{für $k \ge 1$}
        \end{cases}
    \end{math}
\end{ex}

Algorithmische Fragen:
\begin{enumerate}[1)]
    \item
        Wie berechnet man effizient das charakteristische Polynom $P_A(x) := \det(x 1_{n\times n} - A)$?
    \item
        Wie bestimmt man hieraus die Signatur?

        Vorzeichenregel von Descartes:
        Sei $P \in \R[x]_n^1$ ein (normiertes) reelles Polynom vom Grad $n$ mit $n$ reellen Nullstellen $\lambda_1, \dotsc, \lambda_n \in \R$
        \begin{math}
            \#\Set{k & \lambda_k > 0} &= \text{Anzahl Vorzeichenwechsel in den Koeff. von $P(x)$} \\
            \#\Set{k & \lambda_k < 0} &= \text{Anzahl Vorzeichenwechsel in den Koeff. von $P(x)$} \\
            \#\Set{k & \lambda_k = 0} &= \max\Set{k & x^k \divs P(x)}
        \end{math}
\end{enumerate}

\Timestamp{2015-06-22}

\subsection{Einfluss von Symmetrien}

\begin{table}[ht]
    \begin{tabular}{c|ccccc}
        Knoten & Or. Knoten & Or. $\R^3$ & Seifert-Fläche & Seifert-Form & Signatur \\ \hline
        $K$   & + & + & $S$   & $\Theta_S$ & $\sigma$ \\
        $K^!$ & - & + & $S^!$ & $\Theta_S^t$ & $\sigma$ \\
        $K^x$ & + & - & $S$   & $-\Theta_S^t$ & $-\sigma$ \\
        $K^*$ & - & - & $S^!$ & $-\Theta_S$ & $-\sigma$
    \end{tabular}
    \caption{Einfluss von Symmetrien auf Knoten-Invarianten}
\end{table}

\begin{nt}
    Dies ist einer der einfachsten Beweise, dass es chirale Knoten gibt, konkret: $3_1^+ \neq 3_1^-$.
    Weitere Kandidaten:
    \begin{itemize}
        \item
            Färbungen: $P_{A_5}^{(12345)}(3_1^\pm) = 1 + 5x^{\mp}$
        \item
            Jones-Polynom (später)
    \end{itemize}
\end{nt}

\subsection{Verhalten unter verbundener Summe}

Knoten werden mit ihrer Seifert-Fläche verbunden: $S = S_1 \#_\Boundary S_2$.
Seifert-Form:
\begin{math}
    H_1(S) &= H_1(S_1) \oplus H_1(S_2), \\
    g(S) &= g(S_1) + g(S_2), \\
    \Theta_S &= \Matrix{\Theta_{S_1} & 0 \\ 0 & \Theta_{S_2}} = \Theta_{S_1} \oplus \Theta_{S_2}.
\end{math}
Signatur:
\begin{math}
    \sgn(K_1 \# K_2) = \sgn(K_1) + \sgn(K_2)
\end{math}

\begin{ex}
    Kreuzknoten: $\sgn(3_1^+ \# 3_1^-) = +2 - 2 = 0$

    Großmutterknoten: $\sgn(3_1^\pm \# 3_1^\pm) = \pm 2  \pm 2 = \pm 4$

    Die Knoten sind also verschieden.
\end{ex}

Wir versuchen
\begin{math}
    \det(q^{-1} \Theta_S - q\Theta_S^t)
\end{math}
(für $q = i$ entspricht das der Signatur).

Basiswechsel:
Sei $M$ eine Matrix zu $\Theta_S$ und $T \in \GL_{2g} \Z$:
\begin{math}
    \det(q^{-1} M' - q {M'}^t)
    &= \det(T^t (q^{-1}M - qM^t) T) \\
    &= \det(T^t) \det(q^{-1} M - qM^t) \det(T)) \\
    &= \det(q^{-1} M - qM^t).
\end{math}

Stabilisierung:
\begin{math}
    M' &=
    \Matrix{
        &  &  &  0 & * \\
        & M & & \vdots & \vdots \\
        & & & 0 & * \\
        0 & \hdots & 0 & 0 & 0 \\
        * & \hdots & * & 1 & 0 \\
    } \\
    q^{-1}M' - q {M'}^t &=
    \Matrix{
        &  &  &  0 & * \\
        & q^{-1}M - qM^t & & \vdots & \vdots \\
        & & & 0 & * \\
        0 & \hdots & 0 & 0 & -q \\
        * & \hdots & * & q^{-1} & 0 \\
    } \\
    &= \det(q^{-1}M - qM^t) \det \Matrix{0 & -q \\ q^{-1} & 0}
    = \det(q^{-1}M - qM^t).
\end{math}

\begin{st}
    Das (Laurent-)Polynom
    \begin{math}
        \Delta(K)
        := \det(q^{-1} \Theta_S - q\Theta_S^t)
        \in \Z[q^\pm]
    \end{math}
    ist invariant unter Basiswechsel und Stabilisierung und somit eine Invariante des Knotens $K$.
    \begin{note}
        \begin{itemize}
            \item
                Dies ist eine Verallgemeinerung der Determinante $\det(K) = \Delta(K)|_{q\mapsto i}$.
        \end{itemize}
    \end{note}
\end{st}

\begin{ex}
    Betrachte den Twistknoten $T_k$:
    \begin{math}
        \Theta_S = \Matrix{-k & 0 \\ -1 & -1}.
    \end{math}
    Es gilt
    \begin{math}
        q^{-1} \Theta_S - q\Theta_S^t
        &= \Matrix{
            -kq^{-1} + k q & 0 + 1q \\
            -1q^{-1} - 0q & -1q^{-1} + 1q
        } \\
        \det(q^{-1} \Theta_S - q \Theta_S^t)
        &= k(q-q^{-1})(q-q^{-1}) - q(-q^{-1}) \\
        &= k(q^2 - 2 + q^{-2}) + 1 \\
        &= kq^2  + (1-2k) q^0 + kq^{-2}.
    \end{math}
    Probe mit $q \mapsto i$:
    \begin{math}
        -k + 1-2k - k = 1 - 4k,
    \end{math}
    was mit dem übereinstimmt, was wir früher für $\det(T_k)$ ermittelt hatten.
    \begin{note}
        \begin{itemize}
            \item
                Für Knoten tauchen nur gerade Potenzen auf.
            \item
                Zudem git $\Delta(q^{-1}) = \Delta (q)$, denn
                \begin{math}
                    \Delta(K)|_{q\mapsto q^{-1}}
                    &= \det(q \Theta_S - q^{-1}0\Theta_S^t) \\
                    &= \det(-q^{-1} \Theta_S + q^{-1} \Theta_S) \\
                    &= (-1)^{2g} \det(q^{-1} \Theta_s - q \Theta_S)
                    = \Delta(K)
                \end{math}
        \end{itemize}
    \end{note}
\end{ex}

\begin{prop}
    \begin{itemize}
        \item
            $
                \Delta(K)
                = \Delta(K^!)
                = \Delta(K^x)
                = \Delta(K^*)
            $,
        \item
            $\Delta(K_1 \# K_2) = \Delta(K_1) \Delta(K_2)$.
    \end{itemize}
\end{prop}

\begin{ex}
    \begin{math}
        \Delta(3_1^\pm) &= q^2 - 1 + q^{-2} \\
        \Delta(3_1^\pm \# 3_1^\pm) &= (q^2 - 1 + q^{-2})^2
    \end{math}
\end{ex}

\begin{df}
    Wir definieren für $p = a_m q^m + \dotsb + a_nq^n$ mit $m \le n$ und $a_m, a_n \neq 0$ den \emphdef{Span}
    \begin{math}
        \span(p) := n - m.
    \end{math}
\end{df}

\begin{prop}
    \begin{math}
        \span_q \Delta(K) \le 4 g(K).
    \end{math}
    \begin{proof}
        Sei $S$ eine Seifert-Fläche zu $K$ mit minimalem Geschlecht, $g(S) = g(K)$.
        Betrachte $\Theta_S$ und $q^{-1} \Theta_S - q\Theta_S^t$:
        \begin{math}
            \Matrix{
                a_{ij} q^{-1} + b_{ij} q^{+1} & \hdots \\
                \vdots & \ddots
            }.
        \end{math}
        \begin{math}
            \det(\dotso) = \sum_{k=-2g}^{2g} c_k q^k.
        \end{math}
        Also $\span \det(\dotso) \le 4g$.
    \end{proof}
    \begin{note}
        \begin{itemize}
            \item
                Dank obiger Beobachtung ist für Knoten stets $\span_q \Delta(K) \in 4 \N$.
            \item
                Dies liefert eine untere Schranke für das Knotengeschlecht: $g(K) \ge \frac{1}{4} \span_q \Delta(K)$.
                Als Anwendung: falls wir eine Seifert-Fläche $S$ mit $g(S) = \frac{1}{4} \span_q \Delta(K)$ finden, dann hat diese Seifert-Fläche also schon minimales Geschlecht.
        \end{itemize}
    \end{note}
\end{prop}

\begin{ex}[Übung]
    Die Konstruktion von $\sgn(K)$ und $\Delta(K)$ verallgemeinert sich von Knoten zu Verschlingungen $L$, $\#L \ge 1$.
    Details ausarbeiten!

    Wir erhalten $\Delta: \scr L \to \Z[q^{\pm 1}]$.
\end{ex}

\begin{st}[Alexander 1928, Conway 1969]
    Das Alexander-Polynom definiert $\Delta: \scr L \to \Z[q^{\pm 1}]$ erfüllt $\Delta(\KnotTriv) = 1$ und folgende Rekursion (lokale Relation, genannt \emphdef{Schienenrelation}):
    \begin{math}
        % fixme: explain L_+, L_-, L_0
        %\Delta(\KnotSect) - \Delta(\KnotSect*) &= (q - q^{-1}) \Delta(\KnotSectSolved).
        \Delta(L_+) - \Delta(L_-) &= (q - q^{-1}) \Delta(L_0).
    \end{math}
    Dies definiert $\Delta$ eindeutig und definiert sogar einen Algorithmus zur Berechnung von $\delta$.
    \begin{note}
        $\# L_+ = \# L_- = \# L_0 \pm 1$ , daher benötigen wir $\scr L$, die Menge $\scr K$ reicht nicht.
    \end{note}
    \begin{proof}
        Es gilt
        \begin{math}
            H_1(S_+)
            \isomorphic H_1(S_0) \oplus \<a\>
            \isomorphic H_1(S_-)
        \end{math}
        \begin{math}
            \Theta_+ &= \Matrix{\Theta_0 & v \\ u^t & w}
            \Theta_- &= \Matrix{\Theta_0 & v \\ u^t & w + 1}
        \end{math}
        Es folgt
        \begin{math}
            q^{-1} \Theta_- - q\Theta_-^t
            &= \Matrix{
                q^{-1} \Theta_0 - q\Theta_0^t & q^{-1} v - qu \\
                q^{-1} u^t - qv^t & (w+1)(q^{-1} - q)
            } \\
            \Delta(L_+) - \Delta(L_-) &= (q - q^{-1}) \Delta(L_0)
        \end{math}
    \end{proof}
\end{st}

\begin{kor}
    Für $q \mapsto i$ erhalten wir
    \begin{math}
        \det(L_+) - \det(L_-) = 2i \det(L_0).
    \end{math}
\end{kor}
