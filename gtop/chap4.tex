% Kapitel D
\chapter{Seifert-Form, Signatur und Alexander-Polynom}

Ziel: Ausgehend von einem Knoten $K$ konstruiren wir eine Seifert-Fläche $S \subset \R^3$, $\Boundary S = K$.
Daraus gewinnen wir die Seifert-Form $\Theta_s: H_1 S \times H_1 S \to \Z$ ($H_1 = \pi_1(S)_{\text{ab}} \isomorphic \Z^{2g}$ nach Flächenklassifikation) und das Alexander-Polynom
\begin{math}
    \Delta(K)
    := \det(q^{-1} \Theta - q \Theta^t) \in \Z[q^{\pm 1}].
\end{math}
Dis ist unabhängig von den gewählten Fläche $S$ und daher eine Invariante des Knotentyps von $K$. 


% §D1
\section{Äquivalenz von Seifert-Flächen}

\begin{note}
    Sei $K$ ein Knoten.
    \begin{itemize}
        \item
            Es gibt zu $K$ unendlich viele Seifert-Flächen (Anhängen von Henkel, Erhöhung des Geschlechts).
        \item
            Auch Seifert-Flächen minimalen Geschlechts nicht nicht eindeutig (d.h. zueinander isotop), siehe Skizze.
            \begin{math}
                \Boundary S_1 &= A \cup B = \Boundary S_2,
                S_0 \cap S_2 &= A \cap B = S_1 \cap S_0
            \end{math}
            $S_0 \cup S_1$ und $S_0 \cup S_2$ sind Seifert-Flächen zu $K$.
            Es gilt $S_0 \cup S_1 \homeomorphic S_0 \cup S_2 \homeomorphic F_{1,1}^+$, also gleiches Geschlecht $1$.

            Alle Seifert-Flächen $S$ zu $K$ mit minimalem Geschlecht $g(S) = g(K)$ sind untereinander homöomorph (nach Flächenklassifikation $\homeomorphic F^+_{g,1}$), aber im allgemeinen sind ihre Einbettungen im $\R^3$ nicht isotop.
    \end{itemize}
\end{note}


\subsection{Chirurgie auf Seifert-Flächen}


Die folgende Chirurgie (eingebette im $\R^3$) ändert die Fläche $S$ zu $S'$, nicht aber ihren Rand $\Boundary S = \Boundary S' = K$.
Bild: Aufschneiden einer Säule.

\begin{st}
    Je zwei Seifert-Flächen $S, S'$ zum selben Knoten $K$ lassen sich ineinander überführen durch eine endliche Folge solcher Chirurgieren und Isotopie.
    Wir sprechen von \emphdef{$S$-Äquivalenz}.
    \begin{proof}
        Betrachte zunächst nur die Seifert-Fläche, die aus dem Seifert-Algorithmus entstehen.
        $D \mapsto S$ Kanonische Seifert-Fläche zum Diagramm $D$.
        Betrachte den Effekt von $R$-Zügen:
        \begin{enumerate}[R1]
            \item
                …
            \item
                …
            \item
                Übung
        \end{enumerate}
    \end{proof}
\end{st}

\Timestamp{2015-06-15}

\begin{lem}
    Jede Seifert-Fläche zu $K$ ist S-äquivalent zu einer kanonischen S-Fläche (d.h. aus dem Algorithmus angewendet auf ein Diagramm zu K entstanden).
    \begin{proof}
        Wir nutzen die Flächenklassifikation: $S \homeomorphic F_{g,1}^+$.
        Die Einbettung in $\R^3$ können wir modulo Isotopie darstellen als Rechteck mit $2g$ Bändern (die unter Wahrung der Orientierung verschlungen sind).
        Wir betrachten $D = \Boundary S$ als Knotendiagramm und wenden den Seifert-Algorithmus an.
        Die so entstehende Fläche $S'$ ist S-äquivalent zur ursprünglichen Fläche $S$.
        Wir betrachten dazu (zwei verschiedenartige) Kreuzungen zweier Bänder und die Verdrillung eines Bandes, jeweils: $S \leadsto D \leadsto D' \leadsto S'$ und $S \sim S'$ durch S-Äquivalenz.
    \end{proof}
\end{lem}

\begin{lem}
    Je zwei kanonische Seifert-Flächen zu $K$ sind $S$-äquivalent.
    \begin{proof}
        Betrachte R-Zügen von $D$ nach $D'$.
    \end{proof}
\end{lem}


% §D2
\section{Seifert-Form}


Wir betrachten einen Knoten $K \subset \R^3$ und wählen eine zugehörige Seifert-Fläche $S \subset \R^3$, also $K = \Boundary S$.
Wir bestimmen nun $\Theta_S: H_1(S) \times H_1(S) \to \Z$.

Erinnerung:
\begin{math}
    \pi_1(S, *)
    = \pi_1(F_{g,1}^+)
    = \Gen{a_1, b_1, \dotsc, a_g, b_g & -}
    \isomorphic F_{2g}
\end{math}
Wir setzen die erste Homologiegruppe als
\begin{math}
    H_1(S) := \pi_1(S)_{\text{ab}} \isomorphic \Z^{2g}.
\end{math}
In der Abelsch-Machung können wir den Fußpunkt „vergessen“.
Alternativ: Betrachtung im Henkelmodell, Visualisierung der Abelsch-Machung im Henkelmodell.

Wir wollen messen, wie $S$ in $\R^3$ „verschlungen“ ist.

\begin{prop}
    Jede Seifert-Fläche $S \subset \R^3$ lässt sich aufdicken zu $[-1,1] \times S \xto[injective] \R^3$, mit $\Set{0} \times S \xto{\text{proj.}} S$ durch $(0,x) \mapsto x$.
    Die Orientierung legen wir so fest, dass die positive Normale auf $S$ „von $-1$ nach $1$ zeigt“.
    Dies ist eindeutig bis auf Isotopie bei fester Fläche $S$.

    Jede Kurve $\alpha$ auf $S$ definiert Kurven $\alpha^\uparrow = \Set{1} \times \alpha$, und $\alpha^\downarrow = \Set{-1} \times \alpha$.

    Wir definieren $\Theta_S: H_1 S \times H_1 S \to \Z$ durch
    \begin{math}
        (\alpha, \beta) \mapsto \lk(\alpha^\downarrow, \beta^\uparrow).
    \end{math}
    Dies ist eine bilineare Abbildung.
    Nach Wahl einer Basis $a_1, b_1, \dotsc, a_g, b_g \in H_1 S$ (wie oben) entspricht dies einer Matrix $\Z^{2g \times 2g}$.
\end{prop}

\begin{ex}
    \begin{itemize}
        \item
            Triviales Beispiel:
            Für $S = F_{g,1}^+$ in der Standard-Einbettung
            \begin{math}
                \lk(a^\uparrow, b^\downarrow) = \lk(b^\downarrow, a^\uparrow) &= 0, \\
                \lk(a^\downarrow, b^\downarrow) = \lk(b^\uparrow, a^\downarrow) &= 1.
            \end{math}
            Wir erhalten für $\Theta_S \in \Z^{2g}$ eine Blockmatrix mit Blöcken
            \begin{math}
                \Matrix*{0 & 1 \\ 0 & 0} \in \Z^{2 \times 2}.
            \end{math}
        \item
            Twist-Knoten $T_k$ mit $2k$ Kreuzungen, entspricht $k$ vollen Drehungen der Seifert-Fläche $S$, $\Boundary S = T_k$.
            \begin{math}
                \lk(a^\downarrow, b^\uparrow) &= 0, \\
                \lk(b^\downarrow, a^\uparrow) &= -1, \\
                \lk(a^\downarrow, a^\uparrow) &= -k, \\
                \lk(b^\downarrow, b^\uparrow) &= -1,
            \end{math}
    \end{itemize}
\end{ex}

Es stellt sich die Frage, ob $\det(\Theta_S)$ invariant ist.
Dies gilt leider nicht.
Für folgende Ausdrücke jedoch schon:
\begin{itemize}
    \item
        $|\det(\Theta_S + \Theta_S^T)|$,
    \item
        $\sgn(\Theta_S + \Theta_S^T)$,
    \item
        $\det(q^{-1}\Theta_S - q\Theta_S^T) \in \Z[q^{\pm 1}]$.
\end{itemize}









