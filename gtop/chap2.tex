% Kapitel B
\chapter{Knotengruppen}

\Timestamp{2015-05-06}

Ziel: Zu Knoten $K \subset \R^3$ wollen wir $\pi_1(\R^3 \setminus K, *)$ berechnen und nutzen.


% §B1
\section{Erinnerung: Präsentation von Gruppen}

\begin{ex}
    \begin{itemize}
        \item
            Zyklische Gruppe:
            \begin{math}
                G = \Set{g, g^2, g^3, \dotsc, g^n = 1},
            \end{math}
            wobei $g^i \neq g^j$ für $0 \le i < j \le n$.
            Wir nutzen dafür die Schreibweise
            \begin{math}
                G = \<g | g^n = 1\>
                = \<g | g^n\>.
            \end{math}
            Dann erhalten wir den Gruppenisomorphismus $\Z / n \to G$, $k + n\Z \mapsto g^k$.
        \item
            Unendliche zyklische Gruppe:
            \begin{math}
                G = \Set{g^k & k \in \Z},
            \end{math}
            mit $g^i \neq g^j$ für $i \neq j$ in $\Z$.
            Wir nutzen die Schreibweise
            \begin{math}
                G = \< g | - \>.
            \end{math}
            Dann haben wir den Gruppenisomorphismus $\Z \to G, k \mapsto g^k$.
        \item
            Wir wollen folgende Notationen nutzen können:
            \begin{math}
                G = \Gen{a,b & ab = ba}
            \end{math}
            Wir haben einen Gruppenisomorphismus $Z^2 \to G$, $(k,l) \mapsto a^kb^l$.
    \end{itemize}
\end{ex}

\subsection{Freie Gruppen}

\begin{df}
    Sei $∈(G, \cdot)$ eine Gruppe, $S \subset G$.
    Die von $S$ erzeugte Untergruppe ist
    \begin{math}
        \<S\> = \Set{s_1^{e_1} \dotsc s_n^{e_n} & n \in \N, s_i \in S, e_i \in \Z}.
    \end{math}
    Wir nennen $s_1^{e_1} \dotsc s_n^{e_n}$, genauer $(s_1,e_1; \dotsc; s_n, e_n) \in (S\times \Z)^n$ ein \emphdef{Wort} über $S$.
    Ein Wort heißt \emphdef{reduziert}, wenn $s_i \neq s_{i+1}$ und $e_i \neq 0$.
\end{df}

\begin{df}
    Eine Gruppe $G$ heißt \emphdef{frei}, über $S \subset G$, wenn sich jedes $g \in G$ eindeutig schreiben lässt als reduziertes Wort über $S$.
\end{df}

\begin{ex}
    \begin{itemize}
        \item
            $G \isomorphic \Z/5$ ist nicht frei über $S = \Set{g}$, weil $1 = g^0 = g^5 = g^{10} = \dotsc$.
        \item
            $G \isomorphic \Z$ ist frei über $S = \Set{g}$.
        \item
            $G \isomorphic \Z^2$ ist nicht frei über $S = \Set{a,b}$, denn $ab = ba$, also
            \begin{math}
                (a,1;b,1) \neq (b,1;a,1).
            \end{math}
    \end{itemize}
\end{ex}

\begin{st}
    Zu jeder Menge $S$ existiert eine freie Gruppe $F(S) = \<S| - \>$ über $S$.
    \begin{proof}
        Übung.
    \end{proof}
\end{st}

\begin{st}[universelle Abbildungseigenschaft]
    Eine Gruppe $F$ ist genau dann frei über $S \subset F$, wenn gilt:
    zu jeder Abbildung $f: S \to G$ in eine Gruppe $G$ existiert genau ein Gruppenhomomorphismus $h: F \to G$ mit $h|_S = f$.
    %\begin{note}
    %    \Hom(F,G) \stack\isomorphic\to \App(S,G),
    %    h \mapsto h|_S.
    %\end{note}
    \begin{proof}
        Sei $F$ frei über $S$, dann besteht $F$ aus reduzierten Wörtern der Form $s_1^{e_1} \dotsc s_n^{e_n}$.
        Setze $h: F \to G$ durch $h(s_1^{e_1} \dotsc, s_n^{e_n}) = f(s_1)^{e_1} \dotsc f(s_n)^{e_n}$, dies ist die einzige Möglichkeit, eine solche Abbildung zu definieren.
        Sie ist wohldefiniert und multiplikativ.

        Die Umkehrung ist abstract general nonsense:
        Angenommen $f$ besitzt die universelle Abbildungseigenschaft.
        Dann existiert genau ein Gruppenhomomorphismus $h: F \to F(S)$ mit $h|_S = \id_S$ und genau ein $k: F(s) \to F$ mit $k|_S = \id_S$.
        Für diese gilt $k \circ h = \id_F$ und $h \circ k = \id_{F(S)}$.
    \end{proof}
\end{st}

\begin{df}
    Sei $S$ eine Menge, $F = F(S)$ eine freie Gruppe über $S$.
    Sei $R \subset F$ eine Menge von reduzierten Gruppenwörtern über $S$.
    Wir nennen $(S, R)$ eine \emphdef{Präsentation} mit Erzeugern $S$ und Relationen $R$.
    Die hierdurch \emphdef{prästentierte Gruppe} ist
    \begin{math}
        \<S |R\> := F / \<R^F\>.
    \end{math}
    Hierbei ist $\<R^F\>$ die von $R$ normal erzeugte Untergruppe in $F$, d.h.
    \begin{math}
        \<R^F\> = \Gen{ r^f & r \in R, f \in F }
    \end{math}
    wird erzeugt von allen Konjugierten von $r \in R$ in $F$.
    Dies ist die kleinste normale Untergruppe, die $R$ enthält.
\end{df}

\begin{ex}
    \begin{enumerate}[1)]
        \item
            Für $R = \emptyset$ ist $\<S | \emptyset\> = \Gen{S & -} = F(S)$.
        \item
            $\Gen{a & -} \leftarrow \Z, a^k \mapsfrom k$,
        \item
            $\Gen{a & a^n} \leftarrow \Z /n, a^k \mapsfrom k$.
        \item
            $\Gen{a, b & aba^{-1}b^{-1}} \leftarrow \Z^2, a^kb^l \mapsfrom (k,l)$
        \item
            Zopfgruppen, symmetrische Gruppen
    \end{enumerate}
\end{ex}

\begin{st}[Universelle Abbildungseigenschaft]
    Sei $(S,R)$ wie oben, $f : S \to G$.
    Dann sind äquivalent:
    \begin{enumerate}[1)]
        \item
            Der Gruppenhomomorphismus $h: F(S) \to G$ mit $h|_S = f$ erfüllt $h(R) = \Set{1}$ (und faktorisiert somit).
        \item
            Es existiert ein Gruppenhomomorphismus $\_h: \<S|R\> \to G$ mit $\_h \circ q = f$ ($q$ sei hierbar Quotientenhomomorphismus).
    \end{enumerate}
    \begin{proof}
        Leichte Übung.
    \end{proof}
\end{st}

\begin{prop}
    Jede Gruppe $G$ erlaubt eine Präsentation, d.h. ein Tripel $(S,R,h)$ mit $h: \<S | R\> \stack\isomorphic\to G$.
\end{prop}

\begin{ex}
    Sei $G = \Z / R = \Set{0,1,2,3,4}$.
    $h: \< a | a^5\> \to \Z^5$, $a \mapsto 1$ (wohldefiniert, surjektiv, injektiv).
    Aber auch $k: \<a|a^5\> \to \Z^5$, $a \mapsto 2$ ist ein Gruppenisomorphismus.
\end{ex}

Zu einer gegebenen Gruppe $G$ gibt es stets unendlich viele Präsentationen!
Die folgenden Tietze-Operationen ändern die Präsentation, nicht aber die präsentierte Gruppe.
\begin{enumerate}[(T1)]
    \item
        Hinzufügen/Entfernen einer redundanten Relation: $(S,R) \leadsto (S,R')$ mit $R' = R \cup \Set{r}$, $r \in \<R^F\> \setminus R$.
    \item
        Hinzufügen/Entfernen eines redundanten Erzeugers: $(S,R) \leadsto (S',R')$ mit $S' = S \dotcup \Set{s}$, $R' = R \cup \Set{s^{-1} w}$, $w \in \<S\>$.
\end{enumerate}

\begin{st}[Tietze, 1908]
    Zwei (endliche) Präsentationen $(S,R)$ und $(S',R')$ präsentieren genau dann isomorphe Gruppen, wenn sie sich durch (endlich viele) Tietze-Transformationen ineinander überführen lassen.
    \begin{enumerate}[(T1)]
        \item
            $(S,R) \leftrightarrow (S, R \sqcup R')$ mit $R' \subset \< R^{F(S)}\>$,
        \item
            $(S,R) \leftrightarrow (S \sqcup S', R \cup R')$ mit $R' = \Set{s'w_{s'}^{-1} & s' \in S'}$ und $w_{s'} \in F(S)$.
    \end{enumerate}
    Genauer gilt: jeder Isomorphismus $\Gen{S&R} \isomorphic \Gen{S'& R'}$ lässt sich durch (T1), (T2) erzeugen.
    \begin{proof}
        \begin{seg}{uiae}
            \begin{enumerate}[(T1)]
                \item
                    Es gilt $\Gen{S & R} = \Gen{S & R \sqcup R'}$. 
                \item
                    $\Gen{S & R} \isomorphic \Gen{S \sqcup S' & R \sqcup R'}$.
                    $h: S \injto S \sqcup S'$ induziert $\_h$.
                    $k: F(S \sqcup S') \to F(S)$ mit $s' \mapsto w_{s'}$ induziert $\_k$.
                    \[
                        \begin{tikzcd}
                            F(S) \ar[r,"h"] \ar[d] \ar[dr] & F(S \sqcup S') \\
                            \<S \;|\; R\> \ar[r,"\_h"] & \<S \sqcup S' \;|\; R \sqcup R'\>.
                        \end{tikzcd}
                    \]
                    Damit gilt $\_k \circ \_h = \id_{\Gen{S & R}}$ und $\_h \circ \_k = \id_{\Gen{S \sqcup S' & R \sqcup R'}}$.
            \end{enumerate}
        \end{seg}
        \begin{seg}{dtrn}
            Seien $\_h: \Gen{S & R} \to \Gen{S' & R'}$ und $\_k: \Gen{S' & R'} \to \Gen{S & R}$ zueinander inverse Isomorphismen.
            Diese heben wir hoch zu $h: F(S) \to F(S')$ und $k: F(S') \to F(S)$.
            \begin{math}
                \begin{tikzcd}
                    F(S) \ar[r,"h",yshift=-0.2em,swap] \ar[d] \ar[dr] & F(S \sqcup S') \ar[l,"k",yshift=0.2em,swap] \\
                    \{S \;|\; R\} \ar[r,"\_h"] & \{S' \;|\; R'\}.
                \end{tikzcd}
            \end{math}
            Wir können $S \cap S' = \emptyset$ annehmen.
            Wir betrachten $S^* = S \sqcup S'$ und $R^* = R \sqcup R' \sqcup R_h \sqcup R_k$ mit
            \begin{math}
                R_h &:= \Set{sh(s)^{-1} & s \in S} \\
                R_k &:= \Set{s'k(s')^{-1} & s' \in S'}
            \end{math}
            Wir erhalten $\phi: \Gen{S & R} \stack\sim\to \Gen{S^* & R^*}$ durch
            \begin{math}
                \Gen{S & R} \xrightarrow{(T2)} \Gen{S \sqcup S' & R \sqcup R_k}
                \xrightarrow{(T1)} \Gen{S^* & R \sqcup R_k \sqcup R'}
                \xrightarrow{(T1)} \Gen{S^* & R \sqcup R_k \sqcup R' \sqcup R_h}.
            \end{math}
            Ähnlich $\psi: \Gen{S' & R'} \stack\sim\to \Gen{S^* & R^*}$ durch
            \begin{math}
                \Gen{S' & R'} \xrightarrow{(T2)} \Gen{S \sqcup S' & R \sqcup R_h}
                \xrightarrow{(T1)} \Gen{S^* & R \sqcup R_h \sqcup R'}
                \xrightarrow{(T1)} \Gen{S^* & R \sqcup R_h \sqcup R' \sqcup R_k}.
            \end{math}
            Schließlich gilt
            \begin{math}
                \psi^{-1} \circ \phi &= \_h: \Gen{S & R} \homto \Gen{S' & R'}, \\
                \phi^{-1} \circ \psi &= \_k: \Gen{S' & R'} \homto \Gen{S & R}.
            \end{math}
        \end{seg}
    \end{proof}
\end{st}

\Timestamp{2015-05-11}

Ziel: Zu jedem Knoten $K \subset \R^3$ wollen wir die Knotengruppe $\pi_K := \pi_1(\R^3 \setminus K, *)$ „berechnen“, d.h. präsentieren und auswerten.


% B2
\section{Wirtinger-Präsentation von Knotengruppen}


Stelle $K$ durch ein ebenes Diagramm $D$ dar, $K$ und $D$ seien orientiert.
Erzeuger sind die Bögen $x_1, \dotsc, x_n$ von $D$.
Relationen sind Kreuzungen: $x_ix_j = x_jx_{i+1}$, $x_i^{-1}x_i x_j = x_{i+1}$, $x_jx_i = x_{i+1}x_j$, $x_jx_ix_j^{-1} = x_{i+1}$, oder zusammengefasst
\begin{math}
    x^{-\eps(i)}_{j(i)} x_i x_{j(i)}^{\eps(i)} = x_{i+1},
\end{math}
die Daten $\eps: \Set{1,\dotsc, n} \to \Set{\pm 1}$ und $j: \Set{1, \dotsc, n} \to \Set{1, \dotsc, n}$ liest man leicht am Diagramm ab.

\begin{df}
    Wir setzen
    \begin{math}
        \pi_D := \Gen{x_1, \dotsc, x_n & x_{j(i)}^{-\eps(i)} x_i x_{j(i)}^{\eps(i)} = x_{i+1}, i = 1,\dotsc, n}
    \end{math}
\end{df}

\begin{ex}
    \begin{itemize}
        \item
            $\pi_{\KnotTriv} = \Gen{x_1 & -} \isomorphic \Z$.
        \item
            \begin{math}
                \pi_{\KnotKlee} &= \Gen{a,b,c & ac = cb, cb = ba, ba = ac} \\
                &= \Gen{a,b,c & a^c = b, b^a = c, c^b = a}
            \end{math}
    \end{itemize}
\end{ex}

\begin{note}
    $\pi_D$ ist unendlich, denn wir haben die Abelschmachung $(\pi_D,\cdot) \to (\Z,+), x_i \mapsto 1$.

    Für Verschlingungen, bzw. Schlingel mit $n$ Komponenten entsprechend $\pi_D \to \Z^n$.
\end{note}

\begin{ex}
    $\pi_{\KnotKlee}$ ist nicht abelsch, also $\pi_{\KnotKlee} \not\isomorphic \pi_{\KnotTriv}$.
    \begin{proof}
        Betrachte $h: \pi_D \to S_3$ mit $a \mapsto (12)$, $b \mapsto (23)$, $c \mapsto (13)$.
        Man rechnet:
        \begin{math}
            a^c &= (23) = b, &
            b^a &= (13) = c, &
            c^b &= (12) = a,
        \end{math}
        Da $h$ surjektiv und $S_3$ nicht abelsch, ist auch $\pi_D$ nicht abelsch.
    \end{proof}
\end{ex}

Noch zu zeigen: $\pi_D$ ist eine Invariante des Knotentyps.
Es gibt folgende Möglichkeiten:
\begin{enumerate}[1)]
    \item
        Reidemeister-Züge verändern die Gruppe nicht: die Präsentationen unterscheiden sich um Tietze-Transformationen.
    \item
        Es existiert ein Isomorphismus $\pi_D \isomorphic \pi_1(\R^3 \setminus K, *)$.
\end{enumerate}


% todo: Appendix:

%\subsection*{Erinnerung: Permutationen, Zykelschreibweise, Konjugation}
%
%In $S_n$ nutzen wir folgende Schreibweise:
%Seien $i_1, \dotsc, i_l \in \Set{1, \dotsc, n}$ verschieden.
%Definiere die Zykel: $c := (i_1, \dotsc, i_l)$ durch $i_1 \mapsto i_2 \mapsto \dotsb \mapsto i_l \mapsto i_1$.
%
%\begin{prop}
%    Jedes $\sigma \in S_i$ ist Produkt disjunkter Zykel.
%    Dieses ist eindeutig bis auf Umordnung der Faktoren.
%\end{prop}
%
%\begin{ex}
%    $\sigma = (1352)(476) = (476)(1352)$.
%\end{ex}
%
%Für die Konjugation gilt
%\begin{math}
%    (i_1, \dotsc, i_l)^\sigma &=
%    \sigma^{-1} (i_1, \dotsc, i_l) \sigma \\
%    &= (\sigma(i_1), \dotsb, \sigma(i_l))
%\end{math}
%

\begin{st}[Wirtinger, <1900]
    Es existiert ein Gruppenisomorphismus $\pi_D \to \pi_1(\R^3 \setminus K, *)$.
    \begin{proof}
        \begin{enumerate}[1),start=0]
            \item
                Konstruktion von $h$:
                Wie in der Skizze, ordnen wir jedem Bogen $b_i$ von $D$ einen (polygonalen) Weg $\gamma_i: [0,1] \to \R^3 \setminus K$ zu.
                Dieser definiert ein Gruppenelement $w_i = [\gamma_i] \in \pi_1(\R^3 \setminus K)$.
                An jeder Kreuzung gilt die Wirtinger-Relation:
                \begin{math}
                    x_i x_j = x_j x_{i+1}
                \end{math}
                und analog die anderen.
                Wir haben nun einen Gruppenhomomorphismus $h: \pi_D \to \pi_1(\R^3 \setminus K, *)$ mit $x_i \mapsto w_i = [\gamma_i]$.
            \item
                $h$ ist surjektiv, d.h. $\pi_1(\R^3 \setminus K, *)$ wird erzeugt von $w_1, \dotsc, w_n$:

                Wir nutzen die polygonale Fundamentalgruppe (mittels polygonaler Approximation)
                \begin{math}
                    \pi_1(\R^3 \setminus K, *)
                    = \frac{\Set{\text{Schleifen}}}{\text{Homotopie}}
                    = \frac{\Set{\text{polygonale Schleifen}}}{\text{polygonale Homotopie}}.
                \end{math}
                Ohne Einschränkung betrachten wir also polygonale Schleifen $\gamma$ in $\R^3 \setminus K$.
                Zu zeigen ist $\gamma \homotopic \gamma_{i_1}^{e_1} \dotsb \gamma_{i_l}^{e_l}$.
                Trick: Betrachte den „Schatten“ des Knotens $K \subset \R^3$ unter senkrecht von oben einfallendem Licht.
                Genauer: Zu $K \subset \R^3$ ist der Schatten $\hat K = \Set{(x,y,z) \in \R^3 & \exists z' \ge z: (x,y,z') \in K}$.
                Dies ist die Vereinigung über alle Schatten $\hat A$ der Kanten $A$ von $K$.
                \begin{prop}
                    Es gilt $\R^3 \setminus \hat K \homequiv *$
                    \begin{proof}
                        Übung: explizite Formel, vgl. Sternförmig bei Zentralprojektion.
                    \end{proof}
                \end{prop}
                Ablesen an $w = [\gamma]$ eines Wortes in $w_i = [\gamma_i]$.
                Wir nehmen an, dass $\gamma$ die Wände $\hat A$ transversal im Inneren trifft.
                Jeder Durchgang liefert einen Erzeuger $w_i^{\pm 1}$.
                Damit gilt $\gamma = \gamma_{i_1}^{e_1} \dotsb \gamma_{i_l}^{e_l}$.
            \item
                $h$ ist injektiv, d.h. die Wirtinger-Relationen erzeugen alle Relationen.
                Polygonale Homotopie:
                \begin{enumerate}[1)]
                    \item
                        Keine Wand wird getroffen: Keine Änderung des Wortes.
                    \item
                        Eine Wand wird geschnitten:
                        Zwei Unterfälle: Jeweils keine Änderung des Wortes.
                    \item
                        Schatten einer Kreuzung wird geschnitten.
                        Dank Wirtinger-Relation keine Änderung des Wortes.
                \end{enumerate}
        \end{enumerate}
    \end{proof}
\end{st}


\Timestamp{2015-05-13}

% B3
\section{Unterscheidung zahmer Knoten}


Für den Kleeblatt-Knoten $K$ hatten wir
\begin{math}
    \pi_K &\isomorphic \Gen{a,b,c & a^c = b, b^a = c, c^b = a} \\
    &= \Gen{a,b,c & b = a^c, c = b^a, a = c^b, b = (a^{-1}ba)a(a^{-1}ba), a = b^{-1}(a^{-1}ba)b} \\
    &= \Gen{a,b,c & b = a^{-1}b^{-1}a b a, c = a^{-1}b a, a = b^{-1} a^{-1} b a b} \\
    &= \Gen{a,b,c & aba = bab, c = a^{-1}ba} \\
    &= \Gen{a,b,c & aba = bab, c = a^{-1}ba} \\
    &\isomorphic \Gen{a,b & aba = bab} \\
    &\isomorphic B_3.
\end{math}

Wie löst man das Wortproblem in dieser speziellen Gruppe?
\begin{itemize}
    \item
        Wir kennen die Abelschmachung $\alpha: \pi_k \to \Z, a, b \to 1$.
    \item
        Wir kennen $\phi: \pi_k \to \SL_2 \Z = \Gen{X,y & XYX = YXY, (XYX)^4 = 1}$ mit $a \mapsto X = \Matrix{1 & 1 \\ 0 & 1}$, $b \mapsto Y = \Matrix{1 & 0 \\ -1 & 1}$.
        Es gilt $\ker \phi = \Gen{(aba)^4}$ (normal erzeugt!).
        In $B_3$ ist
        \begin{math}
            (\sigma_1 \sigma_2 \sigma_1)^4
            = (\sigma_1 \sigma_2 \sigma_1 \sigma_2 \sigma_1 \sigma_2)^2
            = z_2^2.
        \end{math}
        Folglich $\ker(\phi) \subset Z(\pi_k)$.
        Also ist $\alpha \times \phi : \pi_k^2 \to \Z \times \SL_2(\Z)$ injektiv.
\end{itemize}

\begin{ex}
    Ebenso für Verschlingungen:
    Sei $L$ die Hopf-Verschlingung.
    \begin{math}
        \pi_L = \pi_1 (\R^3 \setminus L, *)
        = \Gen{a,b & a^b = a, b^a = b}
        = \Z^2.
    \end{math}
\end{ex}

\begin{ex}
    Für den Schlingel $T$:
    \begin{math}
        \pi_T
        &= \Gen{a,b,a',b' & b^a = b', a^{b'} = a'} \\
        &= \Gen{a,b,a',b' & b^a = b', a' = a^{-1} b^{-1}a b a} \\
        &= \Gen{a,b & -}
    \end{math}
    Für Zöpfe gilt $\pi_T \isomorphic F_n$ (Übung!).
    Dazu später mehr.
\end{ex}

\subsection{Meridian und Longitude}

Betrachte wieder einen Knoten $K$ mit Knotengruppe $\pi_K = (\R^3 \setminus K, *)$.
$m_k = [\mu_k]$, $l_k = [\lambda_k]$.

Sei $K \subset \R^3$ ein orientierter Knoten.
Es existiert eine Tubenumgebung $f: \S^1 \times \D^1 \injto \R^3$ (polygonal/glatt) mit $f(\S^1 \times \Set 0) = K$ (inklusive Orientierung) und
\begin{math}
    \mu_k, \lambda_k: [0,1] &\to \R^3 \setminus K \\
    \mu_k: t &\mapsto f(1, e^{2\pi i t}) \\
    \lambda_k: t & \mapsto f(e^{2\pi i t}, 1).
\end{math}
mit den Bedingungen $\lk(K, \mu_K) = +1$ und $\lk(K, \lambda_K) = 0$.
Solche Tubenumgebungen sind eindeutig bis auf Isotopie.
Insbesondere sind $m_k = [\mu_K]$ und $l_k = [\lambda_K]$ in $\pi_K$ in $\pi_K$ wohldefiniert.

\begin{df}
    Betrachte $K \subset \R^3 \subset \S^3 = \R^3 \cup \Set{\infty}$.
    \begin{math}
        E_K := \S^3 \setminus f(\S^1 \times \B^2) \subset \S^3
    \end{math}
    ist eine kompakte $3$-Mannigfaltigkeit mit Rand $\Boundary E \isomorphic \S^1 \times \S^1$.
    Sie heißt \emphdef{Knotenaußenraum} zu $K$.

    Es gilt Homotopie-Äquivalenz: $\S^3 \setminus K \homequiv E_k$.
    $\pi_1(\S^1 \setminus K, *) \isomorphic \pi_1(E_, *) \leftarrow \pi_1(\S^1 \times \S^1, *) \isomorphic \Z^2$.

    Das Tupel $(\pi_K, m_k, l_k)$ heißt in dieser Vorlesung „Pomelo“.
\end{df}

\paragraph{Ablesen in der Wirtinger-Präsentation:}

$l_k$ lässt sich direkt im Knotendiagramm ablesen (Skizze).
Man unterscheidet zwischen globaler und lokaler Korrektur (entsprechend, wo die zusätzlichen Windungen von $l_k$ um $x_1$ kodiert werden).

\begin{ex}
    Für den Kleeblattknoten:
    \begin{math}
        \pi_k = \Gen{a,b,c & a^c = b, b^a = c, c^b = a}.
    \end{math}
    Wir erkennen die Repräsentation der globalen und lokalen Korrektur:
    \begin{math}
        m_k &= a, \\
        l_k &= c a b a^{-3} \\
        &= c a c^{-3} b \\
        &= c b^{-2} a c^{-1} b \\
        &= a^{-1} c b^{-1} a c^{-1} b
    \end{math}
\end{ex}

Wir vermuten
\begin{itemize}
    \item
        Obversion/Spiegelung $K^\times$: Orientierungsumkehr von $\R^3$: $(\pi_K, m_K^{-1}, l_K)$.
    \item
        Reversion $K^!$: Orientierungsumkehr von $K$: $(\pi_K, m_K^{-1}, l_k^{-1})$.
    \item
        Inversion $K^*$: Orientierungsumkehr von $K$ und $\R^3$: $(\pi_K, m_K, l_K^{-1})$.
\end{itemize}

%Betrachte $\rho: \pi_K \to A_5$ mit $a = m_k \mapsto (12345) = x$.
%$A_5$ hat 60 Elemente.
%$x$ hat 12 Konjugierte in $A_5$.
%Versuche:
%$
%    b \mapsto y = (12453)
%$.
%Setze $c := b^a = (23514)$, dann ist
%\begin{math}
%    a^c = (43521) \neq b.
%\end{math}
%Weitere $5 + 1$ klappen nicht, $5$ weitere klappen.

\Timestamp{2015-05-18}

In $A_5$ hat $x$ genau $12$ Konjugierte.
Setze $a \mapsto x$.
\begin{enumerate}[1.]
    \item
        Für $b \mapsto x$ erhalten wir die triviale (abelsche) Darstellung $\pi_k \to \Z \to G$, $m_k \mapsto 1 \mapsto x$.
    \item
        Setze $b \mapsto (12453)$.        
        Es gilt
        \begin{math}
            c &= b^a \mapsto (23514),
            b &{\stack ?=} a^c \mapsto (43521), 
        \end{math}
        ein Widerspruch.
        Ebenso für $b \mapsto (23514)$, $(34125)$, $(45231)$, $(51342)$.
    \item
        Probiere $b \mapsto (15243)$.
        Es gilt
        \begin{math}
            c &= b^a \mapsto (21354),
            b &{\stack ?=} a^c \mapsto (31524) = (15243),
        \end{math}
        ein Erfolg (auch redundant $c^b = a \mapsto (45123)$).
        Ebenso für $b \mapsto (21354)$, $(32415)$, $(43521)$, $(54132)$.
    \item
        Für $b \mapsto x^{-1} = (54321)$ gilt
        \begin{math}
            c &= b^a \mapsto x^{-1},
            b &{\stack ?=} a^c \mapsto x,
        \end{math}
        ein Widerspruch.
\end{enumerate}
Also ist $|\Hom(\pi_k,m_k,G,x)| = 6$.

Was ist $\rho(l_k)$?
\begin{math}
    l_k = bac a^{-3} & \stack{1.}\mapsto 1 \\
    &\stack{3.}\mapsto (21354)(12345)(15243)(13524) = (15432) = x^{-1}
\end{math}

\begin{df}
    Zu einer endlichen Gruppe $G$ und $x \in G$ definieren wir die \emph{Färbungszahl} $F_G^x: \scr K \to \N$ durch
    \begin{math}
        F_G^x(K) := |\Hom(\pi_k,m_k;G,x)|
    \end{math}
    und das Färbungspolynom $P_G^x: \scr K \to \Z[G]$ durch
    \begin{math}
        P_G^x(K) := \sum_{\rho: (\pi_k, m_k) \to (G,x)} \rho(l_k)
    \end{math}
\end{df}

\begin{ex}
    Sei $K$ der Kleeblattknoten, $(G,x) = (A_5, (12345))$.
    Es gilt wie oben berechnet
    \begin{math}
        F_G^x(K) &= 6, &
        P_G^x(K) &= 1 + 5x^{-1}.
    \end{math}
    Für den inversen Knoten $K^*$ gilt $(\pi_{K^*}, m_{K^*}, l_{K^*}) = (\pi_K, m_K, l_K^{-1})$, also $P_G^x(K^*) = 1 + 5x$ (beachte: $x \neq x^{-1}$ in $A_5$).

    Für den obversen Knoten $K^x: (\pi_K, m_K^{-1}, l_K)$ erhalten wir
    \begin{math}
        (\pi_K, m_K, l_K) \stack{\rho}\to (A_5, x) \xrightarrow[(15)(24)]{\text{konj.}} (A_5, x^{-1}).
    \end{math}
    Damit ist $F_G^x(K^x) = 6$ und $P_G^x(K^x) = 1 + 5x$.
    Für den reversen Knoten $K^!: (\pi_K, m_K^{-1}, l_K^{-1})$ damit
    \begin{math}
        P_G^x(K^!) = 1+ 5x^{-1}.
    \end{math}
\end{ex}

\begin{kor}
    Für die Kleeblattschlinge $K$ gilt $K \neq K^x$, d.h. die Kleeblattschlinge ist nicht chiral.
    \begin{note}
        Übung: $K = K^!$, d.h. sie ist reversibel.
    \end{note}
\end{kor}

\begin{note}
    Knotengruppendarstellungen entsprechen Färbungen.
    Speziell: $p$-Färbungen entsprechen Darstellungen in die Diedergruppen $(D_p,s)$.

    Sei $n \ge 3$, $P_n = [e^{2\pi i \frac{k}{n}}, k =1, \dotsc, n]$.
    Die Diedergruppe ist die Isometriegruppe des regelmäßigen $n$-Ecks:
    \begin{math}
        D_n &:= \Isom(P_n) \\
        &= \left\{
            \begin{aligned}
                r_k &= \Matrix{\cos(2\pi k /n) & -\sin(2\pi k/n) \\ \sin(2\pi k/n) & \cos(2\pi k/n)}, \\
                s_k &= \Matrix{\cos(2\pi k/n) & \sin(2\pi k/n) \\ \sin(2\pi k/n) & -\cos(2\pi k/n)}
            \end{aligned}
            \middle|\; k = 0, \dotsc, n-1
            \right\}
    \end{math}
    Auf Eckpunkte sind die Elemente Permutationen: $r_K, s_K: \Z/n \to \Z/n$, $r_K(x) = k +x$, $s_K(x) = k - x$.
\end{note}

\begin{st}
    Wir erhalten die Präsentationen
    \begin{math}
        D_n &= \Gen{s_0,s_1 & s_0^2, s_1^2, (s_1s_0) = r_1} \\
        &= \Gen{s,r & s^2, r^n, srs = r^{-1}}
    \end{math}
    \begin{proof}
        Übung!
    \end{proof}
\end{st}

Für Färbungen benötigen wir die Konjugationen
\begin{math}
    s_a^{s^b} = s_b^{-1} s_a s_b: \Z/n &\to \Z/n \\
    x &\mapsto b-(a-(b-x)) = (2b-a)-x = s_{2b-a}(x).
\end{math}
Diese Vorschrift entspricht der $n$-Färbungsregel ($c = 2b -a$ an einer Kreuzung).

\begin{ex}
    Der Knoten $8_{17}$ ist der kleinste nicht-reversible Knoten.
    \begin{proof}
        Hierzu nutzen wir die Mathieu-Gruppe $M_{11}$ der Ordnung 7920.
        \begin{math}
            G= \Gen{x,y} \le A
        \end{math}
        mit
        \begin{math}
            x &= (\mathrm{abcdefghijk}), &
            y &= (\mathrm{abcejikdghf})
        \end{math}
        Für $K = 8_{17}$ und $(G,x)$ finden wir
        \begin{math}
            P_G^x(K) &= 1 + 11x^5 + 11x^6, \\
            P_G^x(K!) &= 1
        \end{math}
        (es gilt $K = K^*$, amphichiral)
    \end{proof}
\end{ex}


\subsection{Wichtige Sätze zu Knotengruppen}

Ist $K$ trivial, so ist $\pi_K \isomorphic \Z$ und insbesondere abelsch.
Es folgt $l_k = 1$.

Gilt umgekehrt $l_k = 1$ impliziert $K$ trivial?

\begin{st}[Papakyriakopoulos, 1957, Dehnsches Lemma, 1910]
    Genau dann ist $K$ trivial, wenn $l_K = 1$ in $\pi_K$ gilt.
\end{st}

Für zwei isotopie Knoten $K \sim K'$ ist
\begin{math}
    (\pi_K, m_K, l_K) \isomorphic (\pi_{K'}, m_{K'}, l_{K'})
\end{math}
Gilt auch die Umkehrung?

\begin{st}[Waldhausen, 1968]
    Aus $(\pi_K, m_K, l_K) \isomorphic (\pi_{K'}, m_{K'}, l_{K'})$ folgt $K \sim K'$.
\end{st}

Existiert ein Homomorphismus $\phi: \pi_K \to G$ für $G$ endlich, $\phi(x) \neq 1$, dann ist offenbar $x \neq 1$ in $\pi_K$.
Gilt auch die Umkehrung?

\begin{st}[Thurston, 1983]
    Jede Knotengruppe $\pi_K$ ist residuell endlich, d.h für alle $x \in \pi_K$ mit $x \neq 1$, existiert ein Homomorphismus $\phi: \pi_k \to G$ in eine endliche Gruppe $G$ mit $\phi(x) \neq 1$.
\end{st}

\begin{kor}
    Das Problem, zu einem (zahmen) Knoten zu entscheiden, ob $K$ trivial ist oder nicht ist, ist algorthmisch lösbar.
\end{kor}

\begin{alg}
    \emph{Input}: $K$,\\
    \emph{Output}: Boolean: $K$ trivial oder nicht.
    \begin{enumerate}[1.]
        \item
            Präsentiere $\pi_K, m_K, l_K$, $\pi_K = \Gen{S & R}$
        \item
            \begin{enumerate}[a)]
                \item
                    Generiere aus Relationen alle Sequenzen $\Gen{R^{F(S)}} =: N$.
                    Falls $l_k \in N$, dann ist $l_k = 1$ in $\pi_K$ und $K$ trivial.
                \item
                    Generiere alle Homomorphismen $\phi: \pi_K \to G$ in endliche Gruppen.
                    Falls $\phi(l_K) \neq 1$, dann $l_K \neq 1$, also $K$ nicht-trivial.
            \end{enumerate}
    \end{enumerate}
    Dank Thurston (für 2b) ist dies ein Algorithmus.
    \begin{note}
        Zu 2b: Es genügt $G = S_3, S_4, \dotsc$ auszuprobieren, denn jede endliche Gruppe ist in $S_n$ enthalten.
    \end{note}
\end{alg}


\Timestamp{2015-05-20}

% B4
\section{Anwendung auf wilde Knoten}

Erinnerung: Für jeden einfachen Polygonzug $\gamma: [0,1] \injto \R^3$, $\gamma = |P|$, $P = (v_0, v_1, \dotsc, v_n)$ gilt:
\begin{enumerate}[1)]
    \item
        $P$ ist $\Delta$-äquivalent zu $(-e_1, e_1)$,
    \item
        $\R^3 \setminus \gamma([0,1]) \isomorphic \R^3 \setminus \Set 0$,
    \item
        $\pi_1(\R^3 \setminus \gamma([0,1])) = \Set{1}$
\end{enumerate}
Für beliebige (topologische) Einbettungen $\gamma: [0,1] \injto \R^3$ gelten solche Aussagen nicht!

Konstruktion: zahmer Schlingel $T$ (siehe Skizze), unendliche Verknüpfung von $T$ in beide Richtungen (liefert nur eine Komponente).
Wir erhalten $\gamma: (-1,1) \injto \R^3$.
Dank Konvergenz setzt sich dies stetig fort zu $\gamma: [-1,1] \injto \R^3$ mit $\gamma(\pm 1) = p_\pm$ (kompakt/hausdorffsch, injektiv, stetig liefert Einbettung).

Berechnung von $\pi_1(\R^3 \setminus \gamma([-1,1]))$:
Wir erhalten die Relationen $R$:
\begin{math}
    a_{n+1} &= c_{n+1} c_n c_{n+1}^{-1}, \\
    b_n &= c_{n+1} a_n c_{n+1}^{-1}, \\
    c_{n+1} &= b_n b_{n+1} b_n^{-1}, \\
    1 = &= b_n^{-1} a_n c_n
\end{math}

Wir behaupten $\Gen{S & R} \not\isomorphic \Set 1$.
\begin{itemize}
    \item
        Eliminiere $a_n$:
        \begin{math}
            b_n = c_{n+1} \underbrace{c_n c_{n-1} c_n^{-1}}_{=a_n} c_{n+1}^{-1}.
        \end{math}
    \item
        Eliminiere $b_n$:
        \begin{math}
            c_{n+1} =
            c_{n+1}c_n c_{n-1} c_n^{-1} c_{n+1}^{-1}
            c_{n+2}c_{n+1} c_n c_{n+1}^{-1} c_{n+2}^{-1}
            c_{n+1}c_n c_{n-1}^{-1} c_n^{-1} c_{n+1}^{-1}
        \end{math}
        fixme: rechnen
        \begin{math}
            c_{n+1}c_n c_{n-1}^{-1} c_n^{-1} c_{n+1} c_n c_{n-1} c_n^{-1} c_n &= 1\\
            c_n c_{n-1} c_{n+1} c_n &= c_{n+1} c_n c_{n-1}
        \end{math}
\end{itemize}
Es gilt
\begin{math}
    \pi_1(\R^3 \setminus \gamma([0,1]))
    \isomorphic \Gen{c_n, n \in \Z & c_n c_{n-1} c_{n+1} c_n = c_{n+1} c_n c_{n-1}, n \in \Z }.
\end{math}
Weise nun nach, dass diese Gruppe nicht trivial ist.
Betrachte $\phi: G \to A_5$ gegeben durch
\begin{math}
    c_{2k} &\mapsto (12345),
    c_{2k+1} &\mapsto (14235).
\end{math}
Die Relation ist erfüllt, für $n = 2k$:
\begin{math}
    (12345)(14235)(14235)(12345) &= (1)(2)(345), \\
    (14235)(12345)(14235) &= (1)(2)(345)
\end{math}
Für $n = 2k + 1$ Übung.

\begin{kor}
    Es existieren Einbettungen $\S^2 \to \S^3$ sodass das Komplement nicht aus offenen Bällen besteht, sogar $\pi(\S^3 \setminus f(\S^2), *) \neq \Set 1$.
    \begin{proof}
        Mit obigen erhalten wir eine Einbettung $g: (-1,1) \times \S^1 \injto \R^3$, durch Fortsetzung $f: \S^2 \injto \R^3$.
        Die Fundamentalgruppe ist nicht-trivial.
        $f$ ist „einseitig wild“, ähnlich auch beidseitig möglich.
    \end{proof}
\end{kor}

