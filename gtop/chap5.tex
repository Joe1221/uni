% Kap E
\chapter{Das Jones-Polynom}

\Timestamp{2015-06-29}

% §E1
\section{Die Kaufman-Klammer und das Jones-Polynom}


Wir betrachten die Menge $\scr D^{\text{un}}$ der unorientierten Verschlingungsdiagramme (Nummerierung der Komponenten auch irrelevant).

\begin{prop}
    Sei $R$ ein kommutativer Ring mit 1 und $A, B, C \in R$.
    Dann gibt es genau eine Abbildung: $\<\argdot\> : \scr D^{\text{un}} \to R$, sodass gilt
    \begin{enumerate}[1)]
        \item
            $\<O\> = 1$, $\<D \sqcup O\> = \< D \> \cdot C$
            Insbesondere $\<O \dotso O\> = C^{n-1}$.
        \item
            $\<L_+\> = A \< L_1 \> + B \< L_2 \>$
    \end{enumerate}
    \begin{proof}
        \begin{seg}{Eindeutigkeit}
            Seien $\<\argdot\>, [\argdot]: \scr D^{\text{un}} \to R$ zwei Abbildungen mit den geforderten Eigenschaften.
            Induktion über die Kreuzungszahl $\Cr(D)$:
            Für $\Cr(D) = 0$ ist $\<D\> = C^{n-1} = [D]$ dank 1).
            Für $\Cr(D) \ge 1$ ist
            \begin{math}
                \< D L_+ \>
                = A \< L_1 \> + B \< L_2 \>
                = A [ L_1 ] + B [ L_2 ]
                = [D L_+ ].
            \end{math}
        \end{seg}
        \begin{seg}{Existenz}
            Rekursive Konstruktion von $\<\argdot\>: \scr D^{\text{num}} \to R$ (nummerierte Kreuzungen).
            Für $n = 0$:
            \begin{math}
                \< D \> := C^{\#D - 1}.
            \end{math}
            Für $n = 1$:
            \begin{math}
                \< D L_+ \> := A \< L_1 \> + B \< L_2 \>
            \end{math}
            Für $n = 2$:
            \begin{math}
                \< D L_+^1 L_+^2 \>
                &= A \< L_1 L_+ \> + B \< L_2 L_+ \> \\
                &= AA \< L_1 L_1\> + AB \< L_1 L_2 \> + BA \<L_2 L_1\> + BB \< L_2 L_2\> \\
                &= A \< L_+ L_2\> + B \< L_+ L_2\> \\
                &= \< D L_+^2 L_+^1 \>
            \end{math}
            Also ist $\< \argdot\>$ unabhängig von der Nummerierung der Kreuzungen.
            Wir erhalten $\< \argdot \>: \scr D \to R$.
        \end{seg}
    \end{proof}
\end{prop}

Reidemeister-Züge:
Wir erhalten für den R2-Zug die Bedingungen
\begin{math}
    B &= A^{-1}, &&
    C &= -A^2 - A^{-2}
\end{math}
Der R3-Zug ist damit bereits bedient.
Für den R1-Zug erhalten wir jedoch
\begin{math}
    \<K\> = (-A^{-3}) \<L\>
\end{math}
Wie erreichen wir Invarianz unter R1?
Brutale Methode: $-A^{-3} = 1$ algebraisch im Ring erzwingen.

Geschickter: Wir betrachten orientierte Diagramme und den Drall (w?) $w: \scr D^{\text{or}} \to \Z$, $w(D) = \sum_{p \text{ Kreuz}} \eps(p)$.
Der Drall ist invariant unter R2 und R3, aber $w(K) = w(L) + 1$.

\begin{df}
    Für jedes \emph{orientierte} Diagramm $D$ definieren wir das \emphdef{Jones-Polynom}
    \begin{math}
        V(D) = \<D\> \cdot (-A^{-3})^{w(D)}.
    \end{math}
    Dieses ist invariant unter allen R-Zügen.
    \begin{proof}
        Man betrachte den kritischen R1-Zug: Leicht.
    \end{proof}
\end{df}

\begin{st}
    Das Jones-Polynom definiert eine Abbildung $V: \scr L \to \Z[A^{\pm 1}]$.
\end{st}

\begin{ex}
    Für die Kleeblattschlinge erhalten wir
    \begin{math}
        V(3_1^-) = A^4 + A^{12} - A^{16}.
    \end{math}
\end{ex}

\begin{prop}
    Für $L \in \scr L$ gilt $V(L^!) = V(L)$

    Kehrt man die Orientierung von $L_1 \subset L$ um und lässt $L_2 = L \setminus L_1$ unverändert, dann gilt
    \begin{math}
        V(L_1^! \cup L_2)
        = V(L_1 \cup L_2) A^{6 \lk(L_1, L_2)}.
    \end{math}
\end{prop}

\begin{prop}
    Es gilt $V(L^*) = V(L)^*$, wobei $*: \Z[A^{\pm}] \to \Z[A^\pm]$ mit $A \mapsto A^{-1}$.
    \begin{proof}
        Stelle $L$ durch ein Diagramm $D$ dar.
        Die Aussage folgt induktiv aus
        \begin{math}
            w(D^*) &= -w(D) \\
            \<L_+\> &= A \< L_1\> + A^{-1} \<L_2\> \\
            \<L_-\> &= A^{-1} \< L_1\> + A \<L_2\>.
        \end{math}
        Alternativ mit Eindeutigkeit:
        \begin{math}
            [D] := \<D^*\>^*
        \end{math}
        erfüllt die Bedingung für die Kaufman-Klammer, also $[\argdot] = \<\argdot\>$.
    \end{proof}
\end{prop}

\begin{ex}
    \begin{math}
        V(3_1^-) &= A^4 + A^{12} - A^{16}, \\
        V(3_1^+) &= A^{-4} + A^{-12} - A^{-16}.
    \end{math}
    Das Jones-Polynom unterscheidet also links- und rechtshändige Kleeblattschlinge.
    Dies ist neben Färbungspolynom und Signatur unser dritter Beweis für die Chiralität der Kleeblattschlinge.
\end{ex}

\begin{prop}
    \begin{math}
        V(K \# L) = V(K) V(L)
    \end{math}
    (Vorsicht: linke Seite nur für Knoten wohldefiniert)
    \begin{proof}
        Stelle $K$, $L$ durch Diagramme dar, kurz: $D \# D'$.
        Induktion über $\Cr(D')$: für $\Cr(D') = 0$ ist alles klar.

        Siehe Skizze.
    \end{proof}
\end{prop}

\begin{nt}
    Übliche Schreibweisen für das Polynom $V(K)$: in den Variablen $q = -A^{-2}$, bzw. $t = A^{-4} = q^2$.
\end{nt}

\begin{st}
    Die Invariante $V: \scr D^{or} \to \Z[q^\pm 1] \subset \Z[A^{\pm 1}]$ erfüllt $V(0) = 1$ und die Schienenrelation
    \begin{math}
        q^{-2} V(L_+) - q^2 V(L_-) = (q^{-1} - q^{+1}) V(L_0).
    \end{math}
    Dies legt $V$ eindeutig fest und liefert zugleich einen Algorithmus.
    \begin{proof}
        \begin{math}
            A \<L_+\> &= A^2\<L_1\> + A^0 \< L_2\> \\
            A^{-1} \<L_-\> &= A^{-2}\< L_1\> + A^0 \< L_2 \> \\
            \leadsto q^{-2}V(L_+) - q^2 V(L_-) &= A^4 (-A^{-3})^{w+1} \< L_+\> - A^{-4} (-A^{-3})^{w-1} \<L_-\> \\
            &= (-A^{-3})^w \Big( -A \<L_+\> + A^{-1} \<L_-\> \Big) \\
            &= (-A^{-3})^w (A^{-2} - A^2) \< L_2 \> \\
            &= (q^{-1} - q) V(L_0)
        \end{math}
    \end{proof}
    \begin{nt}
        Das Alexander-Polynom hatte die Eigenschaft $\Delta(0) = 1$,
        \begin{math}
            \Delta(L_+) - \Delta(L_-) = (q-q^{-1}) \Delta(L_0).
        \end{math}
        Die Determinante: $\det(0) = 1$,
        \begin{math}
            \det(L_+) - \det(L_-) = 2i \det(L_0).
        \end{math}
        Es gilt
        \begin{math}
            \det = \Delta|_{q\mapsto i}
            = V|_{q \mapsto i}.
        \end{math}
    \end{nt}
\end{st}

\Timestamp{2015-07-01}

Für das Jones-Polynom gibt es zwei Zugänge
\begin{enumerate}[1)]
    \item
        Kaufmann-Klammer, $\<D\> \in \Z[A^{\pm 1}]$, Drall-Normierung $V(D) = \<D\> (-A^{-3})^{w(D)}$.
    \item
        Schienenrelation:
        \begin{math}
            q^{-2} V(L_+) - q^2 V(L_-) = (q^{-1} - q) V(L_0)
        \end{math}
        übliche Konvention: $q := -A^{-2}$, bzw $t = A^{-4} = q^2$.
\end{enumerate}
1) hatten wir bereits behandelt.
2) lässt sich auch als Konstruktion für das Jones-Polynom verwenden.

Vergleiche das Alexander-Polynom: $\Delta: \scr L \to \Z[q]$:
\begin{math}
    \Delta(L_+) - \Delta(L_-) = (q-q^{-1}) \Delta(L_0).
\end{math}
\begin{note}
    $V$ unterscheidet viele Knoten, die $\Delta$ nicht unterscheidet, z.B. $3_1 \neq 3_1^x$.
    Umgekehrt gibt es Knoten $K, L$ mit $\Delta(K) \neq \Delta(L)$, aber $V(K) = V(L)$.
    Beide Invarianten sind also unabhängig.
\end{note}
Ein Defekt, oder eine Eigenschaft beider Polynome ist die Invarianz unter Mutation (Drehung einer passenden Komponente um 180 Grad):

\begin{ex}
    Conway-Knoten $C$ und Kinoshita-Terasaka-Knoten $K$.
\end{ex}

\begin{st}
    Sei $I: \scr D^{\text{or}} \to R$, $a,b,c \in R$, $a,b \in \R^*$ eine Invariante mit $I(0) = 1$ und
    \begin{math}
        a I(L_+) + b I(L_-) = c I(L_0).
    \end{math}
    Dann ist $I$ invariant unter Mutation.
    \begin{proof}
        Induktion über $\Cr(R)$.
        Für $\Cr(R) = 0$ ist dies klar (2 Fälle mit evtl. Kreiskomponenten, die beliebig verschoben werden können).
        Sei nun $\Cr(R) \ge 1$.
        Mache Strang 1 absteigend, dann 2.
        Wenn absteigend, dann klar: (4 Fälle).

        Nun Induktion über die Anzahl $n$ der zu wechselnden Kreuzungen bis absteigend ($n=0$ trivial).
        Es gilt
        \begin{math}
            \text{Skizze}
        \end{math}
    \end{proof}
\end{st}

\begin{ex}
    Für Conway-Knoten und K-T-Knoten $K$ gilt $\Delta(C) = \Delta(K)$, $V(C) = V(K)$.
    Wie unterscheidet man also $K$ und $C$?
    Es geht mit Färbungspolynomen, z.B. mit $G = \PSL_2 \F_7$, $x = \Matrix{0 & 1 \\ -1 & 1}$, $\od(x) = 3$.
    Es ergibt sich
    \begin{math}
        P_G^x(K) &= 1 + 6x, &&
        P_G^x(C) &= 1 + 12x.
    \end{math}
    Ebenso mit $G = A_7$, $x = (1234567)$,
    \begin{math}
        P_G^x(K) &= 1 + 7x^2 + 28x^5 + 18x^6 \\
        P_G^x(C) &= 1 + 7x^2 + 7x^3 + 21x^5 + 14x^6
    \end{math}
    Mit $G = M_{11} \subset A_{11}$ findet man sogar $K \neq K^x, K^!, K^*$ und $C \neq C^x, C^!, C^*$.
\end{ex}

\begin{nt}
    Die Berechnung des Alexander-Polynoms $\Delta$ ist relativ leicht und effizient (dank linearer Algebra).
    Die Berechnung von $V$ ist aufwändiger.
    Nur wenige spezielle Werte lassen sich anders und schneller berechnen.
\end{nt}

\begin{prop}
    \begin{enumerate}[1)]
        \item
            $V(L)|_{q\mapsto 1} = 2^{\# L - 1}$,
        \item
            $V(L)|_{q\mapsto i} = \det(L)$,
        \item
            $V(L)|_{q\mapsto e^{\frac{\pi i}{6}}} = \pm (i\sqrt{3})^{\Col_3(L)}$
    \end{enumerate}
    \begin{proof}
        \begin{enumerate}[1)]
            \item
                Induktion per Schienenrelation
            \item
                Induktion per Schienenrelation
            \item
                vermutlich nicht-trivial.
        \end{enumerate}
    \end{proof}
\end{prop}

\begin{nt}
    Für den Torusknoten $T(a,b)$ kennt man geschlossene Formeln für $\Delta$ und $V$.
    Im Allgemeinen bleibt es bei mühsamen Rechnungen für Einzelfälle.
\end{nt}


% §E2
\section{Verallgemeinerungen zum HOMFLYPT-Polynom und zum Kauffman-Polynom}

\begin{st}[HOMPFLYPT]
    Sei $R$ ein kommutativer Ring mit Eins, $a, b \in R^*$, $c, d \in R$ mit $a - b = cd$.
    Dann existiert genau eine Abbildung $I: \scr D^{\text{or}} \to R$ mit
    \begin{enumerate}[i)]
        \item
            $I(0) = 1$ und $I(D \sqcup 0) = I(D) \cdot d$,
        \item
            $aI(L_+) - bI(L_-) = cI(L_0)$,
        \item
            $I$ ist invariant unter Reidemeister-Zügen.
    \end{enumerate}
    \begin{proof}
        Die Eindeutigkeit wurde bereits bewiesen (durch Algorithmus).
        Die Existenz verlangt nach einer Konstruktion.

        Sei $\scr D_n$ die Menge der Verschlingungsdiagrammen mit $\le n$ Kreuzungen.
        Sei $\scr D_n^*$ die Menge der Verschlingungsdiagrammen mit $\le n$ Kreuzungen und nummerierten Komponenten mit jeweils einem Basispunkt auf der Komponente.

        Zu einem Diagramm $D \in \scr D_n^*$ ist das absteigende Diagramm $\alpha D$ eindeutig bestimmt (Komponenten jeweils von Basispunkt absteigend).

        Konstruktion von $I_n: \scr D_n^* \to R$ und Nachweis der Eigenschaften:
        \begin{seg}{$n = 0$}
            $I_0: \scr D_0^* \to R$, setze $I_0(D) = d^{\#D - 1}$ ($d$ Anzahl der Komponenten).
        \end{seg}
        \begin{seg}{$n \ge 1$}
            Sei $I_{n-1}$ konstruiert.
            Definiere $I_n$ wiei folgt.
            Sei $w(D)$ die Anzahl der zu wechselnden Kreuzungen von $D$ nach $\alpha D$.
            Für $w(D) = 0$ setze $I_n(D) = d^{\#D - 1}$ gemäß 1).
            Für $w(D) \ge 1$ wechsle die erste Kreuzung und berechne $I_n(D)$ gemäß 2).
        \end{seg}
        Wir zeigen
        \begin{enumerate}[($A_n$)]
            \item
                $I_n(D)$ ist unabhängig von der Reihenfolge der zu wechselnden Kreuzungen.
            \item
                $I_n(D)$ ist unabhängig von den Basispunkten.
            \item
                $I_n(D)$ ist invariant unter Reidemeister-Zügen in $\scr D_n^*$ (solche, die die Kreuzungszahl nicht ändern).
            \item
                $I_n(D)$ ist unabhängig von der Nummerierung der Komponenten.
        \end{enumerate}
        $A_0, B_0, C_0, D_0$ sind klar.
        Seien $A_{n-1}, B_{n-1}, C_{n-1}, D_{n-1}$ bewiesen.
        $A_n$ gilt dank Kommutativität:
        \begin{math}
            a I_n(L_+ L_+) - b I_n(L_- L_+) &= c I_{n-1}(L_0 L_+) \\
            a I_n(L_- L_+) - b I_n(L_- L_-) &= c I_{n-1}(L_- L_0) \\
            \leadsto
            a^2 I_n(L_+ L_+) - b^2 I_n(L_- L_-) &= ac I_{n-1}(L_0 L_+) + bc I_{n-1}(L_- L_0)
        \end{math}
        und
        \begin{math}
            a I_n(L_+ L_+) - b I_n(L_+ L_-) &= c I_{n-1}(L_+ L_0) \\
            a I_n(L_+ L_-) - b I_n(L_- L_-) &= c I_{n-1}(L_0 L_-) \\
            \leadsto
            a^2 I_n(L_+ L_+) - b^2 I_n(L_- L_-) &= ac I_{n-1}(L_+ L_0) + bc I_{n-1}(L_0 L_-)
        \end{math}
        Die Differenz ist
        \begin{math}
            c^2 I_n(L_0 L_0) - c^2 I_n(L_0 L_0) &= 0
        \end{math}
        Damit gilt $A_n$, d.h. $I_n$ erfüllt die Schienenrelation 2) für jede Kreuzung.
    \end{proof}
\end{st}

\begin{lem}
    Zu $D \in \scr D_n^*$ ist das absteigende Diagramm $\alpha D$ trivial.
    Genauer existieret eine Folge von R-Zügen
    \begin{math}
        \alpha D = D_0 \to D_1 \to \dotsb \to D_l = 0
    \end{math}
    mit der Eigenschaft, dass $\Cr(D_0) \ge \Cr(D_1) \ge \dotsb \ge \Cr(D_l) = 0$.
    Hierbei erlauben wir neben $R1, R2, R3$ auch $R2^2$ (zweifache Anwendung von R2, durchschieben von einem Kreis).
    \begin{proof}
        Übung, oder Lickorisk Kap 15.
    \end{proof}
    \begin{note}
        Ohne $R2^2$ ist dies nicht möglich (Skizze: 8 mit zwei zusätzlichen inneren Komponenten).
    \end{note}
\end{lem}









