\documentclass{mycourse}

\newcommand{\LH}{\operatorname{LH}}

\title{Höhere Analysis}

\begin{document}

\maketitle
\tableofcontents

\chapter{Fourierreihen}

\section{Orthonormalsysteme}

\begin{df} \label{1.1}
	Sei $L$ ein linearer Raum über $\C$.
	Eine Abbildung $\<,\>: L \times L \to \C$ heißt \emph{Skalarprodukt}, falls
	\begin{enumerate}[(S1)]
		\item
			$\<\alpha x_1 + \beta x_2, y \> = \alpha \<x_1,y\> + \beta \<x_2, y\>$
		\item
			$\<y,x\> = \_{\<x,y\>}$
		\item
			$\forall x\in L\setminus \{0\} : \<x,x\> > 0$
	\end{enumerate}
\end{df}

\begin{nt} \label{1.2}
	\begin{enumerate}[1)]
		\item
			Aus (S2) folgt $\<x,x\> \in \R$.
		\item
			Aus (S1) und (S2) folgt
			\[
				\<x, \alpha y_1 + \beta y_2\> = \_\alpha \<x,y_1\> + \_\beta \<x,y_2\>
			\]
		\item
			Aus (S1) folgt $\<0,0\> = 0$ und mit (S3):
			\[
				\<x,x\> = 0 \quad\iff\quad x=0
			\]
	\end{enumerate}
\end{nt}

\begin{st} \label{1.3}
	Durch
	\[
		\|x\| := \sqrt{\<x,x\>}
	\]
	wird auf $L$ eine Norm definiert, die \emph{induzierte Norm}.
	Es gilt die sogenante \emph{Cauchy-Schwarz-Bunjakowski-Ungleichung} (CSB):
	\[
		| \<x,y\> | \le \|x\| \cdot \|y\|
		\qquad \forall x,y \in L
	\]
	\begin{proof}
		Die Positivität folgt aus \ref{1.1} (S3) und \ref{1.2}.
		Die Homogenität $\|\alpha x\| = |\alpha| \|x\|$ ist eine leichte Übung.

		Zeige für die Dreiecksungleichung zunächst die CSB für reelles $c$:
		\begin{align*}
			0 &\le \Big\< x + c\<x,y\> y, x + c \<x,y\> y\Big\> \\
			&= \|x\|^2 + \underbrace{c\_{\<x,y\>}\<x,y\> + c \<x,y\>\<y,x\>}_{= 2|\<x,y\>|^2 c} + c^2\<x,y\>\_{\<x,y\>} \|y\|^2 \\
			&= \|x\|^2 + 2|\<x,y\>|^2 c + \|y\|^2 |\<x,y\>|^2 c^2 
			=: p(c)
		\end{align*}
		Wegen $p(c) \ge 0$ für alle $c\in \R$ gilt für die Diskriminante
		\begin{align*}
			0 \ge D 
			&= \tilde b^2 - 4\tilde a \tilde c \\
			&= 4 |\<x,y\>|^4 - 4 \|y\|^2 |\<x,y\>|^2 \|x\|^2
		\end{align*}
		Also
		\[
			|\<x,y\>|^4 \le |\<x,y\>|^2 \|x\|^2 \|y\|^2
		\]
		Daraus folgt die CSB.

		Zeige jetzt die Dreiecksungleichung:
		\begin{align*}
			\|x+y\|^2 = \<x+y, x+y\>
			&= \|x\|^2 + \underbrace{\<x,y\> + \<y,x\>}_{= 2\Re \<x,y\>} + \|y\|^2 \\
			&\le \|x\|^2 + 2 |\<x,y\>|  + \|y\|^2 \\
			&\le \|x\|^2 + 2\|x\|\|y\| + \|y\|^2 \\
			&\le (\|x\| + \|y\|)^2
		\end{align*}
	\end{proof}
\end{st}

\begin{kor}[Stetigkeit des Skalarprodukts]
	Das Skalarprodukt ist bezüglich der induzierten Norm in beiden Argumenten stetig, d.h.
	\[
		x_n \to x \quad\land\quad y_n \to y
		\qquad \implies \qquad
		\<x_n, y_n\> \to \<x,y\>
	\]
	oder
	\[
		\lim_{n\to \infty} \<x_n, y_n\> = \l\<\lim_{n\to \infty} x_n, \lim_{n\to\infty} y_n \r\>
	\]
	\begin{proof}
		\begin{align*}
			|\<x_n,y_n\> - \<x,y\>|
			&= |\<x_n, y_n - y\> - \<x-x_n,y\>| \\
			&\le |\<x_n,y_n -y\>| + |\<x-x_n,y\>| \\
			&\stack{CSB}{\le} \|x_n\| \underbrace{\|y_n -y\|}_{\to 0} + \underbrace{\|x-x_n\|}_{\to 0} \underbrace{\|y\|}_{=\const}
		\intertext{Da $(x_n)$ konvergent in $L$, ist $x_n$ beschränkt (Beweis dazu analog wie in $\C$) und daher}
			&\to 0 \qquad (n\to \infty)
		\end{align*}
	\end{proof}
\end{kor}

\begin{df}[Hilbertraum] \label{1.5}
	$(L, \<,\>)$ heißt \emph{Hilbertraum}, wenn $L$ bezüglich der induzierten Norm vollständig ist, d.h. jede Cauchy-Folge konvergiert.
\end{df}

\begin{ex} \label{1.6}
	\begin{enumerate}[1)]
		\item
			Sei 
			\[
				L = \bigg\{(x_n) \text{ Folge in } \C : \sum_{n=1}^\infty |x_n|^2 < \infty \bigg\}
			\]
			Setze 
			\[
				\<(x_n),(y_n)\> := \sum_{n=1}^\infty x_n\_{y_n}
			\]
			Die Reihe konvergiert, da sie eine Cauchy-Folge in $\C$ ist:
			\begin{align*}
				\bigg| \sum_{j=n}^m x_j\_{y_j} \bigg|
				&=\bigg| \Big\<(x_n,\dotsc,x_m), (y_n,\dotsc,y_m)\Big\>_{\C^{m-n+1}} \bigg|
			\intertext{mit der CSB in $\C^{m-n+1}$ gilt}
				&\le \bigg( \sum_{j=n}^m |x_j|^2 \bigg)^{\f 12} \bigg( \sum_{j=n}^m |y_j|^2 \bigg)^{\f 12}
			\intertext{
				Wegen $\sum_{n=1}^\infty |x_n|^2 < \infty$ (für alle $(x_n) \in L$) sind die beiden Summen $< \eps$ für $m \ge n > N_\eps$ (bzw. $\tilde N_\eps$). 
				Für $m \ge n > \max\{N_\eps, \tilde N_\eps\}$ also}
				&< \eps
			\end{align*}
			Gültigkeit von (S1) bis (S3) ist eine leichte Übung.

			$(L, \<,\>)$ ist ein Hilbertraum (Beweis: Übungsaufgabe 1.2a).
			Man nennt ihn $\ell^2$, je nach Kontext mit oder ohne dem Skalarprodukt aus diesem Beispiel: $\ell^2 := L$ oder $\ell^2 := (L,\<,\>)$.
		\item
			Sei $L = C([a,b] \to \C)$ und
			\[
				\<f,g\> := \int_a^b f(x)\_{g(x)} dx
			\]
			(S1) bis (S3) ist eine Übung.

			$(L,\<,\>)$ ist kein Hilbertraum (Gegenbeispiel Übung).

			Erweitere $L$ zu 
			\begin{align*}
				\tilde L &:= L^2(]a,b[) \\
				\<f,g\>^{\sim} &:= \int_{]a,b[} f\_{g} d\my
			\end{align*}
			$(\tilde L, \<,\>^\sim)$ ist ein Hilbertraum.
			
			Es gelten
			\begin{itemize}
				\item
					$L \subset \tilde L$
				\item
					$\<f,g\>^\sim = \<f,g\>$ für $f,g \in L$
				\item
					$L$ ist dicht in $\tilde L$.
			\end{itemize}
	\end{enumerate}
\end{ex}

\begin{df}[Prä-Hilbertraum] \label{1.7}
	Ein linearer Raum mit Skalarprodukt heißt \emph{Prä-Hilbertraum}.
\end{df}

\begin{st} \label{1.8}
	Zu jedem \emph{Prä-Hilbertraum} $(L,\<,\>_L)$ existiert ein bis auf Isomorphie eindeutiger Hilbertraum $(H,\<,\>_H)$ mit
	\begin{itemize}
		\item
			$L \subset H$
		\item
			$\<,\>_H = \<,\>_L$ auf $L\times L$
		\item
			$L$ ist dicht in $H$.
	\end{itemize}
	$(H,\<,\>_H)$ heißt \emph{Vervollständigung} von $L$.
	\begin{note}
		Die Isomorphie bedeutet in diesem Fall eine lineare, bijektive Funktion $\Phi$
		\[
			\Phi: \Big(H,\<,\>_H\Big) \to \Big(\tilde H,\<,\>_{\tilde H}\Big)
		\]
		mit
		\[
			\Big\<\Phi(f),\Phi(g)\Big\>_{\tilde H} = \<f,g\>_H
		\]
	\end{note}
	\begin{proof}
		Genauso wie die Erweiterung von $\Q$ zu $\R$.
	\end{proof}
\end{st}

\begin{df} \label{1.9}
	Eine Folge $(e_j)$ im Prä-Hilbertraum $(L,\<,\>)$ heißt \emph{Orthonormalsystem}, falls
	\[
		\<e_j,e_k\> = \delta_{j,k}
	\]
	Insbesondere ist damit $\|e_j\| = 1$.
\end{df}

\begin{df}[Vereinbarung] \label{1.10}
	In $(L,\<,\>)$ verwenden wir ab jetzt immer für $\|\cdot\|$ die induzierte Norm.
	Außerdem ist $(L,\<,\>)$ im Folgenden immer ein Prä-Hilbertraum.
\end{df}

\begin{st} \label{1.11}
	Sei $(e_j)$ ein Orthonormalsystem.
	Dann gelten
	\begin{enumerate}[1)]
		\item
			Falls 
			\[
				x = \sum_{j=1}^\infty x_j e_j
			\]
			mit $x_j \in \C$, so ist
			\[
				x_j = \<x,e_j\>
			\]
		\item
			$\{e_1,e_2, \dotsc \}$ ist linear unäbhängig, d.h. jede endliche Teilmenge ist linear unabhängig.
	\end{enumerate}
	\begin{proof}
		\begin{enumerate}[1)]
			\item
				\begin{align*}
					\<x,e_k\> 
					&= \bigg\< \sum_{j=1}^\infty x_j e_j, e_k \bigg\>
				\intertext{wegen der Stetigkeit des Skalarprodukts:}
					&= \sum_{j=1}^\infty \underbrace{\<x_j e_j, e_k\>}_{= 0 \text{ für $j\neq k$}} \\
					&= x_k \<e_k, e_k\> =1
				\end{align*}
			\item
				Sei $\sum_{j=1}^N \alpha_j e_j = 0$.
				Aus 1) folgt
				\[
					\alpha_j = x_j = \<x,e_j\> = \<0,e_j\> = 0
				\]
		\end{enumerate}
	\end{proof}
\end{st}

\begin{st} \label{1.12}
	Sei $(e_j)$ ein Orthonormalsystem in $(L,\<,\>)$ und $x\in L$.
	Dann gelten
	\begin{enumerate}[1)]
		\item
			die \emph{Besselsche Ungleichung}:
			\[ \label{eq:bessel}
				\sum_{j=1}^\infty |\<x,e_j\>|^2 \le \|x\|^2
			\]
		\item
			die \emph{Parsevalsche Gleichung}:
			\[
				\sum_{j=1}^\infty |\<x,e_j\>|^2 = \|x\|^2
			\]
			genau dann, wenn
			\[
				x = \sum_{j=1}^\infty \<x,e_j\> e_j
			\]
		\item
			Die Menge
			\[
				M := \bigg\{ x\in L : x = \sum_{j=1}^\infty \<x,e_j\> e_j \bigg\}
			\]
			ist abgeschlossen in $L$.
		\item
			Ist $(L, \<,\>)$ zusätzlich Hilbertraum, so ist $\sum_{j=1}^\infty \<x,e_j\> e_j$ konvergent (aber nicht unbedingt gegen $x$).
	\end{enumerate}
	\begin{proof}
		\begin{enumerate}[1)]
			\item
				\begin{align*}
					0 &\le \Big\|x - \sum_{j=1}^N \<x,e_j\> e_j \Big\|^2
					= \< \dotso, \dotso \> \\
					&= \|x\|^2 - \underbrace{\Big\<x, \sum_{j=1}^N \<x,e_j\> e_j\Big\>}_{= \sum_{j=1}^N |\<x,e_j\>|^2} - \underbrace{\Big\<\sum_{j=1}^N \<x,e_j\> e_j,x\Big\>}_{= \sum_{j=1}^N |\<x,e_j\>|^2} + \underbrace{\Big\<\sum_{j=1}^N\<x,e_j\> e_j, \sum_{k=1}^N \<x,e_k\> e_k\Big\> }_{= \sum_{j,k=1} \<x,e_j\>\_{\<x,e_k\>}\<e_j,e_k\> 
					= \sum_{j=1}^N |\<x,e_j\>|^2} 
				\end{align*}
				Also
				\[
					0 \le \|x\|^2 - \sum_{j=1}^N |\<x,e_j\>|^2
				\]
			\item[4)]
				Für $m\ge n > N_\eps$ gilt
				\begin{align*}
					\bigg\|\sum_{j=n}^m \<x,e_j\>e_j\bigg\|^2
					&\stack{2)}= \sum_{j=n}^m |\<y,e_j\>|^2 \\
					&= \sum_{j=n}^m |\<x,e_j\>|^2
					\stack{1)}< \eps.
				\end{align*}					
			\item[3)]
				Sei $x \in \_M$, zeige $x \in M$.
				Wähle $(x_n)$ als Folge in $M$ mit $x_n \to x$.
				Zu $\eps > 0$ wähle $N\in \N$ mit $\|x_N-x\| < \eps$.
				Wähle außerdem $M \in \N$ mit
				\[
					\Big\|x_N- \sum_{j=1}^m \<x_N,e_j\> e_j \Big\| < \eps
					\qquad \forall m > M
				\]
				Dann gilt für $m > M$
				\begin{align*}
					\Big\|x_N- \sum_{j=1}^m \<x_N,e_j\> e_j \Big\| 
					&\le \|x-x_N\| + \Big\|x_N - \sum_{j=1}^m \<x_N,e_j\> e_j \Big\| + \Big\| \sum_{j=1}^m \<x_N-x,e_j\> e_j \Big\|	\\
					&\le \eps + \eps + \bigg(\sum_{j=1}^m |\<x_N-x,e_j\>|^2 \bigg)^{\f 12} \\
					&\le 2\eps + \bigg(\sum_{j=1}^\infty |\<x_N-x,e_j\>|^2 \bigg)^{\f 12} \\
					&\stack{1)}\le 2 + \eps\|x_N-x\|
					< 3 \eps
				\end{align*}
		\end{enumerate}
	\end{proof}
\end{st}

\begin{ex} \label{1.13}
	Sei $L = C([-\pi,\pi] \to \C)$ mit Skalarprodukt
	\[
		\<f,g\> := \int_{-\pi}^{\pi}f(x)\_{g(x)} dx
	\]
	und Orthonormalsystem (Beweis Übung)
	\[
		e_j : x \mapsto \f 1{\sqrt{\pi}} \sin(jx)
	\]
	nach \ref{1.12} 3) ist die Menge
	\[
		\Big\{ f \in L : f = \sum_{j=1}^\infty \<f,e_j\> \f 1{\sqrt{\pi}}\sin(j \cdot) \Big\}
	\]
	abgeschlossen in $L$ (Konvergenz der Reihe bezüglich der induzierten Norm).

	Man definiert
	\[
		L^2(]-\pi,\pi[) := \Big\{ f: ]-\pi,\pi[ \to \C : f \text{ messbar } \land \int_{]-\pi,\pi[}|f|^2 d\my < \infty \Big\}.
	\]
	$(e_j)$ ist genauso eine Orthonormalsystem in $L^2(]-\pi,\pi[)$.
	Nach \ref{1.12} 4) ist für alle $f \in L^2(]-\pi,\pi[)$ die Reihe
	\[
		\sum_{j=1}^\infty \<f,e_j\> \f 1{\sqrt{\pi}} \sin(j \cdot)
	\]
	konvergent in $L^2(]-\pi,\pi[)$, aber nicht unbedingt punktweise gegen $f$.
\end{ex}

\begin{df} \label{1.14}
	Sei $(L, \<,\>)$ ein Prä-Hilbertraum und $(e_j)$ ein Orthonormalsystem in $L$.
	\begin{enumerate}[1)]
		\item
			Für $x\in L$ heißt
			\[
				\sum_{j=1}^\infty \<x,e_j\> e_j
			\]
			die \emph{Fourierreihe} von $x$.
			Die Koeffizienten $\<x,e_j\>$ heißen \emph{Fourierkoeffizienten}
		\item
			$(e_j)$ heißt \emph{vollständiges Orthonormalsystem} (VONS), falls
			\[
				\forall x \in L  : x = \sum_{j=1}^\infty \<x,e_j\> e_j
			\]
	\end{enumerate}
\end{df}

\begin{nt} \label{1.15}
	Für ein ONS $(e_j)$ und
	\[
		\LH(\{e_1,e_2,\dotsc\}) = \bigg\{ \sum_{j=1}^N \alpha_j e_j : N \in \N, \alpha_j \in \C \bigg\}
	\]
	dicht in $L$, so ist $(e_j)$ ein VONS.
	\begin{proof}
		Folgt direkt aus \ref{1.12} 4).
	\end{proof}
\end{nt}

\begin{st} \label{1.16}
	Alle Hilberträume mit (abzählbar unendlichen) VONS sind isomorph. 

	Ist $(H,\<,\>)$ Hilbertraum mit VONS $(e_j)$ so ist
	\begin{align*}
		\Phi: H &\to \ell^2 \\
		x &\mapsto (\<x,e_j\>)_{j\in \N}
	\end{align*}
	ein Hilbertraum-Isomorphismus
	\begin{proof}
		Übung
	\end{proof}
\end{st}



\section{Operatoren und Eigenwerte}



\begin{df} \label{1.17}
	\begin{enumerate}[1)]
		\item
			Sei $D(A)$ ein linearer Teilraum von $L$ und
			\[
				A : D(A) \to L.
			\]
			Dann heißt $A$ \emph{linearer Operator} in $L$.
		\item
			Ein linearer Operator $A$ heißt \emph{symmetrisch}, falls
			\[
				\forall x,y \in D(A) : \<Ax,y\> = \<x,Ay\>
			\]
		\item
			$\lambda \in \C$ heißt \emph{Eigenwert} (EW) von $A$, falls
			\[
				\exists x \in D(A) \setminus \{0\} : Ax = \lambda x
			\]
			$x$ heißt \emph{Eigenelement} oder \emph{Eigenvektor} von $A$ zum Eigenwert $\lambda$.

			Man nennt
			\[
				N(\lambda) := \dim (\ker(A-\lambda \Id))
			\]
			(geometrische) \emph{Vielfachheit} von $\lambda$.
	\end{enumerate}
\end{df}

\begin{st} \label{1.18}
	Ist $A: D(A) \to L$ symmetrisch, so gelten
	\begin{enumerate}[1)]
		\item
			Alle Eigenwerte sind reell.
		\item
			Eigenelemente zu verschiedenen Eigenwerten sind orthogonal.
	\end{enumerate}
	\begin{proof}
		Übung, oder siehe LAAG1.
	\end{proof}
\end{st}

\begin{ex} \label{1.19}
	Sei $(e_n)$ ein ONS im Hilbertraum $H$ und $(\lambda_j)$ ein Folge in $\C$.
	Definiere
	\begin{align*}
		D(A) &:= \Big\{x \in H : \sum_{j=1}^\infty |\lambda_j \<x,e_j\> |^2 < \infty \Big\} \\
		Ax &:= \sum_{j=1}^\infty \lambda_j \<x,e_j\> e_j \qquad \text{für } x \in D(A)
	\end{align*}
	\begin{enumerate}[1)]
		\item
			$D(A)$ ist linearer Teilraum von $H$ (leicht nachzurechnen).
			Die Reihe für $Ax$ konvergiert, da nach Parseval:
			\begin{align*}
				\Big\| \sum_{j=n}^m \lambda_j \<x,e_j\> e_j \Big\|^2
				= \sum_{j=n}^m |\lambda_j \<x,e_j\> |^2
				< \eps
				\qquad \text{für } m \ge n > N_\eps
			\end{align*}
			Also ist $A$ linearer Operator (Linearität leicht nachzurechnen).
		\item
			\begin{enumerate}[$\alpha$)]
				\item
					Alle $\lambda_j$ sind Eigenwerte, da für $x = e_k \in D(A)$
					\[
						Ax = \sum_{j=1}^\infty \lambda_j \<e_k,e_j\> e_j = \lambda_k x
					\]
				\item
					Falls $\exists x \in H\setminus \{0\} \forall j \in \N : \<x,e_j\> = 0$, dann ist $Ax = 0$.
					Also ist $\lambda = 0$ eventuell ein zusätzlicher Eigenwert.
				\item
					Es gibt keine weiteren Eigenwerte.
					\begin{proof}
						Sei $Ax = \lambda x$ mit $\lambda \neq \lambda_j$ für $j \in \N$.
						Dann gilt für alle $k \in \N$:
						\begin{align*}
							\lambda \<x,e_k\> 
							&= \<Ax, e_k\> \\ 
							&= \Big\< \sum_{j=1}^\infty \lambda_j \<x,e_j\> e_j, e_k \Big\> \\
							&= \lambda_k \<x,e_k\> \<e_k, e_k\>
							= \lambda_k \<x,e_k\> \\
							\implies \qquad \underbrace{(\lambda - \lambda_k)}_{\neq 0}\<x,e_k\> &= 0
						\end{align*}
						Also $x = 0$ oder $Ax = 0$.
						Damit sind alle Eigenwerte von $A$ gegeben durch
						\[
							\{ \lambda_j : j \in \N \} \quad \text{oder}\quad \{\lambda_j : j \in \N\} \cup \{0\}
						\]
					\end{proof}
			\end{enumerate}
		\item
			$A$ symmetrisch $\iff \forall j \in \N : \lambda_j \in \R$
			\begin{proof}
				\begin{seg}[$\implies$]
					Gilt nach \ref{1.18}.
				\end{seg}
				\begin{seg}[$\Longleftarrow$]
					Sei $x,y \in D(A)$, dann ist wegen der Stetigkeit des Skalarproduktes
					\begin{align*}
						\<Ax,y\>
						&= \sum_{j=1}^\infty \<\lambda_j \<x,e_j\> e_j, y\> \\
						&= \sum_{j=1}^\infty \lambda_j \<x,e_j\> \<e_j, y\> \\
						&= \sum_{j=1}^\infty \lambda_j \_{\<y,e_j\>} \<x, e_j\> \\
						&= \sum_{j=1}^\infty \<x,\lambda_j \<y,e_j\> e_j\> \\
						&= \<x, Ay\>.
					\end{align*}
				\end{seg}
			\end{proof}
		\item
			Falls $(\lambda_j)$ beschränkt, so ist $D(A) = H$
			\begin{proof}
				Siehe Besselsche Ungleichung (\ref{eq:bessel}).
			\end{proof}
		\item
			Falls $(e_j)$ VONS in $H$, so ist $D(A)$ dicht in $H$.
			\begin{proof}
				Sei $x \in H$, $x = \sum_{j=1}^\infty \<x,e_j\> e_j$.
				Setze $x_n := \sum_{j=1}^n \<x,e_j\> e_j$.
				Dann konvergiert $x_n \to x$ und $x_n \in D(A)$ (da $\<x_n,e_k\> = 0$ für $k > n$, also endliche Summe).
			\end{proof}
	\end{enumerate}
\end{ex}

\begin{df} \label{1.20}
	Die Darstellung 
	\[
		Ax := \sum_{j=1}^\infty \lambda_j \<x,e_j\> e_j
	\]
	des Operators $A$ heißt \emph{Spektraldarstellung} von $A$.
	Die Menge
	\[
		\sigma(A) := \_{\{\lambda \in \C : \lambda \text{ ist Eigenwert von $A$} \}}
	\]
	heißt \emph{Spektrum} von $A$.
	Ein Operator $A$, der durch diese Darstellung gegeben ist, heißt \emph{diskreter Spektraloperator}.
	\begin{note}
		Die Spektraldarstellung entspricht der Diagonalisierung bei Matrizen.
	\end{note}
\end{df}

\begin{ex} \label{1.21}
	Sei $H = L^2([0,\pi])$ und
	\begin{align*}
		D(A) &:= \Big\{ f \in C^2 ([0,\pi] \to \C) : f(0) = f(\pi) = 0 \Big\} \\
		Af &:= -f'' \qquad \text{für } f \in D(A)
	\end{align*}
	\begin{enumerate}[1)]
		\item
			$A$ ist symmetrisch
			\begin{proof}
				\begin{align*}
					\<Af, g\> 
					&= - \int_0^{\pi} f'' \_g dx
				\intertext{2 mal partiell integriert ergibt sich}
					&= \underbrace{\Big[ - f'\_g + f \_g' \Big]_{x=0}^\pi}_{=0} - \int_0^\pi f \_g'' dx \\
					&= \<f, Ag\>
				\end{align*}
			\end{proof}
			Insbesondere hat $A$ damit nur reelle Eigenwerte.
		\item
			$A$ ist positiv, d.h. $\forall f \in D(A) : \<Af, f\> \ge 0$.
			\begin{proof}
				\begin{align*}
					\<Af,f\> 
					&= - \int_0^\pi f'' \_f dx \\
					&= - \underbrace{\big[f' \_f \big]_{x=0}^\pi}_{=0} + \int_0^\pi \underbrace{f' \_f'}_{= |f'|^2 \ge 0} dx
					\ge 0
				\end{align*}
			\end{proof}
			Insbesondere sind damit alle Eigenwerte $\ge 0$.
		\item
			Alle Eigenwert sind von der Form $\lambda_j = j^2$ mit Eigenfunktionen $e_j(x) = \sqrt{\f 2 \pi} \sin (jx)$.
			\begin{proof}
				Für die Eigenwerte gilt
				\begin{align*}
					Af = \lambda f
					\iff -f'' = \lambda f
					\iff f(x) = c_1 \sin(\sqrt \lambda x) + c_2 \cos (\sqrt \lambda x) \qquad c_1,c_2 \in \C.
				\end{align*}
				Wegen $f(0)=0$ ist $c_2 = 0$ und wegen $f(\pi)$ ist $\lambda = j^2$.

				Da $A$ symmetrisch und $\|e_j\| = 1$ bilden die $(e_j)$ ein ONS.
			\end{proof}

	\end{enumerate}
\end{ex}

\end{document}
