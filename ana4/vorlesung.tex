\documentclass{mycourse}

\newcommand{\LH}{\operatorname{LH}}
\newcommand{\esssup}{\operatorname{ess\,sup}}

\title{Höhere Analysis}

\begin{document}

\maketitle
\tableofcontents

\chapter{Fourierreihen}

\section{Orthonormalsysteme}

\begin{df} \label{1.1}
	Sei $L$ ein linearer Raum über $\C$.
	Eine Abbildung $\<\cdot, \cdot\>: L \times L \to \C$ heißt \emph{Skalarprodukt}, falls
	\begin{enumerate}[(S1)]
		\item
			$\<\alpha x_1 + \beta x_2, y \> = \alpha \<x_1,y\> + \beta \<x_2, y\>$;
		\item
			$\<y,x\> = \_{\<x,y\>}$;
		\item
			$\forall x\in L\setminus \{0\} : \<x,x\> > 0$.
	\end{enumerate}
\end{df}

\begin{nt} \label{1.2}
	\begin{enumerate}[1)]
		\item
			Aus (S2) folgt $\<x,x\> \in \R$.
		\item
			Aus (S1) und (S2) folgt
			\[
				\<x, \alpha y_1 + \beta y_2\> = \_\alpha \<x,y_1\> + \_\beta \<x,y_2\>.
			\]
		\item
			Aus (S1) folgt $\<0,0\> = 0$ und mit (S3)
			\[
				\<x,x\> = 0 \quad\iff\quad x=0.
			\]
	\end{enumerate}
\end{nt}

\begin{st} \label{1.3}
	Durch
	\[
		\|x\| := \sqrt{\<x,x\>}
	\]
	wird auf $L$ eine Norm definiert, die \emph{induzierte Norm}.
	Es gilt die sogenante \emph{Cauchy-Schwarz-Bunjakowski-Ungleichung} (CSB)
	\[
		| \<x,y\> | \le \|x\| \cdot \|y\|,
		\qquad x,y \in L
	\]
	\begin{proof}
		Die Positivität folgt aus \ref{1.1} (S3) und \ref{1.2}.
		Die Homogenität $\|\alpha x\| = |\alpha| \|x\|$ ist eine leichte Übung.

		Zeige für die Dreiecksungleichung zunächst die CSB. Für reelles $c$ gilt:
		\begin{align*}
			0 &\le \Big\< x + c\<x,y\> y, x + c \<x,y\> y\Big\> \\
			&= \|x\|^2 + \underbrace{c\_{\<x,y\>}\<x,y\> + c \<x,y\>\<y,x\>}_{= 2|\<x,y\>|^2 c} + c^2\<x,y\>\_{\<x,y\>} \|y\|^2 \\
			&= \|x\|^2 + 2|\<x,y\>|^2 c + \|y\|^2 |\<x,y\>|^2 c^2 
			=: p(c).
		\end{align*}
		Wegen $p(c) \ge 0$ für alle $c\in \R$, gilt für die Diskriminante
		\begin{align*}
			0 \ge D 
			&= \tilde b^2 - 4\tilde a \tilde c \\
			&= 4 |\<x,y\>|^4 - 4 \|y\|^2 |\<x,y\>|^2 \|x\|^2,
		\end{align*}
		also
		\[
			|\<x,y\>|^4 \le |\<x,y\>|^2 \|x\|^2 \|y\|^2.
		\]
		Daraus folgt die CSB.

		Zeige jetzt die Dreiecksungleichung. Es gilt
		\begin{align*}
			\|x+y\|^2 = \<x+y, x+y\>
			&= \|x\|^2 + \underbrace{\<x,y\> + \<y,x\>}_{= 2\Re \<x,y\>} + \|y\|^2 \\
			&\le \|x\|^2 + 2 |\<x,y\>|  + \|y\|^2 \\
			&\le \|x\|^2 + 2\|x\|\|y\| + \|y\|^2 \\
			&\le (\|x\| + \|y\|)^2.
		\end{align*}
	\end{proof}
\end{st}

\begin{kor}[Stetigkeit des Skalarprodukts]
	Das Skalarprodukt ist bezüglich der induzierten Norm in beiden Argumenten stetig, d.h.
	\[
		x_n \to x \quad\land\quad y_n \to y
		\qquad \implies \qquad
		\<x_n, y_n\> \to \<x,y\>
	\]
	oder
	\[
		\lim_{n\to \infty} \<x_n, y_n\> = \l\<\lim_{n\to \infty} x_n, \lim_{n\to\infty} y_n \r\>.
	\]
	\begin{proof}
	Seien $ (x_n) $ und $ (y_n) $ konvergente Folgen, sodass $ x_n \to x, y_n \to x $, dann folgt
		\begin{align*}
			|\<x_n,y_n\> - \<x,y\>|
			&= |\<x_n, y_n - y\> - \<x-x_n,y\>| \\
			&\le |\<x_n,y_n -y\>| + |\<x-x_n,y\>| \\
			&\stack{CSB}{\le} \|x_n\| \underbrace{\|y_n -y\|}_{\to 0} + \underbrace{\|x-x_n\|}_{\to 0} \underbrace{\|y\|}_{=\const}
			\end{align*}
		Da $(x_n)$ konvergent in $L$, ist $x_n$ beschränkt (Beweis dazu analog wie in $\C$) und der obige Ausdruck konvergiert gegen $ 0 $ für $ n\to \infty $
			
		
	\end{proof}
\end{kor}

\begin{df}[Hilbertraum] \label{1.5}
	$(L, \<\cdot, \cdot\>)$ heißt \emph{Hilbertraum}, wenn $L$ bezüglich der induzierten Norm vollständig ist, d.h. jede Cauchy-Folge konvergiert.
\end{df}

\begin{ex} \label{1.6}
	\begin{enumerate}[1)]
		\item
			Sei 
			\[
				L := \bigg\{(x_n) \text{ Folge in } \C : \sum_{n=1}^\infty |x_n|^2 < \infty \bigg\}
			\]
			Setze 
			\[
				\<(x_n),(y_n)\> := \sum_{n=1}^\infty x_n\_{y_n}
			\]
			Die Reihe konvergiert, da sie eine Cauchy-Folge in $\C$ ist:
			\begin{align*}
				\bigg| \sum_{j=n}^m x_j\_{y_j} \bigg|
				&=\bigg| \Big\<(x_n,\dotsc,x_m), (y_n,\dotsc,y_m)\Big\>_{\C^{m-n+1}} \bigg|.
			\intertext{Mit der CSB in $\C^{m-n+1}$ gilt}
				&\le \bigg( \sum_{j=n}^m |x_j|^2 \bigg)^{\f 12} \bigg( \sum_{j=n}^m |y_j|^2 \bigg)^{\f 12}.
				\end{align*}
				Wegen $\sum_{n=1}^\infty |x_n|^2 < \infty$ (für alle $(x_n) \in L$) sind die beiden Summen $< \eps$ für $m \ge n > N_\eps$ (bzw. $\tilde N_\eps$). 
				Für $m \ge n > \max\{N_\eps, \tilde N_\eps\}$ ist der ganze Ausdruck $ < \eps $.
			
			Gültigkeit von (S1) bis (S3) ist eine leichte Übung.

			$(L, \<\cdot,\cdot\>)$ ist ein Hilbertraum (Beweis: Übungsaufgabe 1.2a).
			Man nennt ihn $\ell^2$, je nach Kontext mit oder ohne dem Skalarprodukt aus diesem Beispiel: $\ell^2 := L$ oder $\ell^2 := (L,\<\cdot,\cdot\>)$.
		\item
			Sei $L = C([a,b] \to \C)$ und
			\[
				\<f,g\> := \int_a^b f(x)\_{g(x)} dx.
			\]
			(S1) bis (S3) ist eine Übung.

			$(L,\<,\>)$ ist kein Hilbertraum (Gegenbeispiel Übung).

			Erweitere $L$ zu 
			\begin{align*}
				\tilde L &:= L^2(]a,b[) \\
				\<f,g\>^{\sim} &:= \int_{]a,b[} f\_{g} d\my
			\end{align*}
			$(\tilde L, \<,\>^\sim)$ ist ein Hilbertraum.
			
			Es gelten
			\begin{itemize}
				\item
					$L \subset \tilde L$
				\item
					$\<f,g\>^\sim = \<f,g\>$ für $f,g \in L$
				\item
					$L$ ist dicht in $\tilde L$.
			\end{itemize}
	\end{enumerate}
\end{ex}

\begin{df}[Prähilbertraum] \label{1.7}
	Ein linearer Raum mit Skalarprodukt heißt \emph{Prähilbertraum}.
\end{df}

\begin{st} \label{1.8}
	Zu jedem \emph{Prähilbertraum} $(L,\<\cdot,\cdot\>_L)$ existiert ein bis auf Isomorphie eindeutiger Hilbertraum $(H,\<\cdot,\cdot\>_H)$ mit
	\begin{itemize}
		\item
			$L \subset H$;
		\item
			$\<,\>_H = \<,\>_L$ auf $L\times L$;
		\item
			$L$ ist dicht in $H$.
	\end{itemize}
	$(H,\<\cdot,\cdot\>_H)$ heißt \emph{Vervollständigung} von $L$.
	\begin{note}
		Die Isomorphie bedeutet in diesem Fall die Existenz einer linearen, bijektiven Funktion $\Phi$
		\[
			\Phi: \Big(H,\<\cdot,\cdot\>_H\Big) \to \Big(\tilde H,\<\cdot, \cdot\>_{\tilde H}\Big)
		\]
		mit
		\[
			\Big\<\Phi(f),\Phi(g)\Big\>_{\tilde H} = \<f,g\>_H.
		\]
	\end{note}
	\begin{proof}
		Genauso wie die Erweiterung von $\Q$ zu $\R$.
	\end{proof}
\end{st}

\begin{df} \label{1.9}
	Eine Folge $(e_j)$ im Prähilbertraum $(L,\<,\>)$ heißt \emph{Orthonormalsystem}, falls
	\[
		\<e_j,e_k\> = \delta_{j,k}
	\]
	Insbesondere ist damit $\|e_j\| = 1$.
\end{df}

\begin{conv} \label{1.10}
	Im Folgenden bezeichne $ (L, \<\cdot, \cdot \>) $ immer einen Prähilbertraum und $ \|\cdot \| $ seine induzierte Norm.
\end{conv}

\begin{st} \label{1.11}
	Sei $(e_j)$ ein Orthonormalsystem.
	Dann gelten:
	\begin{enumerate}[1)]
		\item
			Falls 
			\[
				x = \sum_{j=1}^\infty x_j e_j
			\]
			mit $x_j \in \C$, so ist
			\[
				x_j = \<x,e_j\>.
			\]
		\item
			$\{e_1,e_2, \dotsc \}$ ist linear unäbhängig, d.h. jede endliche Teilmenge ist linear unabhängig.
	\end{enumerate}
	\begin{proof}
		\begin{enumerate}[1)]
			\item
			wegen der Stetigkeit des Skalarprodukts gilt:
				\begin{align*}
					\<x,e_k\> 
					&= \bigg\< \sum_{j=1}^\infty x_j e_j, e_k \bigg\>
					&= \sum_{j=1}^\infty \underbrace{\<x_j e_j, e_k\>}_{= 0 \text{ für $j\neq k$}} \\
					&= x_k \<e_k, e_k\> =1.
				\end{align*}
			\item
				Sei $\sum_{j=1}^N \alpha_j e_j = 0$.
				Aus 1) folgt
				\[
					\alpha_j = x_j = \<x,e_j\> = \<0,e_j\> = 0.
				\]
		\end{enumerate}
	\end{proof}
\end{st}

\begin{st} \label{1.12}
	Sei $(e_j)$ ein Orthonormalsystem in $(L,\<,\>)$ und $x\in L$.
	Dann gilt:
	\begin{enumerate}[1)]
		\item
			die \emph{Besselsche Ungleichung}:
			\[
				\sum_{j=1}^\infty |\<x,e_j\>|^2 \le \|x\|^2.
			\]
		\item
			die \emph{Parsevalsche Gleichung}:
			\[
				\sum_{j=1}^\infty |\<x,e_j\>|^2 = \|x\|^2
			\]
			genau dann, wenn
			\[
				x = \sum_{j=1}^\infty \<x,e_j\> e_j.
			\]
			\begin{note}
				Sie gilt also insbesondere dann, wenn $(e_j)$ ein \emph{vollständiges} ONS ist (siehe \ref{1.14} 2)).
			\end{note}
		\item
			Die Menge
			\[
				M := \bigg\{ x\in L : x = \sum_{j=1}^\infty \<x,e_j\> e_j \bigg\}
			\]
			ist abgeschlossen in $L$.
		\item
			Ist $(L, \<,\>)$ zusätzlich Hilbertraum, so ist $\sum_{j=1}^\infty \<x,e_j\> e_j$ konvergent (aber nicht unbedingt gegen $x$).
	\end{enumerate}
	\begin{proof}
		\begin{enumerate}[1)]
			\item Mit der Positivität von $ \|\cdot \| $ folgt
				\begin{align*}
					0 &\le \Big\|x - \sum_{j=1}^N \<x,e_j\> e_j \Big\|^2
					= \< \dotso, \dotso \> \\
					&= \|x\|^2 - \underbrace{\Big\<x, \sum_{j=1}^N \<x,e_j\> e_j\Big\>}_{= \sum_{j=1}^N |\<x,e_j\>|^2} - \underbrace{\Big\<\sum_{j=1}^N \<x,e_j\> e_j,x\Big\>}_{= \sum_{j=1}^N |\<x,e_j\>|^2} + \underbrace{\Big\<\sum_{j=1}^N\<x,e_j\> e_j, \sum_{k=1}^N \<x,e_k\> e_k\Big\> }_{= \sum_{j,k=1} \<x,e_j\>\_{\<x,e_k\>}\<e_j,e_k\> 
					= \sum_{j=1}^N |\<x,e_j\>|^2}, 
				\end{align*}
				also
				\[
					0 \le \|x\|^2 - \sum_{j=1}^N |\<x,e_j\>|^2.
				\]
			\item[4)]
				Für $m\ge n > N_\eps$ gilt
				\begin{align*}
					\bigg\|\sum_{j=n}^m \<x,e_j\>e_j\bigg\|^2
					&\stack{2)}= \sum_{j=n}^m |\<y,e_j\>|^2 \\
					&= \sum_{j=n}^m |\<x,e_j\>|^2
					\stack{1)}< \eps.
				\end{align*}					
			\item[3)]
				Sei $x \in \_M$, zeige $x \in M$.
				Wähle $(x_n)$ als Folge in $M$ mit $x_n \to x$.
				Zu $\eps > 0$ wähle $N\in \N$ mit $\|x_N-x\| < \eps$.
				Wähle außerdem $M \in \N$ mit
				\[
					\Big\|x_N- \sum_{j=1}^m \<x_N,e_j\> e_j \Big\| < \eps
					\qquad \forall m > M.
				\]
				Dann gilt für $m > M$
				\begin{align*}
					\Big\|x_N- \sum_{j=1}^m \<x_N,e_j\> e_j \Big\| 
					&\le \|x-x_N\| + \Big\|x_N - \sum_{j=1}^m \<x_N,e_j\> e_j \Big\| + \Big\| \sum_{j=1}^m \<x_N-x,e_j\> e_j \Big\|	\\
					&\le \eps + \eps + \bigg(\sum_{j=1}^m |\<x_N-x,e_j\>|^2 \bigg)^{\f 12} \\
					&\le 2\eps + \bigg(\sum_{j=1}^\infty |\<x_N-x,e_j\>|^2 \bigg)^{\f 12} \\
					&\stack{1)}\le 2 + \eps\|x_N-x\|
					< 3 \eps.
				\end{align*}
		\end{enumerate}
	\end{proof}
\end{st}

\begin{ex} \label{1.13}
	Sei $L = C([-\pi,\pi] \to \C)$ mit Skalarprodukt
	\[
		\<f,g\> := \int_{-\pi}^{\pi}f(x)\_{g(x)} dx
	\]
	und Orthonormalsystem (Beweis Übung)
	\[
		e_j : x \mapsto \f 1{\sqrt{\pi}} \sin(jx).
	\]
	Nach \ref{1.12} 3) ist die Menge
	\[
		\Big\{ f \in L : f = \sum_{j=1}^\infty \<f,e_j\> \f 1{\sqrt{\pi}}\sin(j \cdot) \Big\}
	\]
	abgeschlossen in $L$ (Konvergenz der Reihe bezüglich der induzierten Norm).

	Man definiert
	\[
		L^2(]-\pi,\pi[) := \bigg\{ f: ]-\pi,\pi[ \to \C : f \text{ messbar } \land \int_{]-\pi,\pi[}|f|^2 d\my < \infty \bigg\}.
	\]
	$(e_j)$ ist genauso ein Orthonormalsystem in $L^2(]-\pi,\pi[)$.
	Nach \ref{1.12} 4) ist für alle $f \in L^2(]-\pi,\pi[)$ die Reihe
	\[
		\sum_{j=1}^\infty \<f,e_j\> \f 1{\sqrt{\pi}} \sin(j \cdot)
	\]
	konvergent in $L^2(]-\pi,\pi[)$, aber nicht unbedingt punktweise gegen $f$.
\end{ex}

\begin{df} \label{1.14}
	Sei $(L, \<,\>)$ ein Prähilbertraum und $(e_j)$ ein Orthonormalsystem in $L$.
	\begin{enumerate}[1)]
		\item
			Für $x\in L$ heißt
			\[
				\sum_{j=1}^\infty \<x,e_j\> e_j
			\]
			die \emph{Fourierreihe} von $x$.
			Die Koeffizienten $\<x,e_j\>$ heißen \emph{Fourierkoeffizienten}
		\item
			$(e_j)$ heißt \emph{vollständiges Orthonormalsystem} (VONS), falls für alle $ x\in L $
			\[
				x = \sum_{j=1}^\infty \<x,e_j\> e_j
			\]
			gilt.
	\end{enumerate}
\end{df}

\begin{nt} \label{1.15}
	Sei $(e_j)$ ein ONS und
	\[
		\LH(\{e_1,e_2,\dotsc\}) = \bigg\{ \sum_{j=1}^N \alpha_j e_j : N \in \N, \alpha_j \in \C \bigg\}
	\]
	dicht in $L$.

	Dann ist $(e_j)$ ein VONS.
	\begin{proof}
		Folgt direkt aus \ref{1.12} 3). Sei $ M $ definiert wie in \ref{1.12} 3), dann folgt zunächst $ LH(\{e_1,e_2,...\})\subset M \subset L $ und somit auch $ L=\_{LH(\{e_1,e_2,...\})}\subset \_ M=M\subset L $, also $ M=L $.
	\end{proof}
\end{nt}

\begin{st} \label{1.16}
	Alle Hilberträume mit (abzählbar unendlichen) VONS sind isomorph. 

	Ist $(H,\<,\>)$ Hilbertraum mit VONS $(e_j)$ so ist
	\begin{align*}
		\Phi: H &\to \ell^2 \\
		x &\mapsto (\<x,e_j\>)_{j\in \N}
	\end{align*}
	ein Hilbertraum-Isomorphismus
	\begin{proof}
		Übungsaufgabe \href{http://www.iadm.uni-stuttgart.de/LstAnaMPhy/Lesky/Vorlesungen/13-Hoehere-Analysis/blatt01.pdf}{1.2b}
	\end{proof}
\end{st}



\section{Operatoren und Eigenwerte}



\begin{df} \label{1.17}
	\begin{enumerate}[1)]
		\item
			Sei $D(A)$ ein linearer Teilraum von $L$ und
			\[
				A : D(A) \to L.
			\]
			Dann heißt $A$ \emph{linearer Operator} in $L$.
		\item
			Ein linearer Operator $A$ heißt \emph{symmetrisch}, falls
			\[
				\forall x,y \in D(A) : \<Ax,y\> = \<x,Ay\>.
			\]
		\item
			$\lambda \in \C$ heißt \emph{Eigenwert} (EW) von $A$, falls
			\[
				\exists x \in D(A) \setminus \{0\} : Ax = \lambda x.
			\]
			$x$ heißt \emph{Eigenelement} oder \emph{Eigenvektor} von $A$ zum Eigenwert $\lambda$.

			Man nennt
			\[
				N(\lambda) := \dim (\ker(A-\lambda \Id))
			\]
			(geometrische) \emph{Vielfachheit} von $\lambda$.
	\end{enumerate}
\end{df}

\begin{st} \label{1.18}
	Ist $A: D(A) \to L$ symmetrisch, so gelten
	\begin{enumerate}[1)]
		\item
			Alle Eigenwerte sind reell.
		\item
			Eigenelemente zu verschiedenen Eigenwerten sind orthogonal.
	\end{enumerate}
	\begin{proof}
		Übung, oder siehe LAAG1.
	\end{proof}
\end{st}

\begin{ex} \label{1.19}
	Sei $(e_n)$ ein ONS im Hilbertraum $H$ und $(\lambda_j)$ ein Folge in $\C$.
	Definiere
	\begin{align*}
		D(A) &:= \Big\{x \in H : \sum_{j=1}^\infty |\lambda_j \<x,e_j\> |^2 < \infty \Big\}, \\
		Ax &:= \sum_{j=1}^\infty \lambda_j \<x,e_j\> e_j \qquad \text{für } x \in D(A).
	\end{align*}
	\begin{enumerate}[1)]
		\item
			$D(A)$ ist linearer Teilraum von $H$ (leicht nachzurechnen).
			Die Reihe für $Ax$ konvergiert, da nach Parseval:
			\begin{align*}
				\Big\| \sum_{j=n}^m \lambda_j \<x,e_j\> e_j \Big\|^2
				= \sum_{j=n}^m |\lambda_j \<x,e_j\> |^2
				< \eps
				\qquad \text{für } m \ge n > N_\eps.
			\end{align*}
			Also ist $A$ linearer Operator (Linearität leicht nachzurechnen).
		\item
			\begin{enumerate}[a)]
				\item
					Alle $\lambda_j$ sind Eigenwerte, da für $x = e_k \in D(A)$
					\[
						Ax = \sum_{j=1}^\infty \lambda_j \<e_k,e_j\> e_j = \lambda_k x
					\]
				\item
					Falls $\exists x \in H\setminus \{0\} \forall j \in \N : \<x,e_j\> = 0$, dann ist $Ax = 0$.
					Also ist $\lambda = 0$ eventuell ein zusätzlicher Eigenwert.
				\item
					Es gibt keine weiteren Eigenwerte.
					\begin{proof}
						Sei $Ax = \lambda x$ mit $\lambda \neq \lambda_j$ für $j \in \N$.
						Dann gilt für alle $k \in \N$:
						\begin{align*}
							\lambda \<x,e_k\> 
							&= \<Ax, e_k\> \\ 
							&= \Big\< \sum_{j=1}^\infty \lambda_j \<x,e_j\> e_j, e_k \Big\> \\
							&= \lambda_k \<x,e_k\> \<e_k, e_k\>
							= \lambda_k \<x,e_k\> \\
							\implies \qquad \underbrace{(\lambda - \lambda_k)}_{\neq 0}\<x,e_k\> &= 0
						\end{align*}
						Aus $\<x,e_k\> = 0$ für alle $k\in \N$ folgt $x = 0$ oder zumindest $Ax = \sum_{j=1}^\infty \lambda_j \<x,e_j\> e_j = 0$.
						Damit sind alle Eigenwerte von $A$ gegeben durch
						\[
							\{ \lambda_j : j \in \N \} \quad \text{oder}\quad \{\lambda_j : j \in \N\} \cup \{0\}
						\]
					\end{proof}
			\end{enumerate}
		\item
			$A$ symmetrisch $\iff \forall j \in \N : \lambda_j \in \R$
			\begin{proof}
				\begin{seg}[$\implies$]
					Gilt nach \ref{1.18}.
				\end{seg}
				\begin{seg}[$\Longleftarrow$]
					Sei $x,y \in D(A)$, dann ist wegen der Stetigkeit des Skalarproduktes
					\begin{align*}
						\<Ax,y\>
						&= \sum_{j=1}^\infty \<\lambda_j \<x,e_j\> e_j, y\> 
						= \sum_{j=1}^\infty \lambda_j \<x,e_j\> \<e_j, y\> \\
						&= \sum_{j=1}^\infty \lambda_j \_{\<y,e_j\>} \<x, e_j\> 
						= \sum_{j=1}^\infty \<x,\lambda_j \<y,e_j\> e_j\> \\
						&= \<x, Ay\>.
					\end{align*}
				\end{seg}
			\end{proof}
		\item
			Falls $(\lambda_j)$ beschränkt, so ist $D(A) = H$
			\begin{proof}
				Siehe Besselsche Ungleichung (\ref{1.12} 1)).
			\end{proof}
		\item
			Falls $(e_j)$ VONS in $H$, so ist $D(A)$ dicht in $H$.
			\begin{proof}
				Sei $x \in H$, $x = \sum_{j=1}^\infty \<x,e_j\> e_j$.
				Setze $x_n := \sum_{j=1}^n \<x,e_j\> e_j$.
				Dann konvergiert $x_n \to x$ und $x_n \in D(A)$ (da $\<x_n,e_k\> = 0$ für $k > n$, also endliche Summe).
			\end{proof}
	\end{enumerate}
\end{ex}

\begin{df} \label{1.20}
	Die Darstellung 
	\[
		Ax := \sum_{j=1}^\infty \lambda_j \<x,e_j\> e_j
	\]
	des Operators $A$ heißt \emph{Spektraldarstellung} von $A$.
	Die Menge
	\[
		\sigma(A) := \_{\{\lambda \in \C : \lambda \text{ ist Eigenwert von $A$} \}}
	\]
	heißt \emph{Spektrum} von $A$.
	Ein Operator $A$, der durch diese Darstellung gegeben ist, heißt \emph{diskreter Spektraloperator}.
	\begin{note}
		Die Spektraldarstellung entspricht der Diagonalisierung bei Matrizen.
	\end{note}
\end{df}

\begin{df*}
	Man nennt $A$ \emph{positiv}, falls
	\[
		\forall x \in D(A) : \<Ax,x\> \ge 0
	\]
	\begin{note} \label{nt:evpos}
		In diesem Fall sind alle Eigenwerte von $A$ positiv.
		\begin{proof}
			Sei $x$ Eigenvektor zum Eigenwert $\lambda$, dann ist
			\[
				0 \le \<Ax,x\> 
				= \lambda \underbrace{\<x,x\>}_{>0}
			\]
			also $\lambda \ge 0$.
		\end{proof}
	\end{note}
\end{df*}

\begin{ex} \label{1.21}
	Sei $H = L^2([0,\pi])$ und
	\begin{align*}
		D(A) &:= \Big\{ f \in C^2 ([0,\pi] \to \C) : f(0) = f(\pi) = 0 \Big\} \\
		Af &:= -f'' \qquad \text{für } f \in D(A)
	\end{align*}
	\begin{enumerate}[1)]
		\item
			$A$ ist symmetrisch.
			\begin{proof}
				\begin{align*}
					\<Af, g\> 
					= \int_0^{\pi} Af g dx
					&= - \int_0^{\pi} f'' \_g dx
				\intertext{2 mal partiell integriert ergibt sich}
					&= \underbrace{\Big[ - f'\_g + f \_g' \Big]_{x=0}^\pi}_{=0} - \int_0^\pi f \_g'' dx \\
					&= \<f, Ag\>
				\end{align*}
			\end{proof}
			Insbesondere hat $A$ damit nach \ref{1.18} 1) nur reelle Eigenwerte.
		\item
			$A$ ist positiv.
			\begin{proof}
				Sei $f \in D(A)$.
				Dann gilt
				\begin{align*}
					\<Af,f\> 
					&= - \int_0^\pi f'' \_f dx \\
					&= - \underbrace{\big[f' \_f \big]_{x=0}^\pi}_{=0} + \int_0^\pi \underbrace{f' \_f'}_{= |f'|^2 \ge 0} dx
					\ge 0
				\end{align*}
			\end{proof}
			Insbesondere sind damit alle Eigenwerte $\ge 0$.
		\item
			Alle Eigenwerte sind von der Form $\lambda_j = j^2$ mit Eigenfunktionen $e_j(x) = \sqrt{\f 2 \pi} \sin (jx)$.
			Die $(e_j)$ bilden ein ONS.
			\begin{proof}
				Sei $f \in D(A) \setminus \{0\}$.
				Für die Eigenwerte gilt:
				\begin{align*}
					Af = \lambda f
					&\iff\quad -f'' = \lambda f \\
					&\iff\quad f(x) = c_1 \sin(\sqrt \lambda x) + c_2 \cos (\sqrt \lambda x) \qquad c_1,c_2 \in \C.
				\end{align*}
				Wegen $f(0)=0$ ist $c_2 = 0$ und wegen $f(\pi) = 0$ muss $\lambda = j^2$ ($\lambda \neq 0$, sonst $f = 0$).
				Also folgt für $f$:
				\[
					f(x) = c_1 \sin(j x)
				\]

				Da $A$ symmetrisch und 
				\[
					\|e_j\|^2 = \int_0^\pi \f 2\pi \sin^2(x) dx = \f 1\pi \int_0^\pi 1 - \cos(2x) dx = 1,
				\]
				bilden die $(e_j)$ ein ONS.
			\end{proof}
		\item
			Sei $e_j : x \mapsto \sqrt{\f 2\pi} \sin(jx)$ (also $(e_j)$ ein ONS in $L^2([0,\pi])$).
			Setze
			\begin{align*}
				D(\tilde A) &:= \Big\{ f\in L^2([0,\pi]) : \sum_{j=1}^\infty j^4 |\<f,e_j\>|^2 < \infty \Big\}, \\
				\tilde A f &:= \sum_{i=1}^\infty j^2 \Big\<f, \sqrt{\tf 2\pi}\sin(j\cdot) \Big\> \sqrt{\tf 2\pi}\sin(j\cdot).
			\end{align*}
			Dann gelten
			\begin{itemize}
				\item
					$D(A) \subset D(\tilde A)$,
				\item
					$\tilde A \Big|_{D(A)} = A$, bzw. $\tilde Af = Af$ für $f \in D(A)$.
			\end{itemize}
			Man sagt: $\tilde A$ ist die \emph{Erweiterung} des Operators $A$ und schreibt $A \subset \tilde A$.
			\begin{proof}
				\begin{enumerate}[a)]
					\item
						Zeige $D(A) \subset D(\tilde A)$.
						Sei $f \in D(A)$, dann ist
						\[
							f \in C^2([0,\pi]\to \C) \subset L^2([0,\pi])
						\]
						und
						\begin{align*}
							j^2\<f,e_j\> = \<f,j^2e_j\> &= \<f,Ae_j\>
							\intertext{Wegen $e_j, f \in D(A)$ und $A$ symmetrisch:}
							&= \<Af, e_j\>
						\end{align*}
						Also
						\[
							\sum_{j=1}^\infty j^4|\<f,e_j\>|^2 
							= \sum_{j=1}^\infty |\<Af,e_j\>|^2 
							\stack{\text{Bessel}}\le \|Af\|^2 
							< \infty
						\]
					\item
						Zeige $\tilde A \Big|_{D(A)} = A$, bzw. $\tilde Af = Af$ für $f \in D(A)$.

						Sei $f \in D(A)$
						\begin{align*}
							\tilde Af 
							&= \sum_{j=1}^\infty j^2 \<f,e_j\> e_j \\
							&= \sum_{j=1}^\infty \<f,\underbrace{j^2e_j}_{=Ae_j}\> e_j \\
							&= \sum_{j=1}^\infty \<Af, e_j\> e_j
							\intertext{da $(e_j)$ ein VONS in $\im A$ ist (wird später gezeigt)}
							&= Af
						\end{align*}
				\end{enumerate}
			\end{proof}
	\end{enumerate}
\end{ex}

\begin{nt} \label{1.22}
	Seien $A$ und $D(A)$ gegeben wie in \ref{1.19}.
	Dann ist
	\[
		A \text{ positiv }  \quad\iff\quad \forall j\in \N : \lambda_j \ge 0
	\]
	\begin{proof}
		Es gilt
		\begin{align*}
			\<Ax, x \> 
			&= \sum_{j=1}^\infty \underbrace{\<\lambda_j \<x,e_j\> e_j, x\>}_{= \lambda_j\<x,e_j\> \<e_j,x\>} \\
			&= \sum_{j=1}^\infty \lambda_j |\<x,e_j\>|^2
		\end{align*}
		Die Rückrichtung erkennt man jetzt leicht.

		Die Hinrichtung wurde in \ref{nt:evpos} gezeigt.
	\end{proof}
\end{nt}

\begin{df} \label{1.23}
	Ein linearer Operator $A$ in $L$ mit $D(A) = L$ heißt \emph{beschränkt}, falls
	\[
		\exists c > 0 \forall x \in L : \|Ax\| \subset c\|x\|.
	\]
	Dann heißt
	\begin{align*}
		\|A\| &:= \inf \Big\{ c > 0 : \forall x \in L : \|Ax\| \le c\|x\| \Big\} \\
		&= \sup_{\|x\|=1} \|Ax\| = \sup_{\|x\|\le 1} \|Ax\| = \sup_{\|x\|\neq 0} \f{\|Ax\|}{\|x\|}
	\end{align*}
	die \emph{(Operator)-Norm} von $A$.

	Insbesondere gilt
	\[
		\forall x\in L : \|Ax\| \le \|A\| \|x\|
	\]
	\begin{note}
		Der Beweis der Gleichheit obiger Sumprema ist eine leichte Übung.
	\end{note}
\end{df}

\begin{nt} \label{1.24}
	Seien $A$ und $D(A)$ gegeben wie in \ref{1.19}.
	Dann gilt
	\[
		(\lambda_j) \text{ beschränkt} \quad\iff\quad D(A)=H \;\land\; A \text{ beschränkt}
	\]
	\begin{proof}
		Zeige durch vollständige Fallunterscheidung:
		\begin{seg}[$(\lambda_j)$ beschränkt: $|\lambda_j| \le c$]
			Sei $x \in L$.
			Nach Bessel ist $\sum_{j=1}^\infty c^2 |\<x,e_j\>|^2 \le c^2 \|x\|^2$ konvergent, also ist
			\[
				\sum_{j=1}^\infty \underbrace{|\lambda_j|^2 |\<x,e_j\>|^2}_{\le c^2|\<x,e_j\>|^2}
			\]
			nach dem Vergleichskriterium konvergent, also $D(A) = H$.

			\begin{align*}
				\|Ax\|^2 
				&= \bigg\|\sum_{j=1}^\infty \lambda_j \<x,e_j\> e_j\bigg\|^2 \\
				&\stack{\text{Parseval}} = \sum_{j=1}^\infty |\lambda_j|^2 |\<x,e_j\>|^2 \\
				&\stack{\text{s.o.}}\le c^2 \|x\|^2
			\end{align*}
			Also $\|Ax\| \le c\|x\|$ und damit $A$ beschränkt.
		\end{seg}
		\begin{seg}[$(\lambda_j)$ nicht beschränkt]
			Es existiert eine Teilfolge mit $|\lambda_{j_k}| \to \infty$ ($k\to \infty$) und damit
			\[
				\|A e_{j_k}\| = \|\lambda_{j_k} e_{j_k}\| = |\lambda_{j_k}| \to \infty
			\]
			und damit $\sup_{\|x\|=1} \|Ax\| = \infty$, also $A$ nicht beschränkt.
		\end{seg}
	\end{proof}
\end{nt}

\begin{nt*}
	$M$ heißt \emph{folgenkompakt}, falls jede Folge in $M$ eine konvergente Teilfolge besitzt.

	$M$ heißt \emph{überdeckungskompakt}, falls für jede offene Überdeckung $(O_\alpha)_{\alpha \in A}$ von $M$ eine endliche Teilmenge $(O_{\alpha_i})_{i=1,\dotsc,N}$ existiert, die $M$ überdeckt.
\end{nt*}

\begin{df} \label{1.25}
	Ein linearer Operator $A$ in $L$ mit $D(A) = L$ heißt \emph{kompakt}, falls
	\[
		(x_j) \text{ beschränkt} \quad\implies\quad \text{$(Ax_j)$ besitzt eine in $L$ konvergente Teilfolge}
	\]
	oder äquivalent
	\[
		M \subset L \text{ beschränkt} \quad\implies\quad \text{$A(M)$ ist präkompakt, d.h. $\_{A(M)}$ ist kompakt in $L$}
	\]
\end{df}

\begin{st} \label{1.26}
	Jeder kompakte Operator $A$ ist beschränkt.
	\begin{proof}
		Zeige durch Kontraposition.
		Sei $A$ nicht beschränkt, dann existiert $(x_j) \subset L$ mit 
		\[
			\|x_j\| = 1  \quad\land\quad \|Ax_j\| \to \infty
		\]
		Also für jede Teilfolge $\|Ax_{j_k}\| \to \infty$.
		Damit ist $A$ nicht kompakt.
	\end{proof}
\end{st}

\begin{st} \label{1.27}
	Sei $A$ ein linearer Operator in $L$ mit $D(A) = L$.
	Dann gilt
	\begin{align*}
		A \text{ beschränkt} 
		&\quad\iff\quad A \text{ stetig in } x = 0 \\
		&\quad\iff\quad A \text{ ist stetig}
	\end{align*}
	\begin{proof}
		\begin{seg}[$A$ beschränkt $\implies$ $A$ stetig in $x=0$]
			Sei $x_j \to 0$, dann ist
			\[
				\|Ax_j - A 0 \| = \|Ax_j\| \le \|A\| \|x_j\| \to 0
			\]
			also $Ax_j \to 0$ und somit $A$ stetig in $x=0$.
		\end{seg}
		\begin{seg}[$A$ stetig in $x_0=0$ $\implies$ $A$ stetig]
			Sei $x_j \to x$, also $x_j - x \to 0$., dann ist
			\[
				Ax_j - Ax = A(x_j - x) \to 0,
			\]
			also $Ax_j \to Ax$ und somit $A$ stetig.
		\end{seg}
		\begin{seg}[$A$ stetig $\implies$ $A$ beschränkt]
			Zeige durch Kontraposition.
			Sei $A$ nicht beschränkt und $(x_j) \subset L$ mit $\|x_j\| = 1$, $\|Ax_j\| \to \infty$.
			Definiere $y_j := \f 1{\|Ax_j\|}x_j$.
			Dann ist
			\[
				\|y_j\| = \f {\|x_j\|}{\|Ax_j\|} \to 0
			\]
			und
			\[
				\|Ay_j\| = \f 1{\|Ax_j\|} \|Ax_j\| = 1
			\]
			also konvergiert $Ay_j$ nicht gegen $A0 = 0$ und somit $A$ nicht stetig.
		\end{seg}
	\end{proof}
\end{st}

\begin{ex} \label{1.28}
	Sei $K \subset \R^n$ kompakt, $G \in C(K\times K \to \C)$, $L := C(K \to \C)$ und
	\[
		Af(x) := \int_{K} G(x,y) f(y) dy
		\qquad \text{für $f\in L$}
	\]
	($A$ nennt man auch \emph{Integraloperator})

	Dann ist $A : L \to L$ kompakt.
	\begin{proof}[Beweisskizze]
		Sei $(f_j)$ beschränkt in $L$
		\begin{enumerate}[1)]
			\item
				Zeige: $(A f_j(x))$ gleichmäßig beschränkt auf $K$.

				$K\times K$ ist kompakt, also $|G(x,y)| \le c$ auf $K\times C$.
				Dann ist
				\begin{align*}
					|Af_j(x)| 
					&\le \underbrace{\int_{K} |G(x,y) |f(y)| dy}_{\<c,|f_j|\>} \\
					&\stack{CSB}\le \|c\| \|f_j\| \\
					&\le \tilde c
				\end{align*}
			\item
				Zeige: Die Folge $(Af_j)$ ist gleichgradig stetig auf $K$, d.h.
				\[
					\forall \eps > 0 \exists \delta > 0 \underbrace{\forall x,\tilde x \in C \forall j \in \N}_{\mathclap{\text{gleichmäßig stetig bezüglich $x$ und $j$}}} : |x-\tilde x| < \delta \;\implies\; |f_j(x)-f_j(\tilde x)| < \eps
				\]
				Wegen $K\times K$ kompakt, ist $|G(x,y) - G(\tilde x,y)| \le \eps$ für $|x-\tilde x| < \delta$ ($G$ gleichmäßig stetig).
				Also
				\begin{align*}
					|Af_j(x) - Af_j(\tilde x)|
					&= \underbrace{\bigg| \int_K \Big(\underbrace{G(x,y) - G(\tilde x, y)}_{|\cdot|< \eps}\Big) f_j(y) dy \bigg|}_{= \int_K \eps |f_j(y)| dy = \le \<\eps, |f_j|\>} \\
					&\stack{CSB}\le \|\eps\| \|f_j\| \\
					&= \sqrt{N(K)} \eps \|f_j\|  \qquad\qquad \|\eps\| = \sqrt{\int_{K}|\eps|^2 dx} \\
					&\le c\eps
				\end{align*}
				\fixme[$N(K)$ ??]
			\item
				Nach dem Satz von Arzelá-Ascoli enthält $(Af_j)$ eine gleichmäßig auf $K$ konvergente Teilfolge.
			\item
				Weil $(Af_j)$ gleichmäßig konvergent gegen $g$ ist $g$ stetig auf $K$ (also $g \in L$) und 
				\[
					\|Af_{j_k} - g\| \to 0
				\]
		\end{enumerate}
	\end{proof}
\end{ex}


\begin{st}[Azelá-Ascoli] \label{1.29}
	Sei $K \subset \R^n$ kompakt und $f_j : K \to \C$, $(f_j)$ punktweise beschränkt $(\forall x \in K \exists c_x >0 \forall j\in \N : |f_j(x)|\le c_x)$ und gleichgradig stetig, also:
	\[
		\forall \eps > 0 \exists \delta > 0 \forall x,x' \in K \forall j \in \N : \|x-x'\| < \delta \implies |f_j(x)-f_j(x')| < \eps
	\]
	Dann besitzt $(f_j)$ eine Teilfolge, die auf $K$ gleichmäßig gegen $f\in C(K \to \C)$ konvergiert.
	\begin{proof}
		\begin{enumerate}[1)]
			\item
				Es existiert $M \subset K$ mit $M$ abzählbar und dicht in $K$.
				\begin{proof}
					Da $K$ kompakt, lässt sich folgende endliche Teilüberdeckung (von $\bigcup_{x\in K} K_{\f 1l}(x)$) finden:
					\begin{align*}
						K &\subset \bigcup_{j=1}^{N_1} K_1(x_j^{(1)})
						\qquad x_j^{(1)} \in K, K_\delta(x) := \{y \in \R^n : \|y-x\| < \delta \} \\
						K &\subset \bigcup_{j=1}^{N_1} K_{\f 12}(x_j^{(2)})
						\qquad x_j^{(2)} \in K \\
						K &\subset \bigcup_{j=1}^{N_1} K_{\f 13}(x_j^{(3)})
						\qquad x_j^{(3)} \in K \\
						\vdots \quad &\subset \qquad \vdots
					\end{align*}
					Definiere
					\[
						M := \bigcup_{k=1}^\infty \Big\{x_1^{(k)}, \dotsc, x_{N_k}^{(k)}\Big\} =: \{\xi_1, \xi_2, \dotsc \}
					\]
					Offensichtlich ist $M \subset K$, $M$ dicht und abzählbar.
				\end{proof}
			\item
				Konstruiere konvergente Teilfolge $(g_j)$ von $(f_j)$.
				\begin{proof}
					Wegen $(f_j(\xi_1))$ beschränkt existiert nach Bolzano Weierstraß eine konvergente Teilfolge $(f_j^{(1)}(\xi_1))$.
					Diese ist wiederum beschränkt in $\xi_2$, also existiert eine konvergente Teilfolge $(f_j^{(2)}(\xi_2))$ (auch $(f_j^{(2)}(\xi_1))$ konvergent), usw.

					Wähle jetzt $g_j := f_j^{(j)}$ als Diagonalfolge (Teilfolge von $(f_j)$).

					Dann ist $(g_j)$ Teilfolge von $(f_j^{(k)})$ und $(g_j(\xi_k))$ konvergent für alle $\xi_k \in M$.
				\end{proof}
			\item
				$(g_j)$ konvergiert gleichmäßig auf $K$
				\begin{proof}
					Zeige: $(g_j)$ ist gleichmäßige Cauchy-Folge:
					\[
						|g_j(x) - g_k(x)| < \eps
						\qquad \forall j,k > J, x \in K
					\]
					Es gilt
					\begin{align*}
						|g_j(x) - g_k(x)| 
						&\le \underbrace{\Big|g_j(x)-g_j(\xi_l)\Big|}_{\text{$< \eps$ falls $x-\xi_l|<\delta$}} 
						+ \underbrace{\Big|g_j(\xi_l) - g_k(\xi_l)\Big|}_{\text{$<\eps$ falls $j,k > J$, $J$ abhängig von $\xi_l$}}  + \underbrace{\Big| g_k(\xi_l) - g_k(x)\Big|}_{\text{$<\eps$ falls $x-\xi_l < \delta$, $\delta$ unabhängig von $k$ bzw. $j$}} 
					\end{align*}
					\begin{enumerate}[a)]
						\item
							Wähle $\{\xi_{l_1}, \dotsc, \xi_{l_N} \subset M$ so dass $K \in \bigcup_{m=1}^N K_\delta (\xi_{l_m})$ (vgl 1.).
						\item
							Wähle $J \in \N$, sodass 
							\[
								\Big| g_j(\xi_{l_m}) - g_k{\xi_{l_m}} \Big| < \eps
							\]
							für $j,k > J$ für alle $\xi_{l_1}, \dotsc, \xi_{l_N}$ (endlich viele).
					\end{enumerate}
					Also ist $g_j(x) - g_k(x)| < 3 \eps$ für $j,k > J$, $x\in K$.
				\end{proof}
			\item
				Die $g_j$ sind stetig und gleichmäßig konvergent, also ist
				\begin{align*}
					g: K &\to \C \\
					x &\mapsto \lim_{j\to \infty} g_j(x)
				\end{align*}
		\end{enumerate}
	\end{proof}
\end{st}



\chapter{\texorpdfstring{$L^p$}{Lp}-Räume}


\begin{df} \label{2.1}
	\begin{enumerate}[1)]
		\item
			Ein Maßraum $(\Omega, \Sigma, \my)$ besteht aus einer Menge $\Omega$, einer $\sigma$-Algebra $\Sigma \subset P(\Omega)$ (d.h. $\emptyset \in \Sigma, A^C \in \Sigma, \bigcup_{j\in \N} Aj \in \Sigma$) und einem Maß $\my$ (d.h. $\my(\emptyset) = 0, \my(A) \ge 0, \my(\dot\bigcup_{j\in \N}A_j = \sum_{j\in \N} \my(A_j))$).
		\item
			$f: \Omega \to \Omega'$ heißt messbar, falls
			\[
				\forall A' \in \Sigma' : f^{-1}(A') \in \Sigma
			\]
			(ideal ist $\Sigma'$ „klein“, $\Sigma$ „groß“, also „viele“ $f$ messbar)
	\end{enumerate}
\end{df}

\begin{ex}[Anwendung]
	Sei $\Sigma = \R^n$ und $\Sigma$ die Borel $\sigma$-Algebra, d.h. die von den offenen Intervallen $\bigtimes_{j=1}^n ]a_j,b_j[$ erzeugte (kleinste) $\sigma$-Algebra
	\[
		\my \bigg( \bigtimes_{j=1}^n ]a_j,b_j[ \bigg) := \prod_{j=1}^n (b_j - a_j)
	\]
	für $b_j > a_j$ fortgesetzt auf $\Sigma$ heißt \emph{Lebesgue-Borel-Maß}.

	Definiere die Vervollständigung:
	\[
		\Sigma^* := \bigg\{ A \subset \Omega : \exists B,C \in \Sigma : B \subset A \subset C \land \my(C \setminus B) = 0 \bigg\}
	\]
	Dann heißt
	\[
		\my^*(A) := \my(B) = \my(C)
	\]
	\emph{Lebesgue-Maß} auf $\R^n$.

	Für $A \in \Sigma$ verwende
	\begin{align*}
		\Sigma_A := \{ B \cap A : B \in \Sigma \} \\
		\my_A := \my^* \Big|_{\Sigma_A}
	\end{align*}
\end{ex}

\begin{df*}
	Eine Funktion $\chi_M: \Sigma \to \{0,1\}$ mit
	\[
		\chi_M (x) = \begin{cases}
			1 & x\in M \\
			0 & x \in \Omega \setminus M
		\end{cases}
	\]
	nennt man \emph{Indikatorfunktion} oder \emph{charakteristische Funktion}.

	$s$ ist genau dann eine \emph{einfache} Funktion, wenn
	\[
		\exists N \in \N, \alpha_1, \dotsc, \alpha_N \in \R, M_1, \dotsc, M_N \in \Sigma : s = \sum_{j=1}^N \alpha_j \chi_{M_j}
	\]
	Dann kann man
	\[
		\int_{\Omega} s d\my := \sum_{j=1}^N \alpha_j \my(M_j)
	\]
	für $s \ge 0$ (also  $\alpha_j \ge 0$) wählen.
\end{df*}

\begin{df}[Lebesgue-Integral]
	Sei $(\Omega, \Sigma, \my)$ ein Maßraum.
	
	\begin{enumerate}[1)]
		\item
			Sei $f: \Omega \to \R$ Borel-messbar (d.h. im Bildraum wird die Borel-$\sigma$-Algebra verwendet).

			\begin{enumerate}[a)]
				\item
					Definiere für $f \ge 0$:
					\[
						\int_{\Omega} f d\my 
						:= \sup \bigg\{ \int_{\Omega} s d\my : 0 \le s(x) \le f(x), x \in \Omega \land s : \Omega \to \R \text{ einfach} \bigg\}
					\]
				\item
					Für $f: \Omega \to \R$ ohne Vorzeichenbedingung definiert man
					\begin{align*}
						f_+ (x) &:= \max\{f(x), 0 \} \qquad x \in \Omega \\
						f_- (x) &:= \min\{ f(x), 0 \} \qquad x \in \Omega
					\end{align*}
					Dann ist $f_+, f_- \ge 0$, $f = f_+ - f_-$ und $f_+, f_-$ messbar.
					
					Definiere
					\[
						\int_{\Omega} f d\my := \int_{\Omega} f_+ d\my - \int_{\Omega} f_- d\my
					\]
					solange nicht beide Integrale der rechten Seite $\infty$ sind.

					Wir nennen $f$ integrierbar, wenn $f$ messbar ist und obiges Integral definiert ist.
			\end{enumerate}
		\item
			$f : \Omega \to \C$ heißt \emph{messbar}, falls $\Re f$ und $\Im f$ Borel-messbar sind.

			$f$ heißt \emph{integrierbar}, falls $\Re f$ und $\Im f$ integrierbar sind und endlich. Dann
			\[
				\int_{\Omega} f d\my := \int_{\Omega} (\Re f) d\my + i \int_{\Omega} (\Im f) d\my 
			\]
	\end{enumerate}

\end{df}
	

\begin{df} \label{2.4}
	Sei $f: (\Omega, \Sigma, \my) \to \C$ messbar.
	\begin{enumerate}[1)]
		\item
			Für $1 \le p < \infty$ ist
			\[
				N_p(f) := \bigg( \int_{\Omega} |f|^p d\my \bigg)^{\f 1p}
				\qquad \Big( |f|^n = (\cdot)^n \circ |f| \Big)
			\]
			messbar.
		\item
			Wir nennen
			\begin{align*}
					\displaystyle N_\infty(f) 
					&:= \inf \Big\{ c \in [0,\infty] : |f|\le c \text{ $\my$-fast-überall} \Big\} \\
					&= \esssup_{\omega \in \Omega} |f(\omega)|
			\end{align*}
			\emph{wesentliches Supremum}.
	\end{enumerate}
\end{df}

\begin{st}[Eigenschaften] \label{2.5}
	\begin{enumerate}[1)]
		\item
			Für $1 \le p \le \infty$ gilt $0 \le N_p(f) \le \infty$ und $N_p(\alpha f) = |\alpha| N_p(f)$ für $\alpha \in \C$.
		\item
			Es gilt $|f| \le N_\infty(f)$ $\my$-fast-überall und
			\begin{align*}
				\my \Big\{ \omega \in \Omega : |f(\omega)| > N_\infty(f) \Big\}
				&= \my \bigg( \bigcup_{n\in \N} \Big\{ \omega \in \Omega : |f(\omega)| > N_\infty(f) + \f 1n \Big\} \bigg) \\
				&= \lim_{n\to \infty }\underbrace{\my \bigg( \Big\{ \omega \in \Omega : |f(\omega)| > N_\infty(f) + \f 1n \Big\} \bigg)}_{=0} \\
				&=  0
			\end{align*}
			Insbesondere ist das Infimum in \ref{2.4} 2) ein Minimum.
		\item
			Es gilt
			\[
				N_\infty(f+g) \le N_\infty (f) + N_\infty(g)
			\]
			\begin{proof}
				\begin{align*}
					|(f+g)(\omega)|
					& \le |f(\omega)| + |g(\omega)| \\
					& \le N_\infty(f) + N_\infty(g) \qquad \text{$\my$-fast-überall}
				\end{align*}
			\end{proof}
	\end{enumerate}
\end{st}

\begin{df} \label{2.6}
	$p,q$ mit $1 \le p,q \le \infty$ heißen \emph{konjugiert}, falls
	\[
		1 < p, q < \infty \quad\land\quad \f 1p + \f 1q = 1
	\]
	oder
	\[
		p = 1 \quad\land\quad q = \infty
	\]
	oder
	\[
		p = \infty \quad\land\quad q = 1
	\]
	\begin{note}
		Ein wichtiger Spezialfall ist $p=q=2$.
	\end{note}
\end{df}


\begin{st} \label{2.7}
	Seien $f,g : (\Omega, \Sigma, \my) \to \C$ messbar.
	\begin{enumerate}[1)]
		\item
			Für $1 < p,q < \infty$ und $p,q$ konjugiert gilt die \emph{Höldersche Ungleichung}:
			\[
				\int_{\Omega} |fg| d\my \le \bigg(\int_{\Omega}|f|^p d\my \bigg)^{\f 1p} \bigg( \int_{\Omega} |g|^q d\my \bigg)^{\f 1q}
			\]
			oder in anderer Schreibweise:
			\[
				N_1(fg) \le N_p(f) \cdot N_q(g)
			\]
		\item
			Für $1 \le p < \infty$ gilt die \emph{Minkowskische Ungleichung}:
			\[
				\bigg( \int_{\Omega} |f+g|^p d\my \bigg)^{\f 1p} \le  \bigg(\int_{\Omega} |f|^p d\my\bigg)^{\f 1p} + \bigg( \int_{\Omega} |g|^p d\my \bigg)^{\f 1p}
			\]
			oder in anderer Schreibweise:
			\[
				N_p(f+g) \le N_p(f) + N_p(g)
			\]
	\end{enumerate}
	\begin{proof}
		\begin{enumerate}[1)]
			\item
				\begin{enumerate}[a)]
					\item
						Für $N_p(f) = 0$ ist $f =0 $ $\my$-fast-überall, also $fg = 0$ $\my$-fast-überall und damit $N_1(fg) = 0$.
						Genauso $N_q(g) = 0 \implies N_1(fg) = 0$.

						Aus $N_p(f) > 0 \land N_q(g) = \infty$ und $N_q(g) \ge 0 \land N_p(f) = \infty$ folgt die Behauptung
					\item
						Für $N_p(f) = N_q(g) = 1$, zeige zunächst $N_1(fg) \le 1$.

						Wegen $t \mapsto e^t$ konvex (\fixme[Zeichnung]) gilt für $0 \le \lambda \le 1$ und $s,t \in \R$.
						\[
							e^{\lambda t + (1-\lambda)s} \le \lambda e^t + (1-\lambda)e^s
						\]
						Für $0 < x,y < \infty$ sei
						\[
							x = e^{\f \alpha p}, y = e^{\f \beta q} = e^{\beta (1 - \f 1p)}
						\]
						damit gilt
						\begin{align*}
							xy = e^{\f 1p \alpha + (1-\f 1p)\beta} 
							&= e^{\lambda \alpha + (1-\lambda) \beta} \qquad \lambda := \f 1p, \quad 1-\lambda = \f 1q \\
							&\le \lambda e^{\alpha}  + (1-\lambda) e^{\beta} \\
							&= \f 1p x^p + \f 1q y^q
						\end{align*}
						Also für $0 \le x,y < \infty$:
						\[
							xy \le \f 1p x^p + \f 1q y^q
						\]

						Damit gilt
						\begin{align*}
							N_1(fg) 
							&= \int_{\Omega} |fg| d\my \\
							&\le \int_{\Omega} \f 1p |f|^p + \f 1q |g|^q d\my \\
							&\le \f 1p \underbrace{N_p(f)^p}_{=1} + \f 1q \underbrace{N_q(g)^q}_{=1} \\
							&\le \f 1p + \f 1q = 1
						\end{align*}
					\item
						Sei $0 < N_p(f), N_q(g) < \infty$, dann ist
						\begin{align*}
							\int_{\Omega}|fg| d\my
							&= N_p(f)N_q(g) \int_{\Omega} \underbrace{\Big| \f f{N_p(f)} \Big|}_{N_p(\dots)=1} \underbrace{\Big| \f g{N_q(g)} \Big|}_{N_q(\dots)=1} d\my \\
							&\le N_p(f) N_q(g) \cdot 1
						\end{align*}
				\end{enumerate}
			\item
				Der Fall $p=1$ folgt direkt aus $|f+g| \le |f| + |g|$ (punktweise).

				Triviale Fäll sind $N_p(f+g) = 0$, $N_p(f) = \infty$ und $N_p(g) = \infty$.

				Betrachte nun $N_p(f), N_p(g) < \infty$ und $N_p(f+g) > 0$ und $p>1$.
				\begin{enumerate}[a)]
					\item
						Für $N_p(f+g) < \infty$ gilt punktweise:
						\begin{align*}
							|f+g|^p
							&\le \Big( |f| + |g| \Big)^p \\
							&\le \Big( 2 \max \{|f|, |g|\} \Big)^p \\
							&= 2^p \max \{|f|^p, |g|^p \} \\
							&\le 2^p \Big( |f|^p + |g|^p \Big)
						\end{align*}
						und damit
						\[
							N_p(f+g) \le 2^p \Big( N_p(f) + N_q(g) \Big) < \infty
						\]
					\item
						\begin{align*}
							\Big( N_p(f+g) \Big)^p
							&= \int_{\Omega} |f+g|^p d\my \\
							&= \int_{\Omega} |f+g| |f+g|^{p-1} d\my \\
							&\le \int_{\Omega} |f| |f+g|^{p-1} d\my + \int_{\Omega} |g| |f+g|^{p-1} d\my
						\intertext{Wähle $q \in ]1,\infty[$ mit $\f 1q + \f 1p = 1$ (oder äquivalent $\f pq = p-1$) und wende die Höldersche Ungleichung an:}
							&\le \underbrace{\bigg( \int_{\Omega} |f|^p d\my \bigg)^{\f 1p}}_{N_p(f)} \underbrace{\bigg( \int_{\Omega} |f+g|^{q(p-1)} d\my \bigg)^{\f 1q}}_{N_p(f+g)^{\f pq}} 
							+ \underbrace{\bigg( \int_{\Omega} |g|^p d\my \bigg)^{\f 1p}}_{N_p(g)} \underbrace{\bigg( \int_{\Omega} |f+g|^{q(p-1)} d\my \bigg)^{\f 1q} }_{N_p(f+g)^{\f pq}}
						\end{align*}
						Wegen $0 < N_p(f+g)^{\f pq} < \infty$ also
						\[
							\Big(N_p(f+g)\Big)^{p- \f pq} \le N_p(f) + N_p(g)
						\]
						Wegen $p-\f pq = 1$ ist dies genau die Behauptung.
				\end{enumerate}
		\end{enumerate}
	\end{proof}
\end{st}

\begin{kor} \label{2.8}
	\begin{enumerate}[1)]
		\item
			Für $1 \le p,q \le \infty$ und $p,q$ konjugiert gilt
			\[
				N_1(fg) \le N_p(f) N_q(g)
			\]
			\begin{note}
				Wichtiger Spezialfall:
				\begin{align*}
					\bigg| \int_{\Omega} f \_g d\my \bigg|
					&\le \int_{\Omega} |f \_g| d\my \\
					&\le N_2(f) N_2(g) \\
					&= \bigg( \int_{\Omega} |f|^2 d\my \bigg)^{\f 12} \bigg( \int_{\Omega} |g|^2 d\my \bigg)^{\f 12}
				\end{align*}
				Das entspricht der CSB im $L^2$.
			\end{note}
		\item
			Für $1 \le p \le \infty$ gilt die Dreiecksungleichung
			\[
				N_p(f+g) \le N_p(f) + N_p(g)
			\]
	\end{enumerate}
	\begin{proof}
		\begin{enumerate}[1)]
			\item
				Für $1 < p,q < \infty$ siehe Höldersche Ungleichung.

				Sei $p=1, q = \infty$.
				Dann folgt aus \ref{2.5} $|g| \le N_\infty(g)$ $\my$-fast-überall und damit
				\[
					N_1(fg) = \int_{\Omega} |fg| d\my \le \int_{\Omega}|f| N_\infty (g) d\my
				\]
			\item
				Vergleiche \ref{2.5} und Minkowski.
		\end{enumerate}
	\end{proof}
\end{kor}

\begin{df*} 
	$m : L \to \R$ heißt \emph{Halbnorm}, falls
	\begin{enumerate}[a)]
		\item
			$\displaystyle m(\alpha x) = |\alpha| m(x)$
		\item
			$m$ positiv ist.
		\item
			die Dreiecksungleichung erfüllt ist:
			\[
				m(x+y) \le m(x) + m(y)
			\]
	\end{enumerate}
	\begin{note}
		Im Vergleich zur \emph{Norm} wird also auf die positive Definitheit verzichtet.
	\end{note}
\end{df*}

\begin{df} \label{2.9}
	Sei 
	\[
		\tilde {L^p} (\Omega, \Sigma, \my)
		:= \Big\{ f : \Omega \to \C \text{ messbar} : N_p(f) < \infty \Big\}
	\]
	Dann ist $\tilde{L^p}(\dots)$ ein linearer Raum (Vektorraum) und $N_p$ ist Halbnorm auf $\tilde{L^p}(\dots)$.
\end{df}

\begin{df} \label{2.10}
	Sei $1 \le p \le \infty$ und
	\begin{align*}
		N := \Big\{ f \in \tilde {L^p}(\Omega, \Sigma, \my) : N_p(f) = 0 \Big\}
	\end{align*}
	Durch
	\[
		f \sim g :\iff f-g \in N
	\]
	wird eine Äquivalenzrelation auf $\tilde{L^p}(\dots)$ definiert.

	Setze
	\begin{align*}
		L^p(\Omega, \Sigma, \my)
		&:= \Big\{ [f] : f \in \tilde{L^p}(\Omega, \Sigma, \my) \Big\}
		= \tilde {L^p} (\Omega, \Sigma, \my) / N \\
		\| [f] \|_p &:= N_p(f) \qquad \text{ für } [f] \in L^p(\dots)
	\end{align*}
	Dann ist $\|\cdot\|_p$ eine Norm auf $L^p(\dots)$ (leicht zu zeigen).

	Im Folgenden schreiben wir $L^p$ oder $L^p(\Omega)$ statt $L^p(\Omega, \Sigma, \my)$ und $f$ statt $[f]$.
\end{df}

\begin{nt} \label{2.11}
	Sei $\Omega \subset \R^n$ und $f,g \in C(\Omega \to \C) \cap L^p(\Omega)$ mit $\|f-g\|_p = 0$.
	Dann gilt
	\begin{align*}
		\int_{\Omega} |f-g|^p d\my = 0
		\quad\implies\quad
		f = g \text{ auf $\Omega$}
	\end{align*}
	d.h. falls ein Vertreter $f$ der Äquivalenzklasse $[f]$ stetig ist, sind alle anderen Vertreter unstetig.
	Oder mit anderen Worten: jede Äquivalenzklasse $[f]$ enthält höchstens einen stetigen Vertreter.
\end{nt}

\begin{st}[Fischer-Riesz] \label{2.12}
	Für $1 \le p \le \infty$ ist $L^p(\Omega, \Sigma, \my)$ ein Banachraum.
	\begin{proof}
		Wir müssen nur noch die Vollständigkeit nachweisen.
		\begin{seg}[$p=\infty$]
			Sei $(f_j)$ Cauchy-Folge.
			Definiere
			\[
				B_{jk} := \Big\{ \omega \in \Omega : |f_j(\omega) - f_k(\omega)| > \|f_j-f_k\|_\infty \Big\}
			\]
			Nach \ref{2.5} ist dann $\my(B_{jk}) = 0$ für $j,k\in \N$. 
			Setze
			\[
				B := \bigcup_{j,k\in \N} B_{jk}
			\]
			Dann ist wieder $\my(B) = 0$.

			Für $\omega \in \Omega \setminus B$ gilt
			\[
				|f_j(\omega) - f_k(\omega)| \le \|f_j-f_k\|_\infty.
			\]
			Da $\C$ vollständig ist, konvergiert $(f_j)$ auf $\Omega \setminus B$ gleichmäßig.

			Definiere
			\[
				f(\omega) := \begin{cases}
					\lim_{j \to \infty} f_j(\omega) & \omega \in \Omega \setminus B \\
					0 & \text{sonst}
				\end{cases}
			\]
			Dann ist
			\[
				f := \lim_{j\to \infty} \chi_{\Omega \setminus B} f_j
			\]
			als punktweiser Grenzwert von messbaren Funktionen wieder messbar.
			Außerdem gilt
			\begin{align*}
				|f_j(\omega) - f(\omega) |
				= \lim_{h\to \infty} |f_j(\omega) - f_k(\omega)| \\
				\le \lim_{h\to \infty} \|f_j - f_k\|_{\infty}
				\le \eps
			\end{align*}
			für ein $j > J_\eps$ und $\omega \in \Omega \setminus B$.

			Da $B$ Nullmenge, ist $\|f_j - f\|_\infty \le \eps$ und daher $f_j - f \in L^\infty$ für $j > J_\eps$.
			Somit ist
			\[
				f = \underbrace{f - f_j}_{\in L^\infty} + \underbrace{f_j}_{\in L^\infty} \in L^\infty
			\]
		\end{seg}
		\begin{seg}[$1 \le p < \infty$]
			Sei $(f_j)$ Cauchy-Folge.
			\begin{enumerate}[a)]
				\item
					Wähle eine „gut konvergente“ Teilfolge $({f_j}_k)$.
					Sei dazu $j_l$ so, dass 
					\[
						\|{f_j}_l - f_k\|_p < \f 1{2^k} \qquad \text{für $k>j_l$}
					\]
					Damit gilt
					\[
						\|{f_j}_k - {f_j}_{k+1} \|_p < \f 1{2^k} \qquad \text{für $k \in \N$}
					\]
				\item
					Zeige jetzt die Existenz der Grenzfunktion.

					Setze
					\begin{align*}
						g_n(\omega) &:= \sum_{k=1}^n |{f_j}_k(\omega) - {f_j}_{k+1}(\omega)|  \qquad \omega \in \Omega \\
						g(\omega) &:= \sum_{k=1}^\infty |{f_j}_k(\omega) - {f_j}_{k+1}(\omega)| 
					\end{align*}
					(evtl. ist $g(\omega) = \infty$)
					Dann gilt
					\begin{align*}
						\|g_n\|_p &= \bigg\|\sum_{k=1}^n |{f_j}_k(\omega) - {f_j}_{k+1}(\omega)| \bigg\| \\
						&\le \sum_{k=1}^n \|{f_j}_k - {f_j}_{k+1} \|_p \\
						&\le \sum_{k=1}^\infty \f 1{2^k}
						= 1
					\end{align*}
					Damit konvergiert $0 \le g_n$ monoton wachsend gegen $g$ und somit auch $0 \le g_n^p$ monoton wachsend gegen $g^p$ (jeweils punktweise).
					Wegen $g_n$ messbar, ist $g_n^p$ messbar und somit wegen der punktweisen Konvergenz auch $g^p$.

					Nach dem Satz über monotone Konvergenz gilt
					\begin{align*}
						\|g\|_p = \int_{\Omega} g^p d\my
						&= \int_{\Omega} \lim_{n\to \infty} g_n^p d\my \\
						&= \lim_{n\to \infty} \int_{\Omega} g_n^p d\my
						&= \lim_{n\to \infty} \|g_n\|_p^p
						\le 1
					\end{align*}
					Also $\|g\|_p \le 1$ und somit
					\[
						g(\omega) < \infty
					\]
					$\my$-fast-überall.
					Sei $\Omega'$ so, dass $g(\omega) < \infty$ für $\omega \in \Omega'$ und $\my(\Omega \setminus \Omega') = 0$.

					Also ist $g_n(\omega) \to g(\omega)$ in $\R$ für $\omega \in \Omega'$.
					Mit 
					\[
						{f_j}_k(\omega) = {f_j}_1(\omega) + \sum_{l=1}^{k-1} \Big({f_j}_{l+1}(\omega) - {f_j}_l(\omega)\Big)
					\]
					und der absoluten Konvergenz
					\[
						\sum_{l=1}^{k-1} \Big|{f_j}_{l+1}(\omega) - {f_j}_l(\omega)\Big| < \infty
					\]
					folgt, dass $({f_j}_k(\omega))$ konvergent ist für $k \to \infty$ und $\omega \in \Omega'$.

					Setze
					\[
						f(\omega) := \begin{cases}
							\lim_{k\to \infty} {f_j}_k (\omega) & \omega \in \Omega' \\
							0 & \omega \in \Omega \setminus \Omega'
						\end{cases}
					\]
				\item
					Zeige jetzt $f \in L^p$ und $\|f_j - f\|_p \to 0$.

					Sei $\eps > 0$ und $K \in \N$ mit
					\[
						\|{f_j}_k - {f_j}_l \|_p < \eps  \qquad \text{für $k,l > K$.}
					\]
					Nach dem Lemma von Fatou ist
					\begin{align*}
						\int_{\Omega} |{f_j}_k - f|^p d\my
						&= \int_{\Omega} \lim_{l\to \infty} |{f_j}_k - {f_j}_l |^p d\my
						&\stack{\text{Fatou}}\le \liminf_{l\to \infty} \underbrace{\int_{\Omega} |{f_j}_k - {f_j}_l |^p d\my}_{= \|{f_j}_k - {f_j}_l\|_p^p < \eps^p} \\
						&\le \eps^p
					\end{align*}
					($f$ ist messbar als punktweiser Grenzwert messbarer Funktionen)
					Also ist ${f_j}_k - f \in L^p$, $\|{f_j}_k - f\|_p \le \eps$ und somit $f = f - {f_j}_k + {f_j}_k \in L^p$.

					Wegen $(f_j)$ Cauchy-Folge, konvergiert die Teilfolge ${f_j}_k \to f$ bezüglich $\|\cdot\|_p$.
					Also
					\[
						\|f_j - f\|_p 
						\le \underbrace{\|f_j - {f_j}_k \|_p}_{< \eps} + \underbrace{\|{f_j}_k}_{< \eps} - f\|_p 
						< 2 \eps
					\]
					für hinreichend großes $j, j_k$ und $k$.
			\end{enumerate}
		\end{seg}
	\end{proof}
\end{st}

\begin{kor}[Wegl] \label{2.14}
	Sei $1 \le p \le \infty$ und $(f_j)$ Cauchy-Folge in $L^p$.
	
	Dann existiert $f \in L^p$ mit $\|f - f_j\|_p \to 0$ und eine Teilfolge $({f_j}_k)$ mit ${f_j}_k(\omega) \to f(\omega)$ $\my$-fast-überall in $\Omega$.
\end{kor}

\begin{st} \label{2.15}
	Sei $1 \le p < \infty$.
	\begin{enumerate}[1)]
		\item
			Für eine einfache Funktion $s$ gilt
			\[
				s \in L^p
				\quad \iff \quad
				\my\Big( \big\{ \omega \in \Omega : s(\omega) \neq 0 \big\} \Big) < \infty
			\]
		\item
			Die Menge
			\[
				\Big\{ s \in L^p : s \text{ ist einfach} \Big\}
			\]
			ist dicht in $L^p$.
	\end{enumerate}
	\begin{proof}
		\begin{enumerate}[1)]
			\item
				Dies ist eine einfache Übungsaufgabe.
			\item
				\begin{enumerate}[a)]
					\item
						Betrachte den Fall $f \ge 0$, $f \in L^p$.

						Es existiert eine Folge $(s_j)$ einfacher Funktionen mit (punktweise):
						\[
							0 \le s_1 \le s_2 \le \dotsb \le f
							\qquad \text{und} \qquad
							f = \lim_{j\to \infty} s_j
						\]
						(siehe Maßtheorie)
						Damit ist
						\[
							\int_{\Omega} |s_j|^p d\my
							\le \int_{\Omega} |f|^p d\my
							< \infty
						\]
						Also ist $s_j, f-s_j \in L^p$.
						Weiter gilt punktweise
						\[
							|\underbrace{f - s_j}_{\ge 0}|^p \le |f|^p.
						\]
						Mit dem Satz über majorisierte Konvergenz gilt
						\begin{align*}
							\lim_{j\to \infty} \|f - s_j\|^p
							&= \lim_{j\to \infty} \int_{\Omega} \underbrace{|f-s_j|^p}_{|f|^p} d\my  \qquad \int_{\Omega} |f|^p d\my < \infty\\
							&= \int_{\Omega} \lim_{j\to \infty} |f- s_j |^p d\my \\
							&= \int_{\Omega} 0 d\my
							= 0
						\end{align*}
						Also $\|f-s_j\|_p \to 0$.
					\item
						Für beliebige Funktionen $f: \Omega \to \C$ zerlege diese in vier positive Funktionen.
				\end{enumerate}
		\end{enumerate}
	\end{proof}
\end{st}

\begin{st} \label{2.16}
	\begin{enumerate}[1)]
		\item
			Sei $\my(\Omega) < \infty$, $1 \le p \le p' \le \infty$ und $f \in L^{p'}(\Omega)$.

			Dann ist $f \in L^p(\Omega)$ und 
			\[
				\|f\|_p \le \my(\Omega)^{\f 1p - \f 1{p'}} \|f\|_{p'}
			\]
			\begin{note}
				
			\end{note}
		\item
			Sei $1 \le p < p' \le \infty$, dann ist
			\[
				L^p(\R) \setminus L^{p'}(\R) \neq \emptyset \neq L^{p'}(\R) \setminus L^p(\R)
			\]
		\item
			Sei $\Omega = \N$, $\Sigma := P(\Omega)$ und $\my(M) := \#M$ für $M \subset \Omega$.
			Für $1 \le p \le \infty$ ist dann
			\begin{align*}
				L^p(\Omega, \Sigma, \my) &= \ell^p \\
				\ell^p &:= \bigg\{ (x_j) \text{ Folge in $\C$: } \sum_{j=1}^\infty |x_j|^p < \infty \bigg\} \\
				\|(x_j)\|_p &= \bigg( \sum_{j=1}^\infty |x_j|^p \bigg)^{\f 1p}
			\end{align*}
			Für $p = \infty$ ist
			\begin{align*}
				\ell^\infty &= \Big\{ \text{beschränkte Folgen} \Big\} \\
				\|(x_j)\|_\infty &:= \sup_{j\in \N} |x_j|
			\end{align*}
		\item
			Für den Spezialfall $p=2$ ist $L^2(\Omega, \Sigma, \my)$ vollständig und
			\[
				\|f\|_2 = \bigg( \int_{\Omega} |f|^2 d\my \bigg)^{\f 12}
			\]
			wird erzeugt vom Skalarprodukt $\<f,g\> := \int_{\Omega} f \_g d\my$ (Wegen Hölder: $\int_{\Omega} |f \_g|d\my \le \|f\|_2 \|g\|_2 < \infty$, konvergiert das Integral)

			Also ist $L^2$ ein Hilbertraum.
	\end{enumerate}
\end{st}

\end{document}
