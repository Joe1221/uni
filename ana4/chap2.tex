% This work is licensed under the Creative Commons
% Attribution-NonCommercial-ShareAlike 3.0 Unported License. To view a copy of
% this license, visit http://creativecommons.org/licenses/by-nc-sa/3.0/ or send
% a letter to Creative Commons, 444 Castro Street, Suite 900, Mountain View,
% California, 94041, USA.

\chapter{\texorpdfstring{$L^p$}{Lp}-Räume}


\begin{df} \label{2.1}
	\begin{enumerate}[1)]
		\item
			Ein Maßraum $(\Omega, \Sigma, \my)$ besteht aus einer Menge $\Omega$, einer $\sigma$-Algebra $\Sigma \subset P(\Omega)$ (d.h. $\emptyset \in \Sigma, A^C \in \Sigma, \bigcup_{j\in \N} Aj \in \Sigma$) und einem Maß $\my$ (d.h. $\my(\emptyset) = 0, \my(A) \ge 0, \my(\dot\bigcup_{j\in \N}A_j = \sum_{j\in \N} \my(A_j))$).
		\item
			$f: \Omega \to \Omega'$ heißt messbar, falls
			\[
				\forall A' \in \Sigma' : f^{-1}(A') \in \Sigma
			\]
			(ideal ist $\Sigma'$ „klein“, $\Sigma$ „groß“, also „viele“ $f$ messbar)
	\end{enumerate}
\end{df}

\begin{ex}[Anwendung]
	Sei $\Sigma = \R^n$ und $\Sigma$ die Borel $\sigma$-Algebra, d.h. die von den offenen Intervallen $\bigtimes_{j=1}^n ]a_j,b_j[$ erzeugte (kleinste) $\sigma$-Algebra
	\[
		\my \bigg( \bigtimes_{j=1}^n ]a_j,b_j[ \bigg) := \prod_{j=1}^n (b_j - a_j)
	\]
	für $b_j > a_j$ fortgesetzt auf $\Sigma$ heißt \emph{Lebesgue-Borel-Maß}.

	Definiere die Vervollständigung:
	\[
		\Sigma^* := \bigg\{ A \subset \Omega : \exists B,C \in \Sigma : B \subset A \subset C \land \my(C \setminus B) = 0 \bigg\}
	\]
	Dann heißt
	\[
		\my^*(A) := \my(B) = \my(C)
	\]
	\emph{Lebesgue-Maß} auf $\R^n$.

	Für $A \in \Sigma$ verwende
	\begin{align*}
		\Sigma_A := \{ B \cap A : B \in \Sigma \} \\
		\my_A := \my^* \Big|_{\Sigma_A}
	\end{align*}
\end{ex}

\begin{df*}
	Eine Funktion $\chi_M: \Sigma \to \{0,1\}$ mit
	\[
		\chi_M (x) = \begin{cases}
			1 & x\in M \\
			0 & x \in \Omega \setminus M
		\end{cases}
	\]
	nennt man \emph{Indikatorfunktion} oder \emph{charakteristische Funktion}.

	$s$ ist genau dann eine \emph{einfache} Funktion, wenn
	\[
		\exists N \in \N, \alpha_1, \dotsc, \alpha_N \in \R, M_1, \dotsc, M_N \in \Sigma : s = \sum_{j=1}^N \alpha_j \chi_{M_j}
	\]
	Dann kann man
	\[
		\int_{\Omega} s d\my := \sum_{j=1}^N \alpha_j \my(M_j)
	\]
	für $s \ge 0$ (also  $\alpha_j \ge 0$) wählen.
\end{df*}

\begin{df}[Lebesgue-Integral]
	Sei $(\Omega, \Sigma, \my)$ ein Maßraum.
	
	\begin{enumerate}[1)]
		\item
			Sei $f: \Omega \to \R$ Borel-messbar (d.h. im Bildraum wird die Borel-$\sigma$-Algebra verwendet).

			\begin{enumerate}[a)]
				\item
					Definiere für $f \ge 0$:
					\[
						\int_{\Omega} f d\my 
						:= \sup \bigg\{ \int_{\Omega} s d\my : 0 \le s(x) \le f(x), x \in \Omega \land s : \Omega \to \R \text{ einfach} \bigg\}
					\]
				\item
					Für $f: \Omega \to \R$ ohne Vorzeichenbedingung definiert man
					\begin{align*}
						f_+ (x) &:= \max\{f(x), 0 \} \qquad x \in \Omega \\
						f_- (x) &:= \min\{ f(x), 0 \} \qquad x \in \Omega
					\end{align*}
					Dann ist $f_+, f_- \ge 0$, $f = f_+ - f_-$ und $f_+, f_-$ messbar.
					
					Definiere
					\[
						\int_{\Omega} f d\my := \int_{\Omega} f_+ d\my - \int_{\Omega} f_- d\my
					\]
					solange nicht beide Integrale der rechten Seite $\infty$ sind.

					Wir nennen $f$ integrierbar, wenn $f$ messbar ist und obiges Integral definiert ist.
			\end{enumerate}
		\item
			$f : \Omega \to \C$ heißt \emph{messbar}, falls $\Re f$ und $\Im f$ Borel-messbar sind.

			$f$ heißt \emph{integrierbar}, falls $\Re f$ und $\Im f$ integrierbar sind und endlich. Dann
			\[
				\int_{\Omega} f d\my := \int_{\Omega} (\Re f) d\my + i \int_{\Omega} (\Im f) d\my 
			\]
	\end{enumerate}

\end{df}
	

\begin{df} \label{2.4}
	Sei $f: (\Omega, \Sigma, \my) \to \C$ messbar.
	\begin{enumerate}[1)]
		\item
			Für $1 \le p < \infty$ ist
			\[
				N_p(f) := \bigg( \int_{\Omega} |f|^p d\my \bigg)^{\f 1p}
				\qquad \Big( |f|^n = (\cdot)^n \circ |f| \Big)
			\]
			messbar.
		\item
			Wir nennen
			\begin{align*}
					\displaystyle N_\infty(f) 
					&:= \inf \Big\{ c \in [0,\infty] : |f|\le c \text{ $\my$-fast-überall} \Big\} \\
					&= \esssup_{\omega \in \Omega} |f(\omega)|
			\end{align*}
			\emph{wesentliches Supremum}.
	\end{enumerate}
\end{df}

\begin{st}[Eigenschaften] \label{2.5}
	\begin{enumerate}[1)]
		\item
			Für $1 \le p \le \infty$ gilt $0 \le N_p(f) \le \infty$ und $N_p(\alpha f) = |\alpha| N_p(f)$ für $\alpha \in \C$.
		\item
			Es gilt $|f| \le N_\infty(f)$ $\my$-fast-überall und
			\begin{align*}
				\my \Big\{ \omega \in \Omega : |f(\omega)| > N_\infty(f) \Big\}
				&= \my \bigg( \bigcup_{n\in \N} \Big\{ \omega \in \Omega : |f(\omega)| > N_\infty(f) + \f 1n \Big\} \bigg) \\
				&= \lim_{n\to \infty }\underbrace{\my \bigg( \Big\{ \omega \in \Omega : |f(\omega)| > N_\infty(f) + \f 1n \Big\} \bigg)}_{=0} \\
				&=  0
			\end{align*}
			Insbesondere ist das Infimum in \ref{2.4} 2) ein Minimum.
		\item
			Es gilt
			\[
				N_\infty(f+g) \le N_\infty (f) + N_\infty(g)
			\]
			\begin{proof}
				\begin{align*}
					|(f+g)(\omega)|
					& \le |f(\omega)| + |g(\omega)| \\
					& \le N_\infty(f) + N_\infty(g) \qquad \text{$\my$-fast-überall}
				\end{align*}
			\end{proof}
	\end{enumerate}
\end{st}

\begin{df} \label{2.6}
	$p,q$ mit $1 \le p,q \le \infty$ heißen \emph{konjugiert}, falls
	\[
		1 < p, q < \infty \quad\land\quad \f 1p + \f 1q = 1
	\]
	oder
	\[
		p = 1 \quad\land\quad q = \infty
	\]
	oder
	\[
		p = \infty \quad\land\quad q = 1
	\]
	\begin{note}
		Ein wichtiger Spezialfall ist $p=q=2$.
	\end{note}
\end{df}


\begin{st} \label{2.7}
	Seien $f,g : (\Omega, \Sigma, \my) \to \C$ messbar.
	\begin{enumerate}[1)]
		\item
			Für $1 < p,q < \infty$ und $p,q$ konjugiert gilt die \emph{Höldersche Ungleichung}:
			\[
				\int_{\Omega} |fg| d\my \le \bigg(\int_{\Omega}|f|^p d\my \bigg)^{\f 1p} \bigg( \int_{\Omega} |g|^q d\my \bigg)^{\f 1q}
			\]
			oder in anderer Schreibweise:
			\[
				N_1(fg) \le N_p(f) \cdot N_q(g)
			\]
		\item
			Für $1 \le p < \infty$ gilt die \emph{Minkowskische Ungleichung}:
			\[
				\bigg( \int_{\Omega} |f+g|^p d\my \bigg)^{\f 1p} \le  \bigg(\int_{\Omega} |f|^p d\my\bigg)^{\f 1p} + \bigg( \int_{\Omega} |g|^p d\my \bigg)^{\f 1p}
			\]
			oder in anderer Schreibweise:
			\[
				N_p(f+g) \le N_p(f) + N_p(g)
			\]
	\end{enumerate}
	\begin{proof}
		\begin{enumerate}[1)]
			\item
				\begin{enumerate}[a)]
					\item
						Für $N_p(f) = 0$ ist $f =0 $ $\my$-fast-überall, also $fg = 0$ $\my$-fast-überall und damit $N_1(fg) = 0$.
						Genauso $N_q(g) = 0 \implies N_1(fg) = 0$.

						Aus $N_p(f) > 0 \land N_q(g) = \infty$ und $N_q(g) \ge 0 \land N_p(f) = \infty$ folgt die Behauptung
					\item
						Für $N_p(f) = N_q(g) = 1$, zeige zunächst $N_1(fg) \le 1$.

						Wegen $t \mapsto e^t$ konvex (\fixme[Zeichnung]) gilt für $0 \le \lambda \le 1$ und $s,t \in \R$.
						\[
							e^{\lambda t + (1-\lambda)s} \le \lambda e^t + (1-\lambda)e^s
						\]
						Für $0 < x,y < \infty$ sei
						\[
							x = e^{\f \alpha p}, y = e^{\f \beta q} = e^{\beta (1 - \f 1p)}
						\]
						damit gilt
						\begin{align*}
							xy = e^{\f 1p \alpha + (1-\f 1p)\beta} 
							&= e^{\lambda \alpha + (1-\lambda) \beta} \qquad \lambda := \f 1p, \quad 1-\lambda = \f 1q \\
							&\le \lambda e^{\alpha}  + (1-\lambda) e^{\beta} \\
							&= \f 1p x^p + \f 1q y^q
						\end{align*}
						Also für $0 \le x,y < \infty$:
						\[
							xy \le \f 1p x^p + \f 1q y^q
						\]

						Damit gilt
						\begin{align*}
							N_1(fg) 
							&= \int_{\Omega} |fg| d\my \\
							&\le \int_{\Omega} \f 1p |f|^p + \f 1q |g|^q d\my \\
							&\le \f 1p \underbrace{N_p(f)^p}_{=1} + \f 1q \underbrace{N_q(g)^q}_{=1} \\
							&\le \f 1p + \f 1q = 1
						\end{align*}
					\item
						Sei $0 < N_p(f), N_q(g) < \infty$, dann ist
						\begin{align*}
							\int_{\Omega}|fg| d\my
							&= N_p(f)N_q(g) \int_{\Omega} \underbrace{\Big| \f f{N_p(f)} \Big|}_{N_p(\dots)=1} \underbrace{\Big| \f g{N_q(g)} \Big|}_{N_q(\dots)=1} d\my \\
							&\le N_p(f) N_q(g) \cdot 1
						\end{align*}
				\end{enumerate}
			\item
				Der Fall $p=1$ folgt direkt aus $|f+g| \le |f| + |g|$ (punktweise).

				Triviale Fäll sind $N_p(f+g) = 0$, $N_p(f) = \infty$ und $N_p(g) = \infty$.

				Betrachte nun $N_p(f), N_p(g) < \infty$ und $N_p(f+g) > 0$ und $p>1$.
				\begin{enumerate}[a)]
					\item
						Für $N_p(f+g) < \infty$ gilt punktweise:
						\begin{align*}
							|f+g|^p
							&\le \Big( |f| + |g| \Big)^p \\
							&\le \Big( 2 \max \{|f|, |g|\} \Big)^p \\
							&= 2^p \max \{|f|^p, |g|^p \} \\
							&\le 2^p \Big( |f|^p + |g|^p \Big)
						\end{align*}
						und damit
						\[
							N_p(f+g) \le 2^p \Big( N_p(f) + N_q(g) \Big) < \infty
						\]
					\item
						\begin{align*}
							\Big( N_p(f+g) \Big)^p
							&= \int_{\Omega} |f+g|^p d\my \\
							&= \int_{\Omega} |f+g| |f+g|^{p-1} d\my \\
							&\le \int_{\Omega} |f| |f+g|^{p-1} d\my + \int_{\Omega} |g| |f+g|^{p-1} d\my
						\intertext{Wähle $q \in ]1,\infty[$ mit $\f 1q + \f 1p = 1$ (oder äquivalent $\f pq = p-1$) und wende die Höldersche Ungleichung an:}
							&\le \underbrace{\bigg( \int_{\Omega} |f|^p d\my \bigg)^{\f 1p}}_{N_p(f)} \underbrace{\bigg( \int_{\Omega} |f+g|^{q(p-1)} d\my \bigg)^{\f 1q}}_{N_p(f+g)^{\f pq}} 
							+ \underbrace{\bigg( \int_{\Omega} |g|^p d\my \bigg)^{\f 1p}}_{N_p(g)} \underbrace{\bigg( \int_{\Omega} |f+g|^{q(p-1)} d\my \bigg)^{\f 1q} }_{N_p(f+g)^{\f pq}}
						\end{align*}
						Wegen $0 < N_p(f+g)^{\f pq} < \infty$ also
						\[
							\Big(N_p(f+g)\Big)^{p- \f pq} \le N_p(f) + N_p(g)
						\]
						Wegen $p-\f pq = 1$ ist dies genau die Behauptung.
				\end{enumerate}
		\end{enumerate}
	\end{proof}
\end{st}

\begin{kor} \label{2.8}
	\begin{enumerate}[1)]
		\item
			Für $1 \le p,q \le \infty$ und $p,q$ konjugiert gilt
			\[
				N_1(fg) \le N_p(f) N_q(g)
			\]
			\begin{note}
				Wichtiger Spezialfall:
				\begin{align*}
					\bigg| \int_{\Omega} f \_g d\my \bigg|
					&\le \int_{\Omega} |f \_g| d\my \\
					&\le N_2(f) N_2(g) \\
					&= \bigg( \int_{\Omega} |f|^2 d\my \bigg)^{\f 12} \bigg( \int_{\Omega} |g|^2 d\my \bigg)^{\f 12}
				\end{align*}
				Das entspricht der CSB im $L^2$.
			\end{note}
		\item
			Für $1 \le p \le \infty$ gilt die Dreiecksungleichung
			\[
				N_p(f+g) \le N_p(f) + N_p(g)
			\]
	\end{enumerate}
	\begin{proof}
		\begin{enumerate}[1)]
			\item
				Für $1 < p,q < \infty$ siehe Höldersche Ungleichung.

				Sei $p=1, q = \infty$.
				Dann folgt aus \ref{2.5} $|g| \le N_\infty(g)$ $\my$-fast-überall und damit
				\[
					N_1(fg) = \int_{\Omega} |fg| d\my \le \int_{\Omega}|f| N_\infty (g) d\my
				\]
			\item
				Vergleiche \ref{2.5} und Minkowski.
		\end{enumerate}
	\end{proof}
\end{kor}

\begin{df*} 
	$m : L \to \R$ heißt \emph{Halbnorm}, falls
	\begin{enumerate}[a)]
		\item
			$\displaystyle m(\alpha x) = |\alpha| m(x)$
		\item
			$m$ positiv ist.
		\item
			die Dreiecksungleichung erfüllt ist:
			\[
				m(x+y) \le m(x) + m(y)
			\]
	\end{enumerate}
	\begin{note}
		Im Vergleich zur \emph{Norm} wird also auf die positive Definitheit verzichtet.
	\end{note}
\end{df*}

\begin{df} \label{2.9}
	Sei 
	\[
		\tilde {L^p} (\Omega, \Sigma, \my)
		:= \Big\{ f : \Omega \to \C \text{ messbar} : N_p(f) < \infty \Big\}
	\]
	Dann ist $\tilde{L^p}(\dots)$ ein linearer Raum (Vektorraum) und $N_p$ ist Halbnorm auf $\tilde{L^p}(\dots)$.
\end{df}

\begin{df} \label{2.10}
	Sei $1 \le p \le \infty$ und
	\begin{align*}
		N := \Big\{ f \in \tilde {L^p}(\Omega, \Sigma, \my) : N_p(f) = 0 \Big\}
	\end{align*}
	Durch
	\[
		f \sim g :\iff f-g \in N
	\]
	wird eine Äquivalenzrelation auf $\tilde{L^p}(\dots)$ definiert.

	Setze
	\begin{align*}
		L^p(\Omega, \Sigma, \my)
		&:= \Big\{ [f] : f \in \tilde{L^p}(\Omega, \Sigma, \my) \Big\}
		= \tilde {L^p} (\Omega, \Sigma, \my) / N \\
		\| [f] \|_p &:= N_p(f) \qquad \text{ für } [f] \in L^p(\dots)
	\end{align*}
	Dann ist $\|\cdot\|_p$ eine Norm auf $L^p(\dots)$ (leicht zu zeigen).

	Im Folgenden schreiben wir $L^p$ oder $L^p(\Omega)$ statt $L^p(\Omega, \Sigma, \my)$ und $f$ statt $[f]$.
\end{df}

\begin{nt} \label{2.11}
	Sei $\Omega \subset \R^n$ und $f,g \in C(\Omega \to \C) \cap L^p(\Omega)$ mit $\|f-g\|_p = 0$.
	Dann gilt
	\begin{align*}
		\int_{\Omega} |f-g|^p d\my = 0
		\quad\implies\quad
		f = g \text{ auf $\Omega$}
	\end{align*}
	d.h. falls ein Vertreter $f$ der Äquivalenzklasse $[f]$ stetig ist, sind alle anderen Vertreter unstetig.
	Oder mit anderen Worten: jede Äquivalenzklasse $[f]$ enthält höchstens einen stetigen Vertreter.
\end{nt}

\begin{st}[Fischer-Riesz] \label{2.12}
	Für $1 \le p \le \infty$ ist $L^p(\Omega, \Sigma, \my)$ ein Banachraum.
	\begin{proof}
		Wir müssen nur noch die Vollständigkeit nachweisen.
		\begin{seg}[$p=\infty$]
			Sei $(f_j)$ Cauchy-Folge.
			Definiere
			\[
				B_{jk} := \Big\{ \omega \in \Omega : |f_j(\omega) - f_k(\omega)| > \|f_j-f_k\|_\infty \Big\}
			\]
			Nach \ref{2.5} ist dann $\my(B_{jk}) = 0$ für $j,k\in \N$. 
			Setze
			\[
				B := \bigcup_{j,k\in \N} B_{jk}
			\]
			Dann ist wieder $\my(B) = 0$.

			Für $\omega \in \Omega \setminus B$ gilt
			\[
				|f_j(\omega) - f_k(\omega)| \le \|f_j-f_k\|_\infty.
			\]
			Da $\C$ vollständig ist, konvergiert $(f_j)$ auf $\Omega \setminus B$ gleichmäßig.

			Definiere
			\[
				f(\omega) := \begin{cases}
					\lim_{j \to \infty} f_j(\omega) & \omega \in \Omega \setminus B \\
					0 & \text{sonst}
				\end{cases}
			\]
			Dann ist
			\[
				f := \lim_{j\to \infty} \chi_{\Omega \setminus B} f_j
			\]
			als punktweiser Grenzwert von messbaren Funktionen wieder messbar.
			Außerdem gilt
			\begin{align*}
				|f_j(\omega) - f(\omega) |
				= \lim_{h\to \infty} |f_j(\omega) - f_k(\omega)| \\
				\le \lim_{h\to \infty} \|f_j - f_k\|_{\infty}
				\le \eps
			\end{align*}
			für ein $j > J_\eps$ und $\omega \in \Omega \setminus B$.

			Da $B$ Nullmenge, ist $\|f_j - f\|_\infty \le \eps$ und daher $f_j - f \in L^\infty$ für $j > J_\eps$.
			Somit ist
			\[
				f = \underbrace{f - f_j}_{\in L^\infty} + \underbrace{f_j}_{\in L^\infty} \in L^\infty
			\]
		\end{seg}
		\begin{seg}[$1 \le p < \infty$]
			Sei $(f_j)$ Cauchy-Folge.
			\begin{enumerate}[a)]
				\item
					Wähle eine „gut konvergente“ Teilfolge $({f_j}_k)$.
					Sei dazu $j_l$ so, dass 
					\[
						\|{f_j}_l - f_k\|_p < \f 1{2^k} \qquad \text{für $k>j_l$}
					\]
					Damit gilt
					\[
						\|{f_j}_k - {f_j}_{k+1} \|_p < \f 1{2^k} \qquad \text{für $k \in \N$}
					\]
				\item
					Zeige jetzt die Existenz der Grenzfunktion.

					Setze
					\begin{align*}
						g_n(\omega) &:= \sum_{k=1}^n |{f_j}_k(\omega) - {f_j}_{k+1}(\omega)|  \qquad \omega \in \Omega \\
						g(\omega) &:= \sum_{k=1}^\infty |{f_j}_k(\omega) - {f_j}_{k+1}(\omega)| 
					\end{align*}
					(evtl. ist $g(\omega) = \infty$)
					Dann gilt
					\begin{align*}
						\|g_n\|_p &= \bigg\|\sum_{k=1}^n |{f_j}_k(\omega) - {f_j}_{k+1}(\omega)| \bigg\| \\
						&\le \sum_{k=1}^n \|{f_j}_k - {f_j}_{k+1} \|_p \\
						&\le \sum_{k=1}^\infty \f 1{2^k}
						= 1
					\end{align*}
					Damit konvergiert $0 \le g_n$ monoton wachsend gegen $g$ und somit auch $0 \le g_n^p$ monoton wachsend gegen $g^p$ (jeweils punktweise).
					Wegen $g_n$ messbar, ist $g_n^p$ messbar und somit wegen der punktweisen Konvergenz auch $g^p$.

					Nach dem Satz über monotone Konvergenz gilt
					\begin{align*}
						\|g\|_p = \int_{\Omega} g^p d\my
						&= \int_{\Omega} \lim_{n\to \infty} g_n^p d\my \\
						&= \lim_{n\to \infty} \int_{\Omega} g_n^p d\my
						&= \lim_{n\to \infty} \|g_n\|_p^p
						\le 1
					\end{align*}
					Also $\|g\|_p \le 1$ und somit
					\[
						g(\omega) < \infty
					\]
					$\my$-fast-überall.
					Sei $\Omega'$ so, dass $g(\omega) < \infty$ für $\omega \in \Omega'$ und $\my(\Omega \setminus \Omega') = 0$.

					Also ist $g_n(\omega) \to g(\omega)$ in $\R$ für $\omega \in \Omega'$.
					Mit 
					\[
						{f_j}_k(\omega) = {f_j}_1(\omega) + \sum_{l=1}^{k-1} \Big({f_j}_{l+1}(\omega) - {f_j}_l(\omega)\Big)
					\]
					und der absoluten Konvergenz
					\[
						\sum_{l=1}^{k-1} \Big|{f_j}_{l+1}(\omega) - {f_j}_l(\omega)\Big| < \infty
					\]
					folgt, dass $({f_j}_k(\omega))$ konvergent ist für $k \to \infty$ und $\omega \in \Omega'$.

					Setze
					\[
						f(\omega) := \begin{cases}
							\lim_{k\to \infty} {f_j}_k (\omega) & \omega \in \Omega' \\
							0 & \omega \in \Omega \setminus \Omega'
						\end{cases}
					\]
				\item
					Zeige jetzt $f \in L^p$ und $\|f_j - f\|_p \to 0$.

					Sei $\eps > 0$ und $K \in \N$ mit
					\[
						\|{f_j}_k - {f_j}_l \|_p < \eps  \qquad \text{für $k,l > K$.}
					\]
					Nach dem Lemma von Fatou ist
					\begin{align*}
						\int_{\Omega} |{f_j}_k - f|^p d\my
						&= \int_{\Omega} \lim_{l\to \infty} |{f_j}_k - {f_j}_l |^p d\my
						&\stack{\text{Fatou}}\le \liminf_{l\to \infty} \underbrace{\int_{\Omega} |{f_j}_k - {f_j}_l |^p d\my}_{= \|{f_j}_k - {f_j}_l\|_p^p < \eps^p} \\
						&\le \eps^p
					\end{align*}
					($f$ ist messbar als punktweiser Grenzwert messbarer Funktionen)
					Also ist ${f_j}_k - f \in L^p$, $\|{f_j}_k - f\|_p \le \eps$ und somit $f = f - {f_j}_k + {f_j}_k \in L^p$.

					Wegen $(f_j)$ Cauchy-Folge, konvergiert die Teilfolge ${f_j}_k \to f$ bezüglich $\|\cdot\|_p$.
					Also
					\[
						\|f_j - f\|_p 
						\le \underbrace{\|f_j - {f_j}_k \|_p}_{< \eps} + \underbrace{\|{f_j}_k}_{< \eps} - f\|_p 
						< 2 \eps
					\]
					für hinreichend großes $j, j_k$ und $k$.
			\end{enumerate}
		\end{seg}
	\end{proof}
\end{st}

\begin{kor}[Weyl] \label{2.14}
	Sei $1 \le p \le \infty$ und $(f_j)$ Cauchy-Folge in $L^p$.
	
	Dann existiert $f \in L^p$ mit $\|f - f_j\|_p \to 0$ und eine Teilfolge $({f_j}_k)$ mit ${f_j}_k(\omega) \to f(\omega)$ $\my$-fast-überall in $\Omega$.
\end{kor}

\begin{st} \label{2.15}
	Sei $1 \le p < \infty$.
	\begin{enumerate}[1)]
		\item
			Für eine einfache Funktion $s$ gilt
			\[
				s \in L^p
				\quad \iff \quad
				\my\Big( \big\{ \omega \in \Omega : s(\omega) \neq 0 \big\} \Big) < \infty
			\]
		\item
			Die Menge
			\[
				\Big\{ s \in L^p : s \text{ ist einfach} \Big\}
			\]
			ist dicht in $L^p$.
	\end{enumerate}
	\begin{proof}
		\begin{enumerate}[1)]
			\item
				Dies ist eine einfache Übungsaufgabe.
			\item
				\begin{enumerate}[a)]
					\item
						Betrachte den Fall $f \ge 0$, $f \in L^p$.

						Es existiert eine Folge $(s_j)$ einfacher Funktionen mit (punktweise):
						\[
							0 \le s_1 \le s_2 \le \dotsb \le f
							\qquad \text{und} \qquad
							f = \lim_{j\to \infty} s_j
						\]
						(siehe Maßtheorie)
						Damit ist
						\[
							\int_{\Omega} |s_j|^p d\my
							\le \int_{\Omega} |f|^p d\my
							< \infty
						\]
						Also ist $s_j, f-s_j \in L^p$.
						Weiter gilt punktweise
						\[
							|\underbrace{f - s_j}_{\ge 0}|^p \le |f|^p.
						\]
						Mit dem Satz über majorisierte Konvergenz gilt
						\begin{align*}
							\lim_{j\to \infty} \|f - s_j\|^p
							&= \lim_{j\to \infty} \int_{\Omega} \underbrace{|f-s_j|^p}_{|f|^p} d\my  \qquad \int_{\Omega} |f|^p d\my < \infty\\
							&= \int_{\Omega} \lim_{j\to \infty} |f- s_j |^p d\my \\
							&= \int_{\Omega} 0 d\my
							= 0
						\end{align*}
						Also $\|f-s_j\|_p \to 0$.
					\item
						Für beliebige Funktionen $f: \Omega \to \C$ zerlege diese in vier positive Funktionen.
				\end{enumerate}
		\end{enumerate}
	\end{proof}
\end{st}

\begin{st} \label{2.16}
	\begin{enumerate}[1)]
		\item
			Sei $\my(\Omega) < \infty$, $1 \le p \le p' \le \infty$ und $f \in L^{p'}(\Omega)$.

			Dann ist $f \in L^p(\Omega)$ und 
			\[
				\|f\|_p \le \my(\Omega)^{\f 1p - \f 1{p'}} \|f\|_{p'}
			\]
			\begin{note}
				
			\end{note}
		\item
			Sei $1 \le p < p' \le \infty$, dann ist
			\[
				L^p(\R) \setminus L^{p'}(\R) \neq \emptyset \neq L^{p'}(\R) \setminus L^p(\R)
			\]
		\item
			Sei $\Omega = \N$, $\Sigma := P(\Omega)$ und $\my(M) := \#M$ für $M \subset \Omega$.
			Für $1 \le p \le \infty$ ist dann
			\begin{align*}
				L^p(\Omega, \Sigma, \my) &= \ell^p \\
				\ell^p &:= \bigg\{ (x_j) \text{ Folge in $\C$: } \sum_{j=1}^\infty |x_j|^p < \infty \bigg\} \\
				\|(x_j)\|_p &= \bigg( \sum_{j=1}^\infty |x_j|^p \bigg)^{\f 1p}
			\end{align*}
			Für $p = \infty$ ist
			\begin{align*}
				\ell^\infty &= \Big\{ \text{beschränkte Folgen} \Big\} \\
				\|(x_j)\|_\infty &:= \sup_{j\in \N} |x_j|
			\end{align*}
		\item
			Für den Spezialfall $p=2$ ist $L^2(\Omega, \Sigma, \my)$ vollständig und
			\[
				\|f\|_2 = \bigg( \int_{\Omega} |f|^2 d\my \bigg)^{\f 12}
			\]
			wird erzeugt vom Skalarprodukt $\<f,g\> := \int_{\Omega} f \_g d\my$ (Wegen Hölder: $\int_{\Omega} |f \_g|d\my \le \|f\|_2 \|g\|_2 < \infty$, konvergiert das Integral)

			Also ist $L^2$ ein Hilbertraum.
	\end{enumerate}
\end{st}
