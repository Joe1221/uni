% This work is licensed under the Creative Commons
% Attribution-NonCommercial-ShareAlike 3.0 Unported License. To view a copy of
% this license, visit http://creativecommons.org/licenses/by-nc-sa/3.0/ or send
% a letter to Creative Commons, 444 Castro Street, Suite 900, Mountain View,
% California, 94041, USA.

\chapter{\texorpdfstring{$L^p$}{L\textasciicircum p}-Räume}


\begin{df} \label{2.1}
	\begin{enumerate}[1)]
		\item
			Sei $\Omega$ eine Menge und $\scr P(\Omega)$ ihre Potenzmenge.
			Eine Menge $\Sigma \subset \scr P(\Omega)$ heißt \emph{$\sigma$-Algebra (über $\Omega$)}, wenn sie die folgenden Eigenschaften erfüllt:
			\begin{enumerate}[(i)]
				\item
					$\displaystyle \emptyset, \Omega \in \Sigma$
				\item
					$\displaystyle A \in \Sigma \implies \Omega \setminus A \in \Sigma$
				\item
					$\displaystyle A_1, A_2, \dotsc \in \Sigma \implies \bigcup_{j\in \N} A_j \in \Sigma$
			\end{enumerate}
		\item
			Sei $\Sigma$ eine $\sigma$-Algebra über einer Menge $\Omega$.
			Eine Funktion $\my : \Sigma \to \R \cup \{-\infty, \infty\}$ heißt \emph{Maß (auf $\Omega$)}, wenn sie die folgenden Eigenschaften erfüllt:
			\begin{enumerate}[(i)]
				\item
					$\displaystyle \forall A \in \Sigma : \my(A) \ge 0$
				\item
					$\displaystyle \my(\emptyset) = 0$
				\item
					Für paarweise disjunkte $A_i \in \Sigma$ ($i \in I \subset \N$) gilt
					\[
						\my\Big(\bigcup_{i\in I} A_i\Big) = \sum_{i \in I} \my(A_i)
					\]
			\end{enumerate}
		\item
			Sei $\Sigma$ eine $\sigma$-Algebra über einer Menge $\Omega$ und $\my$ ein Maß auf $\Omega$.

			Das Tupel $(\Omega, \Sigma)$ nennt man \emph{Messraum} oder \emph{messbarer Raum} und das Tupel $(\Omega, \Sigma, \my)$ nennt man \emph{Maßraum}.
		\item
			Seien $(\Omega, \Sigma)$ und $(\Omega', \Sigma')$ Messräume.
			Eine Funktion $f: \Omega \to \Omega'$ heißt messbar, falls
			\[
				\forall A' \in \Sigma' : f^{-1}(A') \in \Sigma
			\]
			(ideal ist $\Sigma'$ „klein“, $\Sigma$ „groß“, also „viele“ $f$ messbar).
	\end{enumerate}
\end{df}

\begin{conv*}
	Im Folgenden sei $(\Omega, \Sigma, \my)$ stets ein Maßraum.
\end{conv*}

\begin{ex}[Anwendung]
	Sei $\Omega = \R^n$ und $\Sigma$ die Borel $\sigma$-Algebra, d.h. die von den offenen Intervallen $\bigtimes_{j=1}^n ]a_j,b_j[$ erzeugte (kleinste) $\sigma$-Algebra.
	Das Maß
	\[
		\my \bigg( \bigtimes_{j=1}^n ]a_j,b_j[ \bigg) := \prod_{j=1}^n (b_j - a_j)
		\qquad \text{für $b_j > a_j$}
	\]
	fortgesetzt auf $\Sigma$ heißt \emph{Lebesgue-Borel-Maß}.

	Definiere die Vervollständigung:
	\[
		\Sigma^* := \bigg\{ A \subset \Omega \;\Big|\; \exists B,C \in \Sigma : B \subset A \subset C \land \my(C \setminus B) = 0 \bigg\}
	\]
	Dann heißt
	\[
		\my^*(A) := \my(B) = \my(C)
	\]
	\emph{Lebesgue-Maß} auf $\R^n$.

	Man setzt für $A \in \Sigma$:
	\begin{alignat*}{2}
		\Sigma_A &:= \{ B \cap A : B \in \Sigma \}, &\qquad
		\Sigma_A^* &:= \{ B \cap A : B \in \Sigma^* \}, \\
		\my_A &:= \my \Big|_{\Sigma_A}, &\qquad
		\my_A^* &:= \my^* \Big|_{\Sigma_A^*}.
	\end{alignat*}
\end{ex}

\begin{df*}
	Eine Funktion $\Ind_M: \Omega \to \{0,1\}$ ($M \subset \Omega$), mit
	\[
		\Ind_M (x) = \begin{cases}
			1 & x\in M \\
			0 & x \in \Omega \setminus M
		\end{cases}
	\]
	nennt man \emph{Indikatorfunktion} oder \emph{charakteristische Funktion}.

	$s:\Omega \to \R$ heißt genau dann \emph{einfache} Funktion, wenn
	\[
		\exists N \in \N, \alpha_1, \dotsc, \alpha_N \in \R, M_1, \dotsc, M_N \in \Sigma : s = \sum_{j=1}^N \alpha_j \Ind_{M_j}.
	\]
	In diesem Fall setzt man
	\[
		\int_{\Omega} s \dx[\my] := \sum_{j=1}^N \alpha_j \my(M_j)
	\]
	für $\alpha_j \ge 0$.
\end{df*}

\begin{df}[Lebesgue-Integral]
	\begin{enumerate}[1)]
		\item
			Sei $f: \Omega \to \R$ Borel-messbar (d.h. im Bildraum wird die Borel-$\sigma$-Algebra verwendet).

			\begin{enumerate}[a)]
				\item
					Definiere für $f \ge 0$:
					\[
						\int_{\Omega} f \dx[\my]
						:= \sup \bigg\{ \int_{\Omega} s \dx[\my] : 0 \le s(x) \le f(x), x \in \Omega \land s : \Omega \to \R \text{ einfach} \bigg\}.
					\]
				\item
					Für $f: \Omega \to \R$ ohne Vorzeichenbedingung definiert man
					\begin{align*}
						f_+ (x) &:= \max\{f(x), 0 \} \qquad x \in \Omega \\
						f_- (x) &:= -\min\{ f(x), 0 \} \qquad x \in \Omega
					\end{align*}
					Dann ist $f_+, f_- \ge 0$, $f = f_+ - f_-$ und $f_+, f_-$ messbar.

					Definiere
					\[
						\int_{\Omega} f \dx[\my] := \int_{\Omega} f_+ \dx[\my] - \int_{\Omega} f_- \dx[\my]
					\]
					solange nicht beide Integrale auf der rechten Seite $\infty$ sind.

					Wir nennen $f$ integrierbar, wenn $f$ messbar ist und obiges Integral endlich ist.
			\end{enumerate}
		\item
			$f : \Omega \to \C$ heißt \emph{messbar}, falls $\Re f$ und $\Im f$ Borel-messbar sind.

			$f$ heißt \emph{integrierbar}, falls $\Re f$ und $\Im f$ integrierbar sind.
			Dann setzt man
			\[
				\int_{\Omega} f \dx[\my] := \int_{\Omega} (\Re f) \dx[\my] + i \int_{\Omega} (\Im f) \dx[\my] .
			\]
	\end{enumerate}

\end{df}


\coursetimestamp{22}{4}{2013}
\begin{df} \label{2.4}
	Sei $f: \Omega \to \C$ messbar.
	\begin{enumerate}[1)]
		\item
			Für $1 \le p < \infty$ setze
			\[
				N_p(f) := \bigg( \int_{\Omega} |f|^p \dx[\my] \bigg)^{\f 1p}
				\qquad \Big( |f|^n = (\argdot)^n \circ |f| \Big).
			\]
		\item
			Wir nennen
			\begin{align*}
					\displaystyle N_\infty(f)
					&:= \inf \Big\{ c \in [0,\infty] : |f|\le c \text{ $\my$-fast-überall} \Big\} \\
					&= \esssup_{\omega \in \Omega} |f(\omega)|
			\end{align*}
			\emph{wesentliches Supremum}.
	\end{enumerate}
\end{df}

\begin{st}[Eigenschaften] \label{2.5}
	\begin{enumerate}[1)]
		\item
			Für $1 \le p \le \infty$ gilt $0 \le N_p(f) \le \infty$ und $N_p(\alpha f) = |\alpha| N_p(f)$ für $\alpha \in \C$.
		\item
			Es gilt $|f| \le N_\infty(f)$ $\my$-fast-überall, d.h.
			\[
				\my \Big\{ \omega \in \Omega : |f(\omega)| > N_\infty(f) \Big\} = 0.
			\]
			Insbesondere ist das Infimum in \ref{2.4} 2) ein Minimum.
		\item
			Es gilt
			\[
				N_\infty(f+g) \le N_\infty (f) + N_\infty(g).
			\]
	\end{enumerate}
	\begin{proof}
		\begin{enumerate}[1)]
			\item Leicht zu zeigen.
			\itemdm
				\begin{align*}
					\my \Big\{ \omega \in \Omega : |f(\omega)| > N_\infty(f) \Big\}
					&= \my \bigg( \bigcup_{n\in \N} \Big\{ \omega \in \Omega : |f(\omega)| > N_\infty(f) + \f 1n \Big\} \bigg) \\
					&= \lim_{n\to \infty }\underbrace{\my \bigg( \Big\{ \omega \in \Omega : |f(\omega)| > N_\infty(f) + \f 1n \Big\} \bigg)}_{=0} \\
					&=  0.
				\end{align*}
				Also ist $|f| \le N_\infty(f)$ $\my$-fast-überall und damit das Infimum von $\{c \in [0,\infty] : |f| \le c \text{ $\my$-fast-überall}\}$ ein Minimum.
			\itemdm
				\begin{align*}
					|(f+g)(\omega)|
					& \le |f(\omega)| + |g(\omega)| \\
					& \le N_\infty(f) + N_\infty(g) \qquad \text{$\my$-fast-überall.}
				\end{align*}
		\end{enumerate}
	\end{proof}
\end{st}

\begin{df} \label{2.6}
	$p,q$ mit $1 \le p,q \le \infty$ heißen \emph{konjugiert}, falls
	\[
		\f 1p + \f 1q = 1
	\]
	(mit der Konvention $\f 1\infty = 0$).
	\begin{note}
		Ein wichtiger Spezialfall ist $p=q=2$.
	\end{note}
\end{df}


\begin{st} \label{2.7}
	Seien $f,g : \Omega \to \C$ messbar.
	\begin{enumerate}[1)]
		\item
			Für $1 < p,q < \infty$ und $p,q$ konjugiert gilt die \emph{Höldersche Ungleichung}:
			\[
				\int_{\Omega} |fg| \dx[\my] \le \bigg(\int_{\Omega}|f|^p \dx[\my] \bigg)^{\f 1p} \bigg( \int_{\Omega} |g|^q \dx[\my] \bigg)^{\f 1q}
			\]
			oder in anderer Schreibweise:
			\[
				N_1(fg) \le N_p(f) \cdot N_q(g).
			\]
		\item
			Für $1 \le p < \infty$ gilt die \emph{Minkowskische Ungleichung}:
			\[
				\bigg( \int_{\Omega} |f+g|^p \dx[\my] \bigg)^{\f 1p} \le  \bigg(\int_{\Omega} |f|^p \dx[\my]\bigg)^{\f 1p} + \bigg( \int_{\Omega} |g|^p \dx[\my] \bigg)^{\f 1p}
			\]
			oder in anderer Schreibweise:
			\[
				N_p(f+g) \le N_p(f) + N_p(g).
			\]
	\end{enumerate}
	\begin{proof}
		\begin{enumerate}[1)]
			\item
				\begin{enumerate}[a)]
					\item
						Für $N_p(f) = 0$ ist $f =0 $ $\my$-fast-überall, also $fg = 0$ $\my$-fast-überall und damit $N_1(fg) = 0$.
						Genauso $N_q(g) = 0 \implies N_1(fg) = 0$.

						Für $N_p(f) > 0 \land N_q(g) = \infty$ und $N_q(g) > 0 \land N_p(f) = \infty$ ist die Aussage klar.
					\item
						Für $N_p(f) = N_q(g) = 1$, zeige zunächst $N_1(fg) \le 1$.

						Wegen $t \mapsto e^t$ konvex\fixme[Zeichnung], gilt für $0 \le \lambda \le 1$ und $s,t \in \R$
						\[
							e^{\lambda t + (1-\lambda)s} \le \lambda e^t + (1-\lambda)e^s.
						\]
						Für $0 < x,y < \infty$ sei
						\[
							x = e^{\f \alpha p},\qquad y = e^{\f \beta q} = e^{\beta (1 - \f 1p)}.
						\]
						Damit gilt
						\begin{align*}
							xy = e^{\f 1p \alpha + (1-\f 1p)\beta}
							&= e^{\lambda \alpha + (1-\lambda) \beta} \qquad (\lambda := \f 1p, \quad 1-\lambda = \f 1q) \\
							&\le \lambda e^{\alpha}  + (1-\lambda) e^{\beta} \\
							&= \f 1p x^p + \f 1q y^q.
						\end{align*}
						Also gilt für beliebige $0 \le x,y < \infty$:
						\[
							xy \le \f 1p x^p + \f 1q y^q
						\]
						(auch als \emph{Young'sche Ungleichung} bekannt)
						und damit
						\begin{align*}
							N_1(fg)
							= \int_{\Omega} |fg| \dx[\my]
							&\le \int_{\Omega} \f 1p |f|^p + \f 1q |g|^q \dx[\my] \\
							&= \f 1p \underbrace{N_p(f)^p}_{=1} + \f 1q \underbrace{N_q(g)^q}_{=1}
							= \f 1p + \f 1q = 1.
						\end{align*}
					\item
						Seien $0 < N_p(f), N_q(g) < \infty$.
						Dann ist
						\begin{align*}
							\int_{\Omega}|fg| \dx[\my]
							&= N_p(f)N_q(g) \int_{\Omega} \underbrace{\Big| \f f{N_p(f)} \Big|}_{N_p(\argdot)=1} \underbrace{\Big| \f g{N_q(g)} \Big|}_{N_q(\argdot)=1} \dx[\my] \\
							&\le N_p(f) N_q(g) \cdot 1
						\end{align*}
				\end{enumerate}
			\item
				Die trivialen Fälle sind $N_p(f+g) = 0$, $N_p(f) = \infty$ und $N_p(g) = \infty$.
				Der Fall $p=1$ folgt direkt aus $|f+g| \le |f| + |g|$ (punktweise).

				Betrachte also nun $N_p(f), N_p(g) < \infty$ und $N_p(f+g) > 0$ und $p>1$.

				Punktweise gilt:
					\begin{align*}
						|f+g|^p
						\le \Big( |f| + |g| \Big)^p
						&\le \Big( 2 \max \{|f|, |g|\} \Big)^p \\
						&= 2^p \max \{|f|^p, |g|^p \}
						\le 2^p \Big( |f|^p + |g|^p \Big)
					\end{align*}
				und damit
					\[
						N_p(f+g) \le 2^p \Big( N_p(f) + N_p(g) \Big) < \infty.
					\]
				Daraus folgt
					\begin{align*}
						\Big( N_p(f+g) \Big)^p
						&= \int_{\Omega} |f+g|^p \dx[\my] \\
						&= \int_{\Omega} |f+g| |f+g|^{p-1} \dx[\my] \\
						&\le \int_{\Omega} |f| |f+g|^{p-1} \dx[\my] + \int_{\Omega} |g| |f+g|^{p-1} \dx[\my].
					\intertext{Wähle $q \in ]1,\infty[$ konjugiert zu $p$, d.h. $\f 1q + \f 1p = 1$ (also $q(p-1) = p$) und wende die Höldersche Ungleichung an:}
						&\le \underbrace{\bigg( \int_{\Omega} |f|^p \dx[\my] \bigg)^{\f 1p}}_{N_p(f)} \underbrace{\bigg( \int_{\Omega} |f+g|^{q(p-1)} \dx[\my] \bigg)^{\f 1q}}_{N_p(f+g)^{\f pq}} \\
							&\quad + \underbrace{\bigg( \int_{\Omega} |g|^p \dx[\my] \bigg)^{\f 1p}}_{N_p(g)} \underbrace{\bigg( \int_{\Omega} |f+g|^{q(p-1)} \dx[\my] \bigg)^{\f 1q} }_{N_p(f+g)^{\f pq}} \\
						&= N_p(f+g)^{\f pq} \Big(N_p(f) + N_p(g)\Big).
					\end{align*}
					Da $0 < N_p(f+g)^{\f pq} < \infty$, gilt also
					\[
						N_p(f+g) = \big(N_p(f+g)\big)^{p- \f pq} \le N_p(f) + N_p(g).
					\]
		\end{enumerate}
	\end{proof}
\end{st}

\begin{kor} \label{2.8}
	\begin{enumerate}[1)]
		\item
			Für $1 \le p,q \le \infty$ und $p,q$ konjugiert gilt
			\[
				N_1(fg) \le N_p(f) N_q(g).
			\]
			\begin{note}
				Wichtiger Spezialfall:
				\begin{align*}
					\bigg| \int_{\Omega} f \_g \dx[\my] \bigg|
					&\le \int_{\Omega} |f \_g| \dx[\my]
					\le N_2(f) N_2(g)
					= \bigg( \int_{\Omega} |f|^2 \dx[\my] \bigg)^{\f 12} \bigg( \int_{\Omega} |g|^2 \dx[\my] \bigg)^{\f 12}.
				\end{align*}
				Das entspricht der CSB im $L^2$.
			\end{note}
		\item
			Für $1 \le p \le \infty$ gilt die Dreiecksungleichung
			\[
				N_p(f+g) \le N_p(f) + N_p(g).
			\]
	\end{enumerate}
	\begin{proof}
		\begin{enumerate}[1)]
			\item
				Für $1 < p,q < \infty$ siehe \ref{2.7} 1).

				Sei also $p=1, q = \infty$.
				Dann folgt aus \ref{2.5} $|g| \le N_\infty(g)$ $\my$-fast-überall und damit
				\[
					N_1(fg)
					= \int_{\Omega} |fg| \dx[\my]
					\le N_\infty(g) \int_{\Omega}|f| \dx[\my]
					= N_p(f) N_q(g).
				\]
			\item
				Gilt mit \ref{2.5} 3) und \ref{2.7} 2).
		\end{enumerate}
	\end{proof}
\end{kor}

\begin{df*}
	$m : L \to \R$ heißt \emph{Halbnorm}, falls
	\begin{enumerate}[a)]
		\item
			$\displaystyle m(\alpha x) = |\alpha| m(x)$ gilt,
		\item
			$m$ positiv ist,
		\item
			die Dreiecksungleichung erfüllt ist
			\[
				m(x+y) \le m(x) + m(y).
			\]
	\end{enumerate}
	\begin{note}
		Im Vergleich zur \emph{Norm} wird also auf die positive Definitheit verzichtet.
	\end{note}
\end{df*}

\begin{df} \label{2.9}
	Sei $1 \le p \le \infty$ und
	\[
		\tilde {L^p} (\Omega, \Sigma, \my)
		:= \Big\{ f : \Omega \to \C \text{ messbar} : N_p(f) < \infty \Big\}.
	\]
	Dann ist $\tilde{L^p}(\Omega, \Sigma, \my)$ ein linearer Raum (Vektorraum) und $N_p$ ist Halbnorm auf $\tilde{L^p}(\Omega, \Sigma, \my)$.
\end{df}

\begin{df} \label{2.10}
	Sei $1 \le p \le \infty$ und
	\begin{align*}
		N &:= \Big\{ f \in \tilde {L^p}(\Omega, \Sigma, \my) : N_p(f) = 0 \Big\} \\
		&= \Big\{ f : \Omega \to \C \text{ messbar} : f = 0 \text{ fast überall} \Big\}.
	\end{align*}
	Durch
	\[
		f \sim g :\iff f-g \in N
	\]
	wird eine Äquivalenzrelation auf $\tilde{L^p}(\Omega, \Sigma, \my)$ definiert.

	Definiere mit Hilfe der Äquivalenzklassen nun
	\begin{align*}
		L^p(\Omega, \Sigma, \my)
			&:= \Big\{ [f] : f \in \tilde{L^p}(\Omega, \Sigma, \my) \Big\}
			= \tilde {L^p} (\Omega, \Sigma, \my) / N, \\
		\| [f] \|_p &:= N_p(f) \qquad \text{ für } [f] \in L^p(\Omega, \Sigma, \my).
	\end{align*}
	Dann ist $\|\argdot\|_p$ eine Norm auf $L^p(\Omega, \Sigma, \my)$, was leicht zu zeigen ist.

	Im Folgenden schreiben wir $L^p$ oder $L^p(\Omega)$ statt $L^p(\Omega, \Sigma, \my)$ und $f$ statt $[f]$.
\end{df}

\begin{nt} \label{2.11}
	Sei $\Omega \subset \R^n$ und $f,g \in C(\Omega \to \C) \cap L^p(\Omega)$.
	Dann gilt
	\begin{align*}
		\|f - g \|_p = 0
		\quad\implies\quad
		f = g \text{ auf $\Omega$}
	\end{align*}
	d.h. falls ein Vertreter $f$ der Äquivalenzklasse $[f]$ stetig ist, sind alle anderen Vertreter unstetig.
	Oder mit anderen Worten: jede Äquivalenzklasse $[f]$ enthält höchstens einen stetigen Vertreter.
\end{nt}

\coursetimestamp{24}{4}{2013}
\begin{st}[Fischer-Riesz] \label{2.12}
	Für $1 \le p \le \infty$ ist $L^p(\Omega, \Sigma, \my)$ ein Banachraum.
	\begin{proof}
		Durch $\|\cdot\|_p$ wird eine Norm beschrieben.
		Wir müssen also nur noch die Vollständigkeit nachweisen.
		\begin{seg}[$p=\infty$]
			Sei $(f_j)$ Cauchy-Folge.
			Definiere
			\[
				B_{jk} := \Big\{ \omega \in \Omega : |f_j(\omega) - f_k(\omega)| > \|f_j-f_k\|_\infty \Big\}.
			\]
			Nach \ref{2.5} ist dann $\my(B_{jk}) = 0$ für $j,k\in \N$.
			Setze
			\[
				B := \bigcup_{j,k\in \N} B_{jk}.
			\]
			Dann ist wieder $\my(B) = 0$.

			Für $\omega \in \Omega \setminus B$ gilt
			\[
				|f_j(\omega) - f_k(\omega)| \le \|f_j-f_k\|_\infty < \eps.
			\]
			Da $\C$ vollständig ist, konvergiert $(f_j)$ auf $\Omega \setminus B$ gleichmäßig.

			Definiere
			\[
				f(\omega) := \begin{cases}
					\lim\limits_{j \to \infty} f_j(\omega) & \text{für }\omega \in \Omega \setminus B ,\\
					0 & \text{sonst.}
				\end{cases}
			\]
			Damit ist
			\[
				f := \lim_{j\to \infty} \Ind_{\Omega \setminus B} f_j
			\]
			als punktweiser Grenzwert von messbaren Funktionen wieder messbar.
			Außerdem gilt
			\begin{align*}
				|f_j(\omega) - f(\omega) |
				= \lim_{k\to \infty} |f_j(\omega) - f_k(\omega)| \\
				\le \lim_{k\to \infty} \|f_j - f_k\|_{\infty}
				\le \eps
			\end{align*}
			für ein $j > J_\eps$ und $\omega \in \Omega \setminus B$.

			Da $B$ Nullmenge, ist $\|f_j - f\|_\infty \le \eps$ und daher $f_j - f \in L^\infty$ für $j > J_\eps$.
			Somit ist auch
			\[
				f = \underbrace{f - f_j}_{\in L^\infty} + \underbrace{f_j}_{\in L^\infty} \in L^\infty.
			\]
		\end{seg}
		\begin{seg}[$1 \le p < \infty$]
			Sei $(f_j)$ Cauchy-Folge.
			\begin{enumerate}[a)]
				\item
					Wähle eine „gut konvergente“ Teilfolge $({f_j}_k)$.
					Sei dazu $j_k > j_{k-1}$ mit
					\[
						\|f_{j_k} - f_l\|_p < \f 1{2^k} \qquad \text{für $l>j_k$}.
					\]
					Damit gilt insbesondere
					\[
						\|f_{j_k} - f_{j_{k+1}} \|_p < \f 1{2^k} \qquad \text{für alle $k \in \N$}.
					\]
				\item
					Zeige jetzt die Existenz der Grenzfunktion.

					Setze
					\begin{align*}
						g_n(\omega) &:= \sum_{k=1}^n |f_{j_k}(\omega) - f_{j_{k+1}}(\omega)|, \qquad \omega \in \Omega; \\
						g(\omega) &:= \sum_{k=1}^\infty |f_{j_k}(\omega) - f_{j_{k+1}}(\omega)|
					\end{align*}
					(evtl. ist $g(\omega) = \infty$).
					Damit konvergiert $0 \le g_n$ monoton wachsend gegen $g$ und somit auch $0 \le g_n^p$ monoton wachsend gegen $g^p$ (jeweils punktweise).
					Wegen $g_n$ messbar, ist $g_n^p$ messbar und somit wegen der punktweisen Konvergenz auch $g^p$.

					Es gilt
					\begin{align*}
						\|g_n\|_p &= \bigg\|\sum_{k=1}^n |f_{j_k}(\omega) - f_{j_{k+1}}(\omega)| \bigg\|_p \\
						&\le \sum_{k=1}^n \|f_{j_k} - f_{j_{k+1}} \|_p \\
						&\le \sum_{k=1}^\infty \f 1{2^k}
						= 1.
					\end{align*}
					Mit dem Satz über monotone Konvergenz gilt damit
					\begin{align*}
						\|g\|_p = \int_{\Omega} g^p \dx[\my]
						&= \int_{\Omega} \lim_{n\to \infty} g_n^p \dx[\my] \\
						&= \lim_{n\to \infty} \int_{\Omega} g_n^p \dx[\my]
						= \lim_{n\to \infty} \|g_n\|_p^p
						\le 1.
					\end{align*}
					Also $\|g\|_p \le 1$ und somit
					\[
						g(\omega) < \infty
					\]
					für $\my$-fast-alle $\omega\in \Omega$.
					Sei $\Omega'\subset \Omega$ so, dass $g(\omega) < \infty$ für $\omega \in \Omega'$ und $\my(\Omega \setminus \Omega') = 0$.

					Also konvergiert $g_n(\omega) \to g(\omega)$ in $\R$ für $\omega \in \Omega'$.
					Mit
					\[
						{f_j}_k(\omega) = {f_j}_1(\omega) + \sum_{l=1}^{k-1} \Big({f_j}_{l+1}(\omega) - {f_j}_l(\omega)\Big)
					\]
					und der absoluten Konvergenz
					\[
						\sum_{l=1}^{\infty} \underbrace{\Big|{f_j}_{l+1}(\omega) - {f_j}_l(\omega)\Big|}_{<\f 1{2^l}} < \infty
					\]
					folgt, dass $({f_j}_k(\omega))$ konvergent ist für $k \to \infty$ und $\omega \in \Omega'$.

					Setze
					\[
						f(\omega) := \begin{cases}
							\lim_{k\to \infty} {f_j}_k (\omega) & \omega \in \Omega'; \\
							0 & \omega \in \Omega \setminus \Omega'.
						\end{cases}
					\]
				\item
					Zeige jetzt $f \in L^p$ und $\|f_j - f\|_p \to 0$.

					Sei $\eps > 0$ und $K \in \N$ mit
					\[
						\|{f_j}_k - {f_j}_l \|_p < \eps  \qquad \text{für $k,l > K$.}
					\]
					Nach dem Lemma von Fatou ist
					\begin{align*}
						\int_{\Omega} |{f_j}_k - f|^p \dx[\my]
						&= \int_{\Omega} \lim_{l\to \infty} |{f_j}_k - {f_j}_l |^p \dx[\my] \\
						&\stack{\text{Fatou}}\le \liminf_{l\to \infty} \underbrace{\int_{\Omega} |{f_j}_k - {f_j}_l |^p \dx[\my]}_{= \|{f_j}_k - {f_j}_l\|_p^p < \eps^p} \\
						&\le \eps^p
					\end{align*}
					Also ist ${f_j}_k - f \in L^p$, $\|{f_j}_k - f\|_p \le \eps$ und somit $f = f - {f_j}_k + {f_j}_k \in L^p$.

					Da $(f_j)$ Cauchy-Folge, konvergiert $f_j \to f$ bezüglich $\|\argdot\|_p$, denn
					\[
						\|f_j - f\|_p
						\le \underbrace{\|f_j - {f_j}_k \|_p}_{< \eps} + \underbrace{\|{f_j}_k - f\|_p}_{< \eps}
						< 2 \eps
					\]
					für hinreichend großes $j$ und $k$.
			\end{enumerate}
		\end{seg}
	\end{proof}
\end{st}
\setcounter{thm}{13}
\begin{kor}[Weyl] \label{2.14}
	Sei $1 \le p \le \infty$ und $(f_j)$ Cauchy-Folge in $L^p$.

	Dann existiert $f \in L^p$ mit $\|f - f_j\|_p \to 0$ und eine Teilfolge $({f_j}_k)$ mit ${f_j}_k(\omega) \to f(\omega)$ $\my$-fast-überall in $\Omega$.
\end{kor}

\begin{st} \label{2.15}
	Sei $1 \le p < \infty$.
	\begin{enumerate}[1)]
		\item
			Für eine einfache Funktion $s$ gilt
			\[
				s \in L^p
				\quad \iff \quad
				\my\Big( \big\{ \omega \in \Omega : s(\omega) \neq 0 \big\} \Big) < \infty.
			\]
		\item
			Die Menge
			\[
				\Big\{ s \in L^p : s \text{ ist einfach} \Big\}
			\]
			ist dicht in $L^p$.
	\end{enumerate}
	\begin{proof}
		\begin{enumerate}[1)]
			\item
				Da $s$ einfach, existieren $c, C \in \R$ mit $0 < c \le C$ sodass $c \le |u(x)| \le C$ für $x \in \supp s$.
				Dann gilt
				\[
					c^p \my(\supp s) \le \int |s|^p \dx[\my] \le C^p \my(\supp s)
				\]
				und somit $\my(\supp s) < \infty \iff \int |s|^p \dx[\my] < \infty$.
			\item
				\begin{enumerate}[a)]
					\item
						Betrachte den Fall $f \ge 0$, $f \in L^p$.

						Es existiert eine Folge $(s_j)$ einfacher Funktionen mit (punktweise):
						\[
							0 \le s_1 \le s_2 \le \dotsb \le f
							\qquad \text{und} \qquad
							f = \lim_{j\to \infty} s_j
						\]
						(siehe Maßtheorie).
						Damit ist
						\[
							\int_{\Omega} |s_j|^p \dx[\my]
							\le \int_{\Omega} |f|^p \dx[\my]
							< \infty.
						\]
						Also ist $s_j$ und $f-s_j \in L^p$.
						Weiter gilt punktweise
						\[
							|\underbrace{f - s_j}_{\ge 0}|^p \le |f|^p.
						\]
						Mit dem Satz über majorisierte Konvergenz gilt
						\begin{align*}
							\lim_{j\to \infty} \|f - s_j\|_p^p
							&= \lim_{j\to \infty} \int_{\Omega} \underbrace{|f-s_j|^p}_{\le |f|^p} \dx[\my],  \qquad \Big(\int_{\Omega} |f|^p \dx[\my] < \infty\Big)\\
							&= \int_{\Omega} \lim_{j\to \infty} |f- s_j |^p \dx[\my] \\
							&= \int_{\Omega} 0 \dx[\my]
							= 0,
						\end{align*}
						also $\|f-s_j\|_p \to 0$.
					\item
						Für eine beliebige Funktion $f: \Omega \to \C$ zerlege diese in vier positive Funktionen und folgere die Aussage mit a).
				\end{enumerate}
		\end{enumerate}
	\end{proof}
\end{st}

\begin{st} \label{2.16}
	\begin{enumerate}[1)]
		\item
			Sei $\my(\Omega) < \infty$, $1 \le p \le p' \le \infty$ und $f \in L^{p'}(\Omega)$.

			Dann ist $f \in L^p(\Omega)$ und es gilt
			\[
				\|f\|_p \le \my(\Omega)^{\f 1p - \f 1{p'}} \|f\|_{p'}.
			\]
		\item
			Sei $1 \le p < p' \le \infty$, dann ist
			\[
				L^p(\R) \setminus L^{p'}(\R) \neq \emptyset \neq L^{p'}(\R) \setminus L^p(\R).
			\]
		\item
			Sei $\Omega = \N$, $\Sigma := P(\Omega)$ und $\my(M) := |M|$ (für $M \subset \Omega$) das Zählmaß.
			Für $1 \le p < \infty$ erhält man
			\begin{align*}
				L^p(\Omega, \Sigma, \my) = \ell^p &= \bigg\{ (x_j) \text{ Folge in $\C$: } \sum_{j=1}^\infty |x_j|^p < \infty \bigg\} ,\\
				\|(x_j)\|_p &= \bigg( \sum_{j=1}^\infty |x_j|^p \bigg)^{\f 1p}
			\intertext{und für $p = \infty$}
				L^p(\Omega, \Sigma, \my) = \ell^\infty &= \Big\{ \text{beschränkte Folgen in $\C$} \Big\}, \\
				\|(x_j)\|_\infty &= \sup_{j\in \N} |x_j|.
			\end{align*}
		\item
			$L^2(\Omega, \Sigma, \my)$ ist ein Hilbertraum.
	\end{enumerate}
	\begin{proof}
		\begin{enumerate}[1)]
			\item
				Siehe \coursehref{blatt04.pdf}{Übungsaufgabe 4.2a}.
			\item
				Siehe \coursehref{blatt04.pdf}{Übungsaufgabe 4.3}.
			\item[4)]
				Für $p=2$ ist $L^2(\Omega, \Sigma, \my)$ nach \ref{2.12} vollständig und
				\[
					\|f\|_2 = \bigg( \int_{\Omega} |f|^2 \dx[\my] \bigg)^{\f 12}
				\]
				wird erzeugt vom Skalarprodukt $\<f,g\> := \int_{\Omega} f \_g \dx[\my]$.
				Das Integral des Skalarprodukts konvergiert wegen
				\[
					\int_{\Omega} |f \_g|\dx[\my] \le \|f\|_2 \|g\|_2 < \infty.
				\]
				Also ist $L^2$ ein Hilbertraum.
		\end{enumerate}
	\end{proof}
\end{st}
