% This work is licensed under the Creative Commons
% Attribution-NonCommercial-ShareAlike 3.0 Unported License. To view a copy of
% this license, visit http://creativecommons.org/licenses/by-nc-sa/3.0/ or send
% a letter to Creative Commons, 444 Castro Street, Suite 900, Mountain View,
% California, 94041, USA.

\chapter{Fourierreihen II}
\coursetimestamp{29}{4}{2013}

\section{Kompakte symmetrische Operatoren}


\begin{df} \label{3.1}
	Sei $M \subset L$ ($L$ Prähilbertraum).
	Dann heißt
	\[
		M^\orth := \Big\{ x \in L \suchthat \forall y \in M : \<x,y\> = 0 \Big\}
	\]
	\emph{orthogonales Komplement} von $M$.
\end{df}

\begin{st} \label{3.2}
	\begin{enumerate}[1)]
		\item
			$M^\orth$ ist ein linearer Teilraum und abgeschlossen (d.h. enthält alle Häufungspunkte).
		\item Es gilt
			\begin{align*}
				(\LH \{M\})^\orth &= M^\orth ,\\
				\_M^\orth &= M^\orth.
			\end{align*}
	\end{enumerate}
	\begin{proof}
		Übungsaufgabe \coursehref{blatt07.pdf}{7.2}
	\end{proof}
\end{st}

\begin{conv*}
	Wir betrachten hier und im Folgenden nur \emph{lineare} Operatoren.
\end{conv*}

\begin{st} \label{3.3}
	Sei $A: L \to L$ symmerisch und beschränkt.
	Dann gilt
	\begin{enumerate}[1)]
		\item
			$\displaystyle \<Ax,x\> \in \R$,
		\item
			$\displaystyle \im(A)^\orth = \ker(A)$,
		\item
			$\displaystyle A \Big|_{\_{\im (A)} }$ ist injektiv und $A \Big|_{\ker(A)} = 0$,
		\item
			$\displaystyle \|A\| = \sup_{\|x\|=1} |\<Ax,x\>|$.
	\end{enumerate}
	\begin{proof}
		\begin{enumerate}[1)]
			\item
				Da $A$ symmetrisch, gilt
				\[
					\<Ax, x\> = \<x, Ax\> = \_{\<Ax,x\>}.
				\]
				Also $\<Ax, x\> \in \R$.
			\item
				Da $A$ symmetrisch, gilt
				\begin{align*}
					x \in \im(A)^\orth
					& \iff \forall y \in \im(A) : \<x,y\> = 0 \\
					& \iff \forall \tilde y \in L: \<x, A\tilde y\> = 0 \\
					& \iff \forall \tilde y \in L: \<Ax,\tilde y\> = 0 \\
					& \iff Ax = 0 \\
					& \iff x \in \ker(A)
				\end{align*}
			\item
				Seien $x,y \in \_{\im(A)}$ mit $Ax = Ay$, also auch $x-y \in \_{\im(A)}$ (betrachte dazu $x,y$ als Folgengrenzwerte).

				Dann ist $A(x-y) = 0$, also nach 2)
				\[
					x-y \in \ker (A) = \im(A)^\orth = \_{\im(A)}^\orth
				\]
				Also 
				\[
					x-y \in \_{\im(A)} \cap \_{\im(A)}^\orth = \{0\}.
				\]
				und damit $A$ injektiv.
			\item
				Falls $\|A\| = 0$, also $A=0$, dann ist $0 = \sup |\<Ax,x\>|$ trivial.

				Sei nun $\|A\| > 0$ und $d := \sup_{\|x\|=1} |\<Ax,x\>|$.
				Zeige $d = \|A\|$:
				\begin{enumerate}[a)]
					\item
						Es gilt $d \le \|A\|$, denn
						\begin{align*}
							|\<Ax,x\>| 
							&\stack{\text{CSB}} \le \|Ax\| \underbrace{\|x\|}_{=1} \\
							&\stack{\ref{1.23}}\le \|A\| \|x\|
							= \|A\|.
						\end{align*}
					\item
						Zeige $d \ge \|A\|$.

						Für $y \neq 0$ gilt
						\[
							|\<Ay,y\>| 
							= \|y\|^2 \Big| \< A \tf{y}{\|y\|}, \tf{y}{\|y\|} \> \Big| 
							\le d\|y\|^2.
						\]

						Und für $\alpha > 0$:
						\begin{align*}
							\Big\< A(\alpha x + \tf 1\alpha Ax), \alpha x + \f 1\alpha A x \Big\>
							- \Big\< A(\alpha x - \tf 1\alpha Ax), \alpha x - \f 1\alpha A x \Big\>
							&= 2\< A\alpha x, \tf 1\alpha A x\> + 2\<A (\tf 1\alpha Ax), \alpha x\> \\
							&= 4 \|Ax\|^2.
						\end{align*}
						Falls $\|Ax\| = 0$, dann ist offensichtlich $\|Ax\| \le d$.
						Sei also $\|x\| = 1$ mit $\|Ax\| \neq 0$.
						Setze $\alpha^2 := \|Ax\|$, dann ist
						\begin{align*}
							4\|Ax\|^2
							&= 
							\bigg| \Big\< A(\alpha x + \tf 1\alpha Ax), \alpha x + \tf 1\alpha A x \Big\>
							- \Big\< A(\alpha x - \tf 1\alpha Ax), \alpha x - \tf 1\alpha A x \Big\> \bigg| \\
							&\le \bigg| \Big\< A(\alpha x + \tf 1\alpha Ax), \alpha x + \tf 1\alpha A x \Big\>
							\bigg| + \bigg| \Big\< A(\alpha x - \tf 1\alpha Ax), \alpha x - \tf 1\alpha A x \Big\> \bigg| \\
							&\le  d \Big\| \alpha x + \tf 1\alpha Ax \Big\|^2 + d \Big\| \alpha x - \tf 1\alpha Ax \Big\|^2 \\
							&= d ( \|\alpha x\|^2 + 2 \Re \<\alpha x, \tf 1\alpha Ax\> + \| \tf 1\alpha Ax\|^2
							+  \|\alpha x\|^2 - 2 \Re \<\alpha x, \tf 1\alpha Ax\> + \| \tf 1\alpha Ax\|^2) \\
							&= 2d (\|\alpha x\|^2 + \|\tf 1\alpha Ax\|^2) \\
							&= 2d (\|A x\|\|x\|^2 + \f 1{\|Ax\|}\|Ax\|^2) \\
							&= 4d \|Ax\|.
						\end{align*}
						Also $\|Ax\| \le d$ für $\|x\| = 1$ und damit
						\[
							\|A\| = \sup_{\|x\|=1} \|Ax\| \le d.
						\]
				\end{enumerate}
		\end{enumerate}
	\end{proof}
\end{st}

\begin{nt} \label{3.4}
	Ist zusätzlich zu \ref{3.3} $(L,\<\argdot,\argdot\>)$ ein Hilbertraum, dann kann 2) in der Form
	\[
		L = \_{\im(A)} \oplus \ker(A)
	\]
	als direkte Summe geschrieben werden.
	Dies bedeutet
	\[
		\forall x \in L \exists! y \in \_{\im(A)} \exists! z \in \ker(A) : x = y + z.
	\]
	\begin{proof}
		 Sei $x\in L$ beliebig. Dann folgt aus dem Projektionssatz (aus der Funktionalanalysis, siehe Übungsaufgabe \coursehref{blatt07.pdf}{7.1}) eine eindeutige Zerlegung 
\[
x=\underbrace{y}_{\in \overline{\im(A)}}+\underbrace{z}_{\in \overline{\im(A)}^\orth\stackrel{\ref{3.2}}=\im(A)^\orth\stackrel{\ref{3.3}}=\ker(A)}.
\]
	\end{proof}
\end{nt}


\begin{st}[Hauptsatz über symmetrische, kompakte Operatoren] \label{3.5}
	Sei $(L, \<\argdot,\argdot\>)$ unendlichdimensionaler Prähilbertraum und $A: L \to L$ linearer, symmetrischer, kompakter Operator.

	Dann gilt:
	\begin{enumerate}[1)]
		\item
			$\lambda = \|A \|$ oder $\lambda = - \|A\|$ ist Eigenwert von $A$.
		\item
			Jeder Eigenwert $\lambda \neq 0$ ist reell und hat endliche Vielfachheit.
		\item
			Falls $A$ nur endlich viele Eigenwerte ungleich $0$ besitzt, dann ist $\lambda = 0$ Eigenwert mit unendlicher Vielfachheit (d.h. $\dim (\ker (A)) = \infty)$.

			Falls $A$ unendlich viele Eigenwerte ungleich $0$ besitzt, dann ist die Menge dieser Eigenwerte abzählbar und für jede Folge $(\lambda_j)$ der Eigenwerte gilt $\lambda_j \to 0$ ($j \to \infty$).
		\item
			Sei $(\lambda_j)$ eine Abzählung der Eigenwerte, die so gebildet wird, dass
			\begin{itemize}
				\item
					$|\lambda_j|$ ist monoton fallend.
				\item
					Jeder Eigenwert $\lambda$ kommt in der Folge so oft vor, wie es seiner Vielfachheit entspricht.
			\end{itemize}
			Zu dieser Folge existiert ein ONS $(e_j)$ aus Eigenelementen.
			Dieses ONS ist vollständig in $\_{\im (A)}$, d.h.
			\[
				\forall x \in \_{\im(A)} : x = \sum_{j=1}^\infty \<x,e_j\> e_j.
			\]
	\end{enumerate}
	\begin{proof}
		Falls $A = 0$ ist nichts zu beweisen.
		Sei also $A \neq 0$ und damit insbesondere $\|A\| > 0$.
		\begin{enumerate}[1)]
			\item
				Wegen \ref{3.3} 4) existiert eine Folge $(x_n)$ mit
				\[
					\|x_n\| = 1 \qquad \land \qquad |\<Ax_n,x_n\>| \to \|A\|
				\]
				Da $A$ kompakt, existiert eine Teilfolge $(x_{n_k})_{k\in \N}$ mit $Ax_{n_k} \to y \in L$ für $k \to \infty$.
				Sei 
				\[
					\lambda := \lim_{k\to\infty} \<Ax_{n_k},x_{n_k}\> = \pm \|A\| \neq 0.
				\]
				Es gilt
				\begin{align*}
					\|Ax_{n_k} - \lambda x_{n_k}\|^2
					&= \underbrace{\|Ax_{n_k} \|^2}_{\le \|A\|^2 \|x_{n_k}\|^2} - 2 \Re \underbrace{\< Ax_{n_k}, \lambda x_{n_k} \>}_{\in \R} + \lambda^2 \underbrace{\|x_{n_k}\|^2}_{=1} \\
					&\le \underbrace{\|A\|^2 + \lambda^2}_{=2\lambda^2} - 2 \lambda \underbrace{\<Ax_{n_k}, x_{n_k}\>}_{\to \lambda} \\
					& \to 0 \qquad (k\to \infty).
				\end{align*}
				Also 
				\[
					y = \lim_{k\to \infty} Ax_{n_k} = \lim_{k\to \infty} \lambda x_{n_k}.
				\]
				Weil $A$ stetig ist (da beschränkt, siehe \ref{1.27} und \ref{1.25}), gilt:
				\[
					Ay 
					= A \lim_{k\to\infty} A x_{n_k} 
					= A \lim_{k\to\infty} \lambda x_{n_k}
					= \lambda \lim_{k\to\infty} Ax_{n_k}
					= \lambda y.
				\]
				Weiter ist
				\[
					\|y\| 
					= \Big\|\lim_{k\to\infty} \lambda x_{n_k}\Big\|
					= \lim_{k\to\infty} |\lambda| \|x_{n_k}\|
					= |\lambda|
					> 0.
				\]
				Also ist $\lambda = \pm \|A\|$ Eigenwert zum Eigenvektor $y$. 
			\item
\coursetimestamp{6}{5}{2013}

				Angenommen $\lambda \neq 0$ sei Eigenwert mit unendlicher Vielfachheit, also
				\[
					\dim \ker (A - \lambda \Id) = \infty
				\]
				Sei $\{v_1,v_2,\dotsc\}$ eine abzählbar unendliche linear unabhängige Teilmenge von $\ker(A-\lambda \Id)$.
				Gram-Schmidt auf $\{v_1,v_2,\dotsc\}$ angewandt liefert ein abzählbar unendliches ONS $(e_j)$ in $\ker (A-\lambda \Id)$.
				Dann gilt trivialerweise:
				\begin{itemize}
					\item
						$Ae_j = \lambda e_j$,
					\item
						$(e_j)$ ist beschränkt.
				\end{itemize}
				Da $A$ kompakt, müsste $(Ae_j)$ eine konvergente Teilfolge besitzen.
				$(Ae_j)$ kann jedoch keine konvergente Teilfolge enthalten, denn
				\[
					\|Ae_j - Ae_k\| = |\lambda| \|e_j-e_k\| = |\lambda| \sqrt 2 = \const \neq 0.
				\]
				Dies führt zu einem Widerspruch, also hat $\lambda$ endliche Vielfachheit.

				Dass $\lambda$ reell ist folgt direkt aus der Symmetrie.
			\item
				Wegen 1) existiert $v_1 \neq 0$ mit $Av_1 = \lambda_1v_1$ und $|\lambda_1| = \|A\|$.

				Betrachte nun $A_1 = A \Big|_{\{v_1\}^\orth}$.
				Es gilt für alle $v \orth v_1$:
				\[
					\<A_1v,v_1\> = \<Av,v_1\> = \<v,Av_1\> = \lambda_1\<v,v_1\> = 0
				\]
				also $\im (A_1) \subset \{v_1\}^\orth$, $A$ operiert also im Prähilbertraum $\{v_1\}^\orth$.
				Außerdem gilt
				\begin{align*}
					\|A_1\| 
					&= \sup_{\substack{v\in \{v_1\}^\orth \\ \|v\|=1}} \|A_1 v\| \\
					&\le \sup_{\substack{v\in L \\ \|v\|=1}} \|A v\| 
					= \|A\|.
				\end{align*}
				$A_1$ ist kompakt: für eine beschränkte Folge $\{x_n\}$ in $\{v_1\}^\orth$ existiert eine konvergente Teilfolge bezüglich $A$:
				\[
					A x_{n_k} \to y \in L.
				\]
				Es gilt $A_1 x_{n_k} \in \{v_1\}^\orth$ und $y \in \{v_1\}^\orth$, da $\{v_1\}^\orth$ abgeschlossen.

				Außerdem ist $A_1$ symmetrisch.

				Wegen 1) existiert $v_2 \neq 0$ mit $Av_2 = \lambda_2v_2$ und $|\lambda_2| = \|A_1\| \le \|A\|$.
				Betrachte nun 
				\[
					A_2 = A \Big|_{\{v_1,v_2\}^\orth} 
				\]
				Setze dies induktiv fort und erhalte eine Folge von Eigenwerten $(\lambda_j)$ von $A$ mit $|\lambda_{j+1}| \le |\lambda_j|$.
				Betrachte zwei Fälle:
				\begin{enumerate}[1. {Fall}]
					\item
						Es existiert ein (minimales) $j\in\N$ mit $\lambda_j = 0$, dann ist	wegen $\|A_{j-1}\| = 0$ und $Av_i\neq 0$ (für $1\le i \le j-1$)
						\[
							\{v_1,\dotsc,v_{j-1}\}^\orth = \ker(A_{j-1}) = \ker(A)
						\]
						und somit $\dim (\ker(A)) = \dim \{v_1,\dotsc, v_{j-1}\}^\orth = \infty$.
						Also ist $\lambda = 0$ Eigenwert von $A$ mit unendlicher Vielfachheit.
					\item
						Für alle $j \in \N$ ist $\lambda_j \neq 0$.
						Angenommen $\lambda_j \not\to 0$ ($j \to \infty$), d.h. es existiert $a > 0$ mit $|\lambda_j| \ge a$ für alle $j \in \N$.

						Wende in jedem Eigenraum Gram-Schmidt an (Eigenvektoren zu verschiedenen Eigenwerten sind schon orthogonal) und erhalte ein ONS $(e_j)_{j\in\N}$ aus Eigenvektoren mit
						\[
							A e_j = \lambda_j e_j
							\qquad \text{und} \qquad
							\text{$\lambda_j$ monoton fallend.}
						\]
						$(\f 1 {\lambda_j} e_j)$ ist beschränkt, aber $(A \f 1 {\lambda_j} e_j)$ kann keine konvergente Teilfolge enthalten, da $A \f 1 {\lambda_j} e_j = e_j$.
						Ein Widerspruch, also gilt $\lambda_j \to 0$.
				\end{enumerate}
			\item
				Verwende die Folge $(\lambda_j)$ mit zugehörigem ONS $(e_j)$ wie im Beweis von 3) konstruiert.

				Sei $x \in L$ und definiere
				\[
					x_n := x - \sum_{j=1}^n \<x,e_j\> e_j.
				\]
				Es gilt
				\begin{itemize}
					\item
						$x_n \in \{e_1,\dotsc,e_n\}^\orth$, denn
						\[
							\<x_n, e_k\> = \<x, e_k\> - \sum_{j=1}^n \<x,e_j\> \<e_j, e_k\> = 0
							\qquad \forall k \in \{1, \dotsc, n\}
						\]
					\item
						$\|Ax_n\| \to 0$ für $n\to \infty$, denn
						\begin{align*}
							\|x_n\|^2  = \<x_n, x_n\>
							&= \|x\|^2 - \sum_{j=1}^n \underbrace{\<x,e_j\>\<e_j,x\>}_{=|\<x,e_j\>|^2} - \sum_{j=1}^n \underbrace{\_{\<x,e_j\>} \<x,e_j\>}_{=|\<x,e_j\>|^2} + \sum_{j=1}^n |\<x,e_j\>|^2 \\
							&= \|x\|^2 - \sum_{j=1}^n (\<x,e_j\>)^2 \\
							&\le \|x\|^2
						\end{align*}
						also $\|x_n\| \le \|x\|$ und damit
						\[
							\|Ax_n\| = \|A_{n-1}x_n\| \le \|A_{n-1}\| \|x_n\| \le \|A_{n-1}\| \|x\| \to 0
							\qquad (n \to \infty).
						\]
				\end{itemize}
				Es gilt
				\[
					\<x,e_j\> A e_j
					= \<x,e_j\> \lambda_j e_j
					= \<x,\lambda_j e_j\> e_j
					= \<x,A e_j\> e_j
					= \<Ax,e_j\> e_j.
				\]
				Damit ist
				\[
					Ax = \lim_{n\to\infty} Ax_n + \sum_{j=1}^n \<x,e_j\> Ae_j = \sum_{j=1}^\infty \<Ax, e_j\> e_j \in \im (A)
				\]
				und somit für alle $y \in \im (A)$
				\[
					y = \sum_{j=1}^\infty \<y,e_j\> e_j.
				\]
				Nach \ref{1.12} 3) gilt selbiges für $\_{\im (A)}$.
		\end{enumerate}
	\end{proof}
\end{st}


\section{Randwertprobleme zweiter Ordnung}


\setcounter{thm}{4} % Nummerierung wurde in der Vorlesung wiederholt
\begin{ex}[Schwingende Saite] %\label{3.5}
	$u(t,x)$ gebe die Auslenkung der Saite im Punkt $x$ zur Zeit $t$ an. 
	Die Physik liefert die DGL:
	\[
		\rho(x) \partial_t^2 u(t,x) - \sigma(x) \partial_x^2 u(t,x) = 0
	\]
	Dabei bezeichnet $\rho(x) > 0$ die Massendichte und $\sigma(x) > 0$ die Federkonstante.
	Außerdem gilt die Randbedingung für die eingespannte Saite
	\[
		u(t,0) = u(t,l) = 0
		\qquad t \ge 0
	\]
	und die Anfangswertbedingungen
	\begin{align*}
		u(0,x) &= u_0(x) \\
		\partial_t u(0,x) &= u_1(x)
	\end{align*}
	für gegebene Anfangsauslenkung $u_0(x)$ und Anfangsgeschwindigkeit $u_1(x)$ im Zeitpunkt $t = 0$.

	Wir zeigen jetzt die Existenz einer Lösung.

	Setze $c(x) := \sqrt{\f {\sigma(x)}{\rho(x)}}$ und schreibe
	\[
		\partial_t^2 u(t,x) - c^2(x) \partial_x^2 u(t,x) = 0.
	\]
	Setze als naïven Ansatz $u(t,x) = v(t)w(x)$ in die DGL ein.
	\[
		v''(t) w(x) - c^2(x) v(t) w''(x) = 0.
	\]
	Falls $v,w \neq 0$, ist
	\[
		\f {v''(t)}{v(t)} = c^2(x) \f {w''(x)}{w(x)} = \const =: - \lambda
	\]
	(die linke Seite ist nur von $t$ und die rechte nur von $x$ abhängig, also müssen sie konstant sein).
	Wir erhalten zwei DGLs:
	\begin{align*}
		v''(t) + \lambda v(t) &= 0 \quad \land  \quad v(0), v'(0) \text{ vorgegeben} \qquad t \ge 0 \\
		c^2 w''(t) + \lambda w(t) &= 0 \quad \land \quad w(0) = w(l) = 0 \qquad 0 \le x \le l.
	\end{align*}
	Erstere stellt ein Anfangswertproblem, zweitere ein Randwertproblem dar.

	Betrachte den Spezialfall $c = 1$.
	Das Randwertproblem besitzt nur für 
	\[
		0 < \lambda = \lambda_j = \big( \f {j\pi}{l} \big)^2
	\]
	die Lösungen
	\[
		w_j(x) = \sqrt{\f 2l} \sin( \tf {j\pi}l x).
	\]
	Die $w_j$ sind normiert, denn $\int_{0}^l \omega_j \omega_k \dx = \delta_{jk}$.

	Die dazugehörige Lösung des Anfangswertproblems ist
	\[
		v_j(t) = v_j(0) \cos(\tf {j\pi}l t) + \f l{j\pi} v_j'(0) \sin(\tf{j\pi}l t)
	\]
	Die Lösungen
	\[
		u_j(t,x) = w_j(x) v_j(x)
	\]
	sind Eigenschwingenen der Saite (es sind auch alle Lösungen, ohne Beweis).
	
	Die allgemeine Lösung lässt sich auch als Fourierreihe darstellen:
	\[
		u(t,x) = \sum_{j=1}^\infty w_j(x) v_j(t)
	\]
	Es bleibt zu zeigen, dass die Reihe und ihre Ableitungen gleichmäßig konvergieren.
\end{ex}


\begin{df}[Sturm-Liouville'sches Eigenwertproblem] \label{3.6}
	Gesucht sind $\lambda \in \R$ und $u \in C^2([a,b]\to \R)$ mit $u \neq 0$ und
	\begin{align} \label{eq:3.1}
		\begin{aligned}
		(pu')' - qu + \lambda ru &= 0, \qquad a\le x \le b, \\
		R_1 u := \alpha_1 u(a) + \alpha_2 u'(a) &= 0, \\
		R_2 u := \beta_1 u(b) + \beta_2 u'(b) &= 0,
		\end{aligned}
	\end{align}
	wobei $p \in C^2([a,b] \to \R)$, $p > 0$ auf $[a,b]$, $q,r \in C([a,b] \to \R)$, $r>0$, $\alpha_1^2 + \alpha_2^2 > 0$, $\beta_1^2 + \beta_2^2 > 0$.

	$\lambda$ heißt \emph{Eigenwert}, $u$ \emph{Eigenfunktion} von \eqref{eq:3.1}.
\end{df}

\begin{nt} \label{3.7}
	\begin{enumerate}[1)]
		\item
			Wir unterscheiden folgende Spezialfälle bei den Randwertbedingungen:
			\begin{enumerate}[i)]
				\item
					Die \emph{Dirichletsche Randbedingung}: $\alpha_1 = \beta_1 = 1, \alpha_2 = \beta_2 = 0$, also
					\[
						u(a) = u(b) = 0.
					\]
				\item
					Die \emph{Neumannsche Randbedingung}: $\alpha_1 = \beta_1 = 0, \alpha_2 = \beta_2 = 1$, also
					\[
						u'(a) = u'(b) = 0.
					\]
			\end{enumerate}
		\item
			Es gilt
			\begin{align*}
				R_1 u = 0
				&\iff \begin{pmatrix}
					u(a) \\ u'(a) 
				\end{pmatrix} \orth \begin{pmatrix}
					\alpha_1 \\ \alpha_2
				\end{pmatrix} \\
				&\iff \exists c \in \R : \begin{pmatrix}
					u(a) \\ u'(a)
				\end{pmatrix} = c \begin{pmatrix}
					\alpha_2 \\ -\alpha_1
				\end{pmatrix}.
			\end{align*}


	\end{enumerate}
\end{nt}

\coursetimestamp{8}{5}{2013}
\begin{st} \label{3.8}
	Jeder Eigenwert $\lambda$ von \eqref{eq:3.1} hat Vielfachheit $1$.
	\begin{proof}
		Wir nehmen an, $\lambda$ habe Vielfachheit $\ge 2$.

		Seien $u_1, u_2$ Lösungen von \eqref{eq:3.1} und $\{u_1,u_2\}$ linear unabhängig.
		$\{u_1,u_2\}$ spannt den ganzen Lösungsraum der DGL in \eqref{3.1} auf.

		Es existiert $(c_1,c_2)^T \in \R$ mit
		\[
			\underbrace{\begin{pmatrix}
				u_1(a) & u_2(a) \\
				u_1'(a) & u_2'(a)
			\end{pmatrix}}_{\det(\argdot) \neq 0}
			\begin{pmatrix}
				c_1 \\ c_2
			\end{pmatrix}
			= \begin{pmatrix}
				\alpha_1 \\ \f{\alpha_2}{p(a)}
			\end{pmatrix}.
		\]
		Also gilt für $u:= c_1u_1 + c_2 u_2$:
		\begin{align*}
			R_1 u &= \alpha_1(c_1u_1(a) + c_2u_2(a)) + \alpha_2 p(a) (c_1u_1'(a) + c_2u_2'(a)) \\
			&= \alpha_1^2 + \alpha_2^2
			\neq 0,
		\end{align*}
		was im Widerspruch zu $R_1u = c_1R_1u_1 + c_2R_1u_2 = 0$ steht.
		Also hat $\lambda$ Vielfachheit $1$.
	\end{proof}
\end{st}

\begin{df}[Sturm-Liouville-Operator] \label{3.9}
	Sei $I := [a,b]$, $L := C([a,b] \to \R)$ mit $\<f,g\> := \int_a^b f(x)g(x) \dx$ Prähilbertraum über $\R$.
	Definiere
	\begin{align}
		\label{eq:3.2}
		\begin{aligned}
			\tilde A u &:= -(pu')' + qu, \\
			D(A) &:= \Big\{ u \in C^2([a,b] \to \R) : R_1u = R_2u = 0 \Big\}, \\
			Au &:= \tilde A u \qquad \text{für $u\in D(A)$}.
		\end{aligned}
	\end{align}
\end{df}

\setcounter{thm}{8}
\begin{st} %\label{3.9}
	Mit den Voraussetzungen aus \eqref{eq:3.2} ist $A$ symmetrisch.
	\begin{proof}
		Es gilt
		\begin{align*}
			\<Au,v\> 
			&= \int_a^b (pu')'v - quv) \dx \\
			&= pu'v \Big|_a^b - \int_a^b (pu'v' + quv) \dx \\
			&= \underbrace{\Big[ pu'v - upv' \Big]_{x=a}^b}_{\text{=0 \text{ (s.u.)}}} + \underbrace{\int_a^b (u(pv')' - quv) \dx}_{= \<u,Av\>}.
		\end{align*}
		Außerdem:
		\begin{align*}
			pu'v - upv' 
			&= \f 1{\alpha_1^2 + \alpha_2^2} \Big( \alpha_1^2 (pu'v - upv') + \alpha_2^2 (pu'v-upv') \Big) \displaybreak[0]\\
			&= \f 1{\alpha_1^2 + \alpha_2^2} \bigg( \alpha_1^2 pu'\underbrace{(\alpha_1 v + \alpha_2 pv')}_{=R_1v = 0} - \alpha_1^2 pv'\underbrace{(\alpha_2pu'+ \alpha_1 u)}_{= R_1u = 0} \\
			 & \qquad\qquad\quad + \alpha_2 v\underbrace{(\alpha_2 p u' + \alpha_1 u}_{=R_1 u = 0}  - \alpha_2 u \underbrace{(\alpha_1 v + \alpha_2 pv')}_{= R_1v= 0} \bigg)  \\
			 &= 0.
		\end{align*}
		Also ist $\<Au,v\> = \<Av,u\>$ und damit $A$ symmetrisch.
	\end{proof}
\end{st}

\begin{nt} \label{3.10}
	Nur im Fall $r= 1$ sind die Eigenwerte von $A$ und von \eqref{eq:3.1} die selben.
	Ist $\lambda$ Eigenwert von \eqref{eq:3.1} mit Eigenfunktion $u$, so gilt
	\[
		A u(x) = \lambda r(x) u(x)
	\]
\end{nt}

\begin{df} \label{3.11}
	Sei I wie in \ref{3.9}. Eine Funktion $G: I \times I \to \R$ heißt \emph{Greensche Funktion} zu $A$, falls
	\begin{enumerate}[1)]
		\item
			$G \in C(I\times I \to \R)$;
		\item
			\fixme[Bild für $\Delta_1$ und $\Delta_2$].
			Sei $\Delta_1:=\{(x,y)\in I \times I | x \le y\}, \Delta_2:=\{(x,y)\in I \times I | x \ge y\}$, dann soll gelten.

			$\partial_1 G, \partial_1^2 G \in C(\Delta_1 \to \R)$ stetig fortsetzbar auf $\_{\Delta_1}$, \\
			$\partial_1 G, \partial_1^2 G \in C(\Delta_2 \to \R)$ stetig fortsetzbar auf $\_{\Delta_2}$;
		\item
			Für festes $\xi \in I$
			\begin{align*}
				\tilde A \big(G(\argdot, \xi)\big) &= 0 \qquad \text{in $I \setminus \{\xi\}$} \\
				R_1 \big(G(\argdot, \xi)\big ) = R_2 G(\argdot, \xi) &= 0;
			\end{align*}
		\item
			Für $\xi \in ]a,b[$
			\begin{align*}
				\partial_1 G (\xi + 0, \xi) - \partial_1 G(\xi - 0, \xi) = -\f 1{p(\xi)}.
			\end{align*}
	\end{enumerate}
\end{df}

\begin{st} \label{3.12}
	Existiert eine Greensche Funktion $G$ zu $A$, so sind für $\phi \in C(I \to \R)$ folgende Aussagen äquivalent:
	\begin{enumerate}[(i)]
		\item
			$Au = \phi$, also insbesondere $u \in D(A)$,
		\item
			$u$ ist eindeutig bestimmt durch
			\[
				u(x) = \int_a^b G(x,y) \phi(y) \dx[y].
			\]
	\end{enumerate}
	\begin{proof}
		\begin{seg}[(ii) $\implies$ (i)]
			Wir teilen das Integral auf und leiten es nach der Leibnitzregel ab:
			\begin{align*}
				u(x)
				&= \int_a^x G(\underbrace{x,y}_{\in \Delta_2})\phi(y) \dx[y] + \int_x^b G(\underbrace{x,y}_{\in \Delta_1})\phi(y) \dx[y] \displaybreak[0].\\
				\implies \quad u'(x) &=
				G(x,x) \phi(x) - 0 + \int_a^x \partial_1 G(x,y) \phi(y) \dx[y]  \\
				&\quad + 0 - G(x,x) \phi(x) + \int_x^b \partial_1 G(x,y) \phi(y) \dx[y] \\
				&=\int_a^x \partial_1 G(x,y) \phi(y) \dx[y] + \int_x^b \partial_1 G(x,y) \phi(y) \dx[y]  \displaybreak[0]. \\
				\implies \quad u''(x) &=
				\underbrace{\partial_1 G(x,x-0)}_{= \partial_1 G(x+0,x)} \phi(x) + 0 - \int_a^x \partial_1^2 G(x,y) \phi(y) \dx[y] \\
				&\quad+ 0 - \underbrace{\partial_1 G(x,x+0)}_{=\partial_1 G(x-0,x)} \phi(x) + \int_x^b \partial_1^2 G(x,y) \phi(y) \dx[y] \\
				&= \int_a^x \partial_1^2 G(x,y) \phi(y) \dx[y] + \int_x^b \partial_1^2 G(x,y) \phi(y) \dx[y]  \\
				&\quad + \phi(x)\Big(\underbrace{\partial_1 G(x+0,x) + \partial_1 G(x-0,x)}_{= - \f 1{p(x)}}\Big) \\
				&= \int_a^x \partial_1^2 G(x,y) \phi(y) \dx[y] + \int_x^b \partial_1^2 G(x,y) \phi(y) \dx[y] - \f {\phi(x)}{p(x)}.
			\end{align*}
			Damit ist $u \in C^2$ und es gilt
			\begin{align*}
				\tilde A u(x) 
				&= -p u''(x) - p'u'(x) + qu(x) \\
				&= \phi(x) - \int_a^b \underbrace{\Big( -p \partial_1^2 G(x,y) - p' \partial_1 G(x,y) + qG(x,y) \Big)}_{=\tilde A G(x,y)} \phi(y) \dx[y] \\
				&= \phi(x) + \int_a^b \underbrace{\tilde A G (\argdot, y)}_{=0 \text{ nach 3)}} \dx[y] \\
				&= \phi(x). %\\
			\end{align*}
			Außerdem gilt für die Randbedingungen
			\[
				R_1 u = \int_a^b \underbrace{R_1G(\argdot, y)}_{=0 \text{ nach 3)}} \dx[y] = 0
			\]
			und ebenso $R_2 u = 0$, also $u \in D(A)$.
		\end{seg}
		\begin{seg}[(i) $\implies$ (ii)]
			Sei $x \in ]a,b[$ fest, dann ist
			\begin{align*}
				\int_a^b G(x,y) \phi(y) \dx[y]
				&\stack{\text{(i)}}= \int_a^b G(x,y) Au(y) \dx[y] \\
				&= \int_a^b \underbrace{G(x,y)}_{=G(y,x) \text{ nach \ref{3.13}}} \Big( -(p(y)u'(y) )' + q(y)u(y) \Big) \dx[y] \displaybreak[0]\\
				&= -\int_a^x G(y,x) \Big(p(y)u'(y)\Big)' \dx[y] - \int_x^b G(y,x) \Big(p(y)u'(y)\Big)' \dx[y] \\
					&\quad + \int_a^b G(y,x) q(y) u(y) \dx[y] \displaybreak[0]\\
				&= \underbrace{\bigg[ -G(y,x)p(y)u'(y) \bigg]^{y=b}_{y=a}}_{=0 \text{ da $R_j G = 0$}} + \int_a^x \partial_y G(y,x) p(y)u'(y) \dx[y] \\ 
					&\quad + \int_x^b \partial_y G(y,x) p(y) u'(y) \dx[y] + \int_a^b G(y,x) q(y)u(y) \dx[y] \displaybreak[0]\\
				&= \underbrace{\bigg[ \partial_y G(y,x) p(y) u(y) \bigg]^{y=b}_{y=a}}_{=0 \text{ da $R_j u = 0$}} + \bigg[ \partial_y G(y,x) p(y) u(y) \bigg]_{y=x+0}^{y=x-0} \\
					&\quad - \int_a^b \partial_y \Big( p(y) \partial_y G(y,x) \Big) u(y) \dx[y] + \int_a^b G(y,x) q(y) u(y) \dx[y] \displaybreak[0]\\
				&= p(x)u(x) \underbrace{\Big( \partial_1 G(x-0,x) - \partial_1 G(x+0,x) \Big)}_{= \f 1{p(x)}} \\
					&\quad + \int_a^b u(y) \underbrace{\Big(- \partial_y \big( p(y) \partial_y G(y,x) \big) + q(y) G(y,x)\Big)}_{=\tilde A G(y,x) = 0} \dx[y] \\
				&= u(x).
			\end{align*}
		\end{seg}
	\end{proof}
\end{st}

\begin{st} \label{3.13}
	Es gilt
	\begin{enumerate}[1)]
		\item
			$G$ ist symmetrisch: $G(x,y) = G(y,x)$.
		\item
			Zu $A$ existiert höchstens eine Greensche Funktion.
	\end{enumerate}
	\begin{proof}
		Seien $G,H$ Greensche Funktionen zu $A$ und $\phi, \psi \in C(I \to \R)$.
		Setze
		\begin{align*}
			u(x) := \int_a^b G(x,y) \phi(y) \dx[y] \displaybreak[0]\\
			v(x) := \int_a^b H(x,y) \psi(y) \dx[y] 
		\end{align*}
		Dann ist
		\begin{align*}
			\int_a^b \Big(u \cdot \underbrace{Av}_{= -\psi} - \underbrace{Au}_{= - \phi} \cdot v\Big) \dx
			&= \<u,Av\> - \<Au,v\> = 0
		\end{align*}
		Also
		\begin{align*}
			-\int_a^b \underbrace{\int_a^b G(x,y) \phi(y) \dx[y]}_{= u(x)} \psi(x) \dx + \int_a^b \phi(y) \underbrace{\int_a^b H(y,x) \psi(x) \dx}_{= v(y)} \dx[y] = 0
		\end{align*}
		und
		\[
			\int_a^b \int_a^b (G(x,y - H(y,x)) \phi(y) \dx[y] \psi(x) \dx = 0.
		\]
		Da $\phi, \psi \in C(I \to \R)$ beliebig gewählt waren und $G$ und $H$ stetig sind, folgt $G(x,y) = H(y,x)$ für alle $(x,y) \in I\times I$.

		1) folgt dann für $H:= G$ und 2) gilt, da $H$ symmetrisch nach 1):
		\[
			G(x,y) = H(y,x) = H(x,y).
		\]
	\end{proof}
\end{st}

\coursetimestamp{13}{5}{2013}
\begin{st}[Konstruktion der Greenschen Funktion] \label{3.14}
	Folgende zwei Aussagen sind äquivalent:
	\begin{enumerate}[(i)]
		\item
			Zu $A$ existiert eine Greensche Funktion
		\item
			$\lambda = 0$ ist kein Eigenwert von $A$.
	\end{enumerate}
	\begin{proof}
		\begin{seg}[(i)$\implies$(ii)]
			Aus (i) folgt mit \ref{3.12}, dass
			\[
				Au = 0 \quad\implies\quad u = \int_a^b G(x,y) \cdot 0 \dx[y] = 0
			\]
			Also kann $0$ kein Eigenwert von $A$ sein.
		\end{seg}
		\begin{seg}[(ii)$\implies$(i)]
			Seien $u_1,u_2$ reelle Lösungen von 
			\[
				\tilde A u = -(pu')' + qu = 0
			\]
			mit
			\begin{align*}
				R_1u_1 &= 0, \qquad u_1 \neq 0, \\
				R_2u_2 &= 0, \qquad u_2 \neq 0.
			\end{align*}
			(z.B. $\tilde A u_1 = 0, u_1(a) = \alpha_2 p(a), u_1'(a) = - \alpha_1$).
			Dann gilt
			\begin{itemize}
				\item
					$\{u_1,u_2\}$ ist linear unabhängig:

					Angenommen $u_1 = c u_2$, dann ist
					\[
						R_2 u_1 = c R_2 u_2 = 0 = R_1u_1
					\]
					also erfüllt $u_1$ beide Randbedingungen und $u \in D(A)$.
					Wegen $Au_1 = 0$ und nach Voraussetzung (ii) gilt $u_1 = 0$, ein Widerspruch.
				\item
					Die Wronskideterminante lässt sich mit $c_0 \in \R \setminus \{0\}$ schreiben als:
					\[
						W(x) = \begin{vmatrix}
							u_1 & u_2 \\
							u_1' & u_2'
						\end{vmatrix}
						= u_1u_2' - u_2u_1' = \f {c_0}{p(x)},
					\]
					denn es gilt für $j\in \{1,2\}$
					\[
						0 = \tilde A u_j = -pu_j'' - p'u_j' + qu_j
						\quad\implies\quad
						u_j'' = \f 1{p}(-p'u_j' + qu_j).
					\]
					Eingesetzt in die Wronski-Determinante ergibt sich
					\begin{align*}
						\f {\dx[]}{\dx} W(x) 
						&= u_1'u_2' + u_1u_2'' - u_2'u_1' - u_2u_1'' \\
						&= u_1 \f 1p (-p'u_2' + qu_2) - u_2 \f 1p (-p'u_1' + u_1) \\
						&= -\f {p'}p \underbrace{(u_1u_2' - u_2u_1')}_{=W(x)},
					\end{align*}
					also
					\[
						W'(x) = - \f {p'}p W(x).
					\]
					Als Lösungen für $W(x)$ ergibt sich
					\[
						W(x) = \f {c_0}{p(x)} 
						\qquad \text{für ein $c_0\in \R$ (betrachte nur relle Lösungen)}
					\]
					Da $\{u_1,u_2\}$ linear unabhängig, muss $W(x) \neq 0$ auf $[a,b]$, also $c_0\neq 0$.
				\item
					Definiere
					\[
						G(x,\xi) := \begin{cases}
							-\f 1{c_0} u_2(\xi)u_1(x) & a \le x \le \xi \le b, \\
							-\f 1{c_0} u_1(\xi)u_2(x) & a \le \xi < x \le b,
						\end{cases}
					\]
					wobei $c_0 \in \R \setminus \{0\}$ wie oben.

					Dann sind die Eigenschaften aus \ref{3.11} für die Greensche Funktion erfüllt:
					\begin{enumerate}[1)]
						\item
							$G \in C^0(I \times I \to \R)$, da beide Definitionszweige bei $x = \xi$ übereinstimmen.
						\item
							Es gilt
							\begin{align*}
								\partial_1 G &= \begin{cases}
									-\f 1{c_0} u_2(\xi) u_1'(x) & a \le x < \xi \le b, \text{ bzw. $(x,\xi) \in \Delta_1$} \\
									-\f 1{c_0} u_1(\xi) u_2'(x) & a \le \xi < x \le b, \text{ bzw. $(x,\xi) \in \Delta_2$} 
								\end{cases}, \\
								\partial_2 G &= \begin{cases}
									-\f 1{c_0} u_2(\xi) u_1''(x) & a \le x < \xi \le b, \text{ bzw. $(x,\xi) \in \Delta_1$} \\
									-\f 1{c_0} u_1(\xi) u_2''(x) & a \le \xi < x \le b, \text{ bzw. $(x,\xi) \in \Delta_2$} 
								\end{cases}.
							\end{align*}
							Also sind $\partial_j G \in C(\Delta_j \to \R)$ ($j\in \{1,2\}$) und stetig fortsetzbar auf $\_{\Delta_j}$.
						\item
							Sei $x < \xi$, dann ist
							\begin{align*}
								\tilde A G(\argdot, \xi) 
								&= -(p\partial_1 G(\argdot, \xi))' + qG(\argdot, \xi)  \\
								&= - \f 1{c_0} u_2(\xi) \tilde A u_1 
								= 0.
							\end{align*}
							Analog für $x > \xi$.

							Es gilt
							\begin{align*}
								R_1 G(\argdot, \xi) &\stack{\xi \ge a}= R_1 (-\f 1{c_0} u_2(\xi) u_1) = -\f 1{c_0} u_2(\xi) R_1u_1 = 0, \\
								R_2 G(\argdot, \xi) &\stack{\xi < b}= R_2 (-\f 1{c_0} u_1(\xi) u_2) = -\f 1{c_0} u_1(\xi) R_2u_2 = 0.
							\end{align*}
							Für $\xi = b$ folgt die letzte Aussage durch stetige Fortsetzung von $\partial_1 G $ auf $\_{\Delta_2}$.
						\item
							Es gilt
							\begin{align*}
								\partial_1 G(\xi + 0, \xi) - \partial_1 G(\xi - 0, \xi)
								&= \lim_{x \searrow \xi} \partial_1 G(x, \xi) - \lim_{x \nearrow \xi} \partial_1 G(x,\xi) \\
								&= - \f 1{c} u_1(\xi) u_2'(\xi) + \f 1c u_2(\xi) u_1'(\xi) \\
								&= - \f 1{c_0} W(\xi)
								= - \f 1{c_0} \f {c_0}{p(\xi)} = - \f 1{p(\xi)}.
							\end{align*}
					\end{enumerate}
					Somit ist $G$ die gesuchte Greensche Funktion.
			\end{itemize}
		\end{seg}
	\end{proof}
\end{st}

\begin{lem} \label{3.15}
	Es existiert $c \in \R$ sodass für alle $u \in D(A)$:
	\[
		\<Au,u\> \ge -c \|u\|^2.
	\]
	Falls
	\begin{align*}
		\alpha_1 = \beta_1 &= 0 \qquad \text{ (Neumann-Randbedingung)}  \\
		\text{oder} \qquad
		\alpha_2 = \beta_2 &= 0 \qquad \text{ (Dirichlet-Randbedingung)}
	\end{align*}
	gilt, kann
	\[
		-c = \min_{a \le x \le b} q(x) 
	\]
	gewählt werden.

	Für alle Eigenwerte von $A$ gilt damit
	\[
		\lambda \ge -c.
	\]
	\begin{proof}
		Es gilt 
		\begin{align*}
			\<Au,u\> 
			&= \int_a^b (-(pu')' + qu) u \dx \\
			&= -pu'u \Big|_{x=a}^b + \int_a^b (\underbrace{p(u')^2}_{\ge 0} + qu^2) \dx.
		\end{align*}
		Falls die obigen Randbedingung gelten, dann ist $pu'u \big|_a^b = 0$ und damit 
		\[
			\<Au,u\> 
			\ge  \int_a^b q \underbrace{u^2}_{\ge 0} \dx 
			\ge \underbrace{\min q(x)}_{=: -c} \int_a^b u^2 \dx.
		\]
		Betrachte jetzt den allgemeinen Fall.
		Mit $\phi(x) := \f {x-a}{b-a}$ folgt
		\begin{align*}
			u(b)^2 
			&= \int_a^b \f {\dx[]}{\dx} (\phi(x) u^2(x)) \dx \\
			&= \int_a^b \underbrace{\f 1{b-a}}_{p'(x)} u^2(x) \dx + \int_a^b \underbrace{\phi(x)}_{0\le \phi \le 1} \underbrace{2u(x) u'(x)}_{= 2(\f 1\eps u(x))(\eps u'(x))} \dx \\
			&\le \Big( \f 1{b-a} + \f 1{\eps^2}\Big)\|u\|^2 + \eps^2 \|u'\|^2.
		\end{align*}
		Mit der Wahl $\phi = \f {b-x}{b-a}$ gilt analog
		\[
			u(a)^2 \le \Big( \f 1{b-a} + \f 1{\eps^2}\Big)\|u\|^2 + \eps^2 \|u'\|^2,
		\]
		also
		\begin{align*}
			\<Au,u\> 
			&= -pu'u \Big|_{x=a}^b + \int_a^b \Big(p(u')^2 + \underbrace{qu^2}_{\ge 0}\Big) \dx \\
			&\ge - \max_{a \le x \le b} p(x) \Big( |u'(b)u(b)| + |u'(a)u(a)|\Big)  + \Big( \min_{a \le x \le b} p(x) \Big) \|u'\|^2 + (\min q_{a \le x \le b}) \|u\|^2
		\intertext{Dabei ist $|u'(a)u(a)| = 0$, falls $\alpha_2 = 0$ und sonst $|u'(a)u(a)| = c|u(a)|^2 = cu^2(a)$. 
			Analog für $|u'(b)u(b)$ und $\beta_2$. 
			Es gilt also:}\\
			\<Au,u\>
			&\ge - \Big(\max_{a \le x \le b} p(x)\Big) c \Big( |u(a)|^2 + |u(b)|^2 \Big) + \Big(\min p(x)_{a \le x \le b}\Big) \|u'\|^2 \\
		\intertext{Mit der Abschätzung für $u(b)$, bzw. $u(a)$ von vorhin ergibt sich}\\
			\<Au,u\>
			&\ge - \Big(\max_{a \le x \le b} p(x) \Big) c \cdot 2  \Big( \f 1{b-a} + \f 1{\eps^2} \Big) \|u\|^2 
			 - \Big(\max_{a \le x \le b} p(x) \Big) c \cdot 2  \eps^2 \| u'\|^2 + \underbrace{\min_{a \le x \le b} p(x)\|u'\|^2}_{> 0}
		\intertext{Wählt man $\eps$ so klein, dass $-(\max p) c 2\eps^2 + \min p = 0$}
			\<Au,u\>&\ge - \Big(\max_{a \le x \le b} p\Big) \cdot 2 c \Big(\tf 1{b-a} + \tf 1{\eps^2}\Big) \|u\|^2.
		\end{align*}
		Und damit ist die Aussage gezeigt. \fixme[überprüfe den Beweis auf formaler Richtigkeit].
	\end{proof}
\end{lem}

\coursetimestamp{15}{5}{2013}
\begin{nt*}
	Es gilt
	\begin{align*}
		&\text{$\lambda$ ist Eigenwert von \eqref{eq:3.1} mit Eigenfunktion $u$} \\
		&\qquad \iff \quad -Au + \lambda ru = 0 \\
		&\qquad\iff \quad Au = \lambda ru.
	\end{align*}
\end{nt*}

\begin{st} \label{3.16}
	\begin{enumerate}[1)]
		\item
			Es gilt
			\[
				\inf \Big\{ \lambda \in \R : \text{$\lambda$ ist Eigenwert von \eqref{eq:3.1}} \Big\} > -\infty
			\]
		\item
			Falls $\alpha_1 = \beta_1 = 0$ oder $\alpha_2 = \beta_2 = 0$ gilt, dann ist
			\[
				\inf \Big\{ \lambda \in \R : \text{$\lambda$ ist Eigenwert von \eqref{eq:3.1}} \Big\} \ge \f {\min q(x)}{\max r(x)}
			\]
	\end{enumerate}
	\begin{proof}
		Sei $\lambda$ Eigenwert von \eqref{eq:3.1} mit Eigenfunktion $u$. Dann ist
		\begin{align*}
			\lambda \max r \int_a^b u^2 \dx 
			&\ge \lambda \int_a^b r u^2 \dx \\
			&= \int_a^b Au u \dx \\
			&= \<Au,u\> \\
			&\stack{\ref{3.15}}\ge \begin{cases}
				-c \|u\|^2 \\
				\min q(x) \|u\|^2 & \text{für die Randbedingungen.}
			\end{cases}
		\end{align*}
	\end{proof}
\end{st}

\begin{st} \label{3.17}
	Sei $d\in \R$ so dass $\lambda = d$ kein Eigenwert von \eqref{eq:3.1} ist ($d$ existiert nach \ref{3.16}).

	Dann ist $\my = 0$ kein Eigenwert des Operators
	\[
		A' := A - dr
	\]
	($D(A') = D(A)$, $A'u = Au - dru$).

	Sei weiter $G'$ die Greensche Funktion zu $A'$ (welche nach \ref{3.14} existiert).
	Definiere
	\begin{align*}
		G(x,y) &:= \sqrt {r(x)} G'(x,y) \sqrt{r(y)} \\
		\scr K u(x) &:= \int_a^b G(x,y) u(y) \dx[y],  \qquad \text{$u\in C(I\to \R)$}.
	\end{align*}
	Dann ist $\my = 0$ kein Eigenwert von $\scr K$ und folgende Aussagen sind äquivalent:
	\begin{enumerate}[(i)]
		\item
			$\lambda = \f 1\my + d$ ist Eigenwert von \eqref{3.1} mit Eigenfunktion $u=\f 1{\sqrt r} v$.
		\item
			$\my = \f 1{\lambda -d}$ ist Eigenwert von $\scr K$ mit Eigenfunktion $v = \sqrt{r} u$
	\end{enumerate}
	\begin{proof}
		\begin{enumerate}[1)]
			\item
				Angenommen $\my = 0$ ist Eigenwert von $A'$, also $A'u = 0$.
				Dann ist $Au = dru$ und somit $\lambda = d$ Eigenwert von \eqref{eq:3.1}, ein Widerspruch.
			\item
				Angenommen $\my = 0$ ist Eigenwert von $\scr K$, dann ist
				\begin{align*}
					0 = \scr K u(x) = \int_a^b \sqrt{r(x)} G'(x,y) \sqrt{r(y)} u(y) \dx[y].
				\end{align*}
				Wegen $r(x) > 0$ also
				\begin{align*}
					0 = \int_a^b G'(x,y) \sqrt{r(y)} u(y) \dx[y].
				\end{align*}
				Nach \ref{3.12} gilt 
				\[
					0 = A'(0) = \sqrt{r} u. 
				\]
				Also $u = 0$, ein Widerspruch.
			\item
				Es gilt
				\begin{alignat*}{2}
					\text{(i)} &\iff& Au &= \lambda ru \\
					&\iff& (A-dr)u &= (\lambda - d)ru \\
					&\iff& A' u &= (\lambda -d)ru \\ 
					&\stack{\ref{3.12}}\iff& u(x) &= \int_a^b G'(x,y) (\lambda -d) r(y) u(y) \dx[y] \\
					&\iff& \underbrace{\sqrt{r(x)} u(x)}_{=:v(x)} &= (\lambda - d) \int_a^b \underbrace{\sqrt{r(x)} G'(x,y) \sqrt{r(y)}}_{G(x,y)} \underbrace{\sqrt{r(y)} u(y)}_{v(y)} \dx[y] \\
					&\iff& v &= (\lambda -d) \scr K v \\
					&\iff \text{(ii).}&&
				\end{alignat*}
		\end{enumerate}
	\end{proof}
\end{st}

\begin{lem} \label{3.18} Es gilt:
	\begin{enumerate}[1)]
		\item
			$\scr K$ ist symmetrisch und kompakt als Operator $\scr K : L \to L$ ($L = C(I \to \R)$, $\<f,g\> = \int_a^b fg \dx$).
		\item
			$\im(\scr K) = \{\sqrt{r} u : u \in D(A)\}$.
	\end{enumerate}
	\begin{proof}
		\begin{enumerate}[1)]
			\item
				Nach \ref{3.13} ist $G'(x,y) = G'(y,x)$, also $G(x,y) = G(y,x)$ und somit $\scr K$ symmetrisch.
				Die Kompaktheit folgt aus \ref{1.28}.
			\item
				\begin{enumerate}[a)]
					\item
						Sei $v \in \im(\scr K)$, also
						\begin{align*}
							v(x) 
							&= \int_a^b \sqrt{r(x)} G'(x,y) \sqrt{r(y)} \phi(y) \dx[y] \\
							&= \sqrt{r(x)} \int_a^b G'(x,y) \sqrt{r(y)} \phi(y) \dx[y] =: \sqrt{r(x)} u(x).
						\end{align*}
						Nach \ref{3.12} wird $A'u= \sqrt{r} \phi$ erfüllt und es ist insbesondere $u \in  D(A') = D(A)$.
						Somit ist $v = \sqrt{r} u$ mit $u \in D(A)$.
					\item
						Sei $v = \sqrt{r} u, u \in D(A) = D(A')$. Dann gilt
						\begin{align*}
							&\text{$u$ löst $A'u = A'u =: \phi$} \\
							&\qquad \stack{\ref{3.12}}\iff u(x) = \int_a^b G'(x,y) (A'u)(y) \dx[y] \\
							&\qquad \iff \underbrace{\sqrt{r(x)}  u(x)}_{=v(x)} = \int_a^b \underbrace{\sqrt{r(x)} G'(x,y) \sqrt{r(y)}}_{G(x,y)} \f{1{\sqrt{r(y)}}} A'u(y) \dx[y] \\
							&\qquad \iff v = \scr K (\tf 1{\sqrt{r}}A'u) \in \im (\scr K).
						\end{align*}
				\end{enumerate}
		\end{enumerate}
	\end{proof}
\end{lem}

\begin{st}[Satz von Sturm-Liouville] \label{3.19}
	\eqref{eq:3.1} besitzt abzählbar unendlich viele Eigenwerte $(\lambda_j)_{j\in \N}$.
	Es gelten folgende Aussagen:
	\begin{enumerate}[1)]
		\item
			$\lambda_j \to + \infty$ für $j \to \infty$.
		\item
			Jeder Eigenwert hat Vielfachheit $1$.
		\item
			Die Eigenfunktionen $u_j$ (jeweils zum Eigenwert $\lambda_j$) können so normiert werden, dass $(\sqrt {r} u_j)$ ein ONS in $L = C(I \to \R)$ bilden.
			Dieses ONS ist vollständig in $L^2([a,b])$.
	\end{enumerate}
	\begin{proof}
		\begin{enumerate}[1)]
			\item
				Nach \ref{3.17} ist $\lambda$ Eigenwert von $\eqref{3.1}$ genau dann wenn $\lambda = \f 1\my + d$ und $\my$ Eigenwert von $\scr K$ ist, $\my \neq 0$, $\scr K$ kompakt und symmetrisch.
				Nach dem Hauptsatz \ref{3.5} hat $\scr K$ abzählbar viele Eigenwerte $(\my_j)$ mit $\my_j \to 0$ gelten.
				Also hat \eqref{eq:3.1} abzählbar viele Eigenwerte $\lambda_j = \f 1{\my_j} + d$ mit $|\lambda_j| \to \infty$.
				Da wegen \ref{3.16}  $\inf \lambda_j > -\infty$ muss $\lambda_j \to + \infty$.
			\item
				Siehe \ref{3.8}.
			\item
				Nach dem Hauptsatz \ref{3.5} hat $\scr K$ ein ONS $(e_j)$ mit $\scr K e_j = \my_j e_j$.
				Eigenfunktionen von \eqref{eq:3.1}: $u_j = \f 1{r} e_j$ zum Eigenwert $\lambda_j = \f 1{\my_j} + d$.
				Also bildet $(\sqrt{r}u_j)_{j\in \N}$ ein ONS.

				Wegen \ref{3.5} ist $(e_j) = (\sqrt{r} u_j)$ vollständig in $\_{\im (K)} = \_{\{\sqrt{r} u : u  \in D(A) \}}$ in $L$.
				$\im (\scr K)$ ist dicht in $L^2([a,b])$.
				Nach Konstruktion ist $\ker{\scr K}=\{0\}$. Nach \ref{1.12} ist also
				\[
					\_{\im(\scr K)}^{L_2} = L^2 ([a,b]).
				\]
		\end{enumerate}
	\end{proof}
\end{st}

\begin{st}[Bessere Konvergenz] \label{3.20}
	Sei $K \subset \R^n$ kompakt, $G \in C(K\times K \to \R)$, $G(x,y) = G(y,x)$, $L=C(K\to \R)$.
	\[
		(\scr K f)(x) := \int_K G(x,y) f(y) \dx[y] \qquad \text{für $f \in L$}
	\]
	Sei $(e_j)$ das ONS aus Eigenfunktionen von $\scr K$ mit Eigenwerten $(\lambda_j)$ (existiert nach \ref{1.28} und \ref{3.5}).
	
	Dann konvergiert für $f \in \im(\scr K)$ die Fourierreihe
	\[
		f(x) = \sum_{i=1}^\infty \<f,e_j\>  e_j(x)
	\]
	absolut und gleichmäßig auf $K$.
\coursetimestamp{27}{5}{2013}
	\begin{proof}
		Sei $f = \scr K g$ mit $g \in L$.
		Dann ist
		\begin{align*}
			\<f,e_j\> 
			= \<\scr K g, e_j\>
			= \<g, \underbrace{\scr K e_j}_{= \lambda_j e_j}\>
			= \lambda_j \<g, e_j\>
		\end{align*}
		\begin{align*}
			\bigg( \sum_{j=N}^{M+N} |\<f,e_j\> e_j(x) | \bigg)^2
			&= \bigg( \sum_{j=N}^{M+N} | \lambda_j \<g, e_j\> e_j(x) | \bigg)^2 \\
			&= \l\< \begin{psmallmatrix} 
				|\<g,e_N\>| \\ \vdots \\ |\<g,e_{N+M}\>|
			\end{psmallmatrix}, \begin{psmallmatrix}
				|\lambda_N e_N(x)| \\ \vdots \\ |\lambda_{N+M} e_{N+M}(x) |
			\end{psmallmatrix} \r\>_{\R^{M+1}}^2 \\
			&\stack{CSB}{\le} \bigg( \sum_{j=N}^{M+N} |\<g,e_j\>|^2 \bigg) \bigg(\sum_{j=N}^{M+N}|\lambda_j e_j(x)|^2\bigg) \tag{$*$}
		\end{align*}
		Da $\sum_{j=1}^\infty |\<g,e_j\>|^2 \le \|g\|^2$ (nach Bessel) konvergiert, lässt sich oben die linke Summe durch beliebiges $\eps > 0$ abschätzen für genügend großes $N$.

		Betrachte nun die rechte Summe.
		Es gilt
		\begin{align*}
			|\lambda_j e_j(x)|
			= |(\scr K e_j)(x)| 
			&= \bigg|\int_{K} G(x,y) e_j(y) \dx[y]\bigg|  \\
			&= \bigg|\_{\int_{K} G(x,y) e_j(y) \dx[y]}\bigg| 
			= \bigg|\int_{K} G(x,y) \_{e_j(y)} \dx[y]\bigg| 
			= \Big|\<G(x,\argdot), e_j \>_L \Big|,
		\end{align*}
		also für die Summe
		\begin{align*}
			\sum_{j=N}^{M+N} |\lambda_j e_j(x)|^2
			&= \sum_{j=N}^{M+N} |\<G(x,\argdot), e_j\>|^2 \\
			&\le \sum_{j=1}^\infty |\<G(x,\argdot), e_j\>|^2 \\
			&\stack{Bessel}{\le} \|G(x,\argdot)\|^2 
			= \int_K |G(x,y)|^2 \dx[y]\\
			&\le \max_{x,y \in K} |G(x,y)|^2 \underbrace{\my(K)}_{< \infty}  \qquad \text{($G$ stetig auf kompakter Menge)} \\
			&\le c.
		\end{align*}
		Also lässt sich $(*)$ für genügend großes $N$ abschätzen:
		\[
			\bigg( \sum_{j=N}^{M+N} |\<f,e_j\> e_j(x) | \bigg)^2
			\le \dotsc \le 
			\bigg( \sum_{j=N}^{M+N} |\<g,e_j\>|^2 \bigg) \bigg(\sum_{j=N}^{M+N}|\lambda_j e_j(x)|^2\bigg)
			< \eps \cdot c.
		\]
		Damit ist $\sum_{i=1}^\infty \<f,e_j\>e_j(x)$ absolut und gleichmäßig konvergent.
	\end{proof}
\end{st}


\section{Sinus-Cosinus-Reihen}

Das bekannteste Orthonormalsystem in $L = L^2([-\pi, \pi])$ ist
\begin{align*}
	(e_1, e_2, \dotsc ) = \bigg( x \mapsto \f 1{\sqrt{2\pi}},\; 
	x \mapsto \f 1{\sqrt \pi} \cos x,\; 
	x \mapsto \f 1{\sqrt \pi} \sin x, \;
	x \mapsto \f 1{\sqrt \pi} \cos(2x), \;
	x \mapsto \f 1{\sqrt \pi} \sin(2x),\; \dotsc \bigg)
\end{align*}
Für die Fourierreihe von $f \in L$ ergibt sich dann
\begin{align*}
	\sum_{j=1}^\infty \<f,e_j\> e_j(x)
	= \f 1{2\pi} \int_{-\pi}^{\pi} f(t) \dx[t] + \sum_{j=1}^\infty \f 1{\pi} \int_{-\pi}^{\pi} f(t) \Big( \cos(jt) \cos(jx) + \sin(jt) \sin(jx) \Big) \dx[t]
\end{align*}

\begin{nt}[Notation] \label{3.21}
	Wir schreiben im Folgenden:
	\begin{align*}
		a_j &:= \f 1{\pi} \int_{-\pi}^{\pi} f(t) \cos(jt) \dx[t] \qquad j\in \N_0, \\
		b_j &:= \f 1{\pi} \int_{-\pi}^{\pi} f(t) \sin(jt) \dx[t] \qquad j\in \N, \\
		s_n(x) &:= \f {a_0}2 + \sum_{j=1}^n \Big( a_j \cos(jx) + b_j \sin(jx) \Big), \\
		s(x) &:= \lim_{n\to \infty} s_n(x) \qquad \text{(falls existent)}.
	\end{align*}
\end{nt}

Konvergiert $s_n(x)$? Wenn ja, gilt dann $s(x) = f(x)$?

\begin{nt}[Beobachtungen] \label{3.22}
	\begin{enumerate}[1)]
		\item
			$s_n$ (und gegebenenfalls $s$) ist $2\pi$-periodisch.
		\item
			Ist $f$ auf $\R$ definiert und $2\pi$-periodisch, so können die Integralgrenzen bei $a_j, b_j$ „verschoben“ werden, also für alle $c \in \R$:
			\begin{align*}
				a_j = \f 1{\pi} \int_{c-\pi}^{c+\pi} f(t) \cos(jt) \dx[t] \\
				b_j = \f 1{\pi} \int_{c-\pi}^{c+\pi} f(t) \sin(jt) \dx[t]
			\end{align*}
		\item
			$a_j, b_j$ ändern sich nicht, wenn $f$ auf einer Nullmenge abgeändert wird.
		\item
			Zur Definition von $a_j, b_j$ reicht die Voraussetzung $f \in L^1([-\pi, \pi]) \supsetneq L^2([-\pi, \pi])$.
	\end{enumerate}
\end{nt}

\begin{lem} \label{3.23}
	Für $f \in C^1([-\pi, \pi] \to \C)$ gilt
	\[
		\int_{-\pi}^{\pi} f(t) \cos(\omega t) \dx[t] \to 0, \qquad
		\int_{-\pi}^{\pi} f(t) \sin(\omega t) \dx[t] \to 0 \qquad (\omega \to \infty).
	\]
	\begin{proof}
		Es gilt für $\omega > 0$:
		\begin{align*}
			\bigg| \int_{-\pi}^{\pi} f(t) \cos(\omega t) \dx[t] \bigg|
			&= \bigg| f(t) \f 1{\omega} \sin(\omega t) - \int_{-\pi}^{\pi} f'(t) \f 1{\omega} \sin(\omega t) \bigg| \\
			&\le \underbrace{|f(t)|}_{\le c_1} \underbrace{\f 1{\omega} |\sin(\omega t)|}_{|\argdot| \le \f 1{\omega}} + \int_{-\pi}^{\pi} \underbrace{|f'(t)|}_{\le c_2} \underbrace{\f 1{\omega} |\sin(\omega t)|}_{|\argdot|\le \f 1{\omega}} \dx[t] \\
			&\le \f {c_1 + 2\pi c_2}\omega \to 0 \qquad (\omega \to \infty)
		\end{align*}
	\end{proof}
\end{lem}

\begin{lem} \label{3.24}
	Für $k \in \N_0$ und $1 \le p < \infty$ ist $C^k([-\pi, \pi] \to \C)$ dicht in $L^p([-\pi, \pi])$.
	\begin{proof}
		Nach Satz \ref{4.18} ist $C_0^\infty$ dicht in $L^p(\R)$. Sei $f\in L^p([-\pi,\pi])$ beliebig, dann existiert eine Folge $(f_n)$ in $C_0^\infty$, 
		die gegen $f$ konvergiert. Nun gilt aber nach Voraussetzung $f_n\in C^\infty(\R)$, also insbesondere $f_n \in C^k([-\pi,\pi])$. 
		Und damit ist alles gezeigt.
	\end{proof}
\end{lem}

\begin{st}[Lemma von Riemann] \label{3.25}
	Für $f \in L^1([-\pi, \pi])$ gilt die Aussage aus \ref{3.23}:
	\[
		\int_{-\pi}^{\pi} f(t) \cos(\omega t) \dx[t] \to 0, \qquad
		\int_{-\pi}^{\pi} f(t) \sin(\omega t) \dx[t] \to 0 \qquad (\omega \to \infty).
	\]
	\begin{proof}
		Zu $f$ wähle $f_\eps \in C^1 ([-\pi,\pi] \to \C)$ mit $\|f_\eps - f\|_{C^1} < \eps$ (\ref{3.24} mit $k=1, p=1$).
		Dann gilt
		\begin{align*}
			\bigg|\int_{-\pi}^{\pi} f(t) \cos(\omega t) \dx[t] \bigg|
			&= \bigg| \underbrace{\int_{-\pi}^{\pi} \underbrace{(f(t) - f_\eps(t)) \cos(\omega t)}_{|\argdot| \le |f-f_\eps|} \dx[t]}_{\le \|f - f_\eps\|_{L^1} = \int_{-\pi}^\pi |f-f_\eps| \dx[t] < \eps} + \underbrace{\int_{-\pi}^{\pi} f_\eps (t) \cos(\omega t) \dx[t]}_{\to 0 \text{ für $\omega \to \infty$ nach \ref{3.23}}} \bigg| \\
			&< 2\eps \qquad \text{für $\omega > K_\eps$.}
		\end{align*}
	\end{proof}
\end{st}

\begin{nt*}[Dirichlet-Kern]
	Wir nennen die Funktionenfolge
	\[
		D_n(x) = \f 1{2\pi} \sum_{j=-n}^n e^{ijx}
	\]
	\emph{Dirichlet-Kern}.

	$D_n(x)$ ist symmetrisch und $2\pi$-periodisch und hat die alternative Darstellung
	\[
		D_n(x) = \f 1{2\pi} \f {\sin((n+\f 12)x)}{\sin(\f 12 x)}
	\]
	(an den Stellen $x = 2k\pi$ für $k \in \Z$ durch $2n + 1$ stetig fortsetzbar).
	Außerdem gilt für $c \in \R$:
	\begin{align*}
		\int_{c-\pi}^{c+\pi} D_n(t) \dx[t] = 1.
	\end{align*}
	\begin{proof}
		Für $D_n(x)$ gilt ($x \neq 0$):
		\begin{align*}
			2\pi D_n(x)
			&= \sum_{j=-n}^n e^{ijx}
			= \sum_{j=0}^{2n} e^{i(j-n)x}
			= e^{-inx} \sum_{j=0}^{2n} (e^{ix})^{j} \\
			&= \underbrace{e^{-inx}}_{= \f {e^{-i(n-\f 12)x}}{e^{-i \f{x}2}}} \f {1- e^{i(2n+1)x}}{1- e^{ix}}
			= \f {e^{-i(n+\f 12)x} - e^{i (n+\f 12)x}}{e^{-i\f{x}2}-e^{i \f x 2}}
			= \f {\sin((n+ \f 12)x)}{\sin(\f x 2)}
		\end{align*}
		und damit
		\[
			D_n(x) = \f 1{2\pi} \f {\sin((n+\f 12)x)}{\sin(\f 12 x)} = D_n(-x).
		\]
		Außerdem ist
		\begin{align*}
			\int_{-\pi}^\pi D_n(t) \dx[t]
			&= \f 1{2\pi} \int_{-\pi}^\pi \sum_{j=-n}^n e^{ijt} \dx[t]
			= \f 1{2\pi} \sum_{j=-n}^n \int_{-\pi}^\pi e^{ijt} \dx[t] \\
			&= \f 1{2\pi} \Bigg( \underbrace{\sum_{\substack{j=-n \\ j\neq 0}}^n \Big[ \f 1{ij} e^{ijt} \Big]_{t=-\pi}^{t=\pi}}_{= 0} + 2\pi \Bigg)
			= 1.
		\end{align*}
		Aufgrund der $2\pi$-Periodizität ergibt sich die Translationsinvarianz des Integrals.
	\end{proof}
\end{nt*}

\begin{lem} \label{3.26}
	Für $f \in L^1([-\pi, \pi])$ $2\pi$-periodisch fortgesetzt auf $\R$ gilt

	\begin{align*}
		s_n(x) - f(x) = \int_{-\pi}^{\pi} \Big(f(x+t) - f(x)\Big) D_n(x) \dx[t]
	\end{align*}
	\begin{proof}
		Es gilt
		\begin{align*}
			s_n(x) 
			&= \f {a_0}2 + \sum_{j=1}^n \Big( a_j \cos(jx) + b_j \sin(jx) \Big) \\
			&= \f 1{2 \pi} \int_{-\pi}^{\pi} f(t) \dx[t] + \sum_{j=1}^n \f 1{\pi} \int_{-\pi}^{\pi} f(t) \Big( \underbrace{\cos(jt) \cos(jx) + \sin(jt) \sin(jx)}_{= \f 12 ( e^{ij(x-t)} + e^{ij(t-x)})} \Big) \dx[t] \\
			&= \int_{-\pi}^{\pi} \bigg( \f 1{2\pi} \sum_{j=-n}^n e^{ij(x-t)} \bigg) f(t) \dx[t] \\
			&= \int_{-\pi}^{\pi} D_n(x-t) f(t) \dx[t].
		\end{align*}
		Und damit
		\begin{align*}
			s_n(x) - f(x) 
			&= \int_{-\pi}^{\pi} f(\tau) D_n(x-\tau) \dx[\tau] - f(x) \underbrace{\int_{-\pi}^{\pi} D_n(t) \dx[t]}_{=1} \\
			&= \int_{-\pi-x}^{\pi-x} \underbrace{f(t + x)\underbrace{D_n(-t)}_{=D_n(t)}}_{\text{$2\pi$-periodisch in $t$}} \dx[t] - \int_{-\pi}^\pi f(x) D_n(t) \dx[t] \\
			&= \int_{-\pi}^{\pi} \Big( f(t + x) - f(x) \Big) D_n(t).
		\end{align*}
	\end{proof}
	\begin{note}
		Aus dem Beweis geht insbesondere folgende Darstellung für $s_n(x)$ hervor:
		\begin{align*}
			s_n(x) = \int_{-\pi}^{\pi} D_n(x-t) f(t) \dx[t]
		\end{align*}
	\end{note}
\end{lem}

\coursetimestamp{29}{5}{2013}
\begin{st}[Kriterium von Dini] \label{3.27}
	Sei $f \in L^1([-\pi,\pi])$ $2\pi$-periodisch fortgesetzt, $x \in [-\pi,\pi]$ fest und
	\[
		\exists \delta > 0 : \int_{-\delta}^{\delta} \l| \f{f(x+t)-f(x)}{t} \r| \dx[t] < \infty
	\]
	Dann gilt 
	\[
		f(x) = \lim_{n\to\infty} s_n(x).
	\]
	\begin{proof}
		Nach \ref{3.26} ist
		\begin{align*}
			s_n(x)-f(x)
			&= \int_{-\pi}^\pi \Big( f(t+x)-f(x) \Big) \f 1{2\pi \sin(\f t2)} \sin((n+\f 12) t) \dx[t] \\
			&= \int_{-\pi}^\pi \underbrace{\underbrace{\f {f(t+x)-f(x)}{t}}_{\in L^1([-\pi,\pi])} \underbrace{\f t{2\pi \sin(\f t2)}}_{\text{stetig ergänzbar in $t=0$}}}_{\in L^1([-\pi,\pi])} \sin((n+\f 12) t) \dx[t] \\
			\intertext{Hier liefert \ref{3.25} schließlich:}
			&\to 0 \qquad (n\to \infty).
		\end{align*}
	\end{proof}
\end{st}

\begin{nt} \label{3.28}
	Die Bedingung von Dini (aus \ref{3.27}) ist erfüllt, falls $f \in L^1([-\pi,\pi])$ $2\pi$-periodisch fortgesetzt und $f$ Hölder-stetig bei $x$ ist, d.h.
	\[
		\exists \alpha, c, \delta > 0 \forall t \in [-\delta, \delta] : |f(x+t) - f(x)| < c |t|^\alpha
	\]
	\begin{proof}
		Es gilt
		\[
			\int_{-\delta}^{\delta} \l| \f{f(x+t) - f(x)}t \r| \dx[t] 
			\le c \int_{-\delta}^\delta |t|^{\overbrace{\alpha - 1}^{>-1}} \dx[t]
			< \infty
		\]
		und somit nach \ref{3.27} $f(x) = \lim_{n\to\infty} s_n(x)$.
	\end{proof}
\end{nt}

\begin{st}[Erweitertes Kriterium von Dini] \label{3.29}
	Sei $f \in L^1([-\pi,\pi])$ $2\pi$-periodisch fortgesetzt, $x \in [-\pi,\pi]$ Unstetigkeitsstelle erster Art (d.h. $f(x-0)$ und $f(x+0)$ existieren).
	Gilt
	\[
		\exists \delta > 0 :
		\int_{-\delta}^0 \l| \f{f(x+t)-f(x-0)}t \r| \dx[t] < \infty
		\;\land\; \int_{0}^{\delta} \l| \f{f(x+t)-f(x+0)}t \r| \dx[t] < \infty
	\]
	so konvergiert die Fourierreihe $s(x)$ und es gilt
	\[
		s(x) = \f 12 \Big( f(x+0) + f(x-0) \Big)
	\]
	\begin{proof}
		Wegen $D_n(-t) = D_n(t)$ und $\int_{-\pi}^\pi D_n(t) \dx[t] = 1$ ist
		\[
			\int_{-\pi}^0 D_n(t) \dx[t] = \int_0^\pi D_n(t) \dx[t] = \f 12
		\]
		Also
		\[
			\f 12 \Big(f(x+0) + f(x-0) \Big) = \int_{-\pi}^0 f(x-0) D_n(t) \dx[t] + \int_{0}^\pi f(x+0) D_n(t) \dx[t]
		\]
		und somit
		\begin{align*}
			s_n(x) - \f 12 \Big( f(x-0) + f(x+0)\Big)
			&= \int_{-\pi}^\pi f(t+x) D_n(t) \dx[t] - \f 12 \Big( f(x-0) + f(x+0)\Big) \\
			&= \int_{-\pi}^0 \Big( f(t+x) - f(x-0) \Big) D_n(t) \dx[t] \\
				&\qquad+ \int_{0}^\pi \Big( f(t+x) - f(x+0) \Big) D_n(t) \dx[t] \\
			&= \int_{-\pi}^\pi \underbrace{g_1(t)}_{\mathclap{\in L^1([-\pi,\pi])}} \sin((n+\f 12)t) \dx[t]
			+ \int_{-\pi}^\pi \underbrace{g_2(t)}_{\mathclap{\in L^1([-\pi,\pi])}} \sin((n+\f 12)t) \dx[t] \\
			&\to 0 \qquad (n\to \infty)
		\end{align*}
		nach \ref{3.25}. Die Voraussetzung von \ref{3.25} ist erfüllt, da
		\begin{align*}
			g_1(t) &= \1_{[-\pi,0]} \cdot \f 1{2\pi} \f {f(t+x) - f(x-0)}t  \f {t}{\sin (\f 12 t)} \in L^1([-\pi,\pi]), \\
			g_2(t) &= \1_{[0,\pi]} \cdot \f 1{2\pi} \f {f(t+x) - f(x+0)}t  \f {t}{\sin (\f 12 t)} \in L^1([-\pi,\pi]).
		\end{align*}

	\end{proof}
\end{st}

\begin{st}[Kriterium von Lipschitz] \label{3.30}
	Sei $f \in C(\R \to \C)$ $2\pi$-periodisch mit
	\[
		\exists \alpha, c > 0 \forall x, x' \in \R : |f(x) - f(x')| \le c | x -x'|^\alpha
	\]
	(d.h. $f$ ist Hölder-stetig auf $\R$).
	Dann konvergiert $(s_n)$ auf $\R$ gleichmäßig gegen $f$.
	\begin{proof}
		\begin{enumerate}[1]
			\item
				Es gelten
				\begin{enumerate}[a)]
					\item
						Für die Bedingungen aus \ref{3.28} gilt $\forall x \in \R : s_n(x) \to f(x)$.
					\item
						$s_n$ ist gleichgradig stetig auf $\R$:
						\[
							\forall \eps > 0 \exists \delta > 0 \forall n \in \N \forall x,x' \in \R : |x-x'|< \delta \implies |s_n(x)-s_n(x')| < \eps.
						\]

						Seien dazu $x, x' \in \R$.
						\begin{align*}
							|s_n(x) - s_n(x)|
							&= \bigg| \int_{-\pi}^\pi f(x+t) D_n(t) \dx[t] - \int_{-\pi}^\pi f(x'+t) D_n(t) \dx[t] \bigg| \\
							&= \Bigg| \int\limits_{[-\pi,-\delta] \cup [\delta, \pi]} \Big( f(x+t) - f(x'+t) \Big) D_n(t) \dx[t]  \\
								&\quad\; + \int_{-\delta}^\delta \Big( f(x+t) -f(x) \Big) D_n(t) \dx[t] 
								+ \underbrace{\int_{-\delta}^\delta \Big(f(x) - f(x')\Big) D_n(t) \dx[t]}_{= f(x)-f(x')} \\
								&\quad\; - \int\limits_{[-\pi,-\delta]}^{[\delta,\pi]} \Big(f(x) - f(x')\Big) D_n(t) \dx[t] 
								 - \int\limits_{-\delta}^\delta \Big(f(x'+t) - f(x') \Big) D_n(t) \dx[t]
								\Bigg| \displaybreak[0] \\
							&\le \int\limits_{[-\pi,-\delta]\cup [\delta, \pi]}  \Big(\underbrace{|f(x+t) - f(x'+t)| + | f(x) -f(x')|}_{\le c |x-x'|^\alpha}\Big) \underbrace{D_n(t)}_{\le \f 1{2\pi \sin(\f \delta 2)}} \dx[t] \\
								&\quad + \int_{-\delta}^\delta \Big(\underbrace{|f(x+t) - f(x)| + |f(x'+t)-f(x')|}_{\le c|t|^\alpha}\Big) D_n(t) \dx[t] \\
								&\quad + \underbrace{|f(x) - f(x')|}_{\le c |x-x'|^\alpha} \displaybreak[0]\\
							&\le c |x-x'|^\alpha \bigg( 1 + \int\limits_{[-\pi,-\delta] \cup [\delta, \pi]} \f 1{2\pi \sin(\f \delta 2)} \dx[t] \bigg)
							+ c \int_{-\delta}^\delta \f {|t|^\alpha}{|\sin(\f t2)|} |\sin(n+\f 12) t | \dx[t] \\
							&\le c |x-x'|^\alpha \bigg( 1 + \int\limits_{[-\pi,-\delta] \cup [\delta, \pi]} \f 1{2\pi \sin(\f \delta 2)} \dx[t] \bigg)
							+ c \int_{-\delta}^\delta \f {|t|^\alpha}{|\sin(\f t2)|} \dx[t]
						\intertext{
						 Da $\int_{-1}^1 \f {|t|^\alpha}{|\sin \f t2|} \dx[t] < \infty$, wählen wir $\delta > 0$, so dass
						 \[
						 	c \int_{-\delta}^\delta \f {|t|^\alpha}{|\sin (\f t2)|} \dx[t] < \eps
						 \]
						 Betrachte
						 \[
							 \underbrace{ c |x-x'|^\alpha \bigg( 1 + \int\limits_{[-\pi,-\delta] \cup [\delta, \pi]} \f 1{2\pi \sin(\f \delta 2)} \dx[t] \bigg)}_{ = c(\delta) |x-x'|^\alpha}
						 \]
						 Wähle zu festem $\delta$ $\xi > 0$ mit $c(\delta) |x-x'|^\alpha < \eps$ für $|x-x'| < \eps$.
						 Damit sind beide Summanden abgeschätzt durch
						}
							&< 2 \eps \qquad \text{für $|x-x'| < \xi$}
						\end{align*}
						unabhängig von $n \in \N$ und der Lage von $x$.
					\item
						$f$ ist stetig.
				\end{enumerate}
			\item
				Sei $\eps > 0$, $x_0 \in \R$ vorgegeben, dann folgt aus obigen drei Aussagen:
				\begin{enumerate}[a)]
					\item
						$|s_n(x_0) - f(x_0) < \eps$ für $n > N_{\eps, x_0}$
					\item
						$|s_n(x) - s_n(x_0)| < \eps$ für $|x-x_0| < \delta', n\in \N$.
					\item
						$f(x) - f(x_0)| < \eps$ für $|x-x_0| < \delta''$.
				\end{enumerate}
				Für $\delta := \min\{\delta', \delta''\}$, $|x-x_0| < \delta$, $n > N_{\eps, x_0}$ folgt aus den obigen drei Aussagen:
				\[
					|f(x) - s_n(x)| 
					\le |f(x) - f(x_0)| + |f(x_0)-s_n(x)| + |s_n(x_0) - s_n(x)|
					< 3\eps.
				\]
			\item
				Zu jedem $x_0 \in \R$ wurde eine Umgebung $I := ]x_0-\delta_{\eps, x_0}, x_0 + \delta_{\eps, x_0}[$ gefunden mit $|f(x) - s_n(x)| < \eps$ für $n > N_{\eps, x_0}$, $x \in  I$.

				Da $[-\pi,\pi]$ kompakt, überdecken endlich viele solcher Intervalle $[-\pi, \pi]$:
				\[
					[-\pi,\pi] \subset \bigcup_{k=1}^K \Big] x_k - \delta_{\eps, x_k}, x_k + \delta_{\eps, x_k} \Big[
				\]
				Für $n > N_\eps := \max\{N_{\eps, x_1}, \dotsc, N_{\eps, x_K}\}$ folgt somit
				\[
					|f(x) - s_n(x)| < 3\eps
				\]
				Da $f, s_n$ $2\pi$-periodisch waren, ist also
				\[
					|f(x) - s_n(x)| < 3\eps \qquad \text{für $n> N_\eps, x\in \R$.}
				\]
		\end{enumerate}
	\end{proof}
\end{st}

\begin{nt} \label{3.31}
	Sei $f \in C^1(\R \to \C)$ $2\pi$-periodisch.
	Dann ist
	\[
		|f(x) - f(x')| = |f'(\xi)||x-x'| \le \max_{[-\pi,\pi]} |f'|\cdot |x-x'|.
	\]
	Nach \ref{3.30} also $s_n \to f$ gleichmäßig auf $\R$.
\end{nt}

\coursetimestamp{3}{6}{2013}
\begin{st}[Komplexe Darstellung von Fourierreihen] \label{3.32}
	\begin{align*}
		s_n(x) 
		&= \int_{-\pi}^{\pi} f(t) \f 1{2\pi} D_n(t) \dx[t] \\
		&= \sum_{j=-n}^n \f 1{2\pi} \int_{-\pi}^\pi f(t) e^{ijt} \dx[t] e^{ijx}
	\end{align*}
	Neue Interpretation mit $e_j(x) := \f 1{\sqrt{2\pi}} e^{ijx}$ ergibt
	\[
		s_n(x) = \sum_{j=-n}^n \<f,e_j\> e_j(x)
	\]
	Es gilt
	\begin{align*}
		\<e_j, e_k\> &= \f 1{2\pi} \int_{-\pi}^\pi e^{ijt}\_{e^{ikt}} \dx[t] \\
		&= \begin{cases}
			1 & j=k \\
			0 & j\neq k
		\end{cases}
	\end{align*}
\end{st}

\begin{st} \label{3.33}
	$(e_j)_{j\in \Z}$ ist VONS in $L^2([-\pi,\pi])$.
	Sei $f \in L^2([-\pi,\pi])$, $\eps > 0$. 
	Wähle $f_\eps \in C^1([-\pi,\pi] \to \C)$ mit $\|f_\eps - f\| < \eps$ (vgl. \ref{3.24}).
	$f_\eps$ kann sogar so gewählt werden, dass $f_\eps(\pm\pi) = 0 = f_\eps'(\pm \pi)$.
	Setze $f_\eps$ $2\pi$-periodisch fort. 
	Dann ist $f_\eps \in C^1([-\pi,\pi] \to \C)$.
	Sei $s_n^{(\eps)} := \sum_{j=-n}^n \<f_\eps, e_j\> e_j$. 

	Dann ist
	\begin{align*}
		\|f - s_n\| 
		&\le \underbrace{\|f - f_\eps\|}_{<\eps} + \|f_\eps - s_n^{(\eps)}\| + \|s_n^{(\eps)} - s_n\| \\
		&\le \eps + \|f_\eps - s_n^{(\eps)}\| + \bigg\| \sum_{j=-n}^n \<f_\eps - f, e_j\> e_j \bigg\|^2 \displaybreak[0]\\
		&\stack{Bessel}\le \eps + \|f_\eps - s_n^{(\eps)}\| + \underbrace{\|f_\eps - f\|}_{<\eps} \\
		&< 2\eps + \sqrt{\int_{-\pi}^\pi \Big|f_\eps} - s_n^{(\eps)}(t)\Big|^2 \dx[t] \\
		&\le 2\eps + \sqrt{\max_{[-\pi,\pi]} \Big|f_\eps} - s_n^{(\eps)}\Big|^2 \cdot 2\pi \\
		&< 3\eps  \qquad \text{für $n > N_\eps$ nach \ref{3.31}} \\
	\end{align*}
	und damit $\|f - s_n\| \to 0$ für $n \to \infty$.
\end{st}

\begin{nt} \label{3.34}
	\begin{enumerate}[1)]
		\item
			Genauso wie in \ref{3.33} lässt sich zeigen, dass
			\[
				(e_j)_{j\in \N} = \Bigg( \f 1{\sqrt{2\pi}}, \; \f 1{\sqrt{\pi}} \sin x,\; \f 1{\sqrt{\pi}} \cos x,\; \dotsc \Bigg)
			\]
			ist VONS in $L^2([-\pi,\pi])$ ($s_n$ ist dieselbe Funktion wie vorher).
		\item
			Zusammenfassung:
			\begin{enumerate}[a)]
				\item
					Für $f \in L^2([-\pi,\pi])$ gilt
					\[
						f = \lim_{n\to \infty} \sum_{i=-n}^n \<f,e_j\> e_j
					\]
					(Konvergenz bezüglich $\|\argdot\|$).
				\item
					Wenn $f$ den Satz von Dini in jedem $x\in [-\pi,\pi]$ erfüllt, gilt
					\[
						f(x) = \lim_{n\to \infty} \sum_{j=-n}^n \<f,e_j\> e_j(x)
					\]
					(punktweise Konvergenz in $\C$).
				\item
					Wenn $f$ die Lipschitz-Bedingung erfüllt, gilt
					\[
						f(x) = \lim_{n\to\infty} \sum_{j=-n}^n \<f,e_j\> e_j(x)
					\]
					gleichmäßig auf $[-\pi,\pi]$.
			\end{enumerate}
	\end{enumerate}
\end{nt}

\begin{nt}[Andere Intervalle] \label{3.35}
	Sei $f \in C(\R \to \C)$ $2L$-periodisch und
	\[
		|f(x) - f(x')| \le c |x-x'|^\alpha \qquad c,\alpha > 0.
	\]
	$g(x) := f(\f L\pi x)$ erfüllt \ref{3.31} (Lipschitz-Bedingung).
	Also
	\begin{align*}
		f(\f L\pi x) = g(x) 
		&= \sum_{j=-\infty}^\infty \f 1{2\pi} \int_{-\pi}^\pi \underbrace{g(t)}_{=f(\f L\pi t)} e^{-ijt} \dx[t] \cdot e^{ijx} \\
		&\stack{s = \f L\pi t}= \sum_{j=-\infty}^\infty \f 1{2\pi} \f \pi L \int_{-L}^L f(s) e^{-ij \f \pi L s} \dx[s] \cdot e^{ijx}
	\end{align*}
	und mit $y = \f L \pi x$
	\[
		f(y) = \sum_{j=-\infty}^\infty \f 1{2L} \int_{-L}^L f(s) e^{-ij\f \pi L s} \dx[s] \cdot e^{ij \f \pi L y}.
	\]
	Setze $\tilde e_j(y) := \f 1{\sqrt{2L}} e^{ij \f \pi L y}$.
	Dann ist $(\tilde e_j)_{j\in \Z}$ VONS in $L^2([-L,L])$.
	Mit $\tilde e_j$ anstelle von $e_j$, gilt alles wie vorher im Intervall $[-L,L]$ bzw. für $2L$-periodische Funktionen.
\end{nt}

\begin{ex} \label{3.36}
	Gegeben $a,b \in \R, a,b \neq 0, f \in [-L,L] \to \C$.

	Gesucht $y \in C^2([-L,L] \to \C)$ als Lösung von 
	\begin{gather*}
		ay'' + by' + y = f \qquad \text{ in $[-L,L]$} \\
		y(-L) = y(L) \quad \land \quad y'(-L) = y'(L) \quad \text{(periodische Randbedingung)}
	\end{gather*}
	Zunächst formale Rechnung: Vertausche Ableitung und Reihe, Konvergenzüberlegungen erst später.

	Seien
	\begin{align*}
		f(x) &= \sum_{j=-\infty}^\infty \<f,\tilde e_j\> \tilde e_j(x) \\
		y(x) &= \sum_{j=-\infty}^\infty y_j \tilde e_j(x)
	\end{align*}
	Ziel der formalen Rechnung ist die Bestimmung von $(y_j)$.
	\begin{align*}
		y'(x) = \sum_{j=-\infty}^\infty y_j \tilde e_j'(x) = \sum_{j=-\infty} y_j ij \f {\pi}L \tilde e_j(x) \\
		y''(x) = \sum_{j=-\infty}^\infty y_j \tilde e_j''(x) = -\sum_{j=-\infty} y_j \Big( j \f {\pi}L \Big)^2 \tilde e_j(x) 
	\end{align*}
	Eingesetzt in die DGL ergibt sich
	\begin{align*}
		\sum_{j=-\infty}^\infty y_j\Big( -a (\tf {j\pi}L)^2 + b ij \tf \pi L + 1 \Big) \tilde e_j(x)
		&\stack != \sum_{j=-\infty}^\infty \<f, \tilde e_j\> \tilde e_j(x) \\
		\iff \qquad y_j &= \dfrac 1{-a (\tf \pi L)^2 + b ij \tf \pi L + 1} \<f,\tilde e_j\>
		\qquad \text{(Fourierkoeff. eindeutig)}
	\end{align*}
	Es ergibt sich als Kandidaten für die Lösung also
	\[
		y(x) = \sum_{j=-\infty}^\infty \f 1{-a (\tf {j\pi}L )^2 + b ij \f \pi L + 1} \<f, \tilde e_j\> \tilde e_j(x)
	\]
	Falls Reihe gleichmäßig konvergent und erste und zweite Ableitung mit Reihe vertauschbar sind, dann ist $y \in C^2([-L,L]\to \C)$ und $y$ ist Lösung.

	Setze voraus: $f \in C^1(\R \to \C)$ $2\pi$-periodisch.
	Dann gilt
	\begin{enumerate}[1)]
		\item
			Nach \ref{3.31}, \ref{3.30} ist für $[-L,L]$
			\[
				f(x) = \sum_{j=-\infty}^\infty \<f,\tilde e_j\> \tilde e_j(x)
			\]
			gleichmäßig konvergent auf $\R$.
		\item
			Mit partieller Integration gilt, da $f'$ stetig:
			\[
				|\<f, \tilde e_j\>| \le \f c j \qquad c \in \R.
			\]
		\item
			\begin{enumerate}[a)]
				\item
					$y(x) = \sum_{j=-\infty}^\infty y_j \tilde e_j(x)$ ist nach Weierstraß gleichmäßig konvergent auf $\R$, da 
					\[
						|y_j \tilde e_j(x)| \le \f {c_1}{j^2} |\<f, \tilde e_j\>| |\tilde e_j(x)| \le \f {c_2}{j^3}
					\]
					 und $\sum_{j=0}^\infty \f 1{j^3} < \infty$.
				\item
					$y'(x) = \sum_{j=-\infty}^\infty y_j \tilde e_j'(x)$ ist gleichmäßig konvergent.

					\begin{align*}
						|y_j \tilde e_j'(x)| \le \f {c_1}{j^2} |\<f,\tilde e_j\>| \underbrace{|\tilde e_j'(x)|}_{\le \f j {\sqrt{2L}}} \le \f {c^2}{j^2}
					\end{align*}
				\item
					Die Reihe $y''(x) = \sum_{j=-\infty}^\infty y_j \tilde e_j''(x)$ konvergiert gleichmäßig:

					\begin{align*}
						y_j e_j''(x)
						= \f 1{a} \Big(\underbrace{\<f, \tilde e_j\> \tilde e_j(x)}_{\text{glm. konv. nach 1)}} - \underbrace{b y_j \tilde e_j'(x)}_{\text{glm. konv. nach 3b)}} - \underbrace{y_j \tilde e_j(x)}_{\text{glm. konv. nach 3a)}} \Big)
					\end{align*}
			\end{enumerate}
	\end{enumerate}

	Sichtwechsel: Da die Reihe für $y$ gleichmäßig konvergent ist, gilt
	\begin{align*}
		y(x) = \int_{-\pi}^\pi f(t) \underbrace{\sum_{j=-\infty}^\infty y_j e^{\f \pi L ij(x-t)}}_{=G(x-t)} \dx[t].
	\end{align*}
	$G(x-t)$ entspricht der Greenschen Funktion.
\end{ex}

