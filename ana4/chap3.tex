% This work is licensed under the Creative Commons
% Attribution-NonCommercial-ShareAlike 3.0 Unported License. To view a copy of
% this license, visit http://creativecommons.org/licenses/by-nc-sa/3.0/ or send
% a letter to Creative Commons, 444 Castro Street, Suite 900, Mountain View,
% California, 94041, USA.

\chapter{Fourierreihen II}

\section{Kompakte symmetrische Operatoren}


\begin{df} \label{3.1}
	Sei $M \subset L$ ($L$ Prä-Hilbertraum).
	Dann heißt
	\[
		M^\orth := \Big\{ x \in L \mid| \forall y \in M : \<x,y\> = 0 \Big\}
	\]
	\emph{orthogonales Komplement} von $M$.
\end{df}

\begin{st} \label{3.2}
	\begin{enumerate}[1)]
		\item
			$M^\orth$ ist ein linearer Teilraum und abgeschlossen (d.h. enthält alle Häufungspunkte).
		\item
			\begin{align*}
				(\LH \{M\})^\orth &= M\orth \\
				\_M^\orth = M^\orth
			\end{align*}
	\end{enumerate}
	\begin{proof}
		Einfache Übung.
	\end{proof}
\end{st}

\begin{conv*}
	Wir betrachten hier und im Folgenden nur \emph{lineare} Operatoren.
\end{conv*}

\begin{st} \label{3.3}
	Sei $A: L \to L$ symmerisch und beschränkt.
	Dann gilt
	\begin{enumerate}[1)]
		\item
			$\displaystyle \<Ax,x\> \in \R$,
		\item
			$\displaystyle \im(A)^\orth = \ker(A)$,
		\item
			$\displaystyle A \Big|_{\_{\im (A)} }$ ist injektiv und $A \Big|_{\ker(A)} = 0$,
		\item
			$\displaystyle \|A\| = \sup_{\|x\|=1} |\<Ax,x\>|$.
	\end{enumerate}
	\begin{proof}
		\begin{enumerate}[1)]
			\item
				Da $A$ symmetrisch, gilt
				\[
					\<Ax, x\> = \<x, Ax\> = \_{\<Ax,x\>}.
				\]
				Also $\<Ax, x\> \in \R$.
			\item
				Da $A$ symmetrisch, gilt
				\begin{align*}
					x \in \im(A)^\orth
					& \iff \forall y \in \im(A) : \<x,y\> = 0 \\
					& \iff \forall \tilde y \in L: \<x, A\tilde y\> = 0 \\
					& \iff \forall \tilde y \in L: \<Ax,\tilde y\> = 0 \\
					& \iff Ax = 0 \\
					& \iff x \in \ker(A)
				\end{align*}
			\item
				Seien $x,y \in \_{\im(A)}$, also auch $x-y \in \_{\im(A)}$ (Betrachte dazu $x,y$ als Folgengrenzwerte).

				Sei $Ax = Ay$.
				Dann ist $A(x-y) = 0$, also nach 2)
				\[
					x-y \in \ker (A) = \im(A)^\orth = \_{\im(A)}^\orth
				\]
				Also 
				\[
					x-y \in \_{\im(A)} \cap \_{\im(A)}^\orth = \{0\}.
				\]
				und damit $A$ injektiv.
			\item
				Falls $\|A\| = 0$, also $A=0$, dann ist $0 = \sup |\<Ax,x\>|$ trivial.

				Sei nun $\|A\| > 0$ und $d := \sup_{\|x\|=1} |\<Ax,x\>|$.
				Zeige $d = \|A\|$:
				\begin{enumerate}[a)]
					\item
						Es gilt $d \le \|A\|$, denn
						\begin{align*}
							|\<Ax,x\>| 
							&\stack{\text{CSB}} \le \|Ax\| \underbrace{\|x\|}_{=1} \\
							&\stack{\ref{1.23}}\le \|A\| \|x\|
							= \|A\|.
						\end{align*}
					\item
						Zeige $d \ge \|A\|$.

						Für $y \neq 0$ gilt
						\[
							|\<Ay,y\>| 
							= \|y\|^2 \Big| \< A \tf{y}{\|y\|}, \tf{y}{\|y\|} \> \Big| 
							\le d\|y\|^2
						\]

						Für $\alpha > 0$ gilt
						\begin{align*}
							\Big\< A(\alpha x + \tf 1\alpha Ax), \alpha x + \f 1\alpha A x \Big\>
							- \Big\< A(\alpha x - \tf 1\alpha Ax), \alpha x - \f 1\alpha A x \Big\>
							&= 2\< A\alpha x, \tf 1\alpha A x\> + 2\<A (\tf 1\alpha Ax), \alpha x\> \\
							&= 4 \|Ax\|^2
						\end{align*}
						Falls $\|Ax\| = 0$, dann ist offensichtlich $\|Ax\| \le d$.
						Sei also $\|x\| = 1$ mit $\|Ax\| \neq 0$.
						Setze $\alpha^2 := \|Ax\|$, dann ist
						\begin{align*}
							4\|Ax\|^2
							&= 
							\bigg| \Big\< A(\alpha x + \tf 1\alpha Ax), \alpha x + \tf 1\alpha A x \Big\>
							- \Big\< A(\alpha x - \tf 1\alpha Ax), \alpha x - \tf 1\alpha A x \Big\> \bigg| \\
							&\le \bigg| \Big\< A(\alpha x + \tf 1\alpha Ax), \alpha x + \tf 1\alpha A x \Big\>
							\bigg| + \bigg| \Big\< A(\alpha x - \tf 1\alpha Ax), \alpha x - \tf 1\alpha A x \Big\> \bigg| \\
							&\le  d \Big\| \alpha x + \tf 1\alpha Ax \Big\|^2 + d \Big\| \alpha x - \tf 1\alpha Ax \Big\|^2 \\
							&= d ( \|\alpha x\|^2 + 2 \Re \<\alpha x, \tf 1\alpha Ax\> + \| \tf 1\alpha Ax\|^2
							+  \|\alpha x\|^2 - 2 \Re \<\alpha x, \tf 1\alpha Ax\> + \| \tf 1\alpha Ax\|^2) \\
							&= 2d (\|\alpha x\|^2 + \|\tf 1\alpha Ax\|^2) \\
							&= 2d (\|A x\|\|x\|^2 + \f 1{\|Ax\|}\|Ax\|^2) \\
							&= 4d \|Ax\|
						\end{align*}
						Also $\|Ax\| \le d$ für $\|x\| = 1$ und damit
						\[
							\|A\| = \sup_{\|x\|=1} \|Ax\| \le d
						\]
				\end{enumerate}
		\end{enumerate}
	\end{proof}
\end{st}

\begin{nt} \label{3.4}
	Ist zusätzlich zu \ref{3.3} $(L,\<\argdot,\argdot\>)$ ein Hilbertraum, dann kann 2) in der Form
	\[
		L = \_{\im(A)} \oplus \ker(A)
	\]
	als direkte Summe geschrieben werden.
	Dies bedeutet
	\[
		\forall x \in L \exists! y \in \_{\im(A)} \exists! z \in \ker(A) : x = y + z
	\]
	\begin{proof}
		Folgt aus dem Projektionssatz (siehe Funktionalanalysis).
	\end{proof}
\end{nt}


\begin{st}[Hauptsatz über symmetrische, kompakte Operatoren] \label{3.5}
	Sei $(L, \<\argdot,\argdot\>)$ unendlichdimensionaler Prähilbertraum, $A: L \to L$ linear, symmetrisch und kompakt.

	Dann gelten
	\begin{enumerate}[1)]
		\item
			$\lambda = \|A \|$ oder $\lambda = - \|A\|$ ist Eigenwert von $A$.
		\item
			Jeder Eigenwert $\lambda \neq 0$ ist reell und hat endliche Vielfachheit.
		\item
			Entweder $A$ hat nur endlich viele Eigenvektoren.
			Dann ist $\lambda = 0$ Eigenwert mit unendlicher Vielfachheit (d.h. $\dim (\ker (A)) = \infty)$.

			Oder $A$ hat unendlich viele Eigenwerte.
			Dann ist die Menge der Eigenwerte abzählbar und für die Folge $(\lambda_j)$ der Eigenwerte gilt $\lambda_j \to 0$ ($j \to \infty$).
		\item
			Sei $(\lambda_j)$ eine Abzählung der Eigenwerte, die so gebildet wird, dass
			\begin{itemize}
				\item
					$|\lambda_j|$ ist monoton fallend.
				\item
					Jeder Eigenwert $\lambda$ kommt in der Folge so oft vor, wie es seiner Vielfachheit entspricht.
			\end{itemize}
			Zu dieser Folge existiert ein ONS $(e_j)$ aus Eigenelementen.
			Dieses ONS ist vollständig in $\_{\im (A)}$, d.h.
			\[
				\forall x \in \_{\im(A)} : x = \sum_{j=1}^\infty \<x,e_j\> e_j
			\]
	\end{enumerate}
	\begin{proof}
		Falls $A = 0$ ist nichts zu beweisen.
		Sei also $A \neq 0$ und damit insbesondere $\|A\| > 0$.
		\begin{enumerate}[1)]
			\item
				Wegen \ref{3.3} 4) existiert eine Folge $(x_n)$ mit
				\[
					\|x_n\| = 1 \qquad \land \qquad |\<Ax_n,x_n\>| \to \|A\|
				\]
				Eventuell hat es Teilfolgen von $(x_n)$, die gegen $\pm \|A\|$ konvergieren.

				Da $A$ kompakt existiert eine Teilfolge  $Ax_{n_k} \to y \in L$ ($k \to \infty$).
				Es gilt
				\begin{align*}
					0 
					&\le \|Ax_{n_k} - \lambda x_{n_k}\|^2 \\
					&= \underbrace{\|Ax_{n_k} \|^2}_{\le \|A\|^2 \|x_{n_k}\|^2} - 2 \Re \underbrace{\< Ax_{n_k}, \lambda x_{n_k} \>}_{\in \R} + \lambda^2 \underbrace{\|x_{n_k}\|^2}_{=1} \\
					&\le \underbrace{\|A\|^2 + \lambda^2}_{=2\lambda^2} - 2 \lambda \underbrace{\<Ax_{n_k}, x_{n_k}\>}_{\to \lambda} \\
					& \to 0
				\end{align*}
				Also $\lambda x_{n_k} \to y$, bzw. $x_{n_k} \to \f 1\lambda y$.
				Da $A$ stetig (da beschränkt, siehe \ref{1.27} und \ref{1.25}) gilt $Ax_{n_k} \to \f 1\lambda A y$.

				Wegen $Ax_{n_k} \to y$ ist folglich $Ay = \lambda y$.
				Aus $\|\lambda x_{n_k}\| = |\lambda | \|x_{n_k}\| = |\lambda| \neq 0$ und $\|\lambda x_{n_k}\| \to \|y\|
				$ folgt $y\neq 0$.

				Also ist $\lambda$ Eigenwert und $\lambda = \|A\|$ oder $\lambda = - \|A\|$. 
		\end{enumerate}
	\end{proof}
\end{st}
