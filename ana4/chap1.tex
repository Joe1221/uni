% This work is licensed under the Creative Commons
% Attribution-NonCommercial-ShareAlike 3.0 Unported License. To view a copy of
% this license, visit http://creativecommons.org/licenses/by-nc-sa/3.0/ or send
% a letter to Creative Commons, 444 Castro Street, Suite 900, Mountain View,
% California, 94041, USA.

\chapter{Fourierreihen}
\coursetimestamp{8}{4}{2013}

\section{Orthonormalsysteme}


\begin{df} \label{1.1}
	Sei $L$ ein linearer Raum über $\C$.
	Eine Abbildung $\<\argdot, \argdot\>: L \times L \to \C$ heißt \emph{Skalarprodukt}, falls
	\begin{enumerate}[(S1)]
		\item
			$\<\alpha x_1 + \beta x_2, y \> = \alpha \<x_1,y\> + \beta \<x_2, y\>$,
		\item
			$\<y,x\> = \_{\<x,y\>}$,
		\item
			$\forall x\in L\setminus \{0\} : \<x,x\> > 0$.
	\end{enumerate}
\end{df}

\begin{nt} \label{1.2}
	\begin{enumerate}[1)]
		\item
			Aus (S2) folgt $\<x,x\> \in \R$.
		\item
			Aus (S1) und (S2) folgt
			\[
				\<x, \alpha y_1 + \beta y_2\> = \_\alpha \<x,y_1\> + \_\beta \<x,y_2\>.
			\]
		\item
			Aus (S1) folgt $\<0,0\> = 0$ und mit (S3)
			\[
				\<x,x\> = 0 \quad\iff\quad x=0.
			\]
	\end{enumerate}
\end{nt}

\begin{st} \label{1.3}
	Durch
	\[
		\|x\| := \sqrt{\<x,x\>}
	\]
	wird auf $L$ eine Norm definiert, die \emph{induzierte Norm}.
	Es gilt die sogenante \emph{Cauchy"=Schwarz"=Bunjakowski"=Ungleichung} (CSB)
	\[
		| \<x,y\> | \le \|x\| \cdot \|y\|,
		\qquad x,y \in L.
	\]
	\begin{proof}
		Die Positivität folgt aus \ref{1.1} (S3) und \ref{1.2}.
		Die Homogenität $\|\alpha x\| = |\alpha| \|x\|$ ist eine leichte Übung.

		Zeige für die Dreiecksungleichung zunächst die CSB. Für reelles $c$ gilt:
		\begin{align*}
			0 &\le \Big\< x + c\<x,y\> y, x + c \<x,y\> y\Big\> \\
			&= \|x\|^2 + \underbrace{c\_{\<x,y\>}\<x,y\> + c \<x,y\>\<y,x\>}_{= 2|\<x,y\>|^2 c} + c^2\<x,y\>\_{\<x,y\>} \|y\|^2 \\
			&= \|x\|^2 + 2|\<x,y\>|^2 c + \|y\|^2 |\<x,y\>|^2 c^2
			=: p(c).
		\end{align*}
		Wegen $p(c) \ge 0$ für alle $c\in \R$, gilt für die Diskriminante
		\begin{align*}
			0 \ge D
			&= \tilde b^2 - 4\tilde a \tilde c \\
			&= 4 |\<x,y\>|^4 - 4 \|y\|^2 |\<x,y\>|^2 \|x\|^2,
		\end{align*}
		also
		\[
			|\<x,y\>|^4 \le |\<x,y\>|^2 \|x\|^2 \|y\|^2.
		\]
		Daraus folgt die CSB.

		Zeige jetzt die Dreiecksungleichung. Es gilt
		\begin{align*}
			\|x+y\|^2 = \<x+y, x+y\>
			&= \|x\|^2 + \underbrace{\<x,y\> + \<y,x\>}_{= 2\Re \<x,y\>} + \|y\|^2 \\
			&\le \|x\|^2 + 2 |\<x,y\>|  + \|y\|^2 \\
			&\le \|x\|^2 + 2\|x\|\|y\| + \|y\|^2 \\
			&\le (\|x\| + \|y\|)^2.
		\end{align*}
	\end{proof}
\end{st}

\begin{kor}[Stetigkeit des Skalarprodukts] \label{1.4}
	Das Skalarprodukt ist bezüglich der induzierten Norm in beiden Argumenten stetig, d.h.
	\[
		x_n \to x \quad\land\quad y_n \to y
		\qquad \implies \qquad
		\<x_n, y_n\> \to \<x,y\>
	\]
	oder
	\[
		\lim_{n\to \infty} \<x_n, y_n\> = \l\<\lim_{n\to \infty} x_n, \lim_{n\to\infty} y_n \r\>.
	\]
	\begin{proof}
	Seien $ (x_n) $ und $ (y_n) $ konvergente Folgen, sodass $ x_n \to x, y_n \to y $. Dann folgt
		\begin{align*}
			|\<x_n,y_n\> - \<x,y\>|
			&= |\<x_n, y_n - y\> - \<x-x_n,y\>| \\
			&\le |\<x_n,y_n -y\>| + |\<x-x_n,y\>| \\
			&\stack{CSB}{\le} \|x_n\| \underbrace{\|y_n -y\|}_{\to 0} + \underbrace{\|x-x_n\|}_{\to 0} \underbrace{\|y\|}_{=\const}.
			\end{align*}
		Da $(x_n)$ konvergent in $L$, ist $x_n$ beschränkt (Beweis dazu analog wie in $\C$) und der obige Ausdruck konvergiert gegen $ 0 $ für $ n\to \infty $.
	\end{proof}
\end{kor}

\begin{df}[Hilbertraum] \label{1.5}
	$(L, \<\argdot, \argdot\>)$ heißt \emph{Hilbertraum}, wenn $L$ bezüglich der induzierten Norm vollständig ist, d.h. jede Cauchy-Folge konvergiert.
\end{df}

\begin{ex} \label{1.6}
	\begin{enumerate}[1)]
		\item
			Setze
			\[
				L := \bigg\{(x_n)_{n\in\N} \text{ Folge in } \C : \sum_{n=1}^\infty |x_n|^2 < \infty \bigg\},
			\]
			mit dem Skalarprodukt
			\[
				\big\<(x_n),(y_n)\big\> := \sum_{n=1}^\infty x_n\_{y_n}.
			\]
			Das Skalarprodut ist wohldefiniert, da die Reihe Grenzwert einer Cauchy-Folge in $\C$ ist und somit konvergiert:
			\begin{align*}
				\bigg| \sum_{j=n}^m x_j\_{y_j} \bigg|
				&=\bigg| \Big\<(x_n,\dotsc,x_m), (y_n,\dotsc,y_m)\Big\>_{\C^{m-n+1}} \bigg|
			\intertext{
				Mit der CSB in $\C^{m-n+1}$ und da $\sum_{n=1}^\infty |x_n|^2 < \infty$ für alle $(x_n) \in L$ gilt
			}
				&\le \bigg( \underbrace{\sum_{j=n}^m |x_j|^2}_{<\eps \text{ für $m\ge n\ge N_\eps$}} \bigg)^{\f 12} \bigg( \underbrace{\sum_{j=n}^m |y_j|^2}_{<\eps\text{ für $m\ge n\ge \tilde N_\eps$}} \bigg)^{\f 12}
				<\eps \qquad \text{für $m\ge n \ge \max\{N_\eps, \tilde N_\eps\}$}.
			\end{align*}
			Die Gültigkeit von (S1) bis (S3) aus \ref{1.1} ist eine leichte Übung.

			$(L, \<\argdot,\argdot\>)$ ist ein Hilbertraum (Beweis: \coursehref{blatt01.pdf}{Übungsaufgabe 1.2a}).
			Man nennt ihn $\ell^2$, je nach Kontext mit oder ohne dem Skalarprodukt: $\ell^2 := L$ oder $\ell^2 := (L,\<\argdot,\argdot\>)$.
		\item
			Setze $L := C([a,b] \to \C)$ mit Skalarprodukt
			\[
				\<f,g\> := \int_a^b f(x)\_{g(x)} \dx.
			\]
			Für die Gültigkeit von (S1) bis (S3) aus \ref{1.1}, siehe \coursehref{blatt01.pdf}{Übungsaufgabe 1.3a}.

			Allerdings ist $(L,\<\argdot,\argdot\>)$ kein Hilbertraum (Gegenbeispiel \coursehref{blatt01.pdf}{Übungsaufgabe 1.3b}).

			Man erweitert $L$ deshalb zu
			\begin{align*}
				\tilde L &:= L^2(]a,b[) \\
				\<f,g\>^{\sim} &:= \int_{]a,b[} f\_{g} \dx[\my]. %sieht zwar komisch aus ist aber richtig so.
			\end{align*}
			$(\tilde L, \<\argdot,\argdot\>^\sim)$ ist ein Hilbertraum.

			Es gelten
			\begin{itemize}
				\item
					$L \subset \tilde L$;
				\item
					$\<f,g\>^\sim = \<f,g\>$ für $f,g \in L$;
				\item
					$L$ ist dicht in $\tilde L$.
			\end{itemize}
			Siehe auch \ref{2.10}.
	\end{enumerate}
\end{ex}

\begin{df}[Prähilbertraum] \label{1.7}
	Ein linearer Raum mit Skalarprodukt heißt \emph{Prähilbertraum}.
\end{df}

\begin{st} \label{1.8}
	Zu jedem \emph{Prähilbertraum} $(L,\<\argdot,\argdot\>_L)$ existiert ein bis auf Isomorphie eindeutiger Hilbertraum $(H,\<\argdot,\argdot\>_H)$ mit
	\begin{itemize}
		\item
			$L \subset H$;
		\item
			$\<\argdot,\argdot\>_H = \<\argdot,\argdot\>_L$ auf $L\times L$;
		\item
			$L$ ist dicht in $H$.
	\end{itemize}
	$(H,\<\argdot,\argdot\>_H)$ heißt \emph{Vervollständigung} von $L$.
	\begin{note}
		Die Isomorphie bedeutet in diesem Fall die Existenz einer linearen, bijektiven Funktion $\Phi$
		\[
			\Phi: \Big(H,\<\argdot,\argdot\>_H\Big) \to \Big(\tilde H,\<\argdot, \argdot\>_{\tilde H}\Big)
		\]
		mit
		\[
			\Big\<\Phi(f),\Phi(g)\Big\>_{\tilde H} = \<f,g\>_H.
		\]
	\end{note}
	\begin{proof}
		Genauso wie die Erweiterung von $\Q$ zu $\R$.
	\end{proof}
\end{st}

\begin{df} \label{1.9}
	Eine Folge $(e_j)_{j\in\N}$ im Prähilbertraum $(L,\<\argdot,\argdot\>)$ heißt \emph{Orthonormalsystem}, falls
	\[
		\<e_j,e_k\> = \delta_{jk}.
	\]
	Insbesondere ist damit $\|e_j\| = 1$.
\end{df}

\begin{conv} \label{1.10}
	Im Folgenden bezeichne $ (L, \<\argdot, \argdot \>) $ immer einen Prähilbertraum und $ \|\argdot \| $ seine induzierte Norm.
\end{conv}

\begin{st} \label{1.11}
	Sei $(e_j)$ ein Orthonormalsystem in $(L, \<\argdot, \argdot\>)$ und $x \in L$.
	Dann gelten:
	\begin{enumerate}[1)]
		\item
			Falls
			\[
				x = \sum_{j=1}^\infty x_j e_j
			\]
			mit $x_j \in \C$, so ist
			\[
				x_j = \<x,e_j\>.
			\]
		\item
			$\{e_1,e_2, \dotsc \}$ ist linear unäbhängig, d.h. jede endliche Teilmenge ist linear unabhängig.
	\end{enumerate}
	\begin{proof}
		\begin{enumerate}[1)]
			\item
			Wegen der Stetigkeit des Skalarprodukts gilt:
				\begin{align*}
					\<x,e_k\>
					= \bigg\< \sum_{j=1}^\infty x_j e_j, e_k \bigg\>
					= \sum_{j=1}^\infty \underbrace{\<x_j e_j, e_k\>}_{= 0 \text{ für $j\neq k$}}
					= x_k \<e_k, e_k\>
					= x_k.
				\end{align*}
			\item
				Sei $0 = \sum_{j=1}^N \alpha_j e_j = \sum_{j=1}^\infty \alpha_j e_j$ ($\alpha_j = 0$ für $j > N$).
				Nach 1) gilt für $j \in \N$
				\[
					\alpha_j = \<0,e_j\> = 0,
				\]
				also ist $\{e_j\}_{j\in \N}$ linear unabhängig.
		\end{enumerate}
	\end{proof}
\end{st}

\begin{st} \label{1.12}
	Sei $(e_j)$ ein Orthonormalsystem in $(L,\<\argdot,\argdot\>)$ und $x\in L$.
	Dann gilt:
	\begin{enumerate}[1)]
		\item
			die \emph{Besselsche Ungleichung}:
			\[
				\sum_{j=1}^\infty |\<x,e_j\>|^2 \le \|x\|^2.
			\]
		\item
			die \emph{Parsevalsche Gleichung}:
			\[
				\sum_{j=1}^\infty |\<x,e_j\>|^2 = \|x\|^2
			\]
			genau dann, wenn
			\[
				x = \sum_{j=1}^\infty \<x,e_j\> e_j.
			\]
			\begin{note}
				Sie gilt also insbesondere genau dann, wenn $(e_j)$ ein \emph{vollständiges} ONS ist (siehe \ref{1.14} 2).
			\end{note}
		\item
			Die Menge
			\[
				M := \bigg\{ x\in L : x = \sum_{j=1}^\infty \<x,e_j\> e_j \bigg\}
			\]
			ist abgeschlossen in $L$.
		\item
			Ist $(L, \<\argdot,\argdot\>)$ zusätzlich Hilbertraum, so ist $\sum_{j=1}^\infty \<x,e_j\> e_j$ konvergent (aber nicht unbedingt gegen $x$).
	\end{enumerate}
\coursetimestamp{10}{4}{2013}
	\begin{proof}
		\begin{enumerate}[1)]
			\item Mit der Positivität von $ \|\argdot \| $ folgt
				\begin{align*}
					0 &\le \Big\|x - \sum_{j=1}^N \<x,e_j\> e_j \Big\|^2
					= \< \dotsc, \dotsc \> \\
					&= \|x\|^2 - \underbrace{\Big\<x, \sum_{j=1}^N \<x,e_j\> e_j\Big\>}_{= \sum_{j=1}^N |\<x,e_j\>|^2} - \underbrace{\Big\<\sum_{j=1}^N \<x,e_j\> e_j,x\Big\>}_{= \sum_{j=1}^N |\<x,e_j\>|^2} + \underbrace{\Big\<\sum_{j=1}^N\<x,e_j\> e_j, \sum_{k=1}^N \<x,e_k\> e_k\Big\> }_{= \sum_{j,k=1} \<x,e_j\>\_{\<x,e_k\>}\<e_j,e_k\>
					= \sum_{j=1}^N |\<x,e_j\>|^2} \\
					&= \|x\|^2 - \sum_{j=1}^N |\<x,e_j\>|^2,
				\end{align*}
				also im Grenzwert $N \to \infty$
				\[
					\sum_{j=1}^\infty |\<x,e_j\>|^2 \le \|x\|^2.
				\]
			\item Insbesondere folgt aus dem Beweis von 1)
				\[
					\Big\|x - \sum_{j=1}^N \<x,e_j\> e_j \Big\|^2=\|x\|^2 - \sum_{j=1}^N |\<x,e_j\>|^2.
				\]
				$\|x\|^2 = \sum_{j=1}^\infty |\<x,e_j\>|^2$ impliziert damit für $N \to \infty$ sofort $x = \sum_{j=1}^\infty \<x,e_j\> e_j$, analog für die Rückrichtung.
			\item
				Sei $x \in \_M$, zeige $x \in M$.
				Wähle $(x_n)$ als Folge in $M$ mit $x_n \to x$.
				Zu $\eps > 0$ wähle $N\in \N$ mit $\|x_N-x\| < \eps$.
				Wähle außerdem $M \in \N$ mit
				\[
					\Big\|x_N- \sum_{j=1}^m \<x_N,e_j\> e_j \Big\| < \eps
					\qquad \forall m > M.
				\]
				Dann gilt für alle $m > M$
				\begin{align*}
					\Big\|x_N- \sum_{j=1}^m \<x,e_j\> e_j \Big\|
					&\le \|x-x_N\| + \Big\|x_N - \sum_{j=1}^m \<x_N,e_j\> e_j \Big\| + \Big\| \sum_{j=1}^m \<x_N-x,e_j\> e_j \Big\|	\\
					&\le \eps + \eps + \bigg(\sum_{j=1}^m |\<x_N-x,e_j\>|^2 \bigg)^{\f 12} \\
					&\le 2\eps + \bigg(\sum_{j=1}^\infty |\<x_N-x,e_j\>|^2 \bigg)^{\f 12} \\
					&\stack{1)}\le 2 + \eps\|x_N-x\|
					< 3 \eps.
				\end{align*}
				Für $m \to \infty$ also
				\[
					\|x_N - \sum_{j=1}^\infty \<x,e_j\>e_j \| \le 3\eps.
				\]
				Für $\eps \to 0$ gilt $x_N \to x$ und es folgt die Behauptung.
			\item
				Setze $y := \sum_{j=n}^m \<x,e_j\>$, dann ist $\<x,e_j\> = \<y,e_j\>$ und es gilt:
				\begin{align*}
					\|y\|^2
					\stack{2)}= \sum_{j=n}^m |\<y,e_j\>|^2
					= \sum_{j=n}^m |\<x,e_j\>|^2
					\stack{1)}< \eps
				\end{align*}
				für $m\ge n \ge N_\eps$, also ist $\sum_{j=1}^\infty \<x,e_j\> e_j$ Cauchyfolge und konvergent, da $H$ vollständig.
		\end{enumerate}
	\end{proof}
\end{st}

\begin{ex} \label{1.13}
	Sei $L = C([-\pi,\pi] \to \C)$ mit Skalarprodukt (siehe \coursehref{blatt01.pdf}{Übungsaufgabe 1.3a})
	\[
		\<f,g\> := \int_{-\pi}^{\pi}f(x)\_{g(x)} \dx
	\]
	und Orthonormalsystem (siehe \coursehref{blatt01.pdf}{Übungsaufgabe 2.1a})
	\[
		e_j : x \mapsto \f 1{\sqrt{\pi}} \sin(jx).
	\]
	Nach \ref{1.12} 3) ist die Menge
	\[
		\bigg\{ f \in L : f = \sum_{j=1}^\infty \<f,e_j\> \f 1{\sqrt{\pi}}\sin(j \argdot) \bigg\}
	\]
	abgeschlossen in $L$.

	Sei jetzt
	\[
		L^2(]-\pi,\pi[) := \bigg\{ f: ]-\pi,\pi[ \to \C : f \text{ messbar } \land \int_{]-\pi,\pi[}|f|^2 \dx[\my] < \infty \bigg\}
	\]
	mit gleichem ONS $(e_j)$ und analogem Skalarprodukt wie für $L$ (Elemente aus $L$ werden als gleich angesehen, wenn sie $\my$-fast-überall übereinstimmen, siehe dazu auch \ref{2.10}).
	$L^2$ ist ein Hilbertraum (die Vollständigkeit bezüglich der induzierten Norm wird später in \ref{2.12} gezeigt).
	Damit ist nach \ref{1.12} 4) für alle $f \in L^2(]-\pi,\pi[)$ die Reihe
	\[
		\sum_{j=1}^\infty \<f,e_j\> \f 1{\sqrt{\pi}} \sin(j \argdot)
	\]
	konvergent in $L^2(]-\pi,\pi[)$ (d.h. bezüglich der induzierten Norm), aber nicht unbedingt punktweise gegen $f$.
\end{ex}

\begin{df} \label{1.14}
	Sei $(e_j)$ ein ONS in $(L, \<\argdot,\argdot\>)$.
	\begin{enumerate}[1)]
		\item
			Für $x\in L$ heißt
			\[
				\sum_{j=1}^\infty \<x,e_j\> e_j
			\]
			die \emph{Fourierreihe} von $x$.
			Die Koeffizienten $\<x,e_j\>$ heißen \emph{Fourierkoeffizienten}.
		\item
			$(e_j)$ heißt \emph{vollständiges Orthonormalsystem} (VONS), falls für alle $ x\in L $
			\[
				x = \sum_{j=1}^\infty \<x,e_j\> e_j
			\]
			gilt.
	\end{enumerate}
\end{df}

\begin{nt} \label{1.15}
	Sei $(e_j)$ ein ONS in $(L, \<\argdot,\argdot\>)$ und
	\[
		\LH\big(\{e_1,e_2,\dotsc\}\big) = \bigg\{ \sum_{j=1}^N \alpha_j e_j : N \in \N, \alpha_j \in \C \bigg\}
	\]
	dicht in $L$.

	Dann ist $(e_j)$ ein VONS.
	\begin{proof}
		Betrachte $M$ aus \ref{1.12} 3), zeige $L \subset M$ ($M \subset L$ ist klar).

		Wegen $\LH(\{e_j\}_{j\in \N}) \subset M$ gilt auch
		\[
			L = \_{\LH(\{e_j\}_{j\in \N})} \subset \_{M} = M
		\]
	\end{proof}
\end{nt}

\begin{st} \label{1.16}
	Alle Hilberträume mit abzählbar unendlichen VONS sind isomorph.

	Ist $(H,\<\argdot, \argdot\>)$ Hilbertraum mit VONS $(e_j)$ so ist
	\begin{align*}
		\Phi: H &\to \ell^2 \\
		x &\mapsto (\<x,e_j\>)_{j\in \N}
	\end{align*}
	ein Hilbertraum-Isomorphismus
	\begin{proof}
		\coursehref{blatt01.pdf}{Übungsaufgabe 1.2b}
	\end{proof}
\end{st}



\section{Operatoren und Eigenwerte}



\begin{df} \label{1.17}
	\begin{enumerate}[1)]
		\item
			Sei $D(A)$ ein linearer Teilraum von $L$ und
			\[
				A : D(A) \to L
			\]
			eine lineare Abbildung.
			Dann heißt $A$ \emph{linearer Operator} in $L$.
		\item
			Ein linearer Operator $A$ heißt \emph{symmetrisch}, falls
			\[
				\forall x,y \in D(A) : \<Ax,y\> = \<x,Ay\>.
			\]
		\item
			$\lambda \in \C$ heißt \emph{Eigenwert} (EW) von $A$, falls
			\[
				\exists x \in D(A) \setminus \{0\} : Ax = \lambda x.
			\]
			$x$ heißt \emph{Eigenelement} oder \emph{Eigenvektor} von $A$ zum Eigenwert $\lambda$.

			Man nennt
			\[
				N(\lambda) := \dim (\ker(A-\lambda \Id))
			\]
			(geometrische) \emph{Vielfachheit} von $\lambda$.
	\end{enumerate}
\end{df}

\begin{st} \label{1.18}
	Ist $A: D(A) \to L$ symmetrisch, so gelten
	\begin{enumerate}[1)]
		\item
			Alle Eigenwerte sind reell.
		\item
			Eigenelemente zu verschiedenen Eigenwerten sind orthogonal.
	\end{enumerate}
	\begin{proof}
		Übung, oder siehe LAAG1.
	\end{proof}
\end{st}

\begin{ex}[Diskreter Spektraloperator] \label{1.19}
	Sei $(e_n)$ ein ONS im Hilbertraum $H$ und $(\lambda_j)$ ein Folge in $\C$.
	Definiere
	\begin{align*}
		D(A) &:= \Big\{x \in H : \sum_{j=1}^\infty |\lambda_j \<x,e_j\> |^2 < \infty \Big\}, \\
		Ax &:= \sum_{j=1}^\infty \lambda_j \<x,e_j\> e_j \qquad \text{für } x \in D(A).
	\end{align*}
	\begin{enumerate}[1)]
		\item
			$D(A)$ ist linearer Teilraum von $H$ (leicht nachzurechnen).
			Die Reihe für $Ax$ konvergiert, da nach Parseval:
			\begin{align*}
				\Big\| \sum_{j=n}^m \lambda_j \<x,e_j\> e_j \Big\|^2
				= \sum_{j=n}^m |\lambda_j \<x,e_j\> |^2
				< \eps
				\qquad \text{für } m \ge n > N_\eps.
			\end{align*}
			Also ist $A$ linearer Operator (Linearität leicht nachzurechnen).
		\item
			\begin{enumerate}[a)]
				\item
					Alle $\lambda_j$ sind Eigenwerte, da für $x := e_j \in D(A)$
					\[
						Ax = \sum_{k=1}^\infty \lambda_k \<e_j,e_k\> e_k = \lambda_j x.
					\]
				\item
					Falls $\exists x \in H\setminus \{0\} \forall j \in \N : \<x,e_j\> = 0$, dann ist $Ax = 0$.
					Also ist $\lambda = 0$ eventuell ein zusätzlicher Eigenwert.
				\item
					Es gibt keine weiteren Eigenwerte.
					\begin{proof}
						Sei $Ax = \lambda x$ mit $\lambda \neq \lambda_j$ für alle $j \in \N$.
						Dann gilt für alle $k \in \N$:
						\begin{align*}
							\lambda \<x,e_k\>
							&= \<Ax, e_k\> \\
							&= \Big\< \sum_{j=1}^\infty \lambda_j \<x,e_j\> e_j, e_k \Big\> \\
							&= \lambda_k \<x,e_k\> \<e_k, e_k\>
							= \lambda_k \<x,e_k\>.
							\end{align*}
						Damit gilt $ (\lambda-\lambda_k)\<x,e_k\>=0$ und wegen $\lambda\neq \lambda_k$ auch $\<x,e_k\>=0 $ für alle $ k\in \N $.
						Dies impliziert $Ax = \sum_{j=1}^\infty \lambda_j \<x,e_j\> e_j = 0$.
						Damit sind alle Eigenwerte von $A$ gegeben durch
						\[
							\{ \lambda_j : j \in \N \} \quad \text{oder}\quad \{\lambda_j : j \in \N\} \cup \{0\}.
						\]
					\end{proof}
			\end{enumerate}
		\item
			$A$ symmetrisch $\iff \forall j \in \N : \lambda_j \in \R$.
			\begin{proof}
				\begin{segnb}[$\implies$]
					Gilt nach \ref{1.18}.
				\end{segnb}
				\begin{segnb}[$\impliedby$]
					Seien $x,y \in D(A)$. Dann gilt wegen der Stetigkeit des Skalarproduktes
					\begin{align*}
						\<Ax,y\>
						&= \Big\< \sum_{j=1}^\infty \lambda_j \<x,e_j\> e_j, y \Big\>
						= \sum_{j=1}^\infty \lambda_j \<x,e_j\> \<e_j, y\> \\
						&= \sum_{j=1}^\infty \_{\lambda_j \<y,e_j\>} \<x, e_j\>
						= \Big\< x, \sum_{j=1}^\infty \lambda_j \<y,e_j\> e_j \Big\>
						= \<x, Ay\>.
					\end{align*}
				\end{segnb}
			\end{proof}
		\item
			Falls $(\lambda_j)$ beschränkt, so ist $D(A) = H$
			\begin{proof}
				Sei $x \in H$ und $\lambda_j$ beschränkt durch $C$, dann ist
				\[
					\sum_{j=1}^\infty |\lambda_j\<x,e_j\>|^2
					\le \sum_{j=1}^\infty |\<Cx, e_j\>|^2
					\le \|C x\|
					< \infty
				\]
				und somit $x \in D(A)$.
			\end{proof}
		\item
			Falls $(e_j)$ ein VONS in $H$ ist, so ist $D(A)$ dicht in $H$.
			\begin{proof}
				Sei $x \in H$, $x = \sum_{j=1}^\infty \<x,e_j\> e_j$.
				Setze $x_n := \sum_{j=1}^n \<x,e_j\> e_j$.
				Dann konvergiert $x_n \to x$ und $x_n \in D(A)$ (da $\<x_n,e_k\> = 0$ für $k > n$, also endliche Summe).
			\end{proof}
	\end{enumerate}
\end{ex}

\begin{df} \label{1.20}
	Die Darstellung
	\[
		Ax := \sum_{j=1}^\infty \lambda_j \<x,e_j\> e_j
	\]
	des Operators $A$ heißt \emph{Spektraldarstellung} von $A$.
	Die Menge
	\[
		\sigma(A) := \_{\{\lambda \in \C : \lambda \text{ ist Eigenwert von $A$} \}}
	\]
	heißt \emph{Spektrum} von $A$.
	Ein Operator $A$, der durch diese Darstellung gegeben ist, heißt \emph{diskreter Spektraloperator}.
	\begin{note}
		Die Spektraldarstellung entspricht der Diagonalisierung bei Matrizen.
	\end{note}
\end{df}

\begin{df*}
	Man nennt $A$ \emph{positiv}, falls
	\[
		\forall x \in D(A) : \<Ax,x\> \ge 0
	\]
	\begin{note}
		In diesem Fall sind alle Eigenwerte von $A$ nicht-negativ.
		\begin{proof}
			Sei $x$ Eigenvektor zum Eigenwert $\lambda$, dann ist
			\[
				0 \le \<Ax,x\>
				= \lambda \underbrace{\<x,x\>}_{>0},
			\]
			also $\lambda \ge 0$.
		\end{proof}
	\end{note}
\end{df*}

\begin{ex} \label{1.21}
	Sei $H = L^2([0,\pi])$ und
	\begin{align*}
		D(A) &:= \Big\{ f \in C^2 ([0,\pi] \to \C) : f(0) = f(\pi) = 0 \Big\} \\
		Af &:= -f'' \qquad \text{für } f \in D(A)
	\end{align*}
	\begin{enumerate}[1)]
		\item
			$A$ ist symmetrisch.
			\begin{proof}
			Es gilt
				\begin{align*}
					\<Af, g\>
					&= \int_0^{\pi} Af \_g \dx
					= - \int_0^{\pi} f'' \_g \dx \\
					&= -\big[ f'\_{g} \big]_{0}^\pi + \int_{0}^\pi f'\_{g'} \dx \\
					&= \underbrace{\big[ - f'\_g + f \_g' \big]_{x=0}^\pi}_{=0} - \int_0^\pi f \_g'' \dx
					= \<f, Ag\>.
				\end{align*}
			\end{proof}
			Insbesondere hat $A$ damit nach \ref{1.18} 1) nur reelle Eigenwerte.
		\item
			$A$ ist positiv.
			\begin{proof}
				Sei $f \in D(A)$.
				Dann gilt
				\begin{align*}
					\<Af,f\>
					&= - \int_0^\pi f'' \_f \dx \\
					&= - \underbrace{\big[f' \_f \big]_{x=0}^\pi}_{=0} + \int_0^\pi \underbrace{f' \_f'}_{= |f'|^2 \ge 0} \dx
					\ge 0.
				\end{align*}
			\end{proof}
			Insbesondere sind damit alle Eigenwerte $\ge 0$.
		\item
			Alle Eigenwerte sind von der Form $\lambda_j = j^2$ mit Eigenfunktionen $e_j(x) = \sqrt{\f 2 \pi} \sin (jx)$.
			Die $(e_j)$ bilden ein ONS.
			\begin{proof}
				Sei $f \in D(A) \setminus \{0\}$.
				Für die Eigenwerte gilt:
				\begin{align*}
					Af = \lambda f
					&\iff\quad -f'' = \lambda f \\
					&\iff\quad f(x) = c_1 \sin(\sqrt \lambda x) + c_2 \cos (\sqrt \lambda x) \qquad c_1,c_2 \in \C.
				\end{align*}
				Wegen $f(0)=0$ ist $c_2 = 0$ und wegen $f(\pi) = 0$ muss $\lambda = j^2$ für $j\in \N$ gelten ($\lambda, c_1 \neq 0$, sonst $f = 0$).
				Also folgt
				\[
					f(x) = c_1 \sin(j x) \qquad  \text{für $c_1 \neq 0, j \in \N$.}
				\]

				Da $A$ symmetrisch (also die Eigenfunktionen $e_j$ orthogonal) und
				\[
					\|e_j\|^2 = \int_0^\pi \f 2\pi \sin^2(x) \dx = \f 1\pi \int_0^\pi 1 - \cos(2x) \dx = 1,
				\]
				bilden die $(e_j)$ ein ONS.
			\end{proof}
\coursetimestamp{15}{4}{2013}
		\item
			Sei $e_j := x \mapsto \sqrt{\f 2\pi} \sin(jx)$ das ONS aus 3).
			Setze
			\begin{align*}
				D(\tilde A) &:= \Big\{ f\in L^2([0,\pi]) : \sum_{j=1}^\infty j^4 |\<f,e_j\>|^2 < \infty \Big\}, \\
				\tilde A f &:= \sum_{j=1}^\infty j^2 \<f, e_j \> e_j.
			\end{align*}
			Dann gelten
			\begin{enumerate}[a)]
				\item
					$D(A) \subset D(\tilde A)$,
				\item
					$\tilde A \Big|_{D(A)} = A$, bzw. $\tilde Af = Af$ für $f \in D(A)$.
			\end{enumerate}
			Man sagt: $\tilde A$ ist die \emph{Erweiterung} des Operators $A$ und schreibt: $A \subset \tilde A$.
			\begin{proof}
				\begin{enumerate}[a)]
					\item
						Zeige $D(A) \subset D(\tilde A)$.
						Sei $f \in D(A)$. Dann ist
						\[
							f \in C^2([0,\pi]\to \C) \subset L^2([0,\pi])
						\]
						und
						\[
							j^2\<f,e_j\> = \<f,j^2e_j\> = \<f,Ae_j\>= \<Af, e_j\>
						\]
						da $e_j, f \in D(A)$ und $A$ symmetrisch.
						Also,
						\[
							\sum_{j=1}^\infty j^4|\<f,e_j\>|^2
							= \sum_{j=1}^\infty |\<Af,e_j\>|^2
							\stackrel{\text{Bessel}}\le \|Af\|^2
							= \|f''\|^2
							< \infty
						\]
						und damit $f \in D(\tilde A)$.
					\item
						Zeige $\tilde A \Big|_{D(A)} = A$, bzw. $\tilde Af = Af$ für $f \in D(A)$.

						Sei $f \in D(A)$. Dann gilt
						\begin{align*}
							\tilde Af = \sum_{j=1}^\infty j^2 \<f,e_j\> e_j = \sum_{j=1}^\infty \<f,\underbrace{j^2e_j}_{=Ae_j}\> e_j = \sum_{j=1}^\infty \<Af, e_j\> e_j= Af,
						\end{align*}
						da $(e_j)$ ein VONS von $\im(A)$ ist (wird später in \ref{3.5} 4) gezeigt).
				\end{enumerate}
			\end{proof}
	\end{enumerate}
\end{ex}

\begin{nt} \label{1.22}
	Seien $A$ und $D(A)$ gegeben wie in \ref{1.19}.
	Dann ist
	\[
		A \text{ positiv }  \quad\iff\quad \forall j\in \N : \lambda_j \ge 0
	\]
	\begin{proof}
		Es gilt
		\begin{align*}
			\<Ax, x \>
			= \sum_{j=1}^\infty \underbrace{\<\lambda_j \<x,e_j\> e_j, x\>}_{= \lambda_j\<x,e_j\> \<e_j,x\>}
			= \sum_{j=1}^\infty \lambda_j |\<x,e_j\>|^2.
		\end{align*}
		Die Rückrichtung erkennt man jetzt leicht.

		Die Hinrichtung wurde bereits gezeigt (wenn ein Operator $A$ positiv ist, gilt allgemein, dass alle seine Eigenwerte nichtnegativ sind).
	\end{proof}
\end{nt}

\begin{df} \label{1.23}
	Ein linearer Operator $A$ in $L$ mit $D(A) = L$ heißt \emph{beschränkt}, falls
	\[
		\exists c > 0 \forall x \in L : \|Ax\| \le c\|x\|.
	\]
	Dann heißt
	\begin{align*}
		\|A\| &:= \inf \Big\{ c > 0 : \forall x \in L : \|Ax\| \le c\|x\| \Big\} \\
		&= \sup_{\|x\|=1} \|Ax\| = \sup_{\|x\|\le 1} \|Ax\| = \sup_{\|x\|\neq 0} \f{\|Ax\|}{\|x\|}
	\end{align*}
	die \emph{(Operator)-Norm} von $A$.

	Insbesondere gilt für alle $ x\in L $
	\[
		\|Ax\| \le \|A\| \|x\|.
	\]
	\begin{note}
		Der Beweis der Gleichheit obiger Sumprema ist eine leichte Übungsaufgabe.
	\end{note}
\end{df}

\begin{nt} \label{1.24}
	Seien $A$ und $D(A)$ gegeben wie in \ref{1.19}.
	Dann gilt
	\[
		(\lambda_j) \text{ beschränkt} \quad\iff\quad D(A)=H \text{ und } A \text{ beschränkt.}
	\]
	\begin{proof}
		Zeige durch vollständige Fallunterscheidung:
		\begin{seg}[$(\lambda_j)$ beschränkt]
			Sei $ c\ge 0 $ mit $ |\lambda_j|\le c $ und $x \in L$ beliebig.
			Nach Bessel ist $\sum_{j=1}^\infty c^2 |\<x,e_j\>|^2 \le c^2 \|x\|^2$ konvergent, also ist
			\[
				\sum_{j=1}^\infty \underbrace{|\lambda_j|^2 |\<x,e_j\>|^2}_{\le c^2|\<x,e_j\>|^2}
			\]
			nach dem Vergleichskriterium konvergent, und somit $D(A) = H$.
			Wegen
			\begin{align*}
				\|Ax\|^2
				&= \bigg\|\sum_{k=1}^\infty \lambda_k \<x,e_k\> e_k\bigg\|^2
				\stackrel{\text{Parseval}} = \sum_{j=1}^\infty \Big| \Big\< \sum_{k=1}^\infty \lambda_k \<x,e_k\> e_k, e_j \Big\> \Big|^2 \\
				&\le c^2 \sum_{j=1}^\infty |\<x,e_j\>|^2
				\stackrel{\text{Bessel}} \le c^2 \|x\|^2,
			\end{align*}
			also $\|Ax\| \le c\|x\|$, ist außerdem $A$ beschränkt.
		\end{seg}
		\begin{seg}[$(\lambda_j)$ nicht beschränkt]
			Es existiert eine Teilfolge mit $|\lambda_{j_k}| \to \infty$ ($k\to \infty$) und damit
			\[
				\|A e_{j_k}\| = \|\lambda_{j_k} e_{j_k}\| = |\lambda_{j_k}| \to \infty,
			\]
			insbesondere $\sup_{\|x\|=1} \|Ax\| = \infty$, also ist $A$ nicht beschränkt.
		\end{seg}
	\end{proof}
\end{nt}

\begin{nt*}
	$M$ heißt \emph{folgenkompakt}, falls jede Folge in $M$ eine konvergente Teilfolge besitzt.

	$M$ heißt \emph{überdeckungskompakt}, falls für jede offene Überdeckung $(O_\alpha)_{\alpha \in A}$ von $M$ eine endliche Teilmenge $(O_{\alpha_i})_{i=1,\dotsc,N}$ existiert, die $M$ überdeckt.
\end{nt*}

\begin{df} \label{1.25}
	Ein linearer Operator $A$ in $L$ mit $D(A) = L$ heißt \emph{kompakt}, falls
	\[
		(x_j) \text{ beschränkt} \quad\implies\quad \text{$(Ax_j)$ besitzt eine in $L$ konvergente Teilfolge}
	\]
	oder äquivalent
	\[
		M \subset L \text{ beschränkt} \quad\implies\quad \text{$A(M)$ ist präkompakt, d.h. $\_{A(M)}$ ist kompakt in $L$.}
	\]
\end{df}

\begin{st} \label{1.26}
	Jeder kompakte Operator $A$ ist beschränkt.
	\begin{proof}
		Zeige durch Kontraposition:
		Sei $A$ nicht beschränkt, dann existiert $(x_j) \subset L$ mit
		\[
			\|x_j\| = 1  \quad\land\quad \|Ax_j\| \to \infty.
		\]
		Also für jede Teilfolge $\|Ax_{j_k}\| \to \infty$.
		Damit ist $A$ nicht kompakt.
	\end{proof}
\end{st}

\begin{st} \label{1.27}
	Sei $A$ ein linearer Operator in $L$ mit $D(A) = L$.
	Dann gilt
	\begin{align*}
		A \text{ beschränkt}
		&\quad\iff\quad A \text{ stetig in } x = 0 \\
		&\quad\iff\quad A \text{ ist stetig.}
	\end{align*}
	\begin{proof}
		\begin{seg}[$A$ beschränkt $\implies$ $A$ stetig in $x=0$]
			Sei $x_j \to 0$, dann gilt
			\[
				\|Ax_j\| \le \|A\| \|x_j\| \to 0,
			\]
			also $Ax_j \to 0$ und somit ist $A$ stetig in $x=0$.
		\end{seg}
		\begin{seg}[$A$ stetig in $x_0=0$ $\implies$ $A$ stetig]
			Sei $x_j \to x$, also $x_j - x \to 0$, dann gilt, da $A$ stetig in $x_0=0$.
			\[
				Ax_j - Ax = A(\underbrace{x_j - x}_{\to 0}) \to 0,
			\]
			also $Ax_j \to Ax$ und somit ist $A$ stetig.
		\end{seg}
		\begin{seg}[$A$ stetig $\implies$ $A$ beschränkt]
			Zeige durch Kontraposition.
			Sei $A$ nicht beschränkt und $(x_j) \subset L$ mit $\|x_j\| = 1$, $\|Ax_j\| \to \infty$.
			Definiere $y_j := \f 1{\|Ax_j\|}x_j$.
			Dann folgt
			\[
				\|y_j\| = \f {\|x_j\|}{\|Ax_j\|} \to 0
			\]
			und
			\[
				\|Ay_j\| = \f 1{\|Ax_j\|} \|Ax_j\| = 1 \not\to 0.
			\]
			Somit ist $A$ nicht stetig.
		\end{seg}
	\end{proof}
\end{st}

\begin{ex} \label{1.28}
	Sei $K \subset \R^n$ kompakt, $G \in C(K\times K \to \C)$, $L := C(K \to \C)$.
	Definiere den \emph{Integraloperator} $A$ durch
	\[
	 Af := \int_{K} G(\argdot,y) f(y) \dx[y]
		\qquad \text{für $f\in L$}
	\]

	Dann ist $A : L \to L$ kompakt.
	\begin{proof}[Beweisskizze]
		Sei $(f_j)$ beschränkt in $L$.
		\begin{enumerate}[1)]
			\item
				Zeige: Die Folge $(A f_j(x))$ ist bezüglich $x \in K$ gleichmäßig beschränkt.

				$K\times K$ ist kompakt, also existiert ein $ c\ge0 $ mit $|G(x,y)| \le c$  für alle $x,y \in K\times K$.
				Dann ist
				\[
					|Af_j(x)| \le \underbrace{\int_{K} |G(x,y) |f(y)| \dx[y]}_{\le\<c,|f_j|\>} \stackrel{CSB}\le \|c\| \|f_j\| \le \tilde c.
				\]
			\item
				Zeige: Die Folge $(Af_j)$ ist gleichgradig stetig auf $K$, d.h.
				\[
					\forall \eps > 0 \exists \delta > 0 \underbrace{\forall x,x' \in K \forall j \in \N}_{\mathclap{\text{gleichmäßig stetig bezüglich $x$ und $j$}}} :
					 |x-x'| < \delta \;\implies\; |Af_j(x)-Af_j(x')| < \eps.
				\]
				Wegen $K\times K$ kompakt, ist $G$ gleichmäßig stetig in $y$, d.h. $|G(x,y) - G(\tilde x,y)| \le \eps$ für $|x-\tilde x| < \delta$.
				Also
				\begin{align*}
					|Af_j(x) - Af_j(\tilde x)|
					&= \bigg| \int_K \Big(\underbrace{G(x,y) - G(\tilde x, y)}_{|\argdot| < \eps}\Big) f_j(y) \dx[y] \bigg| \\
					&\le \int_K \eps |f_j(y)| \dx[y]
					= \<\eps, |f_j|\> \\
					&\stack{CSB}\le \|\eps\| \|f_j\| = \sqrt{\int_{K}|\eps|^2 \dx} \|f_j\|=\sqrt{\mu(K)} \eps \|f_j\|\le c \eps.
				\end{align*}
			\item
				Nach dem Satz \ref{1.29} von Arzelá-Ascoli enthält $(Af_j)$ eine auf $K$ gleichmäßig konvergente Teilfolge $(Af_{j_k})$.
			\item
				Da $(Af_{j_k})$ gleichmäßig konvergiert, ist die Grenzfunktion $g$ stetig auf $K$ (also $g \in L$) und $\|Af_{j_k} - g\| \to 0$.
				Damit ist $A$ kompakt.
		\end{enumerate}
	\end{proof}
\end{ex}

\coursetimestamp{17}{4}{2013}
\begin{st}[Arzelá-Ascoli] \label{1.29}
	Sei $K \subset \R^n$ kompakt und $f_j : K \to \C$ eine Folge mit $(f_j)$ punktweise beschränkt $(\forall x \in K \exists c_x >0 \forall j\in \N : |f_j(x)|\le c_x)$ und gleichgradig stetig, also:
	\[
		\forall \eps > 0 \exists \delta > 0 \forall x,x' \in K \forall j \in \N : \|x-x'\| < \delta \implies |f_j(x)-f_j(x')| < \eps.
	\]
	Dann besitzt $(f_j)$ eine Teilfolge, die auf $K$ gleichmäßig gegen $f\in C(K \to \C)$ konvergiert.
	\begin{proof}
		\begin{enumerate}[1)]
			\item
				\textbf{Es existiert $M \subset K$ mit $M$ abzählbar und dicht in $K$:}

				  Wir setzten $ K_\delta(x):= \{y\in \R^n; \|y-x\|<\delta \} $.

				  Da $ K $ (überdeckungs-)kompakt, lassen sich wegen $ K\subset \bigcup_{x\in K} K_{\f{1}{l}}(x) $ für alle $ l\in \N $ jeweils endlich viele Punkte $ x_1^{(l)},..., x^{(l)}_{N_l} $ finden, so dass
				  \[
				  K\subset \bigcup_{j=1}^{N_l} K_{\frac{1}{l}}(x_j^{(l)}).
				  \]

					Definiere
					\[
						M := \bigcup_{k=1}^\infty \Big\{x_1^{(k)}, \dotsc, x_{N_k}^{(k)}\Big\}
					\]
					$M \subset K$ ist abzählbar und dicht in $K$.
					Insbesondere finden wir eine bijektive Abbildung $ \xi_\argdot: \N \to M $.

			\item
				\textbf{Konstruiere konvergente Teilfolge $(g_j)$ von $(f_j)$:}

					Wegen $(f_j(\xi_1))$ beschränkt existiert nach Bolzano"=Weierstraß eine konvergente Teilfolge $(f_j^{(1)}(\xi_1))$.
					Diese ist wiederum beschränkt in $\xi_2$, also existiert eine konvergente Teilfolge $(f_j^{(2)}(\xi_2))$, usw.
					Wähle jetzt $(g_j) := (f_j^{(j)})$ als Diagonalfolge der so erhaltenen Teilfolgen von $(f_j)$.
					Dann ist für jedes $\xi_k \in \{\xi_1, \xi_2, \dotsc\}$ $(g_j(\xi_k))$ konvergent für $j \to \infty$.
			\item
				\textbf{$(g_j)$ konvergiert gleichmäßig auf $K$:}

					Zeige: $(g_j)$ ist gleichmäßige Cauchy-Folge, das heißt
					\[
						\forall \eps > 0 \exists J \in \N \forall j,k \ge J, x \in K : |g_j(x) - g_k(x)| < \eps
					\]

					Sei dazu $ \eps>0 $ beliebig.
					Wähle $ \delta >0 $ passend zu $\eps$ gemäß der gleichgradigen Stetigkeit der Funktionen $g_j$ (ererbt von $f_j$, da Teilfolgen), also:
					\[
						\forall x,x' \in K \forall j \in \N : \|x - x'\| < \delta \implies | g_j(x) - g_j(x') | < \eps.
					\]
					\begin{enumerate}[a)]
						\item
							Wähle $\{\xi_{l_1}, \dotsc, \xi_{l_N} \}\subset M$ so dass $K \subset \bigcup_{m=1}^N K_\delta (\xi_{l_m})$ (funktioniert, da $K$ (üb"-er"-deck"-ungs-)kompakt, siehe Konstruktion von $M$).
						\item
							Da $g_j$ punktweise konvergent (weil Teilfolgen von $f_j$), wähle $J \in \N$, sodass
							\[
								\forall j,k > J \forall m \in \{1, \dotsc, N\} : \big| g_j(\xi_{l_m}) - g_k(\xi_{l_m}) \big| < \eps.
							\]
					\end{enumerate}
					Wir finden jetzt für jedes $ x $ ein $ \xi_{l_i}, i\in\{1,...,N\} $, sodass $ x \in K_\delta(\xi_{l_i}) $.
					Wegen
					\begin{enumerate}[(I)]
						\item $\big|g_j(x)-g_j(\xi_{l_i})\big|< \eps$ da $|x-\xi_{l_i}|<\delta$,
						\item $ \big|g_j(\xi_{l_i}) - g_k(\xi_{l_i})\big| <\eps$ da $j,k > J$,
						\item $ \big| g_k(\xi_{l_i}) - g_k(x)\big| <\eps$ da $|x-\xi_{l_i}| < \delta$
					\end{enumerate}
					gilt
					\begin{align*}
						|g_j(x) - g_k(x)|
						&\le \underbrace{\big|g_j(x)-g_j(\xi_{l_i})\big|}_{\text{(I)}}
							+ \underbrace{\big|g_j(\xi_{l_i}) - g_k(\xi_{l_i})\big|}_{\text{(II)}}
							+ \underbrace{\big| g_k(\xi_{l_i}) - g_k(x)\big|}_{\text{(III)}}. \\
						&< 3\eps
					\end{align*}
					für alle $j,k > J$, $x\in K$.
					Also konvergiert $(g_j)$ gleichmäßig in $K$.
			\item
				Die $g_j$ sind stetig und gleichmäßig konvergent, also ist
				\begin{align*}
					g: K &\to \C \\
					x &\mapsto \lim_{j\to \infty} g_j(x)
				\end{align*}
				der stetige Grenzwert der Funktionenfolge $ (g_j) $.
		\end{enumerate}
	\end{proof}
\end{st}
