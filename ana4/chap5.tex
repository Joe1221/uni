% This work is licensed under the Creative Commons
% Attribution-NonCommercial-ShareAlike 3.0 Unported License. To view a copy of
% this license, visit http://creativecommons.org/licenses/by-nc-sa/3.0/ or send
% a letter to Creative Commons, 444 Castro Street, Suite 900, Mountain View,
% California, 94041, USA.

\chapter{Distributionen}


Betrachten $f : \R^3 \setminus \{0\} \to \R : x \mapsto \f 1{|x|}$.
Für $x \neq 0$ gilt
\begin{align*}
	\grad f &= - \f x{|x|^3} \\
	\Delta f = \div \grad f &= 0
\end{align*}
Physiker sagen: $-\Delta f = \delta_0$ (Diracsche $\delta$-Funktion):
„$\delta_0 = 0$ für $x \neq 0$, $\delta_0(0)$ so, dass $\int \delta_0(x)g(x) \dx = g(0)$“

Arbeitsplan: Erweitere $C^1(\R^n \to \C)$ zu einem topologischen Raum $D'(\R^n)$ in dem gelten:
\begin{enumerate}[1)]
	\item
		$D_j : D'(\R^n) \to D'(\R^n) : f \mapsto \partial_{x_j} f$ sind stetige Abbildungen.
	\item
		Zeige: $f : x \mapsto \f 1{|x|}$, $\delta_0$ sind Elemente von $D'(\R^3)$ und
		\[
			-\Delta f = \delta_0.
		\]
\end{enumerate}


\section{Konstruktion des Raumes}


\begin{conv} \label{5.1}
	Für $\alpha \in \N_0^n$ schreiben wir
	\[
		\nabla^\alpha := \partial_{x_1}^{\alpha_1} \dotso \partial_{x_n}^{\alpha_n}.
	\]
\end{conv}

\begin{df}[Raum der Testfunktionen] \label{5.2}
	Seien $\phi_j, \phi \in C_0^\infty(\R^n)$.
	Dann heißt die Folge $(\phi_j)_{j\in \N}$ \emph{$D$-konvergent} gegen $\phi$, falls
	\begin{enumerate}[1)]
		\item
			$\exists K \subset \R^n \text{ kompakt } \forall j \in \N: \supp \phi_j \subset K$,
		\item
			$\forall \alpha \in \N_0^n: \nabla^{\alpha} \phi_j \to \nabla^\alpha \phi$ auf $\R^n$ gleichmäßig (d.h. $\|\nabla^\alpha \phi_j - \nabla^\alpha \phi\|_{L^\infty(\R^n)} \to 0$ $\forall \alpha \in \N_0^n$).
	\end{enumerate}
	(insbesondere $\supp \phi \subset K$)

	Wir schreiben dann
	\[
		\phi_j \overset{D} \longrightarrow \phi.
	\]
	Der lineare Raum $C_0^\infty(\R^n)$ versehen mit diesem Konvergenzbegriff heißt \emph{Raum der Testfunktionen, $D(\R^n)$}  (Präziser: $D(\R^n)$  ist ein lokalkonvexer topologischer Raum, siehe z.B. \cite{Reed-Simon}).
\end{df}

\begin{st} \label{5.3}
	$D(\R^n)$ ist vollständig.
	\begin{proof}
		Sei $(\phi_j)$ Cauchy-Folge in $D(\R^n)$, d.h. $\exists K \subset \R^n \text{ kompakt } \forall j \in \N: \supp \phi_j \subset K$ und
		\[
			\forall \alpha \in \N_0^n \forall \eps > 0 \exists J_\eps \in \N \forall j,k > J_\eps : \|\nabla^\alpha \phi_j - \nabla^\alpha \phi_k \|_{L^\infty(\R^n)} < \eps.
		\]
		Also ist für jedes feste $\alpha \in \N_0^n$ ist $(\nabla^\alpha \phi_j)_{j\in \N}$ eine Cauchy-Folge bezüglich $\|\argdot\|_{L^\infty(\R^n)}$.
		Damit gilt $\nabla^\alpha \phi_j \to \psi_\alpha \in C(\R^n \to \C)$ gleichmäßig.

		$\psi_0 := \lim{n\to \infty} \nabla^0 \phi_j$ ist differenzierbar, $\nabla^\alpha \psi_0 = \psi_\alpha \in C(\R^n \to \C)$, also $\psi_0 \in C^\infty (\R^n \to \C)$.

		Wegen $\supp \phi_j \subset K$, $\phi_j \to \psi_0$ ist $\supp \psi_0 \subset K$, also $\psi_0 \in C_0^\infty(\R^n)$ und $\nabla^\alpha \psi_j \to \nabla^\alpha \psi_0$ bezüglich $\|\argdot\|_{L^\infty(\R^n)}$.
	\end{proof}
\end{st}

\begin{df}[Schwartzsche Distribution] \label{5.4}
	Eine \emph{(Schwartzsche) Distribution} ist eine lineare, stetige Abbildung $T: D(\R^n) \to \C$.
	Durch
	\begin{align*}
		(T+S)(\phi) &:= T\phi + S\phi,  \qquad T,S \in D(\R^n);\\
		(\alpha T)(\phi) &:= \alpha T\phi,  \qquad T\in D(\R^n), \alpha \in \C
	\end{align*}
	wird $\{ T : D(\R^n) \to \C \suchthat T \text{ linear und stetig}\}$ zu einem linearen Raum, dem Raum $D'(\R^n)$ der Schwartzschen Distributionen.

	Eine Folge $(T_n)$ in $D'(\R^n)$ heißt konvergent gegen $T \in D'(\R^n)$, falls
	\[
		\forall \phi \in C_0^\infty(\R^n) : T_n\phi \to T \phi
	\]
	(Präzizer: $D'(\R^n)$ ist topologischer Dualraum mit von der Topologier auf $D(\R^n)$ induzierten Topologie; zu dieser Topologie gehört unser Konvergenzbegriff).
\end{df}


\begin{ex} \label{5.5}
	\begin{enumerate}[1)]
		\item
			Sei $x_0 \in \R^n$ fest.
			\[
				\delta_{x_0}(\phi) := \phi(x_0) \qquad \text{für $\phi \in C_0^\infty(\R^n)$}
			\]
			heißt \emph{Diracsche $\delta$-Distribution}.

			$\delta_{x_0}$ ist linear:
			\[
				\delta_{x_0}(\alpha \phi + \beta \psi) 
				= (\alpha \phi + \beta \psi)(x_0)
				= \alpha \phi(x_0) + \beta \psi(x_0)
				= \alpha \delta_{x_0}(\phi) + \beta \delta_{x_0}(\psi)
			\]
			$\delta_{x_0}$ ist stetig: Es gelte $\phi_j \overset{D}\to \phi$, also insbesondere $\phi_j \to \phi$ gleichmäßig und punktweise, also
			\[
				\delta_{x_0}(\phi_j) = \phi_j(x_0) \quad \to \quad \phi(x_0) = \delta_{x_0}(\phi).
			\]
		\item
			Sei
			\[
				L_{\text{loc}}^1 (\R^n) := \Big\{ u : \R^n \to \C  \suchthat \forall K \subset \R^n  \text{kompakt} : u_K \in L^1(K) \Big\}
			\]
			(z.B. $f(x) = \f 1{|x|}$, dann ist $f \in L_{\text{loc}}^1(\R^n)$).

			Zu $u \in L_{\text{loc}}^1(\R^n)$ setze
			\[
				T_u(\phi) := \int_{\R^n} u(x) \phi(x) \dx \qquad \text{für $\phi \in C_0^\infty(\R^n)$}.
			\]
			$T_u : C_0^\infty(\R^n) \to \C$ ist linear.

			$T_u$ ist stetig:
			Sei $\phi_j \overset{D}{\to} \phi$, insbesondere $\supp \phi_j \subset K, \supp \phi \subset K$, also
			\begin{align*}
				|T_u \phi_j - T_u \phi| 
				&= \Big| \int_{\R^n} u (\phi_j-\phi) \dx \Big|
				= \Big| \int_{K} u (\phi_j-\phi) \dx \Big| \\
				&\le \underbrace{\|\phi_j - \phi\|_{L^\infty(\R^n)}}_{\to 0} \underbrace{\int_{K} |u| \dx}_{\in \R \text{ da $u\in L_{\text{loc}}^1(\R^n)$}}
			\end{align*}
			Also $T_u \phi_j \to T_u \phi$.
	\end{enumerate}
\end{ex}


\begin{st} \label{5.6}
	$D'(\R^n)$ ist vollständig.
	\begin{proof}
		Sei $(T_n)$ Cauchy-Folge in $D'(\R^n)$, d.h.
		\[
			\forall \phi \in C_0^\infty(\R^n) : (T_n \phi) \text{ Cauchy-Folge in $\C$}
		\]
		(insbesondere $T_n \phi$ konvergent).

		Definiere $T\phi := \lim_{n\to \infty} (T_n \phi)$.
		Dann ist $T$ linear, da $T_n, \lim$ linear.
		$T$ ist stetig, siehe dazu Satz von Banach-Steinhaus oder direkt in \cite{Walter}.
	\end{proof}
\end{st}

\section{Einbettung klassischer Funktionenräume}


\begin{st} \label{5.7}
	Die Abbildung $L_{\text{loc}}^1(\R^n) \ni u \mapsto T_u \in D'(\R^n)$ ist linear und injektiv.
	\begin{proof}
		\begin{enumerate}[1)]
			\item
				Linearität: einfache Übung.
			\item
				Injektivität:
				
				Es gilt $T_u = T_v \iff T_u - T_v = 0 \iff T_{u-v} = 0$, zeige $T_u = 0 \implies u = 0$ fast überall in $\R^n$.
				\begin{enumerate}[a)]
					\item
						Abschneiden:

						Wähle $\psi_R \in C_0^\infty(\R^n), \psi_R(x) = 1, x \in \_{K_R(0)}, 0 \le \psi_{R}(x) \le 1$.
						Dann ist
						\[
							T_{\psi_R \cdot u} (\phi) 
							= \int_{\R^n} (\psi_R u) \phi \dx 
							= T_u (\underbrace{\psi_R \phi}_{\in C_0^\infty(\R^n)})
							= 0
						\]
						Also $\psi_R u \in L^1(\R^n), T_{\psi_R u} = 0, \supp(\psi_R u) \subset \_{K_{R+1}(0)}$.
					\item
						Approximation:

						Setze
						\begin{align*}
							j(x) &:= \begin{cases}
								c e^{-\f 1{1-|x|^2}} & |x| < 1 \\
								0 & |x| \ge 1
							\end{cases} \qquad \text{mit }
							c := \dfrac 1{\int_{|x|< 1} e^{-\f 1{1-|x|^2}} \dx} \\
							j_\eps(x) &:= \f 1{\eps^n} j(\f x\eps)
						\end{align*}
						Also $j_\eps \in C_0^\infty(\R^n), \supp j_\eps = \_{K_\eps(0)}, j_\eps(x) \ge 0, \int_{\R^n} j_\eps \dx = 1$.

						Für festes $y\in \R^n$ ist $j_\eps(y-\argdot) \in C_0^\infty(\R^n)$.
						Da $T_{\psi_R u} = 0$ ist
						\[
							0 = T_{\psi_R u} (j_\eps(y-\argdot))
							= \int_{\R^n} \psi_R(x) u(x) j_\eps(y-x) \dx
							= J_\eps(\psi_R u)(y)
						\]
						mit Glättungsoperator $J_\eps$.
						Nach \ref{4.20} ist
						\[
							\|\psi_R u \|_{L^1(\R^n)}
							= \|\psi_R u - \underbrace{J_\eps(\psi_R u)}_{=0} \|_{L^1(\R^n)}
							\to 0 \qquad \text{für $\eps \searrow 0$.}
						\]
						also $\|\psi_R u \|_{L^1(\R^n)}$, bzw. $\psi_R u = 0$ fast überall in $\R^n$.
						Insbesondere ist
						\[
							\my \Big(\big\{ x \in \R^n : |x| \le R \land u(x) \neq 0 \big\} \Big) = 0,
						\]
						also
						\begin{align*}
							\my(\{x\in \R^n : u(x) \neq 0\})
							&= \my \Big( \bigcup_{k=1}^\infty \{|x| \le k : u(x) \neq 0\} \Big) \\
							&\le \sum_{k=1}^\infty \my \big( \{|x|\le k : u(x)\neq 0\}\big)
							= 0
						\end{align*}
						und somit $u= 0$ fast überall in $\R^n$.
				\end{enumerate}
		\end{enumerate}
	\end{proof}
\end{st}

\begin{df} \label{5.8}
	Wir identifizieren $u$ und $T_u$.
	Dann ist $L_{\text{loc}}^1 (\R^n) \subset D'(\R^n)$ (insbesondere $C^1(\R^n \to \C) \subset D'(\R^n)$).

	Eine Distribution $T \in D'(\R^n)$ heißt \emph{regulär}, falls
	\[
		\exists u \in L_{\text{loc}}^1 (\R^n) : T = T_u,
	\]
	sonst \emph{singulär}.

	Falls $T = T_u$, schreibt man auch $T_u(\phi) =: (u,\phi) =: (T_u,\phi)$ (oft auch $T(\phi) =: (T,\phi)$).
\end{df}

\begin{ex} \label{5.9}
	\begin{enumerate}[1)]
		\item
			$\delta_{x_0} : \phi \mapsto \phi(x_0)$ ist singulär.
		\item
			Im $\R^1$ bezeichne
			\[
				T(\phi) := \lim_{\eps \searrow 0} \Big( \int_{-\infty}^{-\eps} \f{\phi(x)}x \dx + \int_{\eps}^\infty \f {\phi(x)}x \dx \Big)
				= \CH \int_{-\infty}^\infty \f {\phi(x)}x \dx
			\]
			den \emph{Cauchy'schen Hauptwert}.
			$T$ ist singulär.
		\item
			Im $\R^3$ ist $T = T_{\f 1{|x|}}$ regulär.
	\end{enumerate}
\end{ex}

\begin{st} \label{5.10}
	$C_0^\infty(\R^n)$ ist dicht in $D'(\R^n)$.
	\begin{proof}
		Sei $T \in D'(\R^n)$. Konstruiere $t_k \in C_0^\infty(\R^n)$ mit $t_k \to T$, d.h $T_{t_k} \to T$.

		Wähle $\psi_R \in C_0^\infty(\R^n)$, $\psi_R = 1$ für $|x| \le R$, $\psi_R = 0$ für $|x| \ge R + 1$, $  \le \psi_R(x) \le 1$.
		Setze
		\[
			t_k(x) := \psi_k (x) T(j_{\f 1k}(x - \argdot))
		\]
		Dann ist
		\begin{enumerate}[a)]
			\item
				$t_k \in C_0^\infty(\R^n)$:

				Es ist $\supp(t_k) \subset \sup(\psi_k) \subset \_{K_{R+1}(0)}$.
				Sei $x \in \R^n$ fest, $\alpha \in \N_0^n$, dann ist
				\begin{align*}
					\f {(\nabla^\alpha j_{\f 1k})(x+he_j - y) - (\nabla^\alpha j_{\f 1k})(x-y)}h
					\to (\partial_{x_j} \nabla^\alpha j_{\f 1k})(x-y) \qquad \text{gleichmäßig bzgl. $y\in \R^n$ für $h\to 0$.}
				\end{align*}
				Außerdem ist
				\[
					\supp \f{j_{\f 1k}(x+he_j - \argdot) - j_{\f 1k}(x-\argdot)}h
					\;\subset \_{K_{h+ \f 1k}(x)} \;
					\stack[k \ge 1]{|h| \le 1}{\subset} K_2(x).
				\]
				Also gilt für festes $x \in \R^n$, $k \in \N$
				\[
					\f {j_{\f 1k} (x+he_j - \argdot) - j_{\f 1k}(x-\argdot)}h
					\stack{D}\to \partial_{x_j} j_{\f 1k} (x-\argdot)
				\]
				und da $T$ stetig
				\[
					\underbrace{T \Big( \f {j_{\f 1k} (x+he_j - \argdot) - j_{\f 1k}(x-\argdot)}h \Big)}_{= \f 1h ( T(j_{\f 1k}x+he_j-\argdot)) - T(j_{\f 1k}(x-\argdot))}
					\to T \Big( \partial_{x_j} j_{\f 1k} (x-\argdot) \Big).
				\]
				Also ist
				\[
					\f{\partial}{\partial x_j} T (j_{\f 1k}(x-\argdot)) = T(\partial_{x_j} j_{\f 1k}(x-\argdot))
				\]
				und somit $T(j_{\f 1k}(x-\argdot)) \in C^\infty(\R^n \to \C)$ und $t_k \in C_0^\infty(\R^n)$.
			\item
				Zeige: $\forall \phi \in C_0^\infty(\R^n) : T_{t_k}(\phi) \to T(\phi)$ (vgl. \ref{5.4}).

				Es gilt
				\begin{align*}
					T_{t_k}(\phi) 
					&= \int_{\R^n} t_k(x) \phi(x) \dx \\
					&= \int_{\R^n} \psi_k(x) T(j_{\f 1k}(x-\argdot)) \phi(x) \dx \\
					&= \int_{|x|\le k+1} \psi_k(x) T(j_{\f 1k}(x-\argdot)) \phi(x) \dx \qquad (\supp \psi_k \subset \_{K_{k+1}(0)}) \\
					&= \lim_{n\to\infty} \sum_{l=0}^{M(m)} \psi_k(\xi_l^{(m)}) T(j_{\f 1k}(\xi_l^{(m)}-\argdot)) \phi(\xi_l^{(m)})\my(I_l^{(m)}) \\
					&= \lim_{n\to\infty} T \underbrace{\Big( \sum_{l=0}^{M(m)} \psi_k(\xi_l^{(m)}) (j_{\f 1k}(\xi_l^{(m)}-\argdot)) \phi(\xi_l^{(m)})\my(I_l^{(m)}) \Big)}_{\overset{D}\to \int_{|x|\le k+1} \psi_k(x) j_{\f 1k}(x-\argdot) \phi(x) \dx} \\
					&= T \underbrace{\Big( \int_{|x|\le k+1} \psi_{k}(x) j_{\f 1k} (x-\argdot) \phi(x) \dx \Big)}_{\stack[k \to \infty]{D}\longrightarrow \phi} \\
					&\to T(\phi)
				\end{align*}
			\item
				$T_{t_k} \to T$ in $D'(\R^n)$:
		\end{enumerate}
	\end{proof}
\end{st}


\section{Differentiation}


\begin{lem} \label{5.11}
	Für $u \in C_0^\infty(\R^n), \alpha \in \N_0^n$ gilt $T_{\nabla^\alpha u} (\phi) = (-1)^{|\alpha|} T_u(\nabla^\alpha \phi)$.
	\begin{proof}
		Für $\alpha = e_j$ (Standardbasis), dann ist
		\begin{align*}
			T_{\nabla^{e_j} u}(\phi) 
			&= \int_{\R^n} (\partial_{x_j} u) \phi \dx \\
			&= - \int_{\R^n} u \partial_{x_j} \phi \dx \qquad \text{(part. Int. in $j$-ter Richtung, Randterme verschwinden)} \\
			&= - T_u (\partial_{x_j} \phi).
		\end{align*}
	\end{proof}
\end{lem}

\begin{st} \label{5.12}
	Die Abbildungen
	\[
		\nabla^\alpha : C_0^\infty(\R^n) \to C_0^\infty(\R^n)
	\]
	sind stetig bezüglich des Konvergenzbegriffs in $D'(\R^n)$.
	\begin{proof}
		Zeige: Wenn $u_k, u \in C_0^\infty(\R^n)$ und $T_{u_k} \to T_u$ konvergiert, dann konvergiert $T_{\nabla^\alpha u_k} \to T_{\nabla^\alpha u}$.

		Für $\phi \in C_0^\infty(\R^n)$ ist
		\begin{align*}
			T_{\nabla^\alpha u_k} (\phi) 
			&= (-1)^{|\alpha|} T_{u_k} (\underbrace{\nabla^\alpha \phi}_{\in C_0^\infty(\R^n)}) \\
			&\to (-1)^{|\alpha|} T_u(\nabla^\alpha \phi)
			= T_{\nabla^\alpha u}(\phi).
		\end{align*}
	\end{proof}
\end{st}

\begin{st} \label{5.13}
	Für $\alpha \in \N_0^n$ ist die Abbildung $\nabla^\alpha: C_0^\infty(\R^n) \to C_0^\infty(\R^n)$ eindeutig fortsetzbar zu einer stetigen Abbildung $D^\alpha: D'(\R^n) \to D'(\R^n)$.
	Die Fortsetzung ist gegeben durch
	\[
		(D^\alpha T)(\phi) := (-1)^{|\alpha|} T(\nabla^\alpha \phi)
	\]
	und heißt \emph{Distributionenableitung} oder \emph{schwache Ableitung}.
	\begin{proof}
		\begin{enumerate}[1)]
			\item
				Sei $T \in D'(\R^n)$, zeige $D^\alpha T \in D'(\R^n)$.

				$D^\alpha T$ ist linear, denn
				\begin{align*}
					D^\alpha T(a \phi + b \psi)
					&= (-1)^{|\alpha|} T(\nabla^{\alpha}(a \phi + b\psi)) \\
					&= (-1)^{|\alpha|} \big( aT(\nabla^\alpha \phi) + bT(\nabla^\alpha \psi) \big) \\
					&= a D^\alpha T(\phi) + b D^\alpha T(\psi).
				\end{align*}
				$D^\alpha$ ist stetig, denn für $\phi_n \stack D\to \phi$ gilt
				\[
					D^\alpha T(\phi_n)
					= (-1)^{|\alpha|} T(\nabla^\alpha \phi_n)
					\to D^\alpha T(\phi) \qquad \text{$T$ stetig}
				\]
			\item
				Nach \ref{5.11} ist $D^\alpha \big|_{C_0^\infty(\R^n)} = \nabla^\alpha$, also ist $D^\alpha$ Fortsetzung von $\nabla^\alpha$.
			\item
				Zeige die Eindeutigkeit:

				Sei $T \in D'(\R^n)$.
				Wähle Folge $(t_k)$ in $C_0^\infty(\R^n)$ mit $T_{t_k} \to T$ in $D'(\R^n)$.
				Da $D^\alpha$ stetige Fortsetzung sein soll, muss
				\begin{align*}
					&\implies \quad D^\alpha T_{t_k} \to D^\alpha T \\
					&\iff \quad \underbrace{D^\alpha T_{t_k}(\phi)}_{\mathclap{(-1)^{|\alpha|}T_{t_k}(\nabla^\alpha \phi)) \to (-1)^{|\alpha|}T(\nabla^\alpha \phi)}} \to D^\alpha T(\phi)
				\end{align*}
		\end{enumerate}
	\end{proof}
\end{st}

