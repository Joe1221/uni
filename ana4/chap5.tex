% This work is licensed under the Creative Commons
% Attribution-NonCommercial-ShareAlike 3.0 Unported License. To view a copy of
% this license, visit http://creativecommons.org/licenses/by-nc-sa/3.0/ or send
% a letter to Creative Commons, 444 Castro Street, Suite 900, Mountain View,
% California, 94041, USA.

\chapter{Distributionen}
\coursetimestamp{24}{6}{2013}


Betrachte $f : \R^3 \setminus \{0\} \to \R : x \mapsto \f 1{|x|}$.
Für $x \neq 0$ gilt
\begin{align*}
	\grad f &= - \f x{|x|^3} \\
	\Delta f = \div \grad f &= 0
\end{align*}
Physiker sagen: $-\Delta f = \delta_0$ (Diracsche $\delta$-Funktion):
„$\delta_0 = 0$ für $x \neq 0$, $\delta_0(0)$ so, dass $\int \delta_0(x)g(x) \dx = g(0)$“

Arbeitsplan: Erweitere $C^1(\R^n \to \C)$ zu einem topologischen Raum $\scr D'(\R^n)$ in dem gelten:
\begin{enumerate}[1)]
	\item
		$D_j : \scr D'(\R^n) \to \scr D'(\R^n) : f \mapsto \partial_{x_j} f$ sind stetige Abbildungen.
	\item
		Zeige: $f : x \mapsto \f 1{|x|}$, $\delta_0$ sind Elemente von $\scr D'(\R^3)$ und
		\[
			-\Delta f = \delta_0.
		\]
\end{enumerate}


\section{Konstruktion des Raumes}


\begin{conv} \label{5.1}
	Für $\alpha \in \N_0^n$ schreiben wir
	\[
		\nabla^\alpha := \partial_{x_1}^{\alpha_1} \dotso \partial_{x_n}^{\alpha_n}.
	\]
\end{conv}

\begin{df}[Raum der Testfunktionen] \label{5.2}
	Seien $\phi_j, \phi \in C_0^\infty(\R^n)$ ($j \in \N$).
	Dann heißt die Folge $(\phi_j)_{j\in \N}$ \emph{$\scr D$-konvergent} gegen $\phi$ (schreibe $\phi_j \stack {\scr D}\to \phi$), falls
	\begin{enumerate}[1)]
		\item
			eine kompakte Menge $K \subset \R^n$ existiert, sodass
			\[
				\forall j \in \N: \supp \phi_j \subset K,
			\]
		\item
			für alle $\alpha \in \N_0^n$:
			\[
				\|\nabla^\alpha \phi_j - \nabla^\alpha \phi\|_{L^\infty(\R^n)} \to 0
				\qquad \text{für $j \to \infty$,}
			\]
			$\nabla^\alpha \phi_j$ also auf $\R^n$ gleichmäßig gegen $\nabla^\alpha \phi$ konvergiert.
	\end{enumerate}
	(insbesondere ist auch $\supp \phi \subset K$)

	Der lineare Raum $C_0^\infty(\R^n)$ versehen mit diesem Konvergenzbegriff heißt \emph{Raum der Testfunktionen, $\scr D(\R^n)$}.
	Genauer gesagt, ist $\scr D(\R^n)$ ein lokalkonvexer topologischer Raum, siehe z.B. \cite{reedsimon80}).
\end{df}

\begin{st} \label{5.3}
	$\scr D(\R^n)$ ist vollständig.
	\begin{proof}
		Sei $(\phi_j)$ Cauchy-Folge in $\scr D(\R^n)$, d.h. $\exists K \subset \R^n \text{ kompakt } \forall j \in \N: \supp \phi_j \subset K$ und
		\[
			\forall \alpha \in \N_0^n \forall \eps > 0 \exists J_\eps \in \N \forall j,k > J_\eps : \|\nabla^\alpha \phi_j - \nabla^\alpha \phi_k \|_{L^\infty(\R^n)} < \eps.
		\]
		Für jedes feste $\alpha \in \N_0^n$ ist $(\nabla^\alpha \phi_j)_{j\in \N}$ also eine Cauchy-Folge in $C_0^\infty(\R^n \to \C)$ bezüglich $\|\argdot\|_{L^\infty(\R^n)}$.
		Damit konvergiert $\nabla^\alpha \phi_j$ gleichmäßig gegen einen Grenzwert $\psi_\alpha \in C(\R^n \to \C)$.

		Weil $\psi_0 = \lim_{n\to \infty} \nabla^0 \phi_j$ differenzierbar (alle $(\nabla^\alpha \phi_j)_{j\in\N}$ mit $|\alpha|=1$ konvergieren gleichmäßig gegen einen stetigen Grenzwert) und $\nabla^\alpha \psi_0 = \psi_\alpha \in C(\R^n \to \C)$ ist auch $\psi_0 \in C^\infty (\R^n \to \C)$.

		Wegen $\supp \phi_j \subset K$, $\phi_j \to \psi_0$ ist $\supp \psi_0 \subset K$ und somit $\psi_0 \in C_0^\infty(\R^n)$ und $\nabla^\alpha \psi_j \to \nabla^\alpha \psi_0$ bezüglich $\|\argdot\|_{L^\infty(\R^n)}$.
	\end{proof}
\end{st}

\begin{df}[Schwartzsche Distribution] \label{5.4}
	Eine \emph{(Schwartzsche) Distribution} $T$ ist eine lineare, stetige Abbildung $T: \scr D(\R^n) \to \C$.

	Die Menge der Schwartzschen Distributionen
	\[
		\big\{ T : \scr D(\R^n) \to \C \suchthat \text{$T$ linear und stetig} \big\}
	\]
	wird durch
	\begin{align*}
		(T+S)(\phi) &:= T\phi + S\phi,  \qquad T,S \in \scr D(\R^n);\\
		(\alpha T)(\phi) &:= \alpha T\phi,  \qquad T\in \scr D(\R^n), \alpha \in \C
	\end{align*}
	zu einem linearen Raum, dem \emph{Raum $\scr D'(\R^n)$ der Schwartzschen Distributionen}.

	Eine Folge $(T_n)$ in $\scr D'(\R^n)$ heißt konvergent gegen $T \in \scr D'(\R^n)$, falls
	\[
		\forall \phi \in C_0^\infty(\R^n) : T_n\phi \to T \phi
	\]
	$\scr D'(\R^n)$ ist ein topologischer Dualraum mit der von $\scr D(\R^n)$ induzierten Topologie; 
	obiger Konvergenzbegriff gehört zu dieser Topologie.
\end{df}


\begin{ex} \label{5.5}
	\begin{enumerate}[1)]
		\item
			Für festes $x_0 \in \R^n$ heißt
			\[
				\delta_{x_0}(\phi) := \phi(x_0) \qquad \text{für $\phi \in C_0^\infty(\R^n)$}
			\]
			\emph{Diracsche $\delta$-Distribution}.

			$\delta_{x_0}$ ist linear:
			\[
				\delta_{x_0}(\alpha \phi + \beta \psi) 
				= (\alpha \phi + \beta \psi)(x_0)
				= \alpha \phi(x_0) + \beta \psi(x_0)
				= \alpha \delta_{x_0}(\phi) + \beta \delta_{x_0}(\psi).
			\]
			$\delta_{x_0}$ ist stetig: Es gelte $\phi_j \overset{\scr D}\to \phi$, also insbesondere $\phi_j \to \phi$ gleichmäßig und punktweise, also
			\[
				\delta_{x_0}(\phi_j) = \phi_j(x_0) \stack{j\to \infty}\longrightarrow  \phi(x_0) = \delta_{x_0}(\phi).
			\]
		\item
			Sei
			\[
				L_{\text{loc}}^1 (\R^n) := \Big\{ u : \R^n \to \C  \suchthat \forall K \subset \R^n  \text{ kompakt} : u\big|_K \in L^1(K) \Big\}
			\]
			(z.B. $f(x) = \f 1{|x|}$, dann ist $f \in L_{\text{loc}}^1(\R^n)$).

			Zu $u \in L_{\text{loc}}^1(\R^n)$ setze
			\[
				T_u(\phi) := \int_{\R^n} u(x) \phi(x) \dx \qquad \text{für $\phi \in C_0^\infty(\R^n)$}.
			\]
			$T_u : C_0^\infty(\R^n) \to \C$ ist linear.

			$T_u$ ist stetig:
			Sei $\phi_j \overset{\scr D}{\to} \phi$, insbesondere $\supp \phi_j \subset K$, $\supp \phi \subset K$, also
			\begin{align*}
				|T_u \phi_j - T_u \phi| 
				&= \Big| \int_{\R^n} u (\phi_j-\phi) \dx \Big|
				= \Big| \int_{K} u (\phi_j-\phi) \dx \Big| \\
				&\le \underbrace{\|\phi_j - \phi\|_{L^\infty(\R^n)}}_{\to 0} \underbrace{\int_{K} |u| \dx}_{\in \R \text{ da $u\in L_{\text{loc}}^1(\R^n)$}}
			\end{align*}
			Also $T_u \phi_j \to T_u \phi$.
	\end{enumerate}
\end{ex}


\begin{st} \label{5.6}
	$\scr D'(\R^n)$ ist vollständig.
	\begin{proof}
		Sei $(T_n)$ Cauchy-Folge in $\scr D'(\R^n)$, d.h.
		\[
			\forall \phi \in C_0^\infty(\R^n) : (T_n \phi) \text{ Cauchy-Folge in $\C$}
		\]
		(insbesondere $T_n \phi$ konvergent).

		Definiere $T\phi := \lim_{n\to \infty} (T_n \phi)$.
		Dann ist $T$ linear, da $T_n$ linear.
		$T$ ist stetig, siehe dazu Satz von Banach-Steinhaus oder direkt in \cite{walter94}.
	\end{proof}
\end{st}

\section{Einbettung klassischer Funktionenräume}


\begin{st} \label{5.7}
	Sei $T_u$ wie in \ref{5.5} 2):
	\[
		T_u(\phi) := \int_{\R^n} u(x) \phi(x) \dx \qquad \text{für $\phi \in C_0^\infty(\R^n)$}.
	\]
	Dann ist die Abbildung $L_{\text{loc}}^1(\R^n) \ni u \mapsto T_u \in \scr D'(\R^n)$ linear und injektiv.
	\begin{proof}
		\begin{enumerate}[1)]
			\item
				Linearität: einfache Übung.
			\item
				Injektivität:
				
				Es gilt $T_v = T_w \iff T_v - T_w = 0 \iff T_{v-w} = 0$. 

				Sei also $T_u = 0$ und zeige $u = 0$ fast überall in $\R^n$.
				\begin{enumerate}[a)]
					\item
						Abschneiden:

						Wähle $\psi_R \in C_0^\infty(\R^n)$ mit $0 \le \psi_{R}(x) \le 1$ und $\psi_R(x) = 1$ für $x \in \_{K_R(0)}$.
						Dann ist
						\[
							T_{\psi_R u} (\phi) 
							= \int_{\R^n} (\psi_R u) \phi \dx 
							= T_u (\underbrace{\psi_R \phi}_{\mathclap{\in C_0^\infty(\R^n)}})
							= 0,
						\]
						also $\psi_R u \in L^1(\R^n), T_{\psi_R u} = 0$ und $\supp(\psi_R u) \subset \_{K_{R+1}(0)}$.
					\item
						Approximation:

						Setze
						\begin{align*}
							j(x) &:= \begin{cases}
								c e^{-\f 1{1-|x|^2}} & |x| < 1 \\
								0 & |x| \ge 1
							\end{cases} \qquad \text{mit }
							c := \dfrac 1{\int_{|x|< 1} e^{-\f 1{1-|x|^2}} \dx} \\
							j_\eps(x) &:= \f 1{\eps^n} j(\f x\eps)
						\end{align*}
						Also $j_\eps \in C_0^\infty(\R^n), \supp j_\eps = \_{K_\eps(0)}, j_\eps(x) \ge 0, \int_{\R^n} j_\eps \dx = 1$.

\coursetimestamp{26}{6}{2013}
						Für festes $y\in \R^n$ ist $j_\eps(y-\argdot) \in C_0^\infty(\R^n)$.
						Da $T_{\psi_R u} = 0$ ist
						\[
							0 = T_{\psi_R u} (j_\eps(y-\argdot))
							= \int_{\R^n} \psi_R(x) u(x) j_\eps(y-x) \dx
							= J_\eps(\psi_R u)(y)
						\]
						mit Glättungsoperator $J_\eps$.
						Nach \ref{4.20} ist
						\[
							\|\psi_R u \|_{L^1(\R^n)}
							= \|\psi_R u - \underbrace{J_\eps(\psi_R u)}_{=0} \|_{L^1(\R^n)}
							\to 0 \qquad \text{für $\eps \searrow 0$.}
						\]
						also $\|\psi_R u \|_{L^1(\R^n)} = 0$, bzw. $\psi_R u = 0$ fast überall in $\R^n$.
					\item
						Insbesondere ist
						\[
							\my \Big(\big\{ x \in \R^n : |x| \le R \land u(x) \neq 0 \big\} \Big) = 0
							\qquad \forall R > 0,
						\]
						also
						\begin{align*}
							\my(\{x\in \R^n : u(x) \neq 0\})
							&= \my \Big( \bigcup_{k=1}^\infty \{|x| \le k : u(x) \neq 0\} \Big) \\
							&\le \sum_{k=1}^\infty \my \big( \{|x|\le k : u(x)\neq 0\}\big)
							= 0
						\end{align*}
						und somit $u= 0$ fast überall in $\R^n$.
				\end{enumerate}
		\end{enumerate}
	\end{proof}
\end{st}

\begin{df} \label{5.8}
	Wir identifizieren $u$ mit $T_u$.
	Dann ist $L_{\text{loc}}^1 (\R^n) \subset \scr D'(\R^n)$ (insbesondere $C^1(\R^n \to \C) \subset \scr D'(\R^n)$).

	Eine Distribution $T \in \scr D'(\R^n)$ heißt \emph{regulär}, falls
	\[
		\exists u \in L_{\text{loc}}^1 (\R^n) : T = T_u,
	\]
	sonst \emph{singulär}.

	Man schreibt für $T(\phi)$ oft auch $(T,\phi)$ und falls $T = T_u$ regulär auch $(T_u, \phi)$, oder $(u, \phi)$.
\end{df}

\begin{ex} \label{5.9}
	\begin{enumerate}[1)]
		\item
			$\delta_{x_0} : \phi \mapsto \phi(x_0)$ ist singulär.
		\item
			Im $\R^1$ bezeichne
			\[
				T(\phi)
				:= \pv (\f 1x)(\phi)
				:= \CH \int_{-\infty}^\infty \f {\phi(x)}x \dx
				:= \lim_{\eps \searrow 0} \Big( \int_{-\infty}^{-\eps} \f{\phi(x)}x \dx + \int_{\eps}^\infty \f {\phi(x)}x \dx \Big)
			\]
			den \emph{Cauchy'schen Hauptwert} (oder \emph{principal value}).

			$T$ ist eine singuläre Distribution auf $\R$ (siehe \coursehref{blatt12.pdf}{Übungsaufgabe 12.2 b),c)}).
		\item
			Im $\R^3$ ist $T := T_{\f 1{|x|}}$ regulär.
	\end{enumerate}
\end{ex}

\begin{st} \label{5.10}
	$C_0^\infty(\R^n)$ ist dicht in $\scr D'(\R^n)$.
	\begin{proof}
		Sei $T \in \scr D'(\R^n)$. Konstruiere $t_k \in C_0^\infty(\R^n)$ mit $t_k \to T$, bzw. $T_{t_k} \to T$.

		Wähle $\psi_R \in C_0^\infty(\R^n)$ mit $0 \le \psi_R(x) \le 1$ und $\psi_R = 1$ für $|x| \le R$ und $\psi_R = 0$ für $|x| \ge R + 1$.
		Setze
		\[
			t_k(x) := \psi_k (x) T\big(j_{\f 1k}(x - \argdot)\big)
		\]
		Dann ist
		\begin{enumerate}[a)]
			\item
				$t_k \in C_0^\infty(\R^n)$:

				Es ist $\supp(t_k) \subset \sup(\psi_k) \subset \_{K_{R+1}(0)}$.

				Zeige die Differenzierbarkeit von $T(j_{\f 1k}(x-\argdot))$ bezüglich $x$.
				Sei $x \in \R^n$ fest, $\alpha \in \N_0^n$, dann konvergiert
				\begin{align*}
					\dfrac {(\nabla^\alpha j_{\f 1k})(x+he_j - y) - (\nabla^\alpha j_{\f 1k})(x-y)}h
					\to (\partial_{x_j} \nabla^\alpha j_{\f 1k})(x-y)
				\end{align*}
				gleichmäßig bezüglich $y \in \R^n$ für $h\to 0$.
				Außerdem ist
				\[
					\supp \bigg( \dfrac{j_{\f 1k}(x+he_j - \argdot) - j_{\f 1k}(x-\argdot)}h \bigg)
					\;\subset \_{K_{h+ \f 1k}(x)} \;
					\stack[k \ge 1]{|h| \le 1}{\subset} K_2(x).
				\]
				Also gilt für festes $x \in \R^n$, $k \in \N$
				\[
					\dfrac {j_{\f 1k} (x+he_j - \argdot) - j_{\f 1k}(x-\argdot)}h
					\stack{\scr D}\longrightarrow \partial_{x_j} j_{\f 1k} (x-\argdot)
				\]
				und da $T$ stetig
				\begin{align*}
					&\dfrac {T(j_{\f 1k}(x+he_j-\argdot)) - T(j_{\f 1k}(x-\argdot))}h \\
					&\qquad = T \bigg( \dfrac {j_{\f 1k} (x+he_j - \argdot) - j_{\f 1k}(x-\argdot)}h \bigg)
					\to T \Big( \partial_{x_j} j_{\f 1k} (x-\argdot) \Big).
				\end{align*}
				Also ist
				\[
					\f{\partial}{\partial x_j} T (j_{\f 1k}(x-\argdot)) = T(\partial_{x_j} j_{\f 1k}(x-\argdot)).
				\]
				Damit ist $T(j_{\f 1k}(x-\argdot)) \in C^\infty(\R^n \to \C)$ und schließlich $t_k \in C_0^\infty(\R^n)$.
			\item
				Zeige: $\forall \phi \in C_0^\infty(\R^n) : T_{t_k}(\phi) \to T(\phi)$ (dies ist der Konvergenzbegriff in $\scr D'(\R^n)$, siehe \ref{5.4}).

				Es gilt
				\begin{align*}
					T_{t_k}(\phi) 
					&= \int_{\R^n} t_k(x) \phi(x) \dx \\
					&= \int_{\R^n} \psi_k(x) T(j_{\f 1k}(x-\argdot)) \phi(x) \dx \\
					&= \int\limits_{|x|\le k+1} \psi_k(x) T(j_{\f 1k}(x-\argdot)) \phi(x) \dx \qquad (\supp \psi_k \subset \_{K_{k+1}(0)}) \\
					&= \lim_{n\to\infty} \sum_{l=0}^{M(n)} \psi_k(\xi_l^{(n)}) T\big(j_{\f 1k}(\xi_l^{(n)}-\argdot)\big) \phi(\xi_l^{(n)})\my(I_l^{(n)}) \displaybreak[0]\\
					&= \lim_{n\to\infty} T \underbrace{\bigg( \sum_{l=0}^{M(n)} \psi_k(\xi_l^{(n)}) j_{\f 1k}(\xi_l^{(n)}-\argdot) \phi(\xi_l^{(n)})\my(I_l^{(n)}) \bigg)}_{\overset{\scr D}\to \int\limits_{|x|\le k+1} \psi_k(x) j_{\f 1k}(x-\argdot) \phi(x) \dx} \\
					&= T \bigg(\underbrace{ \int\limits_{|x|\le k+1} \psi_{k}(x) j_{\f 1k} (x-\argdot) \phi(x) \dx }_{\stack[k \to \infty]{\scr D}\longrightarrow \phi} \bigg)\\
					&\to T(\phi) \qquad (k \to \infty)
				\end{align*}
			\item
				$T_{t_k} \to T$ in $\scr D'(\R^n)$:
		\end{enumerate}
	\end{proof}
\end{st}


\section{Differentiation}


\begin{lem} \label{5.11}
	Für $u \in C_0^\infty(\R^n), \alpha \in \N_0^n$ gilt $T_{\nabla^\alpha u} (\phi) = (-1)^{|\alpha|} T_u(\nabla^\alpha \phi)$.
	\begin{proof}
		Für $\alpha = e_j$ (Standardbasis), dann ist
		\begin{align*}
			T_{\nabla^{e_j} u}(\phi) 
			&= \int_{\R^n} (\partial_{x_j} u) \phi \dx \\
			&= - \int_{\R^n} u \partial_{x_j} \phi \dx \qquad \text{(part. Int. in $j$-ter Richtung, Randterme verschwinden)} \\
			&= - T_u (\partial_{x_j} \phi).
		\end{align*}
	\end{proof}
\end{lem}

\begin{st} \label{5.12}
	Die Abbildungen
	\[
		\nabla^\alpha : C_0^\infty(\R^n) \to C_0^\infty(\R^n)
	\]
	sind stetig bezüglich des Konvergenzbegriffs in $\scr D'(\R^n)$.
	\begin{proof}
		Zeige: Wenn $u_k, u \in C_0^\infty(\R^n)$ und $T_{u_k} \to T_u$ konvergiert, dann konvergiert $T_{\nabla^\alpha u_k} \to T_{\nabla^\alpha u}$.

		Für $\phi \in C_0^\infty(\R^n)$ ist
		\begin{align*}
			T_{\nabla^\alpha u_k} (\phi) 
			&= (-1)^{|\alpha|} T_{u_k} (\underbrace{\nabla^\alpha \phi}_{\in C_0^\infty(\R^n)}) \\
			&\to (-1)^{|\alpha|} T_u(\nabla^\alpha \phi)
			= T_{\nabla^\alpha u}(\phi).
		\end{align*}
	\end{proof}
\end{st}

\begin{st} \label{5.13}
	Für $\alpha \in \N_0^n$ ist die Abbildung $\nabla^\alpha: C_0^\infty(\R^n) \to C_0^\infty(\R^n)$ eindeutig fortsetzbar zu einer stetigen Abbildung $D^\alpha: \scr D'(\R^n) \to \scr D'(\R^n)$.
	Die Fortsetzung ist gegeben durch
	\[
		(D^\alpha T)(\phi) := (-1)^{|\alpha|} T(\nabla^\alpha \phi)
	\]
	und heißt \emph{Distributionenableitung} oder \emph{schwache Ableitung}.
	\begin{proof}
		\begin{enumerate}[1)]
			\item
				Sei $T \in \scr D'(\R^n)$, zeige $D^\alpha T \in \scr D'(\R^n)$.

				$D^\alpha T$ ist linear, denn
				\begin{align*}
					D^\alpha T(a \phi + b \psi)
					&= (-1)^{|\alpha|} T(\nabla^{\alpha}(a \phi + b\psi)) \\
					&= (-1)^{|\alpha|} \big( aT(\nabla^\alpha \phi) + bT(\nabla^\alpha \psi) \big) \\
					&= a D^\alpha T(\phi) + b D^\alpha T(\psi).
				\end{align*}
				$D^\alpha$ ist stetig, denn für $\phi_n \stack {\scr D}\to \phi$ gilt
				\[
					D^\alpha T(\phi_n)
					= (-1)^{|\alpha|} T(\nabla^\alpha \phi_n)
					\to D^\alpha T(\phi) \qquad \text{$T$ stetig}
				\]
			\item
				Nach \ref{5.11} ist $D^\alpha \big|_{C_0^\infty(\R^n)} = \nabla^\alpha$, also ist $D^\alpha$ Fortsetzung von $\nabla^\alpha$.
			\item
				Zeige die Eindeutigkeit:

				Sei $T \in \scr D'(\R^n)$.
				Wähle Folge $(t_k)$ in $C_0^\infty(\R^n)$ mit $T_{t_k} \to T$ in $\scr D'(\R^n)$.
				Da $D^\alpha$ stetige Fortsetzung sein soll, muss
				\begin{align*}
					&\implies \quad D^\alpha T_{t_k} \to D^\alpha T \\
					&\iff \quad \underbrace{D^\alpha T_{t_k}(\phi)}_{\mathclap{(-1)^{|\alpha|}T_{t_k}(\nabla^\alpha \phi)) \to (-1)^{|\alpha|}T(\nabla^\alpha \phi)}} \to D^\alpha T(\phi)
				\end{align*}
		\end{enumerate}
	\end{proof}
\end{st}


\coursetimestamp{1}{7}{2013}
\begin{nt} \label{5.14}
	\begin{enumerate}[1)]
		\item
			Jede Distribution ist beliebig oft differenzierbar (Definition gilt für alle $\alpha \in \N_0^n$).
			Insbesondere sind alle $f \in L_{\text{loc}}^1(\R^n)$ beliebig oft schwach differenzierbar.
		\item
			Es gilt der Satz von Schwartz:
			\[
				\forall \alpha, \beta \in \N_0^n : D^\alpha (D^\beta T) = D^\beta (D^\alpha T) = D^{\alpha + \beta} T,
			\]
			denn
			\begin{align*}
				D^\alpha (D^\beta T)(\phi)
				&= (-1)^{|\alpha|}(D^\beta T) (\nabla^\alpha \phi) \\
				&\stack[\text{in $C_0^\infty$}]{\text{Schwartz}}= (-1)^{|\alpha|}(-1)^{|\beta|} T(\nabla^\beta(\nabla^\alpha \phi)) \\
				&= (-1)^{|\alpha|+|\beta|} T(\nabla^{\alpha + \beta}) \\
				&= (-1)^{|\alpha + \beta|} T(\nabla^{\alpha + \beta}) \\
				&= (D^{\alpha + \beta} T)(\phi)
			\end{align*}
	\end{enumerate}
\end{nt}

\begin{ex} \label{5.15}
	\begin{enumerate}[1)]
		\item
			Sei $f : \R \to \C$ gegeben durch
			\[
				x \mapsto \begin{cases}
					x & x \ge 0 \\
					0 & x < 0
				\end{cases}.
			\]			
			Dann ist $D^1 f = h $ mit
			\[
				h(x) = \begin{cases}
					1 & x > 0 \\
					0 & x < 0 \\
					\text{egal} & x = 0
				\end{cases},
			\]
			denn
			\begin{align*}
				(\underbrace{D^1 T_f}_{\mathclap{\text{identifiziert mit $D^1 f$}}})(\phi) 
				&= - T_f (\phi') \\
				&= - \int_0^\infty x \phi'(x) \dx \\
				&= \underbrace{- x \phi(x) \Big|_{x=0}^\infty}_{= -0 + 0 \cdot \phi(0) = 0} + \int_{0}^\infty \phi(x) \dx \\
				&= \int_{-\infty}^\infty  h(x) \phi(x) \dx
				= T_h(\phi)
			\end{align*}
			Also $D^1 T_f = T_h$ und $D^1 f = h$.

			Für die zweite Ableitung gilt
			\begin{align*}
				D^2 f = D^1 h = \delta_0,
			\end{align*}
			denn
			\begin{align*}
				D^1 T_h(\phi)
				= - T_h(\phi')
				= - \int_0^\infty 1 \phi'(x) \dx
				= - 0 + \phi(0)
				= \delta_0(\phi).
			\end{align*}

			Bei höheren Ableitungen gilt
			\[
				D^k T_f(\phi)
				= D^{k-2} \delta_0(\phi)
				= (-1)^{k-2} \delta_0(\phi^{(k-2)})
				= (-1)^{k-2} \phi^{(k-2)}(0) \qquad k \ge 2
			\]
		\item
			Sei $f : \R \to \C$, $f \in C^1(]-\infty, x_0[)$ und $f \in C^1(]x_0,\infty[)$ mit existierenden Grenzwerten $f(x_0\pm 0), f'(x_0 \pm 0)$.
			Dann ist
			\[
				Df = f' + \big(f(x_0+0) - f(x_0-0)\big) \delta_{x_0}
			\]
			Wobei $f'$ die klassische Ableitung für $x \neq x_0$ ist und ihr Wert in $x = x_0$ egal ist.

			Also führt ein „Knick“ in der schwachen Ableitung zu einer Sprungstelle erster Art, und eine Sprungstelle erste Art führt zur $\delta$-Distribution (selber nachrechnen).
		\item
			Im $\scr D'(\R^n)$ gilt
			\[
				(D^\alpha \delta_{x_0}) \phi
				= (-1)^{|\alpha|} \delta_{x_0} (\nabla^\alpha \phi)
				= (-1)^{|\alpha|} (\nabla^\alpha \phi) (x_0).
			\]
		\item
			Betrachte den $\scr D'(\R^3)$.
			Definiere $f : \R^3 \to \C$ durch
			\[
				x \mapsto \begin{cases}
					\f 1{|x|} & x \neq 0 \\
					0
				\end{cases}
			\]
			Dann ist $f \in L_\text{loc}^1(\R^3)$:

			Für $x \neq 0$ ist $\grad f = - \f {x}{|x|^3}$ und $\nabla f = \div \grad f = 0$.

			Berechne $\Delta_W f = (-D^{(2,0,0)} - D^{(0,2,0)} - D^{(0,0,2)}) f$.
			Für $\phi \in C_0^\infty(\R^3)$ gilt
			\begin{align*}
				(-\Delta_W T_f)(\phi)
				&= (-D^{(2,0,0)}T_f)(\phi) - (D^{0,2,0} T_f)(\phi) - (D^{(0,0,2)}T_f)(\phi) \\
				&= - \big((-1)^{|(2,0,0)|} T_f(\nabla^{(2,0,0)}\phi) + T_f (\nabla^{(0,2,0)\phi}) + T_f (\nabla^{(0,0,2)}\phi)\big) \\
				&= - T_f( \nabla^{(2,0,0)}\phi + \nabla^{(0,2,0)}\phi + \nabla^{(0,0,2)}\phi) \\
				&= - T_f(\Delta \phi).
			\end{align*}

			Es gilt die \emph{Greensche Formel} (Beweis dazu später) für Mannigfaltigkeiten $S \subset \R^n$:
			\[
				\int_S (f \cdot \Delta g - \Delta f \cdot g) \dx
				= \int_{\partial S} \big((f \nabla g) \cdot n_0 - (g \nabla f) \cdot n_0\big) \dx[\sigma],
			\]
			wobei $n_0$ der Normaleneinheitsvektor auf $\partial S$ ist, der ins Äußere von $S$ zeigt.

			Sei $R$ so groß, dass $\supp \phi \subset K_R(0)$, dann gilt
			\begin{align*}
				T_f (\Delta \phi)
				&= \int_{\R^3} \f 1{|x|} \Delta \phi(x) \dx \\
				&= \int_{|x| \le R} \f 1{|x|} \Delta \phi(x) \dx \\
				&= \lim_{\eps \searrow 0} \int_{\eps < |x| < R} \f 1{|x|} \Delta \phi(x) \dx \\
				&= \lim_{\eps \searrow 0} \int_{\eps < |x| < R} \f 1{|x|} \Delta \phi(x) - \underbrace{(\Delta \f 1{|x|})\phi(x)}_{= 0} \dx
			\intertext{
				mit $S = \{ x \in \R^3 : \eps \le |x| \le \R\}$ und $\partial S = \{x : |x| = \eps \lor |x| = R\}$ lässt sich die Greensche Formel anwenden:
			}
				&= \lim_{\eps \searrow 0} \int_{|x|=\eps} \Big( \big(\f 1{|x|} \grad \phi \big) \cdot (-\f x{|x|}) - \phi(x) (- \f 1{|x|^3} x)\cdot (-\f x{|x|}) \Big) \dx[\sigma]
					+ \underbrace{\int_{|x| = R} \ldots \dx[\sigma]}_{= 0 \text{, da $\supp \phi \subset K_R(0)$}} \\
				&= \lim_{\eps \searrow 0} \int_{|x|= \eps}  \Big( - \f x{|x|} \cdot \grad \phi(x) - \phi(x) \f 1{|x|^2} \Big) \dx[\sigma] \\
				&\stack{\text{s.u.}}= - 4\pi \phi(0).
			\end{align*}
			Also $T_f(\Delta \phi) = - 4\pi \phi(0) = - 4\pi \delta_0(\phi)$, bzw. $-\Delta_w T_f = 4\pi \delta_0$, bzw. $\Delta_w f = 4\pi \delta_0$.

			Zeige die letzte Gleichheit in obiger Umformung:
			\begin{align*}
				\bigg| \int_{|x|=\eps} \f x{|x|^2} \cdot \grad \phi(x) \dx[\sigma] \bigg|
				&\le \int_{|x|= \eps} \big| \f x{|x|^2} \cdot \grad \phi(x) \big| \dx[\sigma] \\
				&\le \int_{|x|= \eps} \underbrace{|\f x{|x|^2}|}_{= \f 1\eps} \cdot |\grad \phi(x)| \dx[\sigma] \\
				&\le \int_{|x|=\eps} \f c \eps \dx[\sigma]\\
				&\stack{\eps \le 1}\le \f c\eps \int_{|x|=\eps} 1 \dx[\sigma] \\
				&= \f c\eps \cdot 4 \pi \eps^2
				\to 0 \qquad \text{für $\eps \searrow 0$.}
			\end{align*}
			Außerdem ist
			\begin{align*}
				\bigg| \int_{|x| = \eps} \f 1{|x|^2} (\phi(x) - \phi(0)) \dx[\sigma] \bigg|
				&\le \f 1{\eps^2} \max_{|x|\le \eps} |\phi(x) - \phi(0)| \cdot 4\pi \eps^2 \\
				&\to 0 \qquad \text{für $\eps \searrow 0$, da $\phi$ stetig in $x = 0$}
			\end{align*}
			für $\eps \searrow 0$, da $\phi$ stetig in $x=0$ und
			\[
				\int_{|x|=\eps} \f 1{|x|^2} \phi(0) \dx
				= \f 1{\eps^2} \phi(0) \cdot 4\pi \eps^2
				= 4 \pi \phi(0).
			\]

			Beweise nun noch die Greenschen Formel:
			seien $f, g \in C^2$, dann ist
			\begin{align*}
				\div \big( f \grad g - g \grad f \big)
				&= (\grad f)(\grad g) + f \div \grad g
					-(\grad g)(\grad f) - g \div \grad f \\
				&= f \Delta g - g \Delta f
			\end{align*}
			und damit
			\begin{align*}
				\int_{S} (f \Delta g - g \Delta f) \dx
				&= \int_{S} \div \big( f \grad g - g \grad f\big) \dx \\
			\intertext{mit dem Satz von Gauß-Ostrogradski gilt dann}
				&= \int_{\partial S} n_0 (f \grad g - g \grad f) \dx[\sigma] \\
				&= \int_{\partial S} \big( f\underbrace{\grad g}_{=\nabla g} \cdot n_0 - g \underbrace{\grad f}_{\nabla f} \cdot n_0  \big) \dx[\sigma].
			\end{align*}
	\end{enumerate}
\end{ex}

\begin{st} \label{5.16}
	\begin{enumerate}[1)]
		\item
			Sei $T_j \to T$ in $\scr D'(\R^n)$, $\alpha \in \N_0^n$, dann
			\[
				D^\alpha T_j \to D^\alpha T
			\]
			in $\scr D'(\R^n)$.
		\item
			Sei $\sum_{j=1}^\infty T_j = T$ in $\scr D'(\R^n)$, $\alpha \in \N_0^n$, dann ist
			\[
				\sum_{j=1}^\infty D^\alpha T_j = D^\alpha T.
			\]
	\end{enumerate}
	\begin{proof}
		\begin{enumerate}[1)]
			\item
				$T_j \to T$ bedeutet $\forall \phi \in C_0^\infty(\R^n) : T_j(\phi) \to T(\phi)$.
				Mit $\phi \in C_0^\infty(\R^n)$ ist auch $\nabla^\alpha \phi \in C_0^\infty (\R^n)$ und damit
				\begin{align*}
					D^\alpha T_j(\phi)
					&= (-1)^{|\alpha|} T_j(\nabla^\alpha \phi) \\
					&\to (-1)^{|\alpha|} T(\nabla^\alpha \phi) \qquad (j\to \infty) \\
					&= D^\alpha T(\phi)
				\end{align*}
				Also $D^\alpha T_j \to D^\alpha T$.
			\item
				Folgt aus 1).
		\end{enumerate}
	\end{proof}
\end{st}

\begin{ex} \label{5.17}
	\begin{enumerate}[1)]
		\item
			Sei $f_j: \R \to \C$ gegeben durch
			\begin{align*}
				x \mapsto \begin{cases}
					0 & x \le 0 \\
					x^j & 0 < x \le 1 \\
					1 & x > 1
				\end{cases}.
			\end{align*}
			dann konvergiert $f$ punktweise für $x \in \R$ gegen
			\[
				f: x \mapsto \begin{cases}
					0 & x < 1 \\
					1 & x \ge 1
				\end{cases}
			\]
			In $\scr D'(\R)$ konvergiert $f_j \to f$, denn für $\phi \in C_0^\infty(\R)$ gilt
			\begin{align*}
				T_{f_j}(\phi) 
				&= \int_{\R} f_j(x) \phi(x) \dx \\
				&= \underbrace{\int_{0}^1 x^j \phi(x) \dx}_{\mathclap{|\argdot| \le \max_{\R}|\phi| \int_0^1 x^j \dx = \max |\phi| \f 1{j+1} \to 0}} + \int_{1}^\infty \phi(x) \dx \\
				&\to \int_{1}^\infty \phi(x) \dx
				= T_f(\phi).
			\end{align*}
			Für die Ableitung gilt
			\begin{align*}
				D^1  f_j = \begin{cases}
					j \cdot x^{j-1} & 0 < x < 1 \\
					0 & \text{sonst}
				\end{cases},
			\end{align*}
			denn
			\begin{align*}
				D^1 T_{f_j}(\phi)
				&= \int_{0}^1 j x^{j-1} \phi(x) \dx \\
				&= \underbrace{x^j \phi(x) \Big|_0^1}_{=1 \phi(1) - 0} - \underbrace{\int_{0}^1 x^j \phi'(x) \dx}_{\to 0} \\
				&\to \phi(1) = \delta_1(\phi).
			\end{align*}
			Also $D^1 f_j \to \delta_1 = D^1 f$.
\coursetimestamp{3}{7}{2013}
		\item
			Betrachte
			\[
				f(x) = \f \pi4 |x|
				\qquad (-\pi < x \le \pi)
			\]
			$2\pi$-periodisch fortgesetzt.
			
			Aus der Bedingung von Dini konvergiert die Fourierreihe in jedem Punkt:
			\[
				f(x) = g(x) := \f {\pi^2}8 t ( \cos x + \f 1{3^2} \cos (3x) + \f 1{5^2} \cos(5x) + \dotsb )
			\]
			für alle $x \in \R$.
			Die Fourierreihe ist sogar gleichmäßig konvergent (nach Weierstraß: $\sum_{k=1}^\infty) \f 1{(2k+1)^2} < \infty$).

			\begin{align*}
				D^1 f &= \begin{cases}
					\f \pi4 & 0 < x < \pi \\
					-\f \pi4 & -\pi < x < 0
				\end{cases}, \\
				D^1 g &= \sin x + \f 1{3} \sin (3x) + \f 15 \sin (5x) + \dotsb, \\
				g'(x) &= \sin x + \f 1{3} \sin (3x) + \f 15 \sin (5x) + \dotsb \\
				&\stack[\text{Dini}]{\text{verallg.}}= \begin{cases}
					\f \pi4 & 0 < x < \pi \\
					0 & x = \pi \\
					-\f \pi4 & -\pi < x < 0
				\end{cases} \qquad \text{$2\pi$-periodisch fortgesetzt}.
			\end{align*}
			Aus \ref{5.16} folgt
			\[
				T_f = T_{\f {\pi^2}8} - T_{\cos x} - T_{\f {\cos (3x)}{3^2}} - \dotsb
			\]
			und für die Ableitungen
			\[
				D^k f = D^k T_{\f {\pi^2}8} - D^k T_{\cos x} - \dotsb.
			\]
			Also
			\begin{align*}
				D^2 g &= \cos x + \cos (3x) + \cos (5x) + \dotsb \\
				&\stack[\scr D'(\R)]{\text{in}}= D^2 f \\
				&= \f \pi 2 \delta_0 - \f \pi2 \delta_\pi + \f \pi2 \delta_{2\pi} + \dotsb \\
				& \qquad - \f \pi 2 \delta{-\pi} + \f \pi 2 \delta_{-2\pi} + \dotsb.
			\end{align*}
			Also gilt in $\scr D'(\R)$:
			\[
				\sum_{j=1}^\infty \cos(2j + 1) x 
				= \sum_{k\in \Z} \f \pi 2 (-1)^k \delta_{k\pi}
			\]
			Anwendung auf $\phi \in C_0^\infty(\R)$ liefert dann eine endliche Summe (über diejenigen $k$, für die $k \pi \in \supp \phi$).
	\end{enumerate}
\end{ex}

\begin{st}[Definition und Satz] \label{5.18}
	Sei $a \in C^\infty (\R^n \to \C)$
	\begin{enumerate}[1)]
		\item
			Für $T \in \scr D'(\R^n)$ ist $a T \in \scr D'(\R^n)$ definiert durch
			\[
				a T (\phi) := T(a \phi)
				\qquad \phi \in C_0^\infty (\R^n).
			\]
		\item
			Für $u \in L_{\text{loc}}^1 (\R^n)$ gilt
			\[
				a T_u = T_{au}.
			\]
		\item
			Die Abbildung $\scr D'(\R^n) \to \scr D'(\R^n) : T \mapsto a T$ ist stetig und damit die eindeutige stetige Fortsetzung der Abbildung $\scr D(\R^n) \to \scr D(\R^n) : \phi \mapsto a \phi$.
	\end{enumerate}
	\begin{proof}
		\begin{enumerate}[1)]
			\item
				Siehe \coursehref{blatt11.pdf}{Übungsaufgabe 11.5a}.
			\item
				Es gilt
				\begin{align*}
					a T_u (\phi)
					= T_u (a \phi)
					= \int_{\R^n} u a \phi \dx
					= T_{au} (\phi)
				\end{align*}
				Damit gilt folgenden kommutative Diagramm:
				\begin{align*}
					L_{\text{loc}}(\R^n) \ni u &\mapsto a u \in L_{\text{loc}}^1(\R^n) \\
					\scr D'(\R^n) \ni T_u &\mapsto a T u \in \scr D'(\R^n)
				\end{align*}

				Zur Verdeutlichung betrachte $\Phi_a : \scr D'(\R^n) \to \scr D'(\R^n) : T \mapsto a T$.
				Die Aussage von 2) besagt, dass $\Phi_a |_{\im(L_\text{loc}^1(\R^n))} : T \mapsto a T$ über die Identifikation von $u$ mit $T_u$ nicht anderees ist, als die Abbildung
				\[
					L_{\text{loc}}^1 (\R^n) \ni u \mapsto a u \in L_{\text{loc}}^1(\R^n)
				\]
				Also ist $\Phi_a$ eine Fortsetzung der bekannten Multiplikation $a u$ für $a \in C^\infty, u \in L_{\text{loc}}^1(\R^n)$ auf $\scr D'(\R^n)$.
				Wenn $\Phi_a$ stetig ist, dann ist $\Phi_a$ eindeutige stetige Fortsetzung, da $L_{\text{loc}}^1(\R^n)$ dicht in $\scr D'(\R^n)$ ist.
			\item
				Zeige: $\Phi_a$ ist stetig.
				Sei dazu $T_n \to T$ in $\scr D'(\R^n)$, dann gilt für alle $\phi \in C_0^\infty(\R^n) : T_n(\phi) \to T(\phi)$.
				Also auch 
				\[
					a T_n(\phi) = T_n(a\phi) \to T(a\phi) = a T(\phi)
				\]
				und somit $a T_n \to a T_n$, bzw. $\Phi_a (T_n) \to \Phi_a (T)$.
		\end{enumerate}
	\end{proof}
\end{st}

\begin{st}[Leibnitz-Regel] \label{5.19}
	Für $\alpha \in \N_0^n$ gilt
	\[
		D^\alpha (a T) = \sum_{\substack{\beta \in \N_0^n \\ \beta \le \alpha}} \binom{\alpha}{\beta} (\nabla^{\alpha-\beta} a) D^\beta T.
	\]
	\begin{proof}
		Siehe Übungen.
	\end{proof}
\end{st}


\section{Lokales Verhalten von Distributionen}

\begin{df} \label{5.20}
	Sei $O \subset \R^n$ offen.
	\begin{enumerate}[1)]
		\item
			Definiere
			\[
				C_0^\infty(O) := \big\{ \phi \in C^\infty(O \to \C) : \supp \phi \text{ kompakt } \land \supp \phi \subset O \big\}.
			\]
			(insbesondere $d(\supp \phi, \R^n \setminus O) > 0$)
		\item
			Zu $\phi \in C_0^\infty (O)$ sei 
			\[
				\tilde \phi(x) := \begin{cases}
					\phi(x) & x \in O \\
					0 & \text{sonst}
				\end{cases}
			\]
			Offensichtlich ist $\tilde \phi \in C_0^\infty (\R^n)$.
		\item
			Für $S,T \in \scr D'(\R^n)$ setzen wir
			\[
				S = T \text{ (in $O$)} \; :\iff \;
				\forall \phi \in  C_0^\infty (O) : T(\tilde \phi) = S(\tilde \phi).
			\]
	\end{enumerate}
\end{df}

\begin{ex} \label{5.21}
	\begin{enumerate}[1)]
		\item
			Sei $u,v \in L_{\text{loc}}^1(\R^n)$, $u =v $ in $O$, dann ist $T_u = T_v$ in $O$.
		\item
			Es gilt
			\[
				\delta_0 = 0
				\qquad \text{in $\R \setminus \{0\}$},
			\]
			denn wegen $\phi \in C_0^\infty (\R \setminus \{0\})$ ist 
			\[
				\delta_0(\tilde \phi) = \tilde \phi(0) = 0 = T_0(\tilde \phi)
			\]
	\end{enumerate}
\end{ex}

\begin{st} \label{5.22}
	Für $T \in \scr D(\R^n)$ sei
	\[
		G := \bigcup_{\mathclap{O \in \{O \subset \R^n : T = 0 \text{ in O}\}}} O.
	\]
	Dann gilt
	\begin{enumerate}[1)]
		\item
			$G$ ist offen.
		\item
			$T = 0$ in $G$.
	\end{enumerate}
	\begin{proof}
		\begin{enumerate}[1)]
			\item
				Beliebige Vereinigungen offenere Mengen sind offen.
			\item
				Sei $\phi \in C_0^\infty (G)$.
				$\{ O \subset \R^n : T = 0 \text{ in $O$}\}$ ist offene Überdeckung von $\supp \phi$ und $\supp \phi$ ist kompakt.
				Also
				\[
					\exists O_1, \dotsc, O_N \subset \R^n : T = 0 \text{ in $O_j$ } \land \supp \phi \subset \bigcup_{j=1}^N O_j.
				\]
				Mit der Zerlegung der Eins folgt: es gibt $\psi_j \in C_0^\infty (O_j)$, so dass $\sum_{j=1}^N \tilde \psi_j (x) = 1$ für $x \in \supp \phi$.
				Schreibe
				\[
					\tilde \phi = \sum_{j=1}^N \tilde \psi_j \tilde \phi.
				\]
				Dann ist
				\[
					T(\tilde \phi) 
					= \sum_{j=1}^N T(\underbrace{\tilde \psi_j \tilde \phi}_{\in C_0^\infty(O_j)}) 
					= 0,
				\]
				da $T = 0$ in $O_j$.
		\end{enumerate}
	\end{proof}
\end{st}

\begin{df} \label{5.23}
	Für $T \in \scr D'(\R^n)$ setzte
	\[
		\supp(T) := \R^n \subset \bigcup_{\mathclap{O \in \{O \subset \R^n \text{ offen} : T = 0 \text{ in $O$}\}}} O.
	\]
	$\supp (T)$ heißt \emph{Träger} oder \emph{Support} von $T$.

	$x \in \supp(T)$ heißt \emph{wesentlicher Punkt} von $T$.

	Falls $\supp(T)$ kompakt ist, heißt $T$ \emph{finit}.
\end{df}

\begin{ex} \label{5.24}
	\begin{enumerate}[1)]
		\item
			Es gilt
			\[
				\supp (\delta_0) = \{ 0 \}.
			\]
		\item
			Sei $\phi \in C_0^\infty(\R^n)$, dann ist
			\[
				\supp T_\phi = \supp \phi,
			\]
			neue und alte Definition stimmen also überein.
		\item
			Für $u \in L_{\text{loc}}^1(\R^n)$ ist (ohne Beweis)
			\[
				\supp(T_u) = \R^n \setminus \big\{x \in \R^n : \exists O \subset \R^n \text{ offen } : x \in O \land u = 0 \text{ in $O$ fast überall} \big\}.
			\]
	\end{enumerate}
\end{ex}

\begin{st} \label{5.25}
	Es sei $T : C_0^\infty (\R^n) \to \C$ linear.
	Dann sind folgende Aussagen äquivalent:
	\begin{enumerate}[(i)]
		\item
			$T \in \scr D'(\R^n)$
		\item
			Für alle kompakten $K \subset \R^n$ gilt
			\[
				\exists k \in \N, c > 0 \forall \phi \in C_0^\infty(\R^n) : \;
				\supp \phi \subset K \implies |T\phi| \le c \sup_{|\alpha| \le k} \|\nabla^\alpha \phi\|_\infty
			\]
	\end{enumerate}
	\begin{proof}
		\begin{seg}[(ii)$\implies$(i)]
			Zeige: $T : D(\R^n) \to \C$ ist stetig.

			Sei $\phi_j \stack D\to \phi$, d.h. es existiert $K \subset \R^n$ mit $\supp \phi_j, \supp \phi \subset K$ und für alle $\alpha \in \N_0^n$ konvergiert $\nabla^\alpha \phi_j \to \nabla^\alpha \phi$ gleichmäßig auf $\R^n$.
			Zeige $T(\phi_j) \to T(\phi)$:
			\begin{align*}
				|T(\phi_j) - T(\phi)|
				= | T(\phi_j-\phi)|
				\le c \sup_{|\alpha| \le k} \| \nabla^\alpha (\phi_j - \phi) \|_\infty
				\to 0,
			\end{align*}
			da $\nabla^\alpha \phi_j \to \nabla^\alpha \phi$ gleichmäßig.
		\end{seg}
		\begin{seg}[(i)$\implies$ (ii)]
			Zeige $\lnot (ii) \implies \lnot (i)$.
			Es gelte also
			\begin{align*}
				\exists K \forall k \in \N \forall c > 0 \exists \phi \in C_0^\infty(\R^n)
				: \supp \phi \subset K \land |T\phi| > c \sup_{|\alpha|\le k} \|\nabla^\alpha \phi \|_\infty
			\end{align*}
			Wähle $c = k$, dann ist $\phi_k \in C_0^\infty (\R^n)$ mit $\supp \phi_k \subset K$ und $|T_{\phi_k}| > k \sup_{|\alpha| \le k} \|\nabla^\alpha \phi\|_\infty$.
			Setze
			\[
				\psi_k := \dfrac 1{k \max\limits_{|\alpha| \le k}\| \nabla^\alpha \phi\|_\infty} \phi_k.
			\]
			Für alle $\beta \in \N_0^n$ konvergiert für $k \ge |\beta|$
			\[
				|D^\beta \psi_k(x)| \le \f 1k \to 0
			\]
			gleichmäßig.
			Außerdem ist $\supp \psi_k = \supp \phi_k \subset K$, also $\psi_k \stack D\to 0$.
			Es gilt $\lnot (T(\psi_k) \to 0)$, denn
			\[
				T(\psi_k) = \dfrac 1{k \max\|\nabla^\alpha \phi_k\|_\infty} T(\phi_k) \ge 1.
			\]
			Damit ist $T$ nicht stetig, also $T \in \scr D'(\R^n)$.
		\end{seg}
	\end{proof}
\end{st}


\coursetimestamp{8}{7}{2013}
\begin{kor} \label{5.26}
	Ist $T \in D'(\R^n)$ finit, so gilt
	\[
		\exists k \in \N_0, c > 0 \forall \phi \in C_0^\infty(\R^n):
		|T(\phi)| \le c \max_{|\alpha| \le k} \|\nabla_\phi^\alpha \|_\infty.
	\]
	\begin{proof}
		Wähle $\psi \in C_0^\infty(\R^n)$ mit $\forall x \in \supp T : \psi(x) = 1$.
		Dann ist
		\[
			T(\phi)
			= T(\psi \phi + (1-\psi) \phi)
			= T(\psi \phi) + \underbrace{T(\underbrace{(1-\psi) \phi}_{\supp (\argdot) \subset \R^n \setminus \supp T})}_{=0}
			= T(\psi \phi).
		\]
		Für jedes $\phi \in C_0^\infty(\R^n)$ gilt
		\[
			\supp(\psi \phi)
			\subset \supp \psi
			=: K.
		\]
		Wende jetzt \ref{5.25} (ii) an, also
		\[
			\exists k \in \N_0, c > 0 :
			|T(\phi)| = |T(\psi\phi)| \subset c \max_{|\alpha| \le k} \| \nabla^\alpha (\psi\phi) \|_\infty.
		\]
		Es gilt
		\begin{align*}
			|\nabla^\alpha (\psi\phi)(x)|
			= \bigg| \sum_{\beta \le \alpha} \binom{\alpha}{\beta} \underbrace{(\nabla^\beta \psi(x))}_{\le c_\beta \text{ für $x\in \R^n$}} \nabla^{\alpha - \beta} \phi(x) \bigg|
			\le \tilde c \max_{|\gamma|\le k} \|\nabla^\gamma \phi\|_\infty.
		\end{align*}
		und damit
		\[
			\|\nabla^\alpha (\psi \phi)\|
			\le \tilde c \max_{|\gamma| \le k} \|\nabla^\gamma \phi \|_\infty.
		\]
	\end{proof}
\end{kor}

\begin{df} \label{5.27}
	$T \in D'(\R^n)$ heißt \emph{von endlicher Ordnung}, falls die Aussage aus \ref{5.26} erfüllt ist:
	\[
		\exists k \in \N_0, c > 0 \forall \phi \in C_0^\infty(\R^n):
		|T(\phi)| \le c \max_{|\alpha| \le k} \|\nabla_\phi^\alpha \|_\infty.
	\]
	Das kleinste solche $k \in \N_0$ heißt \emph{Ordnung} von $T$.
\end{df}

\begin{ex} \label{5.28}
	\begin{enumerate}[1)]
		\item
			Die Delta-0-Distribution ($\delta_0(\phi) = \phi(0)$) hat Ordnung 0.
		\item
			Sei $u \in L^1(\R^n)$, dann ist
			\[
				|T_u(\phi)|
				\le \int_{\R^n} |u \phi| \dx
				\le \|\phi\|_\infty \int_{\R^n} |u| \dx
				= \|\phi\|_\infty \|u\|_{L^1(\R^n)}
			\]
			$T_u$ hat also Ordnung 0.
		\item
			Betrachte $D^\alpha \delta_0$:
			\[
				D^\alpha \delta_0(\phi)
				= (-1)^{|\alpha|} \delta_0(\nabla^\alpha \phi)
				= (-1)^{|\alpha|} (\nabla^\alpha \phi)(0).
			\]
			$D^\alpha \delta_0$ hat also Ordnung $|\alpha|$.
		\item
			Sei $u(x) = x$, dann ist $u \in L^1_{\text{loc}}(\R)$.

			$T_u$ ist nicht von endlicher Ordnung (trotzdem ist $D^2 T_u = T_{u''} = 0$).

			Wir geben eine Folge $(\phi_j)$ in $C_0^\infty(\R^n)$ an mit Folgengliedern, die jeder Abschätzung aus \ref{5.27} widersprechen, unabhängig davon wie groß $k$ und $c$ sind.
			Definiere
			\[
				\phi_l := j_1 ( \argdot - l).
			\]
			Dann gilt
			\begin{align*}
				T_u (\phi_j)
				&= \int_{\R} u \phi_j \dx \\
				&= \int_{l-1}^{l+1} x j_1(x-l) \dx \\
				&\ge (l-1) \int_{l-1}^{l+1} j_1(x-l) \dx
				= (l-1) \int_{-1}^1 j_1(x) \dx \\
				&= l-1
				\to \infty.
			\end{align*}
			Also $|T_u(\phi_l)| \to \infty$ für $l \to \infty$.
			Jedoch ist
			\[
				| \nabla^\alpha \phi_l(x) |
				= |\nabla^\alpha j_1(x-l) |
				\le \|\nabla^\alpha j_1 \|_\infty.
			\]
			also $\|\nabla^\alpha \phi_l \|_\infty \le \const$ unabhängig von $l$.
			Damit kann $T_u$ nicht von endlicher Ordnung sein.
	\end{enumerate}
\end{ex}

\begin{nt} \label{5.29}
	Sei $T \in D'(\R^n), O \subset \R^n$ offen.
	Dann existiert $\alpha \in \N_0^n$, $f \in C(\R^n \to \C)$, so dass
	\[
		T = D^\alpha T_f
		\qquad \text{auf $O$.}
	\]
	\begin{proof}
		Siehe \cite{walter94}.
	\end{proof}
\end{nt}

\begin{nt}[Die Ableitung ist eine lokale Operation] \label{5.30}
	Falls $F = S$ in $O$, dann auch $D^\alpha T = D^\alpha S$ in $O$ für alle $\alpha \in \N_0^n$.
	\begin{proof}
		Für $\phi \in C_0^\infty, \alpha \in \N_0^n$ gilt
		\[
			\nabla^\alpha \tilde \phi
			= \widetilde{\nabla^\alpha \phi}
		\]
		Für alle $\phi \in C_0^\infty(O)$ gilt
		\begin{align*}
			D^\alpha T(\tilde \phi)
			= (-1)^{|\alpha|} T(\nabla^{\alpha} \tilde \phi)
			&= (-1)^{|\alpha|} T( \widetilde{\nabla^{\alpha} \phi} ) \\
			&= (-1)^{|\alpha|} S( \widetilde{\nabla^{\alpha} \phi} )
			= (-1)^{|\alpha|} S(\nabla^{\alpha} \tilde \phi)
			= D^\alpha S(\tilde \phi).
		\end{align*}
	\end{proof}
\end{nt}

\begin{ex}[Anwendung von \ref{5.30}] \label{5.31}
	Betrachte den $\R^1$.
	Sei $u(x) = x, v(x) = -x, w(x) = |x|$.
	Dann ist
	\[
		T_w = T_u
		\qquad \text{in $O_1 = ]0,\infty[$.}
	\]
	Also auch
	\[
		D^1 T_w = D^1 T_u = T_{u'}
		\qquad \text{in $O_1$.}
	\]
	Analog auch $D^1 T_w = T_{v'}$ in $O_2 = ]-\infty, 0[$.
	Damit ist
	\[
		D^1 T_w
		= T_{w'}
		\quad \text{mit} \quad
		w'(x) :=
		\begin{cases}
			1 & x > 0 \\
			-1 & x < 0 \\
			\text{egal} & x =0
		\end{cases}
	\]
\end{ex}


\section{Direktes Produkt von Distributionen}


\begin{nt} \label{5.32}
	Sei $\phi_1 \in C_0^\infty(\R^n), \phi_2 \in C_0^\infty(\R^m)$, dann ist mit
	\[
		\phi_1 \times \phi_2 (x,y)
		:= \phi_1(x) \phi_2(y).
	\]
	$\phi_1 \times \phi_2 \in C_0^\infty(\R^{n+m})$ und $\supp(\phi_1 \times \phi_2) = (\supp \phi_1) \times (\supp \phi_2)$.
	Wir wollen diese Operation auf $D'(\R^n), D'(\R^m), D'(\R^{n+m})$ fortsetzen.
\end{nt}

\begin{lem} \label{5.33}
	Sei $S \in D'(\R^m), \phi \in C_0^\infty(\R^{n+1})$.
	Setze für $x \in \R^n$
	\[
		\psi(x)
		:= S(\underbrace{\phi(x, \argdot)}_{\in C_0^{\R^m}})
		=: S_y(\phi(x,y)).
	\]
	Dann gelten
	\begin{enumerate}[1)]
		\item
			Für $\alpha \in \N_0^n$ ist
			\[
				\nabla^\alpha \psi = S(\nabla^{(\alpha,0)} \phi(x, \argdot)).
			\]
			Insbesondere $\psi \in C^\infty(\R^n \to \C)$.
		\item
			$\psi \in C_0^\infty(\R^n)$
		\item
			Falls $\phi_j \stack{D(\R^{n+m})}\longrightarrow \phi$, $\psi_j(x) = S(\phi_j(x, \argdot))$, dann gilt
			\[
				\psi_j \stack{D(\R^n)}\longrightarrow \psi
			\]
	\end{enumerate}
	\begin{proof}
		\begin{enumerate}[1)]
			\item
				Verläuft analog zu \coursehref{blatt12.pdf}{Übungsaufgabe 12.4}.
			\item
				Es gelte $\supp \phi \in [-R,R]^{n+m}$.
				Für $|x_1| \ge R \lor |x_2| \ge R \lor \dotsb$ gilt $\phi(x) = 0$ und damit auch $\psi(x) = 0$.
				Also $\supp \psi \subset [-R,R]^n$.
			\item
				Es gelte $\supp \phi_j \subset [-R,R]^{n+m}$.
				Nach 2) ist dann $\supp \psi_j \subset [-R,R]^n$ für alle $j \in \N$.

				Weiter gilt
				\begin{align*}
					| \nabla^\alpha \psi_j(x) - \nabla^\alpha \psi(x) |
					&\stack{1)}= \Big|S\big(\nabla^{(\alpha,0)} \phi_j(x,\argdot)\big) - S\big(\nabla^{(\alpha, 0)} \phi(x, \argdot)\big) \Big| \\
					&= \Big|S\big(\underbrace{\overbrace{\nabla^{(\alpha, 0)}(\phi_j(x,\argdot) - \phi(x,\argdot))}^{=:g_{j,x}}}_{\supp(\argdot) \subset [-R,R]^n =: K}\big)\Big|
				\intertext{
					mit \ref{5.25} (ii) angewendet auf $S \in D'(\R^n)$ ergibt sich
				}
					&= c \max_{\substack{|\beta|\le k \\ \beta \in \N_0^m}} \|\nabla^{\beta} g_{j,x} \|_\infty \\
					&= c \max_{|\beta| \le k} \| \nabla^{(\alpha, \beta)} (\phi_j(x,\argdot) - \phi(x, \argdot)) \|_{\infty} \\
					&\le c \max_{|\beta| \le k} \|\nabla^{(\alpha, \beta)}  (\phi_j - \phi) \|_{L^\infty(\R^{n+m})} \\
					&\to 0,
				\end{align*}
				da $\phi_j \stack{D(\R^{n+m})}\longrightarrow \phi$.
		\end{enumerate}
	\end{proof}
\end{lem}
