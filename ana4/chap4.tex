% This work is licensed under the Creative Commons
% Attribution-NonCommercial-ShareAlike 3.0 Unported License. To view a copy of
% this license, visit http://creativecommons.org/licenses/by-nc-sa/3.0/ or send
% a letter to Creative Commons, 444 Castro Street, Suite 900, Mountain View,
% California, 94041, USA.

\chapter{Fouriertransformation}

\section{Grundlagen}


\begin{df} \label{4.1}
	\begin{enumerate}[1)]
		\item
			Für $f \in C^\infty(\R \to \C)$, $j,k \in \N_0$ sei
			\[
				\|f\|_{j,k} := \sup_{x\in \R} |x^j f^{(k)}(x)|
			\]
			(für $k \ge 1$ Halbnorm).
			Der \emph{Schwartz-Raum} über $\R$ ist
			\[
				\scr S (\R) := \Big\{ f \in C^\infty(\R \to \C) \suchthat  \forall j,k \in \N_0 : \|f\|_{j,k} < \infty \Big\}
			\]
		\item
			Für $f : \R \to \C$ ist
			\[
				\supp(f) := \_{\{x \in \R : f(x) \neq 0}
			\]
			der \emph{Träger} oder \emph{Support} von $f$.

			Der Testraum über $\R$ ist
			\[
				C_0^\infty(\R) := \{f \in C^\infty(\R\to\C) : \supp(f) \text{ ist kompakt} \Big\}
			\]
			Offensichtlich ist $C_0^\infty (\R) \subset \scr S(R)$.
	\end{enumerate}
\end{df}

\begin{ex} \label{4.2}
	\begin{enumerate}[1)]
		\item
			$f(x) = x^j e^{-\alpha x^2}, j \in \N_0, \alpha > 0$, dann ist $f \in \scr S(\R) \setminus \C_0^\infty (\R)$.
		\item
			Sei
			\[
				\tilde j_1(x) := \begin{cases}
					e^{-\tf 1{1-x^2}} & -1 < x < 1 \\
					0 & \text{sonst}
				\end{cases}
			\]
			Dann ist $\supp(\tilde j_1) = [-1,1]$, $\tilde j_1 \in C^\infty(\R \to \R)$, also $\tilde j_1 \in C_0^\infty(\R)$.
			Außerdem $\tilde j_1(x) \ge 0$.

			Setze $j_1(x) := c \tilde j_1(x)$ mit
			\[
				c = \dfrac 1{\int_{-1}^1 e^{- \f 1{1-x^2}} \dx}.
			\]
			Dann hat $j_1$ die zusätzliche Eigenschaft $\int_{\R} j_1(x) \dx = 1$.

			Setze
			\[
				j_\eps(x) := \f 1\eps j_1(\f x\eps).
			\]
			Dann ist $\supp(j_\eps) = [-\eps, \eps]$, $\int_\R j_\eps \dx = 1$, $j_\eps \in C_0^\infty$.
		\item
			Abschneidefunktionen

			Definiere für $R > \eps > 0$
			\[
				\psi_{R, \eps}(x) := \int_{-R-\eps}^{R+ \eps} j_\eps(x-y) \dx[y].
			\]
			Für $-R \le x \le \R$ ist
			\begin{align*}
				\psi_{R, \eps}(x) 
				&= \int_{x-\eps}^{x+\eps} j_\eps (x-y) \dx[y] \\
				&= \int_{-\infty}^\infty j_\eps(x-y) \dx[y] \\
				&\stack{\xi = x-y} = -\int_{+\infty}^{-\infty}  j_\eps (\xi) \dx[\xi] = 1
			\end{align*}
			Für $x \ge R + 2\eps$ oder $x \le - R - 2\eps$ ist
			\[
				\psi_{R,\eps}(x) = \int 0 \dx[y] = 0.
			\]
			Es gilt
			\[
				\f {\mathrm{d}^k}{\mathrm{d}x^k} \psi_{R,\eps}(x) 
				= \int_{-R-\eps}^{R+\eps} j_\eps^{(k)} (x-y) \dx[y],
			\]
			also $\psi_{R,\eps} \in C_0^\infty (\R)$.

			Der Betrag der Ableitung lässt sich abschätzen:
			\begin{align*}
				\Big| \f {\mathrm{d}^k}{\mathrm{d}x^k} \psi_{R,\eps}(x) \Big|
				&\le \int_{R-\eps}^{R+\eps} |j_\eps^{(k)}(x-y)| \dx[y] \\
				&\stack{\xi = x-y}\le \int_{-\eps}^\eps |j_\eps^{(k)}(\xi)| \dx[\xi] \\
				&\le c(\eps),
			\end{align*}
			unabhängig von $R$ und $x$.
	\end{enumerate}
\end{ex}

\begin{nt} \label{4.3}
	Mit
	\[
		d(f,g) := \sum_{j,k = 0}^\infty \f 1{2^{j+k}} \f {\|f-g\|_{j,k}}{1+ \|f-g\|_{j,k}}
	\]
	ist $\scr S(\R)$ ein vollständiger metrischer Raum
\end{nt}

\begin{nt}[Eigenschaften] \label{4.4}
	\begin{enumerate}[1)]
		\item
			$\scr S(\R)$ ist linearer Raum über $\C$.
		\item
			Für $f,g \in \scr S (\R)$ ist $fg \in \scr S(\R)$.
			Insbesondere ist mit 1) $\scr S(\R)$ Algebra ohne Einselement.
		\item
			Für $f \in \scr S(R)$, $j,k \in \N$, $g(x) := x^j f^{(k)}(x)$ ist $g \in \scr S(\R)$.
	\end{enumerate}
	Die selben Eigenschaften besitzt auch $C_0^\infty(\R)$.
\end{nt}

\begin{df} \label{4.5}
	Für $f \in \scr S (\R)$ ist
	\[
		Ff(\omega) := \hat f(\omega) := \f 1{\sqrt{2\pi}} \int_{-\infty}^\infty f(x) e^{i\omega x} \dx
	\]
	die \emph{Fourier-Transformierte} von $f$.
\end{df}

\begin{ex} \label{4.6}
	Sei $f(x) = e^{-\f {x^2}2}$, dann ist die Fourier-Transformierte $\hat f(\omega) = e^{-\f {\omega^2}2}$.

	Damit ist $f$ Eigenfunktion von $F$ zum Eigenwert $\lambda = 1$.
\end{ex}

\begin{st} \label{4.7}
	Für $f \in \scr S(\R)$, $j, k \in \N_0$ gelten:
	\begin{gather*}
		\widehat {f^{(j)}}(\omega) = (i \omega)^j \hat f(\omega) \\
		\begin{aligned}
			g(x) := x^k f(x) \quad &\implies \quad \hat g(\omega) = i^k \hat f^{(k)} (\omega) \\
			g(x) := x^k f^{(j)}(x) \quad &\implies \quad \hat g(\omega) = i^{j+k} \omega^j \hat f^{(k)} (\omega) \\
		\end{aligned}
	\end{gather*}
	\begin{proof}
		\begin{enumerate}[1)]
			\item
				Mit partieller Integration ergibt sich
				\begin{align*}
					\sqrt {2\pi} \hat{f'}(\omega) 
					&= \int_{-\infty}^\infty f'(x) e^{-i\omega x} \dx \\
					&= \underbrace{\Big[f(x) e^{-i\omega x}  \Big]_{x=-\infty}^\infty}_{=0 \text{ ($f \in \scr S$)}} + i\omega \underbrace{\int_{-\infty}^\infty f(x) e^{-i\omega x} \dx}_{= \sqrt{2\pi} \hat f(\omega)} \\
					&= \sqrt{2\pi} i \omega \hat f(\omega). \\
				\end{align*}
			\item
				Sei $g(x) = x f(x)$, dann gilt
				\begin{align*}
					\sqrt{2\pi} \hat g(x)
					&= \int_{-\infty}^\infty f(x) \underbrace{ xe^{-i \omega x}}_{= i \f{\mathrm d}{\mathrm d \omega} e^{-i \omega x}} \dx \\
					&= i \int_{-\infty}^\infty f(x) \f {\mathrm d}{\mathrm d \omega} (e^{-i \omega x}) \dx
				\intertext{
					Das Integral konvergiert gleichmäßig bezüglich $\omega$, da $|f(x) x e^{-i \omega x} | \le |x f(x)| \le \f c{1+x^2}$ da $f \in \scr S$.
				}
					&= i \f {\mathrm d}{\mathrm d \omega} \int_{-\infty}^\infty f(x) e^{-i \omega x} \dx \\
					&=\sqrt{2\pi} i \f {\mathrm d}{\mathrm d \omega} \hat f(\omega)
					= i \hat f^{(1)} (\omega) \sqrt{2\pi}
				\end{align*}
		\end{enumerate}
	\end{proof}
\end{st}

\begin{st} \label{4.8}
	Für $F : f \to \hat f$ ist eine lineare Abbildung von $\scr S(\R)$ in $\scr S(\R)$.

	\begin{proof}
		\begin{enumerate}[1)]
			\item
				$F$ linear ist klar.
			\item
				Sei $f \in \scr S(\R)$.
				Zeige $\hat f \in \scr S(\R)$.
				\begin{enumerate}[a)]
					\item
						Es gilt
						\begin{align*}
							|\hat f(\omega)|
							&= \f 1{\sqrt{2\pi}} \bigg| \int_{-\infty}^\infty f(x) e^{-i \omega x} \dx \bigg| \\
							&\le \f 1{\sqrt{2\pi}} \int_{-\infty}^\infty  |f(x)| \dx
							< \infty
						\end{align*}
						(wegen $|f(x)| \le \f {c}{1+x^2}$, da $f \in \scr S(\R)$)
						und damit $\sup_{\omega \in \R}|\hat f(\omega)| < \infty$.
					\item
						Es gilt
						\begin{align*}
							\sqrt{2\pi} \f {\mathrm d^k}{\mathrm d \omega^k} \hat f(\omega)
							&= \f {\mathrm d^k}{\mathrm d \omega^k} \hat f(\omega) \int_{-\infty}^\infty f(x) e^{i \omega x} \dx \\
							&= \int_{-\infty}^\infty \underbrace{f(x) \underbrace{\f {\mathrm d^k}{\mathrm d \omega^k} e^{- i \omega x}}_{= (-ix)^k e^{-i\omega x}}}_{|\argdot| \le \f {c}{1+|x|^{k+2}|x|^k}} \dx
						\end{align*}
						Das Abgeleitete Integral konvergiert gleichmäßig bezüglich $\omega$.
						Damit ist die Vertauschung von Ableitung und Integral gerechtfertigt.
						Außerdem hängt das Integral stetig von $\omega$ ab.

						Somit ist $\hat f \in C^\infty(\R \to \C)$.
				\end{enumerate}
		\end{enumerate}
		Es gilt
		\begin{align*}
			\sup_{\omega \in \R} |\omega^j \hat f^{(k)}(\omega) |
			&\stackrel{\ref{4.7}}= \sup_{\omega \in \R} \Big| \tf {1}{i^{j+k}} \hat g(\omega) \Big| \qquad \text{mit $g(x) = x^k f^{(j)}(x)$}
		\intertext{also insbesondere $g \in \scr S(\R)$}
			&\stack{\text{a)}}< \infty \qquad \text{für alle $j,k \in \N_0$}.
		\end{align*}
	\end{proof}
\end{st}

\begin{nt} \label{4.9}
	$F : \scr S(\R) \to \scr S(\R)$ ist stetig bezüglich der Metrik $d$ aus \ref{4.3}.
\end{nt}

\begin{st} \label{4.10}
	$F: \scr S(\R) \to \scr S(\R): f \mapsto \hat f$ ist bijektiv und $F^{-1}: \scr S(\R) \to \scr S(\R) : g \mapsto \check g$ ist gegeben durch
	\[
		\check g(x) := \f 1{\sqrt{2\pi}} \int_{-\infty}^\infty g(\omega) e^{i \omega x} \dx[\omega]
	\]
	\begin{proof}
		\begin{enumerate}[1)]
			\item
				Sei zunächst $f \in C_0^\infty(\R)$.
				Wähle $\eps > 0$, so dass $\supp f \subset ]-\f 1\eps, \f 1\eps[$.
				Setze $f$ $\f 2\eps$-periodisch auf $\R$ fort, dann ist $f \in C^\infty (\R \to \C)$.
				Aus \ref{3.35} folgt: für $y \in ]-\f 1\eps, \f 1\eps[$ gilt
				\begin{align*}
					f(y) 
					&= \sum_{j=-\infty}^\infty \f 1{2 \f 1\eps} \int_{-\f 1\eps}^{\f 1\eps} e^{ij\pi s \eps} f(s) \dx[s] e^{ij\pi y \eps} \qquad \text{(nicht-fortgesetztes $f$)}\\
					&= \sum_{j=-\infty}^\infty \underbrace{\f \eps 2 \sqrt{2\pi}}_{= \f {\Delta \omega_j}{\sqrt{2\pi}}} \hat f(\underbrace{j\pi \eps}_{\omega_j}) e^{\overbrace{ij\pi y \eps}^{i\omega_j y}} \\
					&\stack{\eps \searrow 0}= \f 1{\sqrt{2\pi}} \int_{-\infty}^\infty f(\omega) e^{i\omega y} \dx[\omega]
				\end{align*}
				Also gilt für $\forall y \in \R$:
				\[
					f(y) = \f 1{\sqrt{2\pi}} \int_{-\infty}^\infty \hat f(\omega) e^{i \omega y} \dx[\omega]
				\]
			\item
				Sei $f \in \scr S(\R)$.
				Multipliziere mit der Abschneidefunktion $\psi_{R,1}$ aus \ref{4.2}: $\psi_{R,1}\cdot f \in C_0^\infty(\R)$.
				Für $-R < y < R$ gilt
				\begin{align*}
					f(y) 
					&= (\psi_{R,1}\cdot f)(y)
					&\stack{1)}= \f 1{\sqrt{2\pi}} \int_{-\infty}^\infty \widehat{\psi_{R,1} \cdot f}(\omega) e^{i\omega y} \dx[\omega]
				\end{align*}
				\begin{align*}
					\widehat{\psi_{R,1} \cdot f}(\omega) 
					&= \f 1{\sqrt{2\omega}} \int_{-\infty}^\infty \psi_{R_1}(x) f(x) e^{-i \omega x} \dx \\
					&\stack{R\to \infty}\to \hat f(\omega)  \qquad \text{(für jedes feste $\omega$)}
				\end{align*}
				Für $|\omega| \ge 1$ ist
				\begin{align*}
					\Big|\widehat{\psi_{R,1} \cdot f}(\omega)  \Big|
					&= \Bigg| \f 1{\sqrt{2\pi}} \f 1{(-i\omega)^2} \int_{-\infty}^\infty (\psi_{R,1}\cdot f)''(x) e^{-i \omega x} \dx[x] \Bigg| \\
					&= \f 1{\sqrt{2\pi} \omega^2} \int_{-\infty}^\infty \underbrace{\Big| (\psi_{R,1} \cdot f)'' (x))\Big|}_{c (|f(x)| + |f'(x)| + |f''(x)| \le \f {\tilde c}{1+x^2}} 1 \dx \\
					&\le \f {c'}{\omega^2}.
				\end{align*}
				Für $|\omega| \le 1$ ist
				\begin{align*}
					\Big|\widehat{\psi_{R,1} \cdot f}(\omega)  \Big|
					&\le  \f 1{\sqrt{2\pi}} \int_{-\infty}^\infty \underbrace{|\psi_{R,1}(x)|}_{\le 1} |f(x)| |e^{-i\omega x}| \dx \\
					&\le c'' \qquad \text{(unabhängig von $R$).}
				\end{align*}
				Nach dem Satz von der majorisierten Konvergenz gilt also
				\begin{align*}
					f(y) 
					&= \lim_{R\to \infty} \f 1{\sqrt{2\pi}} \int_{-\infty}^\infty (\widehat{\psi_{R,1}\cdot f})(\omega) e^{ \omega y} \dx[\omega] \\
					&= \f 1{\sqrt{2\pi}} \int_{-\infty}^\infty \lim_{R\to \infty} (\widehat{\psi_{R,1}\cdot f})(\omega) e^{ \omega y} \dx[\omega] \\
					&= \f 1{\sqrt{2\pi}} \int_{-\infty}^\infty \hat f(\omega) e^{i \omega y} \dx[\omega]
				\end{align*}
				für alle $y \in \R$.
				Damit ist $F$ insbesondere injektiv.
			\item
				$F$ ist surjektiv:

				Sei $g \in \scr S(\R)$.
				Aus 2) folgt
				\begin{align*}
					g(\omega) 
					&= \f 1{\sqrt{2\pi}} \int_{-\infty}^\infty \hat g(x) e^{i\omega x} \dx \\
					&= \f 1{\sqrt{2\pi}} \int_{-\infty}^\infty \hat g(-x) e^{-i\omega x} \dx \\
				\end{align*}
				Wegen $g \in \scr S(\R)$ ist nach \ref{4.8} auch $\hat g \in \scr S(\R)$.
				Also $g = \widehat{\hat g(- \argdot)} \in \im(F)$.
		\end{enumerate}
	\end{proof}
\end{st}

\begin{ex}[Durchbiegung eines unendlich langen Balkens] \label{4.11}
	Sei $u(x)$ der Verlauf des Balkens und $f(x)$ die einwirkende Kraft an jedem Punkt.
	Es gilt folgende DGL:
	\[
		u^{(4)} + \alpha^4 u(x) = f(x)
	\]
	Wir nehmen an, dass $f, u \in \scr S(\R)$.
	Wegen $\widehat{u^{(k)}(\omega)} = (i\omega)^k \hat u(\omega)$ vereinfacht sich die DGL nach Fouriertransformation zu
	\begin{align*}
		\omega^4 \hat u(\omega) + \alpha^4 \hat u(\omega) &= \hat f(\omega) \\
		\iff \qquad  \hat u(\omega) &= \f 1{\omega^4 + \alpha^4} \hat f(\omega)
	\end{align*}
	Wegen $f \in \scr S(\R)$ ist auch $\hat f \in \scr S(\R)$.
	Mit $\phi(\omega) := \f 1{\omega^4 + \alpha^4}$ gelten $\phi \in C^\infty (\R \to \R)$ und $|\phi^{(k)}(\omega)| \le c_k$ (es reicht $\le |x^{N_k}|c_k$).
	Damit ist $\phi \hat f \in \scr S(\R)$ (Beweis: Übung) und somit
	\[
		u = (\phi \hat f)^{\vee} \in \scr S(\R)
	\]
	Also existiert die Lösung und ist eindeutig.

	Manche Leute würden die Lösung auch gerne ausrechnen können.
	Wie berechnet man $(\phi \psi)^{\vee}$ aus $\check \phi$, $\check \psi$?
\end{ex}

\begin{df} \label{4.12}
	Für $f, g \in \scr S(\R)$ ist
	\[
		(f \ast g)(y) := \int_{-\infty}^\infty f(y-x) g(x) \dx
	\]
	die \emph{Faltung} von $f$ mit $g$.
\end{df}

\begin{ex} \label{4.13}
	Betrachte die Abschneidefunktion aus \ref{4.2} 3):
	\begin{align*}
		\psi_{R,\eps} (x) 
		&:= \int_{-R-\eps}^{R+\eps} j_\eps (x-y) \dx[y],
	\intertext{diese lässt sich schreiben als}
		&= \int_{-\infty}^\infty j_\eps(x-y) \cdot  \chi_{[-R-\eps,R+\eps]}(y) \dx[y] \\
		&= \Big( j_\eps \ast \chi_{[-R-\eps, R+\eps]} \Big) (x)
	\end{align*}
\end{ex}

\begin{st} \label{4.13}
	Für $f, g,h  \in \scr S(\R)$ gelten
	\begin{enumerate}[1)]
		\item
			$f \ast g \in \scr S(\R)$
		\item
			Es gilt
			\begin{align*}
				\widehat{f\cdot g} &= \f 1{\sqrt{2\pi}} \hat f \ast \hat g, \\
				\widehat{f\ast g} &= \sqrt{2\pi} \hat f \cdot \hat g.
			\end{align*}
			Insbesondere auch $(\phi \psi)^\vee = \f 1{\sqrt{2\pi}} \check \phi \ast \check \psi$.
		\item
			Es gilt
			\begin{align*}
				f \ast (g \ast h) &=  (f\ast g) \ast h, \\
				f \ast g &= g \ast f.
			\end{align*}
	\end{enumerate}
	\begin{note}
		Mit dem Satz gilt für das Beispiel \ref{4.11}:
		\[
			u = \f 1{\sqrt{2\pi}}\check \phi \ast f
			= \f 1{\sqrt{2\pi}}\check \phi \ast (\hat f)^\vee.
		\]
		Damit ist $u$ explizit berechenbar mittels $(\f 1{\omega^4 + \alpha^4})^\vee$.
	\end{note}
	\begin{proof}
		\begin{enumerate}[1)]
			\item
				Wegen $f, g \in \scr S(\R)$ ist nach \ref{4.8} $\hat f, \hat g \in \scr S(\R)$, also $\hat f \cdot \hat g \in \scr S(\R)$, also nach \ref{4.10} $(\hat f \hat g)^\vee \in \scr S(\R)$ und somit
				\[
					f \ast g \stack{2)}= \sqrt{2\pi} (\hat f \hat g)^\vee \in \scr S(\R)
				\]
			\item
				Die Fouriertransformation erhält das Skalarprodukt (siehe später), daher gilt
				\begin{align*}
					\sqrt{2\pi} \widehat{f g}(\omega) 
					&= \int_{-\infty}^\infty f(x) g(x) e^{-i\omega x} \dx \\
					&= \< g, \_f e^{i\omega \argdot} \>_{L^2(\R)} \\
					&= \Big\< \hat g, \widehat{\_ f e^{i\omega \argdot}} \Big\>_{L^2(\R)} \\
					&= \int_{-\infty}^\infty \hat g(\tilde \omega) \_{\bigg( \f 1{\sqrt{2\pi}} \int_{-\infty}^\infty \_f(x) e^{i\omega x} e^{-i \tilde \omega x} \dx \bigg)} \dx[\tilde \omega] \\
					&= \int_{-\infty}^\infty \hat g(\tilde \omega) \underbrace{\f 1{\sqrt{2\pi}} \int_{-\infty}^{\infty} f(x) e^{-i(\omega - \tilde \omega)} \dx}_{=\hat f(\omega - \tilde \omega)} \dx[\tilde \omega] \\
					&= \int_{-\infty}^\infty \hat g(\tilde \omega)\hat f(\omega - \tilde \omega) \dx[\omega] \\
					&= (\hat f \ast \hat g)(\omega)
				\end{align*}
				Genauso folgt
				\[
					(\hat f \hat g)^\vee = \f 1{\sqrt{2\pi}} f \ast g.
				\]
			\item
				Es gilt
				\[
					(f\ast g) \ast h = \f 1{\sqrt{2\pi}} (\widehat {f \ast g}  \ast \hat h)^\vee = (\hat f \hat g) \hat h
				\]
		\end{enumerate}
	\end{proof}
\end{st}

