% This work is licensed under the Creative Commons
% Attribution-NonCommercial-ShareAlike 3.0 Unported License. To view a copy of
% this license, visit http://creativecommons.org/licenses/by-nc-sa/3.0/ or send
% a letter to Creative Commons, 444 Castro Street, Suite 900, Mountain View,
% California, 94041, USA.

\chapter{Fouriertransformation}

\section{Grundlagen}


\begin{df} \label{4.1}
	\begin{enumerate}[1)]
		\item
			Für $f \in C^\infty(\R \to \C)$, $j,k \in \N_0$ sei
			\[
				\|f\|_{j,k} := \sup_{x\in \R} |x^j f^{(k)}(x)|
			\]
			(für $k \ge 1$ Halbnorm).
			Der \emph{Schwartz-Raum} über $\R$ ist
			\[
				\scr S (\R) := \Big\{ f \in C^\infty(\R \to \C) \suchthat  \forall j,k \in \N_0 : \|f\|_{j,k} < \infty \Big\}
			\]
		\item
			Für $f : \R \to \C$ ist
			\[
				\supp(f) := \_{\{x \in \R : f(x) \neq 0}
			\]
			der \emph{Träger} oder \emph{Support} von $f$.

			Der Testraum über $\R$ ist
			\[
				C_0^\infty(\R) := \{f \in C^\infty(\R\to\C) : \supp(f) \text{ ist kompakt} \Big\}
			\]
			Offensichtlich ist $C_0^\infty (\R) \subset \scr S(R)$.
	\end{enumerate}
\end{df}

\begin{ex} \label{4.2}
	\begin{enumerate}[1)]
		\item
			$f(x) = x^j e^{-\alpha x^2}, j \in \N_0, \alpha > 0$, dann ist $f \in \scr S(\R) \setminus \C_0^\infty (\R)$.
		\item
			Sei
			\[
				\tilde j_1(x) := \begin{cases}
					e^{-\tf 1{1-x^2}} & -1 < x < 1 \\
					0 & \text{sonst}
				\end{cases}
			\]
			Dann ist $\supp(\tilde j_1) = [-1,1]$, $\tilde j_1 \in C^\infty(\R \to \R)$, also $\tilde j_1 \in C_0^\infty(\R)$.
			Außerdem $\tilde j_1(x) \ge 0$.

			Setze $j_1(x) := c \tilde j_1(x)$ mit
			\[
				c = \dfrac 1{\int_{-1}^1 e^{- \f 1{1-x^2}} \dx}.
			\]
			Dann hat $j_1$ die zusätzliche Eigenschaft $\int_{\R} j_1(x) \dx = 1$.

			Setze
			\[
				j_\eps(x) := \f 1\eps j_1(\f x\eps).
			\]
			Dann ist $\supp(j_\eps) = [-\eps, \eps]$, $\int_\R j_\eps \dx = 1$, $j_\eps \in C_0^\infty$.
		\item
			Abschneidefunktionen

			Definiere für $R > \eps > 0$
			\[
				\psi_{R, \eps}(x) := \int_{-R-\eps}^{R+ \eps} j_\eps(x-y) \dx[y].
			\]
			Für $-R \le x \le \R$ ist
			\begin{align*}
				\psi_{R, \eps}(x) 
				&= \int_{x-\eps}^{x+\eps} j_\eps (x-y) \dx[y] \\
				&= \int_{-\infty}^\infty j_\eps(x-y) \dx[y] \\
				&\stack{\xi = x-y} = -\int_{+\infty}^{-\infty}  j_\eps (\xi) \dx[\xi] = 1
			\end{align*}
			Für $x \ge R + 2\eps$ oder $x \le - R - 2\eps$ ist
			\[
				\psi_{R,\eps}(x) = \int 0 \dx[y] = 0.
			\]
			Es gilt
			\[
				\f {\mathrm{d}^k}{\mathrm{d}x^k} \psi_{R,\eps}(x) 
				= \int_{-R-\eps}^{R+\eps} j_\eps^{(k)} (x-y) \dx[y],
			\]
			also $\psi_{R,\eps} \in C_0^\infty (\R)$.

			Der Betrag der Ableitung lässt sich abschätzen:
			\begin{align*}
				\Big| \f {\mathrm{d}^k}{\mathrm{d}x^k} \psi_{R,\eps}(x) \Big|
				&\le \int_{R-\eps}^{R+\eps} |j_\eps^{(k)}(x-y)| \dx[y] \\
				&\stack{\xi = x-y}\le \int_{-\eps}^\eps |j_\eps^{(k)}(\xi)| \dx[\xi] \\
				&\le c(\eps),
			\end{align*}
			unabhängig von $R$ und $x$.
	\end{enumerate}
\end{ex}

\begin{nt} \label{4.3}
	Mit
	\[
		d(f,g) := \sum_{j,k = 0}^\infty \f 1{2^{j+k}} \f {\|f-g\|_{j,k}}{1+ \|f-g\|_{j,k}}
	\]
	ist $\scr S(\R)$ ein vollständiger metrischer Raum
\end{nt}

\begin{nt}[Eigenschaften] \label{4.4}
	\begin{enumerate}[1)]
		\item
			$\scr S(\R)$ ist linearer Raum über $\C$.
		\item
			Für $f,g \in \scr S (\R)$ ist $fg \in \scr S(\R)$.
			Insbesondere ist mit 1) $\scr S(\R)$ Algebra ohne Einselement.
		\item
			Für $f \in \scr S(R)$, $j,k \in \N$, $g(x) := x^j f^{(k)}(x)$ ist $g \in \scr S(\R)$.
	\end{enumerate}
	Die selben Eigenschaften besitzt auch $C_0^\infty(\R)$.
\end{nt}

\begin{df} \label{4.5}
	Für $f \in \scr S (\R)$ ist
	\[
		Ff(\omega) := \hat f(\omega) := \f 1{\sqrt{2\pi}} \int_{-\infty}^\infty f(x) e^{-i\omega x} \dx
	\]
	die \emph{Fourier-Transformierte} von $f$.
\end{df}

\begin{ex} \label{4.6}
	Sei $f(x) = e^{-\f {x^2}2}$, dann ist die Fourier-Transformierte $\hat f(\omega) = e^{-\f {\omega^2}2}$.

	Damit ist $f$ Eigenfunktion von $F$ zum Eigenwert $\lambda = 1$.
\end{ex}

\begin{st} \label{4.7}
	Für $f \in \scr S(\R)$, $j, k \in \N_0$ gelten:
	\begin{gather*}
		\widehat {f^{(j)}}(\omega) = (i \omega)^j \hat f(\omega) \\
		\begin{aligned}
			g(x) := x^k f(x) \quad &\implies \quad \hat g(\omega) = i^k \hat f^{(k)} (\omega) \\
		\end{aligned}
	\end{gather*}
	\begin{proof}
		\begin{enumerate}[1)]
			\item
				Mit partieller Integration ergibt sich
				\begin{align*}
					\sqrt {2\pi} \hat{f'}(\omega) 
					&= \int_{-\infty}^\infty f'(x) e^{-i\omega x} \dx \\
					&= \underbrace{\Big[f(x) e^{-i\omega x}  \Big]_{x=-\infty}^\infty}_{=0 \text{ ($f \in \scr S$)}} + i\omega \underbrace{\int_{-\infty}^\infty f(x) e^{-i\omega x} \dx}_{= \sqrt{2\pi} \hat f(\omega)} \\
					&= \sqrt{2\pi} i \omega \hat f(\omega). \\
				\end{align*}
			\item
				Sei $g(x) = x f(x)$, dann gilt
				\begin{align*}
					\sqrt{2\pi} \hat g(x)
					&= \int_{-\infty}^\infty f(x) \underbrace{ xe^{-i \omega x}}_{= i \f{\mathrm d}{\mathrm d \omega} e^{-i \omega x}} \dx \\
					&= i \int_{-\infty}^\infty f(x) \f {\mathrm d}{\mathrm d \omega} (e^{-i \omega x}) \dx
				\intertext{
					Das Integral konvergiert gleichmäßig bezüglich $\omega$, da $|f(x) x e^{-i \omega x} | \le |x f(x)| \le \f c{1+x^2}$ da $f \in \scr S$.
				}
					&= i \f {\mathrm d}{\mathrm d \omega} \int_{-\infty}^\infty f(x) e^{-i \omega x} \dx \\
					&=\sqrt{2\pi} i \f {\mathrm d}{\mathrm d \omega} \hat f(\omega)
					= i \hat f^{(1)} (\omega) \sqrt{2\pi}
				\end{align*}
		\end{enumerate}
	\end{proof}
\end{st}

\begin{st} \label{4.8}
	Für $F : f \to \hat f$ ist eine lineare Abbildung von $\scr S(\R)$ in $\scr S(\R)$.

	\begin{proof}
		\begin{enumerate}[1)]
			\item
				$F$ linear ist klar.
			\item
				Sei $f \in \scr S(\R)$.
				Zeige $\hat f \in \scr S(\R)$.
				\begin{enumerate}[a)]
					\item
						Es gilt
						\begin{align*}
							|\hat f(\omega)|
							&= \f 1{\sqrt{2\pi}} \bigg| \int_{-\infty}^\infty f(x) e^{-i \omega x} \dx \bigg| \\
							&\le \f 1{\sqrt{2\pi}} \int_{-\infty}^\infty  |f(x)| \dx
							< \infty
						\end{align*}
						(wegen $|f(x)| \le \f {c}{1+x^2}$, da $f \in \scr S(\R)$)
						und damit $\sup_{\omega \in \R}|\hat f(\omega)| < \infty$.
					\item
						Es gilt
						\begin{align*}
							\sqrt{2\pi} \f {\mathrm d^k}{\mathrm d \omega^k} \hat f(\omega)
							&= \f {\mathrm d^k}{\mathrm d \omega^k} \hat f(\omega) \int_{-\infty}^\infty f(x) e^{i \omega x} \dx \\
							&= \int_{-\infty}^\infty \underbrace{f(x) \underbrace{\f {\mathrm d^k}{\mathrm d \omega^k} e^{- i \omega x}}_{= (-ix)^k e^{-i\omega x}}}_{|\argdot| \le \f {c}{1+|x|^{k+2}|x|^k}} \dx
						\end{align*}
						Das Abgeleitete Integral konvergiert gleichmäßig bezüglich $\omega$.
						Damit ist die Vertauschung von Ableitung und Integral gerechtfertigt.
						Außerdem hängt das Integral stetig von $\omega$ ab.

						Somit ist $\hat f \in C^\infty(\R \to \C)$.
				\end{enumerate}
		\end{enumerate}
		Es gilt
		\begin{align*}
			\sup_{\omega \in \R} |\omega^j \hat f^{(k)}(\omega) |
			&\stackrel{\ref{4.7}}= \sup_{\omega \in \R} \Big| \tf {1}{i^{j+k}} \hat g(\omega) \Big| \qquad \text{mit $g(x) = x^k f^{(j)}(x)$}
		\intertext{also insbesondere $g \in \scr S(\R)$}
			&\stack{\text{a)}}< \infty \qquad \text{für alle $j,k \in \N_0$}.
		\end{align*}
	\end{proof}
\end{st}

\begin{nt} \label{4.9}
	$F : \scr S(\R) \to \scr S(\R)$ ist stetig bezüglich der Metrik $d$ aus \ref{4.3}.
\end{nt}

\begin{st} \label{4.10}
	$F: \scr S(\R) \to \scr S(\R): f \mapsto \hat f$ ist bijektiv und $F^{-1}: \scr S(\R) \to \scr S(\R) : g \mapsto \check g$ ist gegeben durch
	\[
		\check g(x) := \f 1{\sqrt{2\pi}} \int_{-\infty}^\infty g(\omega) e^{i \omega x} \dx[\omega]
	\]
	\begin{proof}
		\begin{enumerate}[1)]
			\item
				Sei zunächst $f \in C_0^\infty(\R)$.
				Wähle $\eps > 0$, so dass $\supp f \subset ]-\f 1\eps, \f 1\eps[$.
				Setze $f$ $\f 2\eps$-periodisch auf $\R$ fort, dann ist $f \in C^\infty (\R \to \C)$.
				Aus \ref{3.35} folgt: für $y \in ]-\f 1\eps, \f 1\eps[$ gilt
				\begin{align*}
					f(y) 
					&= \sum_{j=-\infty}^\infty \f 1{2 \f 1\eps} \int_{-\f 1\eps}^{\f 1\eps} e^{ij\pi s \eps} f(s) \dx[s] e^{ij\pi y \eps} \qquad \text{(nicht-fortgesetztes $f$)}\\
					&= \sum_{j=-\infty}^\infty \underbrace{\f \eps 2 \sqrt{2\pi}}_{= \f {\Delta \omega_j}{\sqrt{2\pi}}} \hat f(\underbrace{j\pi \eps}_{\omega_j}) e^{\overbrace{ij\pi y \eps}^{i\omega_j y}} \\
					&\stack{\eps \searrow 0}= \f 1{\sqrt{2\pi}} \int_{-\infty}^\infty f(\omega) e^{i\omega y} \dx[\omega]
				\end{align*}
				Also gilt für $\forall y \in \R$:
				\[
					f(y) = \f 1{\sqrt{2\pi}} \int_{-\infty}^\infty \hat f(\omega) e^{i \omega y} \dx[\omega]
				\]
			\item
				Sei $f \in \scr S(\R)$.
				Multipliziere mit der Abschneidefunktion $\psi_{R,1}$ aus \ref{4.2}: $\psi_{R,1}\cdot f \in C_0^\infty(\R)$.
				Für $-R < y < R$ gilt
				\begin{align*}
					f(y) 
					&= (\psi_{R,1}\cdot f)(y)
					&\stack{1)}= \f 1{\sqrt{2\pi}} \int_{-\infty}^\infty \widehat{\psi_{R,1} \cdot f}(\omega) e^{i\omega y} \dx[\omega]
				\end{align*}
				\begin{align*}
					\widehat{\psi_{R,1} \cdot f}(\omega) 
					&= \f 1{\sqrt{2\omega}} \int_{-\infty}^\infty \psi_{R_1}(x) f(x) e^{-i \omega x} \dx \\
					&\stack{R\to \infty}\to \hat f(\omega)  \qquad \text{(für jedes feste $\omega$)}
				\end{align*}
				Für $|\omega| \ge 1$ ist
				\begin{align*}
					\Big|\widehat{\psi_{R,1} \cdot f}(\omega)  \Big|
					&= \Bigg| \f 1{\sqrt{2\pi}} \f 1{(-i\omega)^2} \int_{-\infty}^\infty (\psi_{R,1}\cdot f)''(x) e^{-i \omega x} \dx[x] \Bigg| \\
					&= \f 1{\sqrt{2\pi} \omega^2} \int_{-\infty}^\infty \underbrace{\Big| (\psi_{R,1} \cdot f)'' (x))\Big|}_{c (|f(x)| + |f'(x)| + |f''(x)| \le \f {\tilde c}{1+x^2}} 1 \dx \\
					&\le \f {c'}{\omega^2}.
				\end{align*}
				Für $|\omega| \le 1$ ist
				\begin{align*}
					\Big|\widehat{\psi_{R,1} \cdot f}(\omega)  \Big|
					&\le  \f 1{\sqrt{2\pi}} \int_{-\infty}^\infty \underbrace{|\psi_{R,1}(x)|}_{\le 1} |f(x)| |e^{-i\omega x}| \dx \\
					&\le c'' \qquad \text{(unabhängig von $R$).}
				\end{align*}
				Nach dem Satz von der majorisierten Konvergenz gilt also
				\begin{align*}
					f(y) 
					&= \lim_{R\to \infty} \f 1{\sqrt{2\pi}} \int_{-\infty}^\infty (\widehat{\psi_{R,1}\cdot f})(\omega) e^{ \omega y} \dx[\omega] \\
					&= \f 1{\sqrt{2\pi}} \int_{-\infty}^\infty \lim_{R\to \infty} (\widehat{\psi_{R,1}\cdot f})(\omega) e^{ \omega y} \dx[\omega] \\
					&= \f 1{\sqrt{2\pi}} \int_{-\infty}^\infty \hat f(\omega) e^{i \omega y} \dx[\omega]
				\end{align*}
				für alle $y \in \R$.
				Damit ist $F$ insbesondere injektiv.
			\item
				$F$ ist surjektiv:

				Sei $g \in \scr S(\R)$.
				Aus 2) folgt
				\begin{align*}
					g(\omega) 
					&= \f 1{\sqrt{2\pi}} \int_{-\infty}^\infty \hat g(x) e^{i\omega x} \dx \\
					&= \f 1{\sqrt{2\pi}} \int_{-\infty}^\infty \hat g(-x) e^{-i\omega x} \dx \\
				\end{align*}
				Wegen $g \in \scr S(\R)$ ist nach \ref{4.8} auch $\hat g \in \scr S(\R)$.
				Also $g = \widehat{\hat g(- \argdot)} \in \im(F)$.
		\end{enumerate}
	\end{proof}
\end{st}

\begin{ex}[Durchbiegung eines unendlich langen Balkens] \label{4.11}
	Sei $u(x)$ der Verlauf des Balkens und $f(x)$ die einwirkende Kraft an jedem Punkt.
	Es gilt folgende DGL:
	\[
		u^{(4)} + \alpha^4 u(x) = f(x)
	\]
	Wir nehmen an, dass $f, u \in \scr S(\R)$.
	Wegen $\widehat{u^{(k)}(\omega)} = (i\omega)^k \hat u(\omega)$ vereinfacht sich die DGL nach Fouriertransformation zu
	\begin{align*}
		\omega^4 \hat u(\omega) + \alpha^4 \hat u(\omega) &= \hat f(\omega) \\
		\iff \qquad  \hat u(\omega) &= \f 1{\omega^4 + \alpha^4} \hat f(\omega)
	\end{align*}
	Wegen $f \in \scr S(\R)$ ist auch $\hat f \in \scr S(\R)$.
	Mit $\phi(\omega) := \f 1{\omega^4 + \alpha^4}$ gelten $\phi \in C^\infty (\R \to \R)$ und $|\phi^{(k)}(\omega)| \le c_k$ (es reicht $\le |x^{N_k}|c_k$).
	Damit ist $\phi \hat f \in \scr S(\R)$ (Beweis: Übung) und somit
	\[
		u = (\phi \hat f)^{\vee} \in \scr S(\R)
	\]
	Also existiert die Lösung und ist eindeutig.

	Manche Leute würden die Lösung auch gerne ausrechnen können.
	Wie berechnet man $(\phi \psi)^{\vee}$ aus $\check \phi$, $\check \psi$?
\end{ex}

\begin{df} \label{4.12}
	Für $f, g \in \scr S(\R)$ ist
	\[
		(f \ast g)(y) := \int_{-\infty}^\infty f(y-x) g(x) \dx
	\]
	die \emph{Faltung} von $f$ mit $g$.
\end{df}

\begin{ex} \label{4.13}
	Betrachte die Abschneidefunktion aus \ref{4.2} 3):
	\begin{align*}
		\psi_{R,\eps} (x) 
		&:= \int_{-R-\eps}^{R+\eps} j_\eps (x-y) \dx[y],
	\intertext{diese lässt sich schreiben als}
		&= \int_{-\infty}^\infty j_\eps(x-y) \cdot  \chi_{[-R-\eps,R+\eps]}(y) \dx[y] \\
		&= \Big( j_\eps \ast \chi_{[-R-\eps, R+\eps]} \Big) (x)
	\end{align*}
\end{ex}

\begin{st} \label{4.14}
	Für $f, g,h  \in \scr S(\R)$ gelten
	\begin{enumerate}[1)]
		\item
			$f \ast g \in \scr S(\R)$
		\item
			Es gilt
			\begin{align*}
				\widehat{f\cdot g} &= \f 1{\sqrt{2\pi}} \hat f \ast \hat g, \\
				\widehat{f\ast g} &= \sqrt{2\pi} \hat f \cdot \hat g.
			\end{align*}
			Insbesondere auch $(\phi \psi)^\vee = \f 1{\sqrt{2\pi}} \check \phi \ast \check \psi$.
		\item
			Es gilt
			\begin{align*}
				f \ast (g \ast h) &=  (f\ast g) \ast h, \\
				f \ast g &= g \ast f.
			\end{align*}
	\end{enumerate}
	\begin{note}
		Mit dem Satz gilt für das Beispiel \ref{4.11}:
		\[
			u = \f 1{\sqrt{2\pi}}\check \phi \ast f
			= \f 1{\sqrt{2\pi}}\check \phi \ast (\hat f)^\vee.
		\]
		Damit ist $u$ explizit berechenbar mittels $(\f 1{\omega^4 + \alpha^4})^\vee$.
	\end{note}
	\begin{proof}
		\begin{enumerate}[1)]
			\item
				Wegen $f, g \in \scr S(\R)$ ist nach \ref{4.8} $\hat f, \hat g \in \scr S(\R)$, also $\hat f \cdot \hat g \in \scr S(\R)$, also nach \ref{4.10} $(\hat f \hat g)^\vee \in \scr S(\R)$ und somit
				\[
					f \ast g \stack{2)}= \sqrt{2\pi} (\hat f \hat g)^\vee \in \scr S(\R)
				\]
			\item
				Die Fouriertransformation erhält das Skalarprodukt (siehe später), daher gilt
				\begin{align*}
					\sqrt{2\pi} \widehat{f g}(\omega) 
					&= \int_{-\infty}^\infty f(x) g(x) e^{-i\omega x} \dx \\
					&= \< g, \_f e^{i\omega \argdot} \>_{L^2(\R)} \\
					&= \Big\< \hat g, \widehat{\_ f e^{i\omega \argdot}} \Big\>_{L^2(\R)} \\
					&= \int_{-\infty}^\infty \hat g(\tilde \omega) \_{\bigg( \f 1{\sqrt{2\pi}} \int_{-\infty}^\infty \_f(x) e^{i\omega x} e^{-i \tilde \omega x} \dx \bigg)} \dx[\tilde \omega] \\
					&= \int_{-\infty}^\infty \hat g(\tilde \omega) \underbrace{\f 1{\sqrt{2\pi}} \int_{-\infty}^{\infty} f(x) e^{-i(\omega - \tilde \omega)} \dx}_{=\hat f(\omega - \tilde \omega)} \dx[\tilde \omega] \\
					&= \int_{-\infty}^\infty \hat g(\tilde \omega)\hat f(\omega - \tilde \omega) \dx[\omega] \\
					&= (\hat f \ast \hat g)(\omega)
				\end{align*}
				Genauso folgt
				\[
					(\hat f \hat g)^\vee = \f 1{\sqrt{2\pi}} f \ast g.
				\]
			\item
				Es gilt
				\[
					(f\ast g) \ast h = \f 1{\sqrt{2\pi}} (\widehat {f \ast g}  \ast \hat h)^\vee = (\hat f \hat g) \hat h
				\]
		\end{enumerate}
	\end{proof}
\end{st}

\begin{st}[Plancherel-Gleichung] \label{4.15}
	Für $f \in \scr S(\R)$ gilt
	\[
		\|Ff\|_{L^2(\R)} = \|f\|_{L^2(\R)}.
	\]
	Die Fouriertransformation erhält also die $L^2$-Norm.
	\begin{proof}
		\begin{enumerate}[1)]
			\item
				Zunächst sei $f \in C_0^\infty(\R)$, $\supp f \subset ]-\f 1\eps, \f 1\eps[$.
				In $L^2(]-\f 1\eps, \f 1\eps[)$ ist $(e_j)_{j\in \Z}$ mit
				\[
					e_j(x) = \sqrt{ \f\eps 2} e^{ij\pi \eps x}
				\]
				ein VONS, also $\forall f \in L^2(\cdots) : f = \sum_{i=-\infty}^\infty \<f,e_j\> e_j$.

				Mit der Parsevalschen Gleichung folgt:
				\begin{align*}
					\|f\|_{L^2(]-\f 1\eps, \f 1\eps[)}^2 
					&= \sum_{j=-\infty}^\infty |\<f,e_j\>|^2 
					&= \sum_{j=-\infty}^\infty \f \eps 2 \bigg| \int_{-\f 1\eps}^{\f 1\eps} f(x) e^{-ij\pi \eps x} \dx \bigg|^2
					&= \sum_{j=-\infty}^\infty \f \eps 2 \bigg| \underbrace{\int_{-\infty}^{\infty} f(x) e^{-ij\pi \eps x} \dx}_{\sqrt{2\pi} \hat f (j\pi \eps)} \bigg|^2 \\
					&= \sum_{j=-\infty}^\infty \underbrace{\pi \eps}_{\Delta \omega} \Big| \hat f(\underbrace{j\pi \eps}_{\omega_j})\Big|^2 \\
					&\stack{\eps \to 0}\to \int_{-\infty}^\infty |\hat f(\omega)|^2 \dx[\omega].
				\end{align*}
				Also gilt für $f \in C_0^\infty(\R) : \|\hat f\|_{L^2(\R)} = \|f\|_{L^2(\R)}$.
			\item
				Sei jetzt $f \in \scr S(\R)$.
				Multipliziere mit Abschneidefunktion: $\psi_{R,1} f \in C_0^\infty(\R)$.
				Nach 1) gilt
				\[
					\underbrace{\Big\|\psi_{R,1} f\Big\|_{L^2(\R)}}_{\stack{R\to \infty}\to \|f\|_{L^2(\R)} \text{ (maj. Konv.)}} = \Big\| \widehat{\psi_{R,1} f} \Big\|_{L^2(\R)}
				\]
				Im Beweis von \ref{4.10} haben wir gesehen
				\begin{align*}
					|\hat(\omega)|^2 &= \lim_{n\to\infty}\Big| \widehat{\psi_{R,1}f}(\omega) \Big|^2 \\
					\Big| \widehat{\psi_{R,1} f}(\omega) \Big| &\le \f {c}{1+\omega^2}
				\end{align*}
				Mit majorisierter Konvergenz folgt
				\[
					\implies \quad \int_{-\infty}^\infty \Big| \widehat{\psi_{R,1}f}(\omega)\Big|^2 \dx[\omega] \to \int_{-\infty}^\infty |\hat f(\omega)|^2 \dx[\omega]
				\]
				Also für $R \to \infty$: $\|f\|_{L^2(\R)} = \|\hat \|_{L^2(\R)}$.
		\end{enumerate}
	\end{proof}
\end{st}

\begin{kor} \label{4.16}
	Für $f,g \in \scr S(\R)$ gilt
	\[
		\<f,g\>_{L^2(\R)} = \< Ff, Fg\>_{L^2(\R)}
	\]
	\begin{proof}
		Folgt direkt mit der Polarisationsformel:
		\[
			\<f,g\> = \f 14 \Big( \|f+g\|^2 - \|f-g\|^2 - i \Big(\|f+ig\|^2 - \|f-ig\|^2 \Big) \Big).
		\]
	\end{proof}
\end{kor}

\begin{st} \label{4.17}
	Für $f \in \scr S(\R)$ gilt
	\[
		\| \hat f \|_{L^\infty(\R)} \le \f 1{\sqrt{2\pi}} \|f\|_{L^1(\R)}.
	\]
	\begin{proof}
		Es gilt
		\begin{align*}
			|\hat f(\omega)|
			&= \bigg| \f 1{\sqrt{2\pi}} \int_{-\infty}^\infty f(x) e^{-i\omega x} \dx \bigg| \\
			&\le \f 1{\sqrt{2\pi}} \int_{-\infty}^\infty |f(x)| 1 \dx
		\end{align*}
		Also ist für alle $\omega \in \R$
		\[
			|\hat f(\omega)| \le \f 1{\sqrt{2\pi}} \|f\|_{L^1(\R)}.
		\]
	\end{proof}
\end{st}


\subsection{Dichte Mengen}


\begin{st} \label{4.18}
	Sei $1 \le p < \infty$.
	Dann ist $C_0^\infty$ dicht in $L^p(\R)$.
	\begin{proof}
		Sei $u \in L^p(\R)$ und o.B.d.A. $u(x) \ge 0$ auf $\R$ (sonst $u = (\Re(u))_+ - (\Re(u))_- + i((\Im(u))_+ - (\Im(u))_-))$).
		Approximiere nun $u$ durch $C_0^\infty(\R)$-Funktionen.
		\begin{enumerate}[1)]
			\item
				Mache Träger kompakt:
				\[
					v_k := \chi_{[-k,k]}\cdot u.
				\]
				Dann gilt
				\begin{itemize}
					\item
						$|v_k(x)-u(x)|^p \to 0$ für $k\to \infty$ für jedes feste $x \in \R$,
					\item
						$|v_k(x)-u(x)|^p \le |u(x)|^p$, $\int_{-\infty}^\infty |u(x)|^p \dx < \infty$.
				\end{itemize}
				Mit majorisierter Konvergenz gilt $\int_{-\infty}^\infty |v_k(x)|^p \dx \to 0$, also
				\[
					\|v_k - u\|_{L^p(\R)} \to 0.
				\]
				Wähle $u_1 \in L^p(\R)$, $\supp u_1 \subset [-R,R]$ mit $\|u-u_1\|_{L^p(\R^n)} < \eps$.
			\item
				Aus der Maßtheorie wissen wir: Es gibt eine Folge $(\phi_k)_{k\in \N}$ einfacher Funktionen mit
				\begin{itemize}
					\item
						$\forall u \in \R : 0 \le \phi_k(x) \le u_1(x)$, insbesondere $\supp \phi_k \subset [-R,R]$,
					\item
						$\phi_k(x) \nearrow u_1(x)$ punktweise auf $\R$.
				\end{itemize}
				Mit majorisierter Konvergenz wissen wir $\|u_1 - \phi_k\|_{L^p(\R)} \to 0$.

				Wähle 
				\[
					u_2 := \sum_{j=1}^n \lambda_j \chi_{B_j},
				\]
				mit $B_j$ messbar und $\|u_1-u_2\|_{L^p(\R)} < \eps$.

				Es gilt $B_j \in \text{Lebesgue-Borel-$\sigma$-Algebra}$.
				Wähle $A \in \text{Borel-$\sigma$-Algebra}$ mit $\my(A_j \setminus B_j) = 0 = \my(B_j \setminus A_j)$.

				Setze 
				\[
					u_3 := \sum_{j=1}^n \lambda_j \chi_{A_j},
				\]
				womit $\|u_2 - u_3\|_{L^p(\R)} = 0$ gilt.
				Es gelte o.B.d.A. $\supp u_3 \subset [-R,R]$ (schneide $A_j$ mit $[-R,R]$).
			\item
				Approximiere jedes $A_j$ durch endlich viele abgeschlossene Intervalle.
				Sei $A \in \text{Borel-$\sigma$-Algebra}$, $A \subset [-R,R]$.
				Das Maß von $A$ ist definiert als
				\[
					\my(A) := \inf \Big\{ \sum_{j=1}^\infty \my(I_j) : A \subset \bigcup_{j=1}^\infty I_j, I_j = [a_j, b_j] \Big\}.
				\]
				Wähle also o.B.d.A $(I_j)$ mit $I_j \subset [-R,R]$, $A \subset \bigcup_{j=1}^\infty I_j$, $\my(A) + \tilde \eps^p \le \sum_{j=1}^\infty \my(I_j)$.
				Wegen $\my(A) \le \sum_{j=1}^\infty \my(I_j)$ ist $0 \le \sum_{j=1}^\infty \my(I_j) - \my(A) \le \tilde \eps^p$.
				Also
				\begin{align*}
					0 
					\le \int_{-\infty}^\infty \underbrace{\Big(\chi_{\bigcup_{j=1}^\infty I_j} - \chi_{A_j}\Big)}_{=(\argdot)^p \text{ da $\in \{0,1\}$}} \dx
					&= \my\bigg(\bigcup_{j=1}^\infty I_j \bigg) - \my(A) \\
					&\le \sum_{j=1}^\infty u(I_j) - \my(A)
					< \tilde \eps^p.
				\end{align*}
				Und somit
				\[
					\Big|\chi_{\bigcup_{j=1}^\infty I_j} - \chi_{A_j}\Big\|_{L^p(\R)} < \tilde \eps.
				\]
				Wähle $N \in \N$ mit $\sum_{i=N+1}^\infty \my(I_j) < \tilde \eps^p$, dann gilt
				\begin{align*}
					0 
					\le \int_{-\infty}^\infty \underbrace{\Big(\chi_{\bigcup_{j=1}^\infty I_j} - \chi_{\bigcup_{j=1}^N I_j}\Big)}_{=(\argdot)^p}
					&\le \int_{-\infty}^\infty \Big(\chi_{\bigcup_{j=N+1}^\infty I_j}\Big) \\
					&\le \sum_{j=N+1}^\infty \my(I_j) 
					< \tilde \eps^p.
				\end{align*}
				Also
				\[
					\Big\| \chi_{\bigcup_{j=1}^\infty I_j} - \chi_{\bigcup_{j=1}^N I_j} \Big\|_{L^p(\R)} < \tilde \eps.
				\]
				Zusammengefasst gilt damit
				\[
					\Big\| \chi_A - \underbrace{\chi_{\bigcup_{j=1}^N I_j}}_{\mathclap{\text{nur noch endlich viele}}} \Big\|_{L^p(\R)} < 2 \tilde \eps.
				\]
				Wähle also
				\[
					u_4 = \sum_{j=1}^m  \tilde \lambda_j \chi_{[a_j,b_j]}
				\]
				mit $\|u_3 - u_4\|_{L^p(\R)} < \eps$.
			\item
				Zu $[a_j,b_j]$ existiert $f_j \in C(\R \to \C)$ mit $\supp f_j \subset [a_j, b_j]$, $\|\chi_{[a_j,b_j]} - f\|_{L^p(\R)} < \tilde \eps$.

				Wähle also
				\[
					u_5 = \sum_{j=1}^m \tilde \lambda_j f_j
				\]
				sodass $\|u_4 - u_5\|_{L^p(\R)} < \eps$, $u_5 \in C(\R \to \C)$ und $\supp u_5 \subset [-R,R]$.
			\item
				Aus \ref{4.20} 4) wissen wir für stetige Funktionen mit kompaktem Träger, dass $j_\delta \ast u_5 \in C_0^\infty(\R \to \C)$ und
				\[
					\Big\|j_\delta \ast u_5 - u_5 \Big\|_{L^p(\R)} < \eps \qquad \text{für $\delta > 0$ genügend klein.}
				\]

				Insgesamt gilt damit
				\[
					\Big\|u - j_\delta \ast u_5 \Big\| < 6\eps
				\]
		\end{enumerate}
	\end{proof}
\end{st}

\begin{df} \label{4.19}
	Für $f \in L^p(\R)$ setze
	\[
		\scr J_\eps f := j_\eps \ast f
	\]
	d.h.
	\[
		\scr J_\eps f(x) = \int_{-\infty}^\infty j_\eps (x-y) f(y) \dx[y].
	\]
	$\scr J_\eps$ heißt \emph{Glättungsoperator} (oder \emph{Mollifier}).
	\begin{note}
		Die wichtigen Eigenschaften von $j_\eps$ waren:

		$j_\eps \in C_0^\infty(\R)$, $j_\eps \ge 0$ auf $\R$, $\supp j_\eps = [-\eps,\eps]$, $\int_{\R} j_\eps(x) \dx = \int_{-\eps}^\eps j_\eps(x) \dx = 1$.
	\end{note}
\end{df}

\begin{st} \label{4.20}
	Für $1 \le p < \infty$, $u \in L^p(\R)$ gelten
	\begin{enumerate}[1)]
		\item
			$J_\eps u \in C^\infty (\R \to \C)$ für $\eps > 0$.
		\item
			$\supp u$ beschränkt $\implies J_\eps u \in C_0^\infty (\R)$.
		\item
			$J_\eps u \in L^p(\R)$, $\|J_\eps u \|_{L^p(\R)} \le \|u\|_{L^p(\R)}$
		\item
			$\|J_\eps u - u \|_{L^p(\R)} \to 0$ für $\eps \searrow 0$.
	\end{enumerate}
	\begin{proof}
		\begin{enumerate}[1)]
			\item
				Zeige
				\[
					\f{\mathrm d^k}{\mathrm x^k} (J_\eps u)(x) = \int_{-\infty}^\infty \f{\mathrm d^k j_\eps}{\mathrm d x^k} (x-y) u(y) \dx[y]
				\]
				für $k \in \N$.

				Für $k=1$ und $\eps$ fest gilt
				\begin{align*}
					&\Bigg| \f {J_\eps u(x+k) - J_\eps u(x)}{k} - \int_{-\infty}^\infty j_\eps' (x-y) u(y) \dx[y] \Bigg|
					= \Bigg| \int_{-\infty}^\infty \Big( \underbrace{\f {j_\eps(x+h-y)-j_\eps(x-y)}h}_{=j_\eps'(x+\theta k -y)} - j_\eps'(x-y) \Big) u(y) \dx[y] \Bigg|
				\intertext{wobei $0 < \theta < 1$ von $x-y$ und $h$ abhängig}
					&\qquad\le \int_{-\infty}^\infty \Big| j_\eps'(x+\theta h -y) - j_\eps'(x-y) \Big| |u(y)| \dx[y] \\
					&\qquad\stack{Hölder}\le \Bigg(\int_{-\infty}^\infty \Big| j_\eps'(x+\theta h -y) - j_\eps'(x-y) \Big|^q \dx[y] \Bigg)^{\f 1q} \|u\|_{L^p(\R)} \qquad \text{ mit $\f 1p + \f 1q = 1$}
				\end{align*}
				(Für $p = 1$ schreibt man: $\|\sup_{x-y \in\R} |j_\eps'(x+\theta h -y) - j_\eps'(x-y)| \|u\|_{L^1(\R)} < \tilde \eps \|u\|_{L^1(\R)}$ (falls $|k| < \delta$)).

				Für den Teil mit der $q$-Norm gilt:
				\begin{align*}
					\Bigg(\int_{-\infty}^\infty \Big| j_\eps'(x+\theta h -y) - j_\eps'(x-y) \Big|^q \dx[y] \Bigg)
					&\stack{z=x-y}= \int_{-\infty}^\infty \Big| j_\eps'(z+\theta h) - j_\eps' (z) \Big|^q \dx[z] \\
					&= \int_{-2\eps}^{2\eps} \Big| j_\eps'(z+\theta h) - j_\eps' (z) \Big|^q \dx[z] \qquad \text{falls $|h| < \eps$}
					\intertext{die stetige Funktion $j_\eps'$ ist auf dem kompakten Intervall $[-2\eps, 2\eps]$ gleichmäßig stetig: $|j_\eps'(z+\theta h) - j_\eps'(z)| < \tilde \eps$ für $|\theta h| < \delta$}
					&= 4\eps \tilde \eps^q \qquad \text{für $|h| < \delta$}
				\end{align*}
			\item
				Falls $\supp u \subset [a,b]$, $a<b$, dann ist $\supp(J_\eps u) \subset [a-\eps, b+\eps]$ und somit $J_\eps u \in C_0^\infty(\R)$.
			\item
				\begin{enumerate}[a)]
					\item
						Zeige $|J_\eps u(x)| \le \Big( \int_{\R} j_\eps(x-y) |u(y)|^p \dx[y] \Big)^{\f 1p}$.

						Für $p=1$ ist die Aussage klar (Betrag ins Integral ziehen).
						Sei also $1 < p < \infty$:
						\begin{align*}
							|J_\eps u(x)| 
							&= \bigg| \int_{-\infty}^\infty \underbrace{j_\eps(x-y)^{\f 1q}}_{\ge 0} j_\eps(x-y)^{\f 1p} u(y) \dx[y] \bigg| \\
							&\stack{\text{Hölder}}\le \underbrace{\bigg( \int_{-\infty}^\infty j_\eps(x-y)^{\f qq} \dx[y] \bigg)^{\f 1q}}_{=1} \bigg( \int_{-\infty}^\infty j_\eps(x-y)^{\f pp} |u(y)|^p \dx[y] \bigg)^{\f 1p}
						\end{align*}
					\item
						Es gilt
						\begin{align*}
							\| J_\eps u \|_{L^p(\R)}^p
							&= \int_{-\infty}^\infty |J_\eps u(x)|^p \dx \\
							&\stack{\text{a)}}\le \int_{-\infty}^\infty \bigg( \int_{-\infty}^\infty j_\eps(x-y) |u(y)|^p \dx[y] \bigg) \dx \\
							&\stack{\text{Fubini}} \int_{-\infty}^\infty |u(y)|^p \underbrace{\bigg( \int_{-\infty}^\infty j_\eps(x-y) \dx \bigg)}_{=1} \dx[y] \qquad \text{($j_\eps \ge 0, |u|^p \ge 0$)} \\
							&= \|u\|_{L^p(\R)}^p < \infty.
						\end{align*}
						Also insbesondere $J_\eps u \in L^p(\R)$.
				\end{enumerate}
			\item
				Sei zunächst $u \in C(\R \to \C)$, $\supp u$ kompakt (z.B. $\supp u \subset [a,b]$).
				Dann gilt wegen $\int_{j_\eps(x)} \dx = 1$:
				\begin{align*}
					|J_\eps u(x) - u(x)|
					&= \bigg| \int_{-\infty}^\infty j_\eps(x-y) \Big(u(y) - u(x)\Big) \dx[y] \bigg| \\
					&\le \int_{x-\eps}^{x+\eps} \underbrace{j_\eps(x-y)}_{\int \argdot \dx[y] = 1} \underbrace{|u(y)-u(x)|}_{\delta \text{ für $\eps < \eps_0$}} \dx[y] \\
					&< \delta
				\end{align*}
				($u$ stetig auf kompakter Menge, also gleichmäßig stetig).

				Also
				\begin{align*}
					\|J_\eps u - u\|_{L^p(\R)}^p
					&= \int_{-\infty}^\infty \underbrace{|J_\eps u(x) - u(x)|^p}_{\supp(\argdot) \subset [a-\eps, b+ \eps]} \dx \\
					&< \delta^p (b-a +z) \qquad \text{für $\eps < \eps_0$} \\
					&\to 0 \qquad \text{für $\eps \searrow 0$}
				\end{align*}
		\end{enumerate}
	\end{proof}
\end{st}

\begin{st}[Satz von Plancherel] \label{4.21}
	\begin{enumerate}[1)]
		\item
			Die Fouriertransformation $F: \scr S(R) \to \scr S(R)$ besitzt eine eindeutige Fortsetzung $\scr F:L^2(\R) \to L^2(\R)$.
			Diese Fortsetzung ist unitär, d.h.
			\[
				\forall f,g \in L^2 (\R) : \<\scr F f, \scr Fg \> = \<f,g\>
			\]
		\item
			Sei $\scr G: L^2(\R) \to L^2(\R)$ die eindeutige Fortsetzungen von $F^{-1}$ auf $L^2(\R)$.
			Dann ist $\scr G = \scr F^{-1}$.
	\end{enumerate}
	\begin{proof}
		\begin{enumerate}[1)]
			\item
				In den Übungen wird gezeigt: $\scr F$ existiert, ist eindeutig und $\|\scr Ff\|_{L^2(\R)} = \|f\|_{L^2(\R)}$ für $f \in L^2(\R)$.
				Mit der Polarisationsformel folgt die Aussage.
			\item
				Dass $\scr G$ existiert und eindeutig ist, wird in den Übungen gezeigt.

				Zeige $\scr G \circ \scr F = \Id = \scr F \circ \scr G$.
				Zu $f \in L^2(\R)$ wähle nach \ref{4.20} $(f_n)$ in $C_0^\infty(\R) \subset \scr S(\R)$ mit $\|f-f_n\|_{L^2(\R)} \to 0$.
				Dann ist
				\[
					\scr F f = \underbrace{\lim[L^2]_{n\to \infty} \scr F f_n}_{\|\scr F f  - \scr F f_n\|_{L^2(\R)}\to 0}
				\]
				 und
				 \[
					\scr G(\scr F f) = \lim[L^2]_{n\to\infty} F^{-1}(F(f_n)) = \lim[L^2]_{n\to\infty} f_n = f.
				 \]
		\end{enumerate}
	\end{proof}
\end{st}

\begin{nt} \label{4.22}
	Für $f \in L^2(\R)$ ist $\int_{-\infty}^\infty f(x) e^{-i \omega x} \dx$ nicht immer konvergent.
	Beispielsweise
	\[
		f(x) :=  \f{\sin(x^2)}{1+|x|}.
	\]
\end{nt}

\begin{st} \label{4.23}
	Für $f \in L^2(\R)$ gilt
	\[
		\scr F f = \f 1{\sqrt{2\pi}} \lim[L^2]_{R\to\infty} \int_{-R}^R f(x)e^{-i\omega x} \dx
	\]
	\begin{proof}
		\begin{enumerate}[1)]
			\item
				Setze $f_R := \chi_{[-R,R]} \cdot f$, also $\|f_R - f\|_{L^2(\R)} \to 0$ für $R \to \infty$.
				Da $\scr F$ stetig ist, gilt weiter $\scr F f = \lim[L^2]_{R\to\infty} \scr F f_R$.
			\item
				Setze 
				\[
					\hat f(\omega)  := \int_{-\infty}^\infty f_R(x) e^{-i\omega x} \dx
					= \int_{-R}^R f(x) e^{-i\omega x} \dx
				\]
				und zeige $\hat f = \scr F f_R$.
							
				Setze 
				\[
					f_n := j_{\f 1n} \ast f_R 
					= J_{\f 1n} (f_R) \in C_0^\infty(\R) \qquad \text{nach \ref{4.20}, 2)},
				\]
				Es gelten
				\begin{enumerate}[a)]
					\item
						Es gilt
						\begin{align*}
							|\hat f_n(\omega) - \hat f(\omega)|
							& \le \f 1{\sqrt{2\pi}} \underbrace{\int_{-R-\f 1n}^{R+\f 1n} |f_n(x) - f_R(x)|\cdot |e^{-i\omega x} \dx}_{= \<|f_n-f_R|,1\>} \\
							&\stack{CSB}\le \f 1{\sqrt{2\pi}} \underbrace{\|f_n - f_R\|_{L^2(\R)}}_{\to 0} \underbrace{\sqrt{\int_{-\R-\f 1n}^{R+\f 1n} \dx}}_{= \sqrt{2R + \f 2n} \to 2R}
							\to 0 \qquad (n\to \infty)
						\end{align*}
						gleichmäßig bezüglich $\omega$.
					\item
						Wegen $f_n \in C_0^\infty$ ist $\hat f_n = F f_n = \scr F f_n$ also
						\begin{align*}
							\|\hat f_n - \scr F f_R\|_{L^2(\R)}
							&= \|\scr F(f_n -f_R)\|_{L^2(\R)} \\
							&= \|f_n - f_R\|_{L^2(\R)}
							\to 0 \qquad (n\to \infty)
						\end{align*}
					\item
						Für alle $\phi \in C_0^\infty(\R)$ gilt $\<\hat f - \scr F f_R, \phi\> = 0$, also
						\begin{align*}
							\< \scr F f_R, \phi\> = \lim_{n\to\infty} \<\hat f_n, \phi\>
							= \<\hat f, \phi\>
						\end{align*}
					\item
						Wähle $(\phi_k)$ in $C_0^\infty(\R)$ mit $\|\phi_k - (\hat f - \scr F f_R)\|_{L^2(\R)} \to 0$.
						Dann ist
						\[
							0 = \<\hat f - \scr F f_R, \phi_k\> \to \|\hat f - \scr F f_R\|_{L^2(\R)}^2 \qquad (k\to \infty)
						\]
						und somit $\hat f = F f_R$.
				\end{enumerate}
		\end{enumerate}

	\end{proof}
\end{st}

