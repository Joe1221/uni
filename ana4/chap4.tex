% This work is licensed under the Creative Commons
% Attribution-NonCommercial-ShareAlike 3.0 Unported License. To view a copy of
% this license, visit http://creativecommons.org/licenses/by-nc-sa/3.0/ or send
% a letter to Creative Commons, 444 Castro Street, Suite 900, Mountain View,
% California, 94041, USA.

\chapter{Fouriertransformation}
\coursetimestamp{5}{6}{2013}

\section{Grundlagen}


\begin{df} \label{4.1}
	\begin{enumerate}[1)]
		\item
			Für $f \in C^\infty(\R \to \C)$, $j,k \in \N_0$ sei
			\[
				\|f\|_{j,k} := \sup_{x\in \R} |x^j f^{(k)}(x)|.
			\]
			Definiere den \emph{Schwartz-Raum über $\R$} als
			\[
				\scr S (\R) := \Big\{ f \in C^\infty(\R \to \C) \suchthat  \forall j,k \in \N_0 : \|f\|_{j,k} < \infty \Big\}.
			\]
		\item
			Für $f : \R \to \C$ sei
			\[
				\supp(f) := \_{\{x \in \R : f(x) \neq 0\}}
			\]
			der \emph{Träger} oder \emph{Support} von $f$.

			Der \emph{Testraum} über $\R$ ist definiert als
			\[
				C_0^\infty(\R) := \Big\{f \in C^\infty(\R\to\C) : \supp(f) \text{ ist kompakt} \Big\}
			\]
			Offensichtlich ist $C_0^\infty (\R) \subset \scr S(R)$.
	\end{enumerate}
	\begin{note}
		Für $f \in C^\infty(\R \to \C)$, $j,k\in\N_0$ ist $\|\argdot\|_{j,k}$ eine Halbnorm und im Fall $k = 0$ sogar eine Norm.
	\end{note}
\end{df}

\begin{ex} \label{4.2}
	\begin{enumerate}[1)]
		\item
			Für $j \in \N_0, \alpha > 0$ setze
			\[
				f(x) = x^j e^{-\alpha x^2}.
			\]
			Dann ist $f \in \scr S(\R) \setminus C_0^\infty (\R)$.
		\item
			Sei
			\[
				\tilde j_1(x) := \begin{cases}
					e^{-\tf 1{1-x^2}} & -1 < x < 1 \\
					0 & \text{sonst}
				\end{cases}
			\]
			Wegen $\supp(\tilde j_1) = [-1,1]$ und $\tilde j_1 \in C^\infty(\R \to \R)$ ist $\tilde j_1 \in C_0^\infty(\R)$.
			Außerdem gilt $\tilde j_1(x) \ge 0$.

			Setze $j_1(x) := c \tilde j_1(x)$ mit
			\[
				c = \dfrac 1{\int_{-1}^1 e^{- \f 1{1-t^2}} \dx[t]}.
			\]
			Dann hat $j_1$ die zusätzliche Eigenschaft $\int_{\R} j_1(x) \dx = 1$.

			Setze
			\[
				j_\eps(x) := \f 1\eps j_1(\f x\eps).
			\]
			Dann hat $j_\eps$ insgesamt folgende Eigenschaften:
			\begin{itemize}
				\item
					$j_\eps \in C_0^\infty$,
				\item
					$\supp(j_\eps) = [-\eps,\eps]$,
				\item
					$\int_\R j_\eps(x) \dx = 1$.
			\end{itemize}
		\item
			Definiere für $R > \eps > 0$ die sogenannte \emph{Abschneidefunktion}:
			\[
				\psi_{R, \eps}(x) := \int_{-R-\eps}^{R+ \eps} j_\eps(x-y) \dx[y].
			\]

			Für $-R \le x \le R$ ist
			\begin{align*}
				\psi_{R, \eps}(x)
				= \int_{-\infty}^\infty j_\eps(x-y) \dx[y]
				\stack{\xi = x-y} = -\int_{+\infty}^{-\infty}  j_\eps (\xi) \dx[\xi]
				= 1.
			\end{align*}

			Für $x \le - R - 2\eps$ oder $x \ge R + 2\eps$ ist
			\[
				\psi_{R,\eps}(x) = \int 0 \dx[y] = 0,
			\]
			also $\supp (\psi_{R,\eps}) \subset [-R-2\eps, R+2\eps]$.

			Mit der Leibnizregel für Parameterintegrale gilt
			\[
				\f {\mathrm{d}^k}{\mathrm{d}x^k} \psi_{R,\eps}(x)
				= \int_{-R-\eps}^{R+\eps} j_\eps^{(k)} (x-y) \dx[y],
			\]
			also $\psi_{R,\eps} \in C_0^\infty (\R)$.

			Der Betrag der Ableitung lässt sich unabhängig von $R$ und $x$ abschätzen:
			\begin{align*}
				\Big| \f {\mathrm{d}^k}{\mathrm{d}x^k} \psi_{R,\eps}(x) \Big|
				\le \int_{-R-\eps}^{R+\eps} |j_\eps^{(k)}(x-y)| \dx[y]
				\stack{\xi = x-y}\le \int_{-\eps}^\eps |j_\eps^{(k)}(\xi)| \dx[\xi]
				\le c(\eps).
			\end{align*}
	\end{enumerate}
\end{ex}

\begin{nt} \label{4.3}
	Mit der Metrik
	\[
		d(f,g) := \sum_{j,k = 0}^\infty \f 1{2^{j+k}} \f {\|f-g\|_{j,k}}{1+ \|f-g\|_{j,k}}
	\]
	ist $(\scr S(\R), d)$ ein vollständiger metrischer Raum.
	% Fixme: Beweis?!
\end{nt}

\begin{nt}[Eigenschaften] \label{4.4}
	\begin{enumerate}[1)]
		\item
			$\scr S(\R)$ ist ein linearer Raum über $\C$.
		\item
			Für $f,g \in \scr S (\R)$ ist $f \cdot g \in \scr S(\R)$.

			Insbesondere ist $(\scr S(\R), \cdot)$ mit 1) eine $\C$-Algebra ohne Einselement.
		\item
			Sei $f \in \scr S(R)$, $j,k \in \N_0$ und $g(x) := x^j f^{(k)}(x)$.
			Dann ist $g \in \scr S(\R)$.
	\end{enumerate}
	$C_0^\infty(\R)$ besitzt die selben Eigenschaften.
\end{nt}

\begin{df}[Fourier-Transformation] \label{4.5}
	Für $f \in \scr S (\R)$ ist
	\[
		(Ff)(\omega) := \hat f(\omega) := \f 1{\sqrt{2\pi}} \int_{-\infty}^\infty f(x) \cdot e^{-i\omega x} \dx
	\]
	die \emph{Fourier-Transformierte} von $f$.
\end{df}

\begin{ex} \label{4.6}
	Sei $f(x) = e^{-\f {x^2}2}$, dann ist die Fourier-Transformierte $\hat f(\omega) = e^{-\f {\omega^2}2}$.

	Damit ist $f$ Eigenfunktion von $F$ zum Eigenwert $\lambda = 1$.
\end{ex}

\begin{st} \label{4.7}
	Sei $f \in \scr S(\R)$ und $k \in \N_0$.
	Dann gelten
	\begin{enumerate}[1)]
		\item
			$\widehat {f^{(k)}}(\omega) = (i \omega)^k \hat f(\omega)$,
		\item
			für $g(x) := x^k f(x)$ ist $\hat g(\omega) = i^k \hat f^{(k)} (\omega)$.
	\end{enumerate}
	\begin{proof}
		\begin{enumerate}[1)]
			\item
				Mit partieller Integration ergibt sich
				\begin{align*}
					\sqrt {2\pi} \hat{f'}(\omega)
					&= \int_{-\infty}^\infty f'(x) e^{-i\omega x} \dx \\
					&= \underbrace{\Big[f(x) e^{-i\omega x}  \Big]_{x=-\infty}^{x=\infty}}_{=0 \text{ ($f \in \scr S$)}} + i\omega \underbrace{\int_{-\infty}^\infty f(x) e^{-i\omega x} \dx}_{= \sqrt{2\pi} \hat f(\omega)} \\
					&= \sqrt{2\pi} i \omega \hat f(\omega).
				\end{align*}
			\item
				Sei $g(x) = x f(x)$.
				Da $f \in \scr S(\R)$ gilt $|g(x) e^{-i\omega x}| = |xf(x)| \le \f c{1+x^2}$, also konvergiert das folgende Integral nach dem Satz über majorisierte Konvergenz gleichmäßig (siehe auch \coursehref{blatt09.pdf}{Übungsaufgabe 9.1a}) und es gilt
				\begin{align*}
					\sqrt{2\pi} \hat g(\omega)
					&= \int_{-\infty}^\infty f(x) \underbrace{ xe^{-i \omega x}}_{= i \f{\mathrm d}{\mathrm d \omega} e^{-i \omega x}} \dx \\
					&= i \int_{-\infty}^\infty f(x) \f {\mathrm d}{\mathrm d \omega} (e^{-i \omega x}) \dx \\
					&= i \f {\mathrm d}{\mathrm d \omega} \int_{-\infty}^\infty f(x) e^{-i \omega x} \dx \\
					&=\sqrt{2\pi} i \f {\mathrm d}{\mathrm d \omega} \hat f(\omega)
					= \sqrt{2\pi} i {\hat f}^{(1)} (\omega).
				\end{align*}
		\end{enumerate}
	\end{proof}
\end{st}

\begin{st} \label{4.8}
	$F : f \to \hat f$ ist eine lineare Abbildung von $\scr S(\R)$ in $\scr S(\R)$.

	\begin{proof}
		% FIXME: Überprüfen
		\begin{enumerate}[1)]
			\item
				$F$ linear ist klar.
			\item
				Zeige $\hat f \in \scr S(\R)$.
				\begin{enumerate}[a)]
					\item
						Sei $k \in \N_0$.
						Da $f \in \scr S(\R)$ gilt $|f(x) \ddx[\omega^k] e^{-i\omega x}| = |f(x)||x|^k \le \f c{1+|x|^{k+2}} |x|^k$, also konvergiert das folgende Integral nach dem Satz über majorisierte Konvergenz gleichmäßig.
						Es gilt
						\begin{align*}
							\sqrt{2\pi} \ddx[\omega^k] \hat f(\omega)
							= \ddx[\omega^k] \int_{-\infty}^\infty f(x) e^{-i \omega x} \dx
							= \int_{-\infty}^\infty f(x) \ddx[\omega^k] e^{- i \omega x} \dx
						\end{align*}
						Wegen der gleichmäßigen Konvergenz des Integrals hängt der Wert stetig von $\omega$ ab, also $\hat f \in C^\infty(\R \to \C)$.

						Außerdem ist $|\hat f(\omega)| \le \int_{-\infty}^\infty |f(x) e^{i\omega x}| \dx \le c \int_{-\infty}^\infty \f 1{1+x^2} \dx < \infty$, also $\sup_{\omega\in\R} |\hat f(\omega)| < \infty$.
					\item
						Seien jetzt $j,k \in \N_0$ und $g(x) := x^k f(x)$, dann ist $g^{(j)} \in \scr S(\R)$ (Produktregel) und nach Teil a) gilt $\sup_{\omega \in \R}|\hat {g^{(j)}}(\omega)| < \infty$, also
						\[
							\|\hat f\|_{j,k}
							= \sup_{\omega\in\R} \Big| \omega^j {\hat f}^{(k)}(\omega) \Big|
							\stack{\ref{4.7}}= \sup_{\omega\in\R} \Big| \omega^j \hat g(\omega) \Big|
							\stack{\ref{4.7}}= \sup_{\omega\in\R} \Big| \hat {g^{(j)}}(\omega) \Big|
							\stack{\text{a)}}< \infty.
						\]
				\end{enumerate}
		\end{enumerate}
	\end{proof}
\end{st}

\begin{nt} \label{4.9}
	$F : \scr S(\R) \to \scr S(\R)$ ist stetig bezüglich der Metrik $d$ aus \ref{4.3}.
	\begin{proof}
		Siehe \coursehref{blatt09.pdf}{Übungsaufgabe 9.4a}.
	\end{proof}
\end{nt}

\coursetimestamp{10}{6}{2013}
\begin{st} \label{4.10}
	$F: \scr S(\R) \to \scr S(\R): f \mapsto \hat f$ ist bijektiv und $F^{-1}: \scr S(\R) \to \scr S(\R) : g \mapsto \check g$ ist gegeben durch
	\[
		(F^{-1}g)(x) := \check g(x) := \f 1{\sqrt{2\pi}} \int_{-\infty}^\infty g(\omega) \cdot e^{i \omega x} \dx[\omega]
	\]
	\begin{proof}
		Zeige zunächst $f(y) = \f 1{\sqrt{2\pi}} \int_{-\infty}^\infty \hat f(\omega) e^{i\omega y} \dx[\omega]$ für $f \in C_0^\infty(\R)$, erweitere dies anschließend mit Hilfe der Abschneidefunktion auf $f \in \scr S(\R)$ (damit ist die Injektivität durch Angabe der Umkehrabbildung auf $\im F$ gezeigt) und zeige zuletzt die Surjektivität.
		\begin{enumerate}[1)]
			\item
				Sei $f \in C_0^\infty(\R)$ und $\eps > 0$ klein genug, sodass $\supp f \subset ]-\f 1\eps, \f 1\eps[$.

				Setze $f$ $\f 2\eps$-periodisch auf $\R$ fort, um $f$ in eine Fourierreihe gemäß \ref{3.35} zu entwickeln.
				Auf $]-\f 1\eps, \f 1\eps[$ stimmen das fortgesetzte $f$ und das nicht-fortgesetzte $f$ überein, also gilt für $y \in ]-\f 1\eps, \f 1\eps[$
				\begin{align*}
					f(y)
					&= \sum_{j=-\infty}^\infty \f 1{2 \f 1\eps} \int_{-\f 1\eps}^{\f 1\eps} e^{-ij\pi \eps s} f(s) \dx[s] e^{ij\pi \eps y} \\
					&= \sum_{j=-\infty}^\infty \f \eps 2 \sqrt{2\pi} \hat f(j\pi \eps) e^{ij\pi \eps y} \\
					&= \f 1{\sqrt{2\pi}} \sum_{j=-\infty}^\infty \Delta \omega_j \hat f(\omega_j) e^{i \omega_j y}
						&& \omega_j := j \pi \eps, \Delta \omega_j = \pi \eps.
				\end{align*}
				Mit $\eps \to 0$ gilt also für $y \in \R$
				\[
					f(y) = \f 1{\sqrt{2\pi}} \int_{-\infty}^\infty \hat f(\omega) e^{i \omega y} \dx[\omega].
				\]

			\item
				Sei $f \in \scr S(\R)$ und $\psi_{R,1} \in C_0^\infty$ die Abschneidefunktion aus $\ref{4.2}$.
				Dann ist $\psi_{R,1} f \in C_0^\infty(\R)$.

				Insgesamt gilt formal
				\begin{align*}
					f(y)
					= \lim_{R\to\infty} (\psi_{R,1} f)(y)
					&\stack{1)}= \f 1{\sqrt{2\pi}} \lim_{R\to\infty} \int_{-\infty}^\infty \hat{\psi_{R,1}f}(\omega) e^{-i\omega y} \dx[\omega] \\
					&\stack{2a)}= \f 1{\sqrt{2\pi}} \int_{-\infty}^\infty \lim_{R\to\infty} \hat{\psi_{R,1}f}(\omega) e^{-i\omega y} \dx[\omega] \\
					&\stack{2b)}= \f 1{\sqrt{2\pi}} \int_{-\infty}^\infty \hat{f}(\omega) e^{-i\omega y} \dx[\omega].
				\end{align*}
				Es bleiben noch 2a) und 2b) zu rechtfertigen:
				\begin{enumerate}[{2}a)]
					\item
						Wir suchen für $|\hat{\psi_{R,1}f}|$ eine von $R$ unabhängige, integrierbare Majorante.
						Da $(\psi_{R,1}f)'' \in \scr S(\R)$ gilt für $|\omega| \ge 1$
						\begin{align*}
							\Big|\widehat{\psi_{R,1} \cdot f}(\omega)  \Big|
							&= \Bigg| \f 1{\sqrt{2\pi}} \f 1{(-i\omega)^2} \int_{-\infty}^\infty (\psi_{R,1}\cdot f)''(x) e^{-i \omega x} \dx[x] \Bigg| \\
							&= \f 1{\sqrt{2\pi} \omega^2} \int_{-\infty}^\infty \underbrace{\Big| (\psi_{R,1} \cdot f)'' (x))\Big|}_{\le \f {c}{1+x^2}} 1 \dx \\
							&\le \f {c'}{\omega^2}.
						\end{align*}
						Für $|\omega| \le 1$ ist
						\begin{align*}
							\Big|\widehat{\psi_{R,1} \cdot f}(\omega)  \Big|
							&\le  \f 1{\sqrt{2\pi}} \int_{-\infty}^\infty \underbrace{|\psi_{R,1}(x)|}_{\le 1} |f(x)| |e^{-i\omega x}| \dx
							\le c''.
						\end{align*}
						Somit konvergiert $\int_{-\infty}^\infty \hat{\psi_{R,1}f}(\omega) e^{-i\omega y} \dx[\omega]$ nach dem Satz über majorisierte Konvergenz gleichmäßig.
					\item
						Da $\psi_{R,1}f \in \scr S(\R)$ ist $|(\psi_{R,1}f)(x)| \le \f c{1+x^2}$ und damit für $\omega \in \R$
						\begin{align*}
							\lim_{R\to \infty} \widehat{\psi_{R,1} \cdot f}(\omega)
							&= \f 1{\sqrt{2\omega}} \lim_{R\to\infty} \int_{-\infty}^\infty \psi_{R,1}(x) f(x) e^{-i \omega x} \dx \\
							&= \f 1{\sqrt{2\omega}} \int_{-\infty}^\infty \lim_{R\to\infty} \psi_{R,1}(x) f(x) e^{-i \omega x} \dx \\
							&= \hat f(\omega).
						\end{align*}
				\end{enumerate}

			\item
				Sei $g \in \scr S(\R)$.
				Aus 2) folgt
				\begin{align*}
					g(\omega)
					&= \f 1{\sqrt{2\pi}} \int_{-\infty}^\infty \hat g(x) e^{i\omega x} \dx
					= \f 1{\sqrt{2\pi}} \int_{-\infty}^\infty \hat g(-x) e^{-i\omega x} \dx
				\end{align*}
				Nach \ref{4.8} ist auch $\hat g \in \scr S(\R)$ und somit $g = \widehat{\hat g(- \argdot)} \in \im(F)$.
				Also ist $F$ surjektiv.
		\end{enumerate}
	\end{proof}
\end{st}

\begin{ex}[Durchbiegung eines unendlich langen Balkens] \label{4.11}
	Sei $u(x)$ der Verlauf des Balkens und $f(x)$ die einwirkende Kraft an jedem Punkt.
	Es gilt folgende DGL:
	\begin{align*}
		& u^{(4)} + \alpha^4 u(x) = f(x), &
		& \alpha \neq 0
	\end{align*}
	Wir nehmen an, dass $f, u \in \scr S(\R)$.
	Wegen $\widehat{u^{(k)}}(\omega) = (i\omega)^k \hat u(\omega)$ vereinfacht sich die DGL nach der Fouriertransformation zu
	\begin{align*}
		\omega^4 \hat u(\omega) + \alpha^4 \hat u(\omega) &= \hat f(\omega) \\
		\iff \qquad  \hat u(\omega) &= \f 1{\omega^4 + \alpha^4} \hat f(\omega)
	\end{align*}
	Wegen $f \in \scr S(\R)$ ist auch $\hat f \in \scr S(\R)$.
	Mit $\phi(\omega) := \f 1{\omega^4 + \alpha^4}$ gelten $\phi \in C^\infty (\R \to \R)$ und $|\phi^{(k)}(\omega)| \le c_k$.
	Damit ist $\phi \hat f \in \scr S(\R)$ (siehe \coursehref{blatt09.pdf}{Übungsaufgabe 9.4b}) und somit
	\[
		u = (\phi \hat f)^{\vee} \in \scr S(\R).
	\]
	Also existiert die Lösung und ist eindeutig.

	Manche Menschen würden die Lösung auch gerne ausrechnen können.
	Wie berechnet man $(\phi \hat f)^{\vee}$ aus $\check \phi$ und $f$?
\end{ex}

\begin{df} \label{4.12}
	Sei $f, g \in \scr S(\R)$.
	Die \emph{Faltung} $f \ast g$ von $f$ mit $g$ ist definiert durch
	\[
		(f \ast g)(y) := \int_{-\infty}^\infty f(y-x) g(x) \dx.
	\]
\end{df}

\begin{ex} \label{4.13}
	Die Abschneidefunktion aus \ref{4.2} lässt sich als Faltung schreiben:
	\begin{align*}
		\psi_{R,\eps} (x)
		&:= \int_{-R-\eps}^{R+\eps} j_\eps (x-y) \dx[y], \\
		&= \int_{-\infty}^\infty j_\eps(x-y) \cdot  \Ind_{[-R-\eps,R+\eps]}(y) \dx[y] \\
		&= \Big( j_\eps \ast \Ind_{[-R-\eps, R+\eps]} \Big) (x)
	\end{align*}
\end{ex}

\begin{st} \label{4.14}
	Für $f, g,h  \in \scr S(\R)$ gelten
	\begin{enumerate}[1)]
		\item
			$f \ast g \in \scr S(\R)$
		\item
			Es gilt
			\begin{align*}
				\widehat{f\cdot g} &= \f 1{\sqrt{2\pi}} \hat f \ast \hat g, \\
				\widehat{f\ast g} &= \sqrt{2\pi} \hat f \cdot \hat g.
			\end{align*}
			Insbesondere also $(f \ast g)^\vee = \sqrt{2\pi} \check f \cdot \check g$ und $(f \cdot g)^\vee = \f 1{\sqrt{2\pi}} \check f \ast \check g$
		\item
			Es gilt
			\begin{align*}
				f \ast (g \ast h) &=  (f\ast g) \ast h, \\
				f \ast g &= g \ast f.
			\end{align*}
	\end{enumerate}
	\begin{note}
		Mit dem Satz gilt für das Beispiel \ref{4.11}:
		\[
			u = (\phi \hat f)^\vee
			= \f 1{\sqrt{2\pi}}\check \phi \ast f
		\]
		Damit ist $u$ explizit berechenbar mittels $(\f 1{\omega^4 + \alpha^4})^\vee$.
		Beachte aber, dass $\f 1{\omega^4 + \alpha^4} \notin \scr S(\R)$.
	\end{note}
	\begin{proof}
		\begin{enumerate}[1)]
			\item
				Wegen $f, g \in \scr S(\R)$ ist nach \ref{4.8} $\hat f, \hat g \in \scr S(\R)$, also $\hat f \cdot \hat g \in \scr S(\R)$, also nach \ref{4.10} $(\hat f \hat g)^\vee \in \scr S(\R)$ und somit
				\[
					f \ast g \stack{2)}= \sqrt{2\pi} (\hat f \hat g)^\vee \in \scr S(\R).
				\]
			\item
				Die Fouriertransformation erhält das $L^2$-Skalarprodukt (siehe \ref{4.16}), daher gilt
				\begin{align*}
					\sqrt{2\pi} \widehat{f g}(\omega)
					&= \int_{-\infty}^\infty f(x) g(x) e^{-i\omega x} \dx \\
					&= \< g, \_f e^{i\omega \argdot} \>_{L^2(\R)}
					= \Big\< \hat g, \widehat{\_ f e^{i\omega \argdot}} \Big\>_{L^2(\R)} \\
					&= \int_{-\infty}^\infty \hat g(\tilde \omega) \_{\bigg( \f 1{\sqrt{2\pi}} \int_{-\infty}^\infty \_f(x) e^{i\omega x} e^{-i \tilde \omega x} \dx \bigg)} \dx[\tilde \omega] \\
					&= \int_{-\infty}^\infty \hat g(\tilde \omega) \underbrace{\f 1{\sqrt{2\pi}} \int_{-\infty}^{\infty} f(x) e^{-i(\omega - \tilde \omega)x} \dx}_{=\hat f(\omega - \tilde \omega)} \dx[\tilde \omega] \\
					&= \int_{-\infty}^\infty \hat g(\tilde \omega)\hat f(\omega - \tilde \omega) \dx[\omega] \\
					&= (\hat f \ast \hat g)(\omega)
				\end{align*}
				Analog zeigt man
				\[
					(\hat f \cdot \hat g)^\vee = \f 1{\sqrt{2\pi}} f \ast g.
				\]
			\item
				Es gilt
				\begin{align*}
					f \ast (g \ast h)
						&= \Big(\sqrt{2\pi} \hat f \cdot (\widehat{g \ast h})\Big)^\vee
						= \Big(2\pi \hat f \cdot \hat g \cdot \hat h \Big)^\vee
						= \dotso
						= (f \ast g) \ast h, \\
					f \ast g
						&= \Big(\sqrt{2\pi} \hat f \hat g\Big)^\vee
						= g \ast f.
				\end{align*}
		\end{enumerate}
	\end{proof}
\end{st}

\coursetimestamp{12}{6}{2013}
\begin{st}[Plancherel-Gleichung] \label{4.15}
	Für $f \in \scr S(\R)$ gilt
	\[
		\|Ff\|_{L^2(\R)} = \|f\|_{L^2(\R)},
	\]
	die Fouriertransformation erhält also die $L^2$-Norm.
	\begin{proof}
		Wir verfahren ähnlich wie im Beweis zu \ref{4.10}.
		\begin{enumerate}[1)]
			\item
				Zunächst sei $f \in C_0^\infty(\R)$, $\eps > 0$ mit $\supp f \subset ]-\f 1\eps, \f 1\eps[ =: I_\eps$.
				In $L^2(I_\eps)$ ist $(e_j)_{j\in \Z}$ mit
				\[
					e_j(x) = \sqrt{ \f\eps 2} e^{ij\pi \eps x}
				\]
				ein VONS, also $\forall f \in L^2(I_\eps) : f = \sum_{i=-\infty}^\infty \<f,e_j\> e_j$.

				Mit der Parsevalschen Gleichung folgt:
				\begin{align*}
					\|f\|_{I_\eps}^2
					&= \sum_{j=-\infty}^\infty |\<f,e_j\>|^2
					= \sum_{j=-\infty}^\infty \f \eps 2 \bigg| \int_{-\f 1\eps}^{\f 1\eps} f(x) e^{-ij\pi \eps x} \dx \bigg|^2 \\
					&= \sum_{j=-\infty}^\infty \f \eps 2 \bigg| \underbrace{\int_{-\infty}^{\infty} f(x) e^{-ij\pi \eps x} \dx}_{\sqrt{2\pi} \hat f (j\pi \eps)} \bigg|^2
					= \sum_{j=-\infty}^\infty \underbrace{\pi \eps}_{\Delta \omega_j} \Big| \hat f(\underbrace{j\pi \eps}_{\omega_j})\Big|^2 \\
					&\stack{\eps \to 0}\to \int_{-\infty}^\infty |\hat f(\omega)|^2 \dx[\omega]
					= \|\hat f\|_{L^2(\R)}.
				\end{align*}
				Also gilt für $f \in C_0^\infty(\R) : \|\hat f\|_{L^2(\R)} = \|f\|_{L^2(\R)}$.
			\item
				Sei jetzt $f \in \scr S(\R)$.
				Multipliziere mit der Abschneidefunktion, dann ist $\psi_{R,1} f \in C_0^\infty(\R)$.
				Aus Teil 1) und dem Beweis von \ref{4.10} entnehmen wir:
				\begin{enumerate}[a)]
					\item
						$|(\psi_{R,1} f)(\omega)| \le \f {c_1}{1+\omega^2}$,
					\item
						$\|\psi_{R,1} f\|_{L^2(\R)} = \| \widehat{\psi_{R,1} f} \|_{L^2(\R)}$,
					\item
						$|\hat{\psi_{R,1} f}(\omega)| \le \f {c_2}{1+\omega^2}$,
					\item
						$\lim\limits_{R\to\infty} \hat{\psi_{R,1} f} = \hat f$.
				\end{enumerate}
				Mit majorisierter Konvergenz folgt ähnlich wie in \ref{4.10}
				\begin{align*}
					\|f\|_{L^2(\R)}
					= \Big\| \lim_{R\to\infty} \psi_{R,1} f \Big\|_{L^2(\R)}
					&\stack{\text{a)}}= \lim_{R\to\infty} \Big\| \psi_{R,1} f \Big\|_{L^2(\R)} \\
					&\stack{\text{b)}}= \lim_{R\to\infty} \Big\| \hat{\psi_{R,1} f} \Big\|_{L^2(\R)}
					\stack{\text{c)}}= \Big\| \lim_{R\to\infty} \hat{\psi_{R,1} f} \Big\|_{L^2(\R)}
					\stack{\text{d)}}= \big\| \hat f \big\|_{L^2(\R)}.
				\end{align*}
		\end{enumerate}
	\end{proof}
\end{st}

\begin{kor} \label{4.16}
	Für $f,g \in \scr S(\R)$ gilt
	\[
		\<f,g\>_{L^2(\R)} = \< Ff, Fg\>_{L^2(\R)}.
	\]
	\begin{proof}
		Folgt direkt mit \ref{4.15} und der Polarisationsformel:
		\[
			\<f,g\> = \f 14 \Big( \|f+g\|^2 - \|f-g\|^2 + i \Big(\|f+ig\|^2 - \|f-ig\|^2 \Big) \Big).
		\]
	\end{proof}
\end{kor}

\begin{st} \label{4.17}
	Für $f \in \scr S(\R)$ gilt
	\[
		\| \hat f \|_{L^\infty(\R)} \le \f 1{\sqrt{2\pi}} \|f\|_{L^1(\R)}.
	\]
	\begin{proof}
		Für alle $\omega \in \R$ ist
		\begin{align*}
			|\hat f(\omega)|
			&= \bigg| \f 1{\sqrt{2\pi}} \int_{-\infty}^\infty f(x) e^{-i\omega x} \dx \bigg|
			\le \f 1{\sqrt{2\pi}} \int_{-\infty}^\infty |f(x)| 1 \dx
			= \f 1{\sqrt{2\pi}} \|f\|_{L^1(\R)}.
		\end{align*}
		Also $\| \hat f \|_{L^\infty(\R)} \le \f 1{\sqrt{2\pi}} \|f\|_{L^1(\R)}$.
	\end{proof}
\end{st}


\section{Dichte Mengen}


\begin{st} \label{4.18}
	Sei $1 \le p < \infty$.
	Dann ist $C_0^\infty$ dicht in $L^p(\R)$.
\coursetimestamp{17}{6}{2013}
	\begin{proof}
		% FIXME: Der ganze Beweis sollte überarbeitet werden
		Sei $u \in L^p(\R)$ und o.B.d.A. $u(x) \ge 0$ auf $\R$ (sonst $u = \Re(u)_+ - \Re(u)_- + i(\Im(u)_+ - \Im(u)_-)$).
		Approximiere nun $u$ durch $C_0^\infty(\R)$-Funktionen.
		\begin{enumerate}[1)]
			\item
				Mache Träger kompakt:
				\[
					v_k := \Ind_{[-k,k]}\cdot u.
				\]
				Dann gilt
				\begin{itemize}
					\item
						$|v_k(x)-u(x)|^p \to 0$ für $k\to \infty$ für jedes feste $x \in \R$,
					\item
						$|v_k(x)-u(x)|^p \le |u(x)|^p$, $\int_{-\infty}^\infty |u(x)|^p \dx < \infty$.
				\end{itemize}
				Mit majorisierter Konvergenz gilt $\int_{-\infty}^\infty |v_k(x) - u(x)|^p \dx \to 0$, also
				\[
					\|v_k - u\|_{L^p(\R)} \to 0.
				\]
				Wähle $u_1 \in L^p(\R)$ mit $\|u-u_1\|_{L^p(\R^n)} < \eps$ und $\supp u_1 \subset [-R,R]$ für ein hinreichend großes $R > 0$.
			\item
				Aus der Maßtheorie wissen wir: Es gibt eine Folge $(\phi_k)_{k\in \N}$ einfacher Funktionen mit
				\begin{itemize}
					\item
						$\forall u \in \R : 0 \le \phi_k(x) \le u_1(x)$ (insbesondere $\supp \phi_k \subset [-R,R]$),
					\item
						$\phi_k(x) \nearrow u_1(x)$ punktweise auf $\R$.
				\end{itemize}
				Mit majorisierter Konvergenz wissen wir $\|u_1 - \phi_k\|_{L^p(\R)} \to 0$.

				Wähle $u_2$ als einfache Funktion
				\[
					u_2 := \sum_{j=1}^n \lambda_j \Ind_{B_j},
				\]
				mit $B_j$ messbar und $\|u_1-u_2\|_{L^p(\R)} < \eps$.

				Es gilt $B_j \in \text{Lebesgue-Borel-$\sigma$-Algebra}$.
				Wähle $A \in \text{Borel-$\sigma$-Algebra}$ mit $\my(A_j \setminus B_j) = 0 = \my(B_j \setminus A_j)$.

				Setze
				\[
					u_3 := \sum_{j=1}^n \lambda_j \Ind_{A_j},
				\]
				womit $\|u_2 - u_3\|_{L^p(\R)} = 0$ gilt.
				Es gelte o.B.d.A. $\supp u_3 \subset [-R,R]$ (schneide $A_j$ mit $[-R,R]$).
			\item
				Approximiere jedes $A_j$ durch endlich viele abgeschlossene Intervalle. \\
				Sei $A \in \text{Borel"=$\sigma$"=Algebra}$, $A \subset [-R,R]$.
				Das Maß von $A$ ist definiert als
				\[
					\my(A) := \inf \Big\{ \sum_{j=1}^\infty \my(I_j) : A \subset \bigcup_{j=1}^\infty I_j, I_j = [a_j, b_j] \Big\}.
				\]
				Wähle für $\tilde \eps > 0$ also o.B.d.A $(I_j)$ mit $I_j \subset [-R,R]$, $A \subset \bigcup_{j=1}^\infty I_j$, $\my(A) + \tilde \eps^p < \sum_{j=1}^\infty \my(I_j)$.
				Wegen $\my(A) \le \sum_{j=1}^\infty \my(I_j)$ ist $0 \le \sum_{j=1}^\infty \my(I_j) - \my(A) \le \tilde \eps^p$.
				Also
				\begin{align*}
					0
					\le \int_{-\infty}^\infty \underbrace{\Big(\Ind_{\bigcup_{j=1}^\infty I_j} - \Ind_{A_j}\Big)}_{=(\argdot)^p \text{ da $\in \{0,1\}$}} \dx
					= \my\bigg(\bigcup_{j=1}^\infty I_j \bigg) - \my(A)
					\le \sum_{j=1}^\infty u(I_j) - \my(A)
					< \tilde \eps^p.
				\end{align*}
				Und somit
				\[
					\Big\|\Ind_{\bigcup_{j=1}^\infty I_j} - \Ind_{A_j}\Big\|_{L^p(\R)} < \tilde \eps.
				\]
				Wähle $N \in \N$ mit $\sum_{i=N+1}^\infty \my(I_j) < \tilde \eps^p$, dann gilt
				\begin{align*}
					0
					\le \int_{-\infty}^\infty \underbrace{\Big(\Ind_{\bigcup_{j=1}^\infty I_j} - \Ind_{\bigcup_{j=1}^N I_j}\Big)}_{=(\argdot)^p}
					\le \int_{-\infty}^\infty \Big(\Ind_{\bigcup_{j=N+1}^\infty I_j}\Big)
					\le \sum_{j=N+1}^\infty \my(I_j)
					< \tilde \eps^p,
				\end{align*}
				also
				\[
					\Big\| \Ind_{\bigcup_{j=1}^\infty I_j} - \Ind_{\bigcup_{j=1}^N I_j} \Big\|_{L^p(\R)} < \tilde \eps.
				\]
				Zusammengefasst gilt damit
				\[
					\Big\| \Ind_A - \underbrace{\Ind_{\bigcup_{j=1}^N I_j}}_{\mathclap{\text{nur noch endlich viele}}} \Big\|_{L^p(\R)} < 2 \tilde \eps.
				\]
				Wähle also
				\[
					u_4 := \sum_{j=1}^m  \tilde \lambda_j \Ind_{[a_j,b_j]}
				\]
				mit $\|u_3 - u_4\|_{L^p(\R)} < \eps$.
			\item
				Zu $[a_j,b_j]$ existiert $f_j \in C(\R \to \C)$ mit $\supp f_j \subset [a_j, b_j]$, $\|\Ind_{[a_j,b_j]} - f\|_{L^p(\R)} < \tilde \eps$.
				% FIXME: Bild: Approximation durch „Trapez“

				Wähle also
				\[
					u_5 = \sum_{j=1}^m \tilde \lambda_j f_j
				\]
				sodass $\|u_4 - u_5\|_{L^p(\R)} < \eps$, $u_5 \in C(\R \to \C)$ und immernoch $\supp u_5 \subset [-R,R]$.
			\item
				Mit \ref{4.20} 4) gilt für stetige Funktionen mit kompaktem Träger, dass $j_\delta \ast u_5 \in C_0^\infty(\R \to \C)$ und
				\[
					\Big\|j_\delta \ast u_5 - u_5 \Big\|_{L^p(\R)} < \eps \qquad \text{für $\delta > 0$ genügend klein.}
				\]
		\end{enumerate}
		Insgesamt gilt damit
		\[
			\Big\|u - j_\delta \ast u_5 \Big\| < 5\eps.
		\]
	\end{proof}
\end{st}

\begin{df} \label{4.19}
	Für $f \in L^p(\R)$ setze
	\[
		J_\eps f := j_\eps \ast f,
	\]
	d.h.
	\[
		J_\eps f(x) = \int_{-\infty}^\infty j_\eps (x-y) f(y) \dx[y].
	\]
	$J_\eps$ heißt \emph{Glättungsoperator} (oder \emph{Mollifier}).
	\begin{note}
		Die wichtigen Eigenschaften von $j_\eps$ waren:

		$j_\eps \in C_0^\infty(\R)$, $j_\eps \ge 0$ auf $\R$, $\supp j_\eps = [-\eps,\eps]$, $\int_{\R} j_\eps(x) \dx = \int_{-\eps}^\eps j_\eps(x) \dx = 1$.
	\end{note}
\end{df}

\begin{st} \label{4.20}
	Für $1 \le p < \infty$, $u \in L^p(\R)$ gelten
	\begin{enumerate}[1)]
		\item
			$J_\eps u \in C^\infty (\R \to \C)$ für $\eps > 0$.
		\item
			$\supp u$ beschränkt $\implies J_\eps u \in C_0^\infty (\R)$.
		\item
			$J_\eps u \in L^p(\R)$, $\|J_\eps u \|_{L^p(\R)} \le \|u\|_{L^p(\R)}$
		\item
			$\|J_\eps u - u \|_{L^p(\R)} \to 0$ für $\eps \searrow 0$.
	\end{enumerate}
	\begin{proof}
		\begin{enumerate}[1)]
			\item
				Zeige
				\[
					\ddx[x^k] (J_\eps u)(x) = \int_{-\infty}^\infty \ddx[x^k] j_\eps (x-y) u(y) \dx[y]
				\]
				für $k \in \N$.

				Betrachte $k=1$ (größere $k$ lassen sich induktiv analog zeigen).
				Sei $\eps > 0$, dann gilt
				\begin{align*}
					&\Bigg| \dfrac {J_\eps u(x+k) - J_\eps u(x)}{h} - \int_{-\infty}^\infty j_\eps' (x-y) u(y) \dx[y] \Bigg| \\
					&\qquad = \Bigg| \int_{-\infty}^\infty \Big( \underbrace{\f 1h \big(j_\eps(x+h-y)-j_\eps(x-y)\big) }_{=j_\eps'(x+\theta h -y)} - j_\eps'(x-y) \Big) u(y) \dx[y] \Bigg|
				\intertext{wobei $0 < \theta < 1$ von $x-y$ und $h$ abhängig}
					&\qquad\le \int_{-\infty}^\infty \Big| j_\eps'(x+\theta h -y) - j_\eps'(x-y) \Big| |u(y)| \dx[y] \\
					&\qquad\stack{\text{Hölder}}\le \; \Bigg(\int_{-\infty}^\infty \Big| j_\eps'(x+\theta h -y) - j_\eps'(x-y) \Big|^q \dx[y] \Bigg)^{\f 1q} \|u\|_{L^p(\R)} \qquad \text{ mit $\f 1p + \f 1q = 1$, $p > 1$}
				\end{align*}
				Für den Teil mit der $q$-Norm gilt:
				\begin{align*}
					\Bigg(\int_{-\infty}^\infty \Big| j_\eps'(x+\theta h -y) - j_\eps'(x-y) \Big|^q \dx[y] \Bigg)
					\;&\stack{z=x-y}=\; \int_{-\infty}^\infty \Big| j_\eps'(z+\theta h) - j_\eps' (z) \Big|^q \dx[z] \\
					&= \int_{-2\eps}^{2\eps} \Big| j_\eps'(z+\theta h) - j_\eps' (z) \Big|^q \dx[z] \qquad \text{falls $|h| < \eps$}
					\intertext{die stetige Funktion $j_\eps'$ ist auf dem kompakten Intervall $[-2\eps, 2\eps]$ gleichmäßig stetig, also $|j_\eps'(z+\theta h) - j_\eps'(z)| < \tilde \eps$ für $|\theta h| < \delta$ und somit}
					&= 4\eps \tilde \eps^q \qquad \text{für $|h| < \delta$}
				\end{align*}
				Für den Fall $p = 1$ schreibt man:
				\begin{align*}
					&\Bigg| \dfrac {J_\eps u(x+k) - J_\eps u(x)}{h} - \int_{-\infty}^\infty j_\eps' (x-y) u(y) \dx[y] \Bigg| \\
					&\qquad \le \dotsb \le \sup_{x-y \in\R} \Big|j_\eps'(x+\theta h -y) - j_\eps'(x-y)\Big| \|u\|_{L^1(\R)} < \tilde \eps \|u\|_{L^1(\R)},
				\end{align*}
				für $|h| < \delta$.
			\item
				Falls $\supp u \subset [a,b]$, $a<b$, dann ist $\supp(J_\eps u) \subset [a-\eps, b+\eps]$ und somit $J_\eps u \in C_0^\infty(\R)$.
			\item
				\begin{enumerate}[a)]
					\item
						Zeige zunächst $|J_\eps u(x)| \le \Big( \int_{\R} j_\eps(x-y) |u(y)|^p \dx[y] \Big)^{\f 1p}$.

						Für $p=1$ ist die Aussage klar (Betrag ins Integral ziehen).
						Sei also $1 < p < \infty$:
						\begin{align*}
							|J_\eps u(x)|
							&= \bigg| \int_{-\infty}^\infty \underbrace{\big(j_\eps(x-y)\big)^{\f 1q}}_{\ge 0} \big(j_\eps(x-y)\big)^{\f 1p} u(y) \dx[y] \bigg| \\
							&\stack{\text{Hölder}}\le \underbrace{\bigg( \int_{-\infty}^\infty \big(j_\eps(x-y)\big)^{\f qq} \dx[y] \bigg)^{\f 1q}}_{=1} \bigg( \int_{-\infty}^\infty \big(j_\eps(x-y)\big)^{\f pp} |u(y)|^p \dx[y] \bigg)^{\f 1p}
						\end{align*}
					\item
						Es gilt
						\begin{align*}
							\| J_\eps u \|_{L^p(\R)}^p
							&= \int_{-\infty}^\infty |J_\eps u(x)|^p \dx \\
							&\stack{\text{a)}}\le \int_{-\infty}^\infty \bigg( \int_{-\infty}^\infty j_\eps(x-y) |u(y)|^p \dx[y] \bigg) \dx \\
							&\stack{\text{Fubini}}=\; \int_{-\infty}^\infty |u(y)|^p \underbrace{\bigg( \int_{-\infty}^\infty j_\eps(x-y) \dx \bigg)}_{=1} \dx[y] \qquad \text{($j_\eps \ge 0, |u|^p \ge 0$)} \\
							&= \|u\|_{L^p(\R)}^p < \infty.
						\end{align*}
						Also insbesondere $J_\eps u \in L^p(\R)$.
				\end{enumerate}
			\item
				% FIXME: Nochmal überprüfen
				Sei zunächst $u \in C_0^\infty(\R)$ und $a\le b$ so, dass $\supp u \subset [a,b]$.
				Dann gilt wegen $\int j_\eps(x) \dx = 1$:
				\begin{align*}
					|J_\eps u(x) - u(x)|
					&= \bigg| \int_{-\infty}^\infty j_\eps(x-y) \Big(u(y) - u(x)\Big) \dx[y] \bigg| \\
					&\le \int_{x-\eps}^{x+\eps} \underbrace{j_\eps(x-y)}_{\int \argdot \dx[y] = 1} \underbrace{|u(y)-u(x)|}_{< \delta \text{ für $\eps < \eps_0$}} \dx[y] \\
					&< \delta
				\end{align*}
				($u$ stetig auf kompakter Menge, also gleichmäßig stetig).
				Also
				\begin{align*}
					\|J_\eps u - u\|_{L^p(\R)}^p
					&= \int_{-\infty}^\infty \underbrace{|J_\eps u(x) - u(x)|^p}_{\supp(\argdot) \subset [a-\eps, b+ \eps]} \dx \\
					&< \delta^p (b-a + 2\eps_0) \qquad \text{für $\eps < \eps_0$} \\
					&\to 0 \qquad \text{für $\eps \searrow 0$}
				\end{align*}
				Für allgemeine $u \in L^p(\R)$ folgt die Aussage aus \ref{4.18}:
				Sei dazu $(u_n)_{n\in \N}$ eine Folge in $C_0^\infty(\R)$ mit $\|u_n - u\|_{L^p} \to 0$ und $\delta > 0$.
				Wähle $N_\delta$ so groß, dass
				\[
					\|J_\eps u - J_\eps u_n\|_{L^p(\R)}
					= \|J_\eps (u-u_n)\|_{L^p(\R)}
					\stack{3)}\le \|u - u_n\|_{L^p(\R)}
					< \delta
				\]
				für $n\ge N_\delta$ ($N_\delta$ unabhängig von $\eps$).
				Wähle anschließend $\eps_0$ so dass $\|J_\eps u_{N_\delta} - u_{N_\delta}\| < \delta$ für $\eps < \eps_0$ (siehe oben).
				Dann gilt
				\begin{align*}
					\|J_\eps - u\|_{L^p}
					&\le \underbrace{\|J_\eps u - J_\eps u_{N_\delta}\|_{L^p(\R)}}_{<\delta} + \underbrace{\|J_\eps u_{N_\delta} - u_{N_\delta}\|_{L^p(\R)}}_{<\delta} + \underbrace{\|u_{N_\delta} - u\|_{L^p(\R)}}_{<\delta} \\
					&< 3\delta
				\end{align*}
				für $\eps < \eps_0$.
		\end{enumerate}
	\end{proof}
\end{st}

\begin{st}[Satz von Plancherel] \label{4.21}
	\begin{enumerate}[1)]
		\item
			Die Fouriertransformation $F: \scr S(R) \to \scr S(R)$ besitzt eine eindeutige stetige Fortsetzung $\scr F:L^2(\R) \to L^2(\R)$.
			Diese Fortsetzung ist unitär, d.h.
			\[
				\forall f,g \in L^2 (\R) : \<\scr F f, \scr Fg \> = \<f,g\>
			\]
		\item
			Sei $\scr G: L^2(\R) \to L^2(\R)$ die eindeutige Fortsetzung von $F^{-1}$ auf $L^2(\R)$.
			Dann ist $\scr G = \scr F^{-1}$.
	\end{enumerate}
	\begin{proof}
		\begin{enumerate}[1)]
			\item
				In \coursehref{blatt10.pdf}{Übungsaufgabe 10.2} wird gezeigt: $\scr F$ existiert, ist eindeutig und $\|\scr Ff\|_{L^2(\R)} = \|f\|_{L^2(\R)}$ für $f \in L^2(\R)$.
				Mit der Polarisationsformel folgt die Aussage.
			\item
				Existenz und Eindeutigkeit folgt wieder aus \coursehref{blatt10.pdf}{Übungsaufgabe 10.2}.

				Zeige $\scr G \circ \scr F = \Id = \scr F \circ \scr G$.
				Zu $f \in L^2(\R)$ wähle nach \ref{4.18} $(f_n)$ in $C_0^\infty(\R) \subset \scr S(\R)$ mit $\|f-f_n\|_{L^2(\R)} \to 0$.
				Dann ist wegen der Linearität und der Isometrie $\|\scr F f  - \scr F f_n\|_{L^2(\R)}\to 0$,
				\[
					\scr F f = \lim[L^2]_{n\to \infty} F f_n,
				\]
				und analog für $\scr G$, also
				\[
				   \scr G(\scr F f) = \lim[L^2]_{n\to\infty} F^{-1}(F(f_n)) = \lim[L^2]_{n\to\infty} f_n = f.
				\]
		\end{enumerate}
	\end{proof}
\end{st}

\begin{nt} \label{4.22}
	Für $f \in L^2(\R)$ ist $\int_{-\infty}^\infty f(x) e^{-i \omega x} \dx$ nicht immer konvergent.
	Beispielsweise
	\[
		f(x) :=  \f{\sin(x^2)}{1+|x|}.
	\]
	Folglich lässt sich die Fouriertransformation auf $L^2(\R)$ im Allgemeinen nicht durch die Integraldarstellung berechnen.
	Stattdessen ist sie lediglich durch das abstrakte Fortsetzungsargument gegeben.
\end{nt}

\begin{st} \label{4.23}
	Für $f \in L^2(\R)$ gilt
	\[
		\scr F f = \f 1{\sqrt{2\pi}} \lim[L^2]_{R\to\infty} \int_{-R}^R f(x)e^{-i\omega x} \dx
	\]
	\begin{proof}
		\begin{enumerate}[1)]
			\item
				Setze $f_R := \Ind_{[-R,R]} \cdot f$, also $\|f_R - f\|_{L^2(\R)} \to 0$ für $R \to \infty$.
				Da $\scr F$ stetig ist, gilt weiter $\scr F f = \lim[L^2]_{R\to\infty} \scr F f_R$.
			\item
				Setze für festes $R$
				% FIXME: Notation verbessern, später wird \hat{f_n} verwendet und die Abhängigkeit von R ist nicht ersichtlich.
				\[
					\hat f(\omega)  := \int_{-\infty}^\infty f_R(x) e^{-i\omega x} \dx
					= \int_{-R}^R f(x) e^{-i\omega x} \dx
				\]
				und zeige $\hat f = \scr F f_R$.

				Setze
				\[
					f_n := j_{\f 1n} \ast f_R
					= J_{\f 1n} (f_R) \in C_0^\infty(\R) \qquad \text{nach \ref{4.20}, 2)},
				\]
				Es gelten
				\begin{enumerate}[a)]
					\item
						Es gilt
						\begin{align*}
							|\hat f_n(\omega) - \hat f(\omega)|
							& \le \f 1{\sqrt{2\pi}} \underbrace{\int_{-R-\f 1n}^{R+\f 1n} |f_n(x) - f_R(x)|\cdot |e^{-i\omega x} \dx|}_{= \<|f_n-f_R|,1\>} \\
							&\stack{CSB}\le \f 1{\sqrt{2\pi}} \underbrace{\|f_n - f_R\|_{L^2(\R)}}_{\to 0} \underbrace{\sqrt{\int_{-\R-\f 1n}^{R+\f 1n} \dx}}_{= \sqrt{2R + \f 2n} \to 2R}
							\to 0 \qquad (n\to \infty).
						\end{align*}
					\item
						Wegen $f_n \in C_0^\infty(\R)$ ist $\hat f_n = F f_n = \scr F f_n$ also
						\begin{align*}
							\|\hat f_n - \scr F f_R\|_{L^2(\R)}
							&= \|\scr F(f_n -f_R)\|_{L^2(\R)} \\
							&= \|f_n - f_R\|_{L^2(\R)}
							\to 0 \qquad (n\to \infty)
						\end{align*}
						Insbesondere ist $\hat f_n$ Cauchy-Folge in $L^p$ und damit existiert nach \ref{2.14} eine Teilfolge $(\hat f_{n_k})$, sodass $\hat f_{n_k}(x) \to \hat f_R(x)$ $\my$-fast-überall für $k\to \infty$.
				\end{enumerate}
				Nach 2a) gilt punktweise $\hat f_{n_k} \to \hat f$ und nach 2b) gilt $\hat f_{n_k} \to \scr F f_R$ $\my$-fast-überall, also muss $\hat f = \scr F f_R$.
		\end{enumerate}
	\end{proof}
\end{st}

\coursetimestamp{19}{6}{2013}
\begin{st} \label{4.24}
	Die Fouriertransformation $F: \scr S(\R) \to \scr S(\R)$ besitzt eine eindeutige Fortsetzung
	\[
		\scr F: L^1(\R) \to L^\infty(\R)
	\]
	und es gilt
	\[
		\|\scr F\| = \f 1{\sqrt{2\pi}}.
	\]
	\begin{note}
		Wegen \coursehref{blatt10.pdf}{Übungsaufgabe 10.1} ist $\scr F$ also weiterhin durch die Integraldarstellung gegeben:
		\[
			(\scr F f)(\omega)
			:= \hat f (\omega)
			:= \f 1{\sqrt{2\pi}} \int_{-\infty}^\infty f(x) e^{-i\omega x} \dx.
		\]
	\end{note}
	\begin{proof}
		\begin{enumerate}[1)]
			\item
				Nach \ref{4.18} ist insbesondere $\scr S(\R)$ dicht in $L^1(\R)$.
				Existenz und Eindeutigkeit der Fortsetzung folgt dann aus \coursehref{blatt10.pdf}{Übungsaufgabe 10.2c)}.
			\item
				In \coursehref{blatt10.pdf}{Übungsaufgabe 10.1} zeigt man die Beschränktheit von $\scr F$ mit Schranke $\f 1{\sqrt{2\pi}}$, also
				\[
					\| \scr F\| = \sup_{\|F\|_{L^1(\R)}=1} \|\scr F f\|_{L^\infty(\R)} \le \f 1{\sqrt{2\pi}}.
				\]
				Sei $f := j_1(x) \in C_0^\infty(\R)$, dann ist
				\[
					\|f\|_{L^1(\R)} = \int_{-\infty}^\infty j_1(x) \dx = 1,
				\]
				und
				\[
					\scr F f(0)
					= F f(0)
					= \f 1{\sqrt{2\pi}} \int_{-\infty}^\infty j_1(x) e^{i\omega x} \dx \bigg|_{\omega = 0}
					= \f 1{\sqrt{2\pi}} \int_{-\infty}^\infty j_1(x) \dx
					= \f 1{\sqrt{2\pi}}
				\]
				also
				\[
					\| \scr F\| = \f 1{\sqrt{2\pi}}.
				\]
		\end{enumerate}
	\end{proof}
\end{st}

\begin{st} \label{4.25}
	Für $\scr F: L^1(\R) \to L^\infty(\R)$ gilt
	\[
		\im (\scr F) \subset \Big\{ \hat f \in C(\R \to \C) : \hat f(\omega) \to 0 \text{ für } \omega \to \pm \infty \Big\}
	\]
	\begin{proof}
		\begin{enumerate}[1)]
			\item
				Zeige zunächst die Stetigkeit von $\hat f$.
				Es gilt
				\begin{align*}
					| \hat f(\omega + h) - \hat f(\omega) |
					&= \f 1{\sqrt{2\pi}} \Big| \int_{-\infty}^\infty f(x) ( e^{-i(\omega + h)x} - e^{-i \omega x} ) \dx \Big| \\
					&\le \f 1{\sqrt{2\pi}} \int_{-\infty}^\infty |f(x)| | e^{-i(\omega + h)x} - e^{-i \omega x} | \dx \\
					&\le \underbrace{\f 1{\sqrt{2\pi}} \int_{-\infty}^{-R} |f(x)| 2 \dx
						+ \f 1{\sqrt{2\pi}} \int_{R}^{\infty} |f(x)| 2 \dx}_{< \eps \text{ für $R$ genügend groß}} \\
						&\quad + \underbrace{\f 1{\sqrt{2\pi}} \int_{-R}^{R} |f(x)| \underbrace{\underbrace{|e^{-i(\omega + h) x} - e^{-i \omega x}|}_{\text{gleichmäßig stetig für $(x,h) \in [-R,R] \times [-1,1]$}}}_{< \tilde \eps \text{ für $|h|<\delta$}} \dx}_{\le \f 1{\sqrt{2\pi}} \tilde \eps \|f\|_{L^1(\R)} < \eps \text{ für $|h|<\delta$}} \\
					&< 2\eps
				\end{align*}
				für $R$ genügend groß und $|h| < \delta$.
			\item
				Zeige jetzt $|\hat f(\omega)| \to 0$ für $\omega \to \infty$:
				\begin{align*}
					\sqrt{2\pi} |\hat f(\omega)|
					&\le  \underbrace{\int_{-\infty}^{-R} |f(x)| \dx + \int_{R}^{\infty} |f(x)| \dx}_{<\eps \text{ für $R > R_\eps$}}
					+ \underbrace{\underbrace{\Big| \int_{-R}^R f(x) e^{i\omega x} \dx \Big|}_{\to 0 \text{ für $\omega \to \infty$ nach \ref{3.25}}}}_{<\eps \text{ für $\omega > \Omega_\eps$}} \\
					&< 2\eps
				\end{align*}
				für $\omega > \Omega_\eps$.
		\end{enumerate}
	\end{proof}
\end{st}

\begin{st} \label{4.26}
	Sei $f \in L^1(\R)$, $x \in \R$ fest und es gelte die Bedingung von Dini aus \ref{3.27}:
	\[
		\exists \delta > 0 : \int_{-\delta}^{\delta} | \f {f(t+x)-f(x)}{t} | \dx[t] < \infty.
	\]
	Dann konvergiert das uneigentliche Riemann-Integral
	\[
		\f 1{\sqrt{2\pi}} \int_{-\infty}^\infty \scr F f(\omega) e^{i \omega x} \dx[\omega] = f(x).
	\]
\end{st}


\section{Fouriertransformation im \texorpdfstring{$\R^n$}{R\textasciicircum n}}


\begin{nt*}[Erinnerung]
	Sei $f : \R^n \to \C$.
	\begin{align*}
		\pddx[x_j] f(x) &:= \lim_{h\to 0} \f {f(x+he_j)-f(x)}{h} \\
		\f{\partial^3 }{\partial x_1 \partial x_3^2}f(x) &:= \pddx[x_1] \pddx[x_3] \pddx[x_3] f \\
		\f{\partial^{\alpha_1 + \alpha_2 + \dotsb + \alpha_n} }{\partial x_1^{\alpha_1} \partial x_2^{\alpha_2} \dotsb \partial x_n^{\alpha_n}}f(x)  &: \quad \text{Ableitung der Ordnung $\alpha_1 + \dotsb + \alpha_n$}
	\end{align*}
\end{nt*}

\begin{df} \label{4.27}
	$\alpha \in \N_0^n$ heißt \emph{Multiindex}.
	Man setzt
	\begin{align*}
		|\alpha| &:= \alpha_1 + \dotsb + \alpha_n, \\
		\alpha \le \beta &:\iff \alpha_1 \le \beta_1 \land \dotsb \land \alpha_n \le \beta_n, \\
		\binom{\alpha}{\beta} &:= \binom{\alpha_1}{\beta_1}  \dotsb \binom{\alpha_n}{\beta_n} \\
		x^\alpha &:= x_1^{\alpha_1}  \dotsb \cdot x_n^{\alpha_n} \qquad \text{für $x\in \R^n$.}
	\end{align*}
\end{df}

\begin{ex}[Anwendung] \label{4.28}
	Für $f, g \in C^\infty (\R^n \to \C)$, $\alpha \in \N_0^n$, $\lambda \in \R, x \in \R^n$ setzt man
	\begin{align*}
		D^\alpha f &:= \f {\partial^{|\alpha|}f}{\partial x_1^{\alpha_1} \dotsb \partial x_n^{\alpha_n}},\\
		(\lambda x)^\alpha &:= \lambda^{|\alpha|} x^\alpha.
	\end{align*}
	Die Leibnitz-Regel lässt sich schreiben als:
	\[
		D^\alpha(f g) = \sum_{\substack{\beta \in \N_0^n \\ \beta \le \alpha}} \binom{\alpha}{\beta} (D^\beta f)D^{\alpha - \beta} g.
	\]
\end{ex}

\begin{df} \label{4.29}
	\begin{enumerate}[1)]
		\item
			Wir nennen
			\[
				\scr S(\R^n) := \Big\{ f \in C^\infty(\R^n \to \C) \suchthat \forall \alpha \in \N_0^n \forall j\in \N_0 : \sup_{x\in \R^n} |x|^j |D^{\alpha} f(x)| < \infty \Big\}
			\]
			\emph{Schwartz-Raum} über $\R^n$.
		\item
			Für $f \in \scr S(\R^n)$ ist die Fouriertransformierte gegeben durch
			\[
				F f(\omega) := \hat f(\omega) := \f 1{(2\pi)^{\f n2}} \int_{\R^n} f(x) e^{-i \omega \cdot x} \dx
				\qquad (\omega \in \R^n)
			\]
			wobei mit $\omega \cdot x := \omega_1 x_1 + \dotsb + \omega_n \cdot x_n$ das Standard-Skalarprodukt im $\R^n$ gemeint ist.
	\end{enumerate}
\end{df}

\begin{st} \label{4.30}
	\begin{enumerate}[1)]
		\item
			Die Fouriertransformation $F: \scr S(\R^n) \to \scr S(\R^n) : f \mapsto \hat f$ ist bijektiv und $F^{-1} : f \mapsto \check f$ ist gegeben durch
			\[
				F^{-1}f(x) := \check f (x) := \f 1{(2\pi)^{\f n2}} \int_{\R^n} f(\omega) e^{i \omega \cdot x} \dx[\omega].
			\]
		\item
			Es gilt für $\omega \in \R^n, \alpha \in \N_0^n$:
			\[
				\widehat{D^\alpha f}(\omega) = (i\omega)^\alpha \hat f(\omega) = i^{|\alpha|} \omega^\alpha \hat f(\omega).
			\]
			Für $g(x) = x^\alpha \cdot f(x)$ ist
			\[
				\hat g(\omega) = i^{|\alpha|} (D^\alpha \hat f) (\omega)
			\]
		\item
			Definiere die Faltung:
			\[
				(f \ast g)(x) := \int_{\R^n} f(x-y) g(y) \dx[y] = \int_{\R^n} f(y) g(x-y) \dx[y].
			\]
			Dann gilt
			\begin{align*}
				\widehat{f\cdot g} &=  \f 1{(2\pi)^{\f n2 }} \hat f \ast \hat g, \\
				\widehat{f\ast g} &= (2\pi)^{\f n2} \hat f \cdot \hat g.
			\end{align*}
		\item
			Es gilt der Satz von Plancherel:
			\[
				\|F f\|_{L^2(\R^n)} = \|f\|_{L^2(\R^n)}.
			\]
		\item
			Für $1 \le p < \infty$ ist $C_0^\infty(\R^n)$ dicht in $L^p(\R^n)$.
	\end{enumerate}
\end{st}

\begin{nt}[Fortsetzungen] \label{4.31}
	\begin{enumerate}[1)]
		\item
			$F: \scr S(\R^n) \to \scr S(\R^n)$ besitzt eine eindeutige Fortsetzung $\scr F: L^2(\R^n) \to L^2(\R^n)$.

			$\scr F$ ist unitär: $\forall f, g \in L^2(\R^n) : \<f,g\> = \<\scr F f, \scr F g\>$.

			$\scr F$ kann berechnet werden durch
			\[
				\scr F f = \lim[L^2(\R^n)]_{R\to \infty} \f 1{(2\pi)^{\f n2}} \int_{|x|\le R} f(x) e^{i \omega\cdot x} \dx
			\]
		\item
			$F : \scr S(\R^n) \to \scr S(\R^n)$ besitzt eine eindeutige Fortsetzung $\scr F: L^1(\R^n) \to L^\infty(\R^n)$ mit $\| \scr F \| = \f 1{(2\pi)^{\f n2}}$ und
			\[
				\im (\scr F) \subset \Big\{ f \in C(\R^n \to \C) : f(x) \to 0 \text{ für $|x| \to \infty$} \Big\}.
			\]
	\end{enumerate}
\end{nt}

\begin{ex} \label{4.30}
	Sei $f \in \scr S(\R^n)$ gegeben.
	Wir suchen $u \in \scr S(\R^n)$ mit
	\[
		- \Delta u + u = f.
	\]

	Wegen $u \in \scr S(\R^n)$ ist auch $-\Delta u + u \in \scr S(\R^n)$.
	Wende die Fouriertransformation auf die DGL an:
	\[
		\omega_1^2 \hat u(\omega) + \dotsb + \omega_n^2 \hat u(\omega) + \hat u(\omega) = \hat f(\omega)
		\hat u (\omega) = \f 1{|\omega|^2 + 1} \hat f(\omega)
	\]

	\begin{st*}
		Sei $\hat f \in \scr S(\R^n), \phi \in C^\infty(\R^n)$ und $\forall \alpha \in \N_0^n : \sup_{x\in \R^n} |D^\alpha \phi(x)| < \infty$.
		Dann ist
		\[
			\phi \cdot \hat f \in \scr S(\R^n)
		\]
		\begin{proof}
			$\phi \cdot \hat f \in C^\infty(\R^n)$ ist klar.
			Es gilt außerdem
			\begin{align*}
				\sup_{x \in \R^n} |x|^j |D^\alpha (\phi \cdot \hat f)(x) |
				&\le \sum_{\beta \le \alpha} \binom{\alpha}{\beta} \sup_{x \in \R^n} |x|^j \underbrace{|D^\beta \phi(x)|}_{\le c_\beta} | D^{\alpha-\beta} \hat f(x)| \\
				&< \infty.
			\end{align*}
		\end{proof}
	\end{st*}

	Wegen $\hat f \in \scr S(\R^n)$ gilt $\f 1{|\argdot|^2+1} \hat f \in \scr S(\R^n)$, also ist $\hat u = \f 1{|\argdot|^2+1} \hat f \in \scr S(\R^n)$ eindeutige Lösung.

	Nach Rücktransformation ergibt sich
	\begin{align*}
		u(x)
		&= \Big( \f 1{|\argdot|^2+1} \hat f \Big)^\vee \\
		&=  \f 1{(2\pi)^{\f n2}} \underbrace{(\f 1{|\argdot|^2+1})^\vee}_{=:G} \ast \underbrace{(\hat f)^\vee}_{=f}\\
		&= \f 1{(2\pi)^{\f n2}} \int_{\R^n} \underbrace{G(x-y)}_{\mathclap{\text{entspricht Greenscher Funktion}}} f(y) \dx[y].
	\end{align*}
\end{ex}

\begin{nt}[Ausblick: Satz von Hausdorff-Young] \label{4.31}
	Für $1 \le p \le 2$, $\f 1p + \f 1q = 1$ besitzt $F : \scr S(\R) \to \scr S(\R)$ eine eindeutige stetige Fortsetzung
	\[
		\scr F_p : L^p(\R^n) \to L^q(\R^n)
	\]
	und es gilt
	\[
		\| \scr F_p \| = \f 1{(2\pi)^{n(\f 1p-\f 12)}}.
	\]
\end{nt}
