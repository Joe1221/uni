\chapter{Schälbarkeit}

\Timestamp{2015-05-20}

Wir betrachten einen Komplex $K$ mit konvexen Polytopen als Facetten und der homogenen Dimension $k$ (d.h. jede Seite ist in einer $k$-dimensionalen Facette enthalten „pure complex“) mit einer Ordnung $F_1, \dotsc, F_s$ der Facetten.

Spezialfall: Simplizialer Komplex der Dimension $k$.

$F_1, \dotsc, F_s$ heißt eine \emphdef{Schälung}, wenn entweder $k=0$ oder folgendes gilt:
\begin{enumerate}[1)]
    \item
        Der Randkomplex von $F_1$ hat eine Schälung (trivial, falls $F_1$ ein Simplex ist)
    \item
        Für $1 < j \le s$ ist der Durchschnitt von $F_j$ mit $F_1 \cup \dotsc \cup F_{j-1}$ nicht leer und ist der Anfang einer Schälung von $\Boundary F_j$ ($(k-1)$-dimensional).
\end{enumerate}
$K$ heißt \emphdef{schälbar} (“shellable”), wenn eine Schälung existiert.

\begin{st}[Euler]
    Sei $P$ ein konvexes $d$-Polytop mit schälbarem Rand $\Boundary P$.
    Dann gilt mit $f_{-1}(P) = f_d(P) = 1$ die Euler-Gleichung
    \begin{math}
        \sum_{i=-1}^d (-1)^i f_i(P) = 0,
    \end{math}
    wobei $f_i$ die Anzahl der $i$-dimensionalen Seiten angibt.
    Alternativ gilt für die Euler-Charakteristik
    \begin{math}
        \chi(P) &= \sum_{i=0}^d (-1)^i f_i(P), \\
        \chi(\Boundary P) &= \sum_{i=0}^d (-1)^i f_i(\Boundary P) = 1 - (-1)^d.
    \end{math}
    \begin{proof}
        Bekannt ist für einen $j$-Simplex $S$, dass $\chi(S) = 1$ und $\chi(\Boundary S) = 1 - (-1)^j$.
        Wir verwenden die Additivität von $\chi$:
        \begin{math}
            \chi(A \cup B) + \chi(A \cap B) = \chi(A) + \chi(B),
        \end{math}
        solange $A, B, A \cap B$ Unterkomplexe von $A \cup B$ sind.

        Der Fall $d = 1$ ist trivial, für $d = 2$ ergibt sich $f_0 = f_1$ und $1 - (-1)^d = 0$.
        Die Behauptung gelte für $d - 1$.
        Sei $P$ ein $d$-Polytop mit Schälung des Randes $F_1, \dotsc, F_s$.
        \begin{math}
            \chi(F_1) &= \chi(\texp{$(d-1)$-Polytop}) = 1, \\
            \chi(F_1 \cup F_2) &= \chi(F_1) + \chi(F_2) - \chi(F_1 \cap F_2) \\
            &= 1 + 1 - 1 = 1.
        \end{math}
        Induktiv folgt
        \begin{math}
            \chi(F_1 \cup \dotsb \cup F_j) = 1
        \end{math}
        für alle $j < s$.
        Nun ist
        \begin{math}
            \chi(\Boundary P) &= \chi(F_1 \cup \dotsb \cup F_{s-1}) + \chi(F_s) - \chi(\underbrace{(F_1 \cup \dotsb \cup F_{s-1}) \cap F_s}_{=\Boundary F_{s}}) \\
            &=1 + 1 - (1 - (-1)^{d-1}) \\
            &=1-(-1)^{d}.
        \end{math}
    \end{proof}
\end{st}

\begin{st}[Brugesser-Mani, 1971]
    Sei $P \subset \R^d$ ein konvexes $d$-Polytop.
    Dann ist der Randkomplex $\Boundary P$ schälbar.
    \begin{proof}[konstruktiv]
        Wähle $y_0 \in \mathring P$ und $x_0 \not\in P$ und betrachte einene laufenden Punkt $x$ auf dem Strahl $l$ von $y_0$ nach $x_0$.
        \begin{enumerate}[1.]
            \item
                Der erste Durchstoßpunkt $x$ mit $\Boundary P$ trifft in eine Facette $F_1$.
                Ab da „sieht“ man $F_1$ von $x$ aus.
            \item
                Später trifft $x$ auf die affine Hülle einer zweiten Facette $F_2$.
                Ab da „sieht“ man $F_1$ und $F_2$.
            \item
                Man setzt dies für $x \to \infty$ bis zu $F_1, \dotsc, F_k$ fort.
            \item
                Für die restlichen $F_{k+1}, \dotsc, F_s$ inverser Prozess nachdem man (projektiv gedacht) $\infty$ durchschritten hat.
        \end{enumerate}
        $F_1, \dotsc, F_s$ bildet eine Schälung.
        $\Boundary F_j \cap (F_1 \cup \dotsb \cup F_{j-1})$ sind alle Facetten von $F_j$, die man von $l \cap \Aff(F_j)$ „sieht“. 
        Durch Induktion liefert die Reihenfolge dafür eine Schälung.
        Der Induktionsanfang ist trivial (Dimension $k = 0$ ist stets schälbar).
        \begin{note}
            Durch die Wahl der Geraden lässt sich $F_1$ und $F_s$ vorschreiben.
        \end{note}
    \end{proof}
\end{st}

\begin{kor}
    Mit $F_1, \dotsc, F_s$ ist auch $F_s, \dotsc, F_1$ eine Schälung.
    \begin{proof}
        Umkehrung der Orientierung von $l$ im Beweis.
    \end{proof}
\end{kor}

\begin{ex}
    Schälung des Oktaeders:
    \begin{math}
        123, 126, 135, 156, 234, 246, 345, 456
    \end{math}
    Gemeinsamen Seiten mit dem Rest:
    \begin{math}
        3, 2, 2, 1, 2, 1, 1, 0
    \end{math}
    Gegenüberliegende Ecken:
    \begin{math}
        0, 1, 1, 2, 1, 2, 2, 3
    \end{math}
    Die Anzahl für $0, 1, 2, 3$ jeweils: $1, 3, 3, 1$ (Binomialkoeffizient).
    Diese Symmetrie ist typisch für simpliziale Polytope, siehe nächster Satz.
\end{ex}
