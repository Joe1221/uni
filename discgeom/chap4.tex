\chapter{Schälbarkeit}

\Timestamp{2015-05-20}

Wir betrachten einen Komplex $K$ mit konvexen Polytopen als Facetten und der homogenen Dimension $k$ (d.h. jede Seite ist in einer $k$-dimensionalen Facette enthalten „pure complex“) mit einer Ordnung $F_1, \dotsc, F_s$ der Facetten.

Spezialfall: Simplizialer Komplex der Dimension $k$.

$F_1, \dotsc, F_s$ heißt eine \emphdef{Schälung}, wenn entweder $k=0$ oder folgendes gilt:
\begin{enumerate}[1)]
    \item
        Der Randkomplex von $F_1$ hat eine Schälung (trivial, falls $F_1$ ein Simplex ist)
    \item
        Für $1 < j \le s$ ist der Durchschnitt von $F_j$ mit $F_1 \cup \dotsc \cup F_{j-1}$ nicht leer und ist der Anfang einer Schälung von $\Boundary F_j$ ($(k-1)$-dimensional).
\end{enumerate}
$K$ heißt \emphdef{schälbar} (“shellable”), wenn eine Schälung existiert.

\begin{st}[Euler]
    Sei $P$ ein konvexes $d$-Polytop mit schälbarem Rand $\Boundary P$.
    Dann gilt mit $f_{-1}(P) = f_d(P) = 1$ die Euler-Gleichung
    \begin{math}
        \sum_{i=-1}^d (-1)^i f_i(P) = 0,
    \end{math}
    wobei $f_i$ die Anzahl der $i$-dimensionalen Seiten angibt.
    Alternativ gilt für die Euler-Charakteristik
    \begin{math}
        \chi(P) &= \sum_{i=0}^d (-1)^i f_i(P), \\
        \chi(\Boundary P) &= \sum_{i=0}^d (-1)^i f_i(\Boundary P) = 1 - (-1)^d.
    \end{math}
    \begin{proof}
        Bekannt ist für einen $j$-Simplex $S$, dass $\chi(S) = 1$ und $\chi(\Boundary S) = 1 - (-1)^j$.
        Wir verwenden die Additivität von $\chi$:
        \begin{math}
            \chi(A \cup B) + \chi(A \cap B) = \chi(A) + \chi(B),
        \end{math}
        solange $A, B, A \cap B$ Unterkomplexe von $A \cup B$ sind.

        Der Fall $d = 1$ ist trivial, für $d = 2$ ergibt sich $f_0 = f_1$ und $1 - (-1)^d = 0$.
        Die Behauptung gelte für $d - 1$.
        Sei $P$ ein $d$-Polytop mit Schälung des Randes $F_1, \dotsc, F_s$.
        \begin{math}
            \chi(F_1) &= \chi(\texp{$(d-1)$-Polytop}) = 1, \\
            \chi(F_1 \cup F_2) &= \chi(F_1) + \chi(F_2) - \chi(F_1 \cap F_2) \\
            &= 1 + 1 - 1 = 1.
        \end{math}
        Induktiv folgt
        \begin{math}
            \chi(F_1 \cup \dotsb \cup F_j) = 1
        \end{math}
        für alle $j < s$.
        Nun ist
        \begin{math}
            \chi(\Boundary P) &= \chi(F_1 \cup \dotsb \cup F_{s-1}) + \chi(F_s) - \chi(\underbrace{(F_1 \cup \dotsb \cup F_{s-1}) \cap F_s}_{=\Boundary F_{s}}) \\
            &=1 + 1 - (1 - (-1)^{d-1}) \\
            &=1-(-1)^{d}.
        \end{math}
    \end{proof}
\end{st}

\begin{st}[Brugesser-Mani, 1971]
    Sei $P \subset \R^d$ ein konvexes $d$-Polytop.
    Dann ist der Randkomplex $\Boundary P$ schälbar.
    \begin{proof}[konstruktiv]
        Wähle $y_0 \in \mathring P$ und $x_0 \not\in P$ und betrachte einene laufenden Punkt $x$ auf dem Strahl $l$ von $y_0$ nach $x_0$.
        \begin{enumerate}[1.]
            \item
                Der erste Durchstoßpunkt $x$ mit $\Boundary P$ trifft in eine Facette $F_1$.
                Ab da „sieht“ man $F_1$ von $x$ aus.
            \item
                Später trifft $x$ auf die affine Hülle einer zweiten Facette $F_2$.
                Ab da „sieht“ man $F_1$ und $F_2$.
            \item
                Man setzt dies für $x \to \infty$ bis zu $F_1, \dotsc, F_k$ fort.
            \item
                Für die restlichen $F_{k+1}, \dotsc, F_s$ inverser Prozess nachdem man (projektiv gedacht) $\infty$ durchschritten hat.
        \end{enumerate}
        $F_1, \dotsc, F_s$ bildet eine Schälung.
        $\Boundary F_j \cap (F_1 \cup \dotsb \cup F_{j-1})$ sind alle Facetten von $F_j$, die man von $l \cap \Aff(F_j)$ „sieht“. 
        Durch Induktion liefert die Reihenfolge dafür eine Schälung.
        Der Induktionsanfang ist trivial (Dimension $k = 0$ ist stets schälbar).
        \begin{note}
            Durch die Wahl der Geraden lässt sich $F_1$ und $F_s$ vorschreiben.
        \end{note}
    \end{proof}
\end{st}

\begin{kor}
    Mit $F_1, \dotsc, F_s$ ist auch $F_s, \dotsc, F_1$ eine Schälung.
    \begin{proof}
        Umkehrung der Orientierung von $l$ im Beweis.
    \end{proof}
\end{kor}

\begin{ex}
    Schälung des Oktaeders:
    \begin{math}
        123, 126, 135, 156, 234, 246, 345, 456
    \end{math}
    Gemeinsamen Seiten mit dem Rest:
    \begin{math}
        3, 2, 2, 1, 2, 1, 1, 0
    \end{math}
    Gegenüberliegende Ecken:
    \begin{math}
        0, 1, 1, 2, 1, 2, 2, 3
    \end{math}
    Die Anzahl für $0, 1, 2, 3$ jeweils: $1, 3, 3, 1$ (Binomialkoeffizient).
    Diese Symmetrie ist typisch für simpliziale Polytope, siehe nächster Satz.
\end{ex}

\Timestamp{2015-06-03}

Wir setzen im folgenden einen simplizialen, schälbaren $(d-1)$-dimensionalen Komplex voraus (z.B. Randkomplex eines konvexen simplizialen $d$-Polytops $P$).üm
Facetten sind $d$-Tupel von Ecken.
Sei $F_1, \dotsc, F_s$ feste Schälung.
Idee: Alle Seiten aufbauen durch diese Schälung.

Definiere $R_j$ (restriction) durch
\begin{math}
    R_j = \Set{v \in F_j & \text{$v$ Ecke, $F_j \setminus \Set{v} \subset F_i$ für ein $1 \le i < j$}}
\end{math}
insbesondere $R_1 = \emptyset$.

Eine $(k-1)$-Seite $G$, die beim Schälen im $j$-ten Schritt vorkommt, ist in $F_j$ enthalten, aber: Wenn ein $v \in R_j$ \emph{nicht} in $G$ ist, dann gab es $G$ schon vorher.
Es folgt $R_j \subset G \subset F_j$.

Umgekehrt: Sei $R_j \subset G \subset F_j$.
Angenommen $G$ kommt schon vorher vor $G \subset F_i$, $i <j$, dann existiert $F_l$ mit $l <j$, $G \subset F_l$ und
\begin{math}
    F_l \cap F_i = F_j \setminus w
\end{math}
für $l < j$, also $w \in R_j$.
Aber $R_j \subset G \subset F_l \cap F_j = F_j \setminus w$, ein Widerspruch.

Also neue Seiten sind genau die in
\begin{math}
    I_j := \Set{G & R_j \subset G \subset F_j}.
\end{math}
Falls $|R_j| = i$, wieviele $G$ gibt es?
Es gibt genau $\binom{d-i}{k-i}$ $k$-Tupel ($(k-1)$-Seiten) zwischen $R_j$ und $F_j$.

Die Anzahl der $k$-Tupel ist
\begin{math}
    f_{k-1} &= \sum_{j=1}^s \# G \\
    &= \sum_{j=1}^s \binom{d- |R_j|}{k-|R_j|} \\
    &= \sum_{i=0}^k \binom{d-i}{k-i} h_i,
\end{math}
wobei $h_i$ die Anzahl der $R_j$ ist mit $|R_j| = i$.
Dies definiert den $h$-Vektor
\begin{math}
    h = (\underbrace{h_0}_{=1}, h_1, \dotsc, h_d).
\end{math}
Der $f$-Vektor
\begin{math}
    f = (\underbrace{f_{-1}}_{=1},f_0, \dotsc, f_{d-1})
\end{math}
ist gegeben durch die Anzahl $f_i$ der $i$-Seiten.
Es gilt
\begin{math}
    f_{k-1} = \sum_{i=0}^k \binom{d-i}{k-i} h_i.
\end{math}

\begin{lem}
    Der $h$-Vektor bestimmt den $f$-Vektor eindeutig und umgekehrt.
    \begin{proof}
        Definiere das $f$-Polynom und das $h$-Polynom durch
        \begin{math}
            f(x) = \sum_{i=0}^d f_{i-1} x^{d-i}
            = \sum_{i=0}^d \sum_{j=0}^i \binom{d-j}{i-j} h_j x^{d-i}
            = \sum_{j=0}^d h_j \underbrace{\sum_{i=j}^d \binom{d-j}{i-j} x^{d-i}}_{=(x+1)^{d-j}}
            = h(x+1)
        \end{math}
        mit
        \begin{math}
            h(x) = \sum_{j=0}^d h_j x^{d-j}.
        \end{math}
        Daraus folgt, dass
        \begin{math}
            h(x) = f(x-1)
            &= \sum_{i=0}^d f_{i-1} (x-1)^{d-i} \\
            &= \sum_{i=0}^d f_{i-1} \sum_{j=0}^{d-i} \binom{d-i}{j} (-1)^j x^{d-i-j} \\
            &= \sum_{k=0} h_k x^{d-k}
        \end{math}
        Koeffizientenvergleich liefert
        \begin{math}
            h_k = \sum_{i=0}^k (-1)^{k-i} \binom{d-i}{k-i} f_{i-1}.
        \end{math}
    \end{proof}
    \begin{note}
        Insbesondere ist der $h$-Vektor unabhängig von der Schälung.
    \end{note}
\end{lem}

\begin{kor}
    Mit $F_1, \dotsc, F_s$ ist auch $F_s, \dotsc, F_1$ eine Schälung.
    Folglich ist $h_i = h_{d-i}$ für alle $i$, der $h$-Vektor ist symmetrisch.
\end{kor}


\section[Dehn-Sommerville-Gleichungen]{Die Dehn-Sommerville-Gleichungen für den $f$-Vektor}

\paragraph{$d$ gerade}

Sei $d$ gerade, dann gilt die Euler-Gleichung
\begin{math}
    f_0 - f_1 + f_2 - f_3 + f_4 - f_5 + f_6 \dotsb - f_{d-1} &= \chi(S^{d-1}) = 0
\end{math}
Für die jede $i$-te Kante gilt die Gleichung
\begin{math}
    f_0^{(i)} - f_{1}^{(i)} + f_2^{(i) \dotsb - f_{d-3}^{(i)}} &= \chi(S^{d-3}) = 0
\end{math}
Summieren ergibt
\begin{math}
    \binom{3}{1} f_2 - \binom{4}{2} f_3 + \binom{5}{3} f_4 \dotsb - \binom{d}{d-2}f_{d-1} = 0
\end{math}
Für den Link des $i$-ten Tetraeders gilt
\begin{math}
    f_0^{(i)} - f_{1}^{(i)} + f_2^{(i) \dotsb - f_{d-5}^{(i)}} &= \chi(S^{d-5}) = 0
\end{math}
Summieren:
\begin{math}
    \binom{5}{1} f_2 - \binom{6}{2} f_3 + \binom{5}{3} f_4 \dotsb - \binom{d}{d-4} f_{d-1} = 0
\end{math}
Bis hin zu
\begin{math}
    f_0^{(i)} - f_1^{(i)} = 0
\end{math}
und
\begin{math}
    \binom{d-1}{1} f_{d-2} - \binom{d}{2} f_{d-1} = 0.
\end{math}
Alle Gleichungen entstehen durch die Euler-Gleichungen für die ungerad-dimensionalen Links.

Dies liefert $\frac{d}{2}$ Gleichungen für die $d$ Variablen $f_0, \dotsc, f_{d-1}$, bzw. für die $d$ Variablen $h_1, \dotsc, h_d$.


\paragraph{$d$ ungerade}

Es gilt
\begin{math}
    f_0 - f_1 + f_2 \dotsb + f_{d-1} = \chi(\S^{d-1}) = 2.
\end{math}
Für den Link einer Ecke:
\begin{math}
    f_0^{(i)} - f_1^{(i)} + f_2^{(i)} \dotsb f_{d-2}^{(i)} &= 0
    2f_1^{(i)} - \binom{3}{2}f_2^{(i)} + \binom{4}{3} f_3^{(i)} \dotsb - \binom{d}{d-1} f_{d-1}^{(i)} &= 0
\end{math}
für den Link eines Dreiecks:
\begin{math}
    f_0^{(i)} - f_1^{(i)} + f_2^{(i)} \dotsb f_{d-4}^{(i)} &= 0
    \binom{4}{1}f_2^{(i)} - \binom{5}{2} f_4^{(i)} \dotsb -\binom{d}{d-3} f_{d-1}^{(i)} &= 0
\end{math}
bis zu
\begin{math}
    \binom{d-1}{1} f_{d-2} - \binom{d}{2} f_{d-1} = 0
\end{math}
Wir erhalten $\frac{d+1}{2}$ Gleichungen für $d$ Variablen $f_0, \dotsc, f_{d-1}$, bzw. für $h_1, \dotsc, h_d$.

\begin{kor}
    Die Dehn-Sommerville-Gleichungen für den $f$-Vektor sind äquivalent zu $h_i = h_{d-i}$ für alle $i$ für den $h$-Vektor.
\end{kor}





