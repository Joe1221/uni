\chapter{Das Schlegel-Diagramm}

\Timestamp{2015-05-14}

Das \emphdef{Schlegel-Diagramm} ist die Projektion des Randkomplex eines konvexen $d$-Polytops (ohne einen Punkt) in den $\R^{d-1}$.

\begin{df}
    Sei $P \subset \R^d$ ein $d$-Polytop mit einer Facette $F \subset \Set{x & \<a,x\> = c}$, sodass $P \subset \Set{x & \<a,x\> \le c}$.
    Es sei $y_F \in \R^d$ ein Punkt mit $\<a, y_F\> = c + \eps$ für ein $\eps > 0$.
    Genauer wählen wir $y_F$ so , dass die affine Hülle von $F$ das Polytop $P$ und $y_F$ trennt (und nicht von anderen Facetten).
    Etwa: $y_F$ als Schwerpunkt von $F + \eps a$ für eine äußere Normale $a$.
    Dann definieren wir eine Projektionsabbildung
    \begin{math}
        \pr(x)
        &:= y_F + \frac{c - \<a,y_F\>}{\<a,x\> - \<a,y_F\>} (x-y_F) \\
        &= y_F - \frac{\eps}{\<a,x\> - (c + \eps)} (x - y_F)
    \end{math}
\end{df}

\begin{note}
    Eigenschaften der Projektion:
    \begin{itemize}
        \item
            $\pr(x) = x$ für $x \in H = \Set{x & \<a,x\> = c}$
        \item
            $\<a, \pr(x)\> = c$ für alle $x$, d.h. $\pr(x) \in H$
    \end{itemize}
    Also $\pr(\Boundary P) \subset H \homeomorphic \R^{d-1}$, $\pr(\Boundary F) = \Boundary F$.
    \begin{proof}
        \begin{itemize}
            \item
            \item
                Es gilt
                \begin{math}
                    \<a, \pr(x)\>
                    = \<a, y_F\> + \frac{c - \<a, y_F\>}{\<a,x\> - \<a,y_F\>} (\<a,x\> - \<a,y_F\>)
                    = c.
                \end{math}
        \end{itemize}
    \end{proof}
\end{note}

\begin{df}
    Das \emphdef{Schlegel-Diagramm} $\scr D(P,F)$ ist definiert als das Bild aller echten Seiten von $P$, die von $F$ verschieden sind.
    Die Facette $F$ erscheint dann gewissermaßen als das Äußere von $\scr D(P,F)$ (mit einem Punkt im Unendlichem).
\end{df}

\begin{kor}
    Das Schlegel-Diagramm $\scr D(P,F)$ ist eine polytopale Unterteilung von $F$, deren Seitenverband kombinatorisch isomorph ist zum Seitenverband von $\Boundary P \setminus F$.
    \begin{proof}
        $\pr: \Boundary P \setminus \mathring F \to F \subset H$ ist eine Bijektion.
        Jede $k$-dimensionale Seite von $\Boundary P \setminus F$ erscheint als $k$-dimensionales Polytop in $F$ (bzw. in $H$).
        \begin{math}
            \pr(F_1 \cap F_2) &= \pr(F_1) \cap \pr(F_2) \\
            \pr(F_1 \cup F_2) &= \pr(F_1) \cup \pr(F_2)
        \end{math}
        Denn $\pr$ ist eine projektive Abbildung, überführt also $k$-dimensionale affine Unterräume in $l$-dimensionale affine Unterräume.
        projektiv:
        \begin{math}
            \R^d \ni x \mapsto (1:x) &\mapsto \Big(\<a,x\> - (c + \eps) : (\<a,x\> + (c + \eps)) y_F - \eps(x-y_F)\Big) \\
            &\quad= \Big(1 : y_F - \frac{\eps(x-y_F)}{\<a,x\> - (c + \eps)}\Big) \\
            &\quad= (1 : \pr(x))
            \mapsto \pr(x) \R^d.
        \end{math}
    \end{proof}
\end{kor}

\begin{ex}
    \begin{itemize}
        \item
            Tetraeder
        \item
            Oktaeder
        \item
            4-Simplex
        \item
            4-Würfel
        \item
            4-Kreuzpolytop
        \item
            Prisma
        \item
            $\Delta^2 \times \Delta^2 \in \R^4$ (Sanduhr?)
        \item
            zyklisches $3$-Polytop $C(3,8)$
            \begin{math}
                &123, 134, 145, 156, 167, 178, \\
                &812, 823, 834, 845, 856, 867
            \end{math}
            Wähle $F = 812$.
        \item
            zyklisches $4$-Polytop $C(4,7)$
            \begin{math}
                &1234, 2345, 3456, 4567, 5671, 6712, 7123, \\
                &1245, 2356, 3467, 4571, 5612, 6723, 7134.
            \end{math}
    \end{itemize}
\end{ex}

\begin{ex}
    Liegt das polyedrische Möbiusband im $\R^3$ mit 6 Ecken, 2 Vierecken, 2 Dreiecken in einem Schlegeldiagramm als Unterkomplex?
    Nein: Es gibt keine Facette $F$ dafür, denn alle $6$ Ecken müssten in $F$ liegen (betrachte beide Vierecke).
    Ein Dreieck wäre im $4$-Polytop eine Diagonale.
\end{ex}

\begin{kor}
    Es gibt polyedrische Diagramme, die keine Schlegel-Diagramme eines Polytops sind.
\end{kor}
