\chapter{Supernachbarschaftliche Mannigfaltigkeiten}

\Timestamp{2015-07-01}


Ränder von $3$-Polytopen sind maximal 1-nachbarschaftlich.
Ränder von $2$-Polytopen sind maximal 2-nachbarschaftlich.

Für eine triangulierte, geschlossene $(2k+1)$-dimensionale Mannigfaltigkeit $M$ sind die Dehn-Sommerville-Gleichungen genau die selben, wie für $(2k+2)$-Polytope (bzw. deren Rändern), weil $\chi(M) = 0 = \chi(S^{2k+1})$.

Anders bei $2k$-Mannigfaltigkeiten:
Hier ist $\sum_{i=0}^{2k} (-1)^i f_i = \chi(M)$, alle anderen wie gehabt.
Dies führt zum $h$-Vektor (setze $d := 2k + 1$, tue so, also ob $M$ Rand eines $d$-Polytops wäre)
\begin{math}
    h_j - h_{2k+1-j} = (-1)^{2k+1-j} \binom{2k+1}{j} (\chi(M) - 2)
\end{math}
Dies ist gleich $0$, wenn $M$ eine $2k$-Sphäre ist.
$h_0 = 1$, $h_d = \chi(M) - 1$, also $h_0 - h_d = 2 - \chi(M)$.

\begin{seg}{Fall $k = 1$}
    2-Mannigfaltigkeiten ohne Rand, bzw. geschlossene 2-Pseudomannigfaltigkeiten mit isolierten Singularitäten.
    geschlossen: Jede Kante liegt in genau 2 Dreiecken.

    Dehn-Sommerville mit $f_0 = n$:
    \begin{math}
        n - f_1 + f_2 &= \chi \\
        2f_1 - 3f_2 &= 0 \\
        \leadsto n - \frac{f_1}{3} &= \chi
    \end{math}
    also ist
    \begin{math}
        f_1 = 3(n-\chi) \le \binom{n}{2} = \frac{1}{2}n(n-1)
    \end{math}
    Umgeformt
    \begin{math}
        3(2-\chi) \le \binom{n-3}{2}
    \end{math}
    mit Gleichheit genau dann, wenn $M$ 2-nachbarschaftlich ist.
    Dies ist äquivalent zur Heawoodsche Ungleichung
    \begin{math}
        n \ge \frac{1}{2}(7 + \sqrt{49 - 24 \chi})
    \end{math}
    \begin{note}
        Wenn der vollständige Graph $K_n$ mit $n$ Ecken und $\binom{n}{2}$ Kanten in eine Fläche $M$ einbettbar ist, dann gilt
        \begin{math}
            n \le \frac{1}{2} (7 + \sqrt{49 - 24 \chi(M)}.
        \end{math}
        Beweis: Übung!
        Für $\chi = 2$ folgt, dass $K_5$ nicht einbettbar ist.

        Beispiel: Kleinsche Flasche.
    \end{note}

    Wann haben wir Gleichheit für gegebenes $n$?
    Gleichheit gilt genau dann, wenn $3 \divs \binom{n-3}{2} = \frac{1}{2}(n-3)(n-4)$, also $n \equiv 0$ oder $n \equiv 1 \bmod 3$ ($n \equiv 2 \bmod 3$ ist nicht möglich).
    \begin{table}[ht]
        \begin{tabular}{ccc}
            3 & 2 & (Dieder) \\
            4 & 2 & Rand des Tetraeders \\
            6 & 1 & $\R P^2$ mit 6 Ecken \\
            7 & 0 & Torus mit 7 Ecken \\
            9 & -3 & $\R P^2 \# T^2 \#T^2$ (Bsp. mit pinchpoints) \\
            10 & -5 & analog \\
            12 & -10 & $T^2 \# T^2 \# T^2 \# T^2 \# T^2 \# T^2$ und nichtori. und pinchpoints \\
            13 & -13 & nicht orientierbar oder pinchpoints \\
            19 & -38
        \end{tabular}
        \caption{Fälle mit Gleichheit}
    \end{table}
    Alle Fälle treten auf, außer kleinsche Flasche mit $n = 7$ (Ringel: Map Color Theorem, 1974).

    Wann steht Gleichheit für gegebenes $n$ und eine orientierbare Fläche?
    $\chi$ muss gerade sein, also $6 \divs \binom{n-3}{2} = \frac{1}{2} (n-3)(n-4)$, also 
    \begin{math}
        n \equiv 0, 3, 4, 7 \bmod 12.
    \end{math}
\end{seg}

\begin{seg}{Fall $k = 2$}
    Dehn-Sommerville:
    \begin{math}
        n - f_1 + f_2 - f_3 + f_4 &= \chi \\
        2 f_1 - 3f_2 + 4f_3 - 5f_4 &= 0 \\
        2 f_3 - 5f_4 &= 0
    \end{math}
    2-nachbarschaftlich (geht für 5-Polytope), 3-nachbarschaftlich: $f_1 = \binom{n}{2}$, $f_2 = \binom{n}{3}$
    \begin{math}
        n - f_1 + f_2 - \frac{3}{5} f_3 &= \chi \\
        2 f_1 - 3f_2 + 2f_3 &= 0
        \leadsto 10 n - 4f_1 + f_2 = 10 \chi
    \end{math}
    Es folgt also
    \begin{math}
        \binom{n}{3} - 4 \binom{n}{2} + 10 n = 10 \chi
    \end{math}
    nach Umformung
    \begin{math}
        \binom{n-4}{3} = 10 (\chi - 2).
    \end{math}
\end{seg}



