\chapter{Supernachbarschaftliche Mannigfaltigkeiten}

\Timestamp{2015-07-01}


Ränder von $3$-Polytopen sind maximal 1-nachbarschaftlich.
Ränder von $4$-Polytopen sind maximal 2-nachbarschaftlich.

Für eine triangulierte, geschlossene $(2k+1)$-dimensionale Mannigfaltigkeit $M$ sind die Dehn-Sommerville-Gleichungen genau die selben, wie für $(2k+2)$-Polytope (bzw. deren Rändern), weil $\chi(M) = 0 = \chi(S^{2k+1})$.

Anders bei $2k$-Mannigfaltigkeiten:
Hier ist $\sum_{i=0}^{2k} (-1)^i f_i = \chi(M)$, alle anderen wie gehabt.
Dies führt zum $h$-Vektor (setze $d := 2k + 1$, tue so, also ob $M$ Rand eines $d$-Polytops wäre)
\begin{math}
    h_j - h_{2k+1-j} = (-1)^{2k+1-j} \binom{2k+1}{j} (\chi(M) - 2)
\end{math}
Dies ist gleich $0$, wenn $M$ eine $2k$-Sphäre ist.
$h_0 = 1$, $h_d = \chi(M) - 1$, also $h_0 - h_d = 2 - \chi(M)$.

\begin{seg}{Fall $k = 1$}
    2-Mannigfaltigkeiten ohne Rand, bzw. geschlossene 2-Pseudomannigfaltigkeiten mit isolierten Singularitäten.
    geschlossen: Jede Kante liegt in genau 2 Dreiecken.

    Dehn-Sommerville mit $f_0 = n$:
    \begin{math}
        n - f_1 + f_2 &= \chi \\
        2f_1 - 3f_2 &= 0 \\
        \leadsto n - \frac{f_1}{3} &= \chi
    \end{math}
    also ist
    \begin{math}
        f_1 = 3(n-\chi) \le \binom{n}{2} = \frac{1}{2}n(n-1)
    \end{math}
    Umgeformt
    \begin{math}
        3(2-\chi) \le \binom{n-3}{2}
    \end{math}
    mit Gleichheit genau dann, wenn $M$ 2-nachbarschaftlich ist.
    Dies ist äquivalent zur Heawoodsche Ungleichung
    \begin{math}
        n \ge \frac{1}{2}(7 + \sqrt{49 - 24 \chi})
    \end{math}
    \begin{note}
        Wenn der vollständige Graph $K_n$ mit $n$ Ecken und $\binom{n}{2}$ Kanten in eine Fläche $M$ einbettbar ist, dann gilt
        \begin{math}
            n \le \frac{1}{2} (7 + \sqrt{49 - 24 \chi(M)}.
        \end{math}
        Beweis: Übung!
        Für $\chi = 2$ folgt, dass $K_5$ nicht einbettbar ist.

        Beispiel: Kleinsche Flasche.
    \end{note}

    Wann haben wir Gleichheit für gegebenes $n$?
    Gleichheit gilt genau dann, wenn $3 \divs \binom{n-3}{2} = \frac{1}{2}(n-3)(n-4)$, also $n \equiv 0$ oder $n \equiv 1 \bmod 3$ ($n \equiv 2 \bmod 3$ ist nicht möglich).
    \begin{table}[ht]
        \centering
        \begin{tabular}{ccl}
            $n$ & $\chi$ & realisiert durch \\ \hline
            3 & 2 & (Dieder) \\
            4 & 2 & Rand des Tetraeders \\
            6 & 1 & $\R P^2$ mit 6 Ecken \\
            7 & 0 & Torus mit 7 Ecken \\
            9 & -3 & $\R P^2 \# T^2 \#T^2$ (Bsp. mit pinchpoints) \\
            10 & -5 & analog \\
            12 & -10 & $T^2 \# T^2 \# T^2 \# T^2 \# T^2 \# T^2$ und nichtori. und pinchpoints \\
            13 & -13 & nicht orientierbar oder pinchpoints \\
            19 & -38
        \end{tabular}
        \caption{Fälle mit Gleichheit}
    \end{table}
    Alle Fälle treten auf, außer kleinsche Flasche mit $n = 7$ (Ringel: Map Color Theorem, 1974).

    Wann steht Gleichheit für gegebenes $n$ und eine orientierbare Fläche?
    $\chi$ muss gerade sein, also $6 \divs \binom{n-3}{2} = \frac{1}{2} (n-3)(n-4)$, also 
    \begin{math}
        n \equiv 0, 3, 4, 7 \bmod 12.
    \end{math}
    $n \equiv 7 \bmod 12$ erlaubt eine einfache Lösung (siehe Übung).
\end{seg}

Ungerad-dimensionale Mannigfaltigkeiten haben Euler-Charakteristik $\chi = 0$.

\begin{seg}{Fall $k = 2$}
    $n \ge 6$ (5-Simplex ist minimaler Fall).
    Dehn-Sommerville:
    \begin{math}
        n - f_1 + f_2 - f_3 + f_4 &= \chi \\
        2 f_1 - 3f_2 + 4f_3 - 5f_4 &= 0 \\
        2 f_3 - 5f_4 &= 0
    \end{math}
    2-nachbarschaftlich (funktioniert noch für 5-Polytope) und zusäztlich 3-nachbarschaftlich: $f_1 = \binom{n}{2}$, $f_2 = \binom{n}{3}$
    \begin{math}
        n - f_1 + f_2 - \frac{3}{5} f_3 &= \chi \\
        2 f_1 - 3f_2 + 2f_3 &= 0 \\
        \leadsto 10 n - 4f_1 + f_2 &= 10 \chi
    \end{math}
    Es folgt also
    \begin{math}
        \binom{n}{3} - 4 \binom{n}{2} + 10 n = 10 \chi
    \end{math}
    nach Umformung
    \begin{math}
        \binom{n-4}{3} = 10 (\chi - 2).
    \end{math}
\Timestamp{2015-07-08}
    3-nachbarschaftliche Mannigfaltigkeiten sind einfach zusammenhängend (jede Kurve kann in jede andere deformiert werden).

    Dies ist genau dann ganzzahlig möglich, wenn $(n-4)(n-5)(n-6) \equiv 20 a$ mit $a \in \Z$, also
    \begin{math}
        n \equiv 0, 1, 4, 5, 6, 9, 10, 14, 16 \bmod 20
    \end{math}
    Fälle von $(n,\chi)$:
    \begin{table}
        \centering
        \begin{tabular}{ccl}
            $n$ & $\chi$ & Beispiele \\ \hline
            (5) & (2) & (Dieder) \\
            6 & 2 & $\Boundary \Delta^5 \isomorphic S^4$ \\
            9 & 3 & $\CP^2$, siehe Übung \\
            10 & 4 & Kandidat $\S^2 \times \S^2$, existiert nicht \\
            14 & 14 & ?, existiert zumindest nicht mit zyklischer Symmetrie \\
            16 & 24 & K3-Fläche ($x^4 + y^4 + z^4 + w^4 = 0$ in $\CP^3$), Gruppe der Ordnung 240, Ecken-Link ist $\S^3$ mit 15 Ecken und 2-nachbarschaftlich und zyklischer Symmetrie
        \end{tabular}
        \caption{Auftretende Fälle}
    \end{table}
\end{seg}

\section{Die Ungleichung vom Heawood-Typ}

\begin{st}[Kühnel, 1985]
    Sei $M^4$ eine triangulierte $4$-Mannigfaltigkeit mit $n$ Ecken (also jeder Ecken-Link ist eine triangulierte 3-Sphäre).
    Dann gilt
    \begin{math}
        \binom{n-4}{3} = 10 (\chi(M) - 2)
    \end{math}
    mit Gleichheit genau dann, wenn $f_2 = \binom{n}{3}$, d.h. $M^4$ ist 3-nachbarschaftlich (also $M^4$ einfach zusammenhängend).
    \begin{proof}
        Dehn-Sommerville ergibt:
        \begin{math}
            10n - 4f_1 + f_2 &= 10 \chi
        \end{math}
        Direktes Abschätzen von $f_1$, $f_2$ funktioniert nicht, wegen alternierender Vorzeichen.
        Es ist bekannt, dass für $n \le 8$ folgt $M \isomorphic \S^4$ und $\chi - 2 = 0$.
        Sei also $n \ge 9$.
        (Falls $f_1 = \binom{n}{2}$, dann ist $10 \chi \le 10n - 4\binom{n}{2} + \binom{n}{3}$).
        Setze $\_f_k := \binom{n}{3} - f_k$, zu zeigen: $\_f_2 \ge 4 \_f_1$.
        Link der Ecke $i$: $f_0(i), f_1(i), f_2(i), \dotsc$ und $f_0(i) \ge 5$.
        Wir bezeichnen wieder mit $\_f_0(i) := n - 1 - f_0(i)$ die Anzahl der Nicht-Ecken, und mit $\_f_1(i) = \binom{n-1}{2} - f_1(i)$ die Anzahl der Nicht-Kanten.

        Wegen $f_0(i) + \_f_0(i) = n-1$ und $\_f_0(i) \le n - 6$
        \begin{math}
            \_f_1(i)
            &\ge f_0(i) \_f_0(i) + \binom{\_f_0(i)}{2} \\
            &= \_f_0(i) \Big( f_0(i) + \frac{1}{2}(\_f_0(i) - 1) \Big) \\
            &= \_f_0(i) \Big( n-1 - \frac{1}{2}(\_f_0(i) + 1) \Big) \\
            &\ge \_f_0(i) \Big( n-1 - \frac{1}{2}(n-5) \Big) \\
            &= \_f_0(i) \big( \frac{n}{2} + \frac{3}{2} \big) \\
            &\ge 6 \_f_0(i)
        \end{math}
        da $n \ge 9$.
        Summiert ergibt sich
        \begin{math}
            \underbrace{\sum_{i} \_f_1(i)}_{3\_f_2}
            \ge 6 \underbrace{\sum_i \_f_0(i)}_{3\_f_1},
        \end{math}
        also $\_f_2 \ge 4 \_f_1$.

        Es gilt mit $4 \_f_1 \le \_f_2$
        \begin{math}
            10(\chi - 2)
            &= 10n - 4 f_1 + f_2 - 20 \\
            &= 10n - 4 (\binom{n}{2} - \_f_1) + \binom{n}{3} - \_f_2 - 20 \\
            &\le 10n - 4 \binom{n}{2} + \binom{n}{3} - 20 \\
            &= \frac{1}{6}(n^3 - 15n^2 + 74n - 120) \\
            &= \binom{n-4}{3}
        \end{math}
        mit Gleichheit genau dann, wenn $\_f_2 = 4 \_f_1$.
        Zeige, dass im Fall der Gleichheit $\_f_2 = \_f_1 = 0$.

        Falls $n \ge 10$ muss $\_f_0(i) = 0$ (betrachte die letzte Ungleichung bei $\_f_1(i) \le 6 \_f_0(i)$) für alle $i$ und somit $\_f_1 = \_f_2 = 0$.
        Sei nun $n = 9$, hier wäre auch $\_f_0(i) = n-6 = 3$ möglich, also $f_0(i) = 5$ und
        \begin{math}
            2 f_1 = \sum_{i} f_0(i) = 5n = 45,
        \end{math}
        ein Widerspruch und $\_f_0(i) = 0$ wie oben.
    \end{proof}
\end{st}

Allgemein:

\begin{conj}[Kühnel]
    Triangulierung von $M^{2k}$ mit $n$ Ecken
    \begin{math}
        \binom{n-k-2}{k+1}
        \ge (-1)^k \binom{2k+1}{k+1} (\chi(M) - 2)
    \end{math}
    mit Gleichheit genau dann, wenn $f_k = \binom{n}{k+1}$, d.h. wenn $M^{2k}$ $(k+1)$-nachbarschaftlich ist.
    \begin{proof}
        Durch Isabella Novik und Ed Swartz, 2012 mit kommutativer Algebra.
    \end{proof}
\end{conj}

Siehe auch Springer Lecture Notess 1612 (1995).

