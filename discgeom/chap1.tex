\chapter{Erstes Kapitel?}

\Timestamp{2015-04-15}

Wir betrachten die Geomtrie von Teilmengen des $\R^n$.
Darunter unterscheiden wir zwischen differenzierbar (siehe Differentialgeometrie) und diskret (polyedrisch).

Das Analogon zur Gauß-Krümmung in der Differentialgeometrie bildet eine sogenannte diskrete Krümmung
\begin{math}
    2\pi - \sum \texp{Innenwinkel}.
\end{math}

Wir betrachten zunächst konvexe Polytope.


% 1.
\section{Konvexe Polytope}

\begin{df}[konvexe Hülle]
    Für $x,y \in \R^n$ definieren wir die \emphdef{Verbindungsstrecke} als
    \begin{math}
        \_{xy} := \Set{\lambda x + (1-\lambda)y & 0 \le \lambda \le 1}.
    \end{math}
    Eine Menge $M \subset \R^n$ heißt \emphdef{konvex}, wenn zu je zwei $x,y \in M$ gilt: $\_{xy} \subset M$.

    Analog ist die \emphdef{konvexe Hülle} von Punkten $x_0, \dotsc, x_k$ definiert:
    \begin{math}
        \CHull{x_0, \dotsc, x_k}
        &:= \Set{x & x = \lambda_0 x_0 + \dotsb + \lambda_k x_k, \lambda_i \in [0,1], \sum_i \lambda_i = 1}.
    \end{math}
    Die \emphdef{konvexe Hülle} einer Menge $M$ sei
    \begin{math}
        \CHull{M}
        &:= \bigcap_k \Set{K \subset \R^n & M \subset K, K \text{ konvex}}.
    \end{math}
    Die \emphdef{affine Hülle} von $M$ ist der kleinste affine Unterraum in $\R^n$, der $M$ enthält.
    \begin{note}
        Es lässt sich schreiben
        \begin{math}
            \lambda_0 x_0 + \dotsb + \lambda_k x_k
            = (1-\lambda_k) \Big( \frac{\lambda_0}{1-\lambda_k} x_0  + \dotsb + \frac{\lambda_{k-1}}{1-\lambda_k} x_{k-1} \Big) + \lambda_k.
        \end{math}
    \end{note}
\end{df}


\paragraph{Standard-Polytope}
Im $\R^2$ wäre ein Beispiel Polygone.
IM $\R^n$ sind $k$-Simplizes ($n \ge k$) als konvexe Hülle von $k + 1$ Punkten in allgemeiner Lage (d.h. affin linear unabhängig) Standard-Polytope.
Spezieller ist der \emphdef{Standard-Simplex} in $\R^k$ für $k \in \N$ gegeben als konvexe Hülle der Punkte
\begin{math}
    x_0 = (0, 0, \dotsc, 0),
    x_1 = (1, 0, \dotsc, 0), \dotsc,
    x_k = (0, 0, \dotsc, 1) \in \R^k.
\end{math}
Der \emphdef[Würfel]{$k$-Würfel} ist die konvexe Hülle der $2^k$ Punkte $(\pm 1, \dotsc, \pm 1) \in \R^k$.

Das \emphdef[Oktaeder]{$k$-dimensionale Oktaeder} ist die konvexe Hülle, der $2k$ Punkte $(0, \dotsc, \pm 1, \dotsc, 0) \in \R^k$ ($\pm 1$ in jeder einzelnen Koordinate).

Das \emphdef{Ikosaeder} ist die konvexe Hülle von $(0,\pm \tau, 1), (\pm 1, 0, \pm \tau), (\pm \tau, \pm 1, 0)$, wobei
\begin{math}
    \tau^2 = \tau + 1,
\end{math}
d.h. $\tau$ ist der \emphdef{goldene Schnitt}, $\tau = \frac{1}{2}(\sqrt 5 + 1) \approx 1.618$.
Wir ordnen dem Ikosaeder das Symbol $\{3,5\}$ zu (5 Dreiecke um jede Ecke), mehr dazu später.

Das \emphdef{Dodekaeder} ist die konvexe Hülle von
\begin{math}
    (0, \pi \tau^{-1}, \pm \tau),
    (\pm \tau, 0, \pm \tau^{-1}),
    (\pm \tau^{-1}, \pm \tau, 0),
    (\pm 1, \pm 1, \pm 1).
\end{math}

\paragraph{Kartesische Produkte}
Seien $K \subset \R^k, L \subset \R^l$ konvexe Polytope, dann ist
\begin{math}
    K \times L \subset \R^k \times \R^l \isomorphic \R^{k+l}
\end{math}
wieder eine konvexes Polytop.
Beispielsweise ist ein $k$-Würfel das Produkt aus einem $(k-1)$-Würfel und dem Intervall $[-1,1]$.

\paragraph{Prisma über $K$}
Für ein konvexes Polytop $K$ ist $P = K \times [0,1]$ das Prisma über $K$.

\paragraph{Pyramide über $K$}
Die Pyramide über $K$ ist $\CHull{K \cup \Set{x_0}}$, wobei $x_0$ außerhalb der affinen Hülle von $K$ liegt.
Beispielsweise ist das $k$-Oktaeder die Doppelpyramide über das $(k-1)$-Oktaeder.

\paragraph{Minkowski-Summe}
Wir bezeichnen
\begin{math}
    K + L := \Set{x + y & x\in K, y \in L}
\end{math}
als \emphdef{Minkowski-Summe} der Polyeder $K$ und $L$.

\begin{df}[$H$-Polyeder, $H$-Polytop]
    Ein \emphdef[H-Polyeder]{$H$-Polyeder} im $\R^n$ ist der Durchschnitt endlich vieler abgeschlossener Halbräume des $\R^n$.

    Ein \emphdef[H-Polytop]{$H$-Polytop} ist ein beschränktes $H$-Polyeder (d.h. es enthält keinen Strahl $\Set{x+ty & t \ge 0}$).
\end{df}

\begin{df}[$V$-Polytop]
    Ein \emphdef[V-Polytop]{$V$-Polytop} ist die konvexe Hülle endlich vieler Punkte.
\end{df}

Die Polytope im folgenden seien nicht degeneriert in dem Sinne, dass sie in einer Hyperebene enthalten sind.

\begin{df}
    Ein \emphdef{exponierter Punkt} oder \emphdef{Ecke} einer konvexen Menge $M$ ist ein Punkt $p \in M$, sodass $M \cap H = \Set{p}$ für eine gewisse Hyperebene $H$.

    Der Punkt $p$ heißt \emphdef{extremer Punkt}, wenn aus $p = \lambda x + (1-\lambda)y$ mit $x,y \in P$ stets folgt: $p = x$ oder $p = y$.
    \begin{note}
        Aus exponiert folgt extrem.
        Die Umkehrung gilt nicht (Rechteck mit zwei Halbkreisen).
    \end{note}
\end{df}

\begin{df}[Eckenfigur]
    Sei $P \subset \R^n$ ein $V$-Polytop, nicht in einer Hyperebene enthalten.
    $v_0$ sei eine Ecke von $P$, d.h. ein exponierter Punkt:
    \begin{math}
        \Set{v_0} =
        P \cap \Set{x & \<c,x\> = c_0}
    \end{math}
    für $P \subset \Set{x & \<c,x\> \le c_0}$.
    Die Hyperebene $H = \Set{x & \<c, x\> = c_0}$ wird auch \emphdef{Stützhyperebene} genannt.

    Dann ist $P / v_0 = P \cap \Set{x & \<c, x\> = c_0 - \epsilon}$ für hinreichend kleines $\epsilon > 0$ die \emphdef{Eckenfigur} von $P$ in $v_0$.

\end{df}







