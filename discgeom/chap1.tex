\chapter{Erstes Kapitel?}

\Timestamp{2015-04-15}

Wir betrachten die Geomtrie von Teilmengen des $\R^n$.
Man unterscheidet zwischen differenzierbar (siehe Differentialgeometrie) und diskret (polyedrisch).

Das Analogon zur Gauß-Krümmung in der Differentialgeometrie bildet eine sogenannte diskrete Krümmung
\begin{math}
    2\pi - \sum \texp{Innenwinkel}.
\end{math}

Wir betrachten zunächst konvexe Polytope.


% 1.
\section{Konvexe Polytope}

\begin{df}[konvexe Hülle]
    Für $x,y \in \R^n$ definieren wir die \emphdef{Verbindungsstrecke} als
    \begin{math}
        \_{xy} := \Set{\lambda x + (1-\lambda)y & 0 \le \lambda \le 1}.
    \end{math}
    Eine Menge $M \subset \R^n$ heißt \emphdef{konvex}, wenn zu je zwei $x,y \in M$ gilt: $\_{xy} \subset M$.

    Analog ist die \emphdef{konvexe Hülle} von Punkten $x_0, \dotsc, x_k \in \R^n$ definiert:
    \begin{math}
        \CHull{x_0, \dotsc, x_k}
        &:= \Set{x \in \R^n & x = \lambda_0 x_0 + \dotsb + \lambda_k x_k, \lambda_i \in [0,1], \sum_i \lambda_i = 1}.
    \end{math}
    Die \emphdef{konvexe Hülle} einer Menge $M$ sei
    \begin{math}
        \CHull{M}
        &:= \bigcap_k \Set{K \subset \R^n & M \subset K, K \text{ konvex}}.
    \end{math}
    Die \emphdef{affine Hülle} von $M$ ist der kleinste affine Unterraum in $\R^n$, der $M$ enthält.
    \begin{note}
        Es gilt
        \begin{math}
            \lambda_0 x_0 + \dotsb + \lambda_k x_k
            = (1-\lambda_k) \Big( \frac{\lambda_0}{1-\lambda_k} x_0  + \dotsb + \frac{\lambda_{k-1}}{1-\lambda_k} x_{k-1} \Big) + \lambda_k x_k.
        \end{math}
        Damit lässt sich jeder Punkt aus $\CHull{x_0,\dotsc,x_k}$ durch mehrfache paarweise Konvexkombinationen darstellen.
    \end{note}
\end{df}


\paragraph{Standard-Polytope}

Im $\R^2$ sind Polygone ein Beispiel.
Im $\R^n$ sind $k$-Simplizes ($n \ge k$) als konvexe Hülle von $k + 1$ Punkten in allgemeiner Lage (d.h. affin linear unabhängig) Standard-Polytope.
Spezieller ist der \emphdef{Standard-Simplex} in $\R^k$ für $k \in \N$ gegeben als konvexe Hülle der Punkte
\begin{math}
    x_0 = (0, 0, \dotsc, 0),
    x_1 = (1, 0, \dotsc, 0), \dotsc,
    x_k = (0, 0, \dotsc, 1) \in \R^k.
\end{math}
Der \emphdef[Würfel]{$k$-Würfel} ist die konvexe Hülle der $2^k$ Punkte $(\pm 1, \dotsc, \pm 1) \in \R^k$.

Das \emphdef[Oktaeder]{$k$-dimensionale Oktaeder} ist die konvexe Hülle, der $2k$ Punkte $(0, \dotsc, \pm 1, \dotsc, 0) \in \R^k$ ($\pm 1$ in jeder einzelnen Koordinate).

Das \emphdef{Ikosaeder} ist die konvexe Hülle von $(0,\pm \tau, 1), (\pm 1, 0, \pm \tau), (\pm \tau, \pm 1, 0)$, wobei
\begin{math}
    \tau^2 = \tau + 1,
\end{math}
d.h. $\tau$ ist der \emphdef{goldene Schnitt}, $\tau = \frac{1}{2}(\sqrt 5 + 1) \approx 1.618$.
Wir ordnen dem Ikosaeder das Symbol $\{3,5\}$ zu (5 Dreiecke um jede Ecke), mehr dazu später.

Das \emphdef{Dodekaeder} ist die konvexe Hülle von
\begin{math}
    (0, \pi \tau^{-1}, \pm \tau),
    (\pm \tau, 0, \pm \tau^{-1}),
    (\pm \tau^{-1}, \pm \tau, 0),
    (\pm 1, \pm 1, \pm 1).
\end{math}

\paragraph{Kartesische Produkte}
Seien $K \subset \R^k, L \subset \R^l$ konvexe Polytope, dann ist
\begin{math}
    K \times L \subset \R^k \times \R^l \isomorphic \R^{k+l}
\end{math}
wieder eine konvexes Polytop.
Beispielsweise ist ein $k$-Würfel das Produkt aus einem $(k-1)$-Würfel und dem Intervall $[-1,1]$.

\paragraph{Prisma über $K$}
Für ein konvexes Polytop $K$ ist $P = K \times [0,1]$ das Prisma über $K$.

\paragraph{Pyramide über $K$}
Die Pyramide über $K$ ist $\CHull{K \cup \Set{x_0}}$, wobei $x_0$ außerhalb der affinen Hülle von $K$ liegt.
Beispielsweise ist das $k$-Oktaeder die Doppelpyramide über das $(k-1)$-Oktaeder.

\paragraph{Minkowski-Summe}
Wir bezeichnen
\begin{math}
    K + L := \Set{x + y & x\in K, y \in L}
\end{math}
als \emphdef{Minkowski-Summe} der Polyeder $K$ und $L$.

\begin{df}[$H$-Polyeder, $H$-Polytop]
    Ein \emphdef[H-Polyeder]{$H$-Polyeder} im $\R^n$ ist der Durchschnitt endlich vieler abgeschlossener Halbräume des $\R^n$.

    Ein \emphdef[H-Polytop]{$H$-Polytop} ist ein beschränktes $H$-Polyeder (d.h. es enthält keinen Strahl $\Set{x+ty & t \ge 0}$).
\end{df}

\begin{df}[$V$-Polytop]
    Ein \emphdef[V-Polytop]{$V$-Polytop} ist die konvexe Hülle endlich vieler Punkte.
\end{df}

\begin{note}
    \begin{itemize}
        \item
            Obige Definitionen beinhalten degenerierte Polytope, d.h. solche, die in einer Hyperebene enthalten sind (z.B. die konvexe Hülle zweier Punkte im $\R^2$ ist in einer Geraden enthalten).
            Oft geht man implizit von nicht degenerierten Polytopen aus (lässt sich durch Verringerung der Dimension des Raumes stets bewerkstelligen).
        \item
            Die Begriffe $V$-Polytop und $H$-Polytop sind äquivalent (siehe später), wir sprechen auch einfach von \emphdef[konvexes Polytop]{konvexen Polytopen}.
    \end{itemize}
\end{note}

\begin{df}
    Ein \emphdef{exponierter Punkt} oder eine \emphdef{Ecke} einer konvexen Menge $M$ ist ein Punkt $p \in M$, sodass $M \cap H = \Set{p}$ für eine gewisse Hyperebene $H$.

    Der Punkt $p$ heißt \emphdef{extremer Punkt}, wenn aus $p = \lambda x + (1-\lambda)y$ mit $x,y \in P$ stets folgt: $p = x$ oder $p = y$.
    \begin{note}
        Aus exponiert folgt extrem.
        Die Umkehrung gilt nicht (Rechteck mit zwei Halbkreisen).
    \end{note}
\end{df}

\begin{df}[Eckenfigur]
    Sei $P \subset \R^n$ ein konvexes Polytop.
    $v$ sei eine Ecke von $P$, d.h.
    \begin{math}
        \Set{v} =
        P \cap \Set{x & \<c,x\> = c_0}
    \end{math}
    für $P \subset \Set{x & \<c,x\> \le c_0}$.
    Die Hyperebene $H := \Set{x & \<c, x\> = c_0}$ wird auch \emphdef{Stützhyperebene} genannt.

    Dann ist für hinreichend kleines $\epsilon > 0$
    \begin{math}
        P / v := P \cap \Set{x & \<c, x\> = c_0 - \epsilon}
    \end{math}
    die \emphdef{Eckenfigur} von $P$ in $v$.
\end{df}

\Timestamp{2015-04-22}

\begin{df}
    Ein \emphdef[konvexes Polytop!konvexes d-Polytop]{konvexes $d$-Polytop} $P$ im $\R^n$ ist ein konvexes Polytop, sodass der durch $P$ aufgespannte affine Unterraum Dimension $d$ hat.
    Wir nennen $d$ auch \emphdef{Dimension} von $P$.
\end{df}

\begin{df}
    Sei $P$ ein konvexes $d$-Polytop.

    Eine Hyperebene $H := \Set{x & \<c, x\> = c_0} \subset \R^n$ für $c \in \R^n$, $c_0 \in \R$ heißt \emphdef{Stützhyperebene} von $P$, falls $H \cap P \neq \emptyset$ und $P \subset \Set{x & \<c, x\> \le c_0}$, d.h. $P$ liegt vollständig in einem durch $H$ bestimmten Halbraum.

    Für eine Stützhyperebene $H$ von $P$ heißt $S := P \cap H$ \emphdef{Seite} von $P$.
    Jede Seite von $P$ ist ein konvexes Polytop (Übung).
    $0$-dimensionale Seiten heißen \emphdef[Ecke]{Ecken}, $1$-dimensionale Seiten heißen \emphdef[Kante]{Kanten}, $(d-1)$-dimensionale Seiten heißen \emphdef[Facette]{Facetten}.
\end{df}

\begin{lem}
    Die Seiten eines konvexen Polytops $P$ bilden (inklusive $\emptyset$ und $P$) mit der Inklusion als Ordnungsrelation einen Verband $\scr F(P)$, genannt \emphdef{Seitenverband} (engl. “face lattice”).
    $\scr F(P)$ besitzt $\emptyset$ als minimales und $P$ als maximales Element.
    Die beiden Operationen im Verband sind gegeben durch
    \begin{math}
        A \vee B &:= \texp{kleinste Element, das $A \cup B$ enthält}, \\
        A \wedge B &:= \texp{größtes Element, das in $A \cap B$ enthalten ist}.
    \end{math}
    Insbesondere gilt
    \begin{math}
        A \wedge B
        = A \cap B
        &= (P \cap H_A) \cap (P \cap H_B) \\
        &= P \cap (H_A \cap H_B)
        = P \cap H_{A\cap B}
    \end{math}
    für entsprechende Hyperebenen $H_A$, $H_B$, $H_{A \cap B}$.
\end{lem}

\begin{ex}
    Hasse-Diagramm für einen Oktaeder.
\end{ex}

\begin{df}
    $P$ und $P'$ heißen \emphdef{kombinatorisch äquivalent}, wenn die Seitenverbände $\scr F(P)$ und $\scr F(P')$ isomorph sind.

    $P$ und $P'$ heißen \emphdef{kombinatorisch dual} (oder \emphdef{polar}), wenn $\scr F(P)$ isomorph zu dem umgekehrtem Verband von $\scr F(P)$ ist (d.h. oben und unten im Hasse-Diagramm vertauschen).
\end{df}

Wie konstruiert man geometrisch ein duales Polytop?

\begin{df}
    $P \subset \R^d$ sei ein konvexes $d$-Polytop.
    Dann erklären wir das duale Polytop $p^\Delta$ durch
    \begin{math}
        p^\Delta = \Set{c \in (\R^d)^* & \forall x \in P: c(x) \le 1}
        \subset (\R^d)^* \isomorphic \R^d.
    \end{math}
    $p^\Delta$ ist konvex, denn für $c_1, c_2 \in p^\Delta$ ist
    \begin{math}
        \lambda \underbrace{c_1(x)}_{\le 1} + (1-\lambda)\underbrace{c_2(x)}_{\le 1} \le 1
    \end{math}
    also $\lambda c_1 + (1-\lambda) c_2 \in c^\Delta$.
\end{df}

\begin{ex}
    Würfel / Oktaeder.
\end{ex}

\begin{df}
    Ein Polytop $P$ heißt \emphdef{simplizial}, wenn alle Seiten (insbesondere alle Facetten) Simplizes sind.

    Es heißt \emphdef{einfach}, wenn alle Eckenfiguren Simplizes sind.
\end{df}

% p^{\text{dual}} : kombinatorisch dual

\begin{kor}
    \begin{enumerate}[(i)]
        \item
            $P$ ist simplizial genau dann, wenn $p^{\text{dual}}$ einfach ist.
        \item
            $P$ ist einfach genau dann, wenn $p^{\text{dual}}$ simplizial ist.
        \item
            $P$ ist simplizial und einfach genau dann, wenn $p$ ein Simplex ist.
    \end{enumerate}
    \begin{proof}
        (i) und (ii) sind klar aufgrund der Dualität.

        Für (iii) ist die Rückrichtung trivial.
        Für die Hinrichtung sei $P$ ein simpliziales, einfaches $d$-Polytop.
        Jede Facette hat als $(d-1)$-Simplex $d$ Ecken.
        Jede Ecke ist in genau $d$ Kanten enthalten (Eckenfigur ist Simplex).
        Betrachte eine $(d-2)$-Seite der Facette.
        $P$ hat also insgesamt nur $d+1$ Ecken.
    \end{proof}
\end{kor}


\Timestamp{2015-04-29}

\begin{kor}
    Der Randkomplex eines simplizialen Polytops ist ein simplizialer Komplex, den man als Triangulierung der $(d-1)$-dimensionalen Sphäre auffassen kann.
\end{kor}

Ein Simplex $\<v_0, \dotsc, v_k\>$ kann mit der Menge der Ecken $\Set{v_0, \dotsc, v_k}$ identifiziert werden.
Beispiel Oktaeder, Liste der Facetten:
\begin{math}
    &123 & &523\\
    &124 & &524\\
    &164 & &564\\
    &136 & &536
\end{math}
Diagonalen sind $26, 15, 34$ (Tupel, die nicht in der Facettenliste auftauchen).

\begin{st}
    Es gibt eine Bijektion zwischen den $k$-dimensionalen Seiten eines konvexen Polytops $P$, die eine Ecke $v$ enthalten und den $(k-1)$-dimensionalen Seiten von $P/v$.
    Diese ist gegeben durch
    \begin{math}
        \pi: F \mapsto F \cap \Set{x & c(x) = c_0 - \eps}.
    \end{math}
    \begin{proof}
        Die inverse Abbildung ist gegeben durch
        \begin{math}
            F' \mapsto P \cap \Aff(\Set{v} \cup F').
        \end{math}
    \end{proof}
\end{st}

\begin{df}
    Falls $P$ simplizial ist, dann ist $P/v$ kombinatorisch äquivalent zum \emphdef{Link einer Ecke} $\Link(v)$, d.h. der Menge aller Seiten $F$ in $P$ mit $v \not\in F$, sodass $\Conv(F \cup \Set{v})$ eine Seite von $P$ ist.
    $\Link(v)$ ist ein simplizialer Komplex.
\end{df}
