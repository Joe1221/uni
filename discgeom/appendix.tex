\Timestamp{2015-07-22}

\section{Orientierung}

\begin{df}
    Ein \emphdef{orientierter} endlich-dimensionaler Vektorraum ist ein Vektorraum zusammen mit einer geordneten Basis $b_1, \dotsc, b_n$.

    Eine \emphdef{Orientierung} ist die Wahl einer solchen geodneten Basis bis auf Basistransformationen mit positiver Determinante (insbesondere auch gerade Permutationen der Basis).
\end{df}

\begin{note}
    \begin{itemize}
        \item
            Also gibt es genau zwei Orientierungen.
        \item
            Komplexe Vektorräume sind als reelle Vektorräume bereits orientiert, denn durch den kanonischen Isomorphismus $\C \isomorphic \R^2$ ist jede komplexe Basistransformation eine orientierungserhaltende im reellen.
        \item
            Im $\R^2$:
            \begin{math}
                \Vektor{1 & 0},
                \Vektor{0 & 1}
            \end{math}
            in einem gewöhnlichen kartesischen Koordinatensystem wird \emphdef{positiv orientiert} genannt.
        \item
            $\R^1$ besitzt zwei Orientierungen, dies hat bei der gewöhnlichen Integration zur Folge
            \begin{math}
                \int_a^b f(x) \di[x] = - \int_b^a f(x) \di[x].
            \end{math}
            Die Notation $\int_{[a,b]} f$ unterschlägt die Orientierung von $[a,b]$, implizit wird hier eine positive Orientierung angenommen.
        \item
            Im $\R^3$: Drei-Finger-Regel.

            Man unterscheidet zwischen „positiven und negativen Volumina“.
        \item
            Die Determinante ist ein Volumen mit Vorzeichen.
            Im $\R^2$ ist $\det(\vec a, \vec b)$ die Fläche eines Parallelograms mit Vorzeichen, es gilt z.B.
            \begin{math}
                \det(a, b) + \det(a, -b) = \det(a, b-b) = 0
            \end{math}
    \end{itemize}
\end{note}


\subsection{Alternierende Differentialformen}


Im $\R^2$: $\dx[x] \wedge \dx[y] = -\dx[y] \wedge \dx[x]$.
\begin{math}
    \int_A f(x,y) \dx[x] \wedge \dx[y]
    = - \int_A f(x,y) \dx[y] \wedge \dx[x]
\end{math}

Flächenelement: $\di[A] = \dx[u^1] \wedge \dx[u^2] =  \sqrt{\det{g_{ij}}}$, wobei $g_{ij}$ erste Fundamentalform der durch $(u^1, u^2)$ parametrisierten Fläche.

\begin{df}
    Eine \emphdef{Determinantenform} oder \emphdef{Determinantenfunktion} $\Phi$ ist eine nichtverschwindende alternierende Multilinearform $n$-Form $\Phi(b_1, \dotsc, b_n) \in \R$.
    \begin{math}
        \Phi(A b_1, \dotsc, A b_n)
        = \det A \Phi(b_1, \dotsc, b_n).
    \end{math}
    für $A : \R^n \to \R^n$ linear.
\end{df}

$\det A$ stellt die Volumenverzerrung durch $A$ mit Vorzeichen dar.

\begin{df}
    Ein orientiertes $k$-dimensionales Simplex $\<v_0, \dotsc, v_k\>$ ist gegeben durch $k + 1$ Punkte im $\R^k$ in allgemeiner Lage mit einer festen Reihenfolge (bis auf gerade Permutationen).

    Für eine Permutation $\sigma$ schreiben wir
    \begin{math}
        \<v_{\sigma(0)}, \dotsc, v_{\sigma(k)}\> = \sgn(\sigma) \<v_0, \dotsc, v_k\>.
    \end{math}
    Insebsondere gibt es genau zwei Orientierungen.
\end{df}

\begin{note}
    Auf diese Weise können ganzzahlige Linearkombinationen von Simplizes betrachtet werden (sogenannte $k$-Ketten, siehe algebraische Topologie).
\end{note}

Der \emphdef{kombinatorische orientierte Rand} ist gegeben durch die formale Summe
\begin{math}
    \Boundary \<v_0, \dotsc, k\>
    := \sum_{i=0}^k (-1)^i \<v_0, \dotsc, v_{i-1}, \hat v_i, v_{i+1}, \dotsc, v_k\>
\end{math}
Notation: $\hat v_i$: $v_i$ weglassen.

\begin{lem}
    $\Boundary \circ \Boundary = 0$ (als formale Null-Linearkombination)
    \begin{proof}
        Es gilt, da $\Boundary$ linear
        \begin{math}
            \Boundary \sum_{i=0}^k (-1)^k \<v_0, \dotsc, \hat v_i, \dotsc, v_k\>
            &= \sum_{i=0}^k (-1)^k \Boundary \<v_0, \dotsc, \hat v_i, \dotsc, v_k\> \\
            &= \sum_{i=0}^k (-1)^i \Big( \sum_{j<i} (-1)^j \<v_0, \dotsc, \hat v_j, \dotsc, \hat v_i, \dotsc, k\> + \sum_{j > i} (-1)^{j-1} \<v_0, \dotsc, \hat v_i, \dotsc, \hat v_j, \dotsc, v_k\> \Big) \\
            &= \sum_{i \neq j} \dotso
            = 0
        \end{math}
    \end{proof}
\end{lem}

\begin{note}
    \begin{itemize}
        \item
            Definition der kombinatorischen Orientierbarkeit von triangulierten Mannigfaltigkeiten als „kohärente Orientierung“ aller höchstdimensionalen Simplizes.
        \item
            Integration über orientierte, glatte Simplizes in diffenzierbaren Mannigfaltigkeiten (bzw. über Ketten).
            $\sigma: \Delta_{\text{or}}^k \to M^n$ differenzierbar von maximalem Rang:
            \begin{math}
                \int_\sigma“ w_k := \int_{\Delta^k} \sigma^*(w_k),
            \end{math}
            wobei $\sigma^*(\dx[x] \wedge \dx[y])(X,Y) = \dx[x] \wedge \dx[y](D\sigma(X), D\sigma(Y))$.
        \item
            Integration über $k$-dimensionale glatte Ketten $\sum_i \sigma_i$ von orientierten Simplizes in $M^n$, $\omega$ eine alternierende $k$-Form, dann ist
            \begin{math}
                \int_{\sum \sigma_i} \omega := \sum_i \int_{\sigma_i} \omega.
            \end{math}
        \item
            Satz von Stokes:
            \begin{math}
                \int_A \di[\omega] = \int_{\Boundary A} \omega,
            \end{math}
            wenn $A$ eine $k$-Mannigfaltigkeit mit Rand ist ($k$-Kette), orientiert und $\omega$ eine $(k-1)$-Form.
        \item
            \begin{math}
                \int_A \dx (\omega + \dx[\eta])
                &= \int_{\Boundary A} (\omega + \dx[\eta])
                = \int_A \dx[\omega] \\
                \int_{A + \Boundary B} \dx[\omega] = \int_{\Boundary A + 0} \omega = \int_A \dx[\omega].
            \end{math}
    \end{itemize}
\end{note}
