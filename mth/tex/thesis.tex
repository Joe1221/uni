\documentclass{mythesis}

\usepackage{titling}

\title{Inpainting mit Eulers Elastica}
\author{Stephan Hilb}


%\DeclareDocumentCommand{\thesection}{}{\arabic{section}}
%\DeclareDocumentCommand{\thesubsection}{}{\thesection.\arabic{subsection}}

% "such that"
\DeclareDocumentCommand{\st}{}{\mathbin{|}}

\DeclareDocumentCommand{\Edat}{}{E_{\mathrm{dat}}}
\DeclareDocumentCommand{\Eimg}{}{E_{\mathrm{img}}}


\begin{document}

%\usepackage{titling}

\begin{titlepage}
  \begin{center}
    ~\par\vspace{4em}
    {
      \fontsize { 16pt } { 16pt } \selectfont
      Masterarbeit
    }
    \par\vspace{3em}
    {
      \fontsize { 24pt } { 24pt } \selectfont \sffamily \bfseries
      \thetitle
    }
    \par\vspace{3em}
    {
      \fontsize { 16pt } { 16pt } \selectfont \scshape
      \theauthor
    }
    \par\vspace{1.5em}
    {
      \fontsize { 14pt } { 14pt } \selectfont %\scshape
      \today
    }
    \par\vspace{4.5em}
    {
    }
    \par\vspace{8em}
    {
      \fontsize { 14pt } { 14pt } \selectfont \scshape
      Universität Stuttgart
    }
    \par\vspace{1em}
    {
      \fontsize { 14pt } { 14pt } \selectfont %\scshape
      Institut für Angewandte Analysis und Numerische Simulation
    }
    \par\vspace{1em}
    {
      \fontsize { 14pt } { 14pt } \selectfont %\scshape
      Betreuer:
      Dr. Claus J. Heine,
      Dr. Andreas Langer
    }
  \end{center}
\end{titlepage}

\chapter*{Zusammenfassung}

Die Zielsetzung, fehlende Teile eines Bildes zu rekonstruieren – auch „Inpainting“ genannt – lässt sich als Minimierung eines
Funktionals für das Gesamtbild modellieren, bestimmt durch ein Datenmodell, das die Übereinstimmung mit dem ursprünglichen Bild auf dem bekannten Gebiet kontrolliert, und einem Bildmodell, welches maßgebend für die Güte der Rekonstruktion ist.

Das Euler Elastica Bildmodell, welches die Niveaulinien eines Bildes nach dem Vorbild elastischer Stäbe modelliert, bietet ein vielversprechendes Bildmodell und kommt in dieser Arbeit zum Einsatz.
Für die numerische Minimierung wird eine bekannte “alternating direction“ Augmented Lagrange Methode angewandt und die entstehenden Teilprobleme erstmalig im Kontext der Finiten Elemente gelöst.


{
  \let\clearpage\relax
  \tableofcontents
  %\addtocentrydefault{chapter}{}{Inhaltsverzeichnis}
}
%\tableofcontents



\chapter{Einführung}


%Mit Blick auf \ref{fig:setting} führen wir zunächst Begrifflichkeiten ein.


\begin{definition} \label{def:image}
    Ein \emphdef{Bild} ist eine Abbildung $u: \Omega \to F$, wobei $\Omega \subset \R^d$
    \emphdef{Trägermenge} genannt wird und $F$ \emphdef{Farbraum}.
    Für einen festen Farbraum $F$ sei $I_X$ die Menge der Bilder mit Trägermenge $X$.
    \begin{note}
	Wir betrachten im weiteren Verlauf Graustufenbilder und setzen daher stets $F := [0,1]$ (mit der
	Interpretation: $0$ entspricht „schwarz“ und $1$ „weiß“).
    \end{note}
\end{definition}

Beim Inpainten gehen wir von einem gegebenen Bild $u^0: \Omega \setminus D \to [0,1]$ und einem \emphdef{Inpainting-Bereich}
$D \subset \Omega$ aus. Ziel ist nun eine Rekonstruktion $u: \Omega \to [0,1]$, die optisch „möglichst gut zu $u^0$ passt“.

Wir werden „möglichst gut“ durch die Minimierung eines Energiefunktionals $E[u]$ bestehend aus einem Datenmodell (engl. “data model”) und einem Bildmodell (engl. “image prior model”) ersetzen.
Der Bayes'sche Ansatz liefert hierfür eine schmackhafte Motivation.


\section{Das Bayes'sche Prinzip}

Es seien hier $I_\Omega$, $I_{\Omega \setminus D}$ endlich.
Dies ist beispielsweise der Fall, wenn $\Omega \subset \Z^d$ beschränkt und $F$ endlich ist, was bei praktisch allen digitalen Bildern zwingend ist.

Geht man davon aus, dass $u^0$ zu einem ursprünglichen, vollständigen Bild $u$ gehört, so lässt sich die Entstehung von $u^0$ als Zufallsexperiment in zwei Schritten modellieren:
\begin{enumerate}
    \item
	Auftreten von $u$ als Bild im diskreten Wahrscheinlichkeitsraum $(I_\Omega, 2^{I_\Omega}, P_{\mathrm{img}})$,
    \item
        Zustandekommen von $u^0$ als Beobachtung von $u|_{\Omega \setminus D}$ im diskreten Wahrscheinlichkeitsraum $(I_{\Omega \setminus D}, 2^{I_{\Omega \setminus D}}, P_{\mathrm{dat},u})$.
\end{enumerate}
Wir betrachten dazu den diskreten Wahrscheinlichkeitsraum $\scr P := (I_\Omega \times I_{\Omega\setminus D}, 2^{I_\Omega \times I_{\Omega\setminus D}}, P)$ mit der sogenannten A-Posteriori-Wahrscheinlichkeitsverteilung
\begin{math}
    P(u, u^0) := P_{\mathrm{img}}(u) \cdot P_{\mathrm{dat},u^0}(u)
\end{math}
In diesem Raum besitzt $P_{\mathrm{dat},u^0}(u)$ die gewünschte Interpretation als bedingte Wahrscheinlichkeit $P(u^0|u)$:
\begin{math}
    P(u) &= P_{\mathrm{img}}(u), \\
    P(u^0|u) &= \frac{P(u, u^0)}{P(u)} = P_{\mathrm{dat},u^0}(u),
\end{math}
wobei wir $u$ und $u^0$ mit $\Set{u} \times I_{\Omega \setminus D}$, bzw. $I_\Omega \times \Set{u_0}$ identifizieren.
Dieser Raum erlaubt uns die Berechnung der bedingten Wahrscheinlichkeit
\begin{math}
    P(u|u^0) &= \frac{P(u, u^0)}{P(u^0)}
    = \frac{P(u^0|u) \cdot P(u)}{P(u^0)}.
\end{math}
Dieser einfache Zusammenhang zwischen den bedingten Wahrscheinlichkeiten ist auch als \emph{Satz von Bayes} bekannt.

Beim Inpainten ist $P(u^0)$ ist konstant und $P(u|u^0)$ die Wahrscheinlichkeit, dass $u$ die korrekte Rekonstruktion von $u^0$ ist, also maximieren wir
\begin{math}
    P(u|u^0) = \const \cdot P_{\mathrm{dat},u^0}(u) \cdot P_{\mathrm{img}}(u).
\end{math}
Wir können nun auf beiden Seiten $-\log(\argdot)$ anwenden und ein äquivalentes, aber durchaus handlicheres Problem in Energie-Form\footnote{Namensgebung stammt wohl aus der Thermodynamik, siehe \ref{??}} betrachten.

Für gegebenes $u^0$, $D$ minimieren wir also die Summe aus Datenterm und Bildterm (die additive Konstante ist für das Minimieren irrelevant)
\begin{math}
    E[u] = \Edat[u] + \Eimg[u].
\end{math}
Die Wahlen von $\Edat[u]$ und $\Eimg[u]$, welche den Wahrscheinlichkeitsverteilungen in unserem zweistufigen Zufallsmodell entsprechen, sind durch das Datenmodell, bzw. das Bildmodell festgelegt.


\section{Das Datenmodell}

Naheliegend ist der Gedanke, als Datenmodell lediglich $u|_{\Omega \setminus D} = u^0|_{\Omega \setminus D}$ zu fordern.
Damit würden feste Randbedingungen für das Minimieren auf $D$ vorgegeben werden.
In der Praxis sind Bilder jedoch meist mit Rauschen, Unschärfe oder anderen Artefakten versetzt, sei es aufgrund technischer Aufnahmebedingungen, natürlichen Zerfalls, digitaler Kompressionsartefakte oder aus anderen Gründen.
Das Datenmodell erlaubt es zu definieren, auf welche Weise $u|_{\Omega \setminus D}$ beim Inpainten an das vorliegende Bild $u^0$ angepasst werden soll und ermöglicht es solche Verunreinigungen auszugleichen.

Wir nutzen den bekannten Datenterm aus \ref{ROF}, welcher die quadrierte $L^2$-Norm verwendet:
\begin{math}
    \Edat[u] = \frac{\eta}{2} \int_{\Omega \setminus D} |u - u^0|^2,
\end{math}
wobei $\eta \in \R_{> 0}$ die spätere Gewichtung des Datenterms zum Bildterm kontrolliert.

In \ref{ROF??} wird gezeigt, dass dieser Datenterm gut geeignet ist, wenn das Orginalbild mit weißem Gaus'schem Rauschen versetzt ist, modelliert als $u^0 = u|_{\Omega \setminus D} + \eta$.
Für Schwarz/Weiß Rauschen (auch “salt-and-pepper noise” gennant) scheint die Wahl einer $L^1$-Norm angebracht, wie in \ref{??} erwähnt.

Denkbar wären auch Datenmodelle wie $u^0 = K[u|_{\Omega\setminus D}] + \epsilon$ für einen Glättungsoperator $K$.
Solche Modelle würden es erlauben, Unschärfe zu reduzieren, siehe z.B. \ref{??}.


\section{Das Bildmodell}

Da der Datenterm sich auf $\Omega \setminus D$ beschränkt hat, trägt das Bildmodell die Hauptrolle bei der eigentlichen Bildrekonstruktion innerhalb von $D$.
Man beachte, dass das Bildmodell unabhängig von $D$ ist (man erinnere sich an das Bayes-Prinzip) und auch für $D = \emptyset$ (z.B. Denoising) die Rekonstruktion des ursprünglichen Bildes verbessert.
In der Literatur zum Denoising wird $\Eimg[u]$ daher auch als “regularization term” bezeichnet.

Zunächst macht man sich schnell anhand von einfachen Beispielen klar, dass die Wahl eines guten Bildmodells keineswegs einfach oder eindeutig ist, da die Bildbewertung oft von der menschlichen Interpretation abhängt.
Welche der beiden vollständigen Bilder aus Abbildung \ref{fig:inpainting_non_unique} ein Betrachter eher für richtig hält, hängt davon ab, ob er Weiß oder Schwarz für vordergründiger hält.

Anhand von Abbildung \ref{fig:inpainting_prefer_smooth_and_straight} sieht man, dass in der Regel Bilder bevorzugt werden, bei denen die Kanten (oder Niveaulinien) des Bildes möglichst gering gekrümmt und kurz sind.
Die Niveaulinien eines Bildes stellen also in vielen Fällen eine intuitive Bewertungsgrundlage dar.
Legt man eine geeignete Bewertung für die Höhenlinien eines Bildes (im Sinne eines Energiefunktionals) fest, so liefert die Summation über alle Höhenlinien eine Bewertung des Gesamtbildes.
Diese Methode wird \emphdef{Levelset-Methode} genannt (siehe Kapitel \ref{chap:image_model}.

Das Bildmodell in dieser Arbeit nutzt die Levelset-Methode und wählt für jede Niveaulinie $\gamma$ die \emphdef{Elastica Energie}
\begin{math}
    \int_{\gamma} \alpha + \beta \kappa^2 \di[s].
\end{math}
Eine Kurve, die diese Energie (unter geeigneten Randbedingungen) minimiert, wird (freie) \emphdef{Elastica} genannt.
In der Physik hat sie die Interpretation der Biegeenergie eines elastischen Stabes.
Interessierte seien auf das umfangreiche Werk \ref{TreatiseElastica} verwiesen.

Die Levelset-Methode liefert uns schließlich
\begin{math}
    \Eimg[u] = \int_{\Omega} \alpha + \beta (\nabla \cdot \frac{\nabla u}{\nabla u})^2 \di[x]
\end{math}
für unser Bildmodell (näheres in Kapitel \ref{chap:image_model}).

Es sei bemerkt, dass diese Levelset-Inpaintingmethoden nicht darauf ausgelegt sind wiederkehrende Strukturen zu reproduzieren, die deutlich kleiner als der Inpaintingbereich $D$ sind (siehe Abbildung \ref{fig:inpainting_texture}).
%Wiederkehrende Strukturen außerhalb $D$ werden nicht innerhalb $D$ fortgesetzt.
Hierfür gibt es sogenanntes “texture based inpainting” mit zahlreichen stochastischen Methoden (siehe z.B. \ref{??}).


\chapter{Das Euler Elastica Bildmodell}




\section{Motivation} % vergleich tv-inpainting

\section{Die Euler Elastica}

\section{Die Level-Set-Methode}

\section{Das Euler Elastica Inpainting Modell} % ?



\chapter{Der ADMM-Algorithmus}


\section{Motivation} % komplexität von euler-lagrange und numerische probl.

\section{Method of Multipliers}

\section{Alternating Direction Method of Multipliers}

\section{Euler Elastica ADMM}



\chapter{Numerische Implementierung und Resultate}


\section{Anmerkungen zur Implementierung}
% Finite Elemente, Wahl der Räume, Dune-Zeug

\section{TV-Inpainting}

\section{Euler Elastica Inpainting}














\section{Einführung/Überblick}

\begin{itemize}
    \item
	Non-texture inpainting, geometry based inpainting
    \item
	Eulers Elastica als Kurvenmodell
    \item
	Level-Set Methode
    \item
	Das Inpainting-Modell
    \item
	Lösungsansätze für das EE inpainting model in der Literatur
    \item
	High-Level-Idee des ADMM-Algorithmus
\end{itemize}

\section{Das Euler Elastica Inpainting Modell}

\subsection{Das Kurven-Modell}

\begin{itemize}
%    \item
%	Cumulative Point-Energy Axiome
%    \item
%	Two-Point Energy und die Längenenergie $e[\Gamma] = \int_\Gamma \di[s]$
%    \item
%	Three-Point Energy und die Krümmungs-Energie $e[\Gamma] = \int_\Gamma \phi(\kappa) \di[s]$
    \item
	$e[\Gamma] = \int_\Gamma \alpha + \beta \kappa^2 \di[s]$
\end{itemize}

\subsection{Das Bild-Modell}

\begin{itemize}
    \item
	Die Level-Set-Methode
	\begin{math}
	    E[u] = \int_{[0,1]} e[\Gamma_\lambda] \di[\lambda]
	    = \int_{[0,1]} \int_{\Gamma_\Lambda} \alpha + \beta \kappa^2 \di[s] \di[\lambda]
	\end{math}
    \item
	Co-Area Formel
    \item
	Euler-Elastica Bild-Modell:
	\begin{math}
	    E[u] = \int_{\Omega} (\alpha + \beta \kappa^2) |\nabla u| \di[x]
	\end{math}
	für $\kappa = \nabla \cdot (\frac{\nabla u}{|\nabla u|})$.
\end{itemize}

\subsection{Das Inpainting-Modell}

\begin{itemize}
    \item
	Daten-Modell, $u^0|_{\Omega\setminus D} = (u + n)|_{\Omega\setminus D}$
    \item
	Für Gauß-Rauschen $n$:
	\begin{math}
	    E[u^0 \st u, D] = \frac{\eta}{2} \int_{\Omega \setminus D} |u - u^0|^2 \di[x]
	\end{math}
    \item
	Inpainting-Modell:
	\begin{math}
	    E[u \st u^0, D]
	    = \int_\Omega (\alpha + \beta \kappa^2) |\nabla u| \di[x] + \frac{\eta}{2} \int_{\Omega\setminus D} |u - u^0|^2 \di[x]
	\end{math}
\end{itemize}

\section{Der ADMM-Algorithmus}

\subsection{Augmented Lagrange und ADMM im Allgemeinen}

\subsection{Euler-Elastica-Inpainting ADMM}

\begin{itemize}
    \item
	Operator-Splitting:
	\begin{math}
	    &\min_{v,u,m,p,n} \int_{\Omega} (\alpha + \beta(\nabla \cdot n)^2) |p| + \frac{\eta}{2} \int_{\Omega\setminus D} |v - u^0|^2 \\
	    &\quad\mathrm{s.t.}\quad v = u, p = \nabla u, n = m, |p| = m \cdot p, |m| \le 1.
	\end{math}
    \item
	Augmented Lagrange Funktional:
	\begin{math}
	    \scr L[v,u,m,p,n;\lambda_1,\lambda_2,\lambda_3,\lambda_4]
	    &= \int_{\Omega} (\alpha + \beta(\nabla \cdot n)^2) |p| + \frac{\eta}{2} \int_{\Omega\setminus\Gamma} |v - u^0|^2 \\
	    &\quad + r_1 \int_\Omega (|p| - m\cdot p) + \int_\Omega \lambda_1 (|p| - m \cdot p) \\
	    &\quad + \frac{r_2}{2} \int_\Omega |p - \nabla u|^2 + \int_\Omega \lambda_2 \cdot (p - \nabla u) \\
	    &\quad + \frac{r_3}{2} \int_\Omega (v - u)^2 + \int_\Omega \lambda_3 (v - u) \\
	    &\quad + \frac{r_4}{2} \int_\Omega |n-m|^2 + \int_\Omega \lambda_4 \cdot (n - m) + \delta_{\ge 1}(m).
	\end{math}
    \item
	Updates:
	\begin{math}
	    \lambda_1 &\gets \lambda_1 + r_1 (|p| - m\cdot p), \\
	    \lambda_2 &\gets \lambda_2 + r_2 (p - \nabla u), \\
	    \lambda_3 &\gets \lambda_3 + r_3 (v - u), \\
	    \lambda_4 &\gets \lambda_4 + r_4 (n - m).
	\end{math}
\end{itemize}

\pagebreak
\subsection{EE-ADMM Unterprobleme}

%\begin{math}
%    \scr E_1[v]
%    &= \frac{\eta}{2} \int_{\Omega\setminus D} |v - u^0|^2 + \frac{r_3}{2} \int_\Omega (v-u)^2 + \int_\Omega \lambda_3(v - u)\\
%    \scr E_2[u]
%    &= \frac{r_2}{2} \int_\Omega |p - \nabla u|^2 + \int_\Omega \lambda_2 \cdot (p - \nabla u) + \frac{r_3}{2} \int_\Omega (v-u)^2 + \int_\Omega \lambda_3 (v-u) \\
%    \scr E_3[m]
%    &= r_1 \int_\Omega(|p| - m\cdot p) + \int_\Omega \lambda_1 (|p| - m \cdot p) + \frac{r_4}{2} \int_\Omega |n-m|^2 + \int_\Omega \lambda_4 \cdot (n-m) + \delta_{\ge 1}(m) \\
%    &= \frac{r_4}{2} \int_\Omega |x-m|^2 + \delta_{\ge 1}(m) + \const\\
%    \scr E_4[p]
%    &= \int_\Omega (\alpha + \beta(\nabla \cdot n)^2) |p| + r_1 \int_\Omega (|p| - m\cdot p) + \int_\Omega \lambda_1 (|p| - m\cdot p) \\
%    &\qquad + \frac{r_2}{2} \int_\Omega |p - \nabla u|^2 + \int_\Omega \lambda_2 \cdot (p - \nabla u) \\
%    &= \int_\Omega \Big(\alpha + \beta (\nabla \cdot n)^2 + r_1 + \lambda_1\Big) |p| + \frac{r_2}{2} \int_\Omega \Big| p - \big( \nabla u + \frac{r_1 + \lambda_1}{r_2} m - \frac{1}{r_2} \lambda_2 \big) \Big|^2 + \const\\
%    \scr E_5[n]
%    &= \int_\Omega (\alpha + \beta(\nabla \cdot n)^2 ) |p| + \frac{r_4}{2} \int_\Omega |n-m|^2 + \int_\Omega \lambda_4 \cdot (n - m)
%\end{math}

\begin{enumerate}[1)]
    \item
	Minimiere
	\begin{math}
	    \scr E_1[v]
	    = \frac{\eta}{2} \int_{\Omega\setminus D} |v - u^0|^2 + \frac{r_3}{2} \int_\Omega (v-u)^2 + \int_\Omega \lambda_3(v - u).
	\end{math}
	Variationsrechnung liefert
	\begin{math}
	    0 &= \ddx[\eps] \scr E_1[u + \eps \phi] \\
	    &= \eta \int_{\Omega \setminus D} (v - u^0) \phi + r_3 \int_\Omega (v - u) \phi + \int_\Omega \lambda_3 \phi \\
	    &= \int_\Omega \Big(\eta (v - u^0)\Ind_{\Omega \setminus D}  + r_3 (v - u) + \lambda_3 \Big) \phi,
	\end{math}
	also nach dem Hauptsatz der Variationsrechnung
	\begin{math}
	    v = \begin{cases}
	        u - \frac{\lambda_3}{r_3} & \text{auf $D$}, \\
		\frac{\eta u^0 + r_3 u - \lambda_3}{\eta + r_3} & \text{auf $\Omega\setminus D$}.
	    \end{cases}
	\end{math}
    \item
	Minimiere
	\begin{math}
	    \scr E_2[u]
	    = \frac{r_2}{2} \int_\Omega |p - \nabla u|^2 + \int_\Omega \lambda_2 \cdot (p - \nabla u) + \frac{r_3}{2} \int_\Omega (v-u)^2 + \int_\Omega \lambda_3 (v-u).
	\end{math}
	Variationsrechnung liefert
	\begin{math}
	    0 &= \ddx[\eps] \scr E_2[u + \eps \phi] \\
	    &= r_2 \int_\Omega (\nabla u - p) \cdot \nabla \phi - \int_\Omega \lambda_2 \cdot \nabla \phi + r_3 \int_\Omega (u - v) \phi - \int_\Omega \lambda_3 \phi \\
	    &= \int_\Omega (r_2 \nabla u - r_2 p - \lambda_2) \cdot \nabla \phi + \int_\Omega (r_3 u - r_3 v - \lambda_3) \phi,
	\end{math}
	als schwache Form einer linearen PDE zweiter Ordnung.
	%Euler-Lagrange liefert PDE:
	%\begin{math}
	%    -r_2 \Laplace u + r_3 u = - r_2 \nabla \cdot p - \nabla \cdot \lambda_2 + r_3 v + \lambda_3.
	%\end{math}
    \item
	Minimiere
	\begin{math}
	    \scr E_3[m]
	    &= r_1 \int_\Omega(|p| - m\cdot p) + \int_\Omega \lambda_1 (|p| - m \cdot p) + \frac{r_4}{2} \int_\Omega |n-m|^2 + \int_\Omega \lambda_4 \cdot (n-m) + \delta_{\ge 1}(m) \\
	    &= \frac{r_4}{2} \int_\Omega |x-m|^2 + \delta_{\ge 1}(m) + \const,
	\end{math}
	wobei $x = \frac{(r_1 + \lambda_1)p + \lambda_4}{r_4} + n$.

	Definiere punktweise
	\begin{math}
	    m^* := \begin{cases}
	        x & \text{für $|x| < 1$}, \\
		\frac{x}{|x|} & \text{für $|x| \ge 1$}.
	    \end{cases}
	\end{math}
	$m^*$ minimiert $\scr E_3$.
	Betrachte dazu $\scr E_3[m^* + \phi]$.
	Wir können ohne Einschränkung fordern, dass $|m^* + \phi| \le 1$ auf $\Omega \setminus N$ für eine Nullmenge $N$, (sonst trivialerweise $\infty = \scr E_3[m^* + \phi] \ge \scr E_3[m^*]$).
	Setze
	\begin{math}
	    M := \Set{x \in \Omega & |x| \ge 1} \setminus N.
	\end{math}
	Dann ist auf $M$
	\begin{math}
	    1 \ge |m^* + \phi|^2 = \l| \frac{x}{|x|} \r|^2 + 2 \< \frac{x}{|x|}, \phi \> + |\phi|^2,
	\end{math}
	also $-2\<\frac{x}{|x|}, \phi\> \ge |\phi|^2$.

	Damit ergibt sich schließlich
	\begin{math}
	    \scr E[m^* + \phi]
	    &= \int_M | \frac{x}{|x|} - x + \phi|^2 \\
	    &= \int_M | \frac{x}{|x|} - x|^2 + 2 \int_M \<\frac{x}{|x|} - x, \phi\> + \int_M |\phi|^2 \\
	    &= \scr E[m^*] + \int_M \underbrace{(|x| - 1)}_{\ge 0} \underbrace{(-2\<\frac{x}{|x|}, \phi\>)}_{\ge |\phi|^2} + \|\phi\|_{L^2} \\
	    &= \scr E[m^*] + \|\phi\|_{L^2}.
	\end{math}
	Also ist $m^*$ ein Minimierer von $\scr E_3$ in $L^2$.

	%Explizite Lösung (Lemma):
	%\begin{math}
	%    m = \operatorname{proj}_{\le 1}(x)
	%    = \operatorname{proj}_{\le 1}\Big( \frac{(r_1 + \lambda_1)p + \lambda_4}{r_4} + n \Big)
	%\end{math}
    \item
	Minimiere
	\begin{math}
	    \scr E_4[p]
	    &= \int_\Omega (\alpha + \beta(\nabla \cdot n)^2) |p| + r_1 \int_\Omega (|p| - m\cdot p) + \int_\Omega \lambda_1 (|p| - m\cdot p) \\
	    &\qquad + \frac{r_2}{2} \int_\Omega |p - \nabla u|^2 + \int_\Omega \lambda_2 \cdot (p - \nabla u) \\
	    &= \int_\Omega \Big(\alpha + \beta (\nabla \cdot n)^2 + r_1 + \lambda_1\Big) |p| + \frac{r_2}{2} \int_\Omega \Big| p - \big( \nabla u + \frac{r_1 + \lambda_1}{r_2} m - \frac{1}{r_2} \lambda_2 \big) \Big|^2 + \const \\
	    &= \int_\Omega c |p|_2 + \frac{r_2}{2} \int_\Omega |p - q|^2 + \const
	\end{math}
	Es gilt $c, r_2 > 0$ (siehe auch Lagrange-Update für $\lambda_1$).
	Variationsrechnung liefert für $|p| \neq 0$
	\begin{math}
	    0 &= \ddx[\eps] \scr E_4[p + \eps\phi] \\
	    &= \int_\Omega c \frac{p}{|p|} \cdot \phi + r_2 \int_\Omega (p - q) \cdot \phi.
	\end{math}
	Nach dem Hauptsatz der Variationsrechnung
	\begin{math}
	    (\frac{c}{|p|} + r_2) p = r_2 q.
	\end{math}
	Setze an $p := \lambda q$ mit $\lambda > 0$ also
	\begin{math}
	    \lambda = 1 - \frac{c}{r_2 |q|}
	\end{math}
	Der Minimierer ist damit gegeben durch
	\begin{math}
	    p := \max\Set{0, 1 - \frac{c}{r_2 |q|}} q
	\end{math}
	%Explizite Lösung (Lemma: soft thresholding \dots):
	%\begin{math}
	%    p = \max\Set{0, 1 - \frac{\alpha + \beta(\nabla \cdot n)^2 + r_1 + \lambda_1}{r_2 |q|}} q
	%\end{math}
    \item
	Minimiere
	\begin{math}
	    \scr E_5[n]
	    &= \int_\Omega (\alpha + \beta(\nabla \cdot n)^2 ) |p| + \frac{r_4}{2} \int_\Omega |n-m|^2 + \int_\Omega \lambda_4 \cdot (n - m)
	\end{math}
	Variationsrechnung liefert
	\begin{math}
	    0 &= \ddx[\eps] \scr E_5[n + \eps \phi] |_{\eps = 0} \\
	    &= \int_\Omega 2\beta |p| (\nabla \cdot n) (\nabla \cdot \phi) + \int_\Omega (r_4 n - r_4 m + \lambda_4) \cdot \phi
	\end{math}
	%Euler-Lagrange liefert PDE-System:
	%\begin{math}
	%    -2 \nabla (\beta |p| \nabla \cdot n) + r_4 (n - m) + \lambda_4 = 0
	%\end{math}
\end{enumerate}

\section{Diskretisierung mit Dune-ACFem und Inpainting-Resultate}

\begin{itemize}
    \item
	Gitterwahl: structured (simplex/kubisch) oder unstructured (simplex, guiding durch Kantendetektor)
    \item
	Lineare Lagrange Basis-Funktionen als nahezu einzige Wahl
    \item
	Adaptive Strategien (z.B. refine/coarse gemäß $n = \nabla u$), Vergleich
    \item
	Parameter-Tweaking ($\alpha$, $\beta$, $\eta$, $r_1, r_2, r_3, r_4$) und -Interpretation
    \item
	Startwert-Tweaking ($v, u, m, p, n, \lambda_1, \lambda_2, \lambda_3, \lambda_4$)
\end{itemize}

\section{Weitere Anwendungen}

\begin{itemize}
    \item
	Debluring
    \item
	Upscaling
\end{itemize}


\section{Appendix}

\subsection{Co-Area-Formel}
\subsection{Euler-Lagrange}
\subsection{Soft-Thresholding?}


\end{document}

