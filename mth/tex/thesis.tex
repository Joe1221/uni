\documentclass{mythesis}

\usepackage{titling}

\title{Inpainting mit Eulers Elastica}
\author{Stephan Hilb}


\DeclareDocumentCommand{\thesection}{}{\arabic{section}}
\DeclareDocumentCommand{\thesubsection}{}{\thesection.\arabic{subsection}}


% "such that"
\DeclareDocumentCommand{\st}{}{\mathbin{|}}


\begin{document}

%\usepackage{titling}

\begin{titlepage}
  \begin{center}
    ~\par\vspace{4em}
    {
      \fontsize { 16pt } { 16pt } \selectfont
      Masterarbeit
    }
    \par\vspace{3em}
    {
      \fontsize { 24pt } { 24pt } \selectfont \sffamily \bfseries
      \thetitle
    }
    \par\vspace{3em}
    {
      \fontsize { 16pt } { 16pt } \selectfont \scshape
      \theauthor
    }
    \par\vspace{1.5em}
    {
      \fontsize { 14pt } { 14pt } \selectfont %\scshape
      \today
    }
    \par\vspace{4.5em}
    {
    }
    \par\vspace{8em}
    {
      \fontsize { 14pt } { 14pt } \selectfont \scshape
      Universität Stuttgart
    }
    \par\vspace{1em}
    {
      \fontsize { 14pt } { 14pt } \selectfont %\scshape
      Institut für Angewandte Analysis und Numerische Simulation
    }
    \par\vspace{1em}
    {
      \fontsize { 14pt } { 14pt } \selectfont %\scshape
      Betreuer:
      Dr. Claus J. Heine,
      Dr. Andreas Langer
    }
  \end{center}
\end{titlepage}

\chapter*{Zusammenfassung}

Die Zielsetzung, fehlende Teile eines Bildes zu rekonstruieren – auch „Inpainting“ genannt – lässt sich als Minimierung eines
Funktionals für das Gesamtbild modellieren, bestimmt durch ein Datenmodell, das die Übereinstimmung mit dem ursprünglichen Bild auf dem bekannten Gebiet kontrolliert, und einem Bildmodell, welches maßgebend für die Güte der Rekonstruktion ist.

Das Euler Elastica Bildmodell, welches die Niveaulinien eines Bildes nach dem Vorbild elastischer Stäbe modelliert, bietet ein vielversprechendes Bildmodell und kommt in dieser Arbeit zum Einsatz.
Für die numerische Minimierung wird eine bekannte “alternating direction“ Augmented Lagrange Methode angewandt und die entstehenden Teilprobleme erstmalig im Kontext der Finiten Elemente gelöst.


{
  \let\clearpage\relax
  \tableofcontents
  %\addtocentrydefault{chapter}{}{Inhaltsverzeichnis}
}
%\tableofcontents



\chapter{Einführung}


Mit Blick auf \ref{fig:setting} führen wir zunächst Begrifflichkeiten ein.


\begin{definition} \label{def:image}
    Ein \emphdef{Bild} ist eine Abbildung $u: \Omega \to F$, wobei $\Omega \subset \R^d$
    \emphdef{Trägermenge} genannt wird und $F$ \emphdef{Farbraum}.
    Wir betrachten im weiteren Verlauf Graustufenbilder und setzen daher $F := [0,1]$ (mit der
    Interpretation: $0$ entspricht „schwarz“ und $1$ „weiß“).
\end{definition}

Beim Inpainten gehen wir von einem gegebenen Bild $u^0: \Omega \setminus D \to [0,1]$ und einem \emphdef{Inpainting-Bereich}
$D \subset \Omega$ aus. Ziel ist nun eine Rekonstruktion $u: \Omega \to [0,1]$, die optisch „möglichst gut zu $u^0$ passt“.

Wir werden „möglichst gut“ durch die Minimierung eines Energiefunktionals $E[u]$ bestehend aus einem Datenmodell (engl. “data model”) und einem Bildmodell (engl. “image prior model”) ersetzen.
Der Bayes'sche Ansatz liefert hierfür eine schmackhafte Motivation.

\section{Das Bayes'sche Prinzip}

Geht man davon aus, dass $u^0$ zu einem vollständigen Bild $u^*$ gehört, so lässt sich die Entstehung von $u^0$ als Zufallsexperiment in zwei Schritten modellieren:
\begin{enumerate}
    \item
	Entstehung von $u^*$ als ursprüngliches Bild und
    \item
        $u^0$ als Beobachtung von $u^*|_{\Omega \setminus D}$.
\end{enumerate}




\begin{math}
    E[u|u^0, D] = E[u^0|u,D] + E[u]
\end{math}







\section{Einführung/Überblick}

\begin{itemize}
    \item
	Motivation
    \item
	Bayes-Framework (Image Model und Data Model)
	\begin{math}
	    P(u \st u^0,D) &= \frac{P(u^0 \st u,D) P(u \st D)}{P(u^0 \st D)} \\
	    &= P(u^0 \st u, D) P(u) \cdot \const \\
	\end{math}
    \item
	Bayes in Energie-Form:
	\begin{math}
	    E[u\st u,D] &= \underbrace{E[u^0 \st u,D]}_{\text{data model}} + \underbrace{E[u]}_{\text{image model}} + \const
	\end{math}
    \item
	Non-texture inpainting, geometry based inpainting
    \item
	Eulers Elastica als Kurvenmodell
    \item
	Level-Set Methode
    \item
	Das Inpainting-Modell
    \item
	Lösungsansätze für das EE inpainting model in der Literatur
    \item
	High-Level-Idee des ADMM-Algorithmus
\end{itemize}

\section{Das Euler Elastica Inpainting Modell}

\subsection{Das Kurven-Modell}

\begin{itemize}
%    \item
%	Cumulative Point-Energy Axiome
%    \item
%	Two-Point Energy und die Längenenergie $e[\Gamma] = \int_\Gamma \di[s]$
%    \item
%	Three-Point Energy und die Krümmungs-Energie $e[\Gamma] = \int_\Gamma \phi(\kappa) \di[s]$
    \item
	$e[\Gamma] = \int_\Gamma \alpha + \beta \kappa^2 \di[s]$
\end{itemize}

\subsection{Das Bild-Modell}

\begin{itemize}
    \item
	Die Level-Set-Methode
	\begin{math}
	    E[u] = \int_{[0,1]} e[\Gamma_\lambda] \di[\lambda]
	    = \int_{[0,1]} \int_{\Gamma_\Lambda} \alpha + \beta \kappa^2 \di[s] \di[\lambda]
	\end{math}
    \item
	Co-Area Formel
    \item
	Euler-Elastica Bild-Modell:
	\begin{math}
	    E[u] = \int_{\Omega} (\alpha + \beta \kappa^2) |\nabla u| \di[x]
	\end{math}
	für $\kappa = \nabla \cdot (\frac{\nabla u}{|\nabla u|})$.
\end{itemize}

\subsection{Das Inpainting-Modell}

\begin{itemize}
    \item
	Daten-Modell, $u^0|_{\Omega\setminus D} = (u + n)|_{\Omega\setminus D}$
    \item
	Für Gauß-Rauschen $n$:
	\begin{math}
	    E[u^0 \st u, D] = \frac{\eta}{2} \int_{\Omega \setminus D} |u - u^0|^2 \di[x]
	\end{math}
    \item
	Inpainting-Modell:
	\begin{math}
	    E[u \st u^0, D]
	    = \int_\Omega (\alpha + \beta \kappa^2) |\nabla u| \di[x] + \frac{\eta}{2} \int_{\Omega\setminus D} |u - u^0|^2 \di[x]
	\end{math}
\end{itemize}

\section{Der ADMM-Algorithmus}

\subsection{Augmented Lagrange und ADMM im Allgemeinen}

\subsection{Euler-Elastica-Inpainting ADMM}

\begin{itemize}
    \item
	Operator-Splitting:
	\begin{math}
	    &\min_{v,u,m,p,n} \int_{\Omega} (\alpha + \beta(\nabla \cdot n)^2) |p| + \frac{\eta}{2} \int_{\Omega\setminus D} |v - u^0|^2 \\
	    &\quad\mathrm{s.t.}\quad v = u, p = \nabla u, n = m, |p| = m \cdot p, |m| \le 1.
	\end{math}
    \item
	Augmented Lagrange Funktional:
	\begin{math}
	    \scr L[v,u,m,p,n;\lambda_1,\lambda_2,\lambda_3,\lambda_4]
	    &= \int_{\Omega} (\alpha + \beta(\nabla \cdot n)^2) |p| + \frac{\eta}{2} \int_{\Omega\setminus\Gamma} |v - u^0|^2 \\
	    &\quad + r_1 \int_\Omega (|p| - m\cdot p) + \int_\Omega \lambda_1 (|p| - m \cdot p) \\
	    &\quad + \frac{r_2}{2} \int_\Omega |p - \nabla u|^2 + \int_\Omega \lambda_2 \cdot (p - \nabla u) \\
	    &\quad + \frac{r_3}{2} \int_\Omega (v - u)^2 + \int_\Omega \lambda_3 (v - u) \\
	    &\quad + \frac{r_4}{2} \int_\Omega |n-m|^2 + \int_\Omega \lambda_4 \cdot (n - m) + \delta_{\ge 1}(m).
	\end{math}
    \item
	Updates:
	\begin{math}
	    \lambda_1 &\gets \lambda_1 + r_1 (|p| - m\cdot p), \\
	    \lambda_2 &\gets \lambda_2 + r_2 (p - \nabla u), \\
	    \lambda_3 &\gets \lambda_3 + r_3 (v - u), \\
	    \lambda_4 &\gets \lambda_4 + r_4 (n - m).
	\end{math}
\end{itemize}

\pagebreak
\subsection{EE-ADMM Unterprobleme}

%\begin{math}
%    \scr E_1[v]
%    &= \frac{\eta}{2} \int_{\Omega\setminus D} |v - u^0|^2 + \frac{r_3}{2} \int_\Omega (v-u)^2 + \int_\Omega \lambda_3(v - u)\\
%    \scr E_2[u]
%    &= \frac{r_2}{2} \int_\Omega |p - \nabla u|^2 + \int_\Omega \lambda_2 \cdot (p - \nabla u) + \frac{r_3}{2} \int_\Omega (v-u)^2 + \int_\Omega \lambda_3 (v-u) \\
%    \scr E_3[m]
%    &= r_1 \int_\Omega(|p| - m\cdot p) + \int_\Omega \lambda_1 (|p| - m \cdot p) + \frac{r_4}{2} \int_\Omega |n-m|^2 + \int_\Omega \lambda_4 \cdot (n-m) + \delta_{\ge 1}(m) \\
%    &= \frac{r_4}{2} \int_\Omega |x-m|^2 + \delta_{\ge 1}(m) + \const\\
%    \scr E_4[p]
%    &= \int_\Omega (\alpha + \beta(\nabla \cdot n)^2) |p| + r_1 \int_\Omega (|p| - m\cdot p) + \int_\Omega \lambda_1 (|p| - m\cdot p) \\
%    &\qquad + \frac{r_2}{2} \int_\Omega |p - \nabla u|^2 + \int_\Omega \lambda_2 \cdot (p - \nabla u) \\
%    &= \int_\Omega \Big(\alpha + \beta (\nabla \cdot n)^2 + r_1 + \lambda_1\Big) |p| + \frac{r_2}{2} \int_\Omega \Big| p - \big( \nabla u + \frac{r_1 + \lambda_1}{r_2} m - \frac{1}{r_2} \lambda_2 \big) \Big|^2 + \const\\
%    \scr E_5[n]
%    &= \int_\Omega (\alpha + \beta(\nabla \cdot n)^2 ) |p| + \frac{r_4}{2} \int_\Omega |n-m|^2 + \int_\Omega \lambda_4 \cdot (n - m)
%\end{math}

\begin{enumerate}[1)]
    \item
	Minimiere
	\begin{math}
	    \scr E_1[v]
	    = \frac{\eta}{2} \int_{\Omega\setminus D} |v - u^0|^2 + \frac{r_3}{2} \int_\Omega (v-u)^2 + \int_\Omega \lambda_3(v - u).
	\end{math}
	Variationsrechnung liefert
	\begin{math}
	    0 &= \ddx[\eps] \scr E_1[u + \eps \phi] \\
	    &= \eta \int_{\Omega \setminus D} (v - u^0) \phi + r_3 \int_\Omega (v - u) \phi + \int_\Omega \lambda_3 \phi \\
	    &= \int_\Omega \Big(\eta (v - u^0)\Ind_{\Omega \setminus D}  + r_3 (v - u) + \lambda_3 \Big) \phi,
	\end{math}
	also nach dem Hauptsatz der Variationsrechnung
	\begin{math}
	    v = \begin{cases}
	        u - \frac{\lambda_3}{r_3} & \text{auf $D$}, \\
		\frac{\eta u^0 + r_3 u - \lambda_3}{\eta + r_3} & \text{auf $\Omega\setminus D$}.
	    \end{cases}
	\end{math}
    \item
	Minimiere
	\begin{math}
	    \scr E_2[u]
	    = \frac{r_2}{2} \int_\Omega |p - \nabla u|^2 + \int_\Omega \lambda_2 \cdot (p - \nabla u) + \frac{r_3}{2} \int_\Omega (v-u)^2 + \int_\Omega \lambda_3 (v-u).
	\end{math}
	Variationsrechnung liefert
	\begin{math}
	    0 &= \ddx[\eps] \scr E_2[u + \eps \phi] \\
	    &= r_2 \int_\Omega (\nabla u - p) \cdot \nabla \phi - \int_\Omega \lambda_2 \cdot \nabla \phi + r_3 \int_\Omega (u - v) \phi - \int_\Omega \lambda_3 \phi \\
	    &= \int_\Omega (r_2 \nabla u - r_2 p - \lambda_2) \cdot \nabla \phi + \int_\Omega (r_3 u - r_3 v - \lambda_3) \phi,
	\end{math}
	als schwache Form einer linearen PDE zweiter Ordnung.
	%Euler-Lagrange liefert PDE:
	%\begin{math}
	%    -r_2 \Laplace u + r_3 u = - r_2 \nabla \cdot p - \nabla \cdot \lambda_2 + r_3 v + \lambda_3.
	%\end{math}
    \item
	Minimiere
	\begin{math}
	    \scr E_3[m]
	    &= r_1 \int_\Omega(|p| - m\cdot p) + \int_\Omega \lambda_1 (|p| - m \cdot p) + \frac{r_4}{2} \int_\Omega |n-m|^2 + \int_\Omega \lambda_4 \cdot (n-m) + \delta_{\ge 1}(m) \\
	    &= \frac{r_4}{2} \int_\Omega |x-m|^2 + \delta_{\ge 1}(m) + \const,
	\end{math}
	wobei $x = \frac{(r_1 + \lambda_1)p + \lambda_4}{r_4} + n$.

	Definiere punktweise
	\begin{math}
	    m^* := \begin{cases}
	        x & \text{für $|x| < 1$}, \\
		\frac{x}{|x|} & \text{für $|x| \ge 1$}.
	    \end{cases}
	\end{math}
	$m^*$ minimiert $\scr E_3$.
	Betrachte dazu $\scr E_3[m^* + \phi]$.
	Wir können ohne Einschränkung fordern, dass $|m^* + \phi| \le 1$ auf $\Omega \setminus N$ für eine Nullmenge $N$, (sonst trivialerweise $\infty = \scr E_3[m^* + \phi] \ge \scr E_3[m^*]$).
	Setze
	\begin{math}
	    M := \Set{x \in \Omega & |x| \ge 1} \setminus N.
	\end{math}
	Dann ist auf $M$
	\begin{math}
	    1 \ge |m^* + \phi|^2 = \l| \frac{x}{|x|} \r|^2 + 2 \< \frac{x}{|x|}, \phi \> + |\phi|^2,
	\end{math}
	also $-2\<\frac{x}{|x|}, \phi\> \ge |\phi|^2$.

	Damit ergibt sich schließlich
	\begin{math}
	    \scr E[m^* + \phi]
	    &= \int_M | \frac{x}{|x|} - x + \phi|^2 \\
	    &= \int_M | \frac{x}{|x|} - x|^2 + 2 \int_M \<\frac{x}{|x|} - x, \phi\> + \int_M |\phi|^2 \\
	    &= \scr E[m^*] + \int_M \underbrace{(|x| - 1)}_{\ge 0} \underbrace{(-2\<\frac{x}{|x|}, \phi\>)}_{\ge |\phi|^2} + \|\phi\|_{L^2} \\
	    &= \scr E[m^*] + \|\phi\|_{L^2}.
	\end{math}
	Also ist $m^*$ ein Minimierer von $\scr E_3$ in $L^2$.

	%Explizite Lösung (Lemma):
	%\begin{math}
	%    m = \operatorname{proj}_{\le 1}(x)
	%    = \operatorname{proj}_{\le 1}\Big( \frac{(r_1 + \lambda_1)p + \lambda_4}{r_4} + n \Big)
	%\end{math}
    \item
	Minimiere
	\begin{math}
	    \scr E_4[p]
	    &= \int_\Omega (\alpha + \beta(\nabla \cdot n)^2) |p| + r_1 \int_\Omega (|p| - m\cdot p) + \int_\Omega \lambda_1 (|p| - m\cdot p) \\
	    &\qquad + \frac{r_2}{2} \int_\Omega |p - \nabla u|^2 + \int_\Omega \lambda_2 \cdot (p - \nabla u) \\
	    &= \int_\Omega \Big(\alpha + \beta (\nabla \cdot n)^2 + r_1 + \lambda_1\Big) |p| + \frac{r_2}{2} \int_\Omega \Big| p - \big( \nabla u + \frac{r_1 + \lambda_1}{r_2} m - \frac{1}{r_2} \lambda_2 \big) \Big|^2 + \const \\
	    &= \int_\Omega c |p|_2 + \frac{r_2}{2} \int_\Omega |p - q|^2 + \const
	\end{math}
	Es gilt $c, r_2 > 0$ (siehe auch Lagrange-Update für $\lambda_1$).
	Variationsrechnung liefert für $|p| \neq 0$
	\begin{math}
	    0 &= \ddx[\eps] \scr E_4[p + \eps\phi] \\
	    &= \int_\Omega c \frac{p}{|p|} \cdot \phi + r_2 \int_\Omega (p - q) \cdot \phi.
	\end{math}
	Nach dem Hauptsatz der Variationsrechnung
	\begin{math}
	    (\frac{c}{|p|} + r_2) p = r_2 q.
	\end{math}
	Setze an $p := \lambda q$ mit $\lambda > 0$ also
	\begin{math}
	    \lambda = 1 - \frac{c}{r_2 |q|}
	\end{math}
	Der Minimierer ist damit gegeben durch
	\begin{math}
	    p := \max\Set{0, 1 - \frac{c}{r_2 |q|}} q
	\end{math}
	%Explizite Lösung (Lemma: soft thresholding \dots):
	%\begin{math}
	%    p = \max\Set{0, 1 - \frac{\alpha + \beta(\nabla \cdot n)^2 + r_1 + \lambda_1}{r_2 |q|}} q
	%\end{math}
    \item
	Minimiere
	\begin{math}
	    \scr E_5[n]
	    &= \int_\Omega (\alpha + \beta(\nabla \cdot n)^2 ) |p| + \frac{r_4}{2} \int_\Omega |n-m|^2 + \int_\Omega \lambda_4 \cdot (n - m)
	\end{math}
	Variationsrechnung liefert
	\begin{math}
	    0 &= \ddx[\eps] \scr E_5[n + \eps \phi] |_{\eps = 0} \\
	    &= \int_\Omega 2\beta |p| (\nabla \cdot n) (\nabla \cdot \phi) + \int_\Omega (r_4 n - r_4 m + \lambda_4) \cdot \phi
	\end{math}
	%Euler-Lagrange liefert PDE-System:
	%\begin{math}
	%    -2 \nabla (\beta |p| \nabla \cdot n) + r_4 (n - m) + \lambda_4 = 0
	%\end{math}
\end{enumerate}

\section{Diskretisierung mit Dune-ACFem und Inpainting-Resultate}

\begin{itemize}
    \item
	Gitterwahl: structured (simplex/kubisch) oder unstructured (simplex, guiding durch Kantendetektor)
    \item
	Lineare Lagrange Basis-Funktionen als nahezu einzige Wahl
    \item
	Adaptive Strategien (z.B. refine/coarse gemäß $n = \nabla u$), Vergleich
    \item
	Parameter-Tweaking ($\alpha$, $\beta$, $\eta$, $r_1, r_2, r_3, r_4$) und -Interpretation
    \item
	Startwert-Tweaking ($v, u, m, p, n, \lambda_1, \lambda_2, \lambda_3, \lambda_4$)
\end{itemize}

\section{Weitere Anwendungen}

\begin{itemize}
    \item
	Debluring
    \item
	Upscaling
\end{itemize}


\section{Appendix}

\subsection{Co-Area-Formel}
\subsection{Euler-Lagrange}
\subsection{Soft-Thresholding?}


\end{document}

