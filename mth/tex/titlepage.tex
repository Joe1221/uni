%\usepackage{titling}

\begin{titlepage}
  \begin{center}
    ~\par\vspace{4em}
    {
      \fontsize { 16pt } { 16pt } \selectfont
      Masterarbeit
    }
    \par\vspace{3em}
    {
      \fontsize { 24pt } { 24pt } \selectfont \sffamily \bfseries
      \thetitle
    }
    \par\vspace{3em}
    {
      \fontsize { 16pt } { 16pt } \selectfont \scshape
      \theauthor
    }
    \par\vspace{1.5em}
    {
      \fontsize { 14pt } { 14pt } \selectfont %\scshape
      \today
    }
    \par\vspace{4.5em}
    {
    }
    \par\vspace{8em}
    {
      \fontsize { 14pt } { 14pt } \selectfont \scshape
      Universität Stuttgart
    }
    \par\vspace{1em}
    {
      \fontsize { 14pt } { 14pt } \selectfont %\scshape
      Institut für Angewandte Analysis und Numerische Simulation
    }
    \par\vspace{1em}
    {
      \fontsize { 14pt } { 14pt } \selectfont %\scshape
      Betreuer:
      Dr. Claus J. Heine,
      Dr. Andreas Langer
    }
  \end{center}
\end{titlepage}

\chapter*{Zusammenfassung}

Die Zielsetzung, fehlende Teile eines Bildes zu rekonstruieren – auch „Inpainting“ genannt – lässt sich als Minimierung eines
Funktionals für das Gesamtbild modellieren, bestimmt durch ein Datenmodell, das die Übereinstimmung mit dem ursprünglichen Bild auf dem bekannten Gebiet kontrolliert, und einem Bildmodell, welches maßgebend für die Güte der Rekonstruktion ist.

Das Euler Elastica Bildmodell, welches die Niveaulinien eines Bildes nach dem Vorbild elastischer Stäbe modelliert, bietet ein vielversprechendes Bildmodell und kommt in dieser Arbeit zum Einsatz.
Für die numerische Minimierung wird eine bekannte “alternating direction“ Augmented Lagrange Methode angewandt und die entstehenden Teilprobleme erstmalig im Kontext der Finiten Elemente gelöst.


{
  \let\clearpage\relax
  \tableofcontents
  %\addtocentrydefault{chapter}{}{Inhaltsverzeichnis}
}
%\tableofcontents

