%\usepackage{titling}

\begin{titlepage}
  \begin{center}
    ~\par\vspace{4em}
    {
      \fontsize { 16pt } { 16pt } \selectfont
      Masterarbeit
    }
    \par\vspace{3em}
    {
      \fontsize { 24pt } { 24pt } \selectfont \sffamily \bfseries
      \thetitle
    }
    \par\vspace{3em}
    {
      \fontsize { 16pt } { 16pt } \selectfont \scshape
      \theauthor
    }
    \par\vspace{1.5em}
    {
      \fontsize { 14pt } { 14pt } \selectfont %\scshape
      \today
    }
    \par\vspace{4.5em}
    {
    }
    \par\vspace{8em}
    {
      \fontsize { 14pt } { 14pt } \selectfont \scshape
      Universität Stuttgart
    }
    \par\vspace{1em}
    {
      \fontsize { 14pt } { 14pt } \selectfont %\scshape
      Institut für Angewandte Analysis und Numerische Simulation
    }
    \par\vspace{1em}
    {
      \fontsize { 14pt } { 14pt } \selectfont %\scshape
      Betreuer:
      Dr. Claus J. Heine,
      Dr. Andreas Langer
    }
  \end{center}
\end{titlepage}

\chapter*{Zusammenfassung}

Die Rekonstruktion fehlender Teile eines Bildes, auch „Inpainting” genannt, lässt sich als Minimierung eines
Energiefunktionals für das Gesamtbild modellieren, formuliert mit einem Datenmodell, das die Übereinstimmung mit dem ursprünglichen Bild auf dem bekannten Gebiet kontrolliert, und einem Bildmodell, welches maßgebend für die Güte der Rekonstruktion ist.

Die Euler Elastica Energie, welche die Niveaulinien des Bildes nach dem Vorbild elastischer Stäbe modelliert, bietet ein vielversprechendes Bildmodell und wird in dieser Arbeit angewendet.
Für die numerische Minimierung wird eine bekannte “alternating direction augmented lagrange” Methode eingesetzt und im neuen Setting der Finiten Elemente implementiert.


{
  \let\clearpage\relax
  \tableofcontents
  %\addtocentrydefault{chapter}{}{Inhaltsverzeichnis}
}
%\tableofcontents

