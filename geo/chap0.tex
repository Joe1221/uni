\chapter{Einführung}


\coursetimestamp{08}{04}{2014}

Es gibt verschiedene Arten der Geometrie

\begin{itemize}
	\item
		Inzidenzgeometrie: Punkte, Geraden, Ebenen.
	\item
		Darstellende Geometrie: Projektionen des Anschauungsraumes in die Zeichenebene.
	\item
		Analytische Geometrie über verschiedene Vektorräume:
		\begin{itemize}
			\item
				affin
			\item
				euklidisch
			\item
				projektiv
		\end{itemize}
	\item
		Konvexgeometrie: Konvexe Körper im euklidischen Raum.
	\item
		Polyedergeometrie: Polyeder im euklidischen Raum.
	\item
		Algebraische Geometrie: Nullstellenmengen algebraischer Funktionen.
	\item
		Differentialgeometrie: differenzierbare Kurven, differenzierbare Flächen.
\end{itemize}
Geometrie entsteht also durch Zusammentreffen verschiediener Strukturen (algebraischer, analytischer, topologischer, kombinatorischer, \dots Art).

Im Laufe dieser Vorlesung konzentrieren wir uns auf die Analytische Geometrie und die Differentialgeometrie.


