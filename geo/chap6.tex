\chapter{Die innere Geometrie von Flächen}


Wir stellen uns „Flachländer“ (2-dim. Lebewesen), die in ihrer 2-dimensionalen Welt leben und können Längen und Winkel messen, d.h. sie kennen die (abstrakte) erste Fundamentalform.

Betrachte die Länge eines Abstandskreises in einem festen Punkt mit Radius $r$:
\[
	L = \begin{cases}
		2\pi \sin r & \text{in $S^2$} \\
		2\pi r & \text{in $E^2$} \\
		2\pi \sinh r & \text{in $H^2$}
	\end{cases}
\]
Welche geometrischen Größen hängen nur von der ersten Fundamentalform ab (und können von den Flachländern erkannt werden)?
Hauptkrümmungen und mittlere Krümmung sicher nicht:
Betrachte dazu Ebene und Zylinder, beide haben erste Fundamentalform $\Matrix{1 & 0 \\ 0 & 1}$.
Erstaunlicherweise sehen wir später, dass die Gauß-Krümmung jedoch aus der ersten Fundamentalform berechnet werden kann.

Welche Größen sind hinreichend, um ein Flächenstück eindeutig (bis auf Kongruenz) festzulegen?
Kongruent impliziert isometrisch, aber isometrisch impliziert \emph{nicht} kongruent.

Richtungsableitungen von skalaren Funktionen sind stets eindeutig und \emph{innerhalb} der Fläche berechenbar, aber Ableitungen von Vektorfeldern?

\[
	D_X f|_p = X(f)|_p
	= \lim_{t\to 0} \f 1t (f(c(t)) - f(p)),
\]
wenn $c(t)$ differenzierbar, $c(0) = p, \dot c(0) = X$.
Im $\R^n$ gilt für ein Vektorfeld $Y$.
\[
	D_X Y = D_X(\sum_{i} y_i E_i)
	= \sum_{i} (D_X y_i) E_i
\]

