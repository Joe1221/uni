\chapter{Die innere Geometrie von Flächen}


Wir stellen uns „Flachländer“ (2-dim. Lebewesen), die in ihrer 2-dimensionalen Welt leben und können Längen und Winkel messen, d.h. sie kennen die (abstrakte) erste Fundamentalform.

Betrachte die Länge eines Abstandskreises in einem festen Punkt mit Radius $r$:
\[
	L = \begin{cases}
		2\pi \sin r & \text{in $S^2$} \\
		2\pi r & \text{in $E^2$} \\
		2\pi \sinh r & \text{in $H^2$}
	\end{cases}
\]
Welche geometrischen Größen hängen nur von der ersten Fundamentalform ab (und können von den Flachländern erkannt werden)?
Hauptkrümmungen und mittlere Krümmung sicher nicht:
Betrachte dazu Ebene und Zylinder, beide haben erste Fundamentalform $\Matrix{1 & 0 \\ 0 & 1}$.
Erstaunlicherweise sehen wir später, dass die Gauß-Krümmung jedoch aus der ersten Fundamentalform berechnet werden kann.

Welche Größen sind hinreichend, um ein Flächenstück eindeutig (bis auf Kongruenz) festzulegen?
Kongruent impliziert isometrisch, aber isometrisch impliziert \emph{nicht} kongruent.

Richtungsableitungen von skalaren Funktionen sind stets eindeutig und \emph{innerhalb} der Fläche berechenbar, aber Ableitungen von Vektorfeldern?

\[
	D_X f|_p = X(f)|_p
	= \lim_{t\to 0} \f 1t (f(c(t)) - f(p)),
\]
wenn $c(t)$ differenzierbar, $c(0) = p, \dot c(0) = X$.
Im $\R^n$ gilt für ein Vektorfeld $Y$.
\[
	D_X Y = D_X(\sum_{i} y_i E_i)
	= \sum_{i} (D_X y_i) E_i
\]

\coursetimestamp{24}{06}{2014}

Wir verwenden die folgenden Konventionen
\[
	(u^1, \dotsc, u^n) \in U,
	(x^1, \dotsc, x^{n+1}) \in \R^{n+1}
\]
Der Grund ist die Summenkonvention
\[
	h_{ij} = \sum_{k} h_i^k g_{kj}
\]

In $\R^{n+1}$ gibt es die Richtungsableitung
\[
	D_xY|_p = DY|_p(X)
	= \lim_{t\to 0} \f 1t (Y(p+tx) - Y(p))
	= \lim_{t\to 0} \f 1t (Y(c(t)) - Y(p))
\]
für $c(0) = p, \dot c(0) = X$.
Die Begründung für die letzte Gleichung ist
\[
	\ddx[t](Y \circ c)(t) \big|_{t=0}
	= \Df[Y]\big|_{c(0)} (\ddx[t]{c}|_{t=0})
	= \Df[Y]\big|_{p}(U)
	= D_X Y|_p
\]

\begin{df}
	Sei $f: U \to \R^{n+1}$, $Y$ sei differenzierbares Vektorfeld längs $f: U \ni u \mapsto Y(u)$, $X \in T_u f$ ein fester Tangentialvektor, $p = f(x)$.
	Dann ist die Richtungsableitung $D_x Y|_p$ definiert durch
	\[
		D_x Y|_p := \lim_{t\to  0} \f 1t \Big( Y(\underbrace{u + t(\Df)^{-1}(X)}_{\gamma(t)}) - Y(u) \Big),
	\]
	wobei $X = \Df(\xi)$ und $c(t) = f(u + t\xi)$ Kurve in der Hyperfläche ist.

	Es gilt $c(0) = p, \dot c(0) = \Df(\xi) = X$ und speziell
	\[
		D_{\f {\partial f}{\partial u^i}} Y
		= \lim_{t\to 0} \f 1t \Big(Y(u^1, \dotsc, u^i + t, \dotsc, u^n) - Y(u^1, \dotsc, u^n))
		= \f {\partial Y}{\partial u^i}.
	\]
	und
	\[
		D_{\f{\partial f}{\partial u^i}} \f{\partial f}{\partial u} = \f{\partial^2 f}{\partial u^i \partial u^j}.
	\]
\end{df}

\begin{df}[kovariante Ableitung]
	Seien $X \in T_u f, Y$ ($Y$ tangential an $f$) wie oben.
	Dann heißt
	\[
		\nabla_X Y|_p := (D_x Y|_p)^{\text{Tangentialanteil}}
		= D_X Y|_p - \<D_X|_p Y, \nu_p \> \nu_p
	\]
	die \emphdef{kovariante Ableitung} von $Y$ in Richtung $X$.
	Dann ist $\nabla_X Y |_p \in T_u f$.
	\[
		(X_p, Y) \mapsto \nabla_X Y|_p \in T_u f
	\]
	$Y$ ist definiert längs einer Kurve $c$ mit $c(0) = p, \dot c(0) = X_p$.
\end{df}

Beachte $\<D_X Y, \nu\> = - \< Y, D_x \nu\> = Ⅱ(X,Y)$, da $\< Y, \nu\> = 0$.
Also
\begin{align*}
	\nabla_X Y|_p &= D_X Y|_p - Ⅱ(X,Y) \nu_p \\
	D_x Y|_p &= \underbrace{\nabla_X Y|_p}_{\in T_u f} + \underbrace{Ⅱ(X,Y) \nu_p}_{\in \Orthspace_u f}.
\end{align*}
Es gelten folgende Rechenregeln für $D$ und $\nabla$.
\begin{itemize}
	\item
		Linearität
	\item
		Additivität
	\item
		Produktregel
	\item
		Verträglichkeit mit dem Skalarprodukt
\end{itemize}

\begin{st}
	Die kovariante Ableitung hängt nur von der ersten Fundamentalform ab (ist also eine Größe der inneren Geometrie).
	\begin{proof}
		Es gilt $X = \sum_i X^i \pddx*[u^i]{f}$, $Y = \sum_j \pddx*[u^j]{f}$ und damit
		\begin{align*}
			\nabla_X Y &= \nabla_{\sum_i X^i \pddx*[u^i]{f}} (\sum_j Y^j \pddx*[u^j]{f}) \\
			&= \sum_i X^i \nabla_{\pddx[u^i]{f}} (\sum_j Y^j \pddx*[u^j]{f}) \\
			&= \sum_{i,j} X^i \Big( \pddx*[u^i]{Y^j} \pddx*[u^j]{f} + Y^j \nabla_{\pddx*[u^i]{f}} \pddx*[u^j]{f} \Big)
		\end{align*}
		$\nabla_{\pddx*[u^i]{f}} \pddx*[u^j]{f}$ ist der Tangentialanteil von $D_{\pddx*[u^i]{f}} \pddx*[u^j]{f} = \f{\partial^2 f}{\partial u^i \partial u^j}$ und eindeutig bestimmt durch alle
		\[
			\Gamma_{ij,k} := \< \f{\partial^2 f}{\partial u^i \partial u^j}, \pddx*[u^k]{f} \>.
		\]
		$\Gamma_{ij,k}$ ist symmetrisch in $i,j$, falls $f \in C^2$.
		Es gilt
		\[
			\pddx*[u^k] \< \pddx*[u^i]{f}, \pddx*[u^j]{f} \>
		= \< \f{\partial^2 f}{\partial u^k \partial u^i}, \pddx*[u^j]{f} \>
			+ \< \pddx*[u^i]{f}, \f{\partial^2 f}{\partial u^k \partial u^j} \>
			= \Gamma_{ki,j} + \Gamma_{kj,i},
		\]
		also
		\begin{align*}
			(g_{ij})_k &= \Gamma_{ik,j} + \Gamma_{jk,i} \\
			(g_{jk})_i &= \Gamma_{ji,k} + \Gamma_{ki,j} \\
			(g_{ki})_j &= \Gamma_{kj,i} + \Gamma_{ij,k}
		\end{align*}
		und damit
		\[
			2 \Gamma_{ij,k} = - (g_{ij})_k + (g_{jk})_i + (g_{ki})_j,
		\]
		hängt also nur von der ersten Fundamentalform ab.
	\end{proof}
\end{st}

\begin{df}
	Der Ausdruck
	\[
		\Gamma_{ij,k} = \f 12 \Big( - (g_{ij})_k + (g_{jk})_i + (g_{ki})_j \Big)
	\]
	heißt \emphdef{Christophersymbol erster Art}.
	Die $\Gamma_{ij}^k$ mit
	\[
		\nabla_{\pddx*[u^i]{f}} \pddx*[u^j]{f}
		= \sum_l \Gamma_{ij}^l \pddx*[u^l]{f}
	\]
	heißen \emphdef{Christophersymbole zweiter Art}.

	Dabei gilt $\Gamma_{ij,k} = \Gamma_{ji,k}, \Gamma_{ij}^k = \Gamma_{ji}^k$, $\Gamma_{ij,k} = \sum_l \Gamma_{ij}^l g_{lk}$ und $\Gamma_{ij}^k = \sum_l \Gamma_{ij,l} g^{lk}$.

	Die kovariante Ableitung $\nabla_X Y$ mit $X = \sum_i X^i \pddx*[u^i]{f}, Y = \sum_j Y^j \pddx*[u^j]{f}$ ist
	\[
		\nabla_X Y|_p = \sum_{ij} X^i \Big( \pddx*[u^i]{Y^k} + \sum_j Y^j \Gamma_{ij}^k \Big) \pddx*[u^k]{f}
	\]
	Im euklidischen Raum sind alle $\Gamma_{ij,k} = 0$.

	Wir schreiben
	\[
		(\nabla_i Y)^k
		= \pddx*[u^i]{Y^k} + \Gamma_{ij}^k Y^j
	\]
\end{df}
