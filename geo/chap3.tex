\chapter{Globale Kurventheorie}



Wir beschäftigen uns in diesem Kapitel im wesentlichen mit Eigenschaften geschlossener Kurven.
Wie unterscheiden sich ein Kreis und eine 8-Kurve?

\begin{df}
	Eine Kurve $c: [a,b] \to \R^n$ heißt \emphdef[Kurve!geschlossen]{geschlossen}, wenn es ein $\tilde c: \R \to \R^n$ gibt mit $\tilde c \big|_{[a,b]} = c$ und $\tilde c(t+b-a) = \tilde c(t)$ für alle $t \in \R$.
	Insbesondere ist dann $c(b) = c(a), c'(b) = c'(a)$.

	$c$ heißt \emphdef[Kurve!einfach geschlossen]{einfach geschlossen}, wenn zusätzlich $c|_{[a,b)}$ injektiv ist.
\end{df}

\begin{df}
	Sei $c: [a,b] \to \R^n$ eine Kurve mit Bogenlänge $L$.
	Die \emphdef{Totalkrümmung} ist
	\[
		\int_0^L \kappa(s) \dx[s]
		= \int_a^b \kappa(t) \|\dot c(t)\| \dx[t]
	\]
\end{df}

\begin{ex}
	Für den Einheitskreis linksherum ist die Totalkrümmung $2\pi$, rechtsherum $-2\pi$.
\end{ex}

Lokal kann man im $\R^2$ schreiben
\[
	e_1(t) = (\cos \phi(t), \sin \phi(t))
\]
und
\[
	\kappa e_2(t)
	= e_1'(t)
	= \dot \phi (-\sin \phi(t), \cos \phi(t)) \ddx[s]{t}
\]
also
\[
	\kappa = \dot \phi \ddx[s]{t} = \ddx[s]{\phi} = \phi'
\]

\begin{kor}
	Solange $\phi$ als differenzierbare Funktion erklärt ist, folgt
	\[
		\int_a^b \kappa(t) \|\dot c(t)\| \dx[t]
		= \int_a^b \phi'(t) \dx[t]
		= \phi(b) - \phi(a)
	\]
\end{kor}

Ist $c$ eine geschlossene Kurve und $\phi(t)$ differenzierbar, dann ist wegen $c(b) = c(a)$ auch
\[
	\phi(b) - \phi(a) = 2\pi k
\]
mit $k \in \Z$.

\begin{lem}
	Sei $\gamma: [a,b] \to \R^2 \setminus \{0\}$ stetig.
	Dann gibt es eine stetige Funktion $\phi: [a,b] \to \R$ mit
	\[
		\gamma(t) = \| \gamma(t) \| (\cos \phi(t), \sin \phi(t))
	\]
	Falls $\gamma$ geschlossen ist, so ist
	\[
		W_\gamma := \f 1{2\pi} (\phi(b) - \phi(a)) \in \Z
	\]
	die \emphdef{Windungszahl} von $\gamma$.
	\begin{proof}
		Lokal ist $\phi$ eindeutig bis auf $2\pi k$ mit $k \in \Z$.
		In jeder Halbebene $\{ x : \<x, x_0\> > 0 \}$ ist $\phi$ eindeutig, wenn ein einzelner Wert fest liegt.

		Wähle eine Unterteilung $a = t_0 < t_1 < \dotsb < t_n = b$, so dass $\gamma|_{[t_i, t_{i+1}]}$ stets in einer Halbebene enthalten ist (gleichmäßige Stetigkeit).
		Für festgelegtes $\phi(a)$ bestimmt $\phi(t_{i})$ sukzessive eindeutig $\phi(t_{i+1})$, also existiert solch ein $\phi(t)$.

		Seien $\phi, \tilde \phi$ zwei solche Funktionen, dann ist $\tilde \phi(t) - \phi(t) \in 2\pi\Z$, insbesondere mit $\tilde \phi(a) - \phi(a) = 2\pi k, \tilde \phi(b) - \phi(b) = 2 \pi k'$ ist
		\[
			\phi(b) - \phi(a) = \tilde \phi(b) - \tilde \phi(a).
		\]
	\end{proof}
\end{lem}

\begin{df}
	Sei $c$ eine reguläre, geschlossene Kurve.
	Dann heißt die Windungszahl der Tangente, $W_{\dot c}$, \emphdef{Umlaufzahl}, oder \emphdef{Rotationsindex} $U_c$ von $c$.
\end{df}

Wir vermuten: $U_c = \pm 1$, falls $c$ einfach geschlossen ist.


\coursetimestamp{08}{05}{2014}

Sei $\gamma$ eine geschlossene stetige Kurve in $\R^2 \setminus \{0\}$.
Dann existiert eine stetige Funktion $\phi$ mit
\[
	\gamma(t) = \| \underbrace{\gamma(t)}_{r(t)} \| ( \cos \phi(t), \sin \phi(t) )
\]
mit $r:[a,b] \to \R^2 \setminus \{0\}$.


\begin{prop}
	Es gilt
	\[
		U_c = \f 1{2\pi} \oint_{\gamma} \kappa(s) \dx[s],
	\]
	„Totalkrümmung“ der Kurve.
	\begin{proof}
		Für $c: [a,b] \to \R^2$ ist
		\begin{align*}
			\oint \kappa(s)\dx[s]
			&= \int_a^b \kappa(t) \|\dot c(t)\| \dx[t] \\
			&= \int_a^b \dot \phi(t) \ddx[s]{t} \|\dot c(t)\| \dx[t]
			= \int_a^b \dot \phi(t) \dx[t]
			= \phi(b) - \phi(a)
		\end{align*}
	\end{proof}
\end{prop}

\paragraph{Topologische Interpretation}

Eine Homotopie zwischen $c_0$ und $c_1$ ist eine stetige Abbildung $H(s, t)$ mit $H_t(s) := H(s, t)$ und $H_0 = c_0, H_1 = c_1$.
Homotopie ist eine Äquivalenzrelation.
Die Windungszahl klassifiziert die Homotopieklassen geschlossener Kurven in $\R^2 \setminus \{0\}$.

Die Umlaufzahl klassifiziert die regulären Homotopieklassen geschlossener Kurven regulärer Kurven in $\R^2$ (Whitney-Graustein, 1937).
Dabei ist eine Homotopieklasse \emphdef[Homotopieklasse!regulär]{regulär}, wenn alle $H_t$ regulär sind.
Tritt auch in einer Dissertation von Werner Boy (Göttingen 1901) auf.

Wir wollen zeigen, dass die Umlaufzahl einer \emph{einfach} geschlossenen Kurve gleich $\pm 1$ ist.

\begin{lem}
	Sei $A \subset \R^2$ sternförmig bezüglich $x_0$, d.h. $\vec{xx_0} \subset A$ für jedes $x \in A$.
	Sei $f: A \to \R^2 \setminus \{0\}$ stetig.
	Dann gibt es eine stetige Funktion $\phi: A \to \R$ mit
	\[
		f(x) = \underbrace{\|f(x)\|}_{=:r} \big( \cos \phi(x), \sin \phi(x) \big),
	\]
	d.h. es existieren globale Polarkoordinaten $r = \|f(x)\|, \phi$.
	\begin{proof}
		Wähle $\phi(x_0)$ fest.
		Für jedes $\_{xx_0}$ ist dann $\phi$ eindeutig stetig erklärbar (obiges Lemma).
		Also ist $\phi$ global erklärbar.
		Ist $\phi$ stetig?
		Zeige Folgenstetigkeit.

		Sei $x_n \to x$ eine konvergente Folge in $A$, zeige $\phi(x_n) \to \phi(x)$.
		„Ausschöpfung von $A$ durch Kompakta“, \oBdA sei $\{x_n, x\}$ in einem Kompaktum enthalten.
		Mit gleichmäßiger Stetigkeit existiert $\delta > 0$, so dass $f(x), f(y)$ stets in einer Halbebene liegen, wenn $|x-y| < \delta$.
		Also $|x_n - x| < \delta$ für hinreichend große $n$ und $|tx_n - tx| < \delta$ für $0 \le t \le 1$.
		$f(tx_n)$ und $f(tx)$ liegen in gemeinamer Halbebene für alle $t \le 1$.
		Also ist $|\phi(tx_1) - \phi(tx)| < \pi$.
		Damit folgt $\phi(x_n) \to \phi(x)$ für $n \to \infty$.
	\end{proof}
\end{lem}

\begin{thm}[Hopfscher Umlaufssatz, 1935]
	Sei $c: [a,b] \to \R^2$ eine einfach geschlossene reguläre Kurve.
	Dann gilt
	\[
		\f 1{2\pi} \oint \kappa(s) \dx[s]
		= \f 1{2\pi} \int_a^b \kappa(t) \|\dot c\|
		= U_c
		= \pm 1.
	\]
	\begin{proof}
		$c$ liege \oBdA auf einer Seite der Tangente durch $c(a) = c(b)$:
		$c(t) = (x(t), y(t))$ mit $y(t) \ge 0$.
		Betrachte die Sekante durch $c(a) = c(b)$ und $c(t)$.
		$\dot c(a) = (\dot x(a), 0)$

		Setze $A := \{ (s,t) \in \R^2 : a \le s \le t \le b \}$.
		Definiere $f: A \to \R^2 \setminus \{0\}$ durch
		\[
			f(s,t)
			= \begin{cases}
				\f {c(t) - c(s)}{\|c(t) - c(s)\|} & s \neq t \land (s,t) \neq (a,b) \\
				\f {\dot c(t)}{\|\dot c(t)\|} & s = t \\
				- \f {\dot c(a)}{\|\dot c(a)\|} & (s,t) = (a,b)
			\end{cases}.
		\]
		Da $c$ einfach geschlosssen, ist $c(t) \neq c(s)$ außer $c(b) = c(a)$, also ist $c$ wohldefiniert und stetig
		\begin{align*}
			\lim_{t\to s} \f{c(t) - c(s)}{\|c(t) - c(s)\|}
			&= \f{\dot c(t)}{\|\dot c(t)\|}, \\
			\lim_{\substack{t\to s \\ s\to a}} \f{c(t) - c(s)}{\|c(t) - c(s)\|}
			&= \f{\dot c(a)}{\|\dot c(a)\|}
			= \f{\dot c(b)}{\|\dot c(b)\|}.
		\end{align*}
		Nach dem vorhergehenden Lemma existiert eine stetige Polarwinkelfunktion $\phi: A \to \R$ mit $f(s, t) = (\cos \phi(s,t), \sin \phi(s,t) )$.
		Da $c$ differenzierbar, ist auch $\phi$ differenzierbar.
		Jetzt ist $\phi(t) := \phi(t, t)$ differenzierbar.
		Es gilt
		\[
			U_c
			= \f 1{2\pi} \int_a^b \kappa(t) \|\dot c(t) \| \dx[t]
			= \f 1{2\pi} \int_a^b \dot \phi(t) \dx[t]
			= \f 1{2\pi} \big( \phi(b,b) - \phi(a,a) \big).
		\]
		Außerdem
		\[
			\phi(b,b) - \phi(a,a)
			= \phi(b,b) - \phi(a,b) + \phi(a,b) - \phi(a,a),
		\]
		wobei $\phi(a,b) - \phi(a,a) = \pm \pi$, je nach $\dot x(a) \gtrless 0$.
		Ebenso für $\phi(b,b) - \phi(a,b)$.
		Damit ist $\phi(b,b) - \phi(a,a) = \pm 2\pi$.
	\end{proof}
	\begin{note}
		Der Beweis bricht offensichtlich zusammen, falls $f(t) = f(s)$, $(s,t) \neq (a,b)$.
	\end{note}
\end{thm}

\begin{kor}
	Sei $c$ einfach geschlossene reguläre Kurve, dann ist
	\[
		\oint |\kappa(s)| \dx[s]
		\ge \Big| \oint \kappa(s) \dx[s] \Big|
		= 2\pi,
	\]
	mit Gleichheit genau dann, wenn $\kappa(s)$ das Vorzeichen nicht wechselt.
\end{kor}

\begin{df}
	$c$ sei einfach geschlossen und die Bildmenge die Randkurve von $C \subset \R^2$.
	Dann heißt $c$ \emphdef{konvex}, wenn $C$ konvex ist, d.h. für $x,y \in C$ ist $\vec{x,y} \subset C$.
\end{df}

\begin{st}
	$c$ sei einfach geschlossen.
	Es sind äquivalent:
	\begin{enumerate}[1)]
		\item
			$c$ ist konvex (bzw. $C$ ist konvex)
		\item
			Jede Gerade trifft die Kurve (wenn überhaupt) in einer Strecke, einem Punkt, oder in genau 2 Punkten.
		\item
			Für jede Tangente an $c$ liegt $C$ stets ganz auf einer Seite davon.
		\item
			Die Krümmung $\kappa$ wechselt nicht das Vorzeichen ($\kappa \ge 0$, oder $\kappa \le 0$ überall).
	\end{enumerate}
	\begin{proof}
		Todo.
	\end{proof}
\end{st}

\begin{kor}
	Für jede geschlossene reguläre Kurve im $\R^2$ gilt
	\[
		\oint |\kappa(s)| \dx[s]
		\ge 2\pi,
	\]
	mit Gleicheit genau dann, wenn die Kurve einfach und konvex ist.
	\begin{proof}
		Idee: Betrachte die konvexe Hülle
	\end{proof}
\end{kor}

\coursetimestamp{13}{05}{2014}

\paragraph{Konzept der Geometrie}

\begin{itemize}
	\item
		ein Raum mit viel Struktur ($E^n, S^n, H^n$)
	\item
		studiere Teilmengen davon (auch parametrisierte Kurven oder durch Gleichungen definiert)
	\item
		Lage dieser Teilmengen im Raum
	\item
		Lage mehrerer Teilmengen zueinander
\end{itemize}
Dies versucht man mit Hilfe der Differentialrechnung, im Raum $E^n$ die „euklidische Differentialgeometrie“.
Wir unterscheiden
\begin{itemize}
	\item
		Kurven (parametrisiert durch einen Parameter)
	\item
		Flächen (parametrisiert durch zwei Parameter)
\end{itemize}



