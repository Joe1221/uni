\chapter{Spezielle Klassen von Flächen}



Man kategorisiert verschiedene Spezialfälle von Flächen:
\begin{itemize}
	\item
		Drehflächen als Drehung einer ebenen Kurve um eine Achse,
	\item
		geradelinige Flächen (man schiebe eine Gerade längs einer Kurve),
	\item
		Schiebflächen (man schiebe eine Kurve längs einer Kurve),
	\item
		Minimalflächen (Flächen mit minimaler Oberfläche, „Seifenhäute“).
\end{itemize}

\begin{df}[Drehfläche]
	Sei $t \mapsto (r(t), h(t))$ eine reguläre Kurve in der $(x,z)$-Ebene.
	Drehung um die $z$-Achse ergibt
	\[
		f(t, \phi) = \big( r(t) \cos \phi, r(t) \sin \phi, h(t) \big).
	\]
	Eine solche Fläche heißt \emphdef{Drehfläche} oder \emphdef{Rotationsfläche}.
\end{df}

Es gilt
\begin{align*}
	f_t &= (\dot r \cos \phi, \dot r \sin \phi, \dot h), &
	f_\phi &= (-r \sin \phi, r \cos \phi, 0),
\end{align*}
also
\[
	I = \Matrix*{\dot r^2 + \dot h^2 & 0 \\ 0 & r^2}.
\]
Dies degeneriert für $r = 0$.
\[
	\nu = \f 1{\sqrt{\dot r^2 + \dot h^2}} (- \dot h\cos \phi, -\dot h \sin \phi, \dot r)
\]
Es gilt
\begin{align*}
	f_{tt} &= \Vector{\ddot r \cos \phi & \ddot r \sin \phi & \ddot h}, &
	f_{t\phi} &= \Vector{-\dot r \sin \phi & \dot r \cos \phi & 0}, &
	f_{\phi\phi} &= \Vector{-r \cos \phi & -r \sin \phi & 0}
\end{align*}
Damit ist
\[
	II = \f 1{\sqrt{\dot r^2 + \dot h^2}} \Matrix*{-\ddot r \dot h + \dot r \ddot h & 0 \\ 0 & r\dot h}.
\]
Also sind die $t$-Linien und $\phi$-Linien überall Eigenrichtungen der Weingartenabbildung.
Also ist
\begin{align*}
	\kappa_1 &= \f 1{(\dot r^2 + \dot h^2)^{\f 32}} (-\ddot r\dot h + \dot r \ddot h), &
	\kappa_2 &= \f 1{\sqrt{\dot r^2 + \dot h^2}} \f {\dot h}r.
\end{align*}
Sei $t = s$ der Bogenlängenparameter, also $\dot r^2 + \dot h^2 = 1$, $r' = \dot r$, $h' = \dot h$, $\dot r\ddot r + \dot h \ddot h = 0$.
Es ergibt sich dann
\begin{align*}
	I &= \Matrix{1 & 0 \\ 0 & r^2}, &
	II &= \Matrix{-r'' h' + r' h'' & 0 \\ 0 & rh'}
\end{align*}
und für die Krümmungen
\begin{align*}
	\kappa_1 &= -r'' h' + r' h'' = \f{h'h''}{r'} h' + r' h'' = \f {h''}{r'}(h'^2 + r'^2) = \f{h''}{r'} = - \f{r''}{h'}, \\
	\kappa_2 &= \f {h'}r.
\end{align*}
Also ist
\begin{align*}
	K &= \kappa_1 \kappa_2 = \f {h'}{r} \cdot (- \f {r''}{h'}) = - \f {r''}{r}, \\
	H &= \f 12(\kappa_1 + \kappa_2) = \f 12 (\f{h''}{r'} + \f{h'}{r}).
\end{align*}
$K = c = \const$ gilt genau dann, wenn $-\f{r''}{r} = c$, also genau dann, wenn $r'' + cr = 0$, die DGL des harmonischen Oszillators.
Die Standard-Lösung für $c = 1$ ist $r(s) = \sin s$, also $h = \pm \cos s$.
Es ergibt sich ein Kreis, bzw. eine Sphäre.

Im Punkt $r = 0, r' = -1, h' = 0, h'' = -1$.
Es gilt die Regel von Bernoulli-l'Hospital:
\begin{align*}
	\kappa_2 = \lim \f{h'}{r} = \f{h''}{r'} = 1, \\
	\kappa_1 = \lim \f{r''}{h'} = \f{r'''}{h''} = -r''' = 1
\end{align*}

\paragraph{Drehflächen mit $K = \const$}

Dies ist genau dann der Fall, wenn $r'' + Kr = 0$, $h' = \pm \sqrt{1 - r^2}$.
Wir unterscheiden drei Fälle:
\begin{itemize}
	\item
		Sei $K = 0$.
		Dann ist $r'' = 0$, also $r(s) = as + b$ und $r' = a$, also $h' = \pm \sqrt{1 - a^2}$ und somit $h$ linear.

		Man erhält die drei Fälle eines Zylinders, einer Ebene und eines Kegels
	\item
		Sei $K > 0$.
		Dann ist $r(s) = a \sin(\sqrt k s), r = b\cos (\sqrt k s)$ und es ergibt sich $h' = \sqrt{1 - a^2 K \cos^2(\sqrt k s)}$.
		Dies geht nur für $0 \le a^2 K \le 1$.

		Für $a^2 K = 1$ ergibt sich die Sphäre mit Radius $\f 1{\sqrt k}$.
		Für $a^2 K < 1$ ergibt sich der \emphdef{Spindeltyp} und für $a^2 K > 1$ der \emphdef{Wulsttyp}.

		Im Fall $a^2 K < 1$ ist $h' \neq 0$, aber $r = 0$ möglich.

		Im Fall $a^2 K > 1$ hat $h'$ Nullstelle mit $r \neq 0$.
		Aus $\kappa_2 = 0$ folgt $\kappa_1 = \infty$ (Krümmung der Kurve).
	\item
		Sei $K < 0$.
		Dann ist $r(s) = a\cosh (\sqrt{-K} s) + b \sinh (\sqrt{-k} s)$.

		Setze $K = -1$, $a > 0$, $r = a \cosh s > 0$ (\emphdef{Kehltyp}).
		Es ist $r' = a \sin h s, h' = \sqrt{1 - a^2 \sinh^2 s}$.
		$h' = 0$, wenn $r' = \pm 1$.

		Setze $K = - 1, b > 0$ (\emphdef{Kegeltyp}).
		$r = \sinh s, r' = b \cos h s > 0, h' = \sqrt{1 - b^2 \cosh^2 s}$ ($b < 0$).
		$r$ hat eine Nullstele und $h'$ hat eine Nullstelle.

		Setze $K = -1$, $a = b$ (\emphdef{Pseudosphäre} doer \emphdef{Beltarmis Fläche}).
		Dann ist $r(s) = a e^s > 0$, $r' > 0$, $h' = \sqrt{1 - 4a^2 e^{2s}}$.
		$h' = 0$ für $4a^2 e^{2s} = 1$, also für $s_0 = \ln \f 1{a}$.
		\begin{align*}
			\kappa_1 &= \f {-r''}{h'} = - \f {ae^s}{\sqrt{1-a^2e^{2s}}} \to \infty \\
			\kappa_0 &= \f {h'}r \to 0
		\end{align*}
		für $s \to s_0$.
		Auch $\kappa_1 \nearrow 0, \kappa_2 \to \infty$ für $s \to -\infty$.
\end{itemize}


\coursetimestamp{27}{05}{2014}

\section{Exkurs: Realisierung der hyperbolischen Ebene (teilweise) im \texorpdfstring{$R^3$}{ℝ³}}


Wir hatten
\[
	H^2 = \{ (x,y) : y > 0 \land \dx[s]^2 = \f 1{y^2} (\dx[y]^2 + \dx^2)
\]
$H^2$ lässt sich bijektiv auf
\[
	\{(\phi, t) : \dx[s]^2 = \dx[t]^2 +  e^{-2t} \dx[\phi]^2 \}
\]
abbilden, bzw.
\[
	H^2 = \{ (x,y) : y > 1 \land \dx[s]^2 = \f 1{y^2} (\dx[y]^2 + \dx^2)
	\bijto
	\{(\phi, t) : t > 0 \land \dx[s]^2 = \dx[t]^2 +  e^{-2t} \dx[\phi]^2 \}
\]
durch
\[
	F(x,y) := (x, \ln y) = (\phi, t).
\]
Definiere nun
\[
	f(t, \phi) = \Vector{e^{-t} \cos \phi & e^{-t} \sin \phi & \int_0^t \sqrt{1-e^{-2s}} \dx[s] }
\]
Für $r = e^{-t}, h = \int_0^t \sqrt{} \dx[s]$ ist $\dot r^2 + \dot h^2 = 1$, also $t$-Linien sind nach Bogenlänge parametrisiert.
Es ergibt sich im Bild die Pseudosphäre, oder Beltramische Fläche.
Es gilt
\[
	Ⅰ= \Matrix{1 & 0 \\ 0 & e^{-2t}}
\]
und $K = - \f{\ddot r}{r} = -1$ konstant, also
\begin{align*}
	\kappa_1 &= - \f{\ddot r}{h} \to -\infty, \\
	\kappa_2 &= - \f{\dot h}{r} \to 0
\end{align*}
für $t \to 0$.

Die Beltramifläche (nach Beltrami 1868) oder Pseudosphäre ist ein (partielles) Modell einer nichteukldischen Geometrie im $\R^3$ mit den „Geraden“ als Bilder der hyperbolischen „Geraden“ in $H^2$.


\section{Exkurs: reguläre Pflastermengen}


Wir kennen drei ebene Geometrien: $S^2, E^2, H^2$ mit „Geraden“, „Dreiecken“, etc.
Welche Zerlegungen gibt es mit $k$-Eken, bei denen alle Seitenlängen und alle Winkel jeweils gleich sind?
Solche bezeichnen wir als „reguläre Pflasterungen“, oder “regular tesselation”.

\paragraph{Euklidisch}
In einem euklidischen $p$-Eck ist die Innenwinkelsumme gleich $(p-2)\pi$.
Wenn $q$ $p$-Ecken um jede Ecke liegen, dann muss gelten
\[
	\f{(p-2)\pi}p = \f {2\pi}q,
\]
also $\f 1p + \f 1q = \f 12$.

Schläftli-Symbol: $\{p, q\}$.
\begin{align*}
	\f 13 + \f 16 &= \f 12 \\
	\f 16 + \f 13 &= \f 12 \\
	\f 14 + \f 14 &= \f 12
\end{align*}

In einem sphärischen $p$-Eck ist die Innenwinkelsumme $> (p-2) \pi$, im hyperbolischen $< (p-2)\pi$.
Sphärisch:
\begin{align*}
	\f 13 + \f 13 &> \f 12 \\
	\f 13 + \f 14 &> \f 12 \\
	\f 13 + \f 15 &> \f 12
\end{align*}
Hyperbolisch ergeben sich viel mehr Möglichkeiten.
% ref: link


\section{Endliche Drehgruppen im \texorpdfstring{$\R^3$}{ℝ³}}


\begin{st}
	Jede endliche Drehgruppe im $\R^3$ ist entweder zyklisch, diedrisch, oder eine der ployedrischen Gruppen Tetraedergruppe ($\isomorphic A_4$), Oktaedergruppe ($\isomorphic S_4$), Ikosaedergruppe ($\isomorphic A_5$) oder eine Untergruppe davon.
	\begin{proof}
		Siehe Homepage
	\end{proof}
\end{st}

\begin{nt}
	Aus \ref{chap:1} wissen wir: eine endlich Untergruppe der euklidischen Bewegungsgruppe hat einen gemeinsamen Fixpunkt (Schwerpunkt einer Bahn).
	% fixme: ref
	Also sind die endlichen Drehgruppen Untergruppen von $\O(3)$ bzw. $\SO(3)$.
	$G \le \SO(3)$ endliche Untegruppe

	Jedes $A \in \SO(3) \setminus \{E\}$ hat eine eindeutig bestimmte Drehachse und einen eindeutig bestimmten Drehwinkel (bis auf Vorzeichen).
\end{nt}

\begin{df}
	$A$ heißt \emphdef{$k$-zählig}, wenn $A^k = E$, aber $A^l \neq E$ für alle $1 \le l < k$.

	Eine Drehachse heißt \emphdef{$k$-zählig}, wenn es in $b$ eine $k$-zählige Drehung um diese Achse gibt, aber keine $m$-zählige Drehung mit $m > k$.
\end{df}


Erster Fall:
In $G$ gibt es nur eine Drehachse und diese ist $k$-zählig mit $k \ge 3$.
$B$ sei die Bahn eines Punktes außerhalb der Achse, $|B| = k$.
Die konvexe Hülle ist ein reguläres $k$-Eck.
$G$ wirkt transitiv auf den Ecken.
$G \isomorphic C_\R$ zyklisch.

Zweiter Fall:
Es gibt mehrere Drehachsen, aber nur eine ist $k$-zählig mit $k \ge 3$.
Die anderen müssen die $k$-zähligen bewahren.
Betrachte die Bahnen $B$ eines Punktes außerhalb.
Die konvexe Hülle ist ein Prisma ($k$ gerade) oder Antiprisma ($k$ ungerade).
Dann ist $G$ die Diedergruppe mit Ordnung $2k$.

Dritter Fall:
Es gibt mehr als eine $k$-zählige Drehacke mit $k \ge 3$
Wähle Punkt $P \neq 0$ auf einer dieser $k$-zähligen Achsen und betrachte die konvexe Hülle davon.
Die konvexe Hülle ist ein 3-dimensionales Polyeder mmit $f_0$ Ecken, $f_1$ Kanten, $f_2$ Flächen.
Jede Ecke ist in derselben Zahl von Kanten enthalten.
Diese Zahl sei $q$, $q \ge 3$.
Es gilt $2 f_1 = q f_0$ und $2f_1 \ge 3 f_2$ (Seiten sind $m$-Ecke mit $m \ge 3$).
Mit dem Eulerschen Polyedersatz $f_0 - f_1 + f_2 = 2$ ergibt sich
\[
	2 = f_0 - f_1 + f_2
	= \f 2q f_1 - f_1 + \underbrace{f_2}_{\le \f 23 f_1}
	\le (\f 2q - 1 + \f 23) f_1
	= \underbrace{(\f 2q - \f 13)}_{\stack !> 0} \underbrace{f_1}_{> 0}
\]
Also muss $q < 6$ und somit $3 \le q \le 5$.
$k$ ist stets Teiler von $q$, also muss $k = q$
Alle Innenwinkel der Seite sind gleich der Zahl $p$.
Mit dem Schläflisymbol $\{p, q\}$, als Projektion auf die Sphäre: $\f 1p + \f 1q > \f 12$, also $(p,q) \in \{ (3,3), (3,4), (4, 3), (3,5), (5,3) \}$.
Als ist die konvexe Hülle eine der platonischen Körper.

Vierter Fall:
Jede Drehachse ist 2-zählig, $A^2 = B^2 = (AB)^2 = E$, $AB = (AB)^{-1} = B^{-1} A^{-1} = BA$, die Gruppe ist abelsch.
Also $G \isomorphic C_2 \times C_2$ oder $G \isomorphic C_2$.
$G$ liefert damit Untergruppen der Oktaedergruppe.


\coursetimestamp{03}{06}{2014}


\section{Regelflächen}


Regelflächen (eigentlich Flächenklassen, engl. “ruled surface”) entstehen durch Verschieben von Geraden im Raum

\begin{df}
	Eine Fläche heißt \emphdef{Regelfläche}, oder \emphdef{geradlinige Fläche}, wenn sie (lokal) eine $C^2$-Parametrisierung der Art
	\[
		f(u,v) = c(u) + v X(u)
	\]
	zulässt, wobei $c$ eine $C^2$-Kurve (nicht notwendig regulär) und $X$ ein nirgends verschwindendes Vektorfeld längs $c$ mit \oBdA $\|x\| = 1$.
	$c(u)$ heißt \emphdef{Leitkurve}.
	Die Geraden $v \mapsto c(u)$ für festes $u$ heißen \emphdef{Erzeugende}, oder \emphdef{Regelgeraden}.
	\begin{note}
		$c$ kann als ebene Kurve gewählt werden.
		Es gilt
		\begin{align*}
			f_u &= c' + vX', &
			f_v &= X
		\end{align*}
	\end{note}
\end{df}

\begin{ex}
	Einfache Beispiele für Regelflächen sind Kegel, Zylinder, Ebene.
	Weiter gibt es
	\begin{itemize}
		\item
			Das \emphdef{Helikoid}, oder \emphdef{Wendelfläche}, oder \emphdef{Schraubfläche} (“\emphdef{staircase surface}”):
			\begin{align*}
				c(u) &= \Vector{0 & 0 & b u}, \\
				X(u) &= \Vector{\cos(\alpha u) & \sin(\alpha u) & 0}
			\end{align*}
	\end{itemize}
\end{ex}

Gibt es ausgezeichnete Parameter für eine Regelfläche?
Wir wollen eine ausgezeichnete Kurve finden.

\begin{lem}[Standardparameter]
	Sei $f: U \to \R^3, f(u,v) = c(u) + v X(u)$ eine reguläre Regelfläche mit $\|X\| = 1$ und überall $X' = \ddx[u]{U} \neq 0$ („$f$ sei nirgends zylindrisch“).
	Dann kann man so umparametrisieren, dass $f_*(u,v) = c_*(u) + vX_*(u)$ mit $\|X_*\| = \|X_*'\| = 1$ und $\<c_*', x_*'\> = 0$.
	\begin{proof}
		Da $X$ als Kurve regulär ist, kann man den Bogenlängenparameter einführen, welchen wir ebenfalls mit $u$ bezeichnen.
		Sei also $X_*(u)$ mit $\|X_*\| = 1 = \|X_*'\|$.
		Umparametrisierte Kurve $c(u)$.

		Es verbleibt $\<c_*', x_*'\> = 0$ zu erreichen.
		Wir setzen an
		\[
			c_*(u) = c(u) + v(u) X(u)
		\]
		für ein $v$.
		Es gilt
		\[
			0 \le \<c_*', X_*'\>
			= \<c' + v(x)X_*' + v'(u) X_*, X_*'\>
			= \<c', X_*'\> + v(u).
		\]
		Damit ist $v(u)$ eindeutig durch $v(u) = -\<c', X_*'\>$ bestimmt.
		$c_*(u)$ heißt \emphdef{Striktionslinie}, oder \emphdef{Kehllinie}.
	\end{proof}
	\begin{note}
		$c_*$ beschreibt sozusagen „die Orte der kleinsten Abstände benachbarter Regelgeraden“.
	\end{note}
\end{lem}

\begin{df}
	Eine Regelfläche $f$ heißt \emphdef{windschief}, wenn alle Regelgeraden windschief sind.
\end{df}

\begin{st}
	In den  Standardparametern ist eine Regelfläche $f(u, v) = c(u) + vX(u)$ bis auf euklidische Bewegungen eindeutig bestimmt durch die folgenden drei Größen
	\begin{align*}
		F &= \<c', X\> \\
		\lambda &:= \<c' \times X, X'\> = \det(c', X, X') \\
		J &:= \<X'', X \times X'\> = \det(X, X', X'') \\
	\end{align*}
	jeweils als Funktion von $u$.
	Umgekehrt bestimmt der Wahl dieser drei Größen eindeutig eine Regelfläche.
	Die Größe $F = g_{12}$ bestimmt dabei den Winkel $\phi$ zwischen der Striktionslinie und $X$ durch $F = \|c'\| \cos \phi$, die Größe $J$ (auch \emphdef{kanonische Krümmung} genannt) bestimt die Krümmung der sphärischen Kurve $X$ und damit $X$ selbst und $\lambda$ nennt man \emphdef{Drall} der Fläche.
\end{st}

Die erste Fundamentalform ist gegeben durch
\[
	I = \Matrix{\<c', c'\> + v^2  & \<c', X\> \\ \<c', X\> & 1}
	= \Matrix{F^2 + \lambda^2 + v^2 & F \\ F & 1 }.
\]

\subsection{Tangentenfläche}

Die Tangentenfläche ist
\[
	f(u, v) := c(u) + v \underbrace{c'(u)}_{X(u)}
\]
mit $f_u = c' + v c'', f_v = c'$ linear abhängig für $v = 0$.

Es gilt
\[
	\lambda = \det(c', X, X')
	= \Det(c', c', c'')
	= 0
\]
Also $\Det I = 0$, falls $v = 0$ (Gratlinie nicht regulär).

\begin{df}
	Eine Regelfläche heißt \emphdef{abwickelbar}, wenn sie lokal in die Ebene abgebildet werden kann, wobei die erste Fundamentalform und die Regelgeraden erhalten bleiben müssen.

	Die Vorstellung dabei ist, dass man eine der Geraden in die Ebene legt und dann den Streifen rechts und links davon in die Ebene „abwickelt“ und zwar unter Bewahrung von Längen und Winkeln.
	Eine nicht abwickelbare Regelfläche heißt \emphdef{windschief}.
\end{df}

\begin{st}
	Für eine Regelfläche sind folgende Bedingungen äquivalent.
	\begin{enumerate}[1)]
		\item
			Die Fläche ist abwickelbar,
		\item
			$K = 0$,
		\item
			Entland jeder der Geraden sind alle Flächennormalen zueinander parallel, d.h. die Gauß-Abbildung ist konstant längs jeder der Geraden.
	\end{enumerate}
	Eine Regelfläche, die diese Bedingungen erfüllt, heißt auch \emphdef{Torse}.
	Falls 3) nur für eine bestimmte Gerade zutrifft, nenn man diese \emphdef{Torsallinie}.

	Eine offene und dichte Teilmenge einer jeden Torse besteht aus Stücken von Ebenen, Kegeln, Zylindern, sowie Tangentenflächen.

	Ein jedes flachpunktfreies Flächenstück mit $K = 0$ ist eine Regelfläche.
\end{st}


\coursetimestamp{05}{06}{2014}

\section{Minimalflächen}


Wir suchen ein Flächenstück mit kleinstmöglicher Oberfläche, das in eine gegebene Kurve eingespannt werden kann.
Physikalisch sind dies gerade Seifenhäute.

Mathematisch geht man wie folgt vor:
\begin{itemize}
	\item
		Herleiten notwendiger Bedingungen,
	\item
		Eingrenzen des Problems,
	\item
		Nachweisen der Existenz,
	\item
		Prüfen der Stabilität.
\end{itemize}

Wir definieren das \emphdef{Oberflächenfunktional} auf einem Flächenstück $f: \_U \to \R^3$ mit kompaktem $\_U$:
\[
	A(f) := \int_{f(\_U)} \di[A]
	= \int_{\_U} \sqrt{\det (g_{ij})} \di[(u_1,u_2)].
\]
Ist $A(f)$ minimal, dann verschwindet die Ableitung in jeder Richtung, bzw. der Gradient von $A$ ist null.
Betrachte die Richtungsableitung von $A$ in Richtung einer \emphdef{normalen Variation} $\phi \nu$.
Verwende als Bedingung
\[
	\f{\partial A(f_\eps)}{\partial \eps} \Big|_{\eps = 0, f} = 0
\]
für $f_\eps(u_1, u_2) := f(u_1, u_2) + \eps \phi(u_1, u_2) \nu(u_1, u_2)$ mit Einheitsnormale $\nu$ und beliebigem, aber festem $\phi$.

Schreibe im Folgenden $\dx[A] = \sqrt{g_{ij}(u_1, u_2)} \dx[u_1] \dx[u_2]$.
Berechne für $f_\eps$ die erste Fundamentalform $g_{ij}^{\eps}$:
\[
	\ddx[u_i]{f_\eps}
	= \ddx[u_i]{f} + \eps \ddx[u_i]{\phi} \nu + \ddx[u_i]{\nu}
\]
Es gilt
\[
	g_{ij}^{\eps}
	= \< \ddx[u_i]{f_\eps} , \ddx[u_j]{f_\eps} \>
	= \< \ddx[u_i]{f}, \ddx[u_j]{f} \> + 2 \eps \phi \underbrace{\<\ddx[u_i]{f}, \ddx[u_j]{\nu} \>}_{=-h_{ij}} + \LandauO(\eps^2)
	= g_{ij} - 2\eps\phi h_{ij} + \LandauO(\eps^2)
\]
Ist $(g_{ij})$ positiv definit, dann ist auch $(g_{ij}^{(\eps)})$ positiv definit für hinreichend kleine $|\eps|$.

Wir setzen an
\begin{align*}
	0 &= \ddx[\eps]{A}\big|_{\eps = 0, f} \\
	&= \ddx[\eps] \big|_{\eps = 0} \int_{\_U} \sqrt{\det(g_{ij}^{(\eps)})} \di[(u_1, u_2)] \\
	&= \int_{\_U} \ddx[\eps] \big|_{\eps = 0} \sqrt{\det(g_{ij}^{(\eps)})} \di[(u_1, u_2)] \\
	= \dotsc
	&= -\int_{\_U} \phi \cdot 2H \di[A].
\end{align*}
Soll diese Bedingung für beliebiges $\phi$ gelten, so muss also $H = 0$ gelten.

\begin{df}[Minimalfläche]
	Sei $f: \_U \to \R^3$ ein Flächenstück, $U \subset \R^2$ sei offen, $\_U$ kompakt mit Rand $\partial U$.
	Eine notwendige Bedingung dafür, dass die Oberfläche von $f$ kleiner oder gleich aller normalen Variationen
	\[
		f_\eps: \_U \to \R^3
	\]
	mit $f_\eps|_{\partial U} = f|_{\partial U}$, ist das Verschwinden der mittleren Krümmung $H$ in ganz $U$.
	Man nennt daher ein Flächenstück mit $H = 0$ eine \emphdef{Minimalfläche}.
	\begin{note}
		$H = 0$ sagt nur, dass die Oberfläche stationär ist, nicht unbedingt das Minimum.
	\end{note}
\end{df}

\begin{df}
	$f: U \to \R^3$ heißt \emphdef{konform}, oder \emphdef{winkeltreu}, wenn
	\[
		(g_{ij}(u_1, u_2))0= \lambda(u_1, u_2) \Matrix{1 & 0 \\ 0 & 1}
	\]
	für eine skalare Funktion $\lambda > 0$ gilt.

	$f, \tilde f$ heißen \emphdef{konform zueinander}, wenn $(\tilde g_{ij}) = \lambda (g_{ij})$, bzw.
	\[
		\< \ddx[u_i]{(\tilde f \circ \Phi)}, \ddx[u_j]{(\tilde f \circ \Phi)} \>
		= \lambda \< \ddx[u_i]{f}, \ddx[u_j]{f} \>
	\]
	für eine Parametertransformation $\Phi$.

	$f, \tilde f$ heißen \emphdef{isometrisch zueinander}, wenn $\lambda = 1$ gewählt werden kann.
\end{df}

\begin{nt}
	Eine komplex-analytische Funktion $f(z)$ mit $f'(z) \neq 0$ überall ist eine konforme Abbildung von einem Teil von $\C$ in einen Teil von $\C$. 
	Für $f'(z_0) = r e^{i\theta}$ ist reell betrachtet
	\[
		\Df \big|_{z_0} = r \Matrix{\cos \theta & -\sin \theta \\ \sin \theta & \cos \theta}
	\]
	eine Drehstreckung.

	Dies sind gerade die Cauchy-Riemann-Gleichungen ($\ddx[u]{x} = \ddx[v]{y}, \ddx[v]{x} = - \ddx[u]{y}$).
\end{nt}

\begin{kor}
	\begin{enumerate}[(i)]
		\item
			Für eine Minimalfläche mit $K \neq 0$ ist die Gaußabbildung $\nu$ konform, d.h. $\nu$ und $f$ sind konform zueinander und der konforme Faktor ist $-K$.
		\item
			Für eine konforme Parametrisierung $f: U \to \R^3$ mit $(g_{ij}) = \lambda \Matrix{1 & 0\\ 0 & 1}$ gilt
			\[
				\Laplace f = \ddx[u_1^2]{f} + \ddx[u_2^2]{f} = 2 H \lambda \nu.
			\]
			Insbesondere definiert eine konforme Parametrisierung $f$ genau dann eine Minimalfläche, wenn die drei Komponentenfunktionen $f_1, f_2, f_3$ von $f$ harmonisch sind, d.h. wenn
			\[
				\Laplace f_i = 0
			\]
			für $i \in \{1, 2, 3\}$.
			Der Vektor $\mathbf H = H \nu$ heißt auch \emphdef{mittlerer Krümmungsvektor}.
	\end{enumerate}
	\begin{proof}
		Folgt aus $Ⅲ - 2HⅡ + KⅠ = 0$.
	\end{proof}
\end{kor}

\begin{kor}
	Ist $f$ komplex-anlytish und $f(u+iv) = x(u,v) + iy(u,v)$, dann sind $x, y$ harmonisch.
	\begin{proof}
		Es gilt
		\[
			x_{uu} + x_{vv} = y_{uv} - y_{vu} = 0
		\]
		und entsprechend $y_{uu} + y_{vv} = -x_{uv} + x_{vu} = 0$.
	\end{proof}
\end{kor}

\begin{st}
	Ist umgekehrt $f$ harmonisch, dann ist $\phi$ holomorphh mit
	\[
		\phi(u + iv) = \ddx[u]{f}(u,v) - i \ddx[v]{f}(u,v)
	\]
	\begin{proof}
		Wegen
		\begin{align*}
			\phi_v &= f_{uv} - v f_{vv}, &
			\phi_u &= f_{uu} - i f_{vu}
		\end{align*}
		ist $\Re \phi_i = - \Im \phi_u$, also $0 = \Re \phi_u - \Im \phi_i = f_{uu} + f_{vv}$. 
	\end{proof}	
\end{st}

\begin{kor}
	Für ein Flächenstück $f: U \to \R^3$ mit Komponenten $f = (f_1, f_2, f_3)$ definieren wir eine Abbildung $\phi: U \to \C^3$ durch $\phi(u + iv) = \ddx[u]{f}(u,v) - i \ddx[v]{f}(u,v)$.

	Dann gilt
	\begin{enumerate}[(i)]
		\item
			$f$ ist konform genau dann, wenn $\phi_1^2 + \phi_2^2 + \phi_3^2 = 0$
		\item
			Falls $f$ eine konforme Parametrisierung ist, so ist $f$ eine Minimalfläche genau dann, wenn $\phi_1, \phi_2, \phi_3$ holomorph sind.
		\item
			Falls umgekehrt $\phi_1, \phi_2, \phi_3$ komplex analytisch sind mit $\phi_1^2 + \phi_2^2 + \phi_3^2 = 0$, dann ist das durch die obigen Gleichungen definierte $f$ regulär (also eine Immersion) genau dann, wenn $\phi_1\_{\phi_1} + \phi_2\_{\phi_2} + \phi_3\_{\phi_3} \neq 0$.
	\end{enumerate}
\end{kor}


