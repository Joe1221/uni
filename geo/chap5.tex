\chapter{Spezielle Klassen von Flächen}



Man kategorisiert verschiedene Spezialfälle von Flächen:
\begin{itemize}
	\item
		Drehflächen als Drehung einer ebenen Kurve um eine Achse,
	\item
		geradelinige Flächen (man schiebe eine Gerade längs einer Kurve),
	\item
		Schiebflächen (man schiebe eine Kurve längs einer Kurve),
	\item
		Minimalflächen (Flächen mit minimaler Oberfläche, „Seifenhäute“).
\end{itemize}

\begin{df}[Drehfläche]
	Sei $t \mapsto (r(t), h(t))$ eine reguläre Kurve in der $(x,z)$-Ebene.
	Drehung um die $z$-Achse ergibt
	\[
		f(t, \phi) = \big( r(t) \cos \phi, r(t) \sin \phi, h(t) \big).
	\]
	Eine solche Fläche heißt \emphdef{Drehfläche} oder \emphdef{Rotationsfläche}.
\end{df}

Es gilt
\begin{align*}
	f_t &= (\dot r \cos \phi, \dot r \sin \phi, \dot h), &
	f_\phi &= (-r \sin \phi, r \cos \phi, 0),
\end{align*}
also
\[
	I = \Matrix*{\dot r^2 + \dot h^2 & 0 \\ 0 & r^2}.
\]
Dies degeneriert für $r = 0$.
\[
	\nu = \f 1{\sqrt{\dot r^2 + \dot h^2}} (- \dot h\cos \phi, -\dot h \sin \phi, \dot r)
\]
Es gilt
\begin{align*}
	f_{tt} &= \Vector{\ddot r \cos \phi & \ddot r \sin \phi & \ddot h}, &
	f_{t\phi} &= \Vector{-\dot r \sin \phi & \dot r \cos \phi & 0}, &
	f_{\phi\phi} &= \Vector{-r \cos \phi & -r \sin \phi & 0}
\end{align*}
Damit ist
\[
	II = \f 1{\sqrt{\dot r^2 + \dot h^2}} \Matrix*{-\ddot r \dot h + \dot r \ddot h & 0 \\ 0 & r\dot h}.
\]
Also sind die $t$-Linien und $\phi$-Linien überall Eigenrichtungen der Weingartenabbildung.
Also ist
\begin{align*}
	\kappa_1 &= \f 1{(\dot r^2 + \dot h^2)^{\f 32}} (-\ddot r\dot h + \dot r \ddot h), &
	\kappa_2 &= \f 1{\sqrt{\dot r^2 + \dot h^2}} \f {\dot h}r.
\end{align*}
Sei $t = s$ der Bogenlängenparameter, also $\dot r^2 + \dot h^2 = 1$, $r' = \dot r$, $h' = \dot h$, $\dot r\ddot r + \dot h \ddot h = 0$.
Es ergibt sich dann
\begin{align*}
	I &= \Matrix{1 & 0 \\ 0 & r^2}, &
	II &= \Matrix{-r'' h' + r' h'' & 0 \\ 0 & rh'}
\end{align*}
und für die Krümmungen
\begin{align*}
	\kappa_1 &= -r'' h' + r' h'' = \f{h'h''}{r'} h' + r' h'' = \f {h''}{r'}(h'^2 + r'^2) = \f{h''}{r'} = - \f{r''}{h'}, \\
	\kappa_2 &= \f {h'}r.
\end{align*}
Also ist
\begin{align*}
	K &= \kappa_1 \kappa_2 = \f {h'}{r} \cdot (- \f {r''}{h'}) = - \f {r''}{r}, \\
	H &= \f 12(\kappa_1 + \kappa_2) = \f 12 (\f{h''}{r'} + \f{h'}{r}).
\end{align*}
$K = c = \const$ gilt genau dann, wenn $-\f{r''}{r} = c$, also genau dann, wenn $r'' + cr = 0$, die DGL des harmonischen Oszillators.
Die Standard-Lösung für $c = 1$ ist $r(s) = \sin s$, also $h = \pm \cos s$.
Es ergibt sich ein Kreis, bzw. eine Sphäre.

Im Punkt $r = 0, r' = -1, h' = 0, h'' = -1$.
Es gilt die Regel von Bernoulli-l'Hospital:
\begin{align*}
	\kappa_2 = \lim \f{h'}{r} = \f{h''}{r'} = 1, \\
	\kappa_1 = \lim \f{r''}{h'} = \f{r'''}{h''} = -r''' = 1
\end{align*}

\paragraph{Drehflächen mit $K = \const$}

Dies ist genau dann der Fall, wenn $r'' + Kr = 0$, $h' = \pm \sqrt{1 - r^2}$.
Wir unterscheiden drei Fälle:
\begin{itemize}
	\item
		Sei $K = 0$.
		Dann ist $r'' = 0$, also $r(s) = as + b$ und $r' = a$, also $h' = \pm \sqrt{1 - a^2}$ und somit $h$ linear.

		Man erhält die drei Fälle eines Zylinders, einer Ebene und eines Kegels
	\item
		Sei $K > 0$.
		Dann ist $r(s) = a \sin(\sqrt k s), r = b\cos (\sqrt k s)$ und es ergibt sich $h' = \sqrt{1 - a^2 K \cos^2(\sqrt k s)}$.
		Dies geht nur für $0 \le a^2 K \le 1$.

		Für $a^2 K = 1$ ergibt sich die Sphäre mit Radius $\f 1{\sqrt k}$.
		Für $a^2 K < 1$ ergibt sich der \emphdef{Spindeltyp} und für $a^2 K > 1$ der \emphdef{Wulsttyp}.

		Im Fall $a^2 K < 1$ ist $h' \neq 0$, aber $r = 0$ möglich.

		Im Fall $a^2 K > 1$ hat $h'$ Nullstelle mit $r \neq 0$.
		Aus $\kappa_2 = 0$ folgt $\kappa_1 = \infty$ (Krümmung der Kurve).
	\item
		Sei $K < 0$.
		Dann ist $r(s) = a\cosh (\sqrt{-K} s) + b \sinh (\sqrt{-k} s)$.

		Setze $K = -1$, $a > 0$, $r = a \cosh s > 0$ (\emphdef{Kehltyp}).
		Es ist $r' = a \sin h s, h' = \sqrt{1 - a^2 \sinh^2 s}$.
		$h' = 0$, wenn $r' = \pm 1$.

		Setze $K = - 1, b > 0$ (\emphdef{Kegeltyp}).
		$r = \sinh s, r' = b \cos h s > 0, h' = \sqrt{1 - b^2 \cosh^2 s}$ ($b < 0$).
		$r$ hat eine Nullstele und $h'$ hat eine Nullstelle.

		Setze $K = -1$, $a = b$ (\emphdef{Pseudosphäre} doer \emphdef{Beltarmis Fläche}).
		Dann ist $r(s) = a e^s > 0$, $r' > 0$, $h' = \sqrt{1 - 4a^2 e^{2s}}$.
		$h' = 0$ für $4a^2 e^{2s} = 1$, also für $s_0 = \ln \f 1{a}$.
		\begin{align*}
			\kappa_1 &= \f {-r''}{h'} = - \f {ae^s}{\sqrt{1-a^2e^{2s}}} \to \infty \\
			\kappa_0 &= \f {h'}r \to 0
		\end{align*}
		für $s \to s_0$.
		Auch $\kappa_1 \nearrow 0, \kappa_2 \to \infty$ für $s \to -\infty$.
\end{itemize}


\coursetimestamp{27}{05}{2014}

\section{Exkurs: Realisierung der hyperbolischen Ebene (teilweise) im \texorpdfstring{$R^3$}{ℝ³}}


Wir hatten
\[
	H^2 = \{ (x,y) : y > 0 \land \dx[s]^2 = \f 1{y^2} (\dx[y]^2 + \dx^2)
\]
$H^2$ lässt sich bijektiv auf
\[
	\{(\phi, t) : \dx[s]^2 = \dx[t]^2 +  e^{-2t} \dx[\phi]^2 \}
\]
abbilden, bzw.
\[
	H^2 = \{ (x,y) : y > 1 \land \dx[s]^2 = \f 1{y^2} (\dx[y]^2 + \dx^2)
	\bijto
	\{(\phi, t) : t > 0 \land \dx[s]^2 = \dx[t]^2 +  e^{-2t} \dx[\phi]^2 \}
\]
durch
\[
	F(x,y) := (x, \ln y) = (\phi, t).
\]
Definiere nun
\[
	f(t, \phi) = \Vector{e^{-t} \cos \phi & e^{-t} \sin \phi & \int_0^t \sqrt{1-e^{-2s}} \dx[s] }
\]
Für $r = e^{-t}, h = \int_0^t \sqrt{} \dx[s]$ ist $\dot r^2 + \dot h^2 = 1$, also $t$-Linien sind nach Bogenlänge parametrisiert.
Es ergibt sich im Bild die Pseudosphäre, oder Beltramische Fläche.
Es gilt
\[
	Ⅰ= \Matrix{1 & 0 \\ 0 & e^{-2t}}
\]
und $K = - \f{\ddot r}{r} = -1$ konstant, also
\begin{align*}
	\kappa_1 &= - \f{\ddot r}{h} \to -\infty, \\
	\kappa_2 &= - \f{\dot h}{r} \to 0
\end{align*}
für $t \to 0$.

Die Beltramifläche (nach Beltrami 1868) oder Pseudosphäre ist ein (partielles) Modell einer nichteukldischen Geometrie im $\R^3$ mit den „Geraden“ als Bilder der hyperbolischen „Geraden“ in $H^2$.


\section{Exkurs: reguläre Pflastermengen}


Wir kennen drei ebene Geometrien: $S^2, E^2, H^2$ mit „Geraden“, „Dreiecken“, etc.
Welche Zerlegungen gibt es mit $k$-Eken, bei denen alle Seitenlängen und alle Winkel jeweils gleich sind?
Solche bezeichnen wir als „reguläre Pflasterungen“, oder “regular tesselation”.

\paragraph{Euklidisch}
In einem euklidischen $p$-Eck ist die Innenwinkelsumme gleich $(p-2)\pi$.
Wenn $q$ $p$-Ecken um jede Ecke liegen, dann muss gelten
\[
	\f{(p-2)\pi}p = \f {2\pi}q,
\]
also $\f 1p + \f 1q = \f 12$.

Schläftli-Symbol: $\{p, q\}$.
\begin{align*}
	\f 13 + \f 16 &= \f 12 \\
	\f 16 + \f 13 &= \f 12 \\
	\f 14 + \f 14 &= \f 12
\end{align*}

In einem sphärischen $p$-Eck ist die Innenwinkelsumme $> (p-2) \pi$, im hyperbolischen $< (p-2)\pi$.
Sphärisch:
\begin{align*}
	\f 13 + \f 13 &> \f 12 \\
	\f 13 + \f 14 &> \f 12 \\
	\f 13 + \f 15 &> \f 12
\end{align*}
Hyperbolisch ergeben sich viel mehr Möglichkeiten.
% ref: link


\section{Endliche Drehgruppen im \texorpdfstring{$\R^3$}{ℝ³}}


\begin{st}
	Jede endliche Drehgruppe im $\R^3$ ist entweder zyklisch, diedrisch, oder eine der ployedrischen Gruppen Tetraedergruppe ($\isomorphic A_4$), Oktaedergruppe ($\isomorphic S_4$), Ikosaedergruppe ($\isomorphic A_5$) oder eine Untergruppe davon.
	\begin{proof}
		Siehe Homepage
	\end{proof}
\end{st}

\begin{nt}
	Aus \ref{chap:1} wissen wir: eine endlich Untergruppe der euklidischen Bewegungsgruppe hat einen gemeinsamen Fixpunkt (Schwerpunkt einer Bahn).
	% fixme: ref
	Also sind die endlichen Drehgruppen Untergruppen von $\O(3)$ bzw. $\SO(3)$.
	$G \le \SO(3)$ endliche Untegruppe

	Jedes $A \in \SO(3) \setminus \{E\}$ hat eine eindeutig bestimmte Drehachse und einen eindeutig bestimmten Drehwinkel (bis auf Vorzeichen).
\end{nt}

\begin{df}
	$A$ heißt \emphdef{$k$-zählig}, wenn $A^k = E$, aber $A^l \neq E$ für alle $1 \le l < k$.

	Eine Drehachse heißt \emphdef{$k$-zählig}, wenn es in $b$ eine $k$-zählige Drehung um diese Achse gibt, aber keine $m$-zählige Drehung mit $m > k$.
\end{df}


Erster Fall:
In $G$ gibt es nur eine Drehachse und diese ist $k$-zählig mit $k \ge 3$.
$B$ sei die Bahn eines Punktes außerhalb der Achse, $|B| = k$.
Die konvexe Hülle ist ein reguläres $k$-Eck.
$G$ wirkt transitiv auf den Ecken.
$G \isomorphic C_\R$ zyklisch.

Zweiter Fall:
Es gibt mehrere Drehachsen, aber nur eine ist $k$-zählig mit $k \ge 3$.
Die anderen müssen die $k$-zähligen bewahren.
Betrachte die Bahnen $B$ eines Punktes außerhalb.
Die konvexe Hülle ist ein Prisma ($k$ gerade) oder Antiprisma ($k$ ungerade).
Dann ist $G$ die Diedergruppe mit Ordnung $2k$.

Dritter Fall:
Es gibt mehr als eine $k$-zählige Drehacke mit $k \ge 3$
Wähle Punkt $P \neq 0$ auf einer dieser $k$-zähligen Achsen und betrachte die konvexe Hülle davon.
Die konvexe Hülle ist ein 3-dimensionales Polyeder mmit $f_0$ Ecken, $f_1$ Kanten, $f_2$ Flächen.
Jede Ecke ist in derselben Zahl von Kanten enthalten.
Diese Zahl sei $q$, $q \ge 3$.
Es gilt $2 f_1 = q f_0$ und $2f_1 \ge 3 f_2$ (Seiten sind $m$-Ecke mit $m \ge 3$).
Mit dem Eulerschen Polyedersatz $f_0 - f_1 + f_2 = 2$ ergibt sich
\[
	2 = f_0 - f_1 + f_2
	= \f 2q f_1 - f_1 + \underbrace{f_2}_{\le \f 23 f_1}
	\le (\f 2q - 1 + \f 23) f_1
	= \underbrace{(\f 2q - \f 13)}_{\stack !> 0} \underbrace{f_1}_{> 0}
\]
Also muss $q < 6$ und somit $3 \le q \le 5$.
$k$ ist stets Teiler von $q$, also muss $k = q$
Alle Innenwinkel der Seite sind gleich der Zahl $p$.
Mit dem Schläflisymbol $\{p, q\}$, als Projektion auf die Sphäre: $\f 1p + \f 1q > \f 12$, also $(p,q) \in \{ (3,3), (3,4), (4, 3), (3,5), (5,3) \}$.
Als ist die konvexe Hülle eine der platonischen Körper.

Vierter Fall:
Jede Drehachse ist 2-zählig, $A^2 = B^2 = (AB)^2 = E$, $AB = (AB)^{-1} = B^{-1} A^{-1} = BA$, die Gruppe ist abelsch.
Also $G \isomorphic C_2 \times C_2$ oder $G \isomorphic C_2$.
$G$ liefert damit Untergruppen der Oktaedergruppe.


