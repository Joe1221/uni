\chapter{Spezielle Klassen von Flächen}



Man kategorisiert verschiedene Spezialfälle von Flächen:
\begin{itemize}
	\item
		Drehflächen als Drehung einer ebenen Kurve um eine Achse,
	\item
		geradelinige Flächen (man schiebe eine Gerade längs einer Kurve),
	\item
		Schiebflächen (man schiebe eine Kurve längs einer Kurve),
	\item
		Minimalflächen (Flächen mit minimaler Oberfläche, „Seifenhäute“).
\end{itemize}

\begin{df}[Drehfläche]
	Sei $t \mapsto (r(t), h(t))$ eine reguläre Kurve in der $(x,z)$-Ebene.
	Drehung um die $z$-Achse ergibt
	\[
		f(t, \phi) = \big( r(t) \cos \phi, r(t) \sin \phi, h(t) \big).
	\]
	Eine solche Fläche heißt \emphdef{Drehfläche} oder \emphdef{Rotationsfläche}.
\end{df}

Es gilt
\begin{align*}
	f_t &= (\dot r \cos \phi, \dot r \sin \phi, \dot h), &
	f_\phi &= (-r \sin \phi, r \cos \phi, 0),
\end{align*}
also
\[
	I = \Matrix*{\dot r^2 + \dot h^2 & 0 \\ 0 & r^2}.
\]
Dies degeneriert für $r = 0$.
\[
	\nu = \f 1{\sqrt{\dot r^2 + \dot h^2}} (- \dot h\cos \phi, -\dot h \sin \phi, \dot r)
\]
Es gilt
\begin{align*}
	f_{tt} &= \Vector{\ddot r \cos \phi & \ddot r \sin \phi & \ddot h}, &
	f_{t\phi} &= \Vector{-\dot r \sin \phi & \dot r \cos \phi & 0}, &
	f_{\phi\phi} &= \Vector{-r \cos \phi & -r \sin \phi & 0}
\end{align*}
Damit ist
\[
	II = \f 1{\sqrt{\dot r^2 + \dot h^2}} \Matrix*{-\ddot r \dot h + \dot r \ddot h & 0 \\ 0 & r\dot h}.
\]
Also sind die $t$-Linien und $\phi$-Linien überall Eigenrichtungen der Weingartenabbildung.
Also ist
\begin{align*}
	\kappa_1 &= \f 1{(\dot r^2 + \dot h^2)^{\f 32}} (-\ddot r\dot h + \dot r \ddot h), &
	\kappa_2 &= \f 1{\sqrt{\dot r^2 + \dot h^2}} \f {\dot h}r.
\end{align*}
Sei $t = s$ der Bogenlängenparameter, also $\dot r^2 + \dot h^2 = 1$, $r' = \dot r$, $h' = \dot h$, $\dot r\ddot r + \dot h \ddot h = 0$.
Es ergibt sich dann
\begin{align*}
	I &= \Matrix{1 & 0 \\ 0 & r^2}, &
	II &= \Matrix{-r'' h' + r' h'' & 0 \\ 0 & rh'}
\end{align*}
und für die Krümmungen
\begin{align*}
	\kappa_1 &= -r'' h' + r' h'' = \f{h'h''}{r'} h' + r' h'' = \f {h''}{r'}(h'^2 + r'^2) = \f{h''}{r'} = - \f{r''}{h'}, \\
	\kappa_2 &= \f {h'}r.
\end{align*}
Also ist
\begin{align*}
	K &= \kappa_1 \kappa_2 = \f {h'}{r} \cdot (- \f {r''}{h'}) = - \f {r''}{r}, \\
	H &= \f 12(\kappa_1 + \kappa_2) = \f 12 (\f{h''}{r'} + \f{h'}{r}).
\end{align*}
$K = c = \const$ gilt genau dann, wenn $-\f{r''}{r} = c$, also genau dann, wenn $r'' + cr = 0$, die DGL des harmonischen Oszillators.
Die Standard-Lösung für $c = 1$ ist $r(s) = \sin s$, also $h = \pm \cos s$.
Es ergibt sich ein Kreis, bzw. eine Sphäre.

Im Punkt $r = 0, r' = -1, h' = 0, h'' = -1$.
Es gilt die Regel von Bernoulli-l'Hospital:
\begin{align*}
	\kappa_2 = \lim \f{h'}{r} = \f{h''}{r'} = 1, \\
	\kappa_1 = \lim \f{r''}{h'} = \f{r'''}{h''} = -r''' = 1
\end{align*}

\paragraph{Drehflächen mit $K = \const$}

Dies ist genau dann der Fall, wenn $r'' + Kr = 0$, $h' = \pm \sqrt{1 - r^2}$.
Wir unterscheiden drei Fälle:
\begin{itemize}
	\item
		Sei $K = 0$.
		Dann ist $r'' = 0$, also $r(s) = as + b$ und $r' = a$, also $h' = \pm \sqrt{1 - a^2}$ und somit $h$ linear.

		Man erhält die drei Fälle eines Zylinders, einer Ebene und eines Kegels
	\item
		Sei $K > 0$.
		Dann ist $r(s) = a \sin(\sqrt k s), r = b\cos (\sqrt k s)$ und es ergibt sich $h' = \sqrt{1 - a^2 K \cos^2(\sqrt k s)}$.
		Dies geht nur für $0 \le a^2 K \le 1$.

		Für $a^2 K = 1$ ergibt sich die Sphäre mit Radius $\f 1{\sqrt k}$.
		Für $a^2 K < 1$ ergibt sich der \emphdef{Spindeltyp} und für $a^2 K > 1$ der \emphdef{Wulsttyp}.

		Im Fall $a^2 K < 1$ ist $h' \neq 0$, aber $r = 0$ möglich.

		Im Fall $a^2 K > 1$ hat $h'$ Nullstelle mit $r \neq 0$.
		Aus $\kappa_2 = 0$ folgt $\kappa_1 = \infty$ (Krümmung der Kurve).
	\item
		Sei $K < 0$.
		Dann ist $r(s) = a\cosh (\sqrt{-K} s) + b \sinh (\sqrt{-k} s)$.

		Setze $K = -1$, $a > 0$, $r = a \cosh s > 0$ (\emphdef{Kehltyp}).
		Es ist $r' = a \sin h s, h' = \sqrt{1 - a^2 \sinh^2 s}$.
		$h' = 0$, wenn $r' = \pm 1$.

		Setze $K = - 1, b > 0$ (\emphdef{Kegeltyp}).
		$r = \sinh s, r' = b \cos h s > 0, h' = \sqrt{1 - b^2 \cosh^2 s}$ ($b < 0$).
		$r$ hat eine Nullstele und $h'$ hat eine Nullstelle.

		Setze $K = -1$, $a = b$ (\emphdef{Pseudosphäre} doer \emphdef{Beltarmis Fläche}).
		Dann ist $r(s) = a e^s > 0$, $r' > 0$, $h' = \sqrt{1 - 4a^2 e^{2s}}$.
		$h' = 0$ für $4a^2 e^{2s} = 1$, also für $s_0 = \ln \f 1{a}$.
		\begin{align*}
			\kappa_1 &= \f {-r''}{h'} = - \f {ae^s}{\sqrt{1-a^2e^{2s}}} \to \infty \\
			\kappa_0 &= \f {h'}r \to 0
		\end{align*}
		für $s \to s_0$.
		Auch $\kappa_1 \nearrow 0, \kappa_2 \to \infty$ für $s \to -\infty$.
\end{itemize}

