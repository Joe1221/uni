\chapter{Lokale Kurventheorie in $E^n$}

\begin{df}
	Sei $I \subset \R$ ein Intervall.
	Eine (parametrisierte) \emphdef{Kurve} ist eine Abbildung $c: I \to \R^n$.
\end{df}

\begin{ex}
	Sei $c: I \to \R^n$ eine Kurve.
	Forderungen an Stetigkeit und Differenzierbarkeit von $c$ liefern teilweise trotzdem merkwürdige Exemplare:
	\begin{itemize}
		\item
			\emphdef{Peano-Kurven} sind stetige, raumfüllende Kurven,
		\item
			die \emphdef{Neilsche Parabel} $c(t) := (t^2, t^3) \in \R^2$ ist eine stetig differenzierbare Kurve, ihr Bild besitzt jedock einen „Knick“.
	\end{itemize}
\end{ex}

\begin{df}
	$c$ heißt \emphdef[Kurve!regulär]{reguläre parametrisierte Kurve}, wenn $c$ stetig differenzierbar ist und $\ddx[t]{c(t)} \neq 0$ für alle $t \in I$.

	Die \emphdef{Tangente} von $c$ in $t_0$ ist $c(t_0) + t \dot c(t_0)$, dies ist gerade der lineare Anteil der Taylorentwicklung von $c$ in $t_0$.

	Eine \emphdef[Kurve!reguläre Kurve]{reguläre Kurve} ist eine Äquivalenzklasse von regulären parametrisierten Kurven mit stetig differenzierbarer Parametertransformation $t \mapsto \tau$ mit $\ddx[t]{\tau} > 0$.

	Die \emphdef{Länge} von $c$ ist definiert als
	\[
		L(c) := \int_a^b \|\dot c(t)\| \dx[t]
	\]
	und unabhängig von der Parametrisierung
	\[
		\int_{\alpha}^{\beta} \| \ddx[\tau]{c} \| \dx[\tau]
		= \int_a^b \|\ddx[t]{c} \| \underbrace{\ddx[\tau]{t} \dx[\tau]}_{= \dx[t]}.
	\]
\end{df}

\begin{lem}
	Jede reguläre Kurve kann nach ihrere Bogenlänge parametrisiert werden $c: [0,L] \to \R^n$ mit $\|\ddx[s]{c}\| = 1$.
	\begin{proof}
		Wir setzen $s(t) := \int_0^t \|\dot c(t)\| \dx[t]$, dann ist $\ddx[t]{s} > 0$.
		Es gilt
		\[
			\big\| \ddx[s] c(t(s)) \big\|
			= \| \ddx[t]{c} \| \ddx[s]{t}
			= \|\dot c(t)\| \cdot \f 1{\|\dot c(t)\|}
			= 1.
		\]
		Dies ist eindeutig bis auf einen “shift” $s \mapsto s + s_0$,
		oder Durchlaufen in der anderen Richtung $s \mapsto s_0 - s$, z.B. $\tilde c: [0,L] \to \R^n : \tilde c(s) = c(L - s)$.
	\end{proof}
\end{lem}

\begin{conv}
	Für reguläre Kurven $c$ vereinbaren wir:
	$t$ sei ein beliebiger Parameter, $s$ sei ein Bogenlängenparameter, $\dot c = \ddx[t]{c}$ Tangente und $c' = \ddx[s]{c}$ Einheits-Tangente.
	Es gilt
	\[
		\dot c = c' \cdot \ddx[t]{s} = c' \|\dot c\|.
	\]
\end{conv}

\begin{ex}
	\begin{itemize}
		\item
			Ein Kreis ist beschrieben durch die Bogenlängenparametrisierung
			\begin{align*}
				c(s) &= (\cos s, \sin s) \\
				c'(s) &= (-\sin s, \cos s)
			\end{align*}
			mit $\|c'\| = 1$, oder durch $c(t) = (\cos (2t), \sin(2t)), \dot c(t) = 2(-\sin 2t, \cos 2t)$.
		\item
			Eine Schraubenlinie ist gegeben durch $c(t) = (\cos t, \sin t, ht)$.
			\[
				\Vector{x & y & z}
				\mapsto
				\Matrix{\cos t & -\sin t & 0 \\ \sin t &\cos t & 0 \\ 0 & 0 & 1}
				\Vector{x & y & z}
				+ \Vector{0 & 0 & ht}
			\]
			mit Ganghöhe $h$.
			Dies bildet die $1$-Parametergruppe von Schraubungen, eine Untergruppe der euklidischen Gruppe $E(3)$.
		\item
			Eine \emphdef{Kettenlinie} ist gegeben durch $c(t) = (t, \cosh t)$ oder $c(t) = (t, a \cos \f ta) = \f 1a (\f ta, \cosh \f ta)$ mit einem Parameter $a$.
	\end{itemize}
\end{ex}

Wie beschreiben wir die Geometrie von regulären Kurven im $\R^n$?
\begin{itemize}
	\item
		Taylorentwicklung
	\item
		ein begleitendes (ON-)Bein (Beschreibung von Krümmungen?)
\end{itemize}

Eine Kurve $c: I \to \R^n$ sei $n$-mal stetig differenzierbar.

\begin{df}
	$c$ und $\tilde c$ \emphdef{berühren sich von der Ordnung $k$}, wenn $c^{(j)}(s_0) = \tilde c^{(j)}(s_0)$ für alle $0 \le j \le k$.
\end{df}

%\section{Raumkurven}

Betrachte eine ebene Kurve ($n=2$) $c: I \to \R^2$, $\|c'\| = 1$ zweimal stetig differenzierbar.
Wir setzen $e_1 := c'$.
Wähle $e_2$ so, dass $e_1, e_2$ ein positiv orientiertes ON-Bein ist.
\[
	0
	= \<c', c'\>'
	= 2 \<c'', c'\>
\]
also ist $c'' \orth c'$ und $c'', e_2$ linear abhängig.
Es gilt also
\begin{align*}
	c'' = e_1' &= \kappa e_2 \\
	e_2' &= -\kappa e_1,
\end{align*}
und wir schreiben
\[
	\Vector{e_1 & e_2}'
	= \Matrix{0 & \kappa \\ -\kappa & 0}
	\Vector{e_1 & e_2},
\]
die sogenannten \emphdef{Frenet-Gleichungen} im $\R^2$.

Ein Standard-Trick, wenn $e_i \orth e_j$ und $\|e_i\| = 1 = \|e_j\|$ ist
\[
	0 = \<e_i, e_j\>'
	= \<e_i', e_j\> + \<e_i, e_j'\>
\]

\paragraph{Rekonstruktion von $c(s)$ mittels $\kappa(s)$}

Ist umgekehrt $\kappa(s)$ stetig gegeben, $c(s)$ gesucht.
Wähle einen Anfangspunkt $c(0) = (0,0)$ und einen Anfangstangente $c'(0) = (1,0) = e_1(0)$.

Man setzt $e_1(s) = (\cos \phi(s), \sin \phi(s))$ und sucht $\phi$ mit $\phi(0) = 0$.
Analog ist um 90 Grad gedreht $e_2(s) = (-\sin \phi(s), \cos \phi(s))$ und
\[
	\kappa e_2
	= e_1' (s)
	= \phi'\cdot e_2 (s).
\]
Daraus folgt $\phi(s) = \int_0^s \kappa(t) \dx[t]$ und
\begin{align*}
	c'(s) &= e_1(s) = \Big( \cos \int_0^s \kappa(t) \dx[t], \sin \int_0^s \kappa(t) \dx[t] \Big), \\
	c(s) &= \bigg( \int_0^s \cos \Big( \int_0^s \kappa(t) \dx[t]\Big) \dx[\sigma], \int_0^s \sin \Big( \int_0^s \kappa(t) \dx[t] \Big) \dx[\sigma] \bigg).
\end{align*}

Im Spezialfall $\kappa = 0$ ergibt sich $\phi(s) = 0$ und $c(s) = (s, 0)$ und für $\kappa \neq 0$ konstant ist
\begin{align*}
	c(s) &= \Big( \int_0^s \cos (\kappa \sigma) \dx[\sigma] , \int_0^s \sin(\kappa \sigma) \dx[\sigma] \Big) \\
	&= \f 1\kappa \big( \sin (\kappa s), 1 - \cos (\kappa s) \big),
\end{align*}
also ergibt sich ein Kreis.

Für konstantes $\f{\kappa(s)}{s}$ ergibt sich eine sogenannte \emphdef{Spinnkurve}.


\section{Raumkurven}

Sei $c$ dreimal stetig differenzierbar, $\|c'\| = 1$ und zusätzlich $c'' \neq 0$.
Wir setzen
\begin{align*}
	e_1 &:= c' \\
	e_2 &:= \f {c''}{\|c''\|} \\
	e_3 &:= e_1 \times e_2,
\end{align*}
$e_1$ ist die Einheitstangente, $e_2$ Hauptnormale, $e_3$ Binormale.
Zusammen bilden sie das \emphdef{Frenet-3-Bein}.
In der Literatur bezeichnet man oft $e_1 = t, e_2 = n, e_3 = b$.

Ableitungsgleichungen $e_1', e_2', e_3'$ berechnen sich als
\begin{align*}
	e_1' &= (c')' = c'' = \|c''\| e_2 = \kappa e_2, \\
	e_2' &= \underbrace{\<e_2', e_1\>}_{=-\<e_2, e_1'\>} e_1 + \underbrace{\<e_2', e_2\>}_{=0} e_2 + \underbrace{\<e_2', e_3\>}_{=: \tau} e_3, \\
	e_3' &= \underbrace{\<e_3', e_1\>}_{=-\<e_3, e_1'\> = 0} e_1 + \underbrace{\<e_3', e_2\>}_{-\<e_3, e_2'\> = -\tau} e_2 + \underbrace{\< e_3', e_3 \>}_{=0} e_3 \\
	&= - \tau e_2
\end{align*}
mit \emphdef{Torsion} $\tau$.
Es ergeben sich die \emphdef{Frenet-Gleichungen} im $\R^3$.
\[
	\Vector{e_1 & e_2 & e_3}'
	= \Matrix{ 0 & \kappa & 0 \\ -\kappa & 0 & \tau \\ 0 & - \tau & 0}
	\Vector{e_1 & e_2 & e_3}.
\]

Können wir bei vorgegebenem $\kappa(s), \tau(s)$ die Kurve rekonstruieren?
Die Lösung existiert und ist eindeutig, aber ist im Allgemeinen nicht explizit darstellbar (aber durch Grenzprozesse, z.B. Picard-Lindelöf).

Taylor-Entwicklung in das Frenet-3-Bein ergibt
\[
	c(s) = c(0) + \alpha(s)e_1(0) + \beta(s) e_2(0) + \gamma(s) e_3(0) + o(s^3)
\]
mit gewissen Koeffizientenfunktionen $\alpha, \beta, \gamma$, die wie folgt bestimmt werden:

Zunächst gilt nach den Frenet-Gleichungen
\begin{align*}
	c' &= e_1 \\
	c'' &= e_1'' = \kappa e_2 \\
	c''' &= (\kappa e_2)' = \kappa' e_2 + \kappa e_2' = \kappa' e_2 + \kappa(-\kappa e_1 + \tau e_3)
\end{align*}
Dies impliziert dann
\begin{align*}
	c(s) &= c(0) + se_1 + \f {s^2}2 \kappa e_2 + \f {s^3}6 \big( \kappa' e_2 - \kappa^2 e_1 + \kappa \tau e_3 \big) + o(e^3) \\
	&= c(0) + \big( s - \f {s^3 \kappa^2}6 \big) e_1 + \big( \f {s^2 \kappa}2 + \f {s^3\kappa'}6 \big) e_2 + \f{s^3 \kappa \tau}6 e_3 + o(s^3)
\end{align*}
mit einem linearen, quadratischen und kubischen Term (in $s$).
Diese lassen sich in die $e_1e_2$-Schmiegebene, die $e_2e_3$-Normalebene und die $e_1e_3$ rektifizierende Ebene projezieren.
Es ergeben sich jeweils eine Parabel, eine Neilsche Parabel (falls $\tau \neq 0$) bzw. eine kubische Parabel (falls $\tau(0) \neq 0$).


Für die Schraubenlinie ist $\kappa, \tau$ konstant, sogar die Umkehrung gilt.


\coursetimestamp{29}{04}{2014}


\begin{df}[Frenet-Kurven im $\R^n$]
	Eine \emphdef{Frenet-Kurve} im $\R^n$ ist definiert durch eine $n$-fach stetig differenzierbare Kurve $c:I \to \R^n$ nach Bogenlänge parametrisiert und $\{c^{(j)}\}_{j=1}^{n-1}$ seien überall linear unabhängig.

	Dann existiert längs der Kurve $c$ das \emph{begleitende Frenet-$n$-Bein} $e_1^{j}, \dotsc, e_{j}$ mit
	\begin{enumerate}[1)]
		\item
			$e_1, \dotsc, e_n$ sind orthonormiert und positiv orientiert ($\det \Matrix{e_1, \dotsc, e_n} = 1$),
		\item
			Für jedes $k$ gilt $\<e_1, \dotsc, e_k\> = \<c', \dotsc, c^{(k)}\>$, genannt \emphdef[Schmiegraum]{$k$-ter Schmiegraum},
		\item
			$\<c^{(k)}, e_k\> > 0$ für $k = 1, \dotsc, n-1$.
	\end{enumerate}
	\begin{note}
		$e_1, \dotsc, e_n$ ist eindeutig durch Gram-Schmidt bestimmt.
		%fixme: genauer
		$\<c^{(n)}, e_n\>$ kann auch $< 0$ sein.
	\end{note}
\end{df}


\begin{st}[Frenet-Gleichungen]
	Sei $c$ eine Frenet-Kurve im $\R^n$ mit Frenet-$n$-Bein $e_1, \dotsc, e_n$.
	Dann gibt es Funktionen $\kappa_1(j), \dotsc, \kappa_{n-1}(j)$ mit $\kappa_1, \dotsc, \kappa_{n-2} > 0$, so dass jedes $\kappa_i$ $(n-1-i)$-mal stetig differenzierbar ist und die \emphdef{Frenet-Gleichungen}
	\[
		\Vector*{e_1 & \vdots & e_n}'
		= \Matrix*{
	                0 & \kappa_1  &          &               &              \\
			-\kappa_1 & 0         & \kappa_2 &               &              \\
					  & -\kappa_2 & \ddots   & \ddots        &              \\
	                  &           & \ddots   & 0             & \kappa_{n-1} \\
		              &           &          & -\kappa_{n-1} & 0
		}
		\Vector*{e_1 & \vdots & e_n}
	\]
	$\kappa_i$ heißt \emphdef[Krümmung]{$i$-te Krümmung} von $c$.
	Die Matrix heißt \emphdef{Frenet-Matrix}.
	Die letzte Krümmung $\kappa_{n-1}$ heißt auch Torsion.
	\begin{proof}
		Betrachte die Komponenten
		\[
			e_i' = \sum_{j=1}^n \<e_i', e_j\> e_j.
		\]
		Nach Konstruktion gilt $\<e_i', e_{i+2} = \<e_i', e_{i+3}\> = \dotsc = 0$ und $\<e_i', e_i\> = 0$.
		Setze $\kappa_i := \< e_i', e_{i+1}\>$ für $i=1, \dotsc, n-1$, dann ist $\kappa_i > 0$ wegen $\<(c^{(k)})', e_{k+1}\> = \<c^{(k+1)}, e_{k+1}\> > 0$ für $i=1,\dotsc, n-2$, $\kappa_{n-1}$ ist noch beliebig.

		Mit dem Standard-Trick ist mit $\<e_i, e_j\> \in \{ 0, 1\}$
		\[
			0 = \<e_i, e_j\>'
			= \<e_i', e_j\> + \<e_i, e_j'\>,
		\]
		also ist die Frenet-Matrix schiefsymmetrisch.
	\end{proof}
\end{st}

\begin{kor}
	Das Frenet-$n$-Bein und die Frenet-Krümmungen sind invariant unter eigentlichen euklidischen Bewegungen (bei uneigentlichen geht $e_n$ in $-e_n$ über).
	\begin{proof}
		Sei $B(\v x) := A\v x + \v b$ mit $A \in \SO(n), \v b \in \R^n$ eine eigentliche euklidische Bewegung.
		Es wird abgebildet
		\begin{align*}
			c &\mapsto B \circ c \\
			c' &\mapsto (B \circ c)' = A c' \\
			\vdots &\qquad \vdots \\
			c^{(j)} &\mapsto Ac^{(j)}
		\end{align*}
		Es gilt
		\begin{align*}
			(A e_i)'
			&= A(e_i') \\
			&= A (-\kappa_{i-1} e_{i-1} + \kappa_i e_{i+1}) \\
			&= - \kappa_{i-1} A e_{i-1} + \kappa_i Ae_{i+1}.
		\end{align*}
	\end{proof}
\end{kor}

\begin{st}[Hauptsatz der lokalen Kurventheorie]
	Es seien $\kappa_i \in C^{n-1-i}((a,b), \R)$ für $i = 1, \dotsc, n-1$ gegeben mit $\kappa_1, \dotsc, \kappa_{n-2} > 0$
	Für ein festes $s_0 \in (a,b)$ sei ein Punkt $q_0 \in \R^n$ und ein positiv orientiertes ON-Bein $e_1(s_0), \dotsc, e_n(s_0)$ vorgegeben.

	Dann gibt es genau eine Frenet-Kurve $c: (a,b) \to \R^n$ mit
	\begin{enumerate}[1)]
		\item
			$c(s_0) = q_0$,
		\item
			$e_1(s_0), \dotsc, e_n(s_0)$ ist das Frenet-$n$-Bein von $c$ in $q_0$,
		\item
			$\kappa_1(s), \dotsc, \kappa_{n-1}(s)$ sind die Frenet-Krümmungen von $c$.
	\end{enumerate}
	\begin{proof}
		Schreibe $F(s) := \Vector{e_1(s) & \vdots & e_n(s)}$ als Matrix.
		$F$ ist orthogonal und $\det(F) = 1$.
		Die Frenet-Gleichungen
		\[
			F'(s) = K(s) F(s)
		\]
		ergeben ein lineares Differenzialgleichungssystem erster Ordnung.
		\begin{enumerate}[1.]
			\item
				Gemäß der Theorie linearer Differenzialgleichungen gibt es genau eine auf ganz $(a,b)$ definierte Lösung $F(s)$ des Systems mit der Anfangsbedingung $F(s_0) = \Vector{e_1(0) & \vdots & e_n(0)}$.
			\item
				Es ist nicht selbstverständlich, dass $F(s)$ der Lösung orthogonal bleibt für beliebige $s \in (a,b)$.
				Es ergibt sich
				\[
					(FF^T)'
					= F' F^T + F (F^T)'
					= K (F F^T) + (F F^T) K^T,
				\]
				eine Differenzialgleichung für $FF^T$.
				Wir wissen $FF^T(s_0) = E$ und $\det F(s_0) = 1$.
				$E$ ist eine Lösung dieser Differenzialgleichung, denn $0 = E' = KE + EK^T$ wegen $K$ schiefsymmetrisch.
				Wegen der Eindeutigkeit muss also $FF^T(s) = E$ für alle $s \in (a,b)$ sein.
				Mit der Stetigkeit der Determinante gilt außerdem auch $\det F(s) = 1$ für alle $s \in (a,b)$.
			\item
				Konstruiere $c(s)$ aus $F(s) = \Vector{e_1(s) & \vdots & e_n(s)}$.
				Es gilt $c' = e_1$ und wir setzen somit zwangsläufig
				\[
					c(s) = c(s_0) + \int_{s_0}^s e_1(t) \dx[t].
				\]
				$c$ ist damit eindeutig bestimmt.
				Wegen $c'' = e_1' = \kappa_1 e_2 \neq 0$ ist e
				\begin{align*}
					e_1' = \kappa_1 e_2 \neq 0 &\implies e_2 = \f{c''}{\|c''\|} \\
					e_2' = -\kappa_1 e_1 + \kappa_2 e_3 &\implies e_3 \\
					\vdots &\qquad \vdots
				\end{align*}
				Wegen $c^{(i)} = \sum_{k=1}^{i-1} \lambda_k e_k + \kappa_1 \dotsb \kappa_{i-1} e_i$ und $\kappa_1 \dotsb \kappa_{i-1} \neq 0$ für $i \le n-1$ sind die $c^{(i)}$ linear unabhängig und $c$ eine Frenet-Kurve.
				$K$ ist das Frenet-$n$-Bein und $\kappa_i$ die Frenet-Krümmungen.
		\end{enumerate}
		Im Fall $n=2$ ist $K= \Matrix{0 & \kappa(s) \\ -\kappa(s) & 0}$ und $\v K(s) = \int_{s_0}^s K(t) \dx[t]$.
		Nutze den Ansatz
		\[
			F(s_0 + s) = \exp (K(s)) \cdot F(s_0).
		\]
		Es gilt
		\[
			F'(s)
			= \v K'(s) \exp(\v K(s)) \cdot F(s_0)
			= K(s) F(s),
		\]
		wegen $(K^n)' = n K' K^{n-1}$, da alle Matrizen kommutieren.
		Im Allgemeinen ist jedoch
		\[
			(\v K^n)'
			= (\v K \dotsc \v K)'
			= \v K' \v K^{n-1} + \v K \v K' \v K^{n-2} + \dotsb + \v K^{n-1} \v K'
		\]
		und geht schief.
		Im Fall $n=3$ funktioniert es nur, falls $K(s) K(t) = K(t) K(s)$ für alle $s, t \in (a,b)$, oder falls $\f {\kappa_2}{\kappa_1} = \f {\tau}{\kappa}$ konstant ist.
		Insbesondere wenn $K$ konstant ist, dann ist $\v K(s) = sK$ und $K' = K$.
	\end{proof}
\end{st}





