\coursetimestamp{08}{04}{2014}

Es gibt verschiedene Arten der Geometrie

\begin{itemize}
	\item
		Inzidenzgeometrie: Punkte, Geraden, Ebenen.
	\item
		Darstellende Geometrie: Projektionen des Anschauungsraumes in die Zeichenebene.
	\item
		Analytische Geometrie über verschiedene Vektorräume:
		\begin{itemize}
			\item
				affin
			\item
				euklidisch
			\item
				projektiv
		\end{itemize}
	\item
		Konvexgeometrie: Konvexe Körper im euklidischen Raum.
	\item
		Polyedergeometrie: Polyeder im euklidischen Raum.
	\item
		Algebraische Geometrie: Nullstellenmengen algebraischer Funktionen.
	\item
		Differentialgeometrie: differenzierbare Kurven, differenzierbare Flächen.
\end{itemize}
Geometrie entsteht also durch Zusammentreffen verschiediener Strukturen (algebraischer, analytischer, topologischer, kombinatorischer, \dots Art).

Im Laufe dieser Vorlesung konzentrieren wir uns auf die Analytische Geometrie und die Differentialgeometrie.


\chapter{Geometrien und ihre Transformationsgruppen}


Wir betrachten im Folgenden den $n$-dimensionalen Vektorraum $\K^n$ über einem Körper $\K$.
$\K$ kann ein endlicher Körper sein, oder $\K \in \{\R, \C\}$.

Auf $\K^n$ setzen wir ein euklidisches, bzw. ein hermitesches Skalarprodukt und eine Norm voraus.

\paragraph{Erlanger Programm}

Felix Klein, 1872: „Es ist eine Mannigfaltigkeit und in derselben eine Transformationsgruppe gegeben.
Man soll die der Mannigfaltigkeit angehörenden Gebilde hinsichtlich solcher Eigenschaften untersuchen, die durch die Transformation der Gruppe nicht geändert werden.“

Im Reellen untersuchen wir affine, euklidische, projektive und hyperbolische, im Komplexen affine, hermitesche, projektive und hyperbolische Räume.


\section{Affine Geometrie}

Bezeichne $\A^n := \{ \text{Punkte} \}$.
Wir bezeichnen Verbindungsvektoren
\begin{align*}
	\A^n \times \A^n &\to \K^n \\
	(P, Q) &\mapsto \vec{PQ} \\
	\A^n \times \K^n &\to \A^n \\
	(P, \vec{PQ}) &\mapsto Q
\end{align*}
Speziell für $\A^n := \K^n$ ist
\[
	\vec{PQ} := Q - P.
\]
Affine Unterräume sind lineare Unterräume plus einer Konstante.

\begin{df}
	Sei $A \in \K^{n\times n}$ eine invertierbare Matrix.
	Eine \emphdef[Abbildung!affin]{affine Abbildung} $f: \A^n \to \A^n$ ist gegeben durch
	\begin{align*}
		f(x) &:= Ax + b, \\
		f^{-1}(y) &:= A^{-1} y  - A^{-1} b
	\end{align*}
	\begin{note}
		Die Menge der $k$-dimensionalen Unterräume bleibt invariant unter affinen Abbildungen.
	\end{note}
	\begin{note}
		Spezialfälle sind
		\begin{itemize}
			\item
				$A = E$: Translation
			\item
				$b = 0$: lineare Abbildung
				\begin{itemize}
					\item
						Mit $b = 0, A = \lambda E$ ergibt sich eine zentrische Streckung am Ursprung
				\end{itemize}
			\item
				$b = (1 - \lambda) p, A = \lambda E$: zentrische Streckung in $p$
		\end{itemize}
	\end{note}
\end{df}

\section{Euklidische Geometrie}


\begin{df}
	Der \emphdef[Raum!affin euklidisch]{affin euklidische Raum} $E^n$ ist definiert als $A^n(\R) = \R^n$ mit dem euklidischen Skalarprodukt für Verbindungsvektoren.

	Eine \emphdef[Transformation!euklidisch]{euklidische Transformation} ist eine affine Abbildung $x \mapsto Ax + b$ mit einer orthogonalen Matrix $A$.
\end{df}

\begin{nt}
	Eine Matrix $A \in \R^n$ ist \emphdef[Matrix!orthogonal]{orthogonal}, falls $A^T A = E$.
	\begin{note}
		$A$ ist orthogonal genau dann, wenn $\<Ax, Ay\> = \<x,y\>$
	\end{note}
	Die Menge der euklidischen Transformationen mit der Hintereinanderausführung bildet die \emphdef[Gruppe!euklidisch]{euklidische Gruppe} $E(n)$.
	Analog $E_*(n)$ mit $\det(A) = 1$.
	\begin{note}
		Eine euklidische Transformation bewahrt
		\begin{itemize}
			\item
				die Menge der affinen Unterräume
			\item
				das Skalarprodukt von Verbidungsvektoren
			\item
				die Winkel zwischen Verbindungsvektioren
			\item
				$\cos \angle (x,y) = \f {\<x,y\>}{\|x\|\|y\|}$
		\end{itemize}
		Spezialfälle sind
		\begin{itemize}
			\item
				Translation $x \mapsto x + b$
			\item
				Drehungen, Spiegelungen: $b = 0$
				\begin{itemize}
					\item
						Normalform einer Drehung
					\item
						Normalform einer Spiegelung
				\end{itemize}
		\end{itemize}
	\end{note}
\end{nt}

\begin{lem}
	Eine endliche Untergruppe $G$ von $E(n)$ fixiert einen Punkt und kann somit als Untergruppe der orthogonalen Gruppe aufgefasst werden.
	\begin{proof}
		Sei $|G| = n$ und $\{P_1, \dotsc P_n\}$ eine Bahn unter $G$
		\[
			P_0 := \f 1n \sum_{k=1}^n P_k
		\]
		bezeichne den Schwerpunkt.
		Es gilt
		\[
			f\Big(\f 1n \sum_{k=1}^n P_k\Big)
			= \f 1n \sum_{k=1}^n (AP_k) + b
			= \f 1n \sum_{k=1}^n \underbrace{(AP_k + b)}_{=P_{\sigma(k)}}
			= \f 1n \sum_{l=1}^n P_l
			= P_0
		\]
	\end{proof}
\end{lem}

\begin{kor}
	Alle endlichen Untergruppen von $E(n)$ sind auch Untergruppen der orthogonalen Gruppe $O(n)$.
\end{kor}

\begin{st}
	Die einzigen endlichen Untergruppen der Drehgruppe $\SO (3) = \< A \in \R^3 : A^TA = E, \det A = 1 \>$ sind folgende
	\begin{itemize}
		\item
			zyklisch beliebiger Ordnung,
		\item
			diädrisch beliebiger Ordnung,
		\item
			eine der Drehgruppen von Tetraeder, Oktaeder, Ikosaeder oder eine Untergruppe davon.
	\end{itemize}
	\begin{proof}
		siehe Homepage.

		Idee: betrachte Bahnen eines Punktes $x$: $\{ gx : g \in G \}$.
	\end{proof}
\end{st}


\section{Sphärische Geometrie}


Bezeichne die $n$-dimensionale Sphäre mit $S^n = \{ x \in \R^{n+1} : \|x\| = 1 \}$.
Die Sphärische Geometrie betrachtet Punkte aus $S^n$, Großkreise als „Geraden“ und Schnitte von linearen Unterräumen mit $S^n$ als Unterräume.

In $S^2$ schneiden sich zwei Großkreise in genau $2$ antipodalen Punkten.
In $S^2 / \pm$ schneiden sich zwei Großkreise in genau einem Punkt und je zwei Punkte liegen in genau einem Großkreis.
Dies ist die reell projektiven Ebene und die Geometrie wird auch \emph{elliptische Geometrie} genannt.

In einem sphärischen Dreieck $\triangle$ gilt
\[
	\pi
	< \alpha + \beta + \gamma
	= \pi + \Area(\triangle)
	< 5 \pi.
\]
Es gilt der Sinussatz
\[
	\f {\sin a}{\sin \alpha}
	= \f {\sin b}{\sin \beta}
	= \f {\sin c}{\sin \gamma}.
\]



