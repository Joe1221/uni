\chapter{Geometrien und ihre Transformationsgruppen} \label{chap:1}


Wir betrachten im Folgenden den $n$-dimensionalen Vektorraum $\K^n$ über einem Körper $\K$.
$\K$ kann ein endlicher Körper sein, oder $\K \in \{\R, \C\}$.

Auf $\K^n$ setzen wir ein euklidisches, bzw. ein hermitesches Skalarprodukt und eine Norm voraus.

\paragraph{Erlanger Programm}

Felix Klein, 1872: „Es ist eine Mannigfaltigkeit und in derselben eine Transformationsgruppe gegeben.
Man soll die der Mannigfaltigkeit angehörenden Gebilde hinsichtlich solcher Eigenschaften untersuchen, die durch die Transformation der Gruppe nicht geändert werden.“

Im Reellen untersuchen wir affine, euklidische, projektive und hyperbolische, im Komplexen affine, hermitesche, projektive und hyperbolische Räume.


\section{Affine Geometrie}

\begin{df}
	Ein \emphdef{affiner Raum} ist ein Tripel $(\A, V, \rightarrow)$, bestehend aus einer Menge $\A$, genannt \emphdef{Punktmenge}, einem Vektorraum $V$ und einer Abbildung $\rightarrow: \A \times \A \to V$, welche zwei Punkten $P, Q \in \A$ ihren \emphdef{Verbindungsvektor} $\vec{PQ} \in V$ zuordnet und folgende Eigenschaften erfüllt:
	\begin{enumerate}[(i)]
		\item
			$\forall P, Q, R \in \A : \vec{PQ} + \vec{QR} = \vec{PR}$,
		\item
			$\forall P \in \A, v \in V \exists! Q \in \A : v = \vec{PQ}$.
	\end{enumerate}
	\begin{note}
		\begin{itemize}
			\item
				Für das $Q$ in (ii) schreiben wir auch kurz $P + v$ und haben somit eine Abbildung $+: \A \times V \to \A$.
			\item
				Für $V = \K^n$ mit einem Körper $\K$, bezeichnen wir den affinen Raum auch kurz mit $\A^n$.
			\item
				Definiert man einen Punkt $O \in \A$ als \emphdef[affiner Raum!Ursprung]{Ursprung}, so lässt sich jeder Punkt $P \in \A$ durch den Vektor $\vec{OP} \in V$ identifizieren.
				Da die konkrete Wahl des Ursprungs zwar relevant für die Zuordnung ist, aber manchmal unwesentlich für die weitere Betrachtung, nutzen wir diese Notation von Punkten als Vektoren überall, wo angebracht, und gehen in diesen Fällen implizit von einem beliebigen – aber festen – Ursprung aus.
		\end{itemize}
	\end{note}
\end{df}

\begin{ex}
	\begin{itemize}
		\item
			Jeder Vektorraum $V$ wird durch $\vec{PQ} := Q - P$ für $P, Q \in V$ zu einem affinen Raum.

			Wenn wir von einem affinen Raum $V^n$ für einen Vektorraum $V$ reden, dann meinen wir gerade diesen affinen Raum mit Ursprung $0$.
		\item
			Sei $U \le V$ ein Untervektorraum und $p \in V$.
			Dann ist $p + U := \{ p + u : u \in U \}$ ein affiner Raum, genannt \emphdef[affiner Raum!affiner Unterraum]{affiner Unterraum}.
	\end{itemize}
\end{ex}

\begin{df}
	Sei $A \in \K^{n\times n}$ eine invertierbare Matrix und $b \in \K^n$.
	Eine Abbildung $f: \A^n \to \A^n$ der Form
	\begin{align*}
		f(x) &:= Ax + b
	\end{align*}
	nennen wir \emphdef[Transformation!affin]{affine Transformation}.
	\begin{note}
		\begin{itemize}
			\item
				Affine Transformationen sind invertierbar und die Inverse sind wieder affine Transformationen:
				\[
					f^{-1}(y) = A^{-1} y  - A^{-1} b.
				\]
			\item
				Die Menge der $k$-dimensionalen Unterräume bleibt invariant unter affinen Abbildungen.
		\end{itemize}
	\end{note}
\end{df}

\begin{ex}
	Folgende Abbildungen sind Spezialfälle von affinen Transformationen $x \mapsto Ax + b$.
	\begin{description}
		\item[Translation]
			Mit der Wahl $A := E$ ergibt sich eine \emphdef{Translation} $x \mapsto x + b$.
		\item[Lineare Abbildung]
			$b := 0$ führt auf eine lineare Abbildung $x \mapsto Ax$ zurück.
			\begin{itemize}
				\item
					Mit $b := 0, A := \lambda E$ ergibt sich eine zentrische Streckung am Ursprung, $x \mapsto \lambda x$.
			\end{itemize}
		\item[Zentrische Streckung]
			Für $p \in \A$ ergibt sich mit $b := (1 - \lambda) p, A := \lambda E$ eine \emphdef{zentrische Streckung} in $p$ um den Faktor $\lambda \in \R$
			\[
				x \mapsto \lambda x + (1 - \lambda) p = \lambda (x - p) + p.
			\]
	\end{description}
\end{ex}


\section{Euklidische Geometrie}

\begin{df}
	Eine Matrix $A \in \R^{n\times n}$ ist \emphdef[orthogonale Matrix]{orthogonal}, falls $A^T A = E$, oder äquivalent $\<Ax, Ay\> = \<x, y\>$ für alle $x, y \in \R^n$.
\end{df}

\begin{df}
	Der \emphdef[euklidischer Raum]{affin euklidische Raum} $E^n$ ist der affine Raum $\R^n$ zusammen mit dem euklidischen Skalarprodukt für Verbindungsvektoren.

	Eine \emphdef[Transformation!euklidisch]{euklidische Transformation} $f: E^n \to E^n$  ist eine affine Abbildung $x \mapsto Ax + b$ mit einer orthogonalen Matrix $A$.
	Ist zusätzlich $\det(A) = 1$, so nennen wir die Transformation \emphdef[eigentliche Transformation]{eigentlich}.

	Die Menge der euklidischen Transformationen mit der Hintereinanderausführung bildet die \emphdef[Gruppe!euklidisch]{euklidische Gruppe} $\E(n)$.
	Die eigentlichen Transformationen bilden ebenfalls eine Gruppe, geschrieben $\E_*(n), \E^+(n)$, oder $\SE(n)$.

	Euklidische Transformationen mit $b = 0$, d.h. $x \mapsto Ax$ mit $A$ orthogonal, bilden die \emphdef{orthogonale Gruppe} $\O(n)$.
	Solche mit $b = 0$ und $\det(A) = 1$ bilden die \emphdef{Drehgruppe} $\SO(n)$.
\end{df}

\begin{nt}
	Eine euklidische Transformation bewahrt
	\begin{itemize}
		\item
			die Menge der affinen Unterräume,
		\item
			das Skalarprodukt von Verbindungsvektoren,
		\item
			die Winkel zwischen Verbindungsvektoren (da eine Größe in Abhängigkeit des Skalarprodukts: $\cos \angle (x,y) = \f {\<x,y\>}{\|x\|\|y\|}$).
	\end{itemize}
	Spezialfälle sind
	\begin{itemize}
		\item
			Translationen $x \mapsto x + b$,
		\item
			Drehungen und Spiegelungen mit $b := 0$:
			\begin{itemize}
				\item
					Normalform einer Drehung
					\[
						A = \diag(R_1, \dotsc, R_k, 1, \dotsc, 1),
						\quad
						R_k = \Matrix{
							\cos \phi & -\sin \phi \\
							\sin \phi & \cos \phi
						}.
					\]
				\item
					Normalform einer Spiegelung
					\[
						A = \diag(R_1, \dotsc, R_k, 1, \dotsc, 1, -1),
						\quad
						R_k = \Matrix{
							\cos \phi & -\sin \phi \\
							\sin \phi & \cos \phi
						}.
					\]
			\end{itemize}
	\end{itemize}
\end{nt}

\begin{lem}
	Eine endliche Untergruppe $G$ von $\E(n)$ fixiert einen Punkt.
	\begin{proof}
		Sei $|G| = n$, $P_1 \in E^n$ und $\{P_1, \dotsc P_n\} := \{f(P_1) : f \in G\}$ eine Bahn unter $G$.
		Definiere den Schwerpunkt als
		\[
			P_0 := \f 1n \sum_{k=1}^n P_k
		\]
		Es gilt für beliebiges $f \in G : x \mapsto Ax + b$:
		\begin{align*}
			f(P_0)
			= f\Big(\f 1n \sum_{k=1}^n P_k\Big)
			&= \f 1n \sum_{k=1}^n (AP_k) + b \\
			&= \f 1n \sum_{k=1}^n \underbrace{(AP_k + b)}_{=f(P_k) = P_{\sigma(k)}}
			= \f 1n \sum_{l=1}^n P_l
			= P_0.
		\end{align*}
	\end{proof}
\end{lem}

\begin{kor}
	Bei geeigneter Wahl des Ursprungs sind alle endlichen Untergruppen von $\E(n)$ auch Untergruppen der orthogonalen Gruppe $\O(n)$.
\end{kor}

\begin{st}
	Die einzigen endlichen Untergruppen der Drehgruppe $\SO (3) = \{ A \in \R^3 : A^TA = E, \det A = 1 \}$ sind folgende:
	\begin{itemize}
		\item
			zyklisch beliebiger Ordnung,
		\item
			diädrisch beliebiger Ordnung,
		\item
			eine der Drehgruppen von Tetraeder, Oktaeder, Ikosaeder oder eine Untergruppe davon.
	\end{itemize}
	\begin{proof}
		siehe Homepage, Idee: betrachte Bahnen eines Punktes $x$.
	\end{proof}
\end{st}


\section{Sphärische Geometrie}


Bezeichne die $n$-dimensionale Sphäre mit $S^n = \{ x \in \R^{n+1} : \|x\| = 1 \}$.
Die Sphärische Geometrie betrachtet Punkte aus $S^n$, Großkreise als „Geraden“ und Schnitte von linearen Unterräumen mit $S^n$ als Unterräume.

In $S^2$ schneiden sich zwei Großkreise in genau $2$ antipodalen Punkten.
In $S^2 / \pm$ schneiden sich zwei Großkreise in genau einem Punkt und je zwei Punkte liegen in genau einem Großkreis.
Dies ist die reell projektiven Ebene und die Geometrie wird auch \emph{elliptische Geometrie} genannt.

In einem sphärischen Dreieck $\triangle$ gilt
\[
	\pi
	< \alpha + \beta + \gamma
	= \pi + \Area(\triangle)
	< 5 \pi.
\]
Es gilt der Sinussatz
\[
	\f {\sin a}{\sin \alpha}
	= \f {\sin b}{\sin \beta}
	= \f {\sin c}{\sin \gamma}.
\]

\coursetimestamp{10}{04}{2014}


\section{Projektive Geometrie}


In der euklidischen Geometrie schneiden sich parallele Geraden nicht.
\begin{align*}
	c_1 + ax_1 + bx_2 &= 0 \\
	c_2 + ax_1 + bx_2 &= 0
\end{align*}
Man führt einen künstlichen „Fernpunkt“ ein, in dem sich diese schneiden.
\begin{align*}
	c_1 x_0 + a x_1 + b x_2 &= 0 \\
	c_2 x_0 + a x_1 + b x_2 &= 0
\end{align*}
in \emph{homogenen Koordinaten}: $[x_0, x_1, x_2]$.
Für $x_0 = 0$ erreicht man den Fernpunkt und für $x_0 = 1$ die ursprünglichen Geradengleichungen.


\begin{df}
	Der \emphdef[projektiver Raum]{$n$-dimensionale projektive Raum $P^n(\K)$} über $\K$ ist definiert als
	\[
		P^n(\K) := \big( \K^{n+1} \setminus \{ 0 \} \big) / \sim,
	\]
	wobei $x \sim \lambda x$ für alle $\lambda \in \K \setminus \{0\}$.
	Wir schreiben Elemente aus $P^n(\K)$ als $[x]$ mit einem Klassenvertreter $x \in \K^{n+1} \setminus \{0\}$.

	Eine \emphdef[Transformation!projektiv]{projektive Transformation} $f: P^n(\K) \to P^n(\K)$ ist eine Abbildung welche von einer invertierbaren linearen Abbildung $A: \K^{n+1} \to \K^{n+1}$ durch
	\[
		f([x]) = [Ax]
	\]
	induziert wird.

	Die Menge der projektiven Transformationen mit der Hintereinanderausführung bilden die \emph{projektive Gruppe} $\PGL(n+1, \K)$ ($n+1$ sinnvoll in Anbetracht der Matrix-Größe).
\end{df}

\begin{ex}
	\begin{itemize}
		\item
			Im Spezialfall $n = 1$, mit einer Matrix $A = \Matrix{a & b \\ c & d} \in \K^{2\times 2}$ ist
			\[
				f\big(\Vector[{ x_0 & x_1 } \big)
				= \l[ A \Vector{x_0 & x_1} \r]
				= \Vector[{ a x_0 + b x_1  &  c x_0 + d x_1 }
				= \begin{cases}
					\Vector[{ \f{a x_0 + b x_1}{c x_0 + d x_1} & 1} & cx_0 + dx_1 \neq 0 \\
					\Vector[{ 1 & 0 } & cx_0 + dx_1 = 0
				\end{cases}.
			\]
			Der affine Teil ($x_1 \neq 0$) von $\Vector[{x_0 & x_1} = \Vector[{\f {x_0}{x_1} & 1}$, liefert
			\[
				f(\Vector[{ x & 1})
				= \Vector[{\f{ax + b}{cx + d} & 1}.
			\]
			Damit können wir $f$ auch als Abbildung $\K \to \K, x \mapsto \f{ax + b}{cx + d}$ auffassen.
			Dies motiviert die Definition der \emph{Möbius-Transformationen}.
		\item
			Wir können $P^1(\C)$ mit der \emphdef[Riemannsche Zahlensphäre]{Riemannschen Zahlensphäre} $\hat \C := \C \cup \infty$ identifizieren, mittels
			\[
				\Vector[{x_0 & x_1}
				\mapsto
				\begin{cases}
					\f {x_0}{x_1} & x_1 \neq 0 \\
					\infty & x_1 = 0
				\end{cases}.
			\]
			Hier ist $\infty$ der Fernpunkt.
	\end{itemize}
\end{ex}

\begin{df}
	Sei $A = \Matrix{a & b \\ c & d} \in \C^{2\times 2}$ mit $\det A = ad - bc \neq 0$.
	Die Abbildung $f: \hat \C \to \hat \C$ mit
	\[
		f(z)
		:= \Matrix*{a & b \\ c & d} \cdot z
		:= \dfrac {a z + b}{c z + d},
	\]
	$f(\infty) = \f ac$ und $f(z) = \infty$ für $c z + d = 0$
	heißt \emphdef{Möbius-Transformation}.
\end{df}

\begin{lem}
	Die Komposition von Möbiustransformationen ist durch die Matrixmultiplikation gegeben.
	\begin{proof}
		nachrechnen oder die projektive Gruppe betrachten
	\end{proof}
\end{lem}

\begin{df}
	Die \emph{stereographische Projektion} $\sigma: \R^3 \supset S^2 \setminus \{ (0, 0, 1) \} \to \R^2 \isomorphic \C$ ist definiert durch
	\[
		\sigma(x_1, x_2, x_3) := \f 1{1-x_3} \Vector*{x_1 & x_2}
	\]
	Falls $x_3 = 0$, so ist $(x_1, x_2, x_3)^T$ ein Fixpunkt unter $\sigma$.

	Die Fortsetzung $\hat \sigma: S^2 \to P^1(\C)$ ist gegeben durch
	\begin{align*}
		\hat \sigma(x_1, x_2, x_3)
		:= \begin{cases}
			\Vector[{1 & 0} & \Vector{x_1 & x_2 & x_3} = \Vector{0 & 0 & 1}\\
			\Vector[{\sigma(x_1, x_2, x_3) & 1} & \text{sonst}
		\end{cases}.
	\end{align*}
\end{df}

\begin{df}
	Eine \emphdef[Möbius-Transformation!der Sphäre]{Möbius-Transformation der Sphäre} $S^2$ ist eine Abbildung $\hat \sigma^{-1} \circ f \circ \hat \sigma$, wobei $f$ eine projektive Transformation ist.

	Die \emphdef{Möbiusgruppe} besteht aus den Möbius-Transformationen $\PGL(2, \C)$, bzw. $\PSL(2, \C)$.
\end{df}

% fixme: Cauchy-Riemann-Gleichungen implizieren winkeltreue ?

\begin{lem}
	$\sigma$ ist winkeltreu
	\begin{proof}
		Zeige: $\sigma^{-1}$ ist winkeltreu.
		\[
			\sigma^{-1}(y_1, y_2)
			= (\lambda y_1, \lambda y_2, 1 - \lambda)
			= \f 1{y_1^2 + y_2^2 + 1} (2y_1, 2y_2, y_1^2 + y_2^2 - 1)
		\]
		mit $\lambda^2 (y_1^2 + y_2^2) + (1-\lambda)^2 = 1$, oder $1 - \lambda = \f{y_1^1 + y_2^2 - 1}{y_1^2 + y_2^2 + 1}$.
		Berechne $\ddx[y_1]{\sigma^{-1}}, \ddx[y_2]{\sigma^{-1}}$.
		Man sieht, dass diese die gleiche Länge haben und orthogonal zueinander stehen.
		Damit ist $D \sigma^{-1}$ eine Drehstreckung und winkeltreu.
	\end{proof}
\end{lem}

\begin{lem}
	$\sigma$ bewahrt Kreise (Kreis durch $(0,0,1)$ wird eine Gerade).
	\begin{proof}
		Schnitt von $S^2$ mit einer Ebene $\alpha x_1 + \beta x_2 + \gamma x_3 + \delta = 0$.
		Betrachte das Bild unter $\sigma^{-1}$, such $y_1, y_2$ mit $\sigma^{-1}(y_1, y_2) = (x_1, x_2, x_3)$.
		\begin{align*}
			\f 1{y_1^2 + y_2^2 + 1} \Big( 2 \alpha y_1 + 2 \beta y_2 + \gamma (y_1^2 + y_2^2 - 1) + \delta (y_1^2 + y_2^2 - 1) &= 0 \\
			(\gamma + \delta)(y_1^2 + y_2^2) + 2 \alpha y_1 + 2 \beta y_2 + \delta - \gamma = 0
		\end{align*}
		Dies ist eine Quadrik, welche für $\gamma + \delta \neq 0$ eine Kreisgleichung und für $\gamma + \delta$ ein Geradengleichung ist.
	\end{proof}
\end{lem}

\begin{lem}
	Jede Möbius-Transformation ist aus den drei Grundtypen zusammengesetzt.
	\begin{proof}
		Sei $\Matrix{a & b \\ c & d}$ gegeben

		Im Fall $c = 0$ ist $f(z) = \f a d z + bd$ eine Drehstreckung und eine Translation.

		Im Fall $c \neq 0$ ist
		\[
			f(z) = \f {az + b}{cz + d}
			= \f ac + \f {\f {bc - ad}c}{cz + d}.
		\]
		Es ergibt sich
		\[
			z \mapsto c z \mapsto c z + d \mapsto \f 1{cz + d} \mapsto \f{\f{bc - ad}{c}}{cz + d} \mapsto \f ac + \f{\f{bc-ad}c}{cz+d} = \f {az + b}{cz + d} = f(z)
		\]
		also Zusammensetzung
	\end{proof}
\end{lem}

\begin{lem}
	$f(z) = \f {az + b} {cz + d}$ bewahrt verallgemeinerte Kreise
	\begin{proof}
		Betrachte Grundtypen von Möbius-Transformationen
		\begin{itemize}
			\item
				Für $\Matrix{a & b \\ c & d} = \Matrix{a & 0 \\ 0 & a^{-1}}$ ist $f(z) = a^2 z$ mit $a \in \C$ eine Drehstreckung mit Zentrum $z=0$ und bewahrt Kreise.
			\item
				Für $\Matrix{a & b \\ c & d} = \Matrix{1 & b \\ 0 & 1}$ ist $f(z) = z + b$ eine Translation und bewahrt Kreise.
			\item
				Für $\Matrix{a & b \\ c & d} = \Matrix{0 & -1 \\ 1 & 0}$ ist $f(z) = - \f 1z$ eine Inversion am Einheitskreis und bewahrt Kreise.
		\end{itemize}
	\end{proof}
\end{lem}

\begin{st}
	\begin{enumerate}[A:]
		\item
			\begin{enumerate}[(i)]
				\item
					Jede Möbiustransformation ist winkeltreu (konform).
				\item
					Jede Möbiustransformation überführt verallgemeinerte Kreise in eben solche um.
			\end{enumerate}
		\item
			\begin{enumerate}[(i)]
				\item
					Jede Möbius-Transformation von $S^2$ ist winkeltreu.
				\item
					Jede Möbius-Transformation von $S^2$ bewahrt Kreise.
			\end{enumerate}
	\end{enumerate}
	\begin{proof}
		Satz A (i) folgt aus der komplexen Differenzierbarkeit.
		Das relle Differential ist eine Drehstreckung (Cauchy-Rieman-Gleichungen).

		Da $\sigma$ winkeltreu ist, folgt Satz B (i).

		Obige Lemmas ergeben Satz A (ii) und Satz B (ii).
	\end{proof}
\end{st}

\coursetimestamp{15}{04}{2014}

% 1.5
\section{Hyperbolische Geometrie}


\subsection{Die hyperbolische Ebene}

\subsubsection{Das Halbebenenmodell}

Wir definieren die Geometrie auf der hyperbolischen Ebene $H^2$ durch die Angabe von Punkten, Geraden und der Parallelitätsrelation:
\begin{itemize}
	\item
		Punkte sind Elemente der oberen komplexen Halbebene $H^2 := \{ z \in \C : \Im z > 0 \}$,
	\item
		Geraden sind (euklidische) Halbkreise mit Mittelpunkt auf der rellen Achse und (euklidische) Halbgeraden, die (euklidisch) senkrecht auf der reellen Achse stehen,
	\item
		zwei Geraden sind parallel, wenn sie keine gemeinsamen Punkte besitzen.
\end{itemize}

Charakteristisch für die hyperbolische Ebene ist die folgende, graphisch leicht nachzuvollziehende Eigenschaft:
\begin{quote}
	Zu einem gegebenen Punkt $p$ und einer Geraden $g$, welche nicht durch $p$ geht, gibt es mindestens zwei zu $g$ parallele Geraden durch $p$.
\end{quote}
Entwickelt man die hyperbolische Geometrie axiomatisch, so wird diese Aussage sogar explizit gefordert.

\subsubsection{Transformationen}

Es gilt $\Iso_+(H^2) = \PSL(2, \R) \le \PGL(2, \C)$ mit Elementen
$f: H^2 \isomorphicto H^2$ mit
\[
	z \mapsto \f{az + b}{cz + d},
\]
% fixme: matrix darstellung
wobei $ad - bc > 0$ und $a,b,c,d \in \R$.
Solche $f \in \Iso_+(H^2)$ sind \emphdef{orientierungserhaltende hyperbolische Bewegungen}.

$\Iso(H^2) = \< \Iso_+(H^2), z \mapsto -\_ z\>$

\begin{st}
	\begin{enumerate}[1)]
		\item
			Jedes Element $f \in \Iso(H^2)$ bildet die obere Halbebene bijektiv auf die obere Halbebene ab und bewahrt hyperbolische Geraden.
		\item
			Zu je zwei Punkten $z, w \in H^2$ existiert eine Abbildung $f \in \Iso(H^2)$ mit $f(z) = w$.
			Zudem kann jede Richtung in jede andere überführt werden.
	\end{enumerate}
	\begin{proof}
		\begin{enumerate}[1)]
			\item
				Für $z \mapsto - \_ z$ ist die Aussage klar.
				Es genügt zu zeigen, dass $f(H^2) \subset H^2$, denn $f \in \Iso_+(H^2) = \PSL(2, \R)$ ist bijektiv.

				Sei $f \in \Iso_+(H^2), z \mapsto \f{az + b}{cz + d}$, dann gilt
				\[
					f(z) = \f {az + b}{cz + d} = \f{(az + b)(c\_z + d)}{|cz + d|^2}
				\]
				und man erhält $\Im f(z) = \f 1{|cz + d|^2} (ad - bc) \Im z > 0$.

				$f$ bewahrt verallgemeinerte Kreise und es gilt
				\[
					f(\_z) = \_{f(z)},
				\]
				also ist $f(\R \cup \{\infty\}) = \R \cup \{\infty\}$.
			\item
				Es reicht zu zeigen, dass $f$ mit $f(i) = w$ existiert.
				Betrachte die Typen
				\begin{itemize}
					\item
						horizontale Verschiebung $z \mapsto z + b$ mit $\Matrix{1 & b \\ 0 & 1}$,
					\item
						zentrische Streckung $z \mapsto a^2 z$ mit $\Matrix{a & 0 \\ 0 & \f 1a}$,
				\end{itemize}
				wobei $a, b \in \R$.
				Mit diesen beiden Typen kann man die gesuchte Funktion konstruieren.

				Zeige, dass jede Richtung in jede andere im Punkt $i$ überführt werden kann.
				Betrachte
				\[
					f_\phi : z \mapsto \f{\cos \phi z - \sin \phi}{\sin \phi z + \cos \phi}
				\]
				mit $\Matrix{\cos \phi & -\sin \phi \\ \sin \phi & \cos \phi}$.
				Es gilt
				\[
					f_\phi(i)
					= \f{i \cos \phi - \sin \phi}{i \sin \phi + \cos \phi}
					= \f {i e^{i\phi}}{e^{i \phi}}
					= i.
				\]
				$f_\phi$ bildet die Halbachse $i \R_{> 0}$ auf den Halbkreis mit Mittelpunkt $m = \f 12 (\cot \phi - \tan \phi)$ und Radius $r = \f 12 (\cot \phi + \tan \phi)$ ab.
				Da $f$ Möbiustransformation und $i$ Fixpunkt unter $f$, zeigen wir zunächst, dass
				\[
					|f_\phi(e) - m | = r.
				\]
				Es gilt
				\begin{align*}
					|f_\phi(i) - m|^2
					&= |i - m|^2 \\
					&= 1 + |m|^2 \\
					&= 1 + \f 14 \big( \cot^2 \phi - \underbrace{2 \cot\phi \tan \phi}_{=1} + \tan^2 \phi \big) \\
					&= \f 14 \big( \cot^2 \phi + 2 + \tan^2 \phi \big)
					= r^2
				\end{align*}
				Da auch $|f_\phi(0) - m| = r$ und $|f_\phi(\infty) - m| = r$ gilt, folgt die Behauptung.
		\end{enumerate}
	\end{proof}
\end{st}

\subsubsection{Längen- und Flächenmessung}

% fixme: Bogelement: länge des tangentialvektors
Das Bogenelement setzen wir als
\[
	d s_H^2 |_z
	:= \f{dz d\_z}{(\Im z)^2}
	= \f{dx^2 + dy^2}{y^2}
\]
Im euklidischen haben wir
\[
	ds^2 |_z
	= dz d\_z
	= dx^2 + dy^2.
\]

Mit Bogenelementen lässt sich gut rechnen:
für eine gegebene Kurve $\gamma: [a,b] \to H^2$ berechnen wir die Länge
\[
	L_H(\gamma)
	= \int_\gamma \dx[s_H]
	= \int_a^b \|\gamma'(t)\|_H \dx[t]
	= \int_a^b \f{\|\gamma'(t)\|}{\Im \gamma(t)} \dx[t].
\]

Da $ds = ds_H \cdot \alpha(p)$, sind die Winkel die gleichen, wie im euklidischen.

\begin{lem}
	Für $f \in \Iso(H^2)$ und $\gamma: [a,b] \to H^2$ gilt
	\[
		L_H(f \circ \gamma) = L_H(\gamma).
	\]
	Ferner gilt $f^* ds_H = ds_H$.
	\begin{proof}
		Sei $\gamma: [a,b] \to H^2$ und $f$ gegeben durch $\Matrix{a & b \\ c & d}$.
		Es gilt
		\[
			\Im((f\circ \gamma)(t))
			= \f {ad - bc}{|c \gamma(t) + d|^2} \Im(\gamma(t))
		\]
		und
		\begin{align*}
			(f\circ \gamma)'(t)
			&= \ddx[z]{f}\big|_{\gamma(t)} \cdot \gamma'(t) \\
			&= \f{a(cz + d) - (az + b)c}{(cz + d)^2} \Big|_{\gamma(t)} \gamma'(t),
		\end{align*}
		also
		\[
			\|(f\circ \gamma)'(t)\|
			= \big| \f{ad - bc}{(c \gamma(t) + d)^2} \big| \|\gamma'(t)\|
		\]
		und
		\[
			\|(f\circ \gamma)'(t)\|_H
			= \f {\|(f\circ \gamma)^{-1}(t)\|}{\Im((f\circ \gamma)(t))}
			= \f {\gamma'(t){\Im(\gamma(t))}}
			= \| \gamma'(t)\|_H.
		\]
		Damit ist die Aussage gezeigt, da $L_H(\gamma) = \int_a^b \|\gamma'(t)\|_H \dx[t]$.
	\end{proof}
\end{lem}

Sei $\gamma: [a,b] \to H^2, t \mapsto it$.
Wir setzen
\[
	d_H(ai, bi) := L_H(\gamma)
	= \int_a^b \f 1t \dx[t]
	= \ln b - \ln a.
\]
Das hyperbolische Flächenelement ist
\[
	dS_H = \f {dx dy}{(\Im y)^2},
\]
im euklidischen: $dS = dxdy$.
Man berechnet damit Flächen für $M \subset H^2$ durch
\[
	\Area_H(M)
	= \int_M \dx[S_H].
\]
Es gilt
\[
	d_H(z, w)
	= \arccos\Big( 1 + \f{|z-w|^2}{2(\Im z)(\Im w)}\Big)
\]

\coursetimestamp{22}{04}{2014}

In der ebenen Geometrie hat es die sphärische Geometrie in $S^2$ (oder elliptische durch Identifizierung der Antipodenpunkte), die euklidische Ebene $E^2$ und die hyperbolische Ebene $H^2 = \{ z \in \C : \Im(z) > 0 \}$.
Jeweils bestehend aus
\begin{itemize}
	\item
		Punktmenge,
	\item
		Menge von „Gerade“,
	\item
		Längen- und Winkelverhältnisse
\end{itemize}
Mit diesen Bestandteilen können wir kongruente Dreiecke und Gruppen von Bewegungen und Spiegelungen beschreiben.

Analog beschreiben wir $S^n, E^n$ und
\[
	H^n := \{ (x_1, \dotsc, x_n) \in \R^n : x_n > 0 \}.
\]
$S^n \subset \R^{n+1}$ erbt die (euklidischen) Längen und Winkeln vom Oberraum.
Wir definieren das hyperbolische Längenelement als
\[
	\f 1{x_n^2} \big( \d x_1^2 + \dotsb + \d x_n^2 \big)
\]
oder alternativ: die Länge eines Vektors $\xi = (\xi_1, \dotsc, \xi_n)$ im Punkt $\v x = (x_1, \dotsc, x_n)$ ist
\[
	\|\xi\|_H
	:= \f 1{x_n} \|\xi\|
	:=\f 1{x_n} \sqrt{\xi_1^2 + \dotsb + \xi_n^2}.
\]

Es gibt allerdings auch eine alternative Beschreibung.
Setze die \emphdef{sphärische Norm}:
\[
	\|\xi\|_S := \f {2\|\xi\|}{1 + |\v x|^2}
\]

Wir betrachten die drei Räume $(\R^n, \|\argdot\|_S), (\R^n, \|\argdot\|), (D^n = \<x \in \R^n : |x| < 1 \}, \|\argdot\|_H)$
mit
\[
	\|\xi\|_H := \f {2\|\xi\|}{1 - |\v x|^2}.
\]
Es gilt
\begin{align*}
	\int_0^\infty \f 2{1+t^2} \dx[t] &= \pi, &
	\int_0^\infty 1 \dx[t] &= \infty, &
	\int_0^1 \f 2{1-t^2} \dx[t] &= \infty.
\end{align*}
Für $n=2$ definieren wir $f: \C \supset D^2 \to H^2 \subset \C$ durch
\[
	f(z) := i \f {1-z}{1+z}.
\]
$f$ ist bijektiv, holomorph (und damit winkeltreu).
% fixme: ref übungsaufgabe

\paragraph{Begründung für $\|\argdot\|_S$}

Wir kennen die stereographische Projektion:
\[
	\sigma(x_1, x_2, x_3) = \f 1{1-x_3} (x_1, x_2)
\]
Bezeichne den Südpol eines Kreises in der $x_1x_3$-Ebene: $x_1 := \sin s, x_2 := 0, x_3 = -\cos s$.
Es gilt
\[
	\sigma(s) = \f 1{1 + \cos s} (\sin s, 0)
\]
und beschreibt eine Halbgerade.
Setze $t(s) := \f {\sin s}{1 + \cos}$, dann ist die Umkehrabbildung gegegben durch
\[
	\sigma^{-1}(t, 0)
	= \f 1{t^2 + 1} (2t, t^2 - 1).
\]
Berechne
\[
	\ddx[t] \sigma^{-1}(t, 0)
	= \f 1{(t^2 + 1)^2} \Big( (t^2 + 1) \cdot 2 - 2t \cdot 2t, (t^2 + 1)\cdot 2t - 2t (t^2 - 1) \Big) \\
	= \f 2{(t^2 + 1)^2} \big( 1 - t^2, 2t \big).
\]
Dann ist
\[
	\| \ddx[t] \sigma^{-1} (t, 0) \|
	= \f 4{(t^2 + 1)^4} \big( (1+t^2)^2 + 4t^2 \big)
	= \f 4{(1+t^2)^2}
\]
Dies rechtfertigt mit $t = |\v x|$ die Wahl $\|\xi\| = \f {2\|\xi\|}{1 + |\v x|^2}$.


% 1.6
\section{Lorentzgeometrie, Lorentzgruppe}

Betrachte $\R_1^{n+1} := \R^{n+1}$ mit dem \emphdef{Lorentz-Skalarprodukt}
\[
	\< \v x, \v y \>_1 :=
	-x_0 y_0 + \sum_{i=1}^n x_i y_i.
\]

\begin{df}
	$x \in \R_1^{n+1}$ heißt
	\begin{itemize}
		\item
			\emphdef{raumartig}, falls $\<\v x, \v x\>_1 > 0$ (zweischalige Hyperboloide),
		\item
			\emphdef{isotop}, oder \emphdef{lichtartig}, oder \emphdef{Nullvektor}, falls $\<\v x, \v x\>_1 = 0$, aber $\v x \neq 0$ (wir bezeichnen $\{ \v x \in \R_1^{n+1} : \<\v x, \v x\>_1 = 0 \}$ als \emphdef{Nullkegel}, oder \emphdef{Lichtkegel}),
		\item
			\emphdef{zeitartig}, falls $\<\v x, \v x\>_1 < 0$ (einschalige Hyperboloide),
	\end{itemize}
\end{df}

Wir betrachten die Geometrie mit denselben Geraden, Ebenen, … wie im euklidischen Raum.

\begin{df}
	Die \emphdef{Lorentzgruppe} $O(1, n)$ ist die Menge aller $(n+1, n+1)$-Matrizen, die $\<\argdot, \argdot\>_1$ bewahren, also
	\[
		\< A\v x, A\v y\>_1 = \< \v x, \v y \>_1
	\]
	für alle $\v x, \v y \in \R^{n+1}$.
	Wir unterscheiden weiterhin
	\begin{itemize}
		\item
			die \emphdef[Lorentzgruppe!eigentliche]{eigentliche Lorentzgruppe} mit $\det A > 0$, $\SO(1, n)$,
		\item
			die \emphdef[Lorentzgruppe!eigentlich orthochrone]{eigentlich orthochrone Lorentzgruppe}, bewahrt die zeitliche Komponente, $\SO^+(1, n)$.
	\end{itemize}
\end{df}

\begin{df}[Hyperboloidmodell des hyperbolischen Raumes]
	Die Quadrik $\{\v x \in \R^{n+1} : \<\v x, \v x\>_1 = -1 \}$ hat zwei Zusammenhangskomponenten (zweischaliges Hyperboloid).
	Die Zusammenhangskomponenten von $(1, 0, \dotsc, 0)$ mit dem von $\<\argdot, \argdot\>_1$ induzierten Längenelement heißt der \emphdef{hyperbolische Raum} $H^n$.
\end{df}

$H^n$ als Quadrik können wir (an $(-1, 0, \dotsc, 0)$) stereographisch auf $D^n$ projezieren, $\sigma: H^n \to D^n$.
\Exercise

Wir parametrisieren eine Hyperbel auf $H^n$ durch
\begin{align*}
	c(s) &:= \big( \cosh s, \sinh s, 0, \dotsc, 0 \big) \\
	\dot c (s) &= \big( \sinh s, \cosh s, 0, \dotsc, 0 \big) \\
	\<\dot c, \dot c\>_1 &= - \sinh^2 s + \cosh^2 s = 1
\end{align*}

\paragraph{Begründung für $\|\argdot\|_H$}.

Die stereographische Projektion wählen wir als
\[
	\sigma(x_0, x_1, \dotsc, x_n) = \f 1{1+ x_0} (x_1, \dotsc, x_n).
\]
Man prüft leicht
\[
	\|\sigma(\v x)\|^2
	= \f 1{(1+x_0)^2}0\big( x_1 ^2 + \dotsb + x_n^2 \big)
	= \f 1{(1+x_0)^2} (x_0^2 - 1) = \f {x_0 - 1}{x_0 + 1} < 1,
\]
also ist $\im \sigma \subset D^n$.

Es gilt
\[
	\sigma(c(s)) = \f 1{1 + \cosh s} \big( \sinh s, 0, \dots, 0 \big).
\]
Für $0 \le t < 1$ erhalten wir die inverse Abbildung
\[
	\sigma^{-1}(t, 0, \dotsc, 0)
	= \big( \f 2{1 - t^2} - 1, \f {2t}{1-t^2} \big)
	= \f 1{1-t^2} \big( 1 + t^2, 2t \big).
\]
In der Tat gilt $\< \dotsc, \dotsc \> = -1$.

Berechne
\begin{align*}
	\ddx[t] \sigma^{-1}()
	&= \f 1{(1-t^2)^2} \Big( (1-t^2) \cdot 2t - (-2t)\cdot (1 + t^2), (1-t^2) \cdot 2 - (-2t)\cdot 2t) \Big) \\
	&= \f 2{(1-t^2)^2} \Big( 2t, 1 + t^2 \Big).
\end{align*}
Für die Länge ergibt sich
\begin{align*}
	\< \ddx[t]{\sigma^{-1}}, \ddx[t]{\sigma^{-1}} \>_1
	&= \f 4{(1-t^2)^4} \big( - 4t^2 + (1+t^2)^2 \big) \\
	&= \f 4{(1-t^2)^2},
\end{align*}
also $\|\ddx[t]{\sigma^{-1}}\|_1 = \f 2{1-t^2}$.
Mit $t = |x|$ rechtfertigt dies wieder die Wahl am Anfang des Abschnitts.

\begin{nt}
	Die eigentlich orthochrone Lorentzgruppe operiert auf $H^n$ durch „hyperbolische Bewegungen“.
	Spezialfälle sind
	\begin{itemize}
		\item
			\emph{räumliche Drehung} mit
			\[
				A = \Matrix{1 & 0 &  \\ 0 & 1 &\\ & & A} \in \SO(n-1),
			\]
		\item
			\emph{„boost“}, oder \emph{Lorentz-Drehung} mit
			\[
				A = \Matrix{\cosh t & \sinh t &  \\ \sinh t & \cosh t &\\ & & I} \in \SO(n-1).
			\]
	\end{itemize}
\end{nt}

\coursetimestamp{24}{04}{2014}

Wir haben Geometrien in verschiedenen Räumen betrachtet, z.B. $S^n, E^n, H^n$.
Als nächstes gehen wir zurück in den euklidischen Raum




