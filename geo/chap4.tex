\chapter{Lokale Flächentheorie}



\begin{df}
	Sei $U \subset \R^2$ offen.

	Ein \emphdef[Flächenstück!parametrisiertes]{parametrisiertes reguläres Flächenstück} ist eine differenzierbare Abbildung $f: U \to \R^3$, wobei $(\ddx[u_j]{f_i})_{i,j} \in \R^{3 \times 2}$ maximalem Rang hat
	($f$ ist eine \emphdef{Immersion}, d.h. ihre Ableitung ist injektiv).

	Elemente $u = (u_1, u_2)$ von $U$ nennen wir \emphdef{Parameter}, Elemente von $f(U)$ \emph{Punkte}.

	Ein unparametrisiertes reguläres \emphdef[Flächenstück!unparametrisiert]{Flächenstück} ist eine Äquivalenzklasse bezüglich $\sim$ von parametrisierten Flächenstücken, wobei $f \sim g$, falls $\Phi$ existiert mit $g = f \circ \Phi$, $\Phi$ bijektiv und $\Phi, \Phi^{-1}$ differenzierbar.
\end{df}

\begin{ex}
	Eine Funktion der Form
	\[
		f(u,v) = \Vector{ u & v & z(u,v)}
	\]
	für differenzierbares $z : U \to \R^3$ definiert stets ein reguläres Flächenstück, denn
	\[
		f_u = \Vector{1 & 0 & z_u},
		f_v = \Vector{0 & 1 & z_v}
	\]
	sind stets linear unabhängig.
\end{ex}

\begin{ex}
	\begin{itemize}
		\item
			Die $2$-Sphäre ohne Äquator ist gegeben durch
			\[
				f(u,v) = \Vector*{ u & v & \pm \sqrt{1 - u^2 - v^2} }.
			\]
			für $u^2 + v^2 < 1$.
			Alternativ
			\[
				f(\phi, \theta) = \Vector*{ \cos \phi \cos \theta & \sin \phi \cos \theta & \sin \theta }
			\]
			mit $0 < \phi < 2\pi$ und $-\f \pi 2 < \theta < \f \pi 2$.
			$f$ ist eine Immersion, denn
			\[
				f_\phi = \Vector{-\sin \phi \cos \theta & \cos \phi \cos \theta & 0},
				f_\theta = \Vector{-\cos \phi \sin \theta & - \sin \phi \sin \theta & \cos \theta}
			\]
			sind wegen $\cos \theta \neq 0$ linear unabhängig.
			Die Koordinaten brechen allerdings im Nord- und Südpol zusammen.

			Es gilt
			\[
				f_\phi \times f_\theta
				= \Vector{\cos \phi \cos^2 \theta & \sin \phi \cos^2 \theta & \sin \theta \cos \theta}
				\neq 0
			\]
			wegen $\cos \theta \neq 0$.
			Anschaulich gesprochen sind $f_\phi, f_\theta$ sind tangential zur Sphäre und $f_\phi \times f_\theta$ ist normal zur Sphäre.
		\item
			Sei
			\[
				f(t,x) := \Vector{t^2 & t^3 & x}
			\]
			dann ist $f_t = \Vector{2t & 3t^2 & 0} = 0$ für $t = 0$, also nicht regulär.
	\end{itemize}
\end{ex}

\begin{conv}
	Der \emphdef[Tangentialraum]{Tangentialraum von $\R^n$ in $x \in \R^n$} ist definiert als $T_p\R^n := \{ u \} \times \R^n \isomorphic \R^n$.

	Sei $f: U \to \R^3$ ein reguläres Flächenstück, $u \in U$ und $p := f(u) \in \R^3$.
	Für einen Punkt $p \in \R^3$ sei $(p, x)$ der Tangentialvektor in $p$ in Richtung $x$.

	Der \emphdef[Tangentialraum!an $U$]{Tangentialraum von $U$ in $u$} ist $T_pU := T_u \R^2$

	Die \emphdef[Tangentialebene]{Tangentialebene von $f$ in $p$} ist
	\[
		T_u f
		:= \Df\big|_u (T_u U)
		\subset T_{f(u)} \R^3
	\]
	und der \emphdef[Normalenraum]{Normalenraum von $f$ in $p$}
	\[
		\Orthspace_u f := \{ f(u) \} \times \{ v \in \R^3 : \forall w \in T_u f : \<v, w\> = 0 \} \subset T_{f(u)} \R^3.
	\]
	\begin{note}
		Es gilt
		\[
			T_u f \oplus \Orthspace_u f = T_{f(u)} \R^3.
		\]
		$T_u f$ wird aufgespannt von $\ddx[u_1]{f}, \ddx[u_2]{f}$ und
		$\Orthspace_u f$ wird aufgespannt von $\ddx[u_1]{f} \times \ddx[u_2]{f}$.
	\end{note}
\end{conv}

\begin{ex}
	Definiere den \emphdef{Rotationstorus} durch
	\[
		f(u, \phi) := \Vector*{ (a+b \cos u) \cos \phi & (a + b \cos u) \sin \phi & b \sin u}
	\]
	mit $0 < \phi, u < 2\pi$ und $0 < b < a$.
	Diese Fläche ist regulär. \Exercise

	Ähnlich wie die Sphäre durch die Gleichung $x^2 + y^2 + z^2 = r^2$ gegeben ist, ist auch der Rotationstorus durch eine Gleichung gegeben, nämlich
	\[
		(a^2 - b^2 + x^2 + y^2 + z^2)^2 = 4 a^2 (x^2 + y^2).
	\]
	Man verifiziert dies mit der Substitution $z^2 = b^2 \sin^2 u$ und $x^2 + y^2 = (a + b \cos u)^2$. \Exercise
\end{ex}

\paragraph{metrische Verhältnisse innerhalb eines Flächenstücks}

Die Länge eine Tangentialvektors $x \in T_v f$ oder $x \in T_p \R^3$ ist
\[
	\|x\| = \sqrt{\<x, x\>}
\]
Die Länge einer Kurve $[a,b] \xrightarrow{c} U \xrightarrow{f} \R^3$ ist
\[
	\int_a^b \| (f \circ c)^\cdot (t)\| \dx[t]
	= \int_a^b \sqrt{\< \Df \dot c, \Df \dot c\>} \dx[t]
\]
mit $\dot c \in T_{c(t)} U$, $(\Df) = (\ddx[u_j]{f_i})_{i,j}$
\[
	\Df \dot c = \Matrix*{ \ddx[u_1]{x_1} & \ddx[u_2]{x_1} \\ \ddx[u_1]{x_2} & \ddx[u_2]{x_2} \\ \ddx[u_1]{x_3} & \ddx[u_2]{x_3} } \Vector{\dot u_1 & \dot u_2}.
\]
Es gilt
\[
	\<\Df \dot c, \Df \dot c\>
	= (\Df \dot c)^T (\Df \dot c)
	= \dot c^T  \Df^T \Df \dot c
\]
mit
\[
	\Df^T \Df = \Matrix*{\< \ddx[u_1]{f}, \ddx[u_1]{f}\> & \<\ddx[u_1]{f}, \ddx[u_2]{f}\> \\ \<\ddx[u_2]{f}, \ddx[u_1]{f}\> & \<\ddx[u_2]{f}, \ddx[u_2]{f}\> },
\]
einer \emph{symmetrischen Billinearform} mit maximalem Rang.

\begin{df}
	Die \emphdef[Fundamentalform!erste]{erste Fundamentalform} $I$ von $f: V \to \R^3$ ist die Einschränkung des euklidischen Skalarprodukts auf alle $T_u f$:
	\[
		I(X, Y) := \< X, Y \>
	\]
	mit $X, Y \in T_u f$.

	In parametrisierter Form haben wir eine symmetrische Billinearform auf $T_u U$:
	\[
		T_u U \times T_u U \ni (V, W) \mapsto \big\< \Df|_u (V), \Df|_u (W) \big\> = I(V, W)
	\]
	mit $V, W \in T_u U$.

	In Koordinaten gegeben durch
	\[
		\Df^T \Df = \Matrix*{\< \ddx[u_1]{f}, \ddx[u_1]{f}\> & \<\ddx[u_1]{f}, \ddx[u_2]{f}\> \\ \<\ddx[u_2]{f}, \ddx[u_1]{f}\> & \<\ddx[u_2]{f}, \ddx[u_2]{f}\> }
		= \Matrix{E & F \\ F & G}
		= \Matrix{g_{11} & g_{12} \\ g_{21} & g_{22}}.
	\]
	Dies ist die Matrix der Ersten Fundamentalform.

	Schreibweise auch
	\[
		\dx[s]^2 = E \dx[y]^2 + 2 E \dx[u_1] \dx[u_2] + G \dx[u_2]^2.
	\]
	Dann ist
	\[
		\| (f \circ c)^\cdot \| = \sqrt{E \dot u_1^2 + 2 F \dot u_1 \dot u_2 + G \dot u_2^2}.
	\]
\end{df}

\begin{ex}
	Betrachte wieder die Sphäre:
	\[
		f_\phi = \Vector{-\sin \phi \cos \theta & \cos \phi \cos \theta & 0},
		f_\theta = \Vector{-\cos \phi \sin \theta & - \sin \phi \sin \theta & \cos \theta}
	\]
	Es gilt
	\begin{align*}
		E &= \< f_\phi, f_\phi\> = \cos^2 \theta \\
		F &= \< f_\phi, f_\theta \> = 0 \\
		G &= \< f_\theta, f_\theta \> = 1
	\end{align*}
	und damit $\Matrix{E & F \\ F & G} = \Matrix{\cos^2 \theta & 0 \\ 0 & 1}$.

	$f_\phi, f_\theta$ sind orthogonal für $\theta = 0$.
	$f_\phi$ wird verkürzt für $\theta \to \pm \f \pi2$.
\end{ex}

\coursetimestamp{15}{05}{2014}

\begin{lem}
	Bei einer Parametertransformation $\tilde f = f \circ \Phi$ ergibt sich für $(g_{ij}), (\tilde g_{ij})$
	\[
		(g_{ij}) = (\Df[\Phi])^T(g_{ij}(\Df[\Phi])
	\]
	\begin{proof}
		Es gilt
		\begin{align*}
			\tilde g_{ij} &= (\Df[\tilde f])^T (\Df[ \tilde f])
			= (\Df \circ \Df[\Phi])^T (\Df \circ \Df[\Phi])
			= (\Df[\Phi])^T \underbrace{\big((\Df)^T \Df\big)}_{=(g_{ij})} \circ \Df[\Phi]
		\end{align*}
	\end{proof}
\end{lem}

Es gilt $\det(\tilde g_{ij}) = \det(g_{ij}) (\det(\Df[\Phi]))^2$.

$g := \det(g_{ij})$ heißt \emphdef{Gramsche Determinante}
\[
	\tilde g = g (\det(\Df[\Phi]))^2
\]
Die \emphdef{Gramsche Matrix} ist
\[
	\Matrix{\<x,x\> & \<x,y\> \\ \<y,x\> & \<y,y\>}
\]
mit $\det() = \|x\|^2 \|y\|^2 \sin^2 \angle$, also ist
\[
	\sqrt{\det}
	= \|x\|\|y\| |\sin \alpha|
\]
der Flächeninhalt des Paralallograms.

Für eine lineare Abbildung $F$ ist die Gramsche Determinante
\[
	\<Fx, Fx\> \<Fy, Fy\> - \<Fx, Fy\>^2
\]
das Quadrat der Flächenverzerrung von $F$.
Damit ist $\sqrt{g} = \sqrt{\det(g_{j})}$ die Flächenverzerrung der Abbildung $f$.

\begin{df}[Oberflächenintegral]
	Sei $f: U \to \R^3$ injektiv, $\alpha$ sei stetig und auf ganz $f(U)$ definiert.
	Dann ist für jedes Kompaktum $Q \subset U$
	\[
		\iint_{f(Q)} \alpha \dx[A]
		:= \iint_{Q} (\alpha \circ f)(u_1, u_2) \sqrt{g} \dx[(u_1, u_2)]
	\]
	das \emphdef{Oberflächenintegral} von $\alpha$ erklärt.
	$\alpha = 1$ liefert den Flächeninhalt.
	$\dx[A] = \sqrt{g} \dx[(u_1, u_2)]$ heißt \emphdef{Flächenelement}.
	\begin{note}
		Diese Definition ist unabhängig von der Parametrisierung, denn für $\tilde f = f \circ \Phi, Q = \Phi(\tilde Q), (u_1,u_2) = \Phi(\tilde u_1, \tilde u_2)$ ist
		\begin{align*}
			\iint_{\tilde f(Q)} \alpha \dx[\tilde A]
			&= \iint_{\tilde Q} (\alpha \circ \tilde f)(\tilde u_1, \tilde u_2) \underbrace{\sqrt{g}}_{=|\det \Df[\Phi]| \sqrt{g}} \dx[(\tilde u_1, \tilde u_2)] \\
			&= \iint_{Q} (\alpha \circ f) (u_1, u_2) \sqrt{g} \dx[(u_1, u_2)] \\
			&= \iint_{f(Q)} \alpha \dx[A]
		\end{align*}
		per Substitution.
	\end{note}
\end{df}

\section{Vektorfelder}


Grundsätzlich ist jede Zuordnung
\[
	p \mapsto (p, x) \in T_p\R^3
\]
ein \emphdef{Vektorfeld}.

\begin{df}[Vektorfeld längs $f: V \to \R^3$]
	ein Vektorfeld längs $f$ ist eine Zuordnung
	\[
		U \ni u \mapsto (f(u), x(u)) \in T_{f(u)} \R^3
	\]
	Analog definiert man stetige und differenzierbare Vektorfelder für stetiges, bzw. differenzierbares $x$.

	Man nennt das Vektorfeld \emphdef[Vektorfeld!tangential]{tangential}, wenn $x(x) \in T_u(f)$ für alle $u \in U$ und \emphdef[Vektorfeld!normal]{normal}, wenn $x(u) \in \orth_u f$ für alle $u \in U$.
\end{df}

\begin{ex}
	$\ddx[u_1]{f}, \ddx[u_2]{f}$ sind tangentiale Vektorfelder längs $f$.
	$\ddx[u_1]{f} \times \ddx[u_2]{f}$ ist ein normales Vektorfeld längs $f$.
\end{ex}

\begin{df}
	Wir nennen
	\[
		\nu := \dfrac{\ddx[u_1]{f} \times \ddx[u_2]{f}}{\|\ddx[u_1]{f} \times \ddx[u_2]{f}}
		\in \orth_u \subset T_{f(u)}\R^3.
	\]
	Die Einheitsnormale ist nicht eindeutig, $-\nu$ ist auch eine Einheitsnormale.
\end{df}

Wann ist eine 2-dimensionale Untermannigfaltigkeit des $R^3$ orientierbar?
Genau dann, wenn es ein stetiges Einheitsnormalenvektorfeld gibt (Forster, Analysis 3). % fixme: ref

\[
	\ddx[u_1]{f}, \ddx[u_2]{f}, \dfrac{\ddx[u_1]{f} \times \ddx[u_2]{f}}{\|\ddx[u_1]{f} \times \ddx[u_2]{f}\|}
\]
ist positiv orientiert, Umdrehen von $\ddx[u_1]{f}, \ddx[u_2]{f}$ ergibt die andere Orientierung.
$(u_1, u_2) \mapsto \Phi(u_1, u_2)$ mit positivem $\det \Df[\Phi]$ ?

Innerhalb von $U \subset \R^2$ gibt es eine feste Orientierung durch Wahl einer Reihenfolge $u_1, u_2$.
Für die Bilmenge $f(U)$ braucht das nicht zu gelten (falls $f$ nicht injektiv ist).

\begin{ex}[Möbiusband]
	Setze
	\[
		f(u,v) := \Vector*{\sin u + v \sin \f{u}2 \sin u & \cos u + v \sin \f{u}2 \cos u & v \cos \f {u}2}.
	\]
	Für $v = 0$ ist $f(u, 0) = \Vector{\sin u & \cos u & 0}$ ein Kreis.
	Für $u = \const$ ergibt sich ein Geradenstück durch $f(u, 0)$.
	$f$ ist regulär für „kleine“ $|v|$ und injektiv für $0 \le u < 2\pi$.
	Für $v \neq 0, |v|$ klein ist
	\[
		v \cos \f u2 = v \cos {\tilde u}2
		\implies u = \tilde u
	\]
	aber $f(2\pi, v) = f(0, -v) = (0,1, -v)$, nicht injektiv.
	$\ddx[u]{f}, \ddx[v]{f}$ liefert Orientierung bei $v = 0$, aber liefert eine andere Orientierung bei $v = 2\pi$, weil $\ddx[v]{f}\big|_{u=2\pi} = - \ddx[v]{f}\big|_{u=0}$.
	Die Einheitsnormalen in $f(2\pi, v)$ und $f(0, -v)$ unterscheiden sich.

	Das Möbiusband ist also als Untermannigfaltigkeit des $\R^3$ \emph{nicht orientierbar}.
\end{ex}

Lokal ist Orientierbarkeit kein Problem (wähle $U$ hinreichend klein).
Die weiteren Betrachtungen sind zunächst lokal.
Wir benötigen 2. Ableitungen, sei also $f$ zweimal stetig differenzierbar.

\begin{df}[Gauß-Abbildung]
	Für $f: U \to \R^3$ ist die \emphdef{Gaußsche Normalenabbildung} oder \emphdef{Gauß-Abbildung} $\nu: U \to S^2$ erklärt als
	\[
		\nu(u_1, u_2) := \dfrac{\ddx[u_1]{f} \times \ddx[u_2]{f}}{\|\ddx[u_1]{f} \times \ddx[u_2]{f}\|} \in S^2 \subset \R^3
	\]
	Identifiziere $T_{f(u)} \R^3$ mit $T_{\nu(u)}\R^3$ durch Translation.
\end{df}

Ideal wäre
\begin{align*}
	\ddx[u_1]{\nu} &= \alpha \ddx[u_1]{f} \\
	\ddx[u_2]{\nu} &= \beta \ddx[u_2]{f}.
\end{align*}
Dann wären $\alpha, \beta$ natürliche Krümmungen.
Dies ist allerdings im Allgemeinen nicht der Fall.
Es läuft auf ein Vergleich von $\Df[\nu]$ mit $\Df$ hinaus.
\begin{align*}
	\Df\big|_U : T_u U  &\bijto T_u f \subset T_{f(u)} \R^3, \\
	\Df[\nu]\big|_U : T_u U &\to T_{\nu(u)} \R^3.
\end{align*}
Nun ist wegen $1 = \<\nu, \nu\>$ gerade $0 = \<\ddx[u_i]{\nu}, \nu\>$ mit $i = 1,2$.
Damit ist $\ddx[u_1]{\nu}, \ddx[u_2]{\nu}$ tangential an der Fläche, also nach Identifikation durch Translation erhalten wir
\[
	\Df[\nu]\big|_U : T_u U \to T_u f.
\]
$\Df$ hat maximalen Rang, also existiert $(\Df)^{-1}: T_\nu f \to T_\nu U$.
Die Ableitungen $\Df, (\Df)^{-1}, \Df[\nu]$ sind parameterunabhängig.

\begin{df}[Weingartenabbildung]
	Die Komposition $L := -\Df[\nu] \circ (\Df)^{-1}$ im Punkt $u$:
	\[
		L_u := -\Df[\nu]\big|_u \circ (\Df\big|_u)^{-1} : T_u f \to T_u f
	\]
	heißt \emphdef{Weingartenabbildung}, oder \emphdef{Formoperator}.
	Insbesondere ist $L(\ddx[u_i]{f}) = - \ddx[u_i]{\nu}$.
\end{df}

$L$ ist ein Endomorphismenfeld der Tangentialebene.

\begin{lem}
	$L$ ist unabhängig von der Parametrisierung (bis auf die Wahl von $\nu$, d.h. ein Vorzeichen) und ist selbstadjungiert bezüglich der ersten Fundamentalform.
	\begin{proof}
		Sei $f = f \circ \Phi, \tilde \nu = \pm \nu \circ \Phi$.
		Dann ist
		\begin{align*}
			\tilde L
			&= -(\Df[\tilde \nu]) (\Df[\tilde f])^{-1}
			= \mp \Df[\nu] \Df[\Phi] (\Df \Df[\Phi])^{-1} \\
			&= \mp \Df[\nu] \Df[\Phi] (\Df[\Phi])^{-1} (\Df)^{-1}
			= \mp \Df[\nu] (\Df)^{-1}
			= \pm L
		\end{align*}
	\end{proof}
	Es gilt
	\[
		I(L \ddx[u_i]{f}, \ddx[u_j]{f})
		= \< - \ddx[u_i]{\nu}, \ddx[u_j]{f} \>
		= - \ddx[u_i]\underbrace{\<\nu, \ddx[u_j]{f}\>}_{=0} + \<\nu, \underbrace{\f{\partial^2 f}{\partial u_i \partial u_j}}_{\text{symm. in $i,j$}} \>
	\]
	Dann ist
	\[
		I(L \ddx[u_i]{f}, \ddx[u_j]{f})
		= I(\ddx[u_i]{f}, L \ddx[u_j]{f}),
	\]
	also \emph{selbstadjungiert}.
	\begin{note}
		Damit ist $L$ insbesondere reell diagonalisierbar mit reellen Eigenwerten.
	\end{note}
\end{lem}

\coursetimestamp{20}{05}{2014}


Sei
\begin{align*}
	LX &= -\Df[\nu](\Df)^{-1}X) = \lambda X \\
	LY &= -\Df[\nu](\Df)^{-1}X) = \my Y
\end{align*}

Setze $X = \Df(\xi), Y = \Df(\eta)$
\begin{align*}
	-\Df[\nu](\xi) &= \lambda \Df(\xi) \\
	-\Df[\nu](\eta) &= \my \Df(\eta)
\end{align*}

Welche Krümmungen einer Kurve können wir erwarten?
$c$ sei Kurve in der Läche, d.h. $c = f \circ \gamma$ für eine Kurve $\gamma$ in $U$.
Die Krümmung $\kappa$ von $c$ ist $\kappa = \|c''\|$.
Als Raumkurve
\begin{align*}
	c'' &= (c'')^{\text{Tangentialanteil}} + (c'')^{\text{Normalanteil}}
	&= (c'')^{\text{Tang}} +\<c'', \ny\> \ny
\end{align*}
und
\[
	\<c'', \ny\> = \<\ddx[s^2]{c}, \ny\> = -\<\ddx[s]{c}, \ddx[s]{\ny} \> = \<\ddx[s]{c}, L \ddx[s]{c}\>
	= I(c', Lc')
\]
hängt nur von der Fläche und $c' = \ddx[s]{c}$ ab.
\[
	L(\ddx[s]{s}) = -\Df[\nu](\Df)^{-1}(\ddx[s]{c}) = - \ddx[s]{\nu}.
\]

\begin{df}
	$\kappa_\nu := \<c'', \nu\>$ heißt \emphdef{Normalkrümmung} von $c$, $\kappa_g := (c'')^{\text{Tang}}$ heißt die \emphdef{geodätische Krümmung}.
	\begin{note}
		Nach Pythagoras ist $\kappa^2 = \|c''\|^2 = \kappa_g^2 + \kappa_\nu^2$.

		Für $\kappa_g = 0$ ergeben sich genau die \emphdef[geodätische Kurve]{geodätische Kurven} (auch \emphdef{Geodätische}).
	\end{note}
\end{df}

\begin{df}
	Die \emphdef[Fundamentalform!zweite]{2. Fundamentalform} wird durch
	\[
		II(X, Y) := I (LX, Y) = I(X, LY)
	\]
	definiert.
	Analog die \emphdef[Fundamentalform!dritte]{3. Fundamentalform} als
	\[
		III(X, Y) := I(L^2X, Y) = I(LX, LY) = I(X, L^2Y).
	\]
\end{df}

\begin{kor}
	\begin{enumerate}[1)]
		\item
			Die Normalkrümmung $\kappa_\nu$ in Richtung $c' = X$ ist durch $II(X, X)$ gegeben:
			\[
				\kappa_\nu = \<c', Lc'\> = II(c', c').
			\]
		\item
			Die 3. Fundamentalform ist gleichzeitig die 1. Fundamentalform $I_\nu$ der Gaußschen-Normalenabbildung $\nu: U \to S^2$ (als Flächenstück betrachtet, sofern sie nicht degeneriert).
			Für $X = \Df(\xi)$ ist
			\[
				III(X, X) = \<LX, LX\> = \<D\nu(\xi), D\nu(\xi)\> = I_\nu(\xi, \xi).
			\]
		\item
			Es gilt der Satz von Cayley-Hamilton:
			\[
				III - \tr(L) II + \det(L) I = 0,
			\]
			$L$ erfüllt sein charakteristisches Polynom.
			\begin{proof}
				Sei $LX = \lambda X, LY = \mu Y$.
				\[
					L^2 - \tr(L) + \det(\Id) = 0
				\]
				einsetzen ergibt
				\begin{align*}
					\lambda^2 X - (\lambda + \mu) \lambda X + \mu \lambda X &= 0 \\
					\mu^2 Y - (\lambda + \mu) \mu Y + \mu \lambda Y &= 0
				\end{align*}
			\end{proof}
	\end{enumerate}
\end{kor}

In Koordinaten ergeben sich die Fundamentalformen als
\begin{align*}
	(g_{ij}) &= I(\ddx[u_i]{f}, \ddx[u_j]{f}) \\
	(h_{ij}) &= II(\ddx[u_i]{f}, \ddx[u_j]{f}) = I(\ddx[u_ji]{f}, -\ddx[u_j]{\nu}) = I(\ddx[u_i]{f}, L \ddx[u_j]{f}) = \< \f{\partial^2 f}{\partial u_i \partial u_j}, \nu \> \\
	(e_{ij}) &= III(\ddx[u_i]{f}, \ddx[u_j]{f}) = \< \ddx[u_i]{\nu}, \ddx[u_j]{\nu} \>
\end{align*}
Wir wollen schreiben $L(\ddx[u]{f}) = \sum_{j} h_i^j \ddx[u_j]{f}$ mit gewissen Koeffizienten $h_i^j$.
Es gilt
\[
	\underbrace{\<L \ddx[u_j]{f}, \ddx[u_k]{f} \>}_{=h_{ik}}
	= \sum_{j} h_i^j \underbrace{\< \ddx[u_j]{f}, \ddx[u_k]{f}\>}_{g_{jk}}
	= \sum_{j} h_i^j g_{jk}
\]
$h_{ij} = \sum_{i} h_i^j g_{jk}$, $h_i^j = \sum_{k} h_{ik} g^{kj}$.
Dann ist $h_{ij}$. Hier fehlt was
\[
	\sum_{k} g_{ik} g^{kj} = \delta_i^l.
\]

\begin{ex}
	Betrachte die Sphäre $S^2(I)$, $f: U\to S^2$.
	Wegen $\nu = \pm f$ ist $L = \mp \Id$.
	Dann ist $II = \pm I, III = I$.
\end{ex}

\begin{df}[Hauptkrümmungen]
	$X$ seie der Einheits-Tangentialvektor in einem festen Punkt, $\|x\| = 1$.
	Dann sind äquivalent:
	\begin{enumerate}[1)]
		\item
			$X$ ist Eigenvektor von $L$ (in diesem Punkt),
		\item
			$II(X, X)$ hat ein Extremum unter allen $X$ mit Nebenbedingung $I(X,X) = 1$.
	\end{enumerate}
	$X$ heißt dann \emphdef{Hauptkrümmungsrichtung} und der Eigenwert $\lambda$ heißt \emphdef{Hauptkrümmung} (extremale Normalkrümmung).
	\begin{proof}
		Nutze die Lagrange-Multiplikatren für Extremwertproblemen unter Nebenbedingungen, wobei $I(X, X) = 1$ die Nebenbedingung und $II(X,X)$ die zu maximierende Zielfunktioni ist.

		$x_1, x_2$ seien kartesische Koordinaten in $U$.
		Die Nebenbedingung ist
		\[
			g(x_1, x_2) = \sum_{i,j} g_{ij} x_i x_j = 1
		\]
		mit konstanter Matrix $g_{ij}$.
		Die Zielfunktion ist
		\[
			h(x_1, x_2) = \sum_{i,j} h_{ij} x_i x_j \to \max
		\]
		Es gilt
		\begin{align*}
			\ddx[x_k]{h} \big|_{(x_1,x_2)} = \ddx[x_k]{f} (\dotsc)
			&= \sum_{i,j} h_{ij} \ddx[x_k]{x_i} x_j  + \sum_{i,j} h_{ij} x_i \ddx[x_k]{x_j} \\
			&= \sum_{j} h_{kj} x_j + \sum_{i} h_{ij} x_i
			= 2\sum_{i} h_{ik} x_i,
		\end{align*}
		Analog ist $\ddx[x_k]{s} = 2 \sum_{i} g_{ik} x_i$.
		Also $\< \grad h|_X, Y \> = 2 II(X, Y) = 2 I(LX, Y)$ und $\<\grad g|_X, Y\> = 2I(X, Y)$
		und somit
		\[
			\grad h = \lambda \grad g
			\iff
			LX = \lambda X.
		\]
	\end{proof}
	\begin{note}
		Bei einer Fläche gibt es zwei Hauptkrümmungen $\kappa_1, \kappa_2$ (Gleichheit ist möglich), deren Hauptkrümmungsrichtungen senkrecht aufeinander stehen:
		für $\kappa_1 \neq \kappa_2$ ist
		\[
			Lx_1 = \kappa_1 X_1, LX_2 = \kappa_2 X_2
		\]
		und
		\[
			\kappa_1 \<X_1, X_2\>
			= \<LX_1, X_2>
			= \<X_1, LX_2\>
			= \kappa_2 \< X_1, X_2,
		\]
		also $\<X_1, X_2 \> = 0$.
	\end{note}
\end{df}

\begin{df}
	Die Determinante $K = \det(L) = \kappa_1 \kappa_2$ heißt \emphdef{Gauß-Krümmung} von $f$.

	Der Mittelwert $H = \f 12 \tr(L) = \f 12 (\kappa_1 + \kappa_2)$ heißt \emphdef{mittlere Krümmung} von $f$.

	Ein Punkt $p$ einer Fläche heißt
	\begin{itemize}
		\item
			\emphdef{elliptisch}, wenn $K(p) > 0$,
		\item
			\emphdef{hyperbolisch}, wenn $K(p) < 0$,
		\item
			\emphdef{parabolisch}, wenn $K(p) = 0$ und $H(p) \neq 0$ oder äquivalent $\rg(L) = 1$.
		\item
			\emphdef{Nabelpunkt}, (engl. “umbilic”), wenn $\kappa_1(p) = \kappa_2(p)$ oder äquivalent $L = \kappa_1 \Id = \kappa_2 \Id$,
		\item
			\emphdef{eigentlicher Nabelpunkt}, wenn $\kappa_1(p) = \kappa_2(p) \neq 0$,
		\item
			\emphdef{Flachpunkt} (engl. “level point”), wenn $\kappa_1(p) = \kappa_2(p) = 0$.
	\end{itemize}
\end{df}

\begin{kor}
	Es gilt stets $H^2 - K = \f 12 (\kappa_1 - \kappa_2)^2 \ge 0$ mit Gleichheit genau für Nabelpunkte.
	In Koordianaten lassen sich $K$ und $H$ nach \ref{??} stets wie folgt ausdrücken:
	\begin{align*}
		K &= \f{\det(h_{ij})}{\det(g_{ij})}
		= \f{h_{11}h_{22} - h_{12}^2}{g_{11}g_{22} - g_{12}^2} \\
		H &= \f 12 \sum_{i} h_{i}^i = \f 12 \sum_{ij} h_{ij} g^{ji}
		= \f 1{2\det(g_{ij})} \big( h_{11} g_{22} - 2h_{12}g_{12} + h_{22}g_{11} \big)
	\end{align*}
\end{kor}

\coursetimestamp{22}{05}{2014}

Anwendung der Mongeschen Koordinaten

\begin{st}
	Ein zusammenhängendes Flächenstück $f: U \to \R^3$ der Klasse $C^2$ besteht genau dann \emph{nur} aus Nabelpunkten, wenn das Bild $f(U)$ in einer Ebene oder einer Sphäre enthalten ist.
	\begin{proof}
		\begin{seg}{$\impliedby$}
			Für eine Ebene ist $\nu$ konstant und $\Df[\nu] = 0$ und $L = 0$, also $\kappa_1 = \kappa_2 = 0$.
			Für eine Sphäre vom Radius $r$ gilt $L = \pm \f 1r \Id$, also $\kappa_1 = \kappa_2 = \f 1r$.
		\end{seg}
		\begin{seg}{$\implies$}
			Es sei $L = \kappa \Id$ mit einer Funktion $\kappa(u, v)$.
			\[
				\iff \Df[\nu](u,v) = - \kappa(u,v) \Df (u,v)
			\]
			Verwende Mongesche Koordinaten: $f(u, v) = (u,v,h(u,v))$.
			\begin{align*}
				\nu_u(u,v) &= -\kappa(u,v) f_u(u, v)
				= -\kappa(u,v)
				(1, 0, h_u) \\
				\nu_i (u,v) &= -\kappa(u,v) f_v(u, v)
				= -\kappa(u,v)
				(0, 1, h_i)
			\end{align*}
			mit
			\[
				\nu = \f {f_u \times f_v}{\|\dotsc\|} = \f 1{\sqrt{1 + h_u^2 + h_v^2}} (-h_u, -h_v, 1)
			\]
			Man erhält
			\begin{align*}
				\partial_u \f {h_u}{\sqrt{\dotsc}}
				&= \kappa(u,v), &
				\partial_v \f {h_u}{\sqrt{\dotsc}}
				&= 0, \\
				\partial_u \f {h_v}{\sqrt{\dotsc}}
				&= 0, &
				\partial_v \f {h_v}{\sqrt{\dotsc}}
				&= \kappa(u,v).
			\end{align*}
			Also ist $\f{h_{u,v}}{\sqrt{\dotsc}}$ eine Funktion nur von $u$, bzw. $v$ abhängig:
			\begin{align*}
				\f{h_u}{\sqrt{\dotsc}} &= a(u), &
				\f{h_v}{\sqrt{\dotsc}} &= b(u).
			\end{align*}
			Dann ist $a'(u) = \kappa(u,v)$, $b'(v) = \kappa(u,v)$, also muss $\kappa$ konstant sein.

			Falls $\kappa = 0$, dann ist $\Df[\nu] = 0$, $\nu$ konstant, also $f$ eine Ebene.

			Falls $\kappa \neq 0$, dann ist $\Df[(\f 1\kappa \nu + f)] = 0$ wegen $\f 1\kappa (\Df[\nu] (\Df)^{-1}) + \Id = 0$
			Also $\f 1\kappa \nu + f$ konstant (Mittelpunkt der Sphäre von Radius $\f 1{|\kappa|})$.
		\end{seg}
	\end{proof}
	\begin{note}
		Wir haben gezeigt: $\kappa_1 = \kappa_2 \implies \kappa_1 = \kappa_2 = \const$.

		Ist $f \in C^3$, so kann man für die Hinrichtung des Beweises auch $\kappa$ ableiten.
	\end{note}
\end{st}









