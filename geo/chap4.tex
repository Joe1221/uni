\chapter{Lokale Flächentheorie}



\begin{df}
	Ein \emphdef[Flächenstück!parametrisiertes]{parametrisiertes (reguläres Flächenstück)} ist eine differenzierbare Abbildung $f: U \to \R^3 \isomorphic E^3$, wobei $(\ddx[u_j]{f_i})$ maximalem Rang hat.
	$f$ ist eine \emphdef{Immersion}.

	Elemente $u = (u_1, u_2)$ von $U$ nennen wir \emphdef{Parameter}, Elemente von $f(U)$ \emph{Punkte}, $f(u_1, u_2) = (x_1(u_1, u_2), x_2(u_1, u_2), x_3(u_1, u_2))$.

	Ein unparametrisiertes (reguläres) \emphdef[Flächenstück!unparametrisiert]{Flächenstück} ist eine Äquivalenzklasse von parametrisierten Flächenstücken, wobei $f \sim f \Phi$, mit
	\[
		\begin{tikzcd}[column sep=tiny]
			\tilde U \arrow{rr}{\Phi} \arrow{dr}[swap]{f \cdot \Phi}  & &  U \arrow{dl}{f} \\
														 & \R^3 &
		\end{tikzcd}
	\]
	wobei $\Phi$ bijektiv und in beiden Richtungen differenzierbar.
\end{df}

\begin{ex}
	Eine Funktion der Form
	\[
		f(u,v) = \Vector{ u & v & z(u,v)}
	\]
	für differenzierbares $z : U \to \R^3$ definiert stets ein reguläres Flächenstück, denn
	\[
		f_u = \Vector{1 & 0 & z_u},
		f_v = \Vector{0 & 1 & z_i}
	\]
	sind linear unabhängig.
\end{ex}

\begin{ex}
	\begin{itemize}
		\item
			Die $2$-Sphäre ohne Äquator ist gegeben durch
			\[
				f(u,v) = \Vector*{ u & v & \pm \sqrt{1 - u^2 - v^2} }.
			\]
			für $u^2 + v^2 < 1$.
			Alternativ
			\[
				f(\phi, \theta) = \Vector{ \cos \phi \cos \theta & \sin \phi \cos \theta & \sin \theta }
			\]
			mit $0 < \phi < 2\pi$ und $-\f \pi 2 < \theta < \f \pi 2$.
			$f$ ist eine Immersion, denn
			\[
				f_\phi = \Vector{-\sin \phi \cos \theta & \cos \phi \cos \theta & 0}
				f_\theta = \Vector{-\cos \phi \sin \theta & - \sin \phi \sin \theta & \cos \theta}
			\]
			sind wegen $\cos \theta \neq 0$ linear unabhängig.
			Die Koordinaten brechen allerdings im Nord- und Südpol zusammen.

			Es gilt
			\[
				f_\phi \times f_\theta
				= \Vector{\cos \phi \cos^2 \theta & \sin \phi \cos^2 \theta & \sin \theta \cos \theta}
				\neq 0.
			\]
			Anschaulich gesprochen sind $f_\phi, f_\theta$ sind tangential zur Sphäre und $f_\phi \times f_\theta$ ist normal zur Sphäre.
		\item
			Sei
			\[
				f(t,x) := \Vector{t^2 & x^3 & x}
			\]
			dann ist $f_t = \Vector{2t & 3t^2 & 0} = 0$ für $t = 0$, also nicht regulär.
	\end{itemize}
\end{ex}

\begin{conv}
	Für einen Punkt $p \in \R^3$ sei $(p, x)$ der Tangentialvektor in $p$ in Richtung $x$.
	Der \emphdef{abstrakte Tangentialraum} ist definiert als
	\[
		T_p\R^3 := \{p\} \times \R^3 \isomorphic \R^3
	\]
	Für $u = (u_1, u_2) \in U$ analog $T_u U = T_u \R^2 = \{ u \} \times \R^2 \isomorphic \R^2$ der Tangentialraum an $U \subset \R^2$.
	\[
		T_u f
		:= Df\big|_u (T_u U)
		\subset T_{f(u)} \R^3
	\]
	der \emphdef{Normalenraum} an der Fläche
	\[
		T_u f \oplus \orth_u f = T_{f(u)} \R^3
	\]
	$T_u f$ wird aufgespannt von $\ddx[u_1]{f}, \ddx[u_2]{f}$ und
	$\orth_u f$ wird aufgespannt von $\ddx[u_1]{f} \times \ddx[u_2]{f}$.
\end{conv}

\begin{ex}
	Definiere den \emphdef{Rotationstorus} durch
	\[
		f(u, \phi) := \Vector*{ (a+b \cos u) \cos \phi & (a + b \cos u) \sin \phi & b \sin u}
	\]
	mit $0 < \phi, u < 2\pi$ und $0 < b < a$.
	Diese Fläche ist regulär. \Exercise

	Ähnlich wie die Sphäre durch die Gleichung $x^2 + y^2 + z^2 = r^2$ gegeben ist, besitzt auch der Rotationstorus eine:
	\[
		(a^2 - b^2 + x^2 + y^2 + z^2)^2 = 4 a^2 (x^2 + y^2).
	\]
	Man verifiziert dies mit der Substitution $z^2 = b^2 \sin^2 u$ und $x^2 + y^2 = (a + b \cos u)^2$. \Exercise
\end{ex}

\paragraph{metrische Verhältnisse innerhalb eines Flächenstücks}

Die Läche eine Tangentialvektors $x \in T_v f$ oder $x \in T_p \R^3$ ist
\[
	\|x\| = \sqrt{\<x, x\>}
\]
Die Länge einer Kurve $[a,b] \xrightarrow{c} U \xrightarrow{f} \R^3$ ist
\[
	\int_a^b \| (f \circ c)^\cdot (t)\| \dx[t]
	= \int_a^b \sqrt{\< Df \dot c, Df \dot c\>} \dx[t]
\]
mit $\dot c \in T_{c(t)} U$, $(Df) = (\ddx[u_j]{f_i})_{i,j}$
\[
	Df \dot c = \Matrix*{ \ddx[u_1]{x_1} & \ddx[u_2]{x_1} \\ \ddx[u_1]{x_2} & \ddx[u_2]{x_2} \\ \ddx[u_1]{x_3} & \ddx[u_2]{x_3} } \Vector{\dot u_1 & \dot u_2}.
\]
Es gilt
\[
	\<Df \dot c, Df \dot c\>
	= (Df \dot c)^T (Df \dot c)
	= \dot c^T  Df^T Df \dot c
\]
mit
\[
	Df^T Df = \Matrix*{\< \ddx[u_1]{f}, \ddx[u_1]{f}\> & \<\ddx[u_1]{f}, \ddx[u_2]{f}\> \\ \<\ddx[u_2]{f}, \ddx[u_1]{f}\> & \<\ddx[u_2]{f}, \ddx[u_2]{f}\> },
\]
einer \emph{symmetrischen Billinearform} mit maximalem Rang.

\begin{df}
	Die \emphdef[Fundamentalform!erste]{erste Fundamentalform} $I$ von $f: V \to \R^3$ ist die Einschränkung des euklidischen Skalarprodukts auf alle $T_u f$:
	\[
		I(X, Y) := \< X, Y \>
	\]
	mit $X, Y \in T_u f$.

	In parametrisierter Form haben wir eine symmetrische Billinearform auf $T_u U$:
	\[
		T_u U \times T_u U \ni (V, W) \mapsto \big\< Df|_u (V), Df|_u (W) \big\> = I(V, W)
	\]
	mit $V, W \in T_u U$.

	In Koordinaten gegeben durch
	\[
		Df^T Df = \Matrix*{\< \ddx[u_1]{f}, \ddx[u_1]{f}\> & \<\ddx[u_1]{f}, \ddx[u_2]{f}\> \\ \<\ddx[u_2]{f}, \ddx[u_1]{f}\> & \<\ddx[u_2]{f}, \ddx[u_2]{f}\> }
		= \Matrix{E & F \\ F & G}
		= \Matrix{g_{11} & g_{12} \\ g_{21} & g_{22}}.
	\]
	Dies ist die Matrix der Ersten Fundamentalform.

	Schreibweise auch
	\[
		\dx[s]^2 = E \dx[y]^2 + 2 E \dx[u_1] \dx[u_2] + G \dx[u_2]^2.
	\]
	Dann ist
	\[
		\| (f \circ c)^\cdot \| = \sqrt{E \dot u_1^2 + 2 F \dot u_1 \dot u_2 + G \dot u_2^2}.
	\]
\end{df}

\begin{ex}
	Betrachte wieder die Sphäre:
	\[
		f_\phi = \Vector{-\sin \phi \cos \theta & \cos \phi \cos \theta & 0},
		f_\theta = \Vector{-\cos \phi \sin \theta & - \sin \phi \sin \theta & \cos \theta}
	\]
	Es gilt
	\begin{align*}
		E &= \< f_\phi, f_\phi\> = \cos^2 \theta \\
		F &= \< f_\phi, f_\theta \> = 0 \\
		G &= \< f_\theta, f_\theta \> = 1
	\end{align*}
	und damit $\Matrix{E & F \\ F & G} = \Matrix{\cos^2 \theta & 0 \\ 0 & 1}$.

	$f_\phi, f_\theta$ sind orthogonal für $\theta = 0$.
	$f_\phi$ wird verkürzt für $\theta \to \pm \f \pi2$.
\end{ex}
