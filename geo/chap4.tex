\chapter{Lokale Flächentheorie}



\begin{df}
	Ein \emphdef[Flächenstück!parametrisiertes]{parametrisiertes (reguläres Flächenstück)} ist eine differenzierbare Abbildung $f: U \to \R^3 \isomorphic E^3$, wobei $(\ddx[u_j]{f_i})$ maximalem Rang hat.
	$f$ ist eine \emphdef{Immersion}.

	Elemente $u = (u_1, u_2)$ von $U$ nennen wir \emphdef{Parameter}, Elemente von $f(U)$ \emph{Punkte}, $f(u_1, u_2) = (x_1(u_1, u_2), x_2(u_1, u_2), x_3(u_1, u_2))$.

	Ein unparametrisiertes (reguläres) \emphdef[Flächenstück!unparametrisiert]{Flächenstück} ist eine Äquivalenzklasse von parametrisierten Flächenstücken, wobei $f \sim f \Phi$, mit
	\[
		\begin{tikzcd}[column sep=tiny]
			\tilde U \arrow{rr}{\Phi} \arrow{dr}[swap]{f \cdot \Phi}  & &  U \arrow{dl}{f} \\
														 & \R^3 &
		\end{tikzcd}
	\]
	wobei $\Phi$ bijektiv und in beiden Richtungen differenzierbar.
\end{df}

\begin{ex}
	Eine Funktion der Form
	\[
		f(u,v) = \Vector{ u & v & z(u,v)}
	\]
	für differenzierbares $z : U \to \R^3$ definiert stets ein reguläres Flächenstück, denn
	\[
		f_u = \Vector{1 & 0 & z_u},
		f_v = \Vector{0 & 1 & z_i}
	\]
	sind linear unabhängig.
\end{ex}

\begin{ex}
	\begin{itemize}
		\item
			Die $2$-Sphäre ohne Äquator ist gegeben durch
			\[
				f(u,v) = \Vector*{ u & v & \pm \sqrt{1 - u^2 - v^2} }.
			\]
			für $u^2 + v^2 < 1$.
			Alternativ
			\[
				f(\phi, \theta) = \Vector{ \cos \phi \cos \theta & \sin \phi \cos \theta & \sin \theta }
			\]
			mit $0 < \phi < 2\pi$ und $-\f \pi 2 < \theta < \f \pi 2$.
			$f$ ist eine Immersion, denn
			\[
				f_\phi = \Vector{-\sin \phi \cos \theta & \cos \phi \cos \theta & 0}
				f_\theta = \Vector{-\cos \phi \sin \theta & - \sin \phi \sin \theta & \cos \theta}
			\]
			sind wegen $\cos \theta \neq 0$ linear unabhängig.
			Die Koordinaten brechen allerdings im Nord- und Südpol zusammen.

			Es gilt
			\[
				f_\phi \times f_\theta
				= \Vector{\cos \phi \cos^2 \theta & \sin \phi \cos^2 \theta & \sin \theta \cos \theta}
				\neq 0.
			\]
			Anschaulich gesprochen sind $f_\phi, f_\theta$ sind tangential zur Sphäre und $f_\phi \times f_\theta$ ist normal zur Sphäre.
		\item
			Sei
			\[
				f(t,x) := \Vector{t^2 & x^3 & x}
			\]
			dann ist $f_t = \Vector{2t & 3t^2 & 0} = 0$ für $t = 0$, also nicht regulär.
	\end{itemize}
\end{ex}

\begin{conv}
	Für einen Punkt $p \in \R^3$ sei $(p, x)$ der Tangentialvektor in $p$ in Richtung $x$.
	Der \emphdef{abstrakte Tangentialraum} ist definiert als
	\[
		T_p\R^3 := \{p\} \times \R^3 \isomorphic \R^3
	\]
	Für $u = (u_1, u_2) \in U$ analog $T_u U = T_u \R^2 = \{ u \} \times \R^2 \isomorphic \R^2$ der Tangentialraum an $U \subset \R^2$.
	\[
		T_u f
		:= Df\big|_u (T_u U)
		\subset T_{f(u)} \R^3
	\]
	der \emphdef{Normalenraum} an der Fläche
	\[
		T_u f \oplus \orth_u f = T_{f(u)} \R^3
	\]
	$T_u f$ wird aufgespannt von $\ddx[u_1]{f}, \ddx[u_2]{f}$ und
	$\orth_u f$ wird aufgespannt von $\ddx[u_1]{f} \times \ddx[u_2]{f}$.
\end{conv}

\begin{ex}
	Definiere den \emphdef{Rotationstorus} durch
	\[
		f(u, \phi) := \Vector*{ (a+b \cos u) \cos \phi & (a + b \cos u) \sin \phi & b \sin u}
	\]
	mit $0 < \phi, u < 2\pi$ und $0 < b < a$.
	Diese Fläche ist regulär. \Exercise

	Ähnlich wie die Sphäre durch die Gleichung $x^2 + y^2 + z^2 = r^2$ gegeben ist, besitzt auch der Rotationstorus eine:
	\[
		(a^2 - b^2 + x^2 + y^2 + z^2)^2 = 4 a^2 (x^2 + y^2).
	\]
	Man verifiziert dies mit der Substitution $z^2 = b^2 \sin^2 u$ und $x^2 + y^2 = (a + b \cos u)^2$. \Exercise
\end{ex}

\paragraph{metrische Verhältnisse innerhalb eines Flächenstücks}

Die Läche eine Tangentialvektors $x \in T_v f$ oder $x \in T_p \R^3$ ist
\[
	\|x\| = \sqrt{\<x, x\>}
\]
Die Länge einer Kurve $[a,b] \xrightarrow{c} U \xrightarrow{f} \R^3$ ist
\[
	\int_a^b \| (f \circ c)^\cdot (t)\| \dx[t]
	= \int_a^b \sqrt{\< Df \dot c, Df \dot c\>} \dx[t]
\]
mit $\dot c \in T_{c(t)} U$, $(Df) = (\ddx[u_j]{f_i})_{i,j}$
\[
	Df \dot c = \Matrix*{ \ddx[u_1]{x_1} & \ddx[u_2]{x_1} \\ \ddx[u_1]{x_2} & \ddx[u_2]{x_2} \\ \ddx[u_1]{x_3} & \ddx[u_2]{x_3} } \Vector{\dot u_1 & \dot u_2}.
\]
Es gilt
\[
	\<Df \dot c, Df \dot c\>
	= (Df \dot c)^T (Df \dot c)
	= \dot c^T  Df^T Df \dot c
\]
mit
\[
	Df^T Df = \Matrix*{\< \ddx[u_1]{f}, \ddx[u_1]{f}\> & \<\ddx[u_1]{f}, \ddx[u_2]{f}\> \\ \<\ddx[u_2]{f}, \ddx[u_1]{f}\> & \<\ddx[u_2]{f}, \ddx[u_2]{f}\> },
\]
einer \emph{symmetrischen Billinearform} mit maximalem Rang.

\begin{df}
	Die \emphdef[Fundamentalform!erste]{erste Fundamentalform} $I$ von $f: V \to \R^3$ ist die Einschränkung des euklidischen Skalarprodukts auf alle $T_u f$:
	\[
		I(X, Y) := \< X, Y \>
	\]
	mit $X, Y \in T_u f$.

	In parametrisierter Form haben wir eine symmetrische Billinearform auf $T_u U$:
	\[
		T_u U \times T_u U \ni (V, W) \mapsto \big\< Df|_u (V), Df|_u (W) \big\> = I(V, W)
	\]
	mit $V, W \in T_u U$.

	In Koordinaten gegeben durch
	\[
		Df^T Df = \Matrix*{\< \ddx[u_1]{f}, \ddx[u_1]{f}\> & \<\ddx[u_1]{f}, \ddx[u_2]{f}\> \\ \<\ddx[u_2]{f}, \ddx[u_1]{f}\> & \<\ddx[u_2]{f}, \ddx[u_2]{f}\> }
		= \Matrix{E & F \\ F & G}
		= \Matrix{g_{11} & g_{12} \\ g_{21} & g_{22}}.
	\]
	Dies ist die Matrix der Ersten Fundamentalform.

	Schreibweise auch
	\[
		\dx[s]^2 = E \dx[y]^2 + 2 E \dx[u_1] \dx[u_2] + G \dx[u_2]^2.
	\]
	Dann ist
	\[
		\| (f \circ c)^\cdot \| = \sqrt{E \dot u_1^2 + 2 F \dot u_1 \dot u_2 + G \dot u_2^2}.
	\]
\end{df}

\begin{ex}
	Betrachte wieder die Sphäre:
	\[
		f_\phi = \Vector{-\sin \phi \cos \theta & \cos \phi \cos \theta & 0},
		f_\theta = \Vector{-\cos \phi \sin \theta & - \sin \phi \sin \theta & \cos \theta}
	\]
	Es gilt
	\begin{align*}
		E &= \< f_\phi, f_\phi\> = \cos^2 \theta \\
		F &= \< f_\phi, f_\theta \> = 0 \\
		G &= \< f_\theta, f_\theta \> = 1
	\end{align*}
	und damit $\Matrix{E & F \\ F & G} = \Matrix{\cos^2 \theta & 0 \\ 0 & 1}$.

	$f_\phi, f_\theta$ sind orthogonal für $\theta = 0$.
	$f_\phi$ wird verkürzt für $\theta \to \pm \f \pi2$.
\end{ex}

\coursetimestamp{15}{05}{2014}

\begin{lem}
	Bei einer Parametertransformation $\tilde f = f \circ \Phi$ ergibt sich für $(g_{ij}), (\tilde g_{ij})$
	\[
		(g_{ij}) = (D\Phi)^T(g_{ij}(D\Phi)
	\]
	\begin{proof}
		Es gilt
		\begin{align*}
			\tilde g_{ij} &= (D \tilde f)^T (D \tilde f)
			= (Df \circ D \Phi)^T (Df \circ D\Phi)
			= (D\Phi)^T \underbrace{\big((Df)^T Df\big)}_{=(g_{ij})} \circ D\Phi
		\end{align*}
	\end{proof}
\end{lem}

Es gilt $\det(\tilde g_{ij}) = \det(g_{ij}) (\det(D\Phi))^2$.

$g := \det(g_{}∫ij)$ heißt \emphdef{Gramsche Determinante}
\[
	\tilde g = g (\det(D\Phi))^2
\]
Die \emphdef{Gramsche Matrix} ist
\[
	\Matrix{\<x,x\> & \<x,y\> \\ \<y,x\> & \<y,y\>}
\]
mit $\det() = \|x\|^2 \|y\|^2 \sin^2 \angle$, also ist
\[
	\sqrt{\det}
	= \|x\|\|y\| |\sin \alpha|
\]
der Flächeninhalt des Paralallograms.

Für eine lineare Abbildung $F$ ist die Gramsche Determinante
\[
	\<Fx, Fx\> <Fy, Fy> - \<Fx, Fy\>^2
\]
das Quadrat der Flächenverzerrung von $F$.
Damit ist $\sqrt{g} = \sqrt{\det(g_{j})}$ die Flächenverzerrung der Abbildung $f$.

\begin{df}[Oberflächenintegral]
	Sei $f: U \to \R^3$ injektiv, $\alpha$ sei stetig und auf ganz $f(U)$ definiert.
	Dann ist für jedes Kompaktum $Q \subset U$
	\[
		\iint_{f(Q)} \alpha \dx[A]
		:= \iint_{Q} (\alpha \circ f)(u_1, u_2) \sqrt{g} \dx[(u_1, u_2)]
	\]
	das \emphdef{Oberflächenintegral} von $\alpha$ erklärt.
	$\alpha = 1$ liefert den Flächeninhalt.
	$\dx[A] = \sqrt{g} \dx[(u_1, u_2)]$ heißt \emphdef{Flächenelement}.
	\begin{note}
		Diese Definition ist unabhängig von der Parametrisierung, denn für $\tilde f = f \circ \Phi, Q = \Phi(\tilde Q), (u_1,u_2) = \Phi(\tilde u_1, \tilde u_2)$ ist
		\begin{align*}
			\iint_{\tilde f(Q)} \alpha \dx[\tilde A]
			&= \iint_{\tilde Q} (\alpha \circ \tilde f)(\tilde u_1, \tilde u_2) \underbrace{\sqrt{g}}_{=|\det D\Phi| \sqrt{g}} \dx[(\tilde u_1, \tilde u_2)] \\
			&= \iint_{Q} (\alpha \circ f) (u_1, u_2) \sqrt{g} \dx[(u_1, u_2)] \\
			&= \iint_{f(Q)} \alpha \dx[A]
		\end{align*}
		per Substitution.
	\end{note}
\end{df}

\section{Vektorfelder}


Grundsätzlich ist jede Zuordnung
\[
	p \mapsto (p, x) \in T_p\R^3
\]
ein \emphdef{Vektorfeld}.

\begin{df}[Vektorfeld längs $f: V \to \R^3$]
	ein Vektorfeld längs $f$ ist eine Zuordnung
	\[
		U \ni u \mapsto (f(u), x(u)) \in T_{f(u)} \R^3
	\]
	Analog definiert man stetige und differenzierbare Vektorfelder für stetiges, bzw. differenzierbares $x$.

	Man nennt das Vektorfeld \emphdef{Vektorfeld!tangential}, wenn $x(x) \in T_u(f)$ für alle $u \in U$ und \emphdef{Vektorfeld!normal}, wenn $x(u) \in \orth_u f$ für alle $u \in U$.
\end{df}

\begin{ex}
	$\ddx[u_1]{f}, \ddx[u_2]{f}$ sind tangentiale Vektorfelder längs $f$.
	$\ddx[u_1]{f} \times \ddx[u_2]{f}$ ist ein normales Vektorfeld längs $f$.
\end{ex}

\begin{df}
	Wir nennen
	\[
		\nu := \dfrac{\ddx[u_1]{f} \times \ddx[u_2]{f}}{\|\ddx[u_1]{f} \times \ddx[u_2]{f}}
		\in \orth_u \subset T_{f(u)}\R^3.
	\]
	Die Einheitsnormale ist nicht eindeutig, $-\nu$ ist auch eine Einheitsnormale.
\end{df}

Wann ist eine 2-dimensionale Untermannigfaltigkeit des $R^3$ orientierbar?
Genau dann, wenn es ein stetiges Einheitsnormalenvektorfeld gibt (Forster, Analysis 3). % fixme: ref

\[
	\ddx[u_1]{f}, \ddx[u_2]{f}, \dfrac{\ddx[u_1]{f} \times \ddx[u_2]{f}}{\|\ddx[u_1]{f} \times \ddx[u_2]{f}}
\]
ist positiv orientiert, Umdrehen von $\ddx[u_1]{f}, \ddx[u_2]{f}$ ergibt die andere Orientierung.
$(u_1, u_2) \mapsto \Phi(u_1, u_2)$ mit positivem $\det D\Phi$ ?

Innerhalb von $U \subset \R^2$ gibt es eine feste orientierung durch Wahl einer Reihenfolge $u_1, u_2$.
Für die Bilmenge $f(U)$ braucht das nicht zu gelten (falls $f$ nicht injektiv ist).

\begin{ex}[Möbiusband]
	Setze
	\[
		f(u,v) := \Vector*{\sin u + v \sin \f{u}2 \sin u & \cos u + v \sin \f{u}2 \cos u & v \cos \f {u}2}.
	\]
	Für $v = 0$ ist $f(u, 0) = \Vector{\sin u & \cos u & 0}$ ein Kreis.
	Für $u = \const$ ergibt sich ein Geradenstück durch $f(u, 0)$.
	$f$ ist regulär für „kleine“ $|v|$ und injektiv für $0 \le u < 2\pi$.
	Für $v \neq 0, |v|$ klein ist
	\[
		v \cos \f u2 = v \cos {\tilde u}2
		\implies u = \tilde u
	\]
	aber $f(2\pi, v) = f(0, -v) = (0,1, -v)$, nicht injektiv.
	$\ddx[u]{f}, \ddx[v]{f}$ liefert Orientierung bei $v = 0$, aber liefert eine andere Orientierung bei $v = 2\pi$, weil $\ddx[v]{f}\big|_{u=2\pi} = - \ddx[v]{f}\big|_{u=0}$.
	Die Einheitsnormalen in $f(2\pi, v)$ und $f(0, -v)$ unterscheiden sich.

	Das Möbiusband ist also als Untermannigfaltigkeit des $\R^3$ \emph{nicht orientierbar}.
\end{ex}

Lokal ist Orientierbarkeit kein Problem (wähle $U$ hinreichend klein).
Die weiteren Betrachtungen sind zunächst lokal.
Wir benötigen 2. Ableitungen, sei also $f$ zweimal stetig differenzierbar.

\begin{df}[Gauß-Abbildung]
	Für $f: U \to \R^3$ ist die \emphdef{Gaußsche Normalenabbildung} oder \emphdef{Gauß-Abbildung} $\nu: U \to S^2$ erklärt als
	\[
		\nu(u_1, u_2)0:= \dfrac{\ddx[u_1]{f} \times \ddx[u_2]{f}}{\|\dfrac{\ddx[u_1]{f} \times \ddx[u_2]{f}}{\|\ddx[u_1]{f} \times \ddx[u_2]{f}}\|} \in S^2 \subset \R^3
	\]
	Identifiziere $T_{f(u)} \R^3$ mit $T_{\nu(u)}\R^3$ durch Translation.
\end{df}

Ideal wäre
\begin{align*}
	\ddx[u_1]{\nu} &= \alpha \ddx[u_1]{f} \\
	\ddx[u_2]{\nu} &= \beta \ddx[u_2]{f}.
\end{align*}
Dann wären $\alpha, \beta$ natürliche Krümmungen.
Dies ist allerdings im Allgemeinen nicht der Fall.
Es läuft auf ein Vergleich mit $D\nu$ mit $Df$ hinaus.
\begin{align*}
	Df\big|_U : T_u U  &\bijto T_u f \subset T_{f(U)} \R^3, \\
	D\nu\big|_U : T_u U &\to T_{v(u)} \R^3.
\end{align*}
Nun ist wegen $1 = \<\nu, \nu\>$ gerade $0 = \<\ddx[u_i]{\nu}, \nu\>$ mit $i = 1,2$.
Damit ist $\ddx[u_1]{\nu}, \ddx[u_2]{\nu}$ tangential an der Fläche, also durch Identifikation durch Translation
\[
	D\nu\big|_U : T_u U \to T_u f.
\]
$Df$ hat maximalen Rang, also existiert $(Df)^{-1}: T_\nu f \to T_\nu U$.
$Df, (Df)^{-1}, D\nu$ sind parameterunabhängig.

\begin{df}[Weingartenabbildung]
	Die Komposition
	\[
		L := -D\nu \circ (Df)^{-1}
	\]
	im Punkt $L_u := -D\nu\big|_u \circ (Df\big|_u)^{-1} : T_u f \to T_u f$ heißt \emphdef{Weingartenabbildung}, oder \emphdef{Formoperator}.
	Insbesondere ist $L(\ddx[u_i]{f}) = - \ddx[u_i]{\nu}$.
\end{df}

$L$ ist ein Endomorphismenfeld der Tangentialebene.

\begin{lem}
	$L$ ist unabhängig von der Parametrisierung (bis auf die Wahl von $\nu$, d.h. ein Vorzeichen) und ist selbstadjungiert bezüglich der ersten Fundamentalform.
	\begin{proof}
		Sei $f = f \circ \Phi, \tilde \nu = \pm \nu \circ \Phi$.
		Dann ist
		\begin{align*}
			\tilde L
			&= -(D \tilde \nu) (D \tilde f)^{-1}
			= \mp D\nu D\Phi (Df D\Phi)^{-1} \\
			&= \mp D\nu D\Phi (D\Phi)^{-1} (Df)^{-1}
			= \mp D\nu (Df)^{-1}
			= \pm L
		\end{align*}
	\end{proof}
	Es gilt
	\[
		I(L \ddx[u_i]{f}, \ddx[u_j]{f})
		= \< - \ddx[u_i]{\nu}, \ddx[u_j]{f} \>
		= - \ddx[u_i]\underbrace{\<\nu, \ddx[u_j]{f}\>}_{=0} + \<\nu, \underbrace{\f{\partial^2 f}{\partial u_i \partial u_j}}_{\text{symm. in $i,j$}} \>
	\]
	Dann ist
	\[
		I(L \ddx[u_i]{f}, \ddx[u_j]{f})
		= I(\ddx[u_i]{f}, L \ddx[u_j]{f}),
	\]
	also \emph{selbstadjungiert.}
\end{lem}








