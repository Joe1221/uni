\documentclass{mycourse}
\renewtheorem{thm}{Theorem}[section]
\renewcommand{\thethm}{\arabic{section}.\arabic{thm}}
\newcommand{\tta}{\vartheta}
\newcommand{\eing}{\mathop |}		% "eingeschränkt auf"
\newcommand{\vekt}[2]{\binom{#1}{#2}}   	% 2-dim Vektoren
\newcommand{\kp}{\times} 	 		% Kreuzprodukt


\title{Geometrie}
\author{}

\begin{document}

\maketitle

\tableofcontents
\newpage

\textbf{Was ist Geometrie?}
\begin{itemize}
	\item
		Keine präzise Definition möglich, umfasst verschiedene mathematische \emph{Teilgebiete} (algebraische Geometrie, Differentialgeometrie, \dots ), eine mögliche Beschreibung ist: \\
		\emph{untersucht Größe und Lage von Figuren im Raum und Eigenschaft des Raumes}
	\item
		Das Wort \emph{Geometrie}  setzt sich zusammen aus dem griechischen $ \gamma \eps \omega $ ( = Land) und $ \mu \eps \tau \rho \iota \alpha$ ( = Vermessung)
	\item
		Eine der ältesten Wissenschaften (> 5000 Jahre)
	\item
		Verschiedene \emph{Zugänge}: 
		\begin{itemize}
			\item axiomatisch (Euklid)
			\item analytisch (Descartes): z.B. kartesisches Modell $\R^2$
		\end{itemize}
	\item
		Verschiedene \emph{Geometrien}:
		\begin{itemize}
			\item euklidische Geometrie
			\item hyperbolische Geometrie: Lobatchewski 1829, Bolyai 1848
			\item sphärische Geometrie
			\item Riemannsche Geometrie: Riemann 1854
		\end{itemize}
	\item 
		Wichtiges Konzept: \emph{Symmetrie} \\
		Felix Klein: Erlanger Programm: Klassifikation geometrischer Teildisziplinen nach den zugelassenen \emph{Symmetrien = Transformationen}
		\begin{itemize}
			\item euklidisch: abstandserhaltende Abbildungen
			\item affine Geometrien: Kollinearität, Abstandsverhältnisse
			\item projektive Geometrie: Doppelverhältnisse \\
				 \vdots
			\item Topologie: Homöomorphismen
		\end{itemize}
	\item
		Starke Bezüge zur \emph{Physik}: \emph{Allgemeine Relativitätstheorie} (Differentialgeometrie) \\
		Raumzeit = 4 - dimensionale pseudo-Riemannsche Mannigfaltigkeit
\end{itemize}

\textbf{In dieser Vorlesung:} Einführung in die (elementare) Differentialgeometrie \\
\textbf{Differentialgeometrie}: Studium von \emph{glatten} Objekten (Kurven, Flächen, \dots) mit Hilfe der Differentialrechnung und Integralrechnung.
Glatt bedeutet hierbei differenzierbar und regulär parametrisiert.
Ein zentraler Begriff in der Geometrie ist die \emph{Krümmung}.

\newpage


\chapter{Kurventheorie}

\section{Bogenlänge einer Kurve}

Eine Kurve ist ein in die Ebene oder den Raum abgebildetes Geradenstück. Solche Kurven wreden durch eine Abbildung $c: I \to \R^n$ beschrieben, wobei  $I$ ein Intervall ist. Für $n = 2$ ist es eine ebene Kurve, für $n = 3$ eine Raumkurve.

\begin{ex}[Beispiele für Kurven]
\label{1.1}
\begin{itemize}
	\item Gerade: $ c: \R \to \R^n , c(t) = a + t v $ für $ a, v \in \R^n$
	\item Kreislinie in der Ebene mit Radius $ r > 0$ und Mittelpunkt $(0,0)$: \[ c: \R \to \R^2, \qquad c(t) = \begin{pmatrix} r \cos(t) \\  r \sin(t) \end{pmatrix} \]
	\item Schraubenlinie im $\R^3$: $c(t) = \begin{pmatrix} r \cos(t) \\ r \sin(t) \\  h \cdot t \end{pmatrix}$
	\item Archimedische Spirale: $ c: (0,\infty) \to \R^2 , \quad a > 0, \quad c(t) = \begin{pmatrix} a  t  \cos(t) \\ a  t  \sin(t)) \end{pmatrix}$
\end{itemize}
\end{ex}

Wie kann man die \textbf{Länge} einer Kurve bestimmen? \\
Seien $a,b \in \R$ und $c: [a,b] \to \R^n$ eine stetige Abbildung („stetiger Weg“). \\

\textbf{Nahe liegende Idee}: Approximation durch einbeschriebenen Polygonzug \\
Sei also $ a = t_0 < t_1 < \dots < t_k = b$ eine \emph{Unterteilung} des Intervalls. Die Länge des dadurch eingeschriebenen Polygonzuges ist 
\[
	||c(a) - c(t_1)|| + ||c(t_1) - c(t_2)|| + \dots + ||c(t_{k-1}) - c(b)|| = \sum_{i=0}^{k-1} ||c(t_i) - c(t_{i+1})||
\]
Wählt man eine \emph{Verfeinerung} $ a = s_0 < s_1 < s_2 < \dots < s_l = b$ mit $l > k$ und $\{t_0, \dots t_k\} \subset \{s_0, \dots s_l\}$, so erhält man einen Polygonzug, dessen Länge größer gleich der Länge des ursprünglichen Polygonzuges ist (aufgrund der Dreiecksungleichung). 

\begin{df}
Sei $c: [a,b] \to \R^n$ stetig. Setze 
\[
     L(c) := \sup \Big\{ \sum_{i=0}^{k-1} ||c(t_i) - c(t_{i+1})|| : a = t_0 < t_1 < \dots < t_k = b \Big\}.
\]
\end{df}

Kann es vorkommen, dass $L(c) = \infty$ ist? Ja! 

\begin{ex}[Kochsche Schneeflocke]
Die Kochsche Schneeflocke oder auch Kochsche Kurve entsteht aus einer Folge von Wegen: \\
$c_0$: $L(c_0) = 1$  \\
$c_1$: $L(c_1) = \f{4}{3}$  \\
$c_2$: $L(c_2) = (\f{4}{3})^2$  \\
$c_3$: $L(c_3) = (\f{4}{3})^3$  \\

Für $c_k$ ist $L(c_k) = (\f{4}{3})^k$. Die $c_k$ konvergieren gleichmäßig gegen einen stetigen Limes-Weg $c_{\infty}$ und es gilt $L(c_{\infty}) = \infty$. \\
Beweis: Übung. \\
\fixme[Bilder]
\end{ex}

\begin{note}
\textbf{Vorsicht:} Im Allgemeinen gilt \underline{nicht} $L(c_k) \to L(c_{\infty})$ falls die $c_k$ gleichmäßig gegen $c_{\infty}$ konvergieren.
\begin{ex*}
\[ 2 = L(c_0) = L(c_1) = L(c_2) = L(c_3), \text{aber } L(c_{\infty}) = \sqrt{2}. \]
\end{ex*}
\fixme[Bilder]
\end{note}

\begin{df}
$c: [a,b] \to \R^n$ heißt \emph{rektifizierbar}, wenn $L(c) < \infty$.
\end{df}

Die Kochsche Kurve ist also eine nicht-rektifizierbare Kurve. Welche Kurven sind rektifizierbar?

\begin{st}
\label{1.5}
Sei $c: [a,b] \to \R^n$ stetig differenzierbar. Dann ist $c$ rektifizierbar und es gilt:
\[ 
	L(c) = \int_a^b ||\dot{c} (t)|| dt , \quad \text{wobei } \dot{c} (t) = \begin{pmatrix} \dot c_1(t) \\ \vdots \\ \dot c_n(t) \end{pmatrix} = \begin{pmatrix} \f{dc_1}{dt} (t) \\ \vdots \\  \f{dc_n}{dt} (t) \end{pmatrix}
\]
\begin{proof}
Wir zeigen, dass der Wert $L(c) = \int_a^b ||\dot{c} (t)|| dt$ beliebig genau durch die Länge eines einbeschriebenen Polygonzuges approximiert werden kann. Dies zeigt auch, dass es keinen Polygonzug der Länge $l' > l$ geben kann, denn sonst gäbe es Verfeinerungen der Unterteilung, deren Länge also $\le l$ sein müssten, mit Länge beliebig nahe an $l'$. Nach dem Satz über Riemannsche Summen (z.B. Foster I), angewandt auf das Integral $\int_a^b ||\dot{c} (t)|| dt$, gibt es zu jedem $\eps'$ ein $\delta_0> 0$, so dass für fast jede Unterteilung $a = t_0 < t_1 < \dots < t_k = b$ mit Feinheit kleiner als $\delta_0$ gilt, dass
\[ \Big | \sum_{i=0}^{k-1} ||\dot{c} (t_{i+1})|| \cdot [t_{i+1} - t_i) - l\Big | < \eps'. \]
Die Komponentenfunktionen $\dot(c_j)$ der Ableitung von $c$ sind als stetige Funktionen auf dem kompakten Intervall $[a,b]$ sogar gleichmäßig stetig, d.h. es gibt zu jedem $\eps' > 0$ ein $\delta_j > 0$, so dass gilt:
\[ |s - t| < \delta_j  \rightarrow |\dot{c_j}(s) - \dot{c_j}(t)| < \eps' .\]
Wähle eine Unterteilung der Feinheit $< \min \{ \delta_1, \dots, \delta_n, \delta_0 \}$. Nun gilt für die Länge des Polygonzuges:
\begin{align*}
 \hat l &:= \sum_{i=0}^{k-1} \|c(t_{i+1}) - c(t_i)\|  = \sum_{i=0}^{k-1} \Big \|  \f{c(t_{i+1}) - c(t_i)}{t_{i+1}-t_i} \Big \| \cdot (t_{i+1} - t_i) \\
 &= \sum_{i=0}^{k-1} \Big \| \Big (  \f{c_1(t_{i+1}) - c_1(t_i)}{t_{i+1}-t_i}, \quad \dots \quad , \f{c_n(t_{i+1}) - c_n(t_i)}{t_{i+1}-t_i} \Big ) \Big \| \cdot (t_{i+1} - t_i) \\
 &= \sum_{i=0}^{k-1} \Big \| \Big (  \dot c_1(\xi_{1,i}), \quad \dots \quad , \dot c_n(\xi_{n,i}) \Big ) \Big \| \cdot (t_{i+1} - t_i)
\end{align*}
Die $\xi_{i,j} \in (t_{i+1},t_i)$ existieren nach dem Mittelwertsatz der Differentialrechnung. 
Ersetzen wir in dem obigen Term die $\dot c(\xi_{i,j})$ durch $\dot c(t_{i+1})$, so haben wir im $i$-ten Summanden den Fehler
\begin{align*}
 (t_{i+1} - t_i) \cdot \Big | \| \big( \dot c_1(\xi_{1,i}),  \dots  , \dot c_n(\xi_{n,i}) \big)\| - \| \big( \dot c_1(t_{i+1}),  \dots  , \dot c_n(t_{i+1}) \big)\| \Big| \\
\le (t_{i+1} - t_i) \cdot \underbrace{\big \| \big(   \dot c_1(\xi_{1,i}) - \dot c_1(t_{i+1}),  \dots  , \dot c_n(\xi_{n,i}) - \dot c_n(t_{i+1}) \big) \big \| }_{\sqrt{n} \cdot \eps'}
\end{align*}
gemacht, insgesamt also höchstens den Fehler $\sqrt{n} \cdot \eps' \cdot (b-a)$. Insgesamt folgt damit
\[ |l-l'| < \eps' + \sqrt{n} \, (b-a) \eps' = \eps' \cdot (1 + \sqrt{n} \,  (b-a)). \]
Wähle also  $\eps' < \f{1}{1+\sqrt{n} \, (b-a)}$.
\end{proof}
\end{st}

\begin{ex}[Bogenlänge für die logarithmische Spirale]
\[ c(t) = \begin{pmatrix} e^{\alpha t} \cos(t) \\ e^{\alpha t} \sin(t) \end{pmatrix} , \quad c: \R \to \R^2, \quad \alpha > 0. \]
Erinnerung: \[ L(c \big |_{[a,b]}) = \int_a^b \| \dot c(t) \| dt  \]
Berechne: 
\begin{align*}
 \dot c (t) &= \begin{pmatrix} e^{\alpha t} (\alpha \cos(t) - \sin(t)) \\ e^{\alpha t} (\alpha \sin(t) + \cos(t)) \end{pmatrix}  \\
  \| \dot c(t) \| &= e^{\alpha t} \sqrt{\alpha^2 \cos^2(t) + \sin^2(t) - 2 \alpha \cos(t) \sin(t) + \alpha^2 \sin^2(t) + cos^2(t) + 2 \alpha \sin(t) \cos(t)} \\
 &= e^{\alpha t} \sqrt{(\alpha^2 +  1) (\sin^2(t) + \cos^2(t) )} = e^{\alpha t} \sqrt{\alpha^2 + 1} 
\end{align*}
Nun folgt: 
\[
 L(c \big |_{[a,b]}) =  \int_a^b \| \dot c(t) \| dt  = \sqrt{\alpha^2 + 1}  \int_a^b  e^{\alpha t} dt  
 = \sqrt{\alpha^2 + 1} \; \Big [ \f{1}{\alpha} e^{\alpha t} \Big ]_a^b  = \f{\sqrt{\alpha^2 + 1}}{\alpha} \; (e^{\alpha b} - e^{\alpha a} ) 
\]

\end{ex}

\section{Kurven im $\R^n$}
Wir wollen nun Kurven im $\R^n$ mit Mitteln der Differential- und Integralrechnung studieren. Wir setzen voraus, dass die Kurven durch beliebig oft differenzierbare Wege gegeben sind.

\begin{df}
Sei $I \subseteq \R$ ein Intervall. Eine \textbf{parametrisierte Kurve} ist eine unendlich oft differenzierbare Abbildung $c: I \to \R^n$. Eine parametrisierte Kurve heißt \textbf{regulär}, falls ihr Geschwindigkeitsvektor nirgends verschwindet, d.h. falls $\dot c (t) \neq 0 \; \forall t \in \R$ gilt.
\end{df}

\begin{ex*}
\begin{itemize}
	\item für regulär parametrisierte Kurven: Alle Beispiele in \ref{1.1}, z.B. ist die Gerade $c(t) = a+tv$ regulär, da $\dot c(t) = v \neq 0$.
	\item  für eine \underline{nicht}-regulär parametrisierte Kurve: Die Neilsche Parabel \[ c:\R \to \R^2, t \mapsto (t^2, t^3) \text{ ist nicht regulär, da } \dot c(0) = (0,0). \]
\end{itemize}
\end{ex*}

\begin{note}
Wir setzen meist voraus, dass Kurven regulär parametrisiert sind. \\
Eine parametrisierte Kurve enthält mehr Informationen als nur das Bild $c(I) \subseteq \R^n$ (die sogenannte \emph{Spur} der Kurve), nämlich auch die Information, wie dieses Bild durchlaufen wird (mit welcher Geschwindigkeit, in welche Richtung). Diese zusätzliche Information wird aber in der Geometrie als eher irrelevant angesehen. \\
Man möchte oft auch die Parametrisierung abändern, ohne dabei das Bild der Kurve zu verändern.
\end{note}

\begin{df}
Sei $c: I \to \R^n$ eine parametrisierte Kurve. Eine \textbf{Parameterstransformation} von $c$ ist eine bijektive Abbildung $\phi: J \to I$, wobei $J$ ein weiteres Intervall ist und sowohl $\phi$ als auch $\phi^{-1}$ unendlich oft differenzierbar sind. Die parametrisierte Kurve $\tilde c = c \circ \phi : J \to \R^n$ nennen wir \textbf{Umparametrisierung} von $c$. 
\end{df}

\begin{note}
Wegen $c = \tilde c \circ \phi^{-1} = c \circ \phi \circ \phi^{-1}$ kann man $c$ aus $\tilde c$ wieder zurückgewinnen. Man beachte, dass die Ableitung einer Parametertransformation nirgends verschwindet, denn:
\[ \phi^{-1} \circ \phi(t) = t \qquad \stack{Kettenregel}{\Longrightarrow} \qquad  (\phi^{-1})' (\phi(t)) \cdot \phi'(t) = 1. \]
Also folgt, dass $\phi'(t) \neq 0 \; \forall t \in J$. \\
Daraus folgt, dass \emph{die Umparametrisierung einer regulär parametrisierten Kurve wieder regulär parametrisiert ist}: 
\begin{align*}
\dot{\tilde{c}}(t) = (c \circ \phi)' (t) = \underbrace{\dot c(\phi(t))}_{\neq 0} \cdot \underbrace{\phi'(t)}_{\neq 0} \neq 0 \\
\text{bzw. } \qquad  \f{d \tilde c}{dt} (t) = \f{d(c \circ \phi)}{dt} (t) = \f{dc}{dt} (\phi (t)) \cdot \f{d \phi}{dt}
\end{align*}
Da die Ableitung einer Parametertransformation nirgends verschwindet, gilt nach dem Zwischenwertsatz, dass entweder $\phi'(t) > 0 \; \forall t \in J$ oder $\phi'(t) < 0 \; \forall t \in J$ ist.
\end{note}

\begin{df}
Eine Parametertransformation heißt \textbf{orientierungserhaltend}, falls $\phi'(t) > 0$ gilt, sonst \textbf{orientierungsumkehrend}.
\end{df}

\begin{df}
Eine \textbf{Kurve}  ist eine Äquivalenzklasse von regulären parametrisierten Kurven, wobei zwei Kurven als äquivalent angesehen werden, wenn sie durch Umparametrisierung auseinander hervorgehen. Eine \textbf{orientierte Kurve} ist eine Äquivalenzklasse von regulären parametrisierten Kurven, wobei zwei Kurven als äquivalent angesehen werden, wenn sie durch eine orientierungserhaltende Parametertransformation auseinander hervorgehen.
\end{df}

\begin{note}
\begin{itemize}
	\item Diese Definition drückt aus, dass nicht die Abbildungen $c: I \to \R^n$ unsere Studienobjekte sind, sondern die Kurven, die sie beschreiben.
	\item Eine orientierte Kurve bestimmt genau eine Kurve, und eine Kurve zwei orientierte Kurven.
	\item \emph{Geometrische Begriffe} sind eben solche, die nicht von der Wahl der Parametrisierung abhängen.
	\item Eine Kurve ist im Allgemeinen \underline{nicht} durch ihre Spur gegeben (da es z.B. bei einer Kurve mit mehreren Schleifen beliebig viele Möglichkeiten gibt, selbige zu durchlaufen, und auch die Anzahl und Reihenfolge der Durchläufe nicht festgelegt sind), dies gilt aber, wenn $c$ injektiv ist.
\end{itemize}
\end{note}

Nach Satz \ref{1.5} ist eine auf einem kompakten Intervall parametrisierbare Kurve rektifizierbar. Die \emph{Länge} ändert sich bei Umparametrisierung nicht, allgemeiner gilt:
\begin{st}
Sei $s: [a,b] \to \R^n$ stetig, $[\tilde a, \tilde b]$ ein Intervall und die stetige Abbildung $\phi: [\tilde a, \tilde b] \to [a,b]$ streng monoton, dann gilt für den stetigen Weg $\tilde c: [\tilde a, \tilde b] \to \R^n, \; \tilde c(t) := c(\phi(t))$, dass $L(\tilde c) = L(c) \in \R \cup \{ \infty \}$.
\begin{proof}
Sei $\tilde a = t_0 < t_1 < \dots < t_k = \tilde b$ eine Unterteilung des Intervalls $[\tilde a, \tilde b]$. Falls $\phi$ streng monoton steigend ist, dann ist $a =\underbrace{\phi(t_0)}_{ = s_0} <\underbrace{\phi( t_1)}_{ = s_1} < \dots < \underbrace{\phi(t_k)}_{ = s_k} = b$ eine Unterteilung von $[a,b]$. Ist $\phi$ streng monoton fallend, so ist $a = \phi(t_k) < \phi(t_{k-1}) < \dots < \phi(t_0) = b$ eine Unterteilung. In beiden Fällen gilt 
\[
\sum_{i=0}^{k-1} \| \tilde c(t_{i+1}) - \tilde c(t_i) \| = \sum_{i=0}^{k-1} \| c(s_{i+1}) - c(s_i)\|.
\]
Zu jedem in $\tilde c$ einbeschriebenen Polygonzug gibt es also somit einen von c in gleicher Länge und umgekehrt $\big ( c(s) = \tilde c(\phi^{-1}(s)) \big)$. Also gilt $L(\tilde c) = L(c)$.
\end{proof}
\end{st}

Unter der Vielzahl von Parametrisierungen gibt es auch einige durch ihre Eigenschaften ausgezeichnete, etwa solche, bei denen die Kurve mit Einheitsgeschwindigkeit durchlaufen wird.

\begin{df}
Eine \textbf{nach der Bogenlänge parametrisierte} (n. Bl. par.) Kurve ist eine parametrisierte Kurve $c:I \to \R^n$ mit $\| \dot c(t) \| = 1 \quad \forall t \in \R$.
\begin{note}
Eine \emph{proportional zur Bogenlänge parametrisierte Kurve} $c:I \to \R^n$ ist eine, für die $\| \dot c(t) \|$ konstant ist.
\end{note}
\end{df}

N.Bl.par. Kurven sind natürlich insbesondere regulär ($\| c(t) \| \neq 0$). \\
Für reguläre Kurven gibt es immer eine Umparametrisierung nach der Bogenlänge:
\begin{st}
Zu jeder regulären Kurve $c$ gibt es eine orientierungserhaltende Umparametrisierung $c \circ \phi$ nach Bogenlänge.
\begin{proof}
Sei $c: I \to \R^n$ regulär par. Kurve  und $t_o \in I$. Setze 
\[ \psi(s) := \int_{t_0}^{s} \| \dot c(t) \| dt = L(c \big |_{[t_0,s]}) \quad \text{ für } s \in I. \]
Da der Integrand positiv ist, ist $\psi$ streng monoton wachsend. Also ist $\psi : I \to J := \psi(I)$. Sei $\phi := \psi^{-1}: J \to I$, dann sind $\psi$ und $\phi$ unendlich oft differenzierbar und es gilt:
 \[ \phi'(t) = \f{1}{\psi^{-1}(\phi(t))}= \f{1}{\| \dot c(\phi(t)) \|} \]
Damit folgt mit der Kettenregel:
\[
\| (c \circ \phi)'(t) \| = \| \dot c(\phi(t))\cdot \phi'(t) \| = \f{\| \dot c(\phi(t)) \|}{\| \dot c(\phi(t)) \|} = 1.
\]
D.h. $c \circ \phi$ ist n.Bl.par.
\end{proof} 
\end{st}

\begin{lem}
\label{2.8}
Sind $c_1: I_1 \to \R^n$ und $c_2: I_2 \to \R^n$ Parametrisierungen n.Bl. der selben Kurve, so sit eine zugehörige Parametertransformation $\phi:I_1 \to I_2$ mit $c_1 = c_2 \circ \phi$ von der Form $\phi(t) = t + t_0$ für ein festes $t_0 \in \R$, falls $c_1$ und $c_2$ gleich orientiert sind, und $\phi(t) = -t + t_0$ falls sie entgegengesetzt orientiert sind.
\begin{proof}
Es gilt:
\[
1 = \| \dot c_1(t) \| = \| \dot c_2(\phi(t)) \cdot \phi'(t) \| = \underbrace{\| \dot c_2(\phi(t))\|}_{=1} \cdot \| \phi'(t) \|.
\]
Also ist $\phi(t) = \pm t + t_0$.
\end{proof}
\end{lem}

\begin{note}
Für nicht-reguläre Kurven gibt es im Allgemeinen \emph{keine} Umparametrisierung n.Bl. (z.B. nicht für eine konstante Kurve).
\end{note}

\section{Krümmung ebener Kurven}

Wir betrachten hier \textbf{ebene} Kurven, also $n = 2$. \\
Sei $c: I \to \R^2$ o.B.d.A eine n.Bl.par. Kurve. Wir nennen $v(t) := \dot c(t)$ das \emph{Tangenten(vektor)feld} und definieren das \emph{Normalenfeld} $n(t)$ durch:
\[ 
n(t) := \begin{pmatrix} 0 & -1 \\ 1 & 0 \end{pmatrix} \cdot v(t) = \begin{pmatrix} 0 & -1 \\ 1 & 0 \end{pmatrix} \cdot \begin{pmatrix} \dot c_1(t) \\ \dot c_2(t) \end{pmatrix}  = \begin{pmatrix} - \dot c_2(t) \\ \dot c_1(t) \end{pmatrix}.
\]
$n$ und $v$ sind normiert. \\
Anschaulich: $n(t)$ ist der um $90^{\circ}$ gegen den Uhrzeigersinn gedrehte Tangentenvektor $v(t)$. \\
Dann ist $(v(t), n(t))$ eine (positiv orientierte) Orthonormalbasis des $\R^2$. Da $c$ n.Bl.par. ist, gilt \\ $\< \dot c(t), \dot c(t) \> = 1$ für alle $t \in I$, kurz $\< \dot c, \dot c\> \equiv 1$, wobei $\< \cdot , \cdot \>$ das übliche Euklidische Skalarprodukt \\ $\Big \< \begin{pmatrix} x_1 \\ x_2 \end{pmatrix}, \begin{pmatrix} y_1 \\ y_2 \end{pmatrix} \Big \> = x_1 y_1 + x_2 y_2$ auf dem $\R^2$ bezeichnet. Durch Differenzieren erhalten wir
\begin{align*}
 1 = \< \dot c(t), \dot c(t) \> &= (\dot c_1(t))^2 + (\dot c_2(t))^2 \\
\quad \stack{Ableiten}{\Longrightarrow} \quad 0 = \f{d}{dt} \< \dot c(t), \dot c(t) \> &= 2 \cdot \ddot c_1(t) \cdot \dot c_1(t) + 2 \cdot \ddot c_2(t) \cdot \dot c(t) = 2 \< \ddot c(t), \dot c(t) \>.
\end{align*}


Also steht $\ddot c(t) $ senkrecht auf $\dot c(t) = v(t)$ und ist somit ein Vielfaches von $n(t)$, es gilt also:
\[
\ddot c(t) = \kappa(t) \cdot n(t).
\]

\begin{df}
Sei $c: I \to \R^2$ eine n.Bl.par. Kurve. Dann heißt die Funktion $\kappa : I \to \R$, definiert durch $\ddot c(t) = \kappa(t) \cdot n(t)$, die \emph{Krümmung} von c.
\end{df}
\begin{note}
Es gilt $\kappa(t) = \< \ddot c(t), n(t) \>$, was auch zeigt, dass $\kappa : I \to \R$ beliebig oft differenzierbar ist. \\
Eine andere Berechnungsmöglichkeit, die hier in dieser Vorlesung aber nicht gebraucht und auch nicht benutzt werden wird, ist $\kappa(t) = \f{\det(\dot c(t), \ddot c(t))}{\| \dot c(t) \|^3}$.

\fixme[Zeichnung Rechts- und Linkskurve kappa]
\end{note}

\begin{ex}
\begin{enumerate}[i)]
	\item 
		Sei $v \in \R^2$ ein Einheitsvektor, $a \in \R^2$ ein beliebiger Punkt. Dann ist $c(t) := tv + a$ eine n.Bl.par. Gerade (denn $\dot c(t) \equiv v$) und es gilt $\ddot c(t) = 0$ für alle $t$, und somit gilt $\kappa \equiv 0$.
	\item
		N.Bl.par. Kreislänge $c: I \to \R^2$ mit Radius $r > 0$.
\begin{align*}
c(t) &= \begin{pmatrix} r \cdot \cos(\f{t}{r}) \\ r \cdot \sin(\f{t}{r}) \end{pmatrix} \\
 \intertext{Dann sind: } \qquad  \dot c(t) &= \begin{pmatrix} - \sin(\f{t}{r}) \\ \cos(\f{t}{r}) \end{pmatrix} = v(t) , \qquad n(t) = \begin{pmatrix} - \cos(\f{t}{r}) \\ - \sin(\f{t}{r}) \end{pmatrix} \\
\intertext{und damit folgt: } \qquad  \ddot c(t) &= \begin{pmatrix} - \f{1}{r} \cos(\f{t}{r}) \\ - \f{1}{r} \sin(\f{t}{r}) \end{pmatrix} = \f{1}{r} \begin{pmatrix} - \cos(\f{t}{r}) \\ - \sin(\f{t}{r}) \end{pmatrix} = \f{1}{r} \cdot n(t)
\intertext{Also ist } \kappa(t) &= \f{1}{r}.
\end{align*}
\end{enumerate}
\end{ex}

% Vorlesung 18.4.13
\begin{df}
Als \textbf{euklidische Bewegungen} oder \textbf{Isometrien} des $\R^2$ bezeichnen wir alle abstandserhaltenden Abbildungen $F: \R^2 \to \R^2$, d.h. solche, für die gilt:
\[ \| F(x) - F(y) \| = \| x - y \| \qquad \forall x,y \in \R^2 \]
\end{df}

\begin{st}
Eine Abbildung $F: \R^2 \to \R^2$ ist genau dann eine Isometrie, wenn sie von der Form $x \mapsto Ax + b$ für ein $A \in O(2)$ ist.
\begin{proof}
\begin{itemize}
	\item Mit der Orthogonalität von $A$ ist \[ \| Ax+b - (Ay + b) \| = \| A(x-y) \| = \underbrace{\| A \|}_{ = \; 1} \cdot \| x - y \| = \| x - y \| .  \]
	\item
	Wenn $F:\R^2 \to \R^2$ eine abstandserhaltende Abbildung ist, dann auch $F_0(x) = F(x) - F(0)$ mit $F_0(0) = 0$. Also ist $\| F_0(x) \| = \| F_0(x) - F_0(0) \| = \| x - 0 \| = \| x\|$, d.h. $F_0$ erhält das Skalarprodukt, denn
	\begin{align*} \< x+y , x+y \> &= \< x,x \> + 2 \< x,y \> + \< y,y \>   \\
	\Longleftrightarrow \quad -2 \< x,y \> = \| x\|^2 + \| y\|^2 - \| x+y \|^2  \quad  &\Longleftrightarrow \quad 2 \< x,y \> =  \| x+y \|^2 - \| x\|^2 - \| y\|^2 \\
\intertext{also } \qquad \< F_0(x), F_0(y) \> = \< x,y \>.
	\end{align*}
	Seien $e_1, e_2$ Standardbasisvektoren, $f_1 := F_0(e_1), f_2 = F_0(e_2)$. Da $F_0$ Norm und Skalarprodukt erhält, ist $f_1, f_2$ eine Orthonormalbasis von $\R^2$. Sei $x = x_1 e_1 + x_2 e_2$, dann folgt $ \< F_0(x), f_i \> = \< F_0(x), F_0(e_i) \> = \< x, e_i \> = x_i \quad \Rightarrow \quad F_0(x) = x_1 f_1 + x_2 f_2$ \\ $ \Longrightarrow \quad F_0$ linear und orthogonal.
\end{itemize}
\end{proof}
\end{st}

\begin{df*}
Wir nennen eine euklidische Bewegung \textbf{orientierungserhaltend}, oder \textbf{eigentliche Bewegung}, falls $ \det(F_0) = + 1 $, oder \textbf{orientierungsumkehrend}, falls $ \det(F_0) = - 1 $.
\end{df*}

\begin{st}
\label{3.5}
Sei $c$ eine n.Bl.par. ebene Kurve und $F: \R^2 \to \R^2$ eine eigentliche Bewegung. Dann ist $F \circ c$ ebenfalls n.Bl.par. Kurve, die dieselbe Krümmung hat.
\begin{proof}
Übung.
\end{proof}
\end{st}

\begin{note}
Falls $F$ uneigentliche Bewegung, ändert sich nur das Vorzeichen der Krümmung. Die Krümmung einer ebenen Kurve ist also eine geometrischen Größe, da
\begin{itemize}
	\item invariant unter (eigentlichen) euklidischen Bewegungen
	\item invariant unter Umparametrisierungen (da zunächst nur für n.Bl.par. Kurven definiert), vgl Übungsblatt 2 Aufgabe 3.
\end{itemize}
Im Folgenden wollen wir zeigen, dass $\kappa(t)$ „im Wesentlichen“ die Kurve bestimmt, d.h. bis auf eigentliche euklidische Bewegungen.
\end{note}

Ab jetzt gilt $J = \begin{pmatrix} 0 & -1 \\ 1 & 0 \end{pmatrix}$.

\begin{lem}
\label{3.6}
Sei $\kappa: I \to \R$ stetig, $t_0 \in I$, $v_0 \in \R^2$ mit $\| v_0 \| = 1$. Dann gibt es genau eine differenzierbare Abbildung $ v: I \to S^1$, so dass $\< \dot v, J v(t) \> = \kappa(t)$ und $v(t_0) = v_0$ gilt, nämlich
\[ v(t) = \begin{pmatrix} \cos(\vartheta(t)) \\ \sin(\vartheta(t)) \end{pmatrix}, \]
wobei $\vartheta: I \to \R$ eine Funktion mit $\vartheta(t) = \vartheta_0 + \int_{t_0}^t \kappa(s) ds$ ist, so dass $\begin{pmatrix} \cos(\vartheta_0) \\ \sin(\vartheta_0) \end{pmatrix} = v_0$ ist.
\begin{proof}
Wir zeigen zunächst, dass die wie oben definierte Abbildung $v: I \to S^1$ die behaupteten Eigenschaften erfüllt. Betrachte
\begin{align*}
\< \dot v(t) , J v(t) \> &= \left \< \begin{pmatrix} (\cos \circ \vartheta)' (t) \\ (\sin \circ \vartheta)'(t) \end{pmatrix} , \begin{pmatrix} - \sin \circ \vartheta (t) \\ \cos \circ \vartheta (t) \end{pmatrix} \right \> = \left \< \begin{pmatrix}-\sin(\vartheta (t)) \cdot \vartheta'(t) \\ \cos(\vartheta(t)) \cdot \vartheta'(t) \end{pmatrix} , \begin{pmatrix} - \sin(\vartheta (t)) \\ \cos(\vartheta (t)) \end{pmatrix} \right \> \\
&= \vartheta'(t) \cdot \underbrace{(\sin(\vartheta(t))^2 + \cos(\vartheta(t))^2)}_1 = \kappa(t) 
\end{align*}
Außerdem gilt $v(t_0) = \begin{pmatrix} \cos(\vartheta_0) \\ \sin(\vartheta_0) \end{pmatrix} = v_0$. Dies zeigt die Existenz. Nun die Eindeutigkeit: \\
Sei $\tilde v: I \to S^1$ eine zweite Abbildung mit $\dot{\tilde v} = \kappa J \tilde v$ und $\tilde v (0) = v_0$ (es gilt, dass $\dot v = \kappa J v$ äquivalent zu $\< \dot v, J v \> = \kappa$ ist, da $\dot v$ auf $v$ senkrecht steht). Dann folgt
\begin{align*} 
\dot v = \kappa J v , \quad \dot{\tilde v} = \kappa J \tilde v. \quad \text{Sei } w &:= v - \tilde v, \quad \dot w = \kappa J w. \\
\Rightarrow \quad \f{d}{dt} \| w(t) \|^2 = \f{d}{dt} \< w(t), w(t)  \> &= 2 \< \dot w(t), w(t) \> = \< \kappa J w, w \> = 0. \\
\intertext{Also ist } \| w \| = \text{konst., und da gilt } w(t_0) = v(t_0) - \tilde v(t_0) &= v_0 - v_0 = 0 \quad \Rightarrow w \equiv 0 \Longrightarrow v = \tilde v. 
\end{align*}
\end{proof}
\end{lem}

% Vorlesung 23.4.13
\begin{st}[Hauptsatz der ebenen Kurventheorie]
\label{3.7}
Ist $I \subset \R$  ein Intervall und $\kappa : I \to \R$ eine beliebig oft differenzierbare Funktion, dann gibt es eine n.Bl.par. Kurve $c: I \to \R^2$ mit Krümmung $\kappa$ und zu jeder weiteren n.Bl.par. Kurve $\tilde c: I \to \R^2$ mit Krümmung $\kappa$ gibt es eine \emph{eigentliche Bewegung} $F$ des $\R^2$, so dass $\tilde c = F \circ c$.

\begin{proof}
\textbf{Existenz:}
Sei $t_0 \in I$ und $v_0 := \begin{pmatrix} 1 \\ 0 \end{pmatrix}$. Nach Lemma \ref{3.6} gibt es eine (sogar beliebig oft) differenzierbare Abbildung $v : I \to S^1$ mit $\< \dot v, Jv \> = \kappa$ und $v(t_0) = v_0$. \\
Dann ist
\[ c(t) := \begin{pmatrix}  \int_{t_0}^t v_1(s) ds \\ \int_{t_0}^t v_2(s) ds \end{pmatrix} , \qquad c:I \to \R^2, \]
eine Abbildung mit 
\[ \< \ddot c(t), J \dot c(t) \> = \kappa(t) \quad \forall t \in I. \]
\textbf{Eindeutigkeit:}
Sei nun $\tilde c: I \to \R^2$ eine weitere Kurve mit Krümmungsfunktion $\kappa$.\\ Setze $\tilde c_0 := \tilde c(t_0) \in \R^2, \quad \tilde v_0 := \dot{\tilde c}(t_0) \in S^1$. \\
Definiere eine eigentliche Bewegung des $\R^2$ durch $F(x) = Ax + b$, wobei $b := \tilde c_0$ und $A$ die Matrix $(\tilde v_0, J \tilde v_0)$ ist. Dann ist
\[ \det(A) = \det \begin{pmatrix} \tilde v_0^1 & - \tilde v_0^2 \\ \tilde v_0^2 & \tilde v_0^1 \end{pmatrix} = (\tilde v_0^1)^2 + (\tilde v_0^2)^2 = 1 \]
(siehe Definition der eigentlichen Bewegung). Setze $d := F \circ c, \: d: I \to \R^2$. \\
Nach Satz \ref{3.5} gilt: $ \< \ddot d, J \dot d \> = \< \ddot c, J \dot c \> = \kappa$. Mit Lemma \ref{3.6} folgt $ \dot d = \dot{\tilde c}, \; \dot d(t_0) = A \dot c(t_0) = \dot{\tilde c}(t_0) = \tilde v_0$.
Also folgt
\[ d(t) = \tilde c_0 + \begin{pmatrix}  \int\limits_{t_0}^t \dot{\tilde c_1}(s) ds \\ \int\limits_{t_0}^t \dot{\tilde c_2}(s) ds \end{pmatrix} = \tilde c(t). \]
\end{proof}
\end{st}

Der Satz liefert ein explizites Verfahren, um Kurven mit vorgegebener Krümmungsfunktion zu konstruieren:

\begin{ex}
\begin{enumerate}[i)]
	\item
	 $\kappa \neq 0$ konstant, $\kappa \equiv \f{1}{r}, \; I = \R, \; t_0 = 0$. Setze $\vartheta_0 = 0$. Dann folgt:
	 \[
	 \vartheta(t)=\int_0^t \frac{1}{r} \mathrm d s= \frac{t}{r}, \quad v(t)=\begin{pmatrix} \cos\left (\frac{t}{r}\right ) \\ -r \cdot \cos\left ( \frac t r \right ) \end{pmatrix},
	 \]
	 somit 
	 \[
	 c(t)=\begin{pmatrix} r \cdot \sin\left ( \frac t r \right ) \\ -r \cdot \cos \left ( \frac t r \right ) +r \end{pmatrix}
	 \]
	 Dies liefert also einen Kreis mit Radius r um den Punkt $(r,0)^T$.
	 
	 \item \emph{ $\kappa$ linear:} $\kappa (t) = \alpha t, \; \alpha > 0, \; I = \R, \; t_0 = 0, \; \vartheta_0 = 0$. Dann gilt: 
	 \[ \vartheta(t) = \f{\alpha}{2} t^2 \qquad \Rightarrow \qquad c(t) = \begin{pmatrix} \int\limits_0^t \cos \left (\f{\alpha}{2} s^2 \right ) \mathrm d s \\ \int\limits_0^t \sin \left (\f{\alpha}{2} s^2 \right ) \mathrm d s \end{pmatrix}
	 \]
	 Diese Kurve wird auch \emph{Klotoide}, \emph{Cornu'sche Spirale} oder \emph{Spinnkurve} genannt. Sie wird im Straßenbau eingesetzt als Übergang zwischen Kurven- und Geradenstücken.  
\end{enumerate}
\end{ex}

\section{Tangentendrehzahl und Umlaufsatz}
Wir wollen nun einige globale Resultate über ebene Kurven beweisen. \\
Dazu betrachten wir \emph{geschlossene Kurven}. \\
\fixme[Bilder]

\begin{df}
Eine parametrisierte Kurve $c: \R \to \R^n$ heißt \textbf{periodisch mit Periode $L$}, falls für alle $t \in \R$ gilt $c(t+L) = c(t)$ für ein festes $L > 0$, und es kein $ 0 < L' < L$ gibt, so dass ebenfalls $c(t+L') = c(t) \forall t \in \R$ gelten würde. \\
Eine Kurve heißt \textbf{geschlossen}, wenn sie eine reguläre periodische Parametrisierung besitzt.
\end{df}

\begin{ex*}
\begin{itemize}
	\item $c: \R \to \R^2, \; t \mapsto \left(\cos(t), \sin(t)\right)$ ist periodisch mit Periode $2 \pi$.
	\item $c: \R \to \R^2, \; t \mapsto \left(\cos(t^3), \sin(t^3)\right)$ ist \emph{geschlossen}, aber \emph{nicht periodisch}. Ist $c: \R \to \R^n$ eine Parametrisierung n.Bl. einer geschlossenen Kurve, so ist $c$ periodisch (Übungsaufgabe).
\end{itemize}
\end{ex*}

\begin{df}
Eine geschlossene Kurve heißt \textbf{einfach geschlossen}, falls sie eine periodische reguläre Parametrisierung $c$ mit Periode $L$ hat, so dass $c \big|_{[0,L)}$ injektiv ist.
\end{df}

\begin{note}
Sei $c: \R \to \R^2$ eine n.Bl.par. periodische Kurve mit Periode $L$. Da $\dot c(0) = \dot c(L)$, hat der Tangentenvektor $\dot c(t)$ nach dem Durchlauf des Intervalls $[0,L]$ eventuell eine oder mehrere Umläufe um den Einheitskreis vollführt:
\fixme[Zeichnungen Anzahl Umläufe]

Wir wollen den \emph{Umlaufsatz} beweisen: bei einer {\color{Green} einfach geschlossenen Kurve} ist die hier veranschaulichte \emph{Tangentendrehzahl} entweder gleich {\color{Orange} $-1$} oder gleich {\color{Orange} $+1$}. \\
Dazu benötigen wir ein paar Vorbereitungen, unter anderem, um die Tangentendrehzahl überhaupt definieren zu können.
\end{note}

\begin{df}
\label{4.3}
Sei $X \subset \R^n$ eine Teilmenge und $x_0 \in X$. Dann heißt $X$ \textbf{sternförmig bezüglich $x_0$}, falls für jeden Punkt $x \in X$ die Strecke zwischen $x_0$ und $x$ ganz in $X$ enthalten ist, d.h. \[ \forall x \in X, \; t \in [0,1]: \quad tx + (1-t) x_0 \in X. \]
\fixme[Zeichnungen sternförmig]
\end{df}

% Vorlesung 25.4.13
\begin{lem}[Hochhebungslemma]
\label{4.4}
Sei die kompakte Menge $X \subset \R^n$ sternförmig bzgl. $x_0$. Sei $e : X \to S^1$ eine stetige Abbildung. Dann existiert eine stetige Abbildung $\vartheta: X \to \R$, so dass 
\[ e(x) = \begin{pmatrix} \cos (\vartheta(x)) \\ \sin(\vartheta(x)) \end{pmatrix} \qquad \forall x \in X. \]
Die Abbildung $\vartheta$ ist durch die Vorgabe $\vartheta(x_0) = \vartheta_0$ eindeutig bestimmt.
\begin{note}
Dies ist i.A. falsch für nicht sternförmige Mengen, z.B. $X = S^1$, $e = \text{id}_{S^1}$. Wir nennen im Fall von \ref{4.4} die Abbildung $\tta$ eine \textbf{Hochhebung} von $e$. Eine Hochhebung von $\dot c$ für eine parametrisierte Kurve $c: I \to \R^2$ nennen wir auch \emph{Winkelfunktion} von $c$.
\end{note}
\begin{proof}
\begin{enumerate}[a)]
	\item 
Wir betrachten zunächst den Fall, dass $e$ nicht surjektiv ist, d.h. $\exists \hat \tta \in \R$, so dass \\  $\begin{pmatrix} \cos(\hat \tta) \\ \sin(\hat \tta) \end{pmatrix} \notin e(X) $. Dann ist die Abbildung $\psi_k: \underbrace{(\hat \tta + 2\pi k, \hat \tta + 2 \pi (k+1))}_{=: I_k} \to S^1 \setminus \begin{pmatrix} \cos(\hat \tta) \\ \sin(\hat \tta) \end{pmatrix} $ gegeben durch $t \mapsto \begin{pmatrix} \cos(t) \\ \sin(t) \end{pmatrix}$ ein Homöomorphismus, und $\psi_k^{-1} \circ e =: \tta$ liefert die gewünschte Funktion, wobei $\tta$ durch die Wahl von $\tta_0$ eindeutig bestimmt ist.

	\item 
Wir betrachten einen weiteren Spezialfall, nämlich $n = 1, \; X = [0,1]$ und $x_0 = 0$, aber $e$ surjektiv. \\
Da $e$ auf dem kompakten Intervall gleichmäßig stetig ist, existiert eine äquidistante Unterteilung $a = t_0 < t_1 < \dots < t_k = b$, so dass $e$ auf keinem der Teilintervalle surjektiv ist. Dann gibt es nach a) genau eine stetige Abbildung $\tta: [t_0, t_1] \to \R$ mit $\tta(t_0) = \tta_0$, so dass \\ $e\big |_{[t_0, t_1]}(s) = \begin{pmatrix} \cos(\tta(s)) \\ \sin(\tta(s)) \end{pmatrix}$ gilt. Nun lässt sich $\tta$ induktiv auf die Teilintervalle $[t_i, t_{i+1}]$ fortsetzen, wobei die Eindeutigkeit durch die Vorgabe von $\tta(t_i)$ gewährleistet ist.

	\item 
Sei nun $X \subset \R^n$ eine allgemeine kompakte, sternförmige Menge bzgl. $x_0$. \\  Dann liegt die Strecke von $x_0$ nach $x$ ganz in $X$ und wir können die stetige Abbildung \\ $e_x: [0,1] \to S^1$,  $ e_x(t) = e(tx + (1-t) x_0)$ betrachten. Angenommen, $\exists \tta: X \to \R$ wie in der Behauptung. Nach b) gibt es ein $\tta_x: [0,1] \to \R$ und $e_x(t) = \begin{pmatrix} \cos(\tta_x(t)) \\ \sin(\tta_x(t)) \end{pmatrix}$. Wegen der Eindeutigkeit von $\tta_x$ folgt  $\tta_x(t) = \tta(tx + (1-t) x_0)$, insbesondere $\tta_x(1) = \tta(x)$. Dies zeigt die Eindeutigkeit von $\tta$.

	\item 
Es bleibt die Existenz zu zeigen (mit anderen Worten: \emph{zz:} $\tta$ (wie oben def./konstruiert) ist stetig). \\
Dazu definiere $\tta(x) := \tta_x(t)$. Dann gilt $\begin{pmatrix} \cos(\tta(x)) \\ \sin(\tta(x)) \end{pmatrix} = \begin{pmatrix} \cos(\tta_x(1)) \\ \sin(\tta_x(1)) \end{pmatrix} = e_x(1) = e(x)$ und $\tta(x_0) = \tta_0$. Zeige also, dass $\tta$ stetig ist: \\
Da $X$ kompakt ist, ist $e$ auf $X$ gleichmäßig stetig, d.h. es gibt zu jedem $\eps > 0$ ein $\delta > 0$, so dass $\| x - y \| < \delta \; \Rightarrow \; \| e(x) - e(y) \| < \eps$. Insbesondere gilt für $x,y \in X$ mit $\| x - y \| < \delta$ und alle $t \in [0,1]$, dass $\| e_x(t) - e_y(t) \| < \eps$. \\
Nun seien $\tta_x,  \tta_y$ zwei Abb. $[0,1] \to \R$ mit $\Big \| \begin{pmatrix} \cos(\tta_x(t)) \\ \sin(\tta_x(t)) \end{pmatrix} - \begin{pmatrix} \cos(\tta_y(t)) \\ \sin(\tta_y(t)) \end{pmatrix} \Big \| < \eps \quad \forall t \in [0,1]$. Aus Stetigkeitsgründen (Lipschitzstetigkeit der in a) konstruierten Funktion $\tta$ mit Faktor $\f{\pi}{2}$) folgt, dass die Funktion $\tta_y$ in einem $\f{\pi}{2} \eps$- „Schlauch“ um die Funktion $\tta_x$ verläuft. Es folgt $|\tta(x) - \tta(y)| < \f{\pi}{2} \eps$.
\end{enumerate}
\end{proof}
\end{lem}

\begin{df}
Sei $c: \R \to \R^2$ eine ebene n.Bl.par. Kurve, periodisch mit Periode $L$. Sei $\tta: \R \to \R$ eine Winkelfunktion von $c$ (d.h. wie in \ref{4.4}, wobei $e(t) := \dot c(t)$). Dann heißt 
\[ \eta_c := \f{1}{2 \pi} (\tta(L) - \tta(0)) \]
die \textbf{Tangentendrehzahl} von $c$.
(Die Funktion $\tta$ ist zwar nur bis auf Addition eines ganzzahligen Vielfachen von $2 \pi$ definiert, dieses fällt aber in der Differenz $\tta(L)  - \tta(0)$ weg.) 
\end{df}

\begin{lem}
\label{4.6}
Seien $c_1, c_2: \R \to \R^2$ zwei ebene n.Bl.par. Kurven, periodisch mit Periode $L$, mit $c_1 = c_2 \circ \phi$, wobei $\phi$ eine Parametertransformation ist. Ist $\phi$ orientierungserhaltend, dann gilt $\eta_{c_1} = \eta_{c_2}$, ist $\phi$ orientierungsumkehrend, so ist $\eta_{c_1} = - \eta_{c_2}$.
\begin{proof}
Nach Lemma \ref{2.8} gilt für die Parametertransformation $\phi(t) = \pm t + t_0$, wobei das Vorzeichen die Orientierungserhaltung (+) bzw. -umkehrung (-) bestimmt. \\
Sei $\tta_2$ eine Winkelfunktion von $c_2$. Dann ist im orientierungserhaltenden Fall $\tta_1 = \tta_2 \circ \phi$, denn $\dot c_1(t) = \dot c_2(t + t_0) = \begin{pmatrix} \cos(\tta_2(t + t_0)) \\ \sin(\tta_2(t + t_0)) \end{pmatrix} $. Mit $\tta_1$ ist aber auch $\tilde \tta_1$ eine Winkelfunktion für $c_1$, wobei $\tilde \tta_1(t) := \tta_1(t + L)$. Es folgt:
\begin{align*} 
2 \pi (\eta_{c_2} - \eta_{c_1}) &= (\tta_2(L) - \tta_2(0)) - (\tta_1(L) - \tta_1(0)) = \tta_1(L - t_0) - \tta_1(-t_0) - \tta_1(L) + \tta_1(0)  \\
& = \tilde \tta_1(-t_0) - \tilde \tta_1(0) - (\tta_1(-t_0) - \tta_1(0)) = (\tilde \tta_1(-t_0) - \tilde \tta_1(0)) - (\tta_1(-t_0) - \tta_1(0)) \\ &= 2 \pi \eta_{c_1} - 2 \pi \eta_{c_1} = 0.
\end{align*}
Im orientierungsumkehrenden Fall ist auch die Funktion $\tta_1(t) := \tta_2(- t + t_0)  + \pi$ eine Winkelfunktion für $c_1$, denn
\[ \dot c_1(t) = \dot c_2(\phi(t)) \phi'(t) = - \dot c_2(-t + t_0) = \begin{pmatrix} -\cos(\tta_2(-t+t_0)) \\ -\sin(\tta_2(-t+t_0)) \end{pmatrix} = \begin{pmatrix} \cos(\tta_2(-t+t_0) + \pi) \\ \sin(\tta_2(-t+t_0) + \pi) \end{pmatrix}. \]
Damit folgt
\begin{align*}
 2 \pi (\eta_{c_2} - \eta_{c_1})  &= (\tta_2(L) - \tta_2(0)) + (\tta_1(L) - \tta_1(0)) = \tta_1(-L + t_0) - \tta_1(t_0) + \tta_1(L) - \tta_1(0)  \\
& =  (\tta_1(-L+t_0) - \tta_1(0)) - (\tilde \tta_1(-L+t_0) - \tilde \tta_1(0)) \\ 
&= (\tta_1(t_0) -\tta_1(L)) - (\tta_1(t_0) - \tta_1(L))  = 0.
\end{align*}
\end{proof}
\end{lem}

% Vorlesung 30.4.13
Das Lemma \ref{4.6} hat gezeigt, dass sich die Tangentendrehzahl $n_c$ bei einer orientierungserhaltenden Umparametrisierung nicht ändert. Somit ist die Tangentendrehzahl einer geschlossenen orientierten ebenen Kurve wohldefiniert. Bei einer orientierungsumkehrenden Umparametrisierung ändert sich das Vorzeichen von $n_c$. Ferner folgt aus der Definition $n_c := \f{1}{2 \pi} (\tta(L) - \tta(0))$, dass $n_c$ eine ganze Zahl ist.

\begin{st}
\label{4.7}
Sei $c: \R \to \R^2$ eine n.Bl.par. periodische ebene Kurve mit Periode $L$. Sei $\kappa : \R \to \R$ die Krümmung von $c$. Dann gilt
\[ 2 \pi n_c = \int_0^L \kappa(t) dt .\]
\begin{proof}
Sei $e(t) = \dot c(t)$ und $\tta$ wie in \ref{4.3}. Aus \ref{3.6} folgt, dass $\tta$ eine Stammfunktion von $\kappa$ ist, also gilt \[ 2 \pi n_c = (\tta(L) - \tta(0)) = \int_0^L \kappa(t) dt . \]
\end{proof}
\end{st}

\begin{st}[Umlaufsatz]
\label{4.8}
Eine einfach geschlossene orientierte ebene Kurve hat Tangentendrehzahl $+1$ oder $-1$.

\begin{proof}
\begin{enumerate}[a)]
	\item
		Sei $c$ eine Parametrisierung n.Bl. einer einfach geschlossenen ebenen Kurve, somit ist $c$ periodisch nach Aufgabe 1, Blatt 4. Sei $L$ die Periode. Wir betrachten zunächst die erste Komponente $c_1$ von $c = \begin{pmatrix} c_1 \\ c_2 \end{pmatrix}$, diese nimmt als stetige Funktion auf dem kompakten Intervall $[0, L]$ ihr Maximum an. Sei $x_0 := \max \{ c_1(t) | t \in \R \}$ und $t_0 \in [0, L]$ eine Zahl, so dass $c_1(t0) = x_0$. Nach einer Parametertransformation $t \mapsto t + t_0$ können wir annehmen, dass $c_1(0) = x_0$ gilt. Da die erste Komponente von $c$ nun bei $0$ ihr Maximum annimmt, gilt $\dot c(0) = \begin{pmatrix} 0 \\ \pm 1 \end{pmatrix}$. Indem wir, falls nötig, noch die orientierungsumkehrende Parametertransformation $t \mapsto -t$ anwenden, können wir annehmen, dass $\dot c = \begin{pmatrix} 0 \\ 1 \end{pmatrix}$ gilt (dadurch ändert sich ja nur das Vorzeichen der Tangentendrehzahl). Nun liegt folgende Situation vor: Kein Bildpunkt von $c$ liegt rechts von der Geraden $c(0) + \R \dot c(0)$. Nun können wir für alle $0 \leq t_1 < t_2 < L$ den Einheitsvektor $\f{c(t_2) - c(t_1)}{\| c(t_2) - c(t_1) \|}$ definieren. Dies ist möglich, da die Kurve aufgrund der Einschränkung injektiv und einfach geschlossen ist, also $c(t_2) \neq c(t_1)$ gilt. Der obige Vektor gibt die Richtung einer Sekante an die Kurve durch $c(t_1)$ und $c(t_2)$ an.

	\item
		Sei nun $X = \{ \begin{pmatrix} t_1 \\ t_2 \end{pmatrix} \in \R^2 \; | \; 0 \leq t_1 \leq t_2 \leq L \}$. Wir können den oben definierten Sekantenvektor stetig auf dieser Menge X fortsetzen, indem wir definieren: 
\[
e : X \to S^1 , \qquad e(t_1, t_2 ) := \begin{cases} 
					\displaystyle \f{c(t_2) - c(t_1)}{\| c(t_2) - c(t_1) \|} & \text{, falls } t_1 < t_2 \text{ und } (t_1, t_2) \neq (0, L) \\
					\displaystyle \dot c(t_1) & \text{ , falls } t_1 = t_2 \\
					\displaystyle - \dot c(0)  & \text{, falls } (t_1, t_2) = (0, L). 
\end{cases}
\]
Nach dem Hochhebungslemma \ref{4.4} können wir eine Funktion $\tta: X \to \R$ wählen, so dass für alle $(t_1, 1_2) \in X$ gilt: 
\[ \begin{pmatrix} \cos(\tta(t_1, t_2)) \\ \sin(\tta(t_1, t_2)) \end{pmatrix} = e(t_1, t_2). \]
Wegen $e(t,t) = \dot c(t) $ ist $t \mapsto \tta(t, t)$ eine Winkelfunktion wie in der Definition der Tangentendrehzahl. Deshalb gilt: 
\[ 2 \pi n_c = (\tta(L,L) - \tta(0, 0)) = \underbrace{\tta(L, L) - \tta(0, L)}_d + \underbrace{\tta(0, L) - \tta(0, 0)}_c .\]

	\item
		Wir betrachten die Abbildung $t \mapsto e(0, t)$. Diese nimmt den Wert $\begin{pmatrix} 1 \\ 0 \end{pmatrix}$ nicht an (denn sonst wäre $c(0)$ kein Bildpunkt der Kurve mit maximaler erster Komponente). Also nimmt $t \mapsto \tta(0, t)$ wie im Beweis des Hochhebungslemmas Teil a) nur Werte in einem Intervall der Form $(2 \pi k, 2 \pi (k+1))$ für ein $k \in \Z$ an. Es gilt:
\begin{align*}
 e(0, L) = - \dot c(0) = \begin{pmatrix} 0 \\ -1 \end{pmatrix} \qquad &\Longrightarrow \quad \tta(0, L) = \f{3}{2} \pi + 2 \pi k \\
\intertext{und} e(0, 0) = \dot c(0) = \begin{pmatrix} 0 \\ 1 \end{pmatrix} \qquad &\Longrightarrow \quad \tta(0, 0) = \f{\pi}{2} + 2 \pi k. \\
& \Longrightarrow \quad \tta(0, L) - \tta(0, 0) = \pi.
\end{align*}
Anschaulich: Da der Sekantenvektor nicht nach rechts zeigen kann, muss er eine $180^{\circ}$-Drehung gegen den Uhrzeigersinn vollführen, um von $\begin{pmatrix} 0 \\ 1 \end{pmatrix}$ nach $\begin{pmatrix} 0 \\ -1 \end{pmatrix}$ zu gelangen.

	\item
		Analog ist  $\begin{pmatrix} -1 \\ 0 \end{pmatrix}$ ist nicht im Bild der Abbildung $t \mapsto e(t, L)$ und es folgt $\tta(L, L) - \tta(0, L) = \pi$.

	\item
		Insgesamt gilt also $2 \pi n_c = \pi + \pi = 2 \pi$.
\end{enumerate}
\end{proof}
\end{st}

\section{Konvexe Kurven und Vierscheitelsatz}
\fixme{Bilder konvex und nicht konvex}
\begin{df}
Eine ebene Kurve heißt \textbf{konvex}, falls für jeden ihrer Punkte gilt: Die Kurve liegt ganz auf einer Seite ihrer Tangente durch jeden Punkt, genauer: Ist $c$ eine Parametrisierung n.Bl. und $n$ das Normalenfeld längs $c$, so gilt für jeden Punkt $c(t_0)$ entweder
\[ \< c(t) - c(t_0), n(t_0) \> \geq 0 \; \forall t \qquad \text{ oder } \qquad \< c(t) - c(t_0), n(t_0) \> \leq 0 \; \forall t. \]
\end{df}

\begin{lem}
Sei $c: I  \to \R^2$ eine n.Bl.par. ebene Kurve mit Normalenfeld $n$. Dann ist $c$ genau dann konvex, wenn (für alle $t, t_0 \in I$ gilt:
\[ \< c(t) - c(t_0), n(t_0) \> \geq 0 \; \forall t, t_0 \in I \quad \text{ oder } \quad \< c(t) - c(t_0) , n(t_0) \> \leq 0 \; \forall t, t_0 \in I. \]
\begin{proof}
Sei $c$ konvex und seien $t_1, t_2$ so, dass $\< c(t) - c(t_1), n(t_1) \> \leq 0 \; \forall t \in I$ und $\< c(t) - c(t_2), n(t_2) \> \geq 0 \; \forall t \in I$. Sei o.E. $t_1 < t_2$. Setze $t_3 := \sup \{\tilde t \in [t_1, t_2] \; \big| \; \< c(t) - c(\tilde t), n(\tilde t) \> \geq 0 \; \forall t \in I \}$. Aus Stetigkeitsgründen gilt $\< c(t) - c(t_3), n(t_3) \> = 0 \; \forall t \in I$ und es folgt, dass $c$ eine Gerade ist (und beide Bedingungen erfüllt sind).
\end{proof}
\end{lem}


% Vorlesung 2.5.13

\newpage
% Vorlesung 7.5.13
\setcounter{thm}{7}
Erinnerung: \textbf{Vierscheitelsatz} \\
Ist $c: \R \to \R^2$ eine periodische n.Bl.par. konvexe ebene Kurve mit Periode $L$, dann hat $c$ mindestens vier Scheitel in $[0, L)$. Scheitel: $t$, falls $\dot \kappa(t) = 0$.

\begin{lem}
\label{5.8}
Schneidet eine einfach geschlossene ebene konvexe Kurve eine Gerade in mehr als 2 Punkten, so enthält die Kurve ein ganzes Segment dieser Geraden und hat damit insbesondere unendlich viele Schnittpunkte. \\
\fixme{Bilder Schnittpunkte}
\begin{proof}
Sei $c: \R \to \R^2$ eine Parametrisierung n.Bl. mit Periode $L$. Nach einer Parametertransformation können wir annehmen, dass $c(0)$ einer der Schnittpunkte mit der Geraden ist. Wir können auch annehmen, dass $\kappa \geq 0$ ist (indem wir eventuell mit $t \mapsto -t$ umparametrisieren). \\
Eine Winkelfunktion $\tta$ erfüllt also $\tta' = \kappa \geq 0$, d.h. sie ist monoton wachsend. Nach dem Umlaufsatz \ref{4.8} gilt $\tta(L) - \tta(0) = 2 \pi \underbrace{n_c}_{+1} = 2 \pi$. Also ist $\tta : [0, L] \to [\tta_0, \tta_0 + 2 \pi]$ eine differenzierbare, monoton wachsende, surjektive Funktion mit $\tta_0 = \tta(0)$. \\
Die Kurve $c$ schneidet die Gerade $G$ bei den Parameterwerten $0 = t_0 < t_1 < t_2 < L$. Sei $G$ parametrisiert durch $t \mapsto p + t \cdot v, \; \| v \| = 1$. 
Sei $I$ eines der Intervalle $[0, t_1], [t_1, t_2], [t_2, L]$. In den Endpunkten von $I$ liegt $c$ auf $G$. Wenn $c(t)$ für alle $t \in I$ auf $G$ liegt, so enthält $c$ ein Segment von $G$ und das Lemma ist bewiesen.

Also angenommen, es gibt auch Punkte $t \in I$, so dass $c(t) \not \in G$. Setze $G_s := \{ sJv + tv \; | \; t \in \R \}$ (Geradenschar) und $s_1 := \sup \{ s > 0 \; | \; G_s \text{ schneidet } c \eing_I \} $. Liegen auf der Seite von $G$, in die $Jv$ zeigt, keine Punkte von $c(I)$, so betrachte stattdessen $s_1 := \inf \{ s < 0 \; | \; G_s \text{ schneidet } c \eing_I \}$. \\
Nach Übungsaufgabe 1, Blatt 5, schneidet $G_{s_1}$ das Kurvenstück $c \eing_I$ in einem Punkt $\tau$ aus dem Inneren von $I$ tangential, d.h. $\dot c(t) = \pm v$. Wenden wir dies auf alle drei Intervalle an, so erhalten wir drei Punkte $\tau_1, \tau_2, \tau_3$ mit $0 < \tau_1 < t_1 < \tau_2 < t_2 < \tau_3 < L$, so dass $\dot c(\tau_j) = \pm v$. 
Es ist $\tta_1$ der eindeutig bestimmte Wert aus $[\tta_0, \tta_0 + 2\pi)$, für den $\begin{pmatrix} \cos(\tta_1) \\ \sin(\tta_1) \end{pmatrix} = +v$ und $\tta_2 = \tta_1 \pm \pi$ derjenige mit $\begin{pmatrix} \cos(\tta_2) \\ \sin(\tta_2) \end{pmatrix} = -v$. Wir können annehmen, dass $\tta_2 = \tta_1 + \pi$ (sonst vertausche $\tta_1$ und $\tta_2$).
\begin{enumerate}[1. \text{Fall:} ]
\item
	 Sei $\tta_1$ (und damit auch $\tta_2$) echt größer als $\tta_0$. Die Winkelfunktion $\tta$ nimmt an den drei Stellen $\tta_1, \tta_2, \tta_3$ jeweils einen der Werte $\tta_1$ oder $\tta_2$ an. Insbesondere muss $\tta$ an mindestens zwei der drei STellen den selben Wert annehmen. Da $\tta$ monoton wächst, muss $\tta$ auf einem der Intervalle $[\tau_1, \tau_2], [\tau_2, \tau_3]$ konstant sein. Dann ist aber $\dot c = \pm v$ auf diesem Intervall, und die Kurve ist dort ein Geradensegment parallel zu $G$. Beide Intervalle enthalten aber einen Punkt, in dem $c$ die Gerade $G$ schneidet (nämlich $t_1$ bzw. $t_2$).

\item 
	$\tta(\tau_1) = \tta_1 = \tta_0, \; \tta(\tau_2) = \tta_2, \; \tta(\tau_3) = \tta_0 + 2 \pi$. Dann is wegen der Monotonie $\tta$ auf dem Intervall $[0, \tau_1]$ konstant und es folgt wie oben, dass $c \eing_[0, \tau_1]$ auf $G$ liegt ($\infty$- viele Schnittpunkte).
\end{enumerate}
\end{proof}
\end{lem}

\begin{lem}
\label{5.9}
Schneidet eine einfach geschlossene, ebene konvexe Kurve eine Gerade in mehr als einem Punkt tangential, so enthält die Kurve ein ganzes Geradenstück.

\begin{proof}
Besitzt die Kurve mehr als 2 Schnittpunkte, so folgt die Behauptung aus Lemma \ref{5.8}. Habe die Kurve also genau 2 tangentiale Berührpunkte $p, q$ mit der Geraden. Da die Kurve konvex ist, liegt sie ganz auf einer Seite der Geraden. Verschieben wir nun $G$ ein hinreichend kleines Stück in Richtung der Kurve, so erhalten wir eine zu $G$ parallele Gerade $G_1$, die aus Stetigkeitsgründen die Kurve in der Nähe von $p$ bzw. $q$ jeweils mindestens zweimal schneidet. Insgesamt hat $G_1$ also mindestens 4 Schnittpunkte mit $c$ und nach \ref{5.8} enthält $c$ ein Geradensegment. 
\end{proof}
\end{lem}

\begin{proof}[Vierscheitelsatz]
Offensichtlich nimmt $\kappa$ auf dem kompakten Intervall $[0, L]$ Maximum und Minimum an, dies liefert bereits 2 Scheitel. 0.E.(Ohne Einschränkungen) können wir annehmen, dass das Minimum in $t = 0$ und das Maximum in $t_0 \in (0, L)$ angenommen wird. Sei $G$ die Gerade durch $c(0)$ und $c(t_0)$. Falls $G$ noch einen weiteren Schnittpunkt mit $c$ hat, enthält $c$ nach Lemma \ref{5.8} ein Geradensegment (und $c$ hat damit $\infty$- viele Scheitelpunkte). Wir nehmen also an, dass $G$ nur in diesen beiden Punkten $c$ schneidet. Außerdem nehmen wir an, dass $G$ die $x$- Achse ist (nach Ausführen einer euklidischen Bewegung).

\textbf{Angenommen}, $c$ hat keinen weiteren Scheitel außer bei $0$ und $t_0$, dann verschwindet $\dot \kappa$ nirgends auf den Intervallen $(0, t_0)$ und $(t_0, L)$. Da außerdem 
\[ \int_0^L \dot \kappa(t) dt = \kappa(L) - \kappa(0) = 0 \]
gilt, muss $\kappa$ auf den beiden Intervallen verschiedene Vorzeichen besitzen. Sei oBdA $\dot \kappa(t) > 0$ auf $(0, t_0)$ und $\dot \kappa(t) < 0$ auf $(t_0, L)$. Wegen \ref{5.9} können wir annehmen, dass $c$ nicht ganz auf einer Seite der $x$-Achse liegt. Wir können annehmen, dass $c \eing_{(0, t_0)}$ oberhalb und $c \eing_{(t_0, L)}$ unterhalb der $x$-Achse liegt. Damit ergibt sich für die $y$-Komponente von $c = \vekt{c_1}{c_2}$: $\dot \kappa(t) \cdot c_2(t) > 0 \; \forall t \in (0, t_0) \cup (t_0, L)$. Insbesondere folgt damit 
\[   \int_0^L \dot \kappa(t) c_2(t) dt > 0. \qquad \text{ (*) }\]
Andererseits gilt mit partieller Integration
\[ \int_0^L  \dot \kappa(t) c(t) dt \stackrel{part. Int.}{=} \underbrace{[\kappa(t) c(t) ]_0^L}_{=0} - \int_0^L \kappa(t) \dot c(t) dt =: I .\]
Wegen \[ \dot n(t) = J \frac{d}{dt} \dot c(t) = J \cdot (\kappa(t) n(t)) = \kappa(t) J^2 \dot c(t) = -\kappa(t) \dot c(t) \]
ist \[ I = \int_0^L \dot n(t) dt = n(L) - n(0) = \vekt{0}{0}. \]
D.h. im Skalarprodukt mir $c_2$ erhalten wir
\[ \int_0^L \dot \kappa(t) c_2(t) dt = \int_0^L \< \dot \kappa(t) c(t), c_2(t) \> dt = 0 \] und somit einen Wiederspruch zu (*).
Es gibt also mindestens 3 Scheitel.

% Vorlesung 14.5.13
\textbf{Angenommen} es gibt genau 3 Scheitel, dann können wir die Kurve in genau 3 Bögen einteilen, auf denen $\dot \kappa$ das Vorzeichen nicht wechselt. Fassen wir also die beiden Bögen, auf denen $\dot \kappa$ das gleiche Vorzeichen hat, zu einem zusammen, so erhalten wir eine Unterteilung von $c$ in zwei Bögen, auf denen - von einem einzigen Punkt abgesehen - $\dot \kappa$ entgegengesetztes Vorzeichen hat. Wir können das selbe noch einmal anwenden, da einziger Punkt, an dem der Integrad zusätzlich Null wird, nichts am Vorzeichen des Integrals ändert. Widerspruch!  \\
Es muss also einen vierten Scheitel geben.
\end{proof}

\begin{note*}
Ein weiteres Mal lässt sich das Argument nicht anwenden, da bei 4 Kreisbogenstücken die mit selben Vorzeichen von $\dot \kappa$ nicht notwendig aneinander anschließen.
\end{note*}

\section{Die isoperimetrische Ungleichung}
Schon im antiken Griechenland bekannt: Unter allen Gebieten in der Ebene, die von einer Kurve der Länge $L$ berandet werden, hat der Kreis den größten Flächeninhalt.

\begin{st}{Isoperimetrische Ungleichung}
\label{6.1}
Sei $G$ ein beschränktes Gebiet, berandet von der einfach geschlossenen Kurve $c$. Sei $A$ der Flächeninhalt von $G$ und $L$ die Länge von $c$. Dann gilt: \[ 4 \pi A \leq L^2 \]
mit Gleichheit genau dann, wenn $c$ ein Kreis ist.
\end{st}

Wir benutzen das folgende Lemma, das zeigt, wie der Flächeninhalt von $G$ durch die Kurve gegeben ist:
\begin{lem}
\label{6.2}
Sei $G \subseteq \R^2$ ein beschränktes Gebiet, das von einer einfach geschlossenen Kurve $c$ berandet wird. Sei $c(t) = \vekt{x(t)}{y(t)}$ eine periodische Parametrisierung von $c$ mit Periode $L$, die das Gebiet in mathematisch positivem Sinne umläuft. Dann gilt: 
\[ A \; \stackrel{1)}{=} \; - \int_0^L x'(t) y(t) dt \; \stackrel{2)}{=} \; \int_0^L x(t) y'(t) dt \; \stackrel{3)}{=} \; \f{1}{2} \int_0^L \left[x(t) y'(t) - x'(t) y(t) \right] dt. \]

\begin{proof}
Wir benutzen den 
\begin{st*}[Divergenzsatz von Gauß]
Sei $v = \vekt{v_1}{v_2}: \R^2 \to \R^2$ eine glatte Abbildung (ein Vektorfeld), $G$ ein Gebiet mit glattem Rand und sei $N: \partial G \to \R^2$ der äußere Einheits-Normalenvektor, dann gilt:
\[ \int_G \div v = \int_{\partial G} \< v, N \>  .\]
Dabei ist die Divergenz definiert als $\div v = \f{\partial \: v_1}{\partial x_1} + \f{\partial \: v_2}{\partial x_2}$.
\end{st*}

Wir wählen $v =\text{ id}_{\R^2}: v(x_1, x_2) = \vekt{x_1}{x_2}$. Dann gilt: \[\div v = \f{\partial \: x_1}{\partial x_1} + \f{\partial \: x_2}{\partial x_2} = 1 + 1 = 2 .\]
Damit gilt für die linke Seite im Gauß'schen Divergenzsatz: $\displaystyle{\int_G \div v = \int_G} 2 = 2 \cdot A$. \\
Andererseits erhalten wir (da $n$ die innere Einheitsnormale ist): 
\begin{align*} \int_0^L \< v(c(t)), N(c(t)) \> dt \; &= \; - \int_0^L \< v(c(t)), n(t) \> dt \\   = \; - \int_0^L \left< \vekt{x(t)}{y(t)} , \vekt{-y'(t)}{x'(t)} \right> dt &= - \int_0^L x'(t) y(t) - x(t) y(t) dt. \end{align*}
Dies zeigt, dass $A = 3)$. 1) und 2) bekommt man durch partielle Integration:
\[ \int_0^L x' y dt = \underbrace{x y \big|_0^L}_{0\text{, da periodisch}} - \int_0^L x y' dt. \]
\end{proof}
\end{lem}

\begin{proof}{der isoperimetrischen Ungleichung \ref {6.1}}
E. Schmidt, 1939. \\
Seien $E$ und $E'$ parallele Geraden, die $c$ nicht schneiden und zwischen denen $c$ liegt. Wir bewegen sie aufeinander zu, bis sie die Kurve $c$ in den Geraden $L$ und $L'$ tangential berühren.

Sei $S^1$ ein Kreis, der sowwohl $L$ als auch $L'$ tangential berührt. Sei das Koordinatensystem so gewählt, dass der Ursprung $0$ der Mittelpunkt des Kreises ist und $L$ und $L'$ parallel zur $y$-Achse liegen. \\
Sei $c(t) = \vekt{x(t)}{y(t)}$ n.Bl. par. und positiv orientiert. Seien $c(0)$ und $c(s_1)$ Tangentenpunkte an $L$ bzw. $L'$. \\
{\color{orange} Wir nehmen an, dass der Kreis $S^1$ durch $ \bar{c}(t) = \vekt{\bar{x}(t)}{\bar{y}(t)} = \vekt{x(t)}{\bar{y}(t)}$ gegeben ist, d.h. $x = \bar {x}$.} \\ 
Sei $2 r$ der Abstand von $L$ und $L'$, d.h. der Radius des Kreises $\bar{c}$ (bzw. $S^1$). Sei $\bar{A}$ der Flächeninhalt von $S^1$. Aus dem Lemma \ref{6.2} erhalten wir 
\[ A = \int_0^L x y' dt \; , \quad \bar{A} = \pi r^2 = - \int_0^L x' \bar{y} dt. \]
Also gilt mit  $0 \leq (x x' + \bar{y} y')^2 \; \Leftrightarrow \; - 2 x x' \bar{y} y' \leq x^2 x'^2 + \bar{y}^2 y'^2$ (*):
\begin{align*}
A + \pi r^2 &= \int_0^L x y' dt - \int_0^L x' \bar{y} dt = \int_0^L (x y' - x' \bar{y}) dt \leq \int_0^L \sqrt{(x y' - x' \bar{y})^2} dt \\
&\stackrel{*}{\leq} \int_0^L \sqrt{(x^2 + \bar{y}^2) \; \cdot \underbrace{(x'^2 + y'^2)}_{= 1, \text{ da $c$ n.Bl.par. }}} dt  \\
&= \int_0^L \sqrt{x^2 + \bar{y}^2} dt = \int_0^L \underbrace{\sqrt{\bar{x}^2 + \bar{y}^2}}_{= r} dt = Lr.
\end{align*}
Wir benutzen jetzt die Ungleichung zwischen geometrischem und arithmetischem Mittel (dabei gilt Gleichheit genau dann, wenn die beiden Zahlen gleich sind):
\[ \sqrt{A} \sqrt{\pi r^2} \leq \f{1}{2} (A + \pi r^2) \leq \f{L}{2} r. \]
Daraus folgt 
\[ 4 \pi A r^2 \leq L^2 r^2 \quad  \Longleftrightarrow \quad 4 \pi A \leq L^2  \qquad \text{(isoper. Ungl.)} .\]

\fixme{noch zu zeigen: Gleichheitsfall}
\end{proof}

% Vorlesung 16.5.13

\section{Raumkurven}

% Vorlesung 28.5.13
\setcounter{thm}{3}

Raumkurven: $c: I \to \R^3$. \\
\emph{Krümmung}: $\kappa(t) := \| \ddot c(t) \| \quad$ ($c$ n.Bl.par.) \\
Beispiel: Die Helix $c(t) = \begin{pmatrix} \cos(\f{t}{\sqrt{2}}) \\ \sin(\f{t}{\sqrt{2}}) \\ \f{t}{\sqrt{2}} \end{pmatrix}$  hat konstante Krümmung $\kappa(t) = \f{1}{2}$, genau wie ein Kreis von Radius $2$, diese beiden Kurven sind \emph{nicht} durch eine euklidische Bewegung zur Deckung zu bringen, denn die Helix ist nicht in einer Ebene enthalten. 

$\leadsto$ Die Krümmung $\kappa(t)$ alleine reicht nicht aus, um die Kurve (analog wie im Hauptsatz über ebene Kurven) eindeutig festzulegen. Der Grund dafür ist, dass sich z.B. die Helix beständig aus ihrer Schmiegeebene herauswindet, z.B. der Kreis aber nicht. Wir benötigen eine weitere Größe, die kennzeichnet, wie stark sich die Kurve an jedem Punkt aus ihrer Schmiegeebene heraus\textbf{windet}. \\
Dazu ergänzen wir $v(t)$ und $n(t)$ zu einer Orthonormalbasis des $\R^3$:

\begin{df}
Sei $c: I \to \R^3$ eine n.Bl.par. Raumkurve und sei $t \in I$ mit $\kappa(t) \neq 0$. Dann ist durch
\[ b(t) := v(t) \kp n(t) \]
der \textbf{Binormalvektor} an $c$ in $t$ definiert.
\end{df}

Wir betrachten die Änderung des Normalenvektors:
\begin{df}
Sei $c: I \to \R^3$ eine n.Bl.par. Raumkurve und $t \in I$ mit $\kappa(t) \neq 0$. Dann heißt
\[ \tau(t) := \< \dot n(t), b(t) \> \]
die \textbf{Torsion} oder \textbf{Windung} von $c$ in $t$.
\end{df}

\begin{note*}
$\left( v(t), n(t), b(t) \right)$ heißt (für $\kappa(t) \neq 0$) das \textbf{begleitende Dreibein} von $c$ in $t$.
\end{note*}

Dass $\tau$ tatsächlich die gewünschte Eigenschaft hat, sieht man an folgendem Lemma:
\begin{lem}
Sei $C: I  \to \R^3$ eine n.Bl.par. Raumkurve mit $\kappa(t) \neq 0 \; \forall t \in I$. Dann ist $c$ eine ebene Kurve, d.h. in einer affinen Ebene des $\R^3$ enthalten, genau dann, wenn $\tau(t) = 0 \; \forall t \in I$ gilt.
\begin{proof}
Übungsaufgabe 2 a) auf Blatt 7.
\end{proof} 
\end{lem}


\begin{ex*}[Fortsetzung von Bsp. \ref{7.3}]

\[ b(t) = v(t) \kp n(t) = \f{1}{\sqrt{2}}  \begin{pmatrix} -\sin(\f{t}{\sqrt{2}}) \\ \cos(\f{t}{\sqrt{2}}) \\ 1 \end{pmatrix} \kp \begin{pmatrix} -\cos(\f{t}{\sqrt{2}}) \\ -\sin(\f{t}{\sqrt{2}}) \\ 0 \end{pmatrix}  = \f{1}{\sqrt{2}} \begin{pmatrix} \sin(\f{t}{\sqrt{2}}) \\ -\cos(\f{t}{\sqrt{2}}) \\ 1 \end{pmatrix}  \]
Damit ergibt sich für die Torsion:
\[ \tau(t) = \< \dot n(t), b(t) \>  = \left\< \f{1}{\sqrt{2}}  \begin{pmatrix} \sin(\f{t}{\sqrt{2}}) \\ -\cos(\f{t}{\sqrt{2}}) \\ 0 \end{pmatrix} , \f{1}{\sqrt{2}} \begin{pmatrix} \sin(\f{t}{\sqrt{2}}) \\ -\cos(\f{t}{\sqrt{2}}) \\ 1 \end{pmatrix} \right\>  = \f{1}{2} \]
Die Helix im Beispiel hat also konstante Torsion $\f{1}{2}$.
\end{ex*}

Wir wollen nun beschreiben, wie sich die Vektoren im begleitenden Dreibein entlang der Kurve zueinander bewegen, das sind die sogenannten \emph{Frenet'schen Gleichungen}:

\begin{st}[Frenet-Gleichungen]
\label{7.7}

Sei $c: I \to \R^3$ eine n.Bl.par. Raumkurve mit positiver Krümmung $\kappa(t) \neq 0 \; \forall t \in I$ (solche Raumkurven heißen \emph{Frenet-Kurven}). Sei ($v, n, b$) das begleitende Dreibein von $c$, dann gilt:
\begin{align*}
\dot v &= \kappa \cdot n \\
\dot n &= - \kappa \cdot v + \tau \cdot b \\
\dot b &= - \tau \cdot n
\end{align*}

bzw. in Matrixschreibweise:
\[
\begin{pmatrix} \dot v \\ \dot n \\ \dot b \end{pmatrix} = \begin{pmatrix} 0 & \kappa & 0 \\ -\kappa & 0 & \tau \\ 0 & -\tau & 0   \end{pmatrix} \cdot \begin{pmatrix} v \\ n \\ b \end{pmatrix} \qquad \text{ oder } \qquad  \begin{pmatrix} \dot v & \dot n & \dot b \end{pmatrix} = \begin{pmatrix} v & n & b \end{pmatrix} \cdot \begin{pmatrix} 0 & \kappa & 0 \\ -\kappa & 0 & \tau \\ 0 & -\tau & 0   \end{pmatrix} \]

\begin{note*}
Dabei sind $\begin{pmatrix} \dot v & \dot n & \dot b \end{pmatrix}$ und $\begin{pmatrix}  v &  n &  b \end{pmatrix}$ Matrizen, da $v, \dot v, n, \dot n, b, \dot b \in \R^3$ sind, und $\begin{pmatrix} \dot v \\ \dot n \\ \dot b \end{pmatrix}, \begin{pmatrix}  v \\  n \\  b \end{pmatrix} \in \R^9$.
\end{note*}
\begin{proof}
\begin{enumerate}[i)]
\item
	Die erste Gleichung ist nichts anderes als unsere Definition von $\kappa$ und $n$.
\item
	Für die zweite Gleichung betrachte 
\[ 0 = \f{d}{dt} \< n, v \> = \< \dot n, v \> + \< n, \dot v \> = \< \dot n, v \> + \kappa ,\]
daraus folgt $\< \dot n, v \> = - \kappa$, und wegen der Definition der Torsion gilt $\< \dot n, b \> =  \tau$. Außerdem gilt $\< \dot n, n \> = 0$ wegen $0 = \f{d}{dt} \underbrace{\< n, n \>}_{1} = \< \dot n, n \> + \< n, \dot n \> = 2 \< \dot n, n \>$.
\item
	Für die dritte Gleichung betrachte
\[ 0 = \f{d}{dt} \< b, v \> = \< \dot b, v \> + \underbrace{\< b, \dot v \>}_{=0} \quad \Rightarrow \quad \< \dot b, v \> = 0 \]
und 
\[ 0 = \f{d}{dt} \< b, n \> = \< \dot b, n \> + \< b, \dot n \> \stackrel{ii)}{=} \< \dot b, n \> + \underbrace{\< b, v \>}_{=0} \cdot (-\kappa) + \tau \cdot \underbrace{\< b, b \>}_{=1} \quad \Rightarrow \quad \< \dot b, n \> = - \tau. \]
Außerdem gilt $\< \dot b, b \> = 0$ analog wie in $ii)$.
\end{enumerate}
\end{proof}
\end{st}

Wir wollen nun zeigen, dass Krümmung und Torsion eine Raumkurve im Wesentlichen eindeutig festlegen. Dazu folgendes 
\begin{lem}
\label{7.8}
Sei $c: I \to \R^3$ eine n.Bl.par. Raumkurve mit positiver Krümmung $\kappa > 0$ und $F: \R^3 \to \R^3$ eine eigentliche (orientierungserhaltende) Euklidische Bewegung, d.h. $F(x) = Ax + b$ für ein $A \in SO(3), b \in \R^3$, dann gilt für die Krümmung und Torsion der Kurve $\tilde c = F  \circ c:$ \[ \tilde \kappa = \kappa, \qquad \tilde \tau = \tau. \]
Ist $F$ eine nicht-orientierungserhaltende Euklidische Bewegung des $\R^3$, d.h. $F(x) = Ax + b$ mit $A \in O(3) \diagdown SO(3), b \in \R^3 $, dann gilt \[ \tilde \kappa = \kappa, \qquad \tilde \tau = - \tau. \]
\begin{proof}
zur Übung selbst.
\end{proof}
\end{lem}

\begin{st}[Hauptsatz über Raumkurven]
\label{7.9}
Sei $I$ ein Intervall und $\kappa , \tau : I \to \R$ glatte Funktionen mit $\kappa > 0$. Dann existiert eine n.Bl.par. Raumkurve $c: I \to \R^3$ mit Krümmung $\kappa$ und Torsion $\tau$. Diese Raumkurve ist bis auf eine orientierungserhaltend Bewegung eindeutig festgelegt.
\begin{proof}
Wir betrachten das lineare Differentialgleichungssystem: 
\[ \begin{matrix} \dot c = v~~ & & & c(t_0) = 0 \\ 
\dot v = \kappa n~ & & & v(t_0) = e_1       \\
\dot n = - \kappa v & + \tau b & \text{ mit Anfangsbedingungen} & n(t_0) = e_2  \\
\dot b = - \tau n & & & b(t_0) = e_3 
\end{matrix}    \]
mit $e_i = i$-ter Einheitsvektor und $t_0 \in I$. 

\textbf{Existenz:}
Da $\kappa, \tau$ glatt sind, existiert eine eindeutige glatte Lösung, und da das System linear ist, ist sie auf ganz $I$ definiert. \\
Wir müssen uns noch überlegen, dass die so definierte Kurve $c$ tatsächlich die Krümmung $\kappa$ und Torsion $\tau$ hat.
Die Anfangswerte sind so gewählt, dass $\left( v(t_0), n(t_0), b(t_0) \right)$ eine positiv orientierte Orthonormalbasis des $\R^3$ sind. Wir wollen zeigen, dass dies für alle $t \in I$ gilt. Wir betrachten dazu 
\begin{align*}
\f{d}{dt} \< v,v \> &= 2 \< \dot v, v \> = 2 \kappa \< n, v \>  \\
\f{d}{dt} \< n, n \> &= 2 \< \dot n, n \> = -2 \kappa \< v, n \> + \tau \< b,n \> \\
\end{align*}
und äquivalent für $ \< b,b \>, \< b,v \>, \< n,v \>, \< b, n \>$. \\
Man erhält ein lineares Differentialgleichungssystem:
\[ 
 \f{d}{dt} \begin{pmatrix} \<v,v \> \\ \< n, n \> \\ \< b,b \> \\ \hline  \< b,v \> \\ \< b, n \> \\ \< n,v \>   \end{pmatrix} = \begin{pmatrix}  0 & 0 & 0 & \vline &  0 & 0 & 2 \kappa \\
0 & 0 & 0 & \vline & 0 & 2 \tau & -2 \kappa \\
0 & 0 & 0 & \vline & 0 & -2 \tau & 0 \\
\hline
0 & 0 & 0 & \vline & 0 & \kappa & - \tau  \\
0 &  - \tau & \tau & \vline & - \kappa & 0 & 0 \\
- \kappa & \kappa & 0 & \vline & \tau & 0 & 0  \end{pmatrix}  \cdot  \begin{pmatrix} \<v,v \> \\ \< n, n \> \\ \< b,b \> \\ \hline  \< b,v \> \\ \< b, n \> \\ \< n,v \>   \end{pmatrix}, \qquad F(t_0) = \begin{pmatrix} 1 \\ 1 \\ 1  \\ 0   \\ 0 \\ 0 \end{pmatrix}
\]
mit Anfangsbedingung $F(t_0)$. Man sieht aber, dass die konstante Funktion $F(t) = \begin{pmatrix} 1 & 1 & 1 & 0 & 0 & 0 \end{pmatrix}^T$ das DGL-System löst.

% Vorlesung 4.6.13
Wegen der Eindeutigkeit folgt, dass für jede Lösung auf dem Intervall $I$ des ersten Systems $\left( v(t), n(t), b(t) \right)$ für alle $t$ eine ONB des $\R^3$ bilden, aus Stetigkeitsgründen eine orientierte ONB des $\R^3$. Wir erhalten insbesondere ein en.Bl.par. Raumkurve $c$. Daraus, dass die Frenet-Gleichungen erfüllt sind, liest man leicht ab, dass $c$ die Krümmung $\kappa$ und Torsion $\tau$ hat.

\textbf{Zur Eindeutigkeit:} Sei $\tilde c: I \to \R^3$ eine weitere Raumkurve mit Krümmung $\kappa$ und Torsion $\tau$. Wir setzen $p := \tilde c(t_0)$ und $A = \left( \tilde v(t_0), \tilde n(t_0), \tilde b(t_0) \right) \in \R^{3 \kp 3}$. Da $\tilde v, \tilde n, \tilde b$ für alle $t$ eine orientierte ONB bilden, ist $A \in SO(3)$ (spezielle orthogonale Gruppe) und damit auch $A^{-1} \in SO(3)$. Sei $F$ die eigentliche Euklidische Bewegung mit $F(x) = A^{-1} x - A^{-1} p$ und setze $\hat c := F \circ \tilde c$. Dann gilt $\tilde c(t_0) = 0$ und $\left(\hat  v(t_0), \hat n(t_0), \hat b(t_0) \right) = \left( e_1, e_2, e_3 \right)$. Wegen der eindeutigen Lösbarkeit des ersten DGL-Systems folgt $\hat c = c$.
\end{proof}
\end{st}

\chapter{Flächentheorie}

\section{Reguläre Flächen}

Bis jetzt haben wir $1-$dimensionale Objekte im Euklidischen Raum betraachtet, nun möchten wir $2-$dimensionale Flächen im Raum studieren. In Analogie zu unserer Definition bei Kurven wäre es naheliegend, Flächenstücke durch Abbildungen $U \to \R^3$ mit einer Regularitätsbedingung zu beschreiben:



% Vorlesung 6.6.13

% Vorlesung 11.6.13

% Vorlesung 13.6.13

% Vorlesung 18.6.13

% Vorlesung 20.6.13

% Vorlesung 25.6.13

\end{document}
