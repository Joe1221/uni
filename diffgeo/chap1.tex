% 1.
\chapter{Mannigfaltigkeiten}

Mannigfaltigkeiten sind geometrische Objekte, die lokal dem euklidischen Räum $\R^n$ ähnlich sind.

% 1.1.
\section{Topologische Mannigfaltigkeiten}

Erinnerung an die Analysis:
Eine Abbildung ist genau dann stetig, wenn Urbilder offener Mengen offen sind.
Um eine Abbildung auf Stetigkeit zu untersuchen genügt es also, jeweils das System der offenen Mengen zu kennen.

\begin{df} \label{1.1} % 1.1.
    Sei $X$ eine Menge.
    Eine \emphdef{Topologie} auf $X$ ist ein System $\scr O$ von Teilmengen von $X$, sodass
    \begin{enumerate}[(i)]
        \item
            $\emptyset \in \scr O$, $X \in \scr O$,
        \item
            $\scr O$ ist abgeschlossen unter endlicher Schnittbildung,
        \item
            $\scr O$ ist abgeschlossen unter (beliebiger) Vereinigung.
    \end{enumerate}
    Elemente aus $\scr O$ nennen wir \emphdef[offene Menge]{offene Mengen}.
    Das Tupel $(X, \scr O)$ heißt \emphdef{topologischer Raum}.

    Eine Abbildung zwischen topologischen Räumen heißt \emphdef{stetig}, wenn die Urbilder offener Mengen offen sind.
\end{df}

\begin{ex*}
    \begin{itemize}
        \item
            $\R^n$ mit den durch die euklidische Abstandsfunktion
            \begin{math}
                d(x,y) = \sqrt{(x_1 - y_1)^2 + \dotsb + (x_n - y_n)^2}
            \end{math}
            festgelegten offenen Mengen bildet einen topologischen Räum.
        \item
            Metrische Räume sind topologische Räume.
    \end{itemize}
\end{ex*}

\begin{df} \label{1.2} % 1.2
    Ein topologischer Raum $X$ heißt \emphdef{hausdorffsch}, wenn es zu je zwei Punkten $a, b \in X$ disjunkte offene Mengen $A, B \in \scr O$ gibt, sodass $a \in A$, $b \in B$.
    \begin{note}
        Metrische Räume sind hausdorffsch.
    \end{note}
\end{df}

\begin{df} \label{1.3} % 1.3
    Ein System von Teilmengen $\scr B$ eines topologischen Räumes heißt \emphdef{Basis} der Topologie, wenn jedes Element in $\scr B$ offen ist sich jede offene Menge als Vereinigung von Mengen in $\scr B$ schreiben lässt.

    Ein topologischer Raum erfüllt das \emphdef{zweite Abzählbarkeitsaxiom} (oder ist \emphdef{zweitabzählbar}, wenn es eine abzählbare Basis der Topologie gibt.
\end{df}

Nun können wir topologische Mannigfaltigkeiten definieren.

\begin{df} \label{1.4} % 1.4
    Ein zweitabzählbarer, hausdorffscher topologischer Raum heißt \emphdef{topologische Mannigfaltigkeit} der Dimension $n$, wenn
    er lokal homöomorph zum $\R^n$ ist (d.h. um jeden seiner Punkte existiert eine offene Umgebung, die zum $\R^n$ homöomorph ist).
\end{df}

\begin{ex*}
    \begin{itemize}
        \item
            $\R^n$ mit der Standard-Topologie ist eine $n$-dimensionale topologische Mannigfaltigkeit.
        \item
            Die Einheitssphäre $S^n \subset \R^{n+1}$ ist eine $n$-dimensionale topologische Mannigfaltigkeit.

            Mit der stereographischen Projektion lässt sich $S^n \setminus \Set{N}$ homöomorph auf $\R^n$ abbilden.
            Tut man das selbe mit dem Südpol, so hat man für jeden Punkt in $S^n$ Umgebungen gefunden, die homöomorph zu $\R^n$ ist.
        \item
            Ein topologischer Raum, der bis auf die hausdorff-Forderung alle anderen Bedingungen für eine topologische Mannigfaltigkeit erfüllt.
            Sei $X = \R \times \Set{0, 1}$.
            Bilde nun $X / \sim$ mit $(x,s) \tilde (y, t) \iff x = y \land x \neq 0$.
        \item
            Ein topologischer Raum, der bis auf die Zweitabzählbarkeit alle anderen Bedingungen für eine topologische Mannigfaltigkeit erfüllt.
            Betrachte die „lange Gerade“.
    \end{itemize}
\end{ex*}


\Timestamp{2015-10-14}

Begrifflichkeiten: Karten, Kartenumgebung.
Eine Mannigfaltigkeit wird von den Kartenumgebungen vollständig überdeckt.
Sie ist ein Gebilde, das aus kleinen Stücken des euklidischen Raums $\R^n$ „zusammengeklebt“ ist.


% 1.2.
\section{Differenzierbare Struktur}

Um die Werkzeuge der Analysis einsetzen zu können, benötigen wir den Begriff der Differenzierbarkeit von Abbildungen zwischen Mannigfaltigkeiten.

\begin{df} \label{1.5}
    Eine Abbildung die eine offene Teilmenge einer topologischen Mannigfaltigkeit homöomorph auf eine offene Teilmenge des $\R^n$ abbildet, wird \emphdef{Karte} der Mannigfaltigkeit genannt.
    Meist werden Umkehrabbildungen von Karten als (lokale) \emphdef{Parametrisierungen} bezeichnet, manchmal auch Karte.
\end{df}

Wir wollen Differenzierbarkeit von Abbildungen zwischen Mannigfaltigkeiten definieren.
Sei $f: M \to \R$ eine reelle Funktion.
Wann ist $f$ differenzierbar?

Naheliegende Idee (die aber nicht ohne Weiteres funktioniert):
Da auf $M$ kein Begriff der Differenzierbarkeit definiert ist, untersuche $f$ „in einer Karte“, d.h. für $p \in M$ wähle eine offene Umgebung$V$ von $p$ in $M$ und einen Homöomorphismus $\phi: V \to U$, wobei $U \subset \R^n$.
Dann erkläre: $f$ ist (bezüglich der Karte $\phi$) differenzierbar im Punkt $p$, wenn
\begin{math}
    f \circ \phi^{-1}: U \to \R
\end{math}
differenzierbar ist.
Problem: Das hängt von der gewählten Karte ab.
In der Tat: Ist $\alpha: U \to U$ ein nicht-differenzierbarer Homöomorphismus, dann ist auch $\psi: \alpha \circ \phi$ eine Karte und $f$ kann bezüglich $\phi$ differenzierbar in $p$ sein und bezüglich $\psi$ nicht.

\begin{df} \label{1.6}
    Sei $M$ eine topologische Mannigfaltigkeit und seien $\phi_i: V_i \to U_i$, $i = 1, \dotsc 2$ zwei Karten.
    Dann heißt der Homöomorphismus $\alpha_{12}: \phi_1(V_1 \cap V_2) \to \phi_2(V_1 \cap V_2)$, definiert durch
    \begin{math}
        \alpha := \phi_2 \circ \phi_1^{-1}|_{\phi(V_1 \cap V_2)}
    \end{math}
    \emphdef{Kartenwechsel} zwischen $\phi_1$ und $\phi_2$.
    Andere gebräuchliche Begriffe sind \emphdef{Koordinatenwechsel}, \emphdef{Koordinatentransformation}, \emphdef{Parameterwechsel}.
    \begin{note}
        Der Name Kartenwechsel kommt daher, dass $\phi_2(x) := \alpha_{12} \circ \phi_1(x)$ für alle $x \in V_1 \cap V_2$ gilt.
    \end{note}
    Zwei Karten heißen \emphdef{differenzierbar verträglich}, wenn ihre Kartenwechsel differenzierbar sind.

    Eine Familie von Karten $(\phi_i)_{i\in I}$, $\phi_i: V_i \to U_i$ einer topologischen Mannigfaltigkeit $M$ heißt \emphdef{Atlas} von $M$, falls $\bigcup_{i \in I} V_i = M$.
    Ein Atlas heißt \emphdef{differenzierbar}, wenn alle Kartenwechsel differenzierbar sind.
    \begin{note}
        Sind $\phi$ und $\psi$ Karten aus einem differenzierbaren Atlas, sodass $p \in M$ im Definitionsbereich beider Karten liegt, dann ist $f$ bei $p \in M$ bezüglich $\phi$ genau dann differenzierbar, wenn dies auch bezüglich $\psi$ gilt.
    \end{note}
    Ein differenzierbarer Atlas heißt \emphdef{maximal}, wenn jede Karte von $M$, die mit allen Karten in $\scr A$ differenzierbar verträglich ist, bereits in $\scr A$ enthalten ist.
\end{df}

\begin{st} \label{1.7}
    Zu jedem differenzierbaren Atlas $\scr A$ einer topologischen Mannigfaltigkeit $M$ gibt es genau einen maximalen Atlas, der diesen enthält.
    \begin{proof}
        Sind zwei Karten mit allen Karten aus $\scr A$ differenzierbar verträglich, dann sind sie auch untereinander differenzierbar veträglich:
        Seien dazu $\phi$ und $\psi$ zwei Karten.
        Wir können annehmen, dass $\phi$ und $\psi$ den selben Definitionsbereich $V$ haben
        Es genügt zu zeigen, dass
        \begin{math}
            \psi \circ \psi^{-1}: \psi(V) \to \phi(V)
        \end{math}
        differenzierbar ist.
        Sei $u \in \psi(V)$.
        Es gibt eine Karte $\tau \in \scr A$, die mit $\phi$ und $\psi$ differenzierbar verträglich ist und die bei $\psi^{-1}(u)$ definiert ist.
        Sei $\tilde V \subset V$ die Schnittmenge von $V$ mit dem Definitionsbereich von $\tau$.
        Durch weiteres Einschränken können wir $V = \tilde V$ annehmen.
        Da $\tau$ mit $\phi$ und mit $\psi$ verträglich ist, wird
        \begin{math}
            \phi \circ \psi^{-1} = \phi \circ \tau^{-1} \circ \tau \circ \psi^{-1}
        \end{math}
        bei $u$ differenzierbar.
        Setze nun
        \begin{math}
            \hat{\scr A} := \Set{ \phi & \text{$\phi$ Karte von $M$ verträglich mit allen Karten in $\scr A$} },
        \end{math}
        dann sind alle Karten in $\scr A$ untereinander verträglich und $\hat{\scr A}$ ist ein maximaler differenzierbarer Atlas.
        $\hat{\scr A}$ enthält jeden differenzierbaren Atlas, der $\scr A$ umfasst und ist damit eindeutig.
    \end{proof}
\end{st}

\begin{df*}
    Zwei differenzierbare Atlanten heißen \emphdef{äquivalent}, wenn sie den selben maximalen Atlas definieren.
\end{df*}

\begin{df} \label{1.8}
    Eine \emphdef{differenzierbare Struktur} auf einer topologischen Mannigfaltigkeit ist die Wahl der Äquivalenzklasse von differenzierbaren Atlanten, oder äquivalent die Wahl eines maximalen differenzierbaren Atlas.

\Timestamp{2015-10-20}
    \begin{note}
        Neben differenzierbaren Strukturen kann man auch fordern, dass alle Kantenwechsel glatt (hier: beliebig oft differenzierbar), holomorph (bei gerad-dimensionalen Mannigfaltigkeiten), analytisch.
        Dann enthält man eine glatte, komplexe, bzw. analytische Struktur, bzw. eine glatte, komplexe, bzw. analytische Mannigfaltigkeit.
    \end{note}
\end{df}

\begin{conv*}
    Wir betrachten im Folgenden (sofern nicht anders festgelegt) glatte Mannigfaltigkeiten
\end{conv*}

Warum ist es wichtig, überhaupt die Wahl der differenzierbaren Struktur vorzugeben?
Gibt es vielleicht eine kanonische Wahl einer differenzierbaren Struktur?

Für viele differenzierbare Mannigfaltigkeiten kann man das tatsächlich tun:
sie tragen (bis auf Diffeomorphie) nur eine differenzierbare Struktur.
Andere topologische Mannigfaltigkeiten können verschiedene (oder gar keine) differenzierbare Strukturen tragen, sodass die entstehenden differenzierbaren Mannigfaltigkeiten nicht zueinander diffeomorph sind.

\begin{ex*}
    Bis auf Diffeomorphie kann man teilweise die Anzahl der differenzierbaren Strukturen zählen.
    \begin{itemize}
        \item
            Für $n \neq 4$ hat der $\R^n$ genau eine differenzierbare Struktur,
            der $\R^4$ jedoch hat überabzählbar viele.
        \item
            \begin{table}[ht]
                \begin{tabular}{r|ccccccccccccccc}
                    $n$ & 1 & 2 & 3 & 4 & 5 & 6 & 7 & 8 & 9 & 10 & 11 & 12 & 13 & 14 & 15 \\ \hline
                    Anzahl & 1 & 1 & 1 & ? & 1 & 1 & 28 & 2 & 8 & 6 & 992 & 1 & 3 & 2 & 16256
                \end{tabular}
                \caption{Anzahl der differenzierbaren Strukturen auf der Sphäre $S^n$ in niedrigen Dimensionen}
            \end{table}
    \end{itemize}
\end{ex*}


\section{Beispiele für (glatte) Mannigfaltigkeiten}


Wir haben gesehen: $\R^n$ und $S^n$ sind topologische Mannigfaltigkeiten.
Sie sind auch glatte Mannigfaltigkeiten.

\begin{itemize}
    \item
        $\id_{\R^n}$ ist eine globale Karte und legt einen maximalen glatten Atlas fest, bestehend aus allen Diffeomorphismen zwischen offenen Teilmengen des $\R^n$.
        Der so definierte Differenzierbarkeitsbegriff stimmt mit dem üblichen überein.
    \item
        Bei $S^n$ ist wie folgt eine glatte Struktur gegeben:
        Überdecke $S^n$ mit den Kartenumgebungen eines Atlas, z.B. $\scr A = \Set{\phi_N, \phi_S}$ (stereographische Projektion am Nord-, bzw. Südpol) und prüfe nach, ob die Kartenwechsel glatt sind (Übungsaufgabe).
\end{itemize}





