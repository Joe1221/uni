% 1.
\chapter{Mannigfaltigkeiten}

Mannigfaltigkeiten sind geometrische Objekte, die lokal dem euklidischen Räum $\R^n$ ähnlich sind.

% 1.1.
\section{Topologische Mannigfaltigkeiten}

Erinnerung an die Analysis:
Eine Abbildung ist genau dann stetig, wenn Urbilder offener Mengen offen sind.
Um eine Abbildung auf Stetigkeit zu untersuchen genügt es also, jeweils das System der offenen Mengen zu kennen.

\begin{df} \label{1.1} % 1.1.
    Sei $X$ eine Menge.
    Eine \emphdef{Topologie} auf $X$ ist ein System $\scr O$ von Teilmengen von $X$, sodass
    \begin{enumerate}[(i)]
        \item
            $\emptyset \in \scr O$, $X \in \scr O$,
        \item
            $\scr O$ ist abgeschlossen unter endlicher Schnittbildung,
        \item
            $\scr O$ ist abgeschlossen unter (beliebiger) Vereinigung.
    \end{enumerate}
    Elemente aus $\scr O$ nennen wir \emphdef[offene Menge]{offene Mengen}.
    Das Tupel $(X, \scr O)$ heißt \emphdef{topologischer Raum}.

    Eine Abbildung zwischen topologischen Räumen heißt \emphdef{stetig}, wenn die Urbilder offener Mengen offen sind.
\end{df}

\begin{ex*}
    \begin{itemize}
        \item
            $\R^n$ mit den durch die euklidische Abstandsfunktion $d(x,y) = \sqrt{(x_1 - y_1)^2 + \dotsb + (x_n - y_n)^2}$ festgelegten offenen Mengen bildet einen topologischen Räum.
        \item
            Metrische Räume sind topologische Räume.
    \end{itemize}
\end{ex*}

\begin{df} \label{1.2} % 1.2
    Ein topologischer Raum $X$ heißt \emphdef{hausdorffsch}, wenn es zu je zwei Punkten $a, b \in X$ disjunkte offene Mengen $A, B \in \scr O$ gibt, sodass $a \in A$, $b \in B$.
    \begin{note}
        Metrische Räume sind hausdorffsch.
    \end{note}
\end{df}

\begin{df} \label{1.3} % 1.3
    Ein System von Teilmengen $\scr B$ eines topologischen Räumes heißt \emphdef{Basis} der Topologie, wenn jedes Element in $\scr B$ offen ist sich jede offene Menge als Vereinigung von Mengen in $\scr B$ schreiben lässt.

    Ein topologischer Raum erfüllt das \emphdef{zweite Abzählbarkeitsaxiom} (oder ist \emphdef{zweitabzählbar}, wenn es eine abzählbare Basis der Topologie gibt.
\end{df}

Nun können wir topologische Mannigfaltigkeiten definieren.

\begin{df} \label{1.4} % 1.4
    Ein zweitabzählbarer, hausdorffscher topologischer Raum heißt \emphdef{topologische Mannigfaltigkeit} der Dimension $n$, wenn
    er lokal homöomorph zum $\R^n$ ist (d.h. um jeden seiner Punkte existiert eine offene Umgebung, die zum $\R^n$ homöomorph ist).
\end{df}

\begin{ex}
    \begin{itemize}
        \item
            $\R^n$ mit der Standard-Topologie ist eine $n$-dimensionale topologische Mannigfaltigkeit.
        \item
            Die Einheitssphäre $S^n \subset \R^{n+1}$ ist eine $n$-dimensionale topologische Mannigfaltigkeit.

            Mit der stereographischen Projektion lässt sich $S^n \setminus \Set{N}$ homöomorph auf $\R^n$ abbilden.
            Tut man das selbe mit dem Südpol, so hat man für jeden Punkt in $S^n$ Umgebungen gefunden, die homöomorph zu $\R^n$ ist.
        \item
            Ein topologischer Raum, der bis auf die hausdorff-Forderung alle anderen Bedingungen für eine topologische Mannigfaltigkeit erfüllt.
            Sei $X = \R \times \Set{0, 1}$.
            Bilde nun $X / \sim$ mit $(x,s) \tilde (y, t) \iff x = y \land x \neq 0$.
        \item
            Ein topologischer Raum, der bis auf die Zweitabzählbarkeit alle anderen Bedingungen für eine topologische Mannigfaltigkeit erfüllt.
            Betrachte die „lange Gerade“.
    \end{itemize}
\end{ex}

