\chapter{Vektorbündel und Tensoren}

\Timestamp{2015-12-09}

Neben dem Tangentialbündel wollen wir nun auch andere Vektorräume punktweise an eine Mannigfaltigkeit „anheften“, vor allem Vektorräume, die sich nicht durch Grundkonstruktionen der Linearen Algebra aus dem Tangentialbündel ergeben, z.B. Linearformen und Multilinearformen auf dem Tangentialraum oder Endomorphismen des Tangentialraums.
Dies führt auf den Begriff des Vektorbündels.

\begin{df}[Faserbündel] \label{5.1}
    Seien $M, B, Z$ glatte Mannigfaltigkeiten, $\pi: M \to B$ eine glatte Abbildung, $(U_j)_{j\in J}$ eine offene Überdeckung von $B$ und seien
    \begin{math}
        h_j: \pi^{-1}(U_j) \to U_j \times Z
    \end{math}
    Diffeomorphismen, so dass $\pi|_{\pi^{-1}(U_j)} = \proj_1 \circ h_j$.
    Dann heißt $\pi$ zusammen mit $(h_j)_{j\in J}$ \emphdef{Faserbündel mit typischer Faser $Z$}.
    Die $h_j$ heißen \emphdef{lokale Trivialisierungen}.
    $B$ heißt \emphdef{Basis}, $M$ \emphdef{Totalraum des Bündels}.

    Gilt $M = B \times Z$ mit $\pi = \proj_1$, dann heißt $\pi$ \emphdef{triviales Bündel}.

    Für $p \in B$ heißt $\pi^{-1}(\Set{p})$ \emphdef{Faser über $p$}.
\end{df}

\begin{ex*}
    \begin{enumerate}[(i)]
        \item
            Das Tangentialbündel $\T M$ mit Basis $M$ und $\pi$ Fußpunktabbildung und typischer Faser $\T_p M$ für ein $p \in M$ bilden ein Faserbündel.
        \item
            $\R^{n+1} \setminus \Set 0$ ist ein Faserbündel über $\RP^n$ mit
            \begin{math}
                \pi: (x_1, \dotsc, x_{n+1}) \mapsto [x_1: \dotsb : x_{n+1}]
            \end{math}
            und typscher Faser $\R \setminus \Set 0$.
        \item
            Hopf-Faserung, siehe Übungsblatt 8.
    \end{enumerate}
\end{ex*}

\begin{st}[Ehresmannscher Faserungssatz] \label{5.2}
    Seien $M, B$ glatte Mannigfaltigkeiten und sei $\pi: M \to B$ eine eigentliche (d.h. Urbilder kompakter Mengen sind kompakt), surjektive Submersion.
    Dann ist $\pi$ ein Faserbündel.
    \begin{proof}
        Siehe Literatur.
    \end{proof}
\end{st}

\begin{df} \label{5.3}
    Ein Faserbündel mit diskreter Faser wird auch \emphdef{Überlagerung} genannt. 
    Ist die Faser endlich mit Kardinalität $n$, so spricht man von $n$-facher Überlagerung.
    \begin{note}
        \begin{itemize}
            \item
                Im Falle einer Überlagerung ist $\pi$ ein lokaler Diffeomorphismus.
            \item
                $U_j$ wird hier auch \emphdef{überlagerte Umgebung} genannt.
        \end{itemize}
    \end{note}
\end{df}

\begin{ex*}
    \begin{enumerate}[(i)]
        \item
            Sei $S^n \subset \R^{n+1}$ die Einheitssphäre, dann ist $\pi: S^n \to \RP^n, x \mapsto [x]$ (homogene Koordinaten) eine $2$-fache Überlagerung.
            Die Faser über $[x] \in \RP^n$ ist $\pi^{-1}([x]) = \Set{x, -x}$.
        \item
            $\pi: \operatorname{Spin}(n) \to \SO(n)$ ist $2$-fache Überlagerung, falls $n \ge 3$.
        \item
            $\pi: \R \mapsto S^1, t \mapsto e^{it}$ ist eine Überlagerung mit abzählbar unendlichen Fasern $t + 2\pi\Z$.
    \end{enumerate}
\end{ex*}

\begin{df} \label{5.4}
    Ein \emphdef{Schnitt} eines Faserbündels $\pi: M \to B$ ist eine glatte Abbildung $s: B \to M$, sodass $\pi \circ s = \id_B$ gilt.

    Die Menge der Schnitte wird mit $\Gamma(B, M)$ bezeichnet.
    \begin{note}
        Ein Schnitt ist eine Einbettung $s: B \to M$.
    \end{note}
\end{df}

\begin{ex*}
    \begin{itemize}
        \item
            Vektorfelder auf einer Mannigfaltigkeit $M$ sind genau die Schnitte im Tangentialbündel $\T M$, d.h. $\scr X(M) = \Gamma(M, \T M)$.
        \item
            $C^\infty(M) = \Gamma(M, M \times \R)$.
    \end{itemize}
\end{ex*}

\begin{nt} \label{5.5}
    Sei $\pi: M \to B$ ein Faserbündel mit lokalen Trivialisierungen $h_j$, $j \in J$.
    Für $j, k \in J$ ist die Abbildung
    \begin{math}
        h_j \circ h_k^{-1}: (U_j \cap U_k) \times Z \to (U_j \cap U_k) \times Z
    \end{math}
    von der Form $(u, z) \mapsto (u, g_{jk}(u,z))$, wobei $g_{jk}(u,\argdot)$ ein Diffeomorphismus auf $Z$ ist (vgl. Übungsaufgabe 28).

    Die Abbildung $g_{jk}(u, \argdot)$ wird \emphdef{Übergangsabbildung} genannt.
\end{nt}

\begin{df} \label{5.6}
    Ein Faserbündel heißt \emphdef{$\K$-Vektor(raum)bündel} vom \emphdef{Rang} $r$, falls die typische Faser $Z$ ein $\K$-Vektorraum der Dimension $r$ ist und alle Übergangsabbildungen $g_{jk}(u, \argdot)$ linear sind.

    Jedes Vektorraumbündel hat als kanonischen Schnitt den \emphdef{Nullschnitt} $s \equiv 0$.

    Ein Vektorraumbündel vom Rang $1$ heißt \emphdef{Linienbündel}.
    \begin{note}
        \begin{itemize}
            \item
                Die Übergangsabbildungen sind quasi Vektorraumautomorphismen von $Z$.
            \item
                $\dim(E) = \dim(Z) + \dim(B)$.
        \end{itemize}
    \end{note}
\end{df}

\begin{ex}
    Vektorraum-Bündel
    \begin{enumerate}[(i)]
        \item
            $\T M$.
        \item
            Zylinder $\Set{(e^{it}, s) & t,s \in \R} \subset \C^2$.
        \item
            Möbiusband $\Set{(e^{it}, s e^{i\frac{t}{2}}) & t,s \in \R} \subset \C^2$.
    \end{enumerate}
\end{ex}

\Timestamp{2015-12-15}

\begin{df}[Pullback] \label{5.7}
    Sei $f: M \to N$ eine glatte Abbildung und $\pi: E \to N$ ein Vektorbündel über $N$.
    Definiere
    \begin{math}
        f^* E := \Set{(p,v) \in M \times E & \pi(v) = f(p)}.
    \end{math}
    Das Vektorbündel $\proj_1: f^* E  \to M$ heißt \emphdef{Pullback} von $E$ unter $f$.
    \begin{note}
        Hat $E$ lokale Trivialisierungen $h_j: \pi^{-1}(U_j) \to U_j \times V$, dann hat $f^* E$ lokale Trivialisierungen
        \begin{math}
            f^* h_j: \proj_1^{-1}(f^{-1}(U_j)) &\to f^{-1}(U_j) \times V, \\
            (p, v) &\mapsto (p, \proj_2(h_j(v))).
        \end{math}
    \end{note}
    Der Pullback von Vektorbündeln induziert auch einen \emphdef{Pullback} von Schnitten:
    \begin{math}
        f^*: \Gamma(N, E) &\to \Gamma(M, f^*E) \\
        s &\mapsto s \circ f
    \end{math}
    \begin{note}
        Anschaulich bedeutet $f^* E$, dass an jedem $p \in M$ der Vektorraum $E_{f(p)}$ „angehängt wird“.
        Die umgekehrte Konstruktion „Pushforward“  funktioniert im Allgemeinen für Vektorbündel nicht (z.B. weil $f$ im Allgemeinen kein Homömorphismus ist).
        Für Diffeomorphismen $g: N \to M$ definieren wir den Pushforward als $g_* E := (g^{-1})^* E$.
    \end{note}
\end{df}

\begin{df} \label{5.8}
    Für zwei Vektorbündel $\pi: E \to M$, $\tilde \pi: F \to 1$ über derselben Mannigfaltigkeit $M$ heißt eine Abbildung $\phi: E \to F$ heißt \emphdef{Vektorbündelhomomorphismus}, falls
    \begin{enumerate}[(i)]
        \item
            das folgende Diagramm kommutiert:
            \begin{math}
                \begin{tikzcd}
                    E \ar[rr,"\phi"] \ar[rd,"\pi"'] & & F \ar[ld,"\tilde \pi"] \\
                    & M
                \end{tikzcd},
            \end{math}
            d.h. $\phi$ bildet für alle $p \in M$ die Faser $E_p$ in die Faser $F_p$ ab
        \item
            und $\phi$ \emphdef{faserweise linear} ist, d.h. die nach (i) induzierte Abbildungen $\phi_p: E_p \to F_p$ sind linear für alle $p \in M$.
    \end{enumerate}
    \begin{note}
        Zum Beispiel ist die Weingartenabbildung einer regulären Fläche ein Vektorbündelendomorphismus.
    \end{note}
\end{df}


\section{Tensoren und ein „Baukasten“ für Vektorräume und Vektorbündel}

Die Grundkonstruktionen (für Vektorräume) der linearen Algebra liefern zu gegebenen Vektorräumen $V, W, \dotsc$ neue, daraus konstruierte, z.B. die direkte Summe $V \oplus W$, den Dualraum $V^*$, die linearen Abbildungen $\Hom(V,W)$, die Bilinearformen $\Bil(V)$.

Diese Konstruktionen wollen wir auf Vektorbündel übertragen und unter einem einheitlichen Gesichtspunkt zusammenfassen.


\begin{df} \label{5.9}
    Seien $\pi: E \to M$, $\tilde \pi: F \to M$ Vektorbündel über der selben Mannigfaltigkeit $M$.
    Seien (ohne Einschränkung durch entsprechende Einschränkung) die lokalen Trivialisierungen
    \begin{math}
        h_j: \pi^{-1}(U_j) &\to U_j \times V,
        \tilde h_j: \tilde \pi^{-1}(U_j) &\to U_j \times W
    \end{math}
    gegeben.

    Dann ist die direkte Summe $E \oplus F \to M$ das Vektorbündel
    \begin{math}
        E \oplus F := \Set{(p,v,w) & p \in M, v \in E_p, w \in F_p}
        \xto{\proj_1} M
    \end{math}
    mit den lokalen Trivialisierungen
    \begin{math}
        \hat h_j: \proj_1^{-1}(U_j) &\to U_j \times (V \oplus W) \\
        (p,v,w) &\mapsto (p, h_j(v), \tilde h_j(w)).
    \end{math}
    Analog definiert man $E^*$, $\Hom(E, F)$, $E / F$ (\emphdef{Quotientenbündel} für ein Unterbündel $F \subset E$), z.B.
    \begin{math}
        E^* = \Set{ (p, \alpha) & p \in M, \alpha \in E_p^*}
    \end{math}
    mit lokaler Trivialisierung $h_j^*: \Set{(p, \alpha) & p \in U_j, \alpha \in E_p^*} \to U_j \times V^*$ durch $(p, \alpha) \mapsto (p, \alpha \circ h_j^{-1}(p, \argdot))$.
\end{df}


Das \emphdef{Tensorprodukt} $V \otimes W$ zweier Vektorräume $V$ und $W$ ist eine Konstruktion mit der man bilineare Abbildungen $V \times W \to Z$ beschreiben kann.

Beispielsweise lässt sich jede Bilinearform $\beta: V \times W \to \K$ durch die Werte von $\beta$ auf Paaren von Basisvektoren $(v_i, w_j)$ beschreiben, wobei $(v_i)_{i\in I}$, $(w_j)_{j\in J}$ Baasen von $V$, bzw. $W$ sind, dann gilt ja:
Ist $v = \sum_i \lambda_i v_i$, $w = \sum_j \mu_j w_j$, so gilt
\begin{math}
    \beta(v,w)
    = \beta\l(\sum_i \lambda_i v_i, \sum_j \mu_j w_j\r)
    = \sum_{i,j} \lambda_i \mu_j \beta(v_i, w_j).
\end{math}
Die \emph{bilineare} Abbildung $\beta$ lässt sich somit durch eine lineare Abbildung $U \to \K$ ersetzen, wobei $U$ der Vektorraum ist, für den die Paare $(v_i, w_j)$ eine Basis bilden.
(Bei endlich-dimensionalen Vektorräumen $V, W$ ordnet man $\beta$ die Strukturmatrix $\beta(v_i, w_j))_{i,j}$ zu.)









