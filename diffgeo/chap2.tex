\chapter{Differenzierbare Abbildungen}


Wir können nun definieren, was differenzierbare Abbildungen zwischen Mannigfaltigkeiten sind.

\begin{df} \label{2.1}
    Sei $M$ eine $m$-dimensionale und $N$ eine $n$-dimensionale hinreichend glatte Mannigfaltigkeit.    
    Eine Abbildung $f: M \to N$ heißt $k$-fach differenzierbar, wenn es zu jedem Punkt $p \in M$ Umgebungen $V \subset M$ um $p$ und $W \subset N$ um $f(p)$ mit Karten $\phi: V \to U$ und $\psi: W \to X$ existieren, sodass
    \begin{math}
        \psi \circ f \circ \phi^{-1}
    \end{math}
    $k$-fach stetig differenzierbar ist.

    Für jede Karte $\phi: V \to U \subset \R^m$ sind die Komponentenfunktionen $\phi^j: V \to \R$, genannt \emphdef{Koordinaten}, definiert.
    
    Mit $C^\infty(M)$ bezeichnen wir die Menge der glatten Funktionen auf $M$.
    \begin{note}
        \begin{itemize}
            \item
                Der Differenzierbarkeits-Begriff ist wohldefiniert, da wegen der Verträglichkeit aller Karten die Wahl der Karten keine Rolle spielt.
            \item
                Der Spezialfall $N = \R$ definiert Differenzierbarkeit von Funktionen (hier: skalare, reellwertige Abbildung).
        \end{itemize}
    \end{note}
\end{df}

\begin{df} \label{2.2}
    Eine Abbildung $f: M \to N$ zwischen Mannigfaltigkeiten heißt (glatter) Diffeomorphismus, wenn sie glatt und bijektiv ist und ihre Umkehrabbildung ebenfalls glatt ist.

    Existieren solche Abbildungen $f: M \to N$, so heißen $M$ und $N$ \emphdef{diffeomorph}.
    \begin{note}
        Diffeomorphie ist der Isomorphiebegriff in der Kategorie der glatten Mannigfaltigkeiten und glatten Abbildungen zwischen ihnen.
    \end{note}
\end{df}

\begin{ex}
    Sei $M = \R$ mit dem Atlab $\Set{\phi: M \to \R, x \mapsto x}$ und $N = \R$ mit dem Atlas $\Set{\psi: N \to \R, x \mapsto x^3}$.
    Damit ist auf $M$ und auf $N$ eine differenzierbare Struktur festgelegt über die Atlanten.
    Dies sind \emph{nicht} äquivalent, in der Tat, einer der beiden Kartenwechsel ist die Umkehrabbildung von $x \mapsto x^3$, also nicht differenzierbar.

    Aber $f: N \to M$, $x \mapsto x^3$ ist ein Diffeomorphismus, denn $\phi \circ f \circ \psi^{-1} = \id_\R$ und $\psi \circ f^{-1} \phi^{-1} = \id_\R$ sind offensichtlich glatt.
\end{ex}

\begin{df} \label{2.3}
    Sei $N$ eine $n$-dimensionale (glatte) Mannigfaltigkeit und $M \subset N$,
    so dass gilt: Es gibt um jeden Punkt $p \in M$ eine Karte $\phi: V \to U$ von $N$ mit der
    Eigenschaft
    \begin{math}
        \phi(V \cap M) = U \cap \R^k,
    \end{math}
    wobei für $k \in \Set{0, \dotsc, n}$ hier $\R^k \subset \R^n$ den Untervektorraum mit letzten $n-k$ Koordinaten gleich Null bezeichnet.

    Dann heißt $M$ eine $k$-dimensionale \emphdef{Untermannigfaltigkeit} von $N$.
\end{df}

\begin{ex}
    \begin{itemize}
        \item
            Reguläre Flächen im $\R^3$ sind 2-dimensionale Untermannigfaltigkeit des $\R^3$.
    \end{itemize}
\end{ex}
