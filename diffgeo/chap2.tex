\chapter{Differenzierbare Abbildungen}


Wir können nun definieren, was differenzierbare Abbildungen zwischen Mannigfaltigkeiten sind.

\begin{df} \label{2.1}
    Sei $M$ eine $m$-dimensionale und $N$ eine $n$-dimensionale hinreichend glatte Mannigfaltigkeit.
    Eine Abbildung $f: M \to N$ heißt $k$-fach differenzierbar, wenn es zu jedem Punkt $p \in M$ Umgebungen $V \subset M$ um $p$ und $W \subset N$ um $f(p)$ mit Karten $\phi: V \to U$ und $\psi: W \to X$ existieren, sodass
    \begin{math}
        \psi \circ f \circ \phi^{-1}
    \end{math}
    $k$-fach stetig differenzierbar ist.

    Für jede Karte $\phi: V \to U \subset \R^m$ sind die Komponentenfunktionen $\phi^j: V \to \R$, genannt \emphdef{Koordinaten}, definiert.

    Mit $C^\infty(M)$ bezeichnen wir die Menge der glatten Funktionen auf $M$.
    \begin{note}
        \begin{itemize}
            \item
                Der Differenzierbarkeits-Begriff ist wohldefiniert, da wegen der Verträglichkeit aller Karten die Wahl der Karten keine Rolle spielt.
            \item
                Der Spezialfall $N = \R$ definiert Differenzierbarkeit von Funktionen (hier: skalare, reellwertige Abbildung).
        \end{itemize}
    \end{note}
\end{df}

\begin{df} \label{2.2}
    Eine Abbildung $f: M \to N$ zwischen Mannigfaltigkeiten heißt (glatter) Diffeomorphismus, wenn sie glatt und bijektiv ist und ihre Umkehrabbildung ebenfalls glatt ist.

    Existieren solche Abbildungen $f: M \to N$, so heißen $M$ und $N$ \emphdef{diffeomorph}.
    \begin{note}
        Diffeomorphie ist der Isomorphiebegriff in der Kategorie der glatten Mannigfaltigkeiten und glatten Abbildungen zwischen ihnen.
    \end{note}
\end{df}

\begin{ex}
    Sei $M = \R$ mit dem Atlab $\Set{\phi: M \to \R, x \mapsto x}$ und $N = \R$ mit dem Atlas $\Set{\psi: N \to \R, x \mapsto x^3}$.
    Damit ist auf $M$ und auf $N$ eine differenzierbare Struktur festgelegt über die Atlanten.
    Dies sind \emph{nicht} äquivalent, in der Tat, einer der beiden Kartenwechsel ist die Umkehrabbildung von $x \mapsto x^3$, also nicht differenzierbar.

    Aber $f: N \to M$, $x \mapsto x^3$ ist ein Diffeomorphismus, denn $\phi \circ f \circ \psi^{-1} = \id_\R$ und $\psi \circ f^{-1} \phi^{-1} = \id_\R$ sind offensichtlich glatt.
\end{ex}

\begin{df} \label{2.3}
    Sei $N$ eine $n$-dimensionale (glatte) Mannigfaltigkeit und $M \subset N$,
    so dass gilt: Es gibt um jeden Punkt $p \in M$ eine Karte $\phi: V \to U$ von $N$ mit der
    Eigenschaft
    \begin{math}
        \phi(V \cap M) = U \cap \R^k,
    \end{math}
    wobei für $k \in \Set{0, \dotsc, n}$ hier $\R^k \subset \R^n$ den Untervektorraum mit letzten $n-k$ Koordinaten gleich Null bezeichnet.

    Dann heißt $M$ eine $k$-dimensionale \emphdef{Untermannigfaltigkeit} von $N$.
    \begin{nt*}
        Eine $k$-dimensionale Untermannigfaltigkeit einer Mannigfaltigkeit $N$ ist selbst eine $k$-dimensionale Mannigfaltigkeit.
    \end{nt*}
\end{df}

\begin{ex*}
    \begin{itemize}
        \item
            Reguläre Flächen im $\R^3$ sind 2-dimensionale Untermannigfaltigkeit des $\R^3$.
\Timestamp{2015-10-21}
        \item
            Die Einheitssphäre $S^n$ im $\R^{n+1}$ ist eine $n$-dimensionale Untermannigfaltigkeit.
        \item
            Jede diskrete Teilmenge $M \subset N$ ist eine $0$-dimensionale Untermannigfaltigkeit von $N$.
        \item
            Die $\dim N$-dimensionalen Untermannigfaltigkeiten von $N$ sind genau die offenen Teilmengen von $N$.
        \item
            Affine Unterräume des $\R^n$
        \item
            Die orthogonalen Matrizen
            \begin{math}
                O(n) = \Set{A \in \R^{n\times n} & A^tA = I}
            \end{math}
            bilden eine $\frac{1}{2}n(n-1)$-dimensionale Untermannigfaltigkeit des $\R^{n\times n} = \R^{n^2}$.
    \end{itemize}
\end{ex*}

\begin{df}
    Sei $U \subset \R^m$ eine offene Teilmenge.
    Eine Abbildung $F: U \to \R^n$ heißt
    \begin{enumerate}[i)]
        \item
            \emphdef{Submersion}, wenn ihre Ableitung $\Df[F] (x)$ eine surjektive,
        \item
            \emphdef{Immersion}, wenn sie eine injektive, oder
        \item
            \emphdef{lokaler Diffeomorphismus}, wenn sie eine bijektive
    \end{enumerate}
    lineare Abbildung $\R^m \to \R^n$ darstellt.

    Eine glatte Abbildung heißt \emphdef{Submersion}, \emphdef{Immersion}, \emphdef{lokaler Diffeomorphismus}, wenn zu jedem $p \in M$ Karten $\phi$ um $p$ und $\psi$ um $f(p)$ exstieren, so dass $\psi \circ f \circ \phi^{-1}$ eine Submersion, Immersion, bzw. lokaler Diffeomorphismus ist.
\end{df}

\begin{st}[Reguläres Urbild] \label{2.5}
    Sei $f: M \to N$ eine Submersion und $q \in N$.
    Dann ist $f^{-1}(\Set{q}) \subset M$ eine $(\dim M - \dim N)$-dimensionale Untermannigfaltigkeit in $M$.
    \begin{proof}
        Betrachte die Situation lokal.
        Satz von der impliziten Funktion auf $\phi^{-1} \circ f \circ \psi$.
        Der so gewonnene Graph ist eine Untermannigfaltigkeit in $\R^m$.
    \end{proof}
\end{st}

\begin{ex}
    Ist $f: U \subset \R^n \to \R$ eine glatte Funktion mit nichtverschwindendem Gradienten, so ist durch $f^{-1}(\Set{0})$ ein reguläre Fläche definiert (evtl. leer).

    Beispielsweise $f(x,y,z) = x^2 + y^2 + z^2 - 1$.
    Die Untermannigfaltigkeitseigenschaft folgt daraus, dass $f:\R^3 \setminus \Set{0} \to \R$ eine Submersion ist.
    Denn es gilt
    \begin{math}
        \Nabla f(x,y,z) = \Vector{2x & 2y & 2z} \neq 0.
    \end{math}
\end{ex}

\begin{df} \label{2.6}
    Eine Immersion $f:M \to N$ heißt \emphdef{Einbettung}, falls $f: M \to f(M)$ ein Homöomorphismus ist.
    \begin{note}
        Diese Bedingung garantiert, dass $f$ keine Selbstannäheruungen hat (insbesondere keine Selbstdurchschneidungen).
        Übung: Das Bild einer Einbettung ist eine Untermannigfaltigkeit.
    \end{note}
\end{df}

\begin{ex}
    \begin{itemize}
        \item
            Standard-Einbettung $\R^k \injto \R^n$, $k \le n$.
        \item
            Gerade mit irrationaler Steigung auf dem Torus $\R \to T^2$ ist keine Einbettung.
    \end{itemize}
\end{ex}

\begin{df} \label{2.7}
    Eine Mannigfaltigkeit $G$ ist eine \emph{Liegruppe}, wenn auf $G$ eine Verknüpfung $G \times G \to G$ gegeben ist, mit der $G$ eine Gruppe ist und die Gruppenoperationen $\cdot: G \times G \to G$, $(a,b) \mapsto ab$ und $\argdot^{-1}: G \to G$, $a \mapsto a^{-1}$ glatte Abbildungen sind.
    \begin{note}
        $G \times G$ ist eine $(2 \dim G)$-dimensionale Mannigfaltigkeit, siehe Übung.
        Äquivalent kann man fordern, dass $G\times G \to G$, $(a,b) \mapsto ab^{-1}$ glatt ist.
    \end{note}
\end{df}

\begin{ex}
    \begin{itemize}
        \item
            Die invertierbaren Matrizen
            \begin{math}
                \Gl(n,\R) = \Set{A \in \R^{n\times n} & \det(A) \neq 0}
            \end{math}
            ist eine Liegruppe.
            \begin{proof}
                Mannigfaltigkeit wegen $\Gl(n,\R) = \det^{-1}(\R \setminus 0)$.
                Die Matrizenmultiplikation sind offensichtlich glatt, benutze die explizite Formel für die Inverse, um zu begründen, dass $A \mapsto A^{-1}$ glatt ist.
            \end{proof}
    \end{itemize}
\end{ex}
