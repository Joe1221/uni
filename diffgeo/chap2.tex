\chapter{Differenzierbare Abbildungen}


Wir können nun definieren, was differenzierbare Abbildungen zwischen Mannigfaltigkeiten sind.
Im Folgenden seinen $M, N$ stets hinreichend glatte Mannigfaltigkeiten der Dimension $m$, bzw. $n$.

\begin{df} \label{2.1}
    Eine Abbildung $f: M \to N$ zwischen Mannigfaltigkeiten heißt $k$-fach differenzierbar, wenn es zu jedem Punkt $p \in M$ Umgebungen $V \subset M$ um $p$ und $W \subset N$ um $f(p)$ mit Karten $\phi: V \to U$ und $\psi: W \to X$ existieren, sodass
    \begin{math}
        \psi \circ f \circ \phi^{-1}
    \end{math}
    $k$-fach stetig differenzierbar ist.

    Für jede Karte $\phi: V \to U \subset \R^m$ sind die Komponentenfunktionen $\phi^j: V \to \R$, genannt \emphdef{Koordinaten}, definiert.

    Mit $C^\infty(M)$ bezeichnen wir die Menge der glatten Funktionen auf $M$.
    \begin{note}
        \begin{itemize}
            \item
                Der Differenzierbarkeits-Begriff ist wohldefiniert, da wegen der Verträglichkeit aller Karten die Wahl der Karten keine Rolle spielt.
            \item
                Der Spezialfall $N = \R$ definiert Differenzierbarkeit von Funktionen (d.h. in dieser Vorlesung skalare, reellwertige Abbildung).
        \end{itemize}
    \end{note}
\end{df}

\begin{df} \label{2.2}
    Eine Abbildung $f: M \to N$ zwischen Mannigfaltigkeiten heißt (glatter) Diffeomorphismus, wenn sie glatt und bijektiv ist und ihre Umkehrabbildung ebenfalls glatt ist.

    Existiert eine solche Abbildung $f: M \to N$, so heißen $M$ und $N$ \emphdef{diffeomorph}.
    \begin{note}
        Diffeomorphie ist der Isomorphiebegriff in der Kategorie der glatten Mannigfaltigkeiten und glatten Abbildungen zwischen ihnen.
    \end{note}
\end{df}

\begin{ex*}
    Sei $M = \R$ mit dem Atlas $\Set{\phi: M \to \R, x \mapsto x}$ und $N = \R$ mit dem Atlas $\Set{\psi: N \to \R, x \mapsto x^3}$.
    Damit ist auf $M$ und auf $N$ eine differenzierbare Struktur festgelegt über die Atlanten.

    Diese sind \emph{nicht} äquivalent: einer der beiden Kartenwechsel ist die Umkehrabbildung von $x \mapsto x^3$, also nicht differenzierbar.

    Trotzdem ist $f: N \to M$, $x \mapsto x^3$ ein Diffeomorphismus, denn $\phi \circ f \circ \psi^{-1} = \id_\R$ und $\psi \circ f^{-1} \circ \phi^{-1} = \id_\R$ sind offensichtlich glatt.
\end{ex*}

\begin{df}[Untermannigfaltigkeit] \label{2.3}
    Sei $M \subset N$.
    Falls um jeden Punkt $p \in M$ eine Karte $\phi: V \to U$ von $N$ mit der
    Eigenschaft
    \begin{math}
        \phi(V \cap M) = U \cap \R^m,
    \end{math}
    existiert (wobei für $m \in \Set{0, \dotsc, n}$ hier $\R^m \subset \R^n$ den Untervektorraum mit letzten $n-m$ Koordinaten gleich Null bezeichnet), dann heißt $M$ eine $m$-dimensionale \emphdef{Untermannigfaltigkeit} von $N$.
    \begin{note}
        Wie die Bezeichnung nahelegt, ist $M$ selbst eine entsprechend glatte $k$-dimensionale Mannigfaltigkeit.
    \end{note}
\end{df}

\begin{ex*}
    \begin{itemize}
        \item
            Reguläre Flächen im $\R^3$ sind 2-dimensionale Untermannigfaltigkeiten des $\R^3$.
\Timestamp{2015-10-21}
        \item
            Die Einheitssphäre $S^n$ im $\R^{n+1}$ ist eine $n$-dimensionale Untermannigfaltigkeit.
        \item
            Jede diskrete Teilmenge $M \subset N$ ist eine $0$-dimensionale Untermannigfaltigkeit von $N$.
        \item
            Die $n$-dimensionalen Untermannigfaltigkeiten von $N$ sind genau die offenen Teilmengen von $N$.
        \item
            Affine Unterräume des $\R^n$ bilden Untermannigfaltigkeiten mit gleicher Dimension wie die der zugrundeliegenden Vektorräume.
        \item
            Die orthogonalen Matrizen
            \begin{math}
                O(n) = \Set{A \in \R^{n\times n} & A^tA = I}
            \end{math}
            bilden eine $\frac{1}{2}n(n-1)$-dimensionale Untermannigfaltigkeit des $\R^{n\times n} = \R^{n^2}$.
    \end{itemize}
\end{ex*}

\begin{df} \label{2.4}
    Sei $U \subset \R^m$ offen, $F: U \to \R^n$ eine Abbildung mit Ableitung $\Df[F]|_{x}: \R^m \to \R^n$ in $x$ (als lineare Abbildung).
    Dann heißt $F$
    \begin{enumerate}[i)]
        \item
            \emphdef{Submersion}, wenn $\Df[F]|_{x}$ surjektiv,
        \item
            \emphdef{Immersion}, wenn $\Df[F]|_{x}$ injektiv und
        \item
            \emphdef{lokaler Diffeomorphismus}, wenn $\Df[F]|_{x}$ bijektiv ist
    \end{enumerate}
    für alle $x \in U$.

    Eine glatte Abbildung $f: M \to N$ zwischen Mannigfaltigkeiten heißt \emphdef{Submersion}, \emphdef{Immersion}, \emphdef{lokaler Diffeomorphismus}, wenn zu jedem $p \in M$ Karten $\phi$ um $p$ und $\psi$ um $f(p)$ existieren, sodass $\psi \circ f \circ \phi^{-1}$ eine Submersion, Immersion, bzw. lokaler Diffeomorphismus ist.
\end{df}

\begin{st}[Reguläres Urbild] \label{2.5}
    Sei $f: M \to N$ eine Submersion und $q \in N$.
    Dann ist $f^{-1}(\Set{q}) \subset M$ eine $(m - n)$-dimensionale Untermannigfaltigkeit in $M$.
    \begin{proof}
        Betrachte die Situation lokal.
        Satz von der impliziten Funktion auf $\psi^{-1} \circ f \circ \phi$.
        Der so gewonnene Graph ist eine Untermannigfaltigkeit in $\R^m$.
    \end{proof}
\end{st}

\begin{ex*}
    Ist $f: U \subset \R^n \to \R$ eine glatte Funktion mit nichtverschwindendem Gradienten, so ist durch $f^{-1}(\Set{0})$ ein reguläre Fläche definiert (evtl. leer).

    Betrachte z.B. $f(x,y,z) = x^2 + y^2 + z^2 - 1$.
    $f:\R^3 \setminus \Set{0} \to \R$ ist eine Submersion,
    denn es gilt
    \begin{math}
        \Nabla f(x,y,z) = \Vector{2x & 2y & 2z} \neq 0.
    \end{math}
    Also ist $f^{-1}(\Set{0})$ eine Untermannigfaltigkeit der Dimension 2.
\end{ex*}

\begin{df} \label{2.6}
    Eine Immersion $f:M \to N$ heißt \emphdef{Einbettung}, falls $f: M \to f(M)$ ein Homöomorphismus ist.
    \begin{note}
        Diese Bedingung garantiert, dass $f$ keine Selbstannäherungen hat (insbesondere keine Selbstdurchschneidungen).
        Übung: Das Bild einer Einbettung ist eine Untermannigfaltigkeit.
    \end{note}
\end{df}

\begin{ex}
    \begin{itemize}
        \item
            Standard-Einbettung $\R^k \injto \R^n$, $k \le n$.
        \item
            Gerade mit irrationaler Steigung auf dem Torus $\R \to T^2$ ist keine Einbettung.
    \end{itemize}
\end{ex}

\begin{df} \label{2.7}
    Sei $G$ eine glatte Mannigfaltigkeit und $\cdot: G \times G \to G$, $\argdot^{-1}: G \to G$ glatte Abbildungen.
    Bildet $(G, \cdot)$ eine Gruppe mit Inversion $\argdot^{-1}$, so nennen wir $G$ eine \emphdef{Lie-Gruppe}.
    \begin{note}
        $G \times G$ ist eine $2g$-dimensionale Mannigfaltigkeit, siehe Übung.
        Äquivalent kann man fordern, dass $G\times G \to G$, $(a,b) \mapsto ab^{-1}$ glatt ist.
    \end{note}
\end{df}

\begin{ex}
    Die invertierbaren Matrizen
    \begin{math}
        \Gl(n,\R) = \Set{A \in \R^{n\times n} & \det(A) \neq 0}
    \end{math}
    bilden eine Liegruppe.
    \begin{proof}
        Mannigfaltigkeit wegen $\Gl(n,\R) = \det^{-1}(\R \setminus \Set 0)$.
        Die Matrizenmultiplikation ist offensichtlich glatt, benutze die explizite Formel für die Inverse, um zu begründen, dass $A \mapsto A^{-1}$ glatt ist.
    \end{proof}
\end{ex}
