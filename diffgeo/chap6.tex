\chapter{Differentialformen}

\begin{df} \label{6.1}
    Sei $V$ ein endlich-dimensionaler reeller Vektorraum und für $q \in \N_0$ sei $\omega \in {V^*}^{\otimes q}$ eine \emphdef{$q$-Form}, d.h. eine multilineare Abbildung $V^k \to \R$, $(v_1, \dotsc, v_q) \mapsto \omega(v_1, \dotsc, v_q)$.

    $\omega$ heißt \emphdef{alternierend}, falls $\omega(v_1, \dotsc, v_q) = 0$ wenn zwei der Argumente $v_1, \dotsc, v_q$ gleich sind.
\end{df}

\begin{ex*}
    Die Determinanten von rellen $n\times n$-Matrizen ist eine alternierende $n$-Form auf den Spalten (Zeilen) einer Matrix, $\det: (\R^n)^n \to \R$.
\end{ex*}

\begin{lem} \label{6.2}
    Sei $\omega \in {V^*}^{\otimes q}$ eine alternierende $q$-Form.
    Dann gilt für $i,j \in \Set{1, \dotsc, q}$, $\lambda \in \R$
    \begin{enumerate}[(i)]
        \item
            Scherung:
            \begin{math}
                \omega(v_1, \dotsc, v_{i-1}, v_i + \lambda v_j, v_{i+1}, \dotsc, v_q) = \omega(v_1, \dotsc, v_q)
            \end{math}
        \item
            Antikommutativität:
            \begin{math}
                \omega(v_1, \dotsc, v_j, \dotsc, v_i, \dotsc, v_q) = - \omega(v_1, \dotsc, v_i, \dotsc, v_j, \dotsc, v_q)
            \end{math}
        \item
            sind $v_1, \dotsc, v_q$ linear abhängig, dann gilt $\omega(v_1, \dotsc, v_q) = 0$.
    \end{enumerate}
    \begin{proof}
        leichte Übung (Linearität und alternierend).
    \end{proof}
\end{lem}

