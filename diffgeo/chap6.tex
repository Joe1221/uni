\chapter{Differentialformen}

\begin{df} \label{6.1}
    Sei $V$ ein endlich-dimensionaler reeller Vektorraum und für $q \in \N_0$ sei $\omega \in {V^*}^{\otimes q}$ eine \emphdef{$q$-Form}, d.h. eine multilineare Abbildung $V^k \to \R$, $(v_1, \dotsc, v_q) \mapsto \omega(v_1, \dotsc, v_q)$.

    $\omega$ heißt \emphdef{alternierend}, falls $\omega(v_1, \dotsc, v_q) = 0$ wenn zwei der Argumente $v_1, \dotsc, v_q$ gleich sind.
\end{df}

\begin{ex*}
    Die Determinanten von rellen $n\times n$-Matrizen ist eine alternierende $n$-Form auf den Spalten (Zeilen) einer Matrix, $\det: (\R^n)^n \to \R$.
\end{ex*}

\begin{lem} \label{6.2}
    Sei $\omega \in {V^*}^{\otimes q}$ eine alternierende $q$-Form.
    Dann gilt für $i,j \in \Set{1, \dotsc, q}$, $\lambda \in \R$
    \begin{enumerate}[(i)]
        \item
            Scherung:
            \begin{math}
                \omega(v_1, \dotsc, v_{i-1}, v_i + \lambda v_j, v_{i+1}, \dotsc, v_q) = \omega(v_1, \dotsc, v_q)
            \end{math}
        \item
            Antikommutativität:
            \begin{math}
                \omega(v_1, \dotsc, v_j, \dotsc, v_i, \dotsc, v_q) = - \omega(v_1, \dotsc, v_i, \dotsc, v_j, \dotsc, v_q)
            \end{math}
        \item
            sind $v_1, \dotsc, v_q$ linear abhängig, dann gilt $\omega(v_1, \dotsc, v_q) = 0$.
    \end{enumerate}
    \begin{proof}
        leichte Übung (Linearität und alternierend).
    \end{proof}
\end{lem}


\Timestamp{2016-01-13}

\begin{df} \label{6.3}
    Die $q$-te \emphdef{äußere Potenz} von $V^*$ ist der Vektorraum
    \begin{math}
        \bigwedge^q V^* := \Set{\omega \in {V^*}^{\otimes q} & \text{$\omega$ ist alternierend} }.
    \end{math}
    \begin{note}
        \begin{itemize}
            \item
                Es gilt $\bigwedge^q V^* = 0$, falls $q > n := \dim(V)$, da dann $q$ Vektoren stets linear abhängig sind.
            \item
                Es gilt $\bigwedge^0 V^* = \R$, $\bigwedge^1 V^* = V^*$.
            \item
                Aus der linearen Algebra ist bekannt, dass $\bigwedge^n V^*$ (mit $n = \dim V$) eindimensional ist, omit wird der Raum von dem (nichttrivialen) Element
                \begin{math}
                    (v_1, \dotsc, v_n) \mapsto \det (e^j(v_k)_{1\le j,k \le n}
                \end{math}
                erzeugt, wobei $e^1, \dotsc, e^n$ eine Basis von $V^*$ bilden.
            \item
                Wir haben gesehen, dass alternierende $q$-Formen antikommutativ sind.
                Daraus folgt für eine Permutation $\sigma \in \Sym(q)$.
                \begin{math}
                    \omega(v_{\sigma(1)}, \dotsc, v_{\sigma(q)})
                    = \sgn(\omega) w(v_1, \dotsc, v_q).
                \end{math}
                Ist $\sigma \in \Sym(q)$ und $\omega \in {V^*}^{\otimes q}$, dann schreiben wir
                \begin{math}
                    (\omega \circ \sigma)(v_1, \dotsc, v_q)
                    := w(v_{\sigma(1)}, \dotsc, v_{\sigma(q)}).
                \end{math}
        \end{itemize}
    \end{note}
\end{df}

\begin{df} \label{6.4}
    Wir definieren die \emphdef{Antisymmetrisierung} von $q$-Formen als
    \begin{math}
        \pi_{\wedge}: {V^*}^{\otimes q} &\to \bigwedge^q V^* \\
        \omega &\mapsto \frac{1}{q!} \sum_{\sigma \in \Sym(q)}  \sgn(\sigma) \omega \circ \sigma.
    \end{math}
    Dies ist eine Vektorraum-Projektion, insbesondere $\pi_{\wedge}(\omega) = \omega$ für $\omega \in \bigwedge^q V^*$.

\end{df}

Sei nun $\omega$ eine alternierende $q$-Form und seien $v_1, \dotsc, v_q \in V$
Sei $e_1, \dotsc, e_n$ eine Basis von $V$ und $e^1, \dotsc, e^n$ die dazu duale Basis von $V^*$, d.h. die Linearformen mit $e^j(e_k) := \delta_{jk}$.
Schreiben wir $v_j = \sum_{k=1}^n \lambda_{jk} e_k \in V$, dann gilt
\begin{math}
    \omega(v_1, \dotsc, v_q)
    &= \sum_{k_1, \dotsc, k_q = 1}^n \lambda_{1,k_1} \dotsb \lambda_{q,k_q} \omega(e_{k_1}, \dotsc, e_{k_q}) \\
    &= \sum_{\substack{k_1, \dotsc, k_q = 1 \\ k_i \neq k_j}}^n \lambda_{1,k_1} \dotsb \lambda_{q,k_q} \omega(e_{k_1}, \dotsc, e_{k_q}) \\
    &= \sum_{1 \le k_1 < \dotsc < k_q \le n} \sum_{\sigma \in \Sym(q)} \lambda_{1,k_{\sigma(1)}} \dotsb \lambda_{q, k_{\sigma(q)}} \omega(e_{k_{\sigma(1)}}, \dotsc, e_{k_{\sigma(q)}}) \\
    &= \sum_{1 \le k_1 < \dotsc < k_q \le n} \sum_{\sigma \in \Sym(q)} \lambda_{1,k_{\sigma(1)}} \dotsb \lambda_{q, k_{\sigma(q)}} \sgn(\omega) \omega(e_{k_1}, \dotsc, e_{k_q}).
\end{math}
Dies zeigt: Es genügt, die Wert von $\omega$ auf $(e_{k_1}, \dotsc e_{k_q})$ mit $1 \le k_1 < \dotsb < k_q \le n$ zu kennen.
Definiere nun für $k_1, \dotsc, k_q \in \Set{1, \dotsc, n}$ beliebig
\begin{math}
    e^{k_1} \wedge \dotsb \wedge e^{k_q}
    := \sum_{\sigma \in \Sym(q)} \sgn(\sigma) e^{k_{\sigma(1)}} \otimes \dotsb \otimes e^{k_{\sigma(q)}}.
\end{math}
Dies ist eine alternierende $q$-Form.

Seien nun für $1 \le k_1 < \dotsc < k_q \le n$ die Zahlen $w_{k_1, \dotsc, k_q}$ definiert als die Werte (für beliebiges $\omega$)
\begin{math}
    w_{k_1, \dotsc, k_q} := \omega(e_{k_1}, \dotsc e_{k_2}, \dotsc, e_{k_q}).
\end{math}
Dann ist die $q$-Form $\omega$ wie oben gegeben durch
\begin{math}
    \omega := \sum_{1\le k_1 < \dotsc < k_q \le n} w_{k_1, \dotsc, k_q} e^{k_1} \wedge \dotsb \wedge e^{k_q}.
\end{math}
Dies zeigt:
\begin{math}
    \Set{e^{k_1} \wedge e^{k_q} & 1 \le k_1 < \dotsb < k_q \le n}
\end{math}
bildet eine Basis von $\bigwedge^q V^*$ (die lineare Unabhängigkeit sieht man, indem man verschiedene $q$-Tupel von Basisvektoren $(e_{j_1}, \dotsc, e_{j_q})$ einsetzt), somit gilt $\dim(\bigwedge^q V^*) = \binom{n}{q}$.

\begin{ex*}
    \begin{math}
        e^1 \wedge e^2 \wedge e^3 (e_2, e_1, e_3)
        = -e^2 \otimes e^1 \otimes e^3 (e_2, e_1, e_3)
        = -1.
    \end{math}
    Weiter gilt $\pi_{\wedge}(e^{k_1} \otimes e^{k_q}) = \frac{1}{q!} e^{k_1} \wedge \dotsb \wedge e^{k_q}$.
    Vergleich mit der Leibnizformen für die Determinante ergibt für $1$-Formen $\alpha_1, \dotsc, \alpha_q$.
    \begin{math}
        \pi_1(\alpha_1 \otimes \dotsb \otimes \alpha_q)
        &= \frac{1}{q!} \sum_{\sigma \in \Sym(q)} \sgn(\sigma) \alpha_1(v_{\sigma(1)}) \dotsb \alpha_q(v_{\sigma(q)}) \\
        &= \frac{1}{q!} \det( \alpha_j(v_k))_{1 \le j,k \le n}.
    \end{math}
\end{ex*}

\begin{df} \label{6.5}
    Sei $\alpha$ eine alternierende $p$-Form und $\beta$ eine alternierende $q$-Form auf $V$ (d.h. $\alpha \in {V^*}^{\otimes p}$, $\beta \in {V^*}^{\otimes q}$).
    Wir definieren das \emphdef{äußere Produkt} $\alpha \wedge \beta$ durch
    \begin{math}
        \alpha \wedge \beta
        &:= \frac{(p+q)!}{p!q!} \pi_\wedge( \alpha \otimes \beta) \\
        &= \frac{1}{p!q!} \sum_{\sigma \in \Sym(p+q)} \sgn(\sigma) (\alpha \otimes \beta) \circ \sigma \\
        &= \sum_{[\sigma] \in \frac{\Sym(p+q)}{\Sym(p) \times \Sym(q)}} \sgn(\sigma) (\alpha \otimes \beta) \circ \sigma,
    \end{math}
    wobei $\sigma$ die eingesetzten Vektoren permutiert.
\end{df}

\Timestamp{2016-01-19}

\begin{ex}
    Beispielsweise gilt für 1-Formen $\alpha, \beta$
    \begin{math}
        (\alpha \wedge \beta)(X,Y)
        &= \frac{(1+1)!}{1!1!} \pi_\wedge(\alpha \otimes \beta)(X,Y) \\
        &= 2 \cdot \frac{1}{2} ( \alpha(X)\beta(Y) - \beta(X) \alpha(Y)).
    \end{math}
\end{ex}

\begin{st}[Eigenschaften von $\wedge$] \label{6.6}
    Das äußere Produkt ist assziativ und \emphdef{superkommutativ} (oder \emphdef{graduiert kommutativ}), d.h. es gilt
    \begin{math}
        \alpha \wedge \beta = (-1)^{pq} \beta \wedge \alpha
    \end{math}
    für alternierende $p$-Formen $\alpha$ und alternierende $q$-Formen $\beta$.

    Der Vektorraum $\bigwedge^{\bullet} V^\ast := \bigoplus_{q=0}^n \bigwedge^q V^*$ wird damit zu einer assoziativen, graduierte Algebra (es gilt $\dim \bigwedge^\bullet V^\ast = 2^n = 2^{\dim V}$).
    \begin{proof}
        Wegen der Bilinearität genügt es, die Assoziativität auf den Basisvektoren nachzurechnen.
        Dazu prüft man, dass
        \begin{math}
            (e^{j_1} \wedge \dotsb \wedge e^{j_p}) \wedge (e^{k_1} \wedge \dotsb \wedge e^{k_q})
            = e^{j_1} \wedge \dotsb \wedge e^{j_p} \wedge e^{k_1} \wedge \dotsb \wedge e^{k_q}.
        \end{math}
        Dies überprüft man, indem man $(p+q)$-Tupel von Basisvektoren einsetzt.
        Die Formel $\alpha \wedge \beta = (-1)^{pq} \beta \wedge \alpha$ folgt, indem man sie für die Basisvektoren $\alpha = e^{j_1} \wedge \dotsb \wedge e^{j_r}$, $\beta = e^{k_1} \wedge \dotsb \wedge e^{k_q}$ überprüft.
        In der Tat benötigt man $pq$ Nachbarvertauschungen.
    \end{proof}
\end{st}

\begin{df} \label{6.7}
    Sei $M$ eine Mannigfaltigkeit und $n := \dim(M)$.
    Mit $\bigwedge^q \T^* M$ bezeichnen wir den Unterbündel der alternierenden $q$-Formen in ${\T^*}^{\otimes q} M$.

    Sei $\bigwedge^\bullet \T^* M := \bigoplus_{q=0}^n \bigwedge^q \T^* M$.
    Die Schnitte in diesem Bündel heißen \emphdef{Differentialformen} auf $M$.
    Die Menge der Schnitte bezeichnen wir mit $\Omega^\bullet(M) := \Gamma(M, \bigwedge^\bullet \T^* M)$, bzw. $\Omega^q(M) := \Gamma(M, \bigwedge^q \T^* M)$.
\end{df}

\begin{st} \label{6.8}
    Es existiert eine eindeutig bestimmte additive Abbildung $\d \in \End(\Omega^\bullet(M))$ mit folgenden Eigenschaften
    \begin{enumerate}[i)]
        \item
            $\d(\Omega^q(M)) \subset \Omega^{q+1}(M)$.
        \item
            $\d(\alpha \wedge \beta) = \d \alpha \wedge \beta + (-1)^p \alpha \wedge \d \beta$ für alle $\alpha \in \Omega^p(M)$, $\beta \in \Omega^\bullet(M)$.
        \item
            $\d: \Omega^0(M) = C^\infty(M) \to \Gamma(M, \T^* M) = \Omega^1(M)$ ist das übliche Differential auf Funktionen.
        \item
            $\d^2 = 0$.
    \end{enumerate}
    Diese Abbildung $\d$ heißt \emphdef{äußere Ableitung} oder \emphdef{de Rham-Operator}.
\end{st}

Wir betrachten zunächst Differentialformen auf $\R^n$.
Wir schreiben dazu in diesem Zusammenhang $\d f$ für die übliche Ableitung einer glatten Funktion $f:\R^n \to \R$, $\d f(p)$ ist eine \emphdef{Linearform}, d.h. eine lineare Abbildung $\R^n \to \R$.
Die Linearform $\d x_i(p)$ bilden in jedem Punkt eine Basis von $\T_p^* \R^n = {\R^n}^*$, nämlich punktweise die Dualbasis zu dem kartesischen Vektorfeld $\pddx[x_j]$, d.h. es gilt
\begin{math}
    \d x_i (\pddx[x_j]) = \d x_i(e_j) = \delta_{ij}.
\end{math}
Es folgt, dass die Differentialformen
\begin{math}
    \Set{\d x_{i_1} \wedge \dotsb \wedge \d x_{i_q} & 1 \le i_1 < \dotsb < i_q \le n }
\end{math}
in jedem Punkt des $\R^n$ eine Basis der alternierenden $q$-Formen bilden.
Eine beliebige alternierende $q$-Form lässt sich schreiben als
\begin{math}
    \omega = \sum_{1 \le i_1 < \dotsb < i_q \le n} f_{i_1, \dotsc, i_q} \d x_{i_1} \wedge \dotsb \wedge \d x_{i_q}
\end{math}
mit Funktionen $f_{i_1, \dotsc, i_q}: \R^n \to \R$.
Zur Vereinfachung der Schreibweise verwendet man einen Multiindex $I = (i_1, \dotsc, i_q)$ mit $1 \le i_1 < \dotsb < i_q \le n$ und schreibt
\begin{math}
    \d x_I := \d x_{i_1} \wedge \dotsb \wedge \d x_{i_q}
\end{math}
und somit $\omega = \sum_I f_I \d x_I$.



