\chapter{Differentialformen}

\begin{df} \label{6.1}
    Sei $V$ ein endlich-dimensionaler reeller Vektorraum und für $q \in \N_0$ sei $\omega \in {V^*}^{\otimes q}$ eine \emphdef{$q$-Form}, d.h. eine multilineare Abbildung $V^k \to \R$, $(v_1, \dotsc, v_q) \mapsto \omega(v_1, \dotsc, v_q)$.

    $\omega$ heißt \emphdef{alternierend}, falls $\omega(v_1, \dotsc, v_q) = 0$ wenn zwei der Argumente $v_1, \dotsc, v_q$ gleich sind.
\end{df}

\begin{ex*}
    Die Determinanten von rellen $n\times n$-Matrizen ist eine alternierende $n$-Form auf den Spalten (Zeilen) einer Matrix, $\det: (\R^n)^n \to \R$.
\end{ex*}

\begin{lem} \label{6.2}
    Sei $\omega \in {V^*}^{\otimes q}$ eine alternierende $q$-Form.
    Dann gilt für $i,j \in \Set{1, \dotsc, q}$, $\lambda \in \R$
    \begin{enumerate}[(i)]
        \item
            Scherung:
            \begin{math}
                \omega(v_1, \dotsc, v_{i-1}, v_i + \lambda v_j, v_{i+1}, \dotsc, v_q) = \omega(v_1, \dotsc, v_q)
            \end{math}
        \item
            Antikommutativität:
            \begin{math}
                \omega(v_1, \dotsc, v_j, \dotsc, v_i, \dotsc, v_q) = - \omega(v_1, \dotsc, v_i, \dotsc, v_j, \dotsc, v_q)
            \end{math}
        \item
            sind $v_1, \dotsc, v_q$ linear abhängig, dann gilt $\omega(v_1, \dotsc, v_q) = 0$.
    \end{enumerate}
    \begin{proof}
        leichte Übung (Linearität und alternierend).
    \end{proof}
\end{lem}


\Timestamp{2016-01-13}

\begin{df} \label{6.3}
    Die $q$-te \emphdef{äußere Potenz} von $V^*$ ist der Vektorraum
    \begin{math}
        \bigwedge^q V^* := \Set{\omega \in {V^*}^{\otimes q} & \text{$\omega$ ist alternierend} }.
    \end{math}
    \begin{note}
        \begin{itemize}
            \item
                Es gilt $\bigwedge^q V^* = 0$, falls $q > n := \dim(V)$, da dann $q$ Vektoren stets linear abhängig sind.
            \item
                Es gilt $\bigwedge^0 V^* = \R$, $\bigwedge^1 V^* = V^*$.
            \item
                Aus der linearen Algebra ist bekannt, dass $\bigwedge^n V^*$ (mit $n = \dim V$) eindimensional ist, omit wird der Raum von dem (nichttrivialen) Element
                \begin{math}
                    (v_1, \dotsc, v_n) \mapsto \det (e^j(v_k)_{1\le j,k \le n}
                \end{math}
                erzeugt, wobei $e^1, \dotsc, e^n$ eine Basis von $V^*$ bilden.
            \item
                Wir haben gesehen, dass alternierende $q$-Formen antikommutativ sind.
                Daraus folgt für eine Permutation $\sigma \in \Sym(q)$.
                \begin{math}
                    \omega(v_{\sigma(1)}, \dotsc, v_{\sigma(q)})
                    = \sgn(\omega) w(v_1, \dotsc, v_q).
                \end{math}
                Ist $\sigma \in \Sym(q)$ und $\omega \in {V^*}^{\otimes q}$, dann schreiben wir
                \begin{math}
                    (\omega \circ \sigma)(v_1, \dotsc, v_q)
                    := w(v_{\sigma(1)}, \dotsc, v_{\sigma(q)}).
                \end{math}
        \end{itemize}
    \end{note}
\end{df}

\begin{df} \label{6.4}
    Wir definieren die \emphdef{Antisymmetrisierung} von $q$-Formen als
    \begin{math}
        \pi_{\wedge}: {V^*}^{\otimes q} &\to \bigwedge^q V^* \\
        \omega &\mapsto \frac{1}{q!} \sum_{\sigma \in \Sym(q)}  \sgn(\sigma) \omega \circ \sigma.
    \end{math}
    Dies ist eine Vektorraum-Projektion, insbesondere $\pi_{\wedge}(\omega) = \omega$ für $\omega \in \bigwedge^q V^*$.

\end{df}

Sei nun $\omega$ eine alternierende $q$-Form und seien $v_1, \dotsc, v_q \in V$
Sei $e_1, \dotsc, e_n$ eine Basis von $V$ und $e^1, \dotsc, e^n$ die dazu duale Basis von $V^*$, d.h. die Linearformen mit $e^j(e_k) := \delta_{jk}$.
Schreiben wir $v_j = \sum_{k=1}^n \lambda_{jk} e_k \in V$, dann gilt
\begin{math}
    \omega(v_1, \dotsc, v_q)
    &= \sum_{k_1, \dotsc, k_q = 1}^n \lambda_{1,k_1} \dotsb \lambda_{q,k_q} \omega(e_{k_1}, \dotsc, e_{k_q}) \\
    &= \sum_{\substack{k_1, \dotsc, k_q = 1 \\ k_i \neq k_j}}^n \lambda_{1,k_1} \dotsb \lambda_{q,k_q} \omega(e_{k_1}, \dotsc, e_{k_q}) \\
    &= \sum_{1 \le k_1 < k_2 < \dotsc < k_q \le n} \sum_{\sigma \in \Sym(q)} \lambda_{1,k_{\sigma(1)}} \dotsb \lambda_{q, k_{\sigma(q)}} \omega(e_{k_{\sigma(1)}}, \dotsc, e_{k_{\sigma(q)}}) \\
    &= \sum_{1 \le k_1 < k_2 < \dotsc < k_q \le n} \sum_{\sigma \in \Sym(q)} \lambda_{1,k_{\sigma(1)}} \dotsb \lambda_{q, k_{\sigma(q)}} \sgn(\omega) \omega(e_{k_1}, \dotsc, e_{k_q}).
\end{math}
Dies zeigt: Es genügt, die Wert von $\omega$ auf $(e_{k_1}, \dotsc e_{k_q})$ mit $1 \le k_1 < \dotsb < k_q \le n$ zu kennen.
Definiere nun für $k_1, \dotsc, k_q \in \Set{1, \dotsc, n}$ beliebig
\begin{math}
    e^{k_1} \wedge \dotsb \wedge e^{k_q}
    := \sum_{\sigma \in \Sym(q)} \sgn(\sigma) e^{k_{\sigma(1)}} \otimes \dotsb \otimes e^{k_{\sigma(q)}}.
\end{math}
Dies ist eine alternierende $q$-Form.

Seien nun für $1 \le k_1 < \dotsc < k_q \le n$ die Zahlen $w_{k_1, \dotsc, k_q}$ definiert als die Werte (für beliebiges $\omega$)
\begin{math}
    w_{k_1, \dotsc, k_q} := \omega(e_{k_1}, \dotsc e_{k_2}, \dotsc, e_{k_q}).
\end{math}
Dann ist die $q$-Form $\omega$ wie oben gegeben durch
\begin{math}
    \omega := \sum_{1\le k_1 < \dotsc < k_q \le n} w_{k_1, \dotsc, k_q} e^{k_1} \wedge \dotsb \wedge e^{k_q}.
\end{math}
Dies zeigt:
\begin{math}
    \Set{e^{k_1} \wedge e^{k_q} & 1 \le k_1 < \dotsb < k_q \le n}
\end{math}
bildet eine Basis von $\bigwedge^q V^*$ (die lineare Unabhängigkeit sieht man, indem man verschiedene $q$-Tupel von Basisvektoren $(e_{j_1}, \dotsc, e_{j_q})$ einsetzt), somit gilt $\dim(\bigwedge^q V^*) = \binom{n}{q}$.

\begin{ex*}
    \begin{math}
        e^1 \wedge e^2 \wedge e^3 (e_2, e_1, e_3)
        = -e^2 \otimes e^1 \otimes e^3 (e_2, e_1, e_3)
        = -1.
    \end{math}
    Weiter gilt $\pi_{\wedge}(e^{k_1} \otimes e^{k_q}) = \frac{1}{q!} e^{k_1} \wedge \dotsb \wedge e^{k_q}$.
    Vergleich mit der Leibnizformen für die Determinante ergibt für $1$-Formen $\alpha_1, \dotsc, \alpha_q$.
    \begin{math}
        \pi_1(\alpha_1 \otimes \dotsb \otimes \alpha_q)
        &= \frac{1}{q!} \sum_{\sigma \in \Sym(q)} \sgn(\sigma) \alpha_1(v_{\sigma(1)}) \dotsb \alpha_q(v_{\sigma(q)}) \\
        &= \frac{1}{q!} \det( \alpha_j(v_k))_{1 \le j,k \le n}.
    \end{math}
\end{ex*}

\begin{df} \label{6.5}
    Sei $\alpha$ eine alternierende $p$-Form und $\beta$ eine alternierende $q$-Form auf $V$ (d.h. $\alpha \in {V^*}^{\otimes p}$, $\beta \in {V^*}^{\otimes q}$).
    Wir definieren das \emphdef{äußere Produkt} $\alpha \wedge \beta$ durch
    \begin{math}
        \alpha \wedge \beta
        &:= \frac{(p+q)!}{p!q!} \pi_\wedge( \alpha \otimes \beta) \circ \sigma \\
        &= \sum_{[\sigma] \in \Sym(p+q) / \Sym(p) \times \Sym(q)} \sgn(\sigma) (\alpha \otimes \beta) \circ \sigma.
    \end{math}
\end{df}





