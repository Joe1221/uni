\chapter{Differentialformen}

\begin{df} \label{6.1}
    Sei $V$ ein endlich-dimensionaler reeller Vektorraum und für $q \in \N_0$ sei $\omega \in {V^*}^{\otimes q}$ eine \emphdef{$q$-Form}, d.h. eine multilineare Abbildung $V^k \to \R$, $(v_1, \dotsc, v_q) \mapsto \omega(v_1, \dotsc, v_q)$.

    $\omega$ heißt \emphdef{alternierend}, falls $\forall 1 \le i,j \le q: v_i = v_j \implies \omega(v_1, \dotsc, v_q) = 0$.
\end{df}

\begin{ex*}
    Die Determinante reeller $n\times n$-Matrizen ist eine alternierende $n$-Form auf den Spalten (bzw. Zeilen) einer Matrix, $\det: (\R^n)^n \to \R$.
\end{ex*}

\begin{lem} \label{6.2}
    Sei $\omega \in {V^*}^{\otimes q}$ eine alternierende $q$-Form.
    Dann gilt für $i,j \in \Set{1, \dotsc, q}$, $i \neq j$, $\lambda \in \R$
    \begin{enumerate}[(i)]
        \item
            Scherung:
            \begin{math}
                \omega(v_1, \dotsc, v_{i-1}, v_i + \lambda v_j, v_{i+1}, \dotsc, v_q) = \omega(v_1, \dotsc, v_q)
            \end{math}
        \item
            Antikommutativität:
            \begin{math}
                \omega(v_1, \dotsc, v_j, \dotsc, v_i, \dotsc, v_q) = - \omega(v_1, \dotsc, v_i, \dotsc, v_j, \dotsc, v_q)
            \end{math}
        \item
            sind $v_1, \dotsc, v_q$ linear abhängig, dann gilt $\omega(v_1, \dotsc, v_q) = 0$.
    \end{enumerate}
    \begin{proof}
        leichte Übung (Linearität und alternierend).
    \end{proof}
\end{lem}


\Timestamp{2016-01-13}

\begin{df} \label{6.3}
    Die $q$-te \emphdef{äußere Potenz} von $V^*$ ist der Vektorraum
    \begin{math}
        \bigwedge^q V^* := \Set{\omega \in {V^*}^{\otimes q} & \text{$\omega$ ist alternierend} }.
    \end{math}
    \begin{note}
        \begin{itemize}
            \item
                Es gilt $\bigwedge^q V^* = 0$, falls $q > n := \dim(V)$, da dann $q$ Vektoren stets linear abhängig sind.
            \item
                Es gilt $\bigwedge^0 V^* = \R$, $\bigwedge^1 V^* = V^*$.
            \item
                Aus der linearen Algebra ist bekannt, dass $\bigwedge^n V^*$ (mit $n = \dim V$) eindimensional ist (Eindeutigkeit der Determinante).
                Somit wird der Raum $\bigwedge^n V^*$ von dem Element
                \begin{math}
                    \det: (v_1, \dotsc, v_n) \mapsto \det (e^j(v_k)_{1\le j,k \le n})
                \end{math}
                erzeugt, wobei $e^1, \dotsc, e^n$ eine Basis von $V^*$ bilden.
            \item
                Wir haben gesehen, dass alternierende $q$-Formen antikommutativ sind.
                Daraus folgt für eine Permutation $\sigma \in \Sym(q)$.
                \begin{math}
                    \omega(v_{\sigma(1)}, \dotsc, v_{\sigma(q)})
                    = \sgn(\omega) w(v_1, \dotsc, v_q).
                \end{math}
                Ist $\sigma \in \Sym(q)$ und $\omega \in {V^*}^{\otimes q}$, dann nutzen wir die Notation
                \begin{math}
                    (\omega \circ \sigma)(v_1, \dotsc, v_q)
                    := w(v_{\sigma(1)}, \dotsc, v_{\sigma(q)}).
                \end{math}
        \end{itemize}
    \end{note}
\end{df}

\begin{df} \label{6.4}
    Wir definieren die \emphdef{Antisymmetrisierung} von $q$-Formen als
    \begin{math}
        \pi_{\wedge}: {V^*}^{\otimes q} &\to \bigwedge^q V^* \\
        \omega &\mapsto \frac{1}{q!} \sum_{\sigma \in \Sym(q)}  \sgn(\sigma) \cdot \omega \circ \sigma.
    \end{math}
    Dies ist eine Vektorraum-Projektion, insbesondere $\pi_{\wedge}(\omega) = \omega$ für $\omega \in \bigwedge^q V^*$.

\end{df}

Sei nun $\omega \in \bigwedge^q V^*$ eine alternierende $q$-Form und seien $v_1, \dotsc, v_q \in V$
Sei $e_1, \dotsc, e_n$ eine Basis von $V$ und $e^1, \dotsc, e^n$ die dazu duale Basis von $V^*$, d.h. die Linearformen mit $e^j(e_k) := \delta_{jk}$.
Schreiben wir $v_j = \sum_{k=1}^n \lambda_{jk} e_k \in V$, dann gilt
\begin{math}
    \omega(v_1, \dotsc, v_q)
    &= \sum_{k_1, \dotsc, k_q = 1}^n \lambda_{1,k_1} \dotsb \lambda_{q,k_q} \omega(e_{k_1}, \dotsc, e_{k_q}) \\
    &= \sum_{\substack{k_1, \dotsc, k_q = 1 \\ k_i \neq k_j}}^n \lambda_{1,k_1} \dotsb \lambda_{q,k_q} \omega(e_{k_1}, \dotsc, e_{k_q}) \\
    &= \sum_{1 \le k_1 < \dotsc < k_q \le n} \sum_{\sigma \in \Sym(q)} \lambda_{1,k_{\sigma(1)}} \dotsb \lambda_{q, k_{\sigma(q)}} \omega(e_{k_{\sigma(1)}}, \dotsc, e_{k_{\sigma(q)}}) \\
    &= \sum_{1 \le k_1 < \dotsc < k_q \le n} \sum_{\sigma \in \Sym(q)} \lambda_{1,k_{\sigma(1)}} \dotsb \lambda_{q, k_{\sigma(q)}} \sgn(\sigma) \omega(e_{k_1}, \dotsc, e_{k_q}).
\end{math}
Dies zeigt: Es genügt, die Wert von $\omega$ auf $(e_{k_1}, \dotsc e_{k_q})$ mit aufsteigenden $1 \le k_1 < \dotsb < k_q \le n$ zu kennen.
Definiere nun für $k_1, \dotsc, k_q \in \Set{1, \dotsc, n}$ beliebig
\begin{math}
    e^{k_1} \wedge \dotsb \wedge e^{k_q}
    := \sum_{\sigma \in \Sym(q)} \sgn(\sigma) e^{k_{\sigma(1)}} \otimes \dotsb \otimes e^{k_{\sigma(q)}}.
\end{math}
Dies ist eine alternierende $q$-Form.

Seien nun für $1 \le k_1 < \dotsc < k_q \le n$ die Zahlen $w_{k_1, \dotsc, k_q}$ definiert als die Werte (für beliebiges $\omega$)
\begin{math}
    w_{k_1, \dotsc, k_q} := \omega(e_{k_1}, \dotsc, e_{k_q}).
\end{math}
Dann ist die $q$-Form $\omega$ wie oben gegeben durch
\begin{math}
    \omega := \sum_{1\le k_1 < \dotsc < k_q \le n} w_{k_1, \dotsc, k_q} e^{k_1} \wedge \dotsb \wedge e^{k_q}.
\end{math}
Dies zeigt:
\begin{math}
    \Set{e^{k_1} \wedge \dotsb \wedge e^{k_q} & 1 \le k_1 < \dotsb < k_q \le n}
\end{math}
bildet eine Basis von $\bigwedge^q V^*$ (die lineare Unabhängigkeit sieht man, indem man verschiedene $q$-Tupel von Basisvektoren $(e_{j_1}, \dotsc, e_{j_q})$ einsetzt), somit gilt $\dim(\bigwedge^q V^*) = \binom{n}{q}$.

\begin{ex*}
    \begin{itemize}
        \item
            $
                e^1 \wedge e^2 \wedge e^3 (e_2, e_1, e_3)
                = -e^2 \otimes e^1 \otimes e^3 (e_2, e_1, e_3)
                = -1.
            $
        \item
            $\pi_{\wedge}(e^{k_1} \otimes \dotsb \otimes e^{k_q}) = \frac{1}{q!} e^{k_1} \wedge \dotsb \wedge e^{k_q}$.
        \item
            Vergleiche mit der Leibnizformel für die Determinante:
            Für $1$-Formen $\alpha_1, \dotsc, \alpha_q$ ist
            \begin{math}
                &\pi_1(\alpha_1 \otimes \dotsb \otimes \alpha_q)(v_1, \dotsc, v_q) \\
                &\qquad= \frac{1}{q!} \sum_{\sigma \in \Sym(q)} \sgn(\sigma) \alpha_1(v_{\sigma(1)}) \dotsb \alpha_q(v_{\sigma(q)}) \\
                &\qquad= \frac{1}{q!} \det\big( \alpha_j(v_k)_{1 \le j,k \le n}\big).
            \end{math}
    \end{itemize}
\end{ex*}

\begin{df} \label{6.5}
    Seien $\alpha \in \bigwedge^p V^*$, $b \in \bigwedge^q V^*$ alternierende $p$-, bzw $q$-Formen.
    Wir definieren das \emphdef{äußere Produkt} $\alpha \wedge \beta$ durch
    \begin{math}
        \alpha \wedge \beta
        &:= \frac{(p+q)!}{p!q!} \pi_\wedge( \alpha \otimes \beta) \\
        &= \frac{1}{p!q!} \sum_{\sigma \in \Sym(p+q)} \sgn(\sigma) (\alpha \otimes \beta) \circ \sigma \\
        &= \sum_{[\sigma] \in \frac{\Sym(p+q)}{\Sym(p) \times \Sym(q)}} \sgn(\sigma) (\alpha \otimes \beta) \circ \sigma,
    \end{math}
    wobei $\sigma$ die eingesetzten Vektoren permutiert.
\end{df}

\Timestamp{2016-01-19}

\begin{ex*}
    Beispielsweise gilt für 1-Formen $\alpha, \beta$
    \begin{math}
        (\alpha \wedge \beta)(X,Y)
        &= \frac{(1+1)!}{1!1!} \pi_\wedge(\alpha \otimes \beta)(X,Y) \\
        &= 2 \cdot \frac{1}{2} ( \alpha(X)\beta(Y) - \beta(X) \alpha(Y)).
    \end{math}
\end{ex*}

\begin{st}[Eigenschaften von $\wedge$] \label{6.6}
    Das äußere Produkt ist assoziativ und \emphdef{superkommutativ} (oder \emphdef{graduiert kommutativ}), d.h. es gilt
    \begin{math}
        \alpha \wedge \beta = (-1)^{pq} \beta \wedge \alpha
    \end{math}
    für alternierende $p$-Formen $\alpha$ und alternierende $q$-Formen $\beta$.

    Der Vektorraum $\bigwedge^{\bullet} V^\ast := \bigoplus_{q=0}^n \bigwedge^q V^*$ wird damit zu einer assoziativen, graduierte Algebra (es gilt $\dim \bigwedge^\bullet V^\ast = 2^n = 2^{\dim V}$).
    \begin{proof}
        Wegen der Bilinearität genügt es, die Assoziativität auf den Basisvektoren nachzurechnen.
        Dazu prüft man, dass
        \begin{math}
            (e^{j_1} \wedge \dotsb \wedge e^{j_p}) \wedge (e^{k_1} \wedge \dotsb \wedge e^{k_q})
            = e^{j_1} \wedge \dotsb \wedge e^{j_p} \wedge e^{k_1} \wedge \dotsb \wedge e^{k_q}.
        \end{math}
        Dies überprüft man, indem man $(p+q)$-Tupel von Basisvektoren einsetzt.
        Die Formel $\alpha \wedge \beta = (-1)^{pq} \beta \wedge \alpha$ folgt, indem man sie für die Basisvektoren $\alpha = e^{j_1} \wedge \dotsb \wedge e^{j_r}$, $\beta = e^{k_1} \wedge \dotsb \wedge e^{k_q}$ überprüft.
        In der Tat benötigt man $pq$ Nachbarvertauschungen.
    \end{proof}
\end{st}

\begin{df} \label{6.7}
    Sei $M$ eine Mannigfaltigkeit und $n := \dim(M)$.
    Mit $\bigwedge^q \T^* M$ bezeichnen wir den Unterbündel der alternierenden $q$-Formen in ${\T^*}^{\otimes q} M$.

    Sei $\bigwedge^\bullet \T^* M := \bigoplus_{q=0}^n \bigwedge^q \T^* M$.
    Die Schnitte in diesem Bündel heißen \emphdef{Differentialformen} auf $M$.
    Die Menge der Schnitte bezeichnen wir mit $\Omega^\bullet(M) := \Gamma(M, \bigwedge^\bullet \T^* M)$, bzw. $\Omega^q(M) := \Gamma(M, \bigwedge^q \T^* M)$.
\end{df}

\begin{st} \label{6.8}
    Es existiert eine eindeutig bestimmte additive Abbildung $\d \in \End(\Omega^\bullet(M))$ mit folgenden Eigenschaften
    \begin{enumerate}[i)]
        \item
            $\d(\Omega^q(M)) \subset \Omega^{q+1}(M)$.
        \item
            $\d(\alpha \wedge \beta) = \d \alpha \wedge \beta + (-1)^p \alpha \wedge \d \beta$ für alle $\alpha \in \Omega^p(M)$, $\beta \in \Omega^\bullet(M)$ (\emphdef{Graduierte Produktregel})
        \item
            $\d: \Omega^0(M) = C^\infty(M) \to \Gamma(M, \T^* M) = \Omega^1(M)$ ist das übliche Differential auf Funktionen.
        \item
            $\d^2 = 0$.
    \end{enumerate}
    Diese Abbildung $\d$ heißt \emphdef{äußere Ableitung} oder \emphdef{de Rham-Operator}.
    \begin{proof}
\Timestamp{2016-01-20}
        Wie in \ref{3.11} zeigt man, dass $\d$ ein lokaler Operator ist.
        Sei $\phi: U \to V$ eine Karte von $M$.
        Dann ist für eine $q$-Form $\omega \in \Omega^q(M)$ die mit $\phi^{-1}$ auf $V$ zurückgeholte Differentialform gleich ${\phi^{-1}}^* \omega = \sum_{I} f_I \d x_I$ mit glatten Funktionen $f_I: V \to \R^n$.
        Somit ist $\omega$ auf $U$ eingeschränkt von der Gestalt
        \begin{math}
            \omega|_U := \sum_I (f_I \circ \phi) \d (x_{i_1} \circ \phi) \wedge \dotsb \wedge \d (x_{i_q} \circ \phi)
        \end{math}
        Sei nun $\d^U$ ein Operator auf $\Omega^\bullet(U)$, der die geforderten Eigenschaften (i)-(iv) hat.
        Dann folgt aus (ii) und (iv), dass
        \begin{math}
            \d^U(\omega|_U)
            = \sum_{I} \d(f \circ \phi) \wedge \d(x_{i_1} \circ \phi) \wedge \dotsb \wedge \d(x_{i_q} \circ \phi)
        \end{math}
        (wobei die $\d$ auf der rechten Seite Differentiale von Funktionen sind).
        Die zeigt, dass $\d^U$ eindeutig ist (und insbesondere, dass es höchstens ein $\d$ auf $M$ mit (i)-(iv) gibt.

        Der erste Faktor lässt sich nach Kettenregel so schreiben:
        \begin{math}
            \d^U(f_I \circ \phi)
            = \phi^* \d f_I
            = \phi^* \sum_{j} \pddx[x_j]{f_I} \d x_j.
        \end{math}
        Somit folgt
        \begin{math}
            {\d^U}^2(f_I \circ \phi)
            = \phi^* \d^2 f_I
            = \phi^* \sum_{j,k} \frac{\partial^2 f_I}{\partial x_k \partial x_j} \d x_k \wedge \d x_j
            = 0.
        \end{math}
        Dies zegit, dass $(d^U)^2 = 0$ gilt, woraus folgt, dass $\d^2 = 0$ ist (falls $\d$ existiert).
        Aus der Darstellung von $\d^U(\omega|_U)$ und (i) folgt, dass auch $\d(\Omega^q(M)) \subset \Omega^{q+1}(M)$ gilt (falls $d$ existiert).
        Wir haben gesehen, dass es in jeder Kartenumgebung $U$ einen eindeutig bestimmten Operator $\d^U$ gibt, der (i)-(iv) erfüllt.
        Für zwei Karten  $\phi: U \to V$, $\phi': U' \to V'$ folgt $\d^U|_{U \cap U'} = \d^{U'}|_{U \cap U'}$ und somit ist $\d$ auf ganz $M$ global definiert und eindeutig bestimmt.
    \end{proof}
\end{st}

Wir betrachten zunächst Differentialformen auf $\R^n$.
Wir schreiben dazu in diesem Zusammenhang $\d f$ für die übliche Ableitung einer glatten Funktion $f:\R^n \to \R$, $\d f(p)$ ist eine \emphdef{Linearform}, d.h. eine lineare Abbildung $\R^n \to \R$.
Die Linearform $\d x_i(p)$ bilden in jedem Punkt eine Basis von $\T_p^* \R^n = {\R^n}^*$, nämlich punktweise die Dualbasis zu dem kartesischen Vektorfeld $\pddx[x_j]$, d.h. es gilt
\begin{math}
    \d x_i (\pddx[x_j]) = \d x_i(e_j) = \delta_{ij}.
\end{math}
Es folgt, dass die Differentialformen
\begin{math}
    \Set{\d x_{i_1} \wedge \dotsb \wedge \d x_{i_q} & 1 \le i_1 < \dotsb < i_q \le n }
\end{math}
in jedem Punkt des $\R^n$ eine Basis der alternierenden $q$-Formen bilden.
Eine beliebige alternierende $q$-Form lässt sich schreiben als
\begin{math}
    \omega = \sum_{1 \le i_1 < \dotsb < i_q \le n} f_{i_1, \dotsc, i_q} \d x_{i_1} \wedge \dotsb \wedge \d x_{i_q}
\end{math}
mit Funktionen $f_{i_1, \dotsc, i_q}: \R^n \to \R$.
Zur Vereinfachung der Schreibweise verwendet man einen Multiindex $I = (i_1, \dotsc, i_q)$ mit $1 \le i_1 < \dotsb < i_q \le n$ und schreibt
\begin{math}
    \d x_I := \d x_{i_1} \wedge \dotsb \wedge \d x_{i_q}
\end{math}
und somit $\omega = \sum_I f_I \d x_I$.


\begin{st}[Eigenschaften von $\d$] \label{6.9}
    Für den de Rahm-Operator gilt
    \begin{enumerate}[(i)]
        \item
            $\d^2 = \d \circ \d = 0 \iff \im(\d) \subset \ker(\d)$.
        \item
            Ist $\phi: M \to N$ eine glatte Abbildung zwischen Mannigfaltigkeiten und $\omega \in \Omega^\bullet(N)$, dann gilt $\d(\phi^* \omega) = \phi^* \d \omega$.
        \item
            Für alle $X \in \scr X(M)$, $w \in \Omega^\bullet(M)$:
            \begin{math}
                L_X \d \omega = \d L_X \omega.
            \end{math}
    \end{enumerate}
    \begin{proof}
        \begin{enumerate}[i)]
            \item
                Siehe \ref{6.8}.
            \item
                Wie im Beweis von \ref{6.8} lässt sich jede Form als Linearkombinationen von Dachprodukten von 1-Formen schrebien, die wiederum lokal von der Form $f \cdot \d g$ sind.
                Für solche Formen gilt die Aussage nach Kettenregel:
                Sind $f, g \in C^\infty(N)$, so gilt $\phi^* f \d g = (f \circ \phi) \d(g \circ \phi)$.
                Also ist
                \begin{math}
                    \d(\phi^* (f \cdot \d g))
                    \stack{\text{6.8 (ii)}}= \d(f \circ g) \wedge \d(g \circ \phi)
                    = \phi^* \d f \wedge \d g.
                \end{math}
                Allgemein folgt die Formel aus $f^*(\alpha \wedge \beta) = (f^* \alpha) \wedge (f^* \beta)$ (was aus $f^*(\alpha \otimes \beta) = (f^* \alpha) \otimes (f^* \beta)$ folgt).
            \item
                Aus (ii) folgt ${\Phi^X_t}^* \d \omega = \d({\Phi_t^X}^* \omega)$, woraus die Behauptung durch Anwenden von $\pddx{t}|_{t=0}$ auf beiden Seiten folgt.
        \end{enumerate}
    \end{proof}
\end{st}

\begin{nt} \label{6.10}
    Der Einsetzungsoperator $\iota_X$ für ein Vektorfeld $X \in \scr X(M)$ hat ähnliche Eigenschaften wie $\d$ auf $\Omega^\bullet(M)$.
    Es gilt
    \begin{itemize}
        \item
            $\iota_X^2 = 0$
        \item
            $\iota_X(\alpha \wedge \beta) = (\iota_X \alpha) \wedge \beta + (-1)^p \alpha \wedge (\iota_X \beta)$.
    \end{itemize}
    siehe Übungsaufgabe 37
\end{nt}

\begin{st}[Homotopieformel von Élie Cartan] \label{6.11}
    Für jedes Vektorfeld $X$ und $\omega \in \Omega^\bullet(M)$ gilt
    \begin{math}
        L_X \omega
        = (d \circ \iota_X + \iota_X \circ d) \omega
        = (d + \iota_X)^2 \omega.
    \end{math}
    Mit anderen Worten: $L_X$ hat die zweite Wurzel $\d + \iota_X$.
    \begin{proof}
        Die zweite Gleichheit folgt aus $\d^2 = 0$, $\iota_X^2 = 0$.

        Mit $\d$ und $\iota_X$ ist auch $K_X := (d + \iota_X)^2$ ein lokaler Operator.
        \begin{enumerate}[a)]
            \item
                Auf Funktionen gilt
                \begin{math}
                    K_X f = \underbrace{\d(\iota_X f)}_{=0} + \iota_X(\d f)
                    = L_X f.
                \end{math}
            \item
                $K_X$ operiert als Derivation auf $\Omega^\bullet(M)$: für $\alpha \in \Omega›p(M)$, $\beta \in \Omega^\bullet(M)$ gilt
                \begin{math}
                    K_X(\alpha \wedge \beta)
                    &= (d + \iota_X) ((\d + \iota_X) \alpha \wedge \beta + (-1)^p \alpha \wedge (d + \iota_X) \beta) \\
                    &= K_X \alpha \wedge \beta + (-1)^{p \pm 1} (d + \iota_X) \alpha \wedge (d + \iota_X) \beta \\
                    &\qquad + (-1)^p ( d+ \iota_X) \alpha \wedge (d + \iota_X) \beta + (-1)^{2p} \alpha \wedge K_X \beta \\
                    &= K_Y \alpha \wedge \beta + \alpha \wedge K_X \beta.
                \end{math}
            \item
                Auf $1$-Formen gilt $K_X = L_X$, denn
                \begin{math}
                    K_X \d f
                    &= (d + \iota_X)(d + \iota_X) \d f
                    = (d + \iota_X) \iota_X \d f \\
                    &= (\d \iota_X \d f)
                    = \d L_X f
                    = L_X \d f
                \end{math}
                Induktiv folgt, dass $L_X$ und $K_X$ auf allen Differentialformen übereinstimmen.
        \end{enumerate}
    \end{proof}
\end{st}

\Timestamp{2016-01-26}

\begin{nt} \label{6.12}
    Für einen $q$-kontravarianten Tensor $\omega \in {V^*}^{\otimes q}$ (nicht notwendig alternierend) gilt für die Lieableitung
    \begin{math}
        L_X(\omega(Y_1, \dotsc, Y_q))
        = L_X(\omega)(Y_1, \dotsc, Y_q) + \sum_{k=1}^q \omega(Y_1, \dotsc, L_X Y_k, \dotsc, Y_q).
    \end{math}
    Dies folgt mit einer Rechnung analog zum Beweis von \ref{5.19}

    \ref{6.11} (Homotopieformel von Élie Cartan) ermöglicht die Berechnung von $L_X \omega$ mit Hilfe von $\d$.
    Umgekehrt lässt sich $\d$ auch aus Liebleitungen berechnen.
\end{nt}

\begin{st} \label{6.13}
    Seien $X_0, \dotsc, X_q \in \scr X(M)$ und $\omega \in \Omega^q(M)$.
    Dann gilt
    \begin{math}
        \d \omega(X_0, \dotsc, X_q)
        = \sum_{j=0}^q (-1)^j X_j( \omega(X_0, \dotsc, \hat X_j, \dotsc, X_q) )
        + \sum_{0 \le j < k \le q} (-1)^{j+k} \omega([X_j, X_k]), X_0, \dotsc, \hat X_j, \dotsc, \hat X_k, \dotsc, X_q)
    \end{math}
    \begin{proof}
        Nutze Induktion über $q$.
        Nach \ref{6.12} gilt
        \begin{math}
            L_{X_0}(\omega(Y_1, \dotsc, Y_q))
            &= \underbrace{(L_{X_0} \omega)}_{\stack{\text{\ref{6.11}}}= \d \iota_{X_0} \omega + \iota_{X_0} \d \omega}(X_1, \dotsc, X_q)
                + \sum_{j=1}^q \omega(X_1, \dotsc, [X_0, X_j], \dotsc, X_q) \\
            &= (\d \iota_{X_0} \omega)(x_1, \dotsc, X_q)
                + \d \omega(X_0, X_1, \dotsc, X_q) - \sum_{k=1}^q (-1)^k \omega([X_0, X_k], X_1, \dotsc, \hat X_k, \dotsc, X_q) \\
            &= \sum_{1 \le j < k \le q} (-1)^{j+k} \omega(X_0, [X_j, X_k], X_1, \dotsc, \hat X_j, \dotsc, \hat X_k, \dotsc, X_q)
                - \sum_{j=1}^a (-1)^j X_j . \omega(X_0, X_1, \dotsc, \hat X_j, \dotsc, X_q)
                + \d \omega(X_0, X_1, \dotsc, X_q) - \sum_{k=1}^q (-1)^k \omega([X_0, X_k], \hat X_0, X_1, \dotsc, \hat X_k, \dotsc, X_q) \\
        \end{math}
        Induktionsanfang $q = 0$:
        \begin{math}
            \d \omega(X_0) = X_0 . \omega,
        \end{math}
        denn $\d$ entspricht auf Funktionen dem gewöhnlichen Differential.
    \end{proof}
    \begin{note}
        Für $\alpha \in \Omega^1(M)$ folgt
        \begin{math}
            \d \alpha(X, Y)
            = X \alpha(Y) - X_1 \alpha(X) - \alpha([X, Y]).
        \end{math}
    \end{note}
\end{st}

Eine Anwendung ist der Satz von Frobenius.

Ein Unterbündel des Tangentialbündels vom Rang $k$ (d.h. eine glatte Auswahl eines $k$-dimensionalen Untervektorraums von $\T_p M$ in jedem Punkt $p \in M$) nennt man eine \emphdef{Distribution}.

Eine Distribution heißt \emphdef{integrabel}, falls es zuu jedem Punkt $p \in M$ eine $k$-dimensionale Untermannigfaltigkeit $N$ gibt, sodass der Tangentialraum von $N$ an jedem Punkt von $N$ mit der Distribution übereinstimmt.

Die Distribution heißt \emphdef{involutiv}, falls für alle Vektorfelder mit Werten in der Distribution (in den Unterbündeln) auch die Lie-Klammer je zwei solcher Vektorfelder wieder in der Distribution liegt.

Der Satz von Frobenius sagt, dass eine Distribution genau dann integrabel ist, wenn sie involutiv ist.
Insbesondere sind eindimensionale Distributionen, d.h. Unterbündel vom Rang 1 stets intgrabel.

\begin{st}[Frobenius] \label{6.14}
    Sei $H \subset \T M$ ein Untervektorbündel vom Rang $k$, so dass für $X, Y \in \Gamma(M, H)$ auch $[X, Y] \in \Gamma(M, H)$ gilt.

    Dann existiert zu jedem $p \in M$ eine $k$-dimensionale Untermannigfaltigkeit $N \subset M$ mit $p \in N$ und $\T N = H |_N := H \cap \pi^{-1}(N)$.
    \begin{proof}
        Sei $\phi$ eine Karte um $p \in M$.
        Dann bilden die Linearformen $\alpha_j := \phi^* \d x_j$, $1 \le j \le n = \dim(M)$ eine Basis von $\T_p M$.
        Also gibt es eine Teilmenge $I \subset \Set{1, \dotsc, n}$ für die $(\alpha_j|_{H_p})_{j \in I}$ eine Basis von $H_p^*$ ist.
        Sei eine offene Umgebung $U$ von $p$, sodass $(\alpha_j| H_p)_{j \in I}$ eine Basis von $H_p^*$ bleibt und seien $(X_j)_{j \in I}$ punktweise die dazu duale Basis von $H$.
        Wegen $\d \alpha_j = 0$ (denn $\d \alpha_j = \d(\phi^* \d x_j) = \phi^*(\d \d x_j) = 0$) gilt für $j, m, l \in I$.
        \begin{math}
            \alpha_j( [X_m, X_l])
            = X_m . \underbrace{\alpha_j(X_l)}_{\in \Set{0,1}} - X_l . \underbrace{\alpha_j(X_m)}_{\in \Set{0, 1}}
            = 0.
        \end{math}
        Da aber die $\alpha_j$ eine Basis von $H^*$ bilden, bedeutet dies wegen $[X_m, X_l] \in H$, dass $[X_m, X_l] = 0$ ist.
        Somit existiert $N \subset U$ nach \ref{3.30}.
    \end{proof}
\end{st}


\Timestamp{2016-01-27}


\section{De Rham-Kohomologie}

Mit Hilfe der Differentialformen und der außeren Ableitung $\d$ lässt sich jeder Mannigfaltigkeit eine graduierte Algebra $H^\bullet(M)$ zuordnen, die eiene differentialtopologische Invariante von $M$ ist.


\begin{df}
    Eien Differentialform $\omega \in \Omega^\bullet(M)$ heißt
    \begin{itemize}
        \item
            \emphdef{geschlossen}, falls $\d \omega = 0$, d.h. $\omega \in \ker \d$,
        \item
            \emphdef{exakt}, falls $\exists \alpha \in \Omega^\bullet(M): \omega = \d \alpha$, d.h. $\omega \in \im \d$.
    \end{itemize}
    Wegen $\d^2 = \d \circ \d = 0$ ist jede exakte Form geschlossen.

    Die \emphdef{De Rham-Kohomologie} einer Mannigfaltigkeit $M$ ist definiert durch
    \begin{math}
        H_{\dR}^q := \ker \d|_{\Omega^q(M)} / \im \d|_{\Omega^{q-1}(M)}.
    \end{math}
    $H^q(M)$ heißt \emphdef[Kohomologiegruppe]{$q$-te (de Rham)-Kohomologiegruppe} von $M$ (sezte $\Omega^q(M) = 0$ für $q < 0$.
    Man setzt
    \begin{math}
        H^\bullet(M) := \bigoplus_{q=0}^{\dim M} H^q(M).
    \end{math}
    Die Elemente von $H^\bullet(M)$ heißen auch \emphdef{Kohomologieklasssen} von $M$.
\end{df}

\begin{note}
    In Aufgabe 35 haben wir die $1$-Form auf $\R^2 \setminus \Set{0}$.
    \begin{math}
        \phi &= - \frac{y}{x^2 + y^2} \d x + \frac{x}{x^2 + y^2} \d y
    \end{math}
    betrachtet und gezeigt, dass sie geschlossen ist, d.h. $\d \phi = 0$, aber nicht exakt.
    Denn $\phi$ ist das Differential der Argumentfunktion auf $\R^2 \setminus \Set{0}$, lokal gegeben durch $\arctan(\frac{x}{y})$.
    Diese Funktion existiert nicht global auf ganz $\R^2 \setminus \Set{0}$, daher ist die Kohomologieklasse $[\phi] = \phi + \im(\d|_{\Omega^0(\R^2 \setminus \Set{0})})$ ein nicht-triviales Element von $H^1(\R^2 \setminus \Set{0})$.

    Die $1$-Form $\phi$ lässt sich nicht auf ganz $\R^2$ fortsetzen, denn in jeder sternförmigen offenen Teilmenge $U$ des $\R^n$ ist jede geschlossene $q$-Form für $q \ge 1$ exakt, d.h. $H^q(U) = 0$.
    Diese Aussage nennt man \emphdef{Poincaré-Lemma}.
    Es gilt also $0 \neq [\phi] \in H^1(\R^2 \setminus \Set{0})) \neq H^1(\R^2)$.
    Insbesondere folgt, dass $\R^2 \setminus \Set{0}$ und $\R^2$ nicht diffeomorph sind (Erinnerung: $\d F^* = F^* \d$)

    Für kompakte $M$ lässt mit harmonischer Analysis zeigen, dass $\dim H^q(M) < \infty$ und die \emphdef{Poincaré-Dualität} gilt, d.h.
    \begin{math}
        H^q(M) \isomorphic H^{\dim M - q}(M).
    \end{math}
\end{note}

\begin{lem} \label{6.16}
    Die De Rham-Kohomologise hat folgende Eigenschaften:
    \begin{enumerate}[(i)]
        \item
            $H^q(M) = 0$ für $q > \dim(M)$ oder $q < 0$.
        \item
            $H^0(M) \isomorphic \R$ kanonisch, falls $M$ zusammenhängend ist
        \item
            Das äußere Produkt auf $\Omega^\bullet(M)$ induziert eine Ring-Struktur, das sogenannte “cup”-Produkt $\cup: H^\bullet(M) \times H^\bullet(M) \to H^\bullet(M)$ eine superkommutative (d.h. graduiert-kommutative), $\Z$-graduierte $\R$-Algebra wird.
        \item
            Jede glatte Abbildung $\phi: M \to N$ induziert einen $\R$-Algebra-Homomorphismus
            \begin{math}
                \phi^*: H^\bullet(N) \to H^\bullet(M).
            \end{math}
    \end{enumerate}
    \begin{proof}
        \begin{enumerate}[(i)]
            \item
                klar, da $\Omega^q(M) = 0$ für $q > \dim(M)$.
            \item
                Ist $f \in \Omega›q(M)$ eine geschlossene Funktion, d.h. $\d f = 0$ dann ist $f$ lokal konstant, d.h.
                $H^0(M)$ ist ein reeller Vektorraum der Dimension $|\pi_0(M)|$.
            \item
                Zeige, $[\alpha] \cup [\beta] := [\alpha \wedge \beta]$ ist unabhängig von der Wahl der Repräsentanten $\alpha$, $\beta$.

                Seien $\gamma \in \Omega^{p-1}(M)$, $\epsilon \in \Omega^{q-1}(M)$, dann ist
                $(\alpha + \d \gamma) \wedge (\beta + \d \epsilon)$ geschlossen.

                Diese Form repräsentiert die selbe Kohomologieklasse $\alpha \wedge \beta$, d.h. $\alpha \wedge \d \epsilon + \d \gamma \wedge \beta + \d \gamma \wedge \d \epsilon$ ist exakt:
                \begin{math}
                    \alpha \wedge \d \epsilon &= \d( (-1)^p \alpha \wedge \epsilon) \\
                    \d \gamma \wedge \beta &= \d( \gamma \wedge \beta) \\
                    \d \gamma \wedge \d \epsilon &= \d ( \gamma \wedge \d \epsilon).
                \end{math}
                Dies zeigt die Wohldefiniertheit des $\cup$-Produkts.
                Die Superkommutativität folgt aus $\alpha \wedge \beta = (-1)^{pq} \beta \wedge \alpha$.
                Graduierte Algebra folgt aus $\alpha \wedge \beta \in \Omega^{p+q}(M)$.
            \item
                Das Zurückholen von Formen $\phi^*: \Omega^\bullet(N) \to \Omega^\bullet(M)$, $\alpha \mapsto \phi^* \alpha$ ist eine $\R$-lineare Abbildung, sie induziert eine wohldefinierte $\R$-lineare Abbildung $H^\bullet(N) \to H^\bullet(M)$ durch $\phi^*[\alpha] := [\phi^* \alpha]$.
                Dies ist wohldefiniert, da
                \begin{math}
                    [\phi^*(\alpha + \d \gamma]]
                    = [\phi^* \alpha + \phi^* \d \gamma]
                    = [\phi^* \alpha + \d \phi^* \gamma]
                    = [\phi^* \alpha].
                \end{math}
                Diese Abbildung ist ein $\R$-Algebra-Homomorphismus, denn
                \begin{math}
                    \phi^*([\alpha] \cup [\beta])
                    &= \phi^* [\alpha \wedge \beta]
                    = [\phi^*(\alpha \wedge \beta)] \\
                    &= [\phi^* \alpha] \cup [\phi^* \beta]
                    = (\phi^* [\alpha]) \cup (\phi^* [\beta]).
                \end{math}
        \end{enumerate}
    \end{proof}
\end{lem}






