\chapter{Der Tangentialraum}

\Timestamp{2015-10-28}

Wir haben nun den Begriff der Differenzierbarkeit definiert, wissen aber noch nicht, was die Ableitung (d.h. das Differential) einer Abbildung ist.

Das Differential (z.B. für Abbildungen zwischen $\R^n$ und $\R^m$) ist eine Annäherung einer Abbildung in der Nähe eines Punktes durch eine (affin-)lineare Abbildung.
Diese bildet einen Vektorraum in einen Vektorraum ab, dieser wird der Tangentialraum sein.

Da eine \emph{abstrakte} Mannigfaltigkeit zunächst nicht im Euklidischen Raum eingebettet ist, kann man nicht wie bei einer Fläche im $\R^3$ einfach die Geschwindigkeitsvektoren von Kurven durch einen Punkt betrachten, man benötigt Hilfskonstruktionen.


\section{Tangentialvektoren in Karten}


\begin{df} \label{3.1}
    Ein \emphdef{Repräsentant eines Tangentialvektors} in $p \in M$ ist ein Paar $(\phi, u) \in \scr A \times \R^m$, bestehend aus einer Karte $\phi$ um $p$ und einem Element $u \in \R^m$.
    
    Zwei Repräsentanten $(\phi, u)$, $(\psi, v)$ heißen \emphdef[äquivalent!Repräsentanten]{äquivalent}, falls
    \begin{math}
        D(\psi \circ \phi^{-1})|_{\phi(p)}(u) = v.
    \end{math}

    Die Äquivalenzklassen $[(\phi, u)]$ heißen \emphdef{Tangentialvektoren} am \emphdef{Fußpunkt} $p$, s bilden den Tangentialraum $\T_p M$
\end{df}

\begin{lem} \label{3.2}
    Mit den Rechenoperationen
    \begin{math}
        [(\phi, u_1)] + [(\phi, u_2)]
            &:= [(\phi, u_1 + u_2)],\\
        \lambda [(\phi, u)]
            &:= [(\phi, \lambda u)]
    \end{math}
    wird $\T_p M$ zu einem $m$-dimensionalen $\R$-Vektorraum.
    \begin{proof}
        Für jede andere Karte $\psi$ ist die Jacobimatrix $D(\psi \circ \phi^{-1})|_{\phi(p)}: \R^m \to \R^m$ ein linearer Isomorphismus, also sind die Addition und die skalare Multiplikation von der Wahl der Karte unabhängig.
        Durch die Wahl einer festen Karte $\phi$ um $p$ bildet $\Set{(\phi, v) & v \in \R^m}$ ein vollständiges Repräsentantensystem für $\T_p M$.
        Es folgt, dass $\T_p M$ zu $\R^m$ isomorph ist.
    \end{proof}
\end{lem}

\begin{df} \label{3.3}
    Seien $M$ und $N$ differenzierbare Mannigfaltigkeiten und $f: M \to N$ eine differenzierbare Abbildung.
    Dann ist das \emphdef{Differenzial} (auch \emphdef{Ableitung} oder \emphdef{Tangential}) von $f$ an $p \in M$ definiert als die lineare Abildung. 
    \begin{math}
        \T_p f: \T_p M &\to \T_{f(p)} N, \\
        [(\phi, u)] &\mapsto [(\psi, D(\psi \circ f \circ \phi^{-1})|_{\phi(p)}(u))]
    \end{math}
    Sei $f: M \to N$ eine glatte Abbildung.
    Für Tangentialbündel ist die Abbildung $\T f: \T M \to \T N$ durch $\T f(v) := \T_p f(v)$ für $v \in \T_p M$ erklärt.
    \begin{proof}
        Wohldefiniertheit:
        Sei $[(\phi,u)] = [\tilde \phi, D(\tilde \phi \circ \phi^{-1})|_{\phi(p)}(u)]$
        Dann ist im Urbild:
        \begin{math}
            (\T_p f) [(\tilde \phi, D(\tilde \phi \circ \phi^{-1})|_{\phi(p)} (u)]
            &= [(\psi, D(\psi \circ f \circ \tilde \phi^{-1})_{\tilde \phi(p)} D(\tilde \phi \circ \phi^{-1})|_{\phi(p)} (u)] \\
            &= [(\psi, D(\psi \circ f \circ \phi^{-1})|_{\phi(p)}(u))] \\
            &= (\T_p f)[(\phi, u)].
        \end{math}
        Im Bild:
        \begin{math}
            [(\psi, D(\psi \circ f \circ \phi^{-1})|_{\phi(p)})]
            &= [(\tilde \psi, D(\tilde \psi \circ \psi^{-1})|_{\psi(f(p)} D(\psi \circ f \circ \phi^{-1})_{\phi(p)}(u))] \\
            &= [(\tilde \psi, D(\tilde \psi \circ f \circ \phi^{-1})|_{\phi(p)} (u))].
        \end{math}
    \end{proof}
\end{df}

\begin{st}[Kettenregel] \label{3.4}
    Seien $M, N, P$ differenzierbare Mannigfaltigkeiten und $M \xto[g] N \xto[f] P$ differenzierbare Abbildungen.
    Dann gilt
    \begin{math}
        \T_p(f \circ g) = (\T_{g(p)} f) \circ (\T_p g)
    \end{math}
    für $p \in M$.
    \begin{proof}
        Wähle Karten $\phi: V \to U$ Karte von $M$ um $p$, $\psi: W \to X$ Karte von $N$ um $g(p)$ und $\tau: Y \to Z$ Karte von $P$ um $f(g(p))$.
        Es gilt
        \begin{math}
            (\T_{g(p)} f) \circ (\T_p g) [(\phi, u)]
            &= (\T_{g(p)} f) [\psi, D(\psi \circ g \circ \phi^{-1})|_{\phi(p)}(u))] \\
            &= [(\tau, D(\tau \circ f \circ \psi^{-1})|_{\psi(g(p))} \circ D(\psi \circ g \circ \phi^{-1})|_{\phi(p)}(u))] \\
            &= [(\tau, D(\tau \circ f \circ g \phi^{-1})|_{\phi(p)}(u))] \\
            &= \T_p(f\circ g) [(\phi, u)].
        \end{math}
    \end{proof}
\end{st}

\begin{df} \label{3.5}
    Sei $M$ eine Mannigfaltigkeit. 
    Wir bezeichnen die disjunkte Vereinigung der Tangentialräume
    \begin{math}
        \T M := \bigdotcup_{p\in M} \T_p M.
    \end{math}
    Sei $\pi: \T M \to M$ die Abbildung, die jeden Tangentialvektor seinen Fußpunkt zuordnet (d.h. $\pi(v) = p$ falls $v \in \T_p M$).
    Wir nennen diese disjunkte Vereinigung aller Tangentialräume an $M$ zusammen mit $\pi$ das \emphdef{Tangentialbündel} von $M$.
\end{df}

\begin{st} \label{3.6}
    Das Tangentialbündel $\T M$ einer Mannigfaltigkeit $M$ ist selbst eine ($2 \dim(M)$)-dimensionale Mannigfaltigkeit.
    %\begin{proof}[Skizze]
        % FIXME: Ordentlich mit Summentopologie, Pullback-Topologie auf \T M.

        %Für jede Karte $\phi: V \to U$ von $M$ kann man wie folgt eine von $\T M$ konstruieren.
        %Sei $\tilde V := \pi^{-1}(V)$ und sei $\tilde p = [(\phi, u)]$.
        %Dann setze $\tilde \phi(\tilde p) = (\pi(\tilde p), u)$.
        %Wir verwenden dann $\tilde \phi$ für jede Karte $\phi$ von $M$ und $\tilde V$, $\tilde U := U \times \R^m$ als Karten von $\T M$. 
    %\end{proof}
\end{st}


\Timestamp{2015-11-03}


\section{Derivationen und Vektorfelder}


\begin{df} \label{3.7}
    Ein Vektorfeld $X$ auf einer Mannigfaltigkeit $M$ ist eine glatte Abbildung $X: M \to \T M$, so dass $X(p) \in \T_p M$.
    Die Menge der Vektorfelder auf $M$ wird mit $\scr X(M)$ bezeichnet.
\end{df}

\begin{ex}
    Durch $p \mapsto e_i$ wird auf $\R^n$ das \emphdef[Vektorfeld!konstant]{konstante Vektorfeld} $e_i$ definiert.
    Man nennt dies auch \emphdef[Vektorfeld!kartesisch]{kartesisches Vektorfeld}.
    (bei $M = \R^n$ kann man $\T_p M = \R^n$ an jedem Punkt kanonisch identifizieren).
\end{ex}

\begin{df} \label{3.8}
    Ist $f: M \to N$ ein Diffeomorphismus von Mannigfaltigkeiten und $X \in \scr X(M)$ ein Vektorfeld auf $M$.
    Dann ist durch
    \begin{math}
        (f^* X)(f(p)) := \T_p f(X(p))
    \end{math}
    ein Vektorfeld auf $N$ definiert, der \emphdef{Pushforward} von $X$ unter $f$.
\end{df}

Eine weitere Charakterisierung des Tangentialbündels und von Tangentiavektoren ist durch \emphdef{Derivationen} (das sind Ableitungsoperatoren erster Ordnung) gegeben.

\begin{df} \label{3.9}
    Mit $C^\infty(M)$ bezeichnen wir die Menge der glatten Funktionen auf $M$.
    \begin{note}
        $C^\infty(M)$ ist ein $\R$-Vektorraum (mit punktweiser Addition und skalarer Multiplikation mit reellen Zahlen).
        Mit der punktweisen Multiplikation von Funktionen ist $C^\infty(M)$ sogar eine $\R$-Algebra.
    \end{note}
\end{df}

\begin{df} \label{3.10}
    Eine \emphdef{Derivation} auf $C^\infty(M)$ ist eine $\R$-lineare Abbildung $\delta: C^\infty(M) \to C^\infty(M)$, welche die Leibnitzregel erfüllt, d.h.
    \begin{math}
        \delta(fg) = \delta(f) g + f \delta(g)
    \end{math}
    für alle $f, g \in C^\infty(M)$.
    \begin{note}
        Für konstante Funktionen $f \equiv c$ gilt $\delta(f) = 0$, denn
        \begin{math}
            \delta(f) = \delta(c 1)
            = \delta(c)1 + c \delta(1)
            = \delta(c) + \delta(c)
            = 2 \delta(f),
        \end{math}
        also $\delta(f) = 0$.
    \end{note}
\end{df}

\begin{lem}[Derivationen sind lokale Operatoren] \label{3.11}
    Der Wert von $\delta(f)$ in eienem Punkt $p$ hängt nur vom Verhalten von $f$ auf einer (beliebig kleinen) Umgebung von $p$ ab. 

    Sei $f \in C^\infty(M)$, $\delta$ eine Derivation und $U \in M$ offen.
    Dann ist $\delta(f)|_{U}$ durch $f|_{U}$ eindeutig bestimmt.
    \begin{proof}
        Sei $p \in U$.
        Wir benutzen, dass es eine glatte Testfunktion $\tau$ gibt, mit $\tau(p) = 1$ und $\tau|_{M \setminus U} \equiv 0$.
        Es gilt für Funktionen $f, \tilde f \in C^\infty(M)$ mit $f|_U = \tilde f|_U$:
        \begin{math}
            0 &= (f - \tilde f) \cdot \tau \\
            \implies 0 & \delta((f - \tilde f) \cdot \tau)(p) \\
            &= \delta(f - \tilde f)(p) \cdot \underbrace{\tau(p)}_{=1} + \underbrace{(f - \tilde f)(p)}_{=0} \delta \tau(p),
        \end{math}
        Also folgt $\delta(f)(p) = \delta(\tilde f)(p)$.
    \end{proof}
\end{lem}

\begin{df} \label{3.12}
    Sei $M$ Mannigfaltigkeit und $v \in \T_p M$.
    Wir sagen, eine glatte Kurve $c: (-\epsilon, \epsilon) \to M$ repräsentiert den Tangentialvektor $v$, falls gilt
    \begin{math}
        c(p) = p
        \qquad\land\qquad
        \xxdot{\phi \circ c} = \ddx[t]|_{t=0}(\phi \circ c)(0) = 0,
    \end{math}
    wobei $v = [(\phi, u)]$.
    \begin{note}
        Indem man zwei solche Kurven $c, \tilde c$ als äquivalent ansieht, falls $\xxdot{\phi \circ c}(0) = \xxdot{\phi \circ \tilde c}(0)$,, erhält man eine weiter Beschreibung von
        \begin{math}
            \T_p M := \Set{[c] & c}.
        \end{math}
        Die Bijektion wird durch $[c] \mapsto [(\phi, \xxdot{\phi \circ c}(0))]$ hergestellt.
        In dieser Beschreibung des Tangentialraums hat das Differential $f: M \to N$ die elegante Beschreibung
        \begin{math}
            \T_p f([c]) = [f \circ c]
        \end{math}
        und die Kettenregel lässt sich wie folgt beweisen
        \begin{math}
            \T_{g(p)} f \circ \T_p g([c])
            = \T_{g(p)} f([g \circ c])
            = [f \circ g \circ c]
            = \T_p(f \circ g)([c]).
        \end{math}
    \end{note}
\end{df}

\begin{df} \label{3.13}
    Sei $X \in \scr X(M)$ ein Vektorfeld auf einer Mannigfaltigkeit $M$ und sei $f \in C^\infty(M)$ eine glatte Funktion $f: M \to \R$.
    Dann ist die Funktion $L_X f \in C^\infty(M)$, genannt \emphdef{Lieableitung} von $f$ nach $X$, definiert durch
    \begin{math}
        L_X f(p) := \ddx[t](f \circ c(t))|_{t = 0},
    \end{math}
    wobei $c: (-\epsilon, \epsilon) \to M$ eine Kurve ist, der den Tangentialvektor $X(p)$ repräsentiert, bzw. durch
    \begin{math}
        L_X f(p) := \Df (f \circ \phi^{-1})|_{\phi(p)}(u),
    \end{math}
    wobei $X(p) = [(\phi, u)]$.
    \begin{note}
        \begin{itemize}
            \item
                Wohldefiniertheit folgt aus der Kettenregel.
            \item
                Für die Lieableitung gibt es die Schreibweisen $X(f)$, $X.f$ und $Xf$.
                Beachte $Xf$ ist die Lieableitung von $f$ nach $X$, $fX$ ist das Vektorfeld $X$ punktweise multipliziert mit dem Wert von $f$.

        \end{itemize}
    \end{note}
\end{df}

\begin{lem} \label{3.14}
    Für jedes Vektorfeld $X \in \scr X(M)$ ist durch $f \mapsto L_x f$ eine Derivation gegeben.
    \begin{proof}
        Aus
        \begin{math}
            L_x(f+g)(p)
            = \Df (f \circ \phi^{-1})|_{\phi(p)}(u) + \Df(g\circ \phi^{-1})|_{\phi(p)}(u)
        \end{math}
        und $L_x(\lambda f)(p) = \lambda \Df (f \circ \phi^{-1})|_{\phi(p)}(u)$ folgt die $\R$-Linearität.

        Die Leibnitzregel folgt aus der gewöhnlichen Produktregel:
        Sei $F = f \circ c$, $G = g \circ c$, wobei $c: (-\epsilon, \epsilon) \to M$ den Vektor $X(p)$ repräsentiert.
        Dann gilt
        \begin{math}
            L_x(fg)(p)
            = \ddx[t] f(c(t))g(c(t)|_{t=0}
            &= \ddx[t] F(t) G(t)|_{t=0} \\
            &= F'(0)G(0) + F(0) G'(0)
            = (L_X f(p))g(p) + f(p) (L_X g(p)).
        \end{math}
    \end{proof}
\end{lem}

\begin{nt} \label{3.15}
    $\scr X(M)$ und $\Der(C^\infty(M))$ (die Menge der Derivationen von $C^\infty(M)$) bilden jeweils auf natürlicher Weise einen reellen Vektorraum und $L: X \mapsto L_X$ (beachte $L_X \in \Der(C^\infty(M))$) ist eine lineare Abbildung $\scr X(M) \to \Der(C^\infty(M))$.
\end{nt}


