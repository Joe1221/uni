\chapter{Der Tangentialraum}

\Timestamp{2015-10-28}

Wir haben nun den Begriff der Differenzierbarkeit definiert, wissen aber noch nicht, was die Ableitung (d.h. das Differential) einer Abbildung ist.

Das Differential (z.B. für Abbildungen zwischen $\R^n$ und $\R^m$) ist eine Annäherung einer Abbildung in der Nähe eines Punktes durch eine (affin-)lineare Abbildung zwischen Vektorräumen.
Diese Vektorräume stellen Tangentialräume dar.

Da eine \emph{abstrakte} Mannigfaltigkeit zunächst nicht im Euklidischen Raum eingebettet ist, kann man nicht wie bei einer Fläche im $\R^3$ einfach die Geschwindigkeitsvektoren von Kurven durch einen Punkt betrachten, man benötigt Hilfskonstruktionen.


\section{Tangentialvektoren in Karten}


\begin{df} \label{3.1}
    Sei $\phi$ eine Karte um $p \in M$ und $u \in \R^m$, dann heißt das Paar $(\phi, u)$ \emphdef{Repräsentant eines Tangentialvektors} in $p$.

    Zweis solche Repräsentanten $(\phi, u)$, $(\psi, v)$ heißen \emphdef[äquivalent!Repräsentant eines Tangentialvektors]{äquivalent}, falls
    \begin{math}
        D(\psi \circ \phi^{-1})|_{\phi(p)} u = v.
    \end{math}

    Die Äquivalenzklassen $[(\phi, u)]$ heißen \emphdef{Tangentialvektoren} am \emphdef{Fußpunkt} $p$.
    Die Menge aller Tangentialvektoren bilden den Tangentialraum $\T_p M$.
\end{df}

\begin{lem} \label{3.2}
    Mit den Rechenoperationen
    \begin{math}
        [(\phi, u_1)] + [(\phi, u_2)]
            &:= [(\phi, u_1 + u_2)],\\
        \lambda [(\phi, u)]
            &:= [(\phi, \lambda u)]
    \end{math}
    wird $\T_p M$ zu einem $m$-dimensionalen $\R$-Vektorraum.
    \begin{proof}
        Für jede andere Karte $\psi$ ist die Jacobimatrix $D(\psi \circ \phi^{-1})|_{\phi(p)}: \R^m \to \R^m$ ein \emph{linearer} Isomorphismus, also sind die Addition und die skalare Multiplikation von der Wahl der Karte unabhängig.
        Durch die Wahl einer festen Karte $\phi$ um $p$ bildet $\Set{(\phi, v) & v \in \R^m}$ ein vollständiges Repräsentantensystem für $\T_p M$.
        Folglich ist $\T_p M$ isomorph zu $\R^m$.
    \end{proof}
\end{lem}

\begin{df} \label{3.3}
    Seien $M$ und $N$ differenzierbare Mannigfaltigkeiten und $f: M \to N$ eine differenzierbare Abbildung.
    Dann ist das \emphdef{Differential} (auch \emphdef{Ableitung} oder \emphdef{Tangential}) von $f$ an $p \in M$ definiert als die lineare Abildung.
    \begin{math}
        \T_p f: \T_p M &\to \T_{f(p)} N, \\
        [(\phi, u)] &\mapsto [(\psi, D(\psi \circ f \circ \phi^{-1})|_{\phi(p)} u)]
    \end{math}
    Sei $f: M \to N$ eine glatte Abbildung.
    Auf Tangentialbündel (siehe \ref{3.5}) ist die Abbildung $\T f: \T M \to \T N$ durch $\T f(v) := \T_p f(v)$ für $v \in \T_p M$ definiert.
    \begin{proof}
        Wohldefiniertheit im Urbild:
        Sei $[(\phi,u)] = [(\tilde \phi, D(\tilde \phi \circ \phi^{-1})|_{\phi(p)} u)]$, dann ist
        \begin{math}
            (\T_p f) [(\tilde \phi, D(\tilde \phi \circ \phi^{-1})|_{\phi(p)} u)]
            &= [(\psi, D(\psi \circ f \circ \tilde \phi^{-1})_{\tilde \phi(p)} D(\tilde \phi \circ \phi^{-1})|_{\phi(p)} u)] \\
            &= [(\psi, D(\psi \circ f \circ \phi^{-1})|_{\phi(p)} u)] \\
            &= (\T_p f)[(\phi, u)].
        \end{math}
        Wohldefiniertheit im Bild:
        Sei $[(\psi,u)] = [(\tilde \psi, D(\tilde \psi \circ \psi^{-1})|_{\psi(p)} u)]$, dann ist
        \begin{math}
            [(\psi, D(\psi \circ f \circ \phi^{-1})|_{\phi(p)} u)]
            &= [(\tilde \psi, D(\tilde \psi \circ \psi^{-1})|_{\psi(f(p)} D(\psi \circ f \circ \phi^{-1})_{\phi(p)} u)] \\
            &= [(\tilde \psi, D(\tilde \psi \circ f \circ \phi^{-1})|_{\phi(p)} u)].
        \end{math}
    \end{proof}
\end{df}

\begin{st}[Kettenregel] \label{3.4}
    Seien $M, N, P$ differenzierbare Mannigfaltigkeiten und $M \xto{g} N \xto{f} P$ differenzierbare Abbildungen.
    Dann gilt
    \begin{math}
        \T_p(f \circ g) = (\T_{g(p)} f) \circ (\T_p g)
    \end{math}
    für $p \in M$.
    \begin{proof}
        Wähle Karten $\phi: V \to U$ von $M$ um $p$, $\psi: W \to X$ von $N$ um $g(p)$ und $\tau: Y \to Z$ von $P$ um $f(g(p))$.
        Es gilt
        \begin{math}
            (\T_{g(p)} f) \circ (\T_p g) [(\phi, u)]
            &= (\T_{g(p)} f) [\psi, D(\psi \circ g \circ \phi^{-1})|_{\phi(p)} u)] \\
            &= [(\tau, D(\tau \circ f \circ \psi^{-1})|_{\psi(g(p))} \circ D(\psi \circ g \circ \phi^{-1})|_{\phi(p)} u )] \\
            &= [(\tau, D(\tau \circ f \circ g \circ \phi^{-1})|_{\phi(p)} u)] \\
            &= \T_p(f\circ g) [(\phi, u)].
        \end{math}
    \end{proof}
\end{st}

\begin{df} \label{3.5}
    Sei $M$ eine Mannigfaltigkeit.
    Setze $\T M$ als die disjunkte Vereinigung der Tangentialräume:
    \begin{math}
        \T M := \bigdotcup_{p\in M} \T_p M.
    \end{math}
    Sei $\pi: \T M \to M$ die Abbildung, die jedem Tangentialvektor seinen Fußpunkt zuordnet (d.h. $\pi(v) = p$ falls $v \in \T_p M$).

    Wir nennen $\T M$ zusammen mit $\pi$ das \emphdef{Tangentialbündel} von $M$.
\end{df}

\begin{st} \label{3.6}
    Das Tangentialbündel $\T M$ einer Mannigfaltigkeit $M$ ist selbst eine $2m$-dimensionale Mannigfaltigkeit.
    %\begin{proof}[Skizze]
        % FIXME: Ordentlich mit Summentopologie, Pullback-Topologie auf \T M.

        %Für jede Karte $\phi: V \to U$ von $M$ kann man wie folgt eine von $\T M$ konstruieren.
        %Sei $\tilde V := \pi^{-1}(V)$ und sei $\tilde p = [(\phi, u)]$.
        %Dann setze $\tilde \phi(\tilde p) = (\pi(\tilde p), u)$.
        %Wir verwenden dann $\tilde \phi$ für jede Karte $\phi$ von $M$ und $\tilde V$, $\tilde U := U \times \R^m$ als Karten von $\T M$.
    %\end{proof}
\end{st}


\Timestamp{2015-11-03}


\section{Derivationen und Vektorfelder}


\begin{df} \label{3.7}
    Ein Vektorfeld $X$ auf einer Mannigfaltigkeit $M$ ist eine glatte Abbildung $X: M \to \T M$, so dass $X(p) \in \T_p M$.
    Die Menge der Vektorfelder auf $M$ wird mit $\scr X(M)$ bezeichnet.
\end{df}

\begin{ex*}
    Durch $p \mapsto e_i$ wird auf $\R^n$ das \emphdef[Vektorfeld!konstant]{konstante Vektorfeld} $e_i$ definiert.
    Man nennt dies auch \emphdef[Vektorfeld!kartesisch]{kartesisches Vektorfeld}.
    (bei $M = \R^n$ kann man $\T_p M = \R^n$ an jedem Punkt kanonisch identifizieren).
\end{ex*}

\begin{df} \label{3.8}
    Ist $f: M \to N$ ein Diffeomorphismus von Mannigfaltigkeiten und $X \in \scr X(M)$ ein Vektorfeld auf $M$.
    Dann ist durch
    \begin{math}
        (f_* X)(f(p)) := \T_p f(X(p))
    \end{math}
    ein Vektorfeld auf $N$ definiert, der \emphdef{Pushforward} von $X$ unter $f$.
\end{df}

Eine weitere Charakterisierung des Tangentialbündels und von Tangentiavektoren ist durch \emphdef{Derivationen} (das sind Ableitungsoperatoren erster Ordnung) gegeben.

\begin{df} \label{3.9}
    Mit $C^\infty(M)$ bezeichnen wir die Menge der glatten Funktionen auf $M$.
    \begin{note}
        $C^\infty(M)$ ist ein $\R$-Vektorraum (mit punktweiser Addition und skalarer Multiplikation mit reellen Zahlen).
        Mit der punktweisen Multiplikation von Funktionen ist $C^\infty(M)$ sogar eine $\R$-Algebra.
    \end{note}
\end{df}

\begin{df} \label{3.10}
    Eine \emphdef{Derivation} auf $C^\infty(M)$ ist eine $\R$-lineare Abbildung $\delta: C^\infty(M) \to C^\infty(M)$, welche die Leibnitzregel erfüllt, d.h.
    \begin{math}
        \delta(fg) = \delta(f) g + f \delta(g)
    \end{math}
    für alle $f, g \in C^\infty(M)$.
    \begin{note}
        Für konstante Funktionen $f \equiv c$ gilt $\delta(f) = 0$, denn
        \begin{math}
            \delta(f) = \delta(c 1)
            = \delta(c)1 + c \delta(1)
            = \delta(c) + \delta(c)
            = 2 \delta(f),
        \end{math}
        also $\delta(f) = 0$.
    \end{note}
\end{df}

\begin{lem}[Derivationen sind lokale Operatoren] \label{3.11}
    Der Wert von $\delta(f)$ in eienem Punkt $p$ hängt nur vom Verhalten von $f$ auf einer (beliebig kleinen) Umgebung von $p$ ab.

    Sei $f \in C^\infty(M)$, $\delta$ eine Derivation und $U \in M$ offen.
    Dann ist $\delta(f)|_{U}$ durch $f|_{U}$ eindeutig bestimmt.
    \begin{proof}
        Sei $p \in U$.
        Wir benutzen, dass es eine glatte Testfunktion $\tau$ gibt, mit $\tau(p) = 1$ und $\tau|_{M \setminus U} \equiv 0$.
        Es gilt für Funktionen $f, \tilde f \in C^\infty(M)$ mit $f|_U = \tilde f|_U$:
        \begin{math}
            0 &= (f - \tilde f) \cdot \tau \\
            \implies 0 & \delta((f - \tilde f) \cdot \tau)(p) \\
            &= \delta(f - \tilde f)(p) \cdot \underbrace{\tau(p)}_{=1} + \underbrace{(f - \tilde f)(p)}_{=0} \delta \tau(p),
        \end{math}
        Also folgt $\delta(f)(p) = \delta(\tilde f)(p)$.
    \end{proof}
\end{lem}

\begin{df} \label{3.12}
    Sei $M$ Mannigfaltigkeit und $v \in \T_p M$.
    Wir sagen, eine glatte Kurve $c: (-\epsilon, \epsilon) \to M$ repräsentiert den Tangentialvektor $v$, falls gilt
    \begin{math}
        c(0) = p
        \qquad\land\qquad
        \dotparen{\phi \circ c} = \ddx[t]|_{t=0}(\phi \circ c)(0) = 0,
    \end{math}
    wobei $v = [(\phi, u)]$.
    \begin{note}
        Indem man zwei solche Kurven $c, \tilde c$ als äquivalent ansieht, falls $\dotparen{\phi \circ c}(0) = \dotparen{\phi \circ \tilde c}(0)$,, erhält man eine weiter Beschreibung von
        \begin{math}
            \T_p M := \Set{[c] & c}.
        \end{math}
        Die Bijektion wird durch $[c] \mapsto [(\phi, \dotparen{\phi \circ c}(0))]$ hergestellt.
        In dieser Beschreibung des Tangentialraums hat das Differential $f: M \to N$ die elegante Beschreibung
        \begin{math}
            \T_p f([c]) = [f \circ c]
        \end{math}
        und die Kettenregel lässt sich wie folgt beweisen
        \begin{math}
            \T_{g(p)} f \circ \T_p g([c])
            = \T_{g(p)} f([g \circ c])
            = [f \circ g \circ c]
            = \T_p(f \circ g)([c]).
        \end{math}
    \end{note}
\end{df}

\begin{df} \label{3.13}
    Sei $X \in \scr X(M)$ ein Vektorfeld auf einer Mannigfaltigkeit $M$ und sei $f \in C^\infty(M)$ eine glatte Funktion $f: M \to \R$.
    Dann ist die Funktion $L_X f \in C^\infty(M)$, genannt \emphdef{Lie-Ableitung} von $f$ nach $X$, definiert durch
    \begin{math}
        L_X f(p) := \ddx[t](f \circ c(t))|_{t = 0},
    \end{math}
    wobei $c: (-\epsilon, \epsilon) \to M$ eine Kurve ist, der den Tangentialvektor $X(p)$ repräsentiert, bzw. durch
    \begin{math}
        L_X f(p) := \Df[(f \circ \phi^{-1})]|_{\phi(p)}(u),
    \end{math}
    wobei $X(p) = [(\phi, u)]$.
    \begin{note}
        \begin{itemize}
            \item
                Wohldefiniertheit folgt aus der Kettenregel.
            \item
                Für die Lieableitung gibt es die Schreibweisen $X(f)$, $X.f$ und $Xf$.
                Beachte $Xf$ ist die Lie-Ableitung von $f$ nach $X$, $fX$ ist das Vektorfeld $X$ punktweise multipliziert mit dem Wert von $f$.

        \end{itemize}
    \end{note}
\end{df}

\begin{lem} \label{3.14}
    Für jedes Vektorfeld $X \in \scr X(M)$ ist durch $f \mapsto L_x f$ eine Derivation gegeben.
    \begin{proof}
        Aus
        \begin{math}
            L_x(f+g)(p)
            = \Df (f \circ \phi^{-1})|_{\phi(p)}(u) + \Df(g\circ \phi^{-1})|_{\phi(p)}(u)
        \end{math}
        und $L_x(\lambda f)(p) = \lambda \Df (f \circ \phi^{-1})|_{\phi(p)}(u)$ folgt die $\R$-Linearität.

        Die Leibnitzregel folgt aus der gewöhnlichen Produktregel:
        Sei $F = f \circ c$, $G = g \circ c$, wobei $c: (-\epsilon, \epsilon) \to M$ den Vektor $X(p)$ repräsentiert.
        Dann gilt
        \begin{math}
            L_x(fg)(p)
            &= \ddx[t] f(c(t))g(c(t)|_{t=0} \\
            &= \ddx[t] F(t) G(t)|_{t=0} \\
            &= F'(0)G(0) + F(0) G'(0) \\
            &= (L_X f(p))g(p) + f(p) (L_X g(p)).
        \end{math}
    \end{proof}
\end{lem}

\begin{nt} \label{3.15}
    $\scr X(M)$ und $\Der(C^\infty(M))$ (die Menge der Derivationen von $C^\infty(M)$) bilden jeweils auf natürlicher Weise einen reellen Vektorraum und $L: X \mapsto L_X$ (beachte $L_X \in \Der(C^\infty(M))$) ist eine lineare Abbildung $\scr X(M) \to \Der(C^\infty(M))$.
\end{nt}


\Timestamp{2015-11-04}

Verschiedene Möglichkeiten Tangentialvektoren zu definieren:
\begin{enumerate}[1.]
    \item
        In Karten:
        \begin{math}
            \T_p M = \Set{[(\phi, u)] & \text{$\phi$ Karte um $p$, $u \in \R^m$}}.
        \end{math}
    \item
        Äquivalenzklassen von Kurven
        \begin{math}
            M = \Set{[c] & \text{$c :(-\epsilon,\epsilon) \to M$ glatt, $c(0) = p$}},
        \end{math}
        wobei $[c] \sim [\tilde c]$, wenn $\dotparen{\phi \circ c}(0) = \dotparen{\phi \circ \tilde c}(0)$.
\end{enumerate}

Lie-Ableitung:
\begin{math}
    L_X f(p) = X(f)(p)
    &:= \ddx[t] (f \circ c)(t)|_{t=0} \\
    &= \Df[(f \circ \phi^{-1})]|_{\phi(p)} (u),
\end{math}
wobei $c$ repräsentiert $X(p)$, d.h. $c: (-\epsilon, \epsilon) \to M$ glatt, $c(0) = p$, $[(\phi, \dotparen{\phi \circ c}(0)] = X(p)$.
Außerdem $[(\phi, u)] = X(p)$.

Derivationen:
$\R$-lineare Abbildung $\delta: C^\infty(M) \to C^\infty(M)$ mit $\delta(fg) = \delta(f)g + f \delta(g)$.
Jedes Vektorfeld liefert eine Derivation, für $X \in \scr X(M)$ ist nämlich $f \mapsto L_X f$ eine Derivation.

\begin{note}
    \begin{itemize}
        \item
            $\scr X(M)$ ist ein $\R$-Vektorraum
        \item
            $\Der(C^\infty(M))$ ist ein $\R$-Vektorraum
            \begin{math}
                (\delta + \eps)(f) &:= \delta(f) + \eps(f), \\
                (\lambda \delta)(f) &:= \lambda \delta(f)
            \end{math}
        \item
            $\scr X(M) \to \Der(C^\infty(M))$, $X \mapsto L_X(\argdot)$ ist eine lineare Abbildung, sogar ein Isomorphismus (siehe \ref{3.16}).
    \end{itemize}
\end{note}

\begin{st} \label{3.16}
    $\scr X(M) \to \Der(C^\infty(M))$, $X \mapsto L_X(\argdot)$ ist ein Isomorphismus von $\R$-Vektorräumen.
    \begin{proof}
        \begin{seg}{Injektivität}
            Angenommen $L_X f \equiv 0$ für alle $f \in C^\infty(M)$ für ein nicht-verschwindendes Vektorfeld $X \in \scr X(M)$, also etwa $X(p) \neq 0$ für ein $p \in M$.
            Wähle eien Karte $\phi: U \to V$ um $p$ und $f: V \to \R$ glatt, so dass die Richtungsableitung $\Df f|_{\phi(p)} u$ nicht verschwindet mit $[(\phi, u)] = X(p)$.
            Wähle um $\phi(p)$ eine glatte Funktion $\tau$, die auf einer Umgebung von $\phi(p)$ konstant $1$ ist und $\supp \tau \subset U$.
            Dann ist durch
            \begin{math}
                \tilde f(q) := \begin{cases}
                    \tau(\phi(q)) f(\phi(q)) & \text{für $q \in U$} \\
                    0 & \text{sonst}
                \end{cases}
            \end{math}
            eine glatte Funktion auf $M$ definert, so dass $L_X f(p) \neq 0$, ein Widerspruch zur Annahme.
        \end{seg}
        \begin{seg}{Surjektivität}
            Sei $\delta$ eine Derivation von $C^\infty(M)$.
            Wir zeigen, dass es $X \in \scr X(M)$ gibt mit $L_X f = \delta(f)$ für alle $f \in C^\infty(M)$.
            Sei $f \in C^\infty(M)$, $p \in M$, $\phi: U \to V$ Karte mit $\phi(p) = 0$ und $g := f \circ \phi^{-1}$.

            In einer sternförmigen Umgebung von $0 = \phi(p)$ in $V$ gilt
            \begin{math}
                g(x) &= g(0) + \int_0^1 \ddx[t] g(tx) \di[t] \\
                &= g(0) + \sum_{i=0}^m x_i \int_0^1 \ddx[x_i]{g}(tx) \di[t]
            \end{math}
            Also gilt (nach \ref{3.11})
            \begin{math}
                \delta(f)(p)
                &= \delta(g \circ p)(p) \\
                &=  \delta(g \circ \phi)(p) + \sum \delta(x_i \circ \phi) \int_0^1 \ddx[x_i]{g}(tx) \di[t]|_{x=0} \\
                &\qquad + \sum (x_i \circ \phi)(p) \delta \Big(\int_0^1 \ddx[x_i]{g}(tx) \di[t] \circ \phi\Big)(p) \\
                &= \sum \delta(x_i \circ \phi)(p) \ddx[x_i]{(f\circ \phi^{-1})}(0) \\
                &= L_X f(p),
            \end{math}
            wobei
            \begin{math}
                X(p) = \l[ \l(\phi, \Vector{\delta(x_1 \circ \phi)(p) & \dots & \delta(x_m \circ \phi)(p)} \r) \r].
            \end{math}
            Die Unabhängigkeit von der Wahl der Karte folgt aus der Injektivität, denn danach müssen zwei Vektorfelder, die auf der Schnittmenge zweier Kartenumgebungen die selbe Derivation definieren, übereinstimmen.
        \end{seg}
        \begin{note}
            Ist $M = U$ eine offene Teilmenge des $\R^m$, dann kann $\T_p U$ mit $\R^m$ kanonisch identifiziert werden.
            Das Vektorfeld, das an jedem Punkt durch $[(\phi, e_i)]$ gegeben ist nennt man \emphdef{kartesisches Vektorfeld}, oder \emphdef{Koordinatenvektorfeld} und man schreibt dafür auch $\ddx[x_i]$, denn die $i$-te partielle Ableitung (als Operator) ist gerade die Derivation von $C^\infty(U)$, die davon erzeugt wird, z.B.
            \begin{math}
                L_{\ddx[x_i]} f(p)
                &= \ddx[t]f(p + te_i)|_{t=0} \\
                &= \ddx[x_i] f(p)
                = \ddx[x_i]{f}(p).
            \end{math}
            Man schreibt auch $\ddx[x]$, $\ddx[y]$, $\ddx[z]$.
            Ist $\phi: U \to V$ eine Karte einer Mannigfaltigkeit $M$, dann sind $\phi^i$, $i = 1, \dotsc, m$ die Koordinatenfunktionen von $\phi$.
            Für die davon erzeugten Vektorfelder schreibt man auch $\ddx[\phi^i]$ (häufiger $\ddx[x^i]$ für Karte $x: U \to V$).
            Sie heißen \emphdef{Koordinatenvektorfelder} (auf $U$), sie sind die Pushforwards der kartesischen Vektorfelder $\ddx[x_i]$.

            (Im Beweis von \ref{3.16} wurde gezeigt, dass für
            \begin{math}
                X = \sum_{i=1}^m \delta \phi^i \ddx[\phi^i]
            \end{math}
            gilt $L_X(\argdot) = \delta$.
        \end{note}
    \end{proof}
\end{st}


\Timestamp{2015-11-10}

Wir haben gesehen, dass Vektorfelder als Derivationen gesehen werden können und umgekehrt.
Für ein Vektorfeld $X \in \scr X(M)$ ist $f \mapsto L_x f = Xf = X(f)$ eine Derivation.

Die Verkettung von zwei Derivationen $\delta \circ \eps$ ist im Allgemeinen keine Derivation mehr (sondern ein Ableitungsoperator zweiter Ordnung).
Aber es gilt

\begin{lem} \label{3.17}
    Der \emphdef{Kommutator} $\gamma \circ \delta - \delta \circ \gamma$ zweier Derivationen $\delta$, $\gamma$ von $C^\infty(M)$ ist wieder eine Derivation.
    \begin{proof}
        Linearität ist klar, wir prüfen die Produktregel:
        \begin{math}
            (\gamma \circ \delta - \delta \circ \gamma)(fg)
            &= \gamma(\delta(f) g + f \delta(g)) - \delta(\gamma(f)g + f \gamma(g)) \\
            &= (\gamma \circ \delta)(f) g + \delta(f) \gamma(g) + \gamma(f) \delta(g) + f (\gamma \circ \delta)(g) \\
            &\qquad - (\delta \circ \gamma)(f) g - \gamma(f) \delta(g) - \delta(f) \gamma(g) - f (\delta \circ \gamma)(g) \\
            &= (\gamma \circ \delta - \delta \circ \gamma)(f)g + f(\gamma \circ \delta - \delta \circ \gamma)(g)
        \end{math}
    \end{proof}
\end{lem}

Für zwei Vektorfelder $X, Y \in \scr X(M)$ muss es also ein Vektorfeld $Z \in \scr X(M)$ geben, sodass
\begin{math}
    L_X(L_Y f) - L_Y(L_X f) = L_Z f.
\end{math}

\begin{df} \label{3.18}
    Seien $X, Y \in \scr X(M)$.
    Dann ist $[X, Y]$, genannt \emphdef{Lie-Klammer} von $X$ und $Y$, das Vektorfeld auf $M$, sodass gilt
    \begin{math}
        [X, Y](f) = X(Y(f)) - Y(X(f)).
    \end{math}
    \begin{note}
        Nach \ref{3.16} und \ref{3.17} ist die Lie-Klammer wohldefiniert.
    \end{note}
\end{df}

Mit dieser Klammer-Verknüpfung ($\R$-linear in beiden Argumenten) wird der $\R$-Vektorraum $\scr X(M)$ zu einer Algebra.
Diese Verknüpfung ist weder kommutativ (sondern $[Y, X] = -[X, Y]$), noch assoziativ (was ein guter Grund ist, die Klammer zu einem Bestandteil der Notation zu machen).
Es gilt jedoch

\begin{st} \label{3.19}
    Die Lie-Klammer auf $\scr X(M)$ erfüllt die \emphdef{Jacobi-Identität}
    \begin{math}
        [X, [Y, Z]] + [Y, [Z, X]] + [Z, [X, Y]] = 0.
    \end{math}
    \begin{proof}
        Man rechne
        \begin{math}
            [\alpha, [\beta, \gamma]]
            &= [\alpha, \beta \circ \gamma - \gamma \circ \beta] \\
            &= \alpha(\beta \circ \gamma - \gamma \circ \beta) - (\beta \circ \gamma - \gamma \circ \beta) \circ a
        \end{math}
        Damit folgt (durch zyklisches Vertauschen $\alpha \mapsto \beta \mapsto \gamma \mapsto \alpha$).
        \begin{math}
            [\alpha, [\beta, \gamma]] + [\beta, [\gamma, \alpha]] + [\gamma, [\alpha, \beta]]
            = 0.
        \end{math}
    \end{proof}
\end{st}

\begin{note}
    Für einen Körper $\K$ heißt ein $\K$-Vektorraum $A$ mit einer bilinearen Verknüpfung $A \times A \to A$ auch $\K$-Algebra.
    Ist die Verknüpfung antisymmetrisch und erfüllt die Jacobi-Identität, so spricht man von einer Lie-Algebra.

    $\scr X(M)$ ist also eine Lie-Algebra.
\end{note}


% 3.3
\section{Flüsse}


Wir wollen noch eine dritte Möglichkeit kennenlernen, wie man Vektorfelder verstehen kann: als infinitesimale Transformationen.

\begin{df} \label{3.20}
    Sei $I \subset \R$ ein offenes Intervall, $c: I \to M$ eine glatte Abbildung (\emphdef{Kurve}).
    Sei $X \in \scr X(M)$ ein Vektorfeld und $\dot c(t) := \ddx[t] c(t) := (T_t c)(e_1) = (T_t c)(\ddx[t])$ der Geschwindigkeitsvektor von $c$ bei $t$.
    Falls
    \begin{math}
        \dot c(t) = X(c(t))
    \end{math}
    für alle $t \in I$ gilt, dann heißt $c$ \emphdef{Integralkurve} von $X$.
\end{df}

Integralkurven sind die Lösunggen einer gewöhnlichen autonomen Differentialgleichung erster ordnung mit Werten auf einer Mannigfaltigkeit.

\begin{st} \label{3.21}
    Sei $M$ eine Mannigfaltigkeit, $X \in \scr X(M)$ und $p \in M$.
    Dann gibt es ein offenes Intervall $I$ um die Null und eine Integralkurve $c$ von $X$ mit $c(0) = p$, eindeutig in dem Sinne, dass jede andere Integralkurve, die bei $0$ den Wert $p$ annimmt mit $c$ übereinstimmt.
    \begin{proof}
        Mit Hilf einer Karte $\phi$ um $p$ lasst sich die Differentialgleichung in den $\R^m$ übertragen.
        Sei $\phi: U \to V$ so eine Karte und sei $F: V \to \R^m$ glatt, so dass $[(\phi, F(x))]$ für jedes $x \in V$ den Tangentialvektor $X(\phi^{-1}(x))$ repräsentiert (mit anderen Worten $F = \phi_* X$, Pushforward).

        Die Aussage folgt aus den üblichen Existenz- und Eindeutigkeitsaussagen für gewöhnliche Differentialgleichungen erster Ordnung im $\R^m$.
        Ist nämlich $y: I \to V$ eine Lösung des Anfangswertproblems
        \begin{math}
            \dot y(t) = F(y(t)), && y(0) = \phi(p),
        \end{math}
        dann ist $\phi^{-1} \circ y$ die gesuchte Integralkurve (und umgekehrt).
    \end{proof}
\end{st}

Wie bei Anfangswertproblemen für (Systeme von) gewöhnlichen Differentialgleichungen in $\R^n$ gibt es auch hier für das Anfangswertproblem $\dot c(t) = X(c(t))$, $c(0) = p$ auf der Mannigfaltigkeit eine \emphdef{maximales Definitionsintervall}, das im Allgemeinen nicht das ganze Intervall ist, da die Lösungskurve den Definitionsbereich in endlicher Zeit „verlassen“ kann.
Beispielsweise ist die Lösung $y(t) = -\frac{1}{t}$ von $y'(t) = \frac{1}{y(t)^2}$ nicht auf ganz $\R$ fortsetzbar.

\begin{df}[Fluss eines Vektorfeldes] \label{3.22}
    Sei $X \in \scr X(M)$.
    Sei $\Phi^X(p): I \to M$ die auf das maximale Definitionsintervall fortgesetzte Integralkurve von $X$ mit $p \in M$, d.h. es gilt für $\Phi_t^X(p) := (\Phi^X(p))(t)$, dass
    \begin{math}
        \ddx[t] \Phi_t^X(p) &= X( \Phi_t^X(p)), \\
        \Phi_0^X(p) &= p.
    \end{math}
    Sei $\Phi^X$ die auf einer offenen Umgebung von $\Set 0 \times M$ definierten Abbildung, gegeben durch $\Phi^X(t,p) := \Phi_t^X(p)$, diese heißt \emphdef{Fluss von $X$}.
\end{df}


\Timestamp{2015-11-11}


\begin{st} \label{3.23}
    Für den Fluss $\Phi^X$ eines Vektorfeldes $X$ gilt
    \begin{enumerate}[i)]
        \item
            $\Phi^X$ ist eine glatte Abbildung,
        \item
            $\Phi^X_0 = \Id_M$,
        \item
            $\Phi^X_s \circ \Phi^X_t = \Phi_{s+t}^X$
        \item
            $\ddx[t] \Phi_t^X(p)|_{t=0} = X(p)$
    \end{enumerate}
    \begin{proof}
        (ii) und (iv) folgen direkt aus der Definition des Flusses.
        (iii) folgt aus der Eindeutigkeit von Integralkurven, denn setze $c(t) := \Phi_t^X \circ \Phi_s^X(p)$, $\tilde c(t) := \Phi_{t+s}^X(p)$ für ein $p \in M$.
        Dann gilt
        \begin{math}
            \dot c(t) &= \ddx[t] \Phi_t^X(\Phi_s^X(p)) = X(\Phi_t^X(\Phi_s^X(p))) \\
            \dot{\tilde c}(t) &= \ddx[t] \Phi_{t+s}^X(p) = X( \Phi_{t+s}^X(p) ),
        \end{math}
        außerdem gilt $c(0) = \Phi_s^X(p) = \tilde c(0$, also lösen $c$ und $\tilde c$ das selbe Anfangswertproblem und stimmen somit überein.
        (i) folgt aus glatter Abhängigkeit von Anfangswerten.
    \end{proof}
\end{st}

\begin{st} \label{3.24}
    Hat ein Vektorfeld $X \in \scr X(M)$ kompakten Träger, dann ist $\Phi^X$ auf ganz $\R \times M$ definiert.
    Dies gilt insbesondere falls $M$ kompakt ist.
    \begin{proof}
        Siehe Köhler, DG und homogene Räume, Satz 1.5.3.
    \end{proof}
\end{st}

\begin{nt*}
    Liegt $\Set{\pm t} \times M$ im Definitionsbereich von $\Phi^X$, so gilt nach \ref{3.23} (iii) und (ii)
    \begin{math}
        \Phi_t^X \circ \Phi_{-t}^X = \Phi_0^X = \Id_M.
    \end{math}
    Also ist $\Phi^X_t$ ein Diffeomorphismus.
    Allgemein ist $\Phi^X_t$ stets lokaler Diffeomorphismus um $\Set{0} \times M$.
    Frü kompakte $M$ gilt nach \ref{3.23} und \ref{3.24}, dass durch die Abbildung
    \begin{math}
        (\R, +) &\to \Diff(M), \\
        t &\mapsto \Phi^X_t
    \end{math}
    ein Gruppenhomomorphismus in die Gruppe der Diffeomorphismen $M \to M$ (mit der Verkettung als Verknüpfung) definiert ist
    $\Diff(M)$ wird auch \emphdef{Einparametergruppe von Diffeomorphismen} genannt.
\end{nt*}

\begin{lem} \label{3.25}
    Sei $f: M \to N$ ein Diffeomorphismus.
    Dann hat das Vektorfeld $f_*X$ (Pushforward) aus $N$ den Fluss $f \circ \Phi_t^X \circ f^{-1}$, also:
    \begin{math}
        \Phi^{f_* X}_t = f \circ \Phi^X_t \circ f^{-1}.
    \end{math}
    \begin{proof}
        Beide Flüsse lösen das selbe Anfangswertproblem:
        \begin{math}
            f \circ \Phi_0^X \circ f^{-1} = f \circ \Id_M \circ f^{-1} = \Id_M = \Phi_0^{f_* X}
        \end{math}
        und für die Differentialgleichung:
        \begin{math}
            \ddx[t](f \circ \Phi_t^X \circ f^{-1})(f(p))|_{t=0}
            &= T_{p} f \ddx[t] \Phi_t^X(p) |_{t=0} )
            = T_{p} f(X(p))
            = (f_* X)(f(p))
        \end{math}
    \end{proof}
\end{lem}

\begin{kor} \label{3.26}
    Sei $f: M \to M$ ein Diffeomorphismus und sei $X \in \scr X(M)$ ein unter $f$ invariantes Vektorfeld, d.h. $f_* X = X$.
    Dannn gilt
    \begin{math}
        \Phi^X_t \circ f = f \circ \Phi^X_t.
    \end{math}
\end{kor}

Mit Hilfe des Flusses lässt sich die Lie-Ableitung $L_X f$ einer Funktion $f$ nach einem Vektorfeld $X$ auf eine weitere Weise schreiben, nämlich
\begin{math}
    X(p)f = Xf(p) = \ddx[t] f(\Phi_t^X(p))|_{t=0},
\end{math}
denn $t \mapsto \Phi_t^X(p)$ repräsentiert den Tangentialvektor $X(p)$.


Die Lie-Ableitung eines Vektorfelds misst „die Änderung“ eines Vektorfelds entlang einer Kurve.

\begin{df} \label{3.27}
    Seien $X, Y \in \scr X(M)$.
    Die Lie-Ableitung von $Y$ nach $X$ bei $p \in M$ ist definiert durch
    \begin{math}
        L_X Y(p) := \ddx[t] {\Phi_{-t}^X}_*Y(p)|_{t = 0}
    \end{math}
    $L_X Y$ ist also ein Vektorfeld auf $M$.
\end{df}

Es gilt
\begin{math}
    L_X Y(p) &= \lim_{t \to 0} \frac{{\Phi_{-t}^X}_* Y(p) - Y(p)}{t} \\
    &= \lim_{t\to 0} \frac{T_{\Phi_t^X(p)} \Phi_{-t}^X (Y(\Phi_t^X(p))) - Y(p)}{t}
\end{math}

\begin{st} \label{3.28}
    Es gilt $L_X Y = [X, Y]$.
    \begin{proof}
        Siehe P. Micha, Topics in DG.
        Für $f \in C^\infty(M)$ definiere die Abbildung
        \begin{math}
            \alpha(t, s) = Y(\Phi^X_t(p)) . (f \circ \Phi_s^X)
        \end{math}
        Hierfür gilt
        \begin{math}
            \alpha(0,0) &= Y(p)(f), \\
            \alpha(t,0) &= Y(\Phi_t^X(p))(f), \\
            \alpha(0,s) &= Y(p)(f \circ \Phi_s^X), \\
            \ddx_{t} \alpha(t,0) |_{t=0} &= \ddx[t] Y(f) (\Phi_t^X(p))|_{t = 0} &
            &= X(Y(f))(p) \\
            \ddx_{s} \alpha(0,s) |_{s=0} &= \ddx[s] Y(p) \ddx[s] (f \circ \Phi_s^X) |_{s = 0} &
            &= Y(X(f))(p).
        \end{math}
        Andererseits gilt
        \begin{math}
            \ddx[u] \alpha(u,-u)|_{u=0}
            &= \ddx[u] Y(\Phi_u^X(p)) (f \circ \Phi_u^X) |_{u=0} \\
            &= \ddx[u] (T \Phi_{-u}^X Y(\Phi_u^X)(f)(p)) |_{u=0}
            = (L_X Y)(f)(p)
        \end{math}
    \end{proof}
\end{st}


\Timestamp{2015-11-17}


\begin{st}
    Die Flüsse zweier Vektorfelder kommutieren genau dann, wenn die Vektorfelder kommutieren, genauer:
    Für $X, Y \in \scr X(M)$ ist $[X, Y] = 0$ äquivalent zu $\Phi_s^X \circ \Phi_t^Y = \Phi_t^Y \circ \Phi_s^X$ ($s, t$ hinreichend klein).
    \begin{proof}
        Angenommen, es gilt $[X, Y] = 0$, dann ist ${\Phi_s^Y}_X = X$, denn
        \begin{enumerate}[(1)]
            \item
                sie gilt für $s = 0$,
            \item
                es gilt
                \begin{math}
                    \ddx[s] {\Phi_s^Y}_* X
                    \stackrel{\ref{3.23}\text{ iii)}}= {\Phi_s^Y}_* \ddx[t] {\Phi_t^Y}_* X |_{t=0}
                    = {\Phi_s^Y}_* [X, Y]
                    = 0.
                \end{math}
        \end{enumerate}
        Nun folgt mit \ref{3.26}
        \begin{math}
            \Phi_s^Y \circ \Phi_t^X = \Phi_t^X \circ \Phi_s^Y.
        \end{math}
        Sei umgekehrt
        \begin{math}
            \ddx[s] \ddx[t] f(\Phi_s^Y \circ \Phi_t^X) |_{t=s=0} &= XYf, \\
            \ddx[s] \ddx[t] f(\Phi_t^X \circ \Phi_s^X) |_{t=s=0} &= YXf.
        \end{math}
        Dann folgt, dass $(XY - YX)f = 0$, falls die Flüsse kommutieren.
    \end{proof}
\end{st}

\begin{note}
    Die Lie-Klammer misst also inifinitesimal, wie weit die Flüsse zweier Vektorfelder davon entfernt sind, zu kommutieren.
    Es gilt sogar
    \begin{math}
        L_X Y = [X, Y] = \ddx[t]|_{t=0} \Phi_{-\sqrt{t}}^Y \circ \Phi_{-\sqrt{t}}^X \circ \Phi_{\sqrt{t}}^Y \circ \Phi_{\sqrt{t}}^X.
    \end{math}
    Kommutieren die beiden Flüsse, dann ist der Ausdruck auf der rechten Seite konstant gleich Null, man umläuft sozusagen eine „Masche“ der Kantenlänge $\sqrt{t}$ und kehrt zum Ausgangspunkt zurück.
\end{note}

\begin{ex*}
    Kartesische Vektorfelder $\ddx[x_i]$ und somit auch Koordinatenvektorfelder $\ddx[\phi_i]$ kommutiren stets, denn
    \begin{math}
        \ddx[x_i] \ddx[x_j] f = \ddx[x_j] \ddx[x_i] f.
    \end{math}
\end{ex*}

\begin{st} \label{3.30}
    Seien $X_1, \dotsc, X_k \in \scr X(N)$ paarweise kommutierende Vektorfelder auf $N$, die bei $p \/n N$ linear unabhängig sind.
    Dann existiert eine $k$-dimensionale Untermannigfaltigkeit $M \subset N$, so dass für alle $q \in M$ die Vektoren $X_1(q), \dotsc, X_k(q)$ eine Basis von $T_q M$ bilden.
    \begin{proof}
        Definiere $F(t_1, \dotsc, t_k) := \Phi_{t_1}^{X_1} \circ \dotsb \circ \Phi_{t_k}^{X_k}(p)$ für $(t_1, \dotsc, t_k) \in B_\eps(0) \subset \R^k$.
        Dann ist
        \begin{math}
            \ddx[t_i] F(t_1, \dotsc, t_k)
            &= \ddx[t_i] \Phi_{t_i}^{X_i} \circ \Phi_{t_1}^{X_1} \circ \dotsb \Phi_{t_{i-1}}^{X_{i-1}} \circ \Phi_{t_{i+1}}^{X_{i+1}} \circ \dotsb \circ \Phi_{t_k}^{X_k}(p) \\
            &= X_i(F(t_1, \dotsc, t_k))
        \end{math}
        und $\ddx[t_i] F(0,\dotsc, 0) = X_i(p)$ für $i = 1, \dotsc, k$ linear unabhängig.
        Somit ist $F$ in einer Umgebung der Null eine Immersion, die nach Aufgabe 10 zu einem Diffeomorphismus erweitert werden kann.
        Dies liefert die Untermannigfaltigkeits-Karte, der letzte Teil der Behauptung folgt aus obiger Rechnung.
    \end{proof}
    \begin{note}
        Dies ist ein Spezialfall des Satzes von Frobenius über Integrabilität von Distributionen.
    \end{note}
\end{st}
