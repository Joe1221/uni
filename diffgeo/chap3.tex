\chapter{Der Tangentialraum}

\Timestamp{2015-10-28}

Wir haben nun den Begriff der Differenzierbarkeit definiert, wissen aber noch nicht, was die Ableitung (d.h. das Differential) einer Abbildung ist.

Das Differential (z.B. für Abbildungen zwischen $\R^n$ und $\R^m$) ist eine Annäherung einer Abbildung in der Nähe eines Punktes durch eine (affin-)lineare Abbildung.
Diese bildet einen Vektorraum in einen Vektorraum ab, dieser wird der Tangentialraum sein.

Da eine \emph{abstrakte} Mannigfaltigkeit zunächst nicht im Euklidischen Raum eingebettet ist, kann man nicht wie bei einer Fläche im $\R^3$ einfach die Geschwindigkeitsvektoren von Kurven durch einen Punkt betrachten, man benötigt Hilfskonstruktionen.


\section{Tangentialvektoren in Karten}


\begin{df} \label{3.1}
    Ein \emphdef{Repräsentant eines Tangentialvektors} in $p \in M$ ist ein Paar $(\phi, u) \in \scr A \times \R^m$, bestehend aus einer Karte $\phi$ um $p$ und einem Element $u \in \R^m$.
    
    Zwei Repräsentanten $(\phi, u)$, $(\psi, v)$ heißen \emphdef[äquivalent!Repräsentanten]{äquivalent}, falls $D(\psi \circ \phi^{-1})|_{\phi(p)}(u) = v$.

    Die Äquivalenzklassen $[(\phi, u)]$ heißen \emphdef{Tangentialvektoren} am \emphdef{Fußpunkt} $p$, s bilden den Tangentialraum $\T_p M$
\end{df}

\begin{lem} \label{3.2}
    Mit den Rechenoperationen
    \begin{math}
        [(\phi, u_1)] + [(\phi, u_2)]
            &:= [(\phi, u_1 + u_2)],\\
        \lambda [(\phi, u)]
            &:= [(\phi, \lambda u)]
    \end{math}
    wird $\T_p M$ zu einem $m$-dimensionalen $\R$-Vektorraum.
    \begin{proof}
        Für jede andere Karte $\psi$ ist die Jacobimatrix $D(\psi \circ \phi^{-1})|_{\phi(p)}: \R^m \to \R^m$ ein linearer Isomorphismus, also sind die Addition und die skalare Multiplikation von der Wahl der Karte unabhängig.
        Durch die Wahl einer festen Karte $\phi$ um $p$ bildet $\Set{(\phi, v) & v \in \R^m}$ ein vollständiges Repräsentantensystem für $\T_p M$.
        Es folgt, dass $\T_p M$ zu $\R^m$ isomorph ist.
    \end{proof}
\end{lem}

\begin{df} \label{3.3}
    Seien $M$ und $N$ differenzierbare Mannigfaltigkeiten und $f: M \to N$ eine differenzierbare Abbildung.
    Dann ist das \emphdef{Differenzial} (auch \emphdef{Ableitung} oder \emphdef{Tangential}) von $f$ an $p \in M$ definiert als die lineare Abildung. 
    \begin{math}
        \T_p f: \T_p M &\to \T_{f(p)} N, \\
        [(\phi, u)] &\mapsto [(\psi, D(\psi \circ f \circ \phi^{-1})|_{\phi(p)}(u))]
    \end{math}
    \begin{proof}
        Wohldefiniertheit:
        Sei $[(\phi,u)] = [\tilde \phi, D(\tilde \phi \circ \phi^{-1})|_{\phi(p)}(u)]$
        Dann ist im Urbild:
        \begin{math}
            (\T_p f) [(\tilde \phi, D(\tilde \phi \circ \phi^{-1})|_{\phi(p)} (u)]
            &= [(\psi, D(\psi \circ f \circ \tilde \phi^{-1})_{\tilde \phi(p)} D(\tilde \phi \circ \phi^{-1})|_{\phi(p)} (u)] \\
            &= [(\psi, D(\psi \circ f \circ \phi^{-1})|_{\phi(p)}(u))] \\
            &= (\T_p f)[(\phi, u)].
        \end{math}
        Im Bild:
        \begin{math}
            [(\psi, D(\psi \circ f \circ \phi^{-1})|_{\phi(p)})]
            &= [(\tilde \psi, D(\tilde \psi \circ \psi^{-1})|_{\psi(f(p)} D(\psi \circ f \circ \phi^{-1})_{\phi(p)}(u))] \\
            &= [(\tilde \psi, D(\tilde \psi \circ f \circ \phi^{-1})|_{\phi(p)} (u))].
        \end{math}
    \end{proof}
\end{df}

\begin{st}[Kettenregel] \label{3.4}
    Seien $M, N, P$ differenzierbare Mannigfaltigkeiten und $M \xto[g] N \xto[f] P$ differenzierbare Abbildungen.
    Dann gilt
    \begin{math}
        \T_p(f \circ g) = (\T_{g(p)} f) \circ (\T_p g)
    \end{math}
    für $p \in M$.
    \begin{proof}
        Wähle Karten $\phi: V \to U$ Karte von $M$ um $p$, $\psi: W \to X$ Karte von $N$ um $g(p)$ und $\tau: Y \to Z$ Karte von $P$ um $f(g(p))$.
        Es gilt
        \begin{math}
            (\T_{g(p)} f) \circ (\T_p g) [(\phi, u)]
            &= (\T_{g(p)} f) [\psi, D(\psi \circ g \circ \phi^{-1})|_{\phi(p)}(u))] \\
            &= [(\tau, D(\tau \circ f \circ \psi^{-1})|_{\psi(g(p))} \circ D(\psi \circ g \circ \phi^{-1})|_{\phi(p)}(u))] \\
            &= [(\tau, D(\tau \circ f \circ g \phi^{-1})|_{\phi(p)}(u))] \\
            &= \T_p(f\circ g) [(\phi, u)].
        \end{math}
    \end{proof}
\end{st}

\begin{df} \label{3.5}
    Sei $M$ eine Mannigfaltigkeit. 
    Wir bezeichnen die disjunkte Vereinigung der Tangentialräume
    \begin{math}
        \T M := \bigdotcup_{p\in M} \T_p M.
    \end{math}
    Sei $\pi: \T M \to M$ die Abbildung, die jeden Tangentialvektor seinen Fußpunkt zuordnet (d.h. $\pi(v) = p$ falls $v \in \T_p M$).
    Wir nennen diese disjunkte Vereinigung aller Tangentialräume an $M$ zusammen mit $\pi$ das \emphdef{Tangentialbündel} von $M$.
\end{df}

\begin{st} \label{3.6}
    Das Tangentialbündel $\T M$ einer Mannigfaltigkeit $M$ ist selbst eine ($2 \dim(M)$)-dimensionale Mannigfaltigkeit.
    %\begin{proof}[Skizze]
        % FIXME: Ordentlich mit Summentopologie, Pullback-Topologie auf \T M.

        %Für jede Karte $\phi: V \to U$ von $M$ kann man wie folgt eine von $\T M$ konstruieren.
        %Sei $\tilde V := \pi^{-1}(V)$ und sei $\tilde p = [(\phi, u)]$.
        %Dann setze $\tilde \phi(\tilde p) = (\pi(\tilde p), u)$.
        %Wir verwenden dann $\tilde \phi$ für jede Karte $\phi$ von $M$ und $\tilde V$, $\tilde U := U \times \R^m$ als Karten von $\T M$. 
    %\end{proof}
\end{st}
