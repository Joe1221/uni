\Timestamp{2015-10-13}

\section*{Was ist Differentialgeometrie?}

\begin{itemize}
    \item
        Ein mathematisches Teilgebiet, in dem geometrische Objekte mit Hilfe von (mehrdimensionaler) Differential- und Integralrechnung studiert werden.
    \item
        „Die Lehre von der Krümmung“
    \item
        Studium glatter Objekte (= glatte Mannigfaltigkeiten) und geometrischer Strukturen.
    \item
        Verallgemeinerung der elementaren Differentialgeometrie, d.h. des Studiums von Kurven und Flächen in der Ebene und dem dreidimensionalen Raum, ihrer Krümmung und globalen Eigenschaften.
\end{itemize}

Angrenzende Teilgebiete:
\begin{itemize}
    \item
        Topologie: insbesondere die Differentialtopologie.
    \item
        Differentialgleichungen und -ungleichungen: z.B. Geodätengleichung, Krümmungsbedingungen (z.B. pos/neg).
    \item
        Liegruppen: die Gruppe der Isometrien einer Riemannschen Mannigfaltigkeit ist eine Liegruppe, Beschreibung von homogenen und symmetrischen Räumen.
    \item
        Variationsrechnung: z.B. Geodätische, Minimalflächen.
    \item
        Funktionentheorie: komplexe Analysis, z.B. Weierstraß-Darstellung einer Minimalfläche.
    \item
        Algebraische Geometrie
    \item
        Kontrolltheorie, etc.
\end{itemize}

Starke Bezüge zur Physik:
Einsteins Allgemeine Relativitätstheorie wird beschrieben mit Begriffen der Differentialgeometrie: die \emph{Raumzeit} ist eine gekrümmte $4$-dimensionale pseudo-riemannsche Mannigfaltigkeit.

Beispiele für typische Sätze/Probleme aus der Differentialgeometrie:

\begin{st}[Gauß-Bonnet]
    Sei $M$ eine zweidimensionale, kompakte, orienterbare riemannsche Mannigfaltigkeit.
    Dann gilt für das Integral über die Gaußkrümmung:
    \begin{math}
        \int_M \kappa = 2 \pi \chi(M),
    \end{math}
    wobei $\chi(M)$ die Eulercharakteristik von $M$ ist.
\end{st}

\begin{prob}
    Klassifikation von positiv gekrümmten riemannschen Mannigfaltigkeiten (bis jetzt nicht vollständig bekannt).
\end{prob}




