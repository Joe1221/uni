\chapter{Lie-Gruppen}

Lie-Gruppen (Mannigfaltigkeiten mit „glatter“ Gruppenstruktur) sind für die Defferentialgeometrie unverzichtbar, vorallem weil sie eine große Fülle von relativ einfach zu studierenden Beispielen liefern.

\begin{note}
    Anstellen der Glattheit von Multiplikation und Inversion kann man auch verlangen, dass $G \times G \to G$, $(a,b) \mapsto a^{-1}b$ glatt ist.
\end{note}

\begin{df} \label{4.1}
    Eien \emphdef{Lie-Untergruppe} $H$ einer Lie-Gruppe $G$ ist eine Untergruppe, die gleichzeitig eine Untermannigfaltigkeit ist.
\end{df}


\Timestamp{2015-11-18}


\begin{lem} \label{4.2}
    Eine Lie-Untergruppe $H \subset G$ ist selbst eine Lie-Gruppe und eine abgeschlossene Teilmenge (im topologischen Sinn) von $G$.
    \begin{proof}
        \begin{seg}{$H$ ist eine Lie-Gruppe}
            $\iota: H \injto G$ ist eine Einbettung und somit (nach Aufgabe 11) ein Diffeomorphismus auf sein Bild, insbesondere $\iota, \iota^{-1}$ glatt.
            Nun sind Multiplikation und Inversion gegeben durch
            \begin{math}
                (a,b) &\mapsto \iota^{-1} (\iota(a) \iota(b)), \\
                a & \mapsto \iota^{-1}(\iota(a)^{-1}),
            \end{math}
            also glatt als Komposition glatter Abbildungen in $G$.
        \end{seg}
        \begin{seg}{$H \subset G$ ist eine abgeschlossene Teilmenge}
            Da $H$ eine Untermannigfaltigkeit ist, ist $H$ lokal abgeschlossen, insbesondere gibt es eine offene Umgebung $U$ von $e$ in $G$, so dass $U \cap H = U \cap \_H$.
            Sei $h \in \_H$.
            Da $hU$ eine offene Umgebung von $h$ in $G$ ist, gilt $hU \cap H \neq \emptyset$.
            Sei $h' \in h U \cap H$, dann gilt $h^{-1} h' \in U$.
            Andererseits, da $h \in \_H$, gibt es eine Folge $h_n$ in $H$, die gegen $h$ konvergiert.
            Es folgt, dass die Folge $h_n^{-1} h'$ gegen $h^{-1} h'$ konvergiert.
            Mit anderen Worten, $h^{-1} h' \in U \cap \_H = U \cap H$, also gilt $h \in H$ und wir haben $\_H \subset H$ gezeigt.
        \end{seg}
    \end{proof}
\end{lem}

\begin{st}[Cartan] \label{4.3}
    Jede abgeschlossene Untergruppe einer Lie-Gruppe ist eine Lie-Untergruppe.
    \begin{proof}
        siehe Literatur.
    \end{proof}
\end{st}

Zusammengefasst: Die Lie-Untergruppen eiener Lie-Gruppe $G$ sind genau die abgeschlossenen Untergruppen.

\begin{ex*}
    Beispiele für Lie-Gruppen:
    \begin{enumerate}[i)]
        \item
            $(\R, +)$,
        \item
            $(\R \setminus \Set 0, \cdot)$,
        \item
            $(\R^n, +)$,
        \item
            $(S^1, \cdot)$, wobei $S^1 = \Set{z \in \C & |z| = 1}$.
            Alternativ
            \begin{math}
                \SO(2) = \Set{\Matrix{\cos t & -\sin t \\ \sin t & \cos t} & t \in \R}.
            \end{math}
        \item
            Jede abgeschlossene Untergruppe von $\GL(n, \R)$ (z.B. als Urbild einer abgeschlossenen Menge)
            \begin{enumerate}[a)]
                \item
                    Sei $f = \det: \R^{n\times n} \to \R$ stetig, dann ist
                    \begin{math}
                        \SL(n, \R)
                        &= \Set{A \in \GL(n, \R) & \det A = 1} \\
                        &= f^{-1}(\Set{1}).
                    \end{math}
                \item
                    Sei $f : \R^{n\times n} \to \R^{n \times n}, A \mapsto AA^t$, dann ist
                    \begin{math}
                        \O(n) &= \Set{A \in \GL(n, \R) & A A^t = I} \\
                        &= f^{-1}(\Set{I})
                    \end{math}
                \item
                    Blockmatrizen mit zwei Blöcken aus $\GL(k, \R)$ und $\GL(n-k, \R)$.
                \item
                    $\GL(n, \C)$ als Untergruppe von $\GL(2n, \R)$,
                    \begin{math}
                        J = \Matrix{0 & -I \\ I & 0} \in \R^{2n \times 2n},
                    \end{math}
                    $I \in \R^n$, dann entspricht $z \mapsto Jz$ der Multiplikation mit $i$.
                    Dann ist
                    \begin{math}
                        \GL(n, \C) = \Set{A \in \GL(2n, \R) & AJ = JA}.
                    \end{math}
            \end{enumerate}
        \item
            Die Schnittmenge zweier Lie-Untergruppen ist wieder eine Lie-Untergruppe, z.B.
            \begin{enumerate}[a)]
                \item
                    $\SO(n) = \O(n) \cap \SL(n, \R)$,
                \item
                    $\U(n) = \GL(n, \C) \cap \O(2n)$,
                \item
                    $\SU(n) = \U(n) \cap \SL(n, \C)$.
            \end{enumerate}
    \end{enumerate}
\end{ex*}

\begin{df} \label{4.4}
    Für eine Lie-Gruppe $G$ und ein Gruppenelement $g \in G$ seien die \emphdef{Linksmultiplikation}, bzw. \emphdef{Rechtsmultiplikation} mit $g$ gegeben durch
    \begin{math}
        L_g: G &\to G,& x &\mapsto gx, \\
        R_g: G &\to G,& x &\mapsto xg.
    \end{math}
    \begin{note}
        \begin{itemize}
            \item
                $L_g$ und $R_g$ sind Diffeomorphismen auf $G$, denn sie siend glatt und bijektiv, ihre Umkehrabbildung ist gegeben durch $(L_g)^{-1} = L_{g^{-1}}$, $(R_g)^{-1} = R_{g^{-1}}$.
            \item
                Eine Lie-Gruppe ist \emphdef{homogen}, d.h. zu $g_1, g_2 \in G$ gibt es einen Diffeomorphismus, der $g_1$ auf $g_2$ abbildet, nämlich $L_{g_2g_1^{-1}}$ und zwar „wirkt“ $G$ auf sich selbst durch Diffeomorphismen.
        \end{itemize}
    \end{note}
\end{df}


\Timestamp{2015-11-24}

Auf Lie-Gruppen haben wir folgende Konstruktion.

\begin{df} \label{4.5}
    Ein \emphdef{linksinvariantes Vektorfeld} auf einer Liegruppe $G$ ist ein Vektorfeld $X \in \scr X(M)$, so dass
    \begin{math}
        {L_g}_* X = X
    \end{math}
    für alle $g \in G$.
    \begin{note}
        \begin{itemize}
            \item
                Ein linksinvariantes Vektorfeld $X \in \scr X(G)$ einer Lie-Gruppe ist durch seinen Wert an einem Punkt $X(g) \in G$ festgelegt, denn es gilt für jedes $h \in G$ dann
                \begin{math}
                    X(h) = ({L_{hg^{-1}}}_* X)(h)
                    = (T_{g} L_{gh^{-1}})(X(g)),
                \end{math}
                Man spricht auch von der \emphdef{linksinvarianten Fortsetzung} von $X(g)$.
                Umgekehrt lässt sich zu jedem Vektor $v \in T_g G$ ein linksinvariantes Vektorfeld $X$ finden, so dass $X(g) = v$, definiert durch obige Formel.
        \end{itemize}
    \end{note}
\end{df}

\begin{ex*}
    \begin{itemize}
        \item
            Fasst man $(\R^n, +)$ als Lie-Gruppe auf, dann sind die linksinvariante Vektorfelder genau die Konstanten.
        \item
            Auf $S^1 = \Set{z \in \C & |z| = 1} = U(1)$ sind die linksinvarianten Vektorfelder durch Rotationen gegeben.
    \end{itemize}
\end{ex*}

\begin{lem} \label{4.6}
    Die linksinvarianten Vektorfelder auf einer Lie-Gruppe bilden eine Lie-Algebra mit der Lie-Klammer von Vektorfelder aus \ref{3.18} als Verknüpfung.
    \begin{proof}
        Offensichtlich bilden die linksinvarianten Vektorfelder einen Untervektorraum von $\scr X(G)$.
        Es bleibt zu zeigen, dass die Lie-Klammer von zwei linksinvarianten Vektorfeldern wieder ein linksinvariantes Vektorfeld ist.
        Dies folgt aus Aufgabe 17, denn ein Vektorfeld auf $G$ ist genau dann linksinvariant, wennn es mit sich selbst $L_g$-verwandt ist für jedes $g \in G$, denn dann
        \begin{math}
            {L_{g}}_* [X_1, X_2]
            = [{L_g}_* X_1, {L_g}_* X_2]
            = [X_1, X_2].
        \end{math}
    \end{proof}
\end{lem}

\begin{df} \label{4.7}
    Wir schreiben $\frk g$ für die Menge der linksinvarianten Vektorfelder.
    Wir identifizieren $\frk g$ mit $T_e G$ in kanonischer Weise (siehe Bemerkung in \ref{4.5}).
    Man nennt $\frk g = T_e G$ auch die \emphdef{Lie-Algebra} von $G$.
\end{df}

\begin{df} \label{4.8}
    Lie-Gruppen, die als abgeschlossene Untergruppen von $\GL(n, \R)$ gegeben sind, nennen wir \emphdef{Matrix-Lie-Gruppen}.
\end{df}

\begin{df} \label{4.9}
    Sei $G$ eine Lie-Gruppe und $\frk g$ ihre Lie-Algebra.
    Wir definieren die Exponentialabbildung $\exp: \frk g \to G$ der Lie-Gruppe $G$ durch
    \begin{math}
        \exp(X) := \Phi_1^X(e),
    \end{math}
    d.h. $\exp(X)$ ist der Fluss des linksinvarianten Vektorfeldes $X$ nach der Zeit $t = 1$, angewendet auf das neutrale Element.
    \begin{note}
        Flüsse linksinvarianter Vektorfelder sind global wohldefiniert.
        Nach \ref{3.26} ($f \circ \Phi_t^X = \Phi_t^X \circ f$ für einen Diffeomorphismus $f$ und ein $f$-invariantes Vektorfeld $X$) folgt $L_g \circ \Phi_t^X = \Phi_t^X \circ L_g$, also
        \begin{math}
            g \Phi_t^X(e) = \Phi_t^X(g).
        \end{math}
        Insbesondere: ist der Fluss beim Startwert $e$ für $t \in (-\eps, \eps)$ definiert, so auch beim Startwert $g$.
        Es folgt globale Wohldefiniertheit, siehe auch Köhler, 1.6.10.
    \end{note}
\end{df}

\begin{ex*}
    \begin{enumerate}[(i)]
        \item
            Sei $G = (\R^n, +)$, dann gilt $\exp(X) = X$, denn $\Phi_t^X(p) = p + tX$.
        \item
            Für $G = \GL(n, \R)$ ist $\exp(X) = \sum_{k=0}^\infty \frac{X^k}{k!}$ die Matrix-Exponentialabbildung, denn
            \begin{math}
                \ddx[t] |_{t=0} e^{tX}
                &=\ddx[t] |_{t=0} \sum_{k=0}^\infty \frac{X^k}{k!} \\
                &=\sum_{k=1}^\infty \frac{X^k}{k!} k t^{k-1} |_{t=0}
                = X.
            \end{math}
            und somit gilt
            \begin{math}
                X(g) = {L_{g}}_* X
                &= g \ddx[t]|_{t=0} \exp(tX) \\
                &= \ddx[t]|_{t=0} g \exp(tX).
            \end{math}
            Außerdem: Für kommutierende Matrizen $X, Y$ gilt $e^{X + Y} = e^X e^Y$ (Beweis wie im eindimensionalen).
            Mit dem obigen ergibt sich
            \begin{math}
                \ddx[s] \exp(sX)
                &= \ddx[t]|_{t=0} \exp((s+t)X) \\
                &= \ddx[t]|_{t=0} \exp(sX) \exp(tX)
                = X(\exp(sX))
            \end{math}
            Außerdem ist $\exp(0 X) = e$.
            Insgesamt haben wir gesehen, dass $s \mapsto \exp(sX)$ eine Integralkurve zu dem linksinvarianten Vektorfeld $X$ mit Anfangswert $e$ ist.
    \end{enumerate}
\end{ex*}

\begin{lem} \label{4.10}
    Auf jeder Mannigfaltigkeit $M$ für jedes $X \in \scr X(M)$ und $s, t$ hinreichend klein gilt
    \begin{math}
        \Phi_{st}^X = \Phi_s^{tX}.
    \end{math}
    \begin{proof}
        \begin{enumerate}[(i)]
            \item
                $\Phi_{0t}^X (p) = p = \Phi_0^{tX}(p)$
            \item
                Es gilt
                \begin{math}
                    \ddx[s]|_{s=0} \Phi_{st}^X(p) = tX(p) = \ddx[s]|_{s=0} \Phi^{tX}_s(p)
                \end{math}
                und $\Phi_{(s_1 + s_2)t}^X = \Phi_{s_1 t}^X \circ \Phi_{s_2 t}$.
        \end{enumerate}
        Zeige
        \begin{math}
            \ddx[s] \Phi_{st}^X(p) = tX(\Phi_{st}^X(p))
        \end{math}
        mit Kettenregeltrick auf $\Phi_{(s_1 + s_2)t}^X(p)$.
    \end{proof}
\end{lem}

Allgemein (nicht nur für Matrix-Lie-Gruppen) gilt:

\begin{lem} \label{4.11}
    Für eine Lie-Gruppe $G$ und ein linksinvariantes Vektorfeld $X \in \frk g$ gilt
    \begin{math}
        \Phi_t^X(g) = g \exp(tX)
    \end{math}
    \begin{proof}
        \begin{math}
            \ddx[t]|_{t=0} g \exp(tX)
            &= \ddx[t]|_{t=0} L_g (\exp(tX)) \\
            &\stack{\ref{4.10}}= \ddx[t]|_{t=0} L_g(\Phi_t^X(e)) \\
            &= T_e L_g(\ddx[t]|_{t=0} \Phi_t^X(e))
            = T_e L_g(X)
            = X(g)
        \end{math}
    \end{proof}
\end{lem}

\begin{st} \label{4.12}
    Für das Differential der Exponentialabbildung $\exp: \frk g \to G$ einer Lie-Gruppe gilt bei der Null:
    \begin{math}
        T_0 \exp = \id_{\frk g}
        :T_0 \frk g = \frk g \to T_e G = \frk g
    \end{math}
    \begin{proof}
        \begin{math}
            T_0 \exp(X)
            = \ddx[t]|_{t=0} \exp(tX)
            = \ddx[t]|_{t=0} \Phi_1^{tX}(e)
            \stack{\ref{4.10}}= \ddx[t]|_{t=0} \Phi_t^X(e)
            = X(e).
        \end{math}
    \end{proof}
\end{st}

\begin{kor} \label{4.13}
    Auf einer Umgebung $V$ der Null in $\frk g = T_e G$ ist $\exp|_V: V \to \exp(V)$ ein Diffeomorphismus und die Umkehrabbildung $\log := {\exp|_{V}}^{-1}$ eine kanonische Karte, genannt \emphdef{logarithmische Karte}.
    \begin{proof}
        Umkehrsatz für Mannigfaltigkeiten, siehe Aufgabe 9.
    \end{proof}
\end{kor}

\begin{st} \label{4.14}
    Für jeden Lie-Gruppenhomomorphismus (d.h. ein glatter Homomorphismus von Lie-Gruppen) $f: G \to H$ gilt
    \begin{math}
        f(\exp(X)) = \exp(T_e f(X))
    \end{math}
    für alle $X \in \frk g$.
    \begin{proof}
        Sei $Y := T_e f(X)$ und seien die linksinvarianten Fortsetzungen von $X \in T_e G$, $Y \in T_e H$ ebenfalls mit $X$, bzw. $Y$ bezeichnet.
        Dann gilt $f(\Phi_t^X(e)) = \Phi_t^Y(e)$, denn $f(\Phi_0^X(e)) = f(e_G) = e_H$ und
        \begin{math}
            \ddx[t]|_{t=t_0} f(\Phi_t^X(e))
            &= \ddx[t]|_{t=0} f(\Phi_t^X(\Phi_{t_0}^X(e))) \\
            &= \ddx[t]|_{t=0} f(\Phi_{t_0}^X(e) \Phi_{t}^X(e)) \\
            &= \ddx[t]|_{t=0} f(\Phi_{t_0}^X(e)) f(\Phi_t^X(e)) \\
            &= \ddx[t]|_{t=0} L_{f(\Phi_{t_0}^X(e))} (f(\Phi_t^X(e))) \\
            &= {L_{f(\Phi_{t_0}^X(e))}}_* T_e f(X) \\
            &= Y(f(\Phi_{t_0}^X(e)))
        \end{math}
    \end{proof}
\end{st}

\begin{kor} \label{4.15}
    Für eine Lie-Untergruppe $H \subset G$ gilt für $X \in \frk h$
    \begin{math}
        \exp_H(X)
        = \iota(\exp_H(X))
        = \exp_G(T_0 \iota(X))
        = \exp_G(X),
    \end{math}
    wobei $\iota: H \injto G$ die Inklusionsabbildung bezeichnet.
    Mit anderen Worten: $\exp_G|_{\frk h} = \exp_H$.
\end{kor}

\begin{ex*}
    Damit folgt, dass für alle Matrix-Lie-Gruppen $G \subset \GL(n, \R)$, dass die Exponentialabbildung $\exp_G$ von $G$ gegeben ist durch die Matrixdarstellung $X \mapsto e^X = \sum_{k=0}^\infty \frac{X^k}{k!}$.
\end{ex*}

\begin{st} \label{4.16}
    Sei $G$ eine zusammenhängende Lie-Gruppe.
    Dann wird $G$ (als Gruppe) von jeder Umgebung des Einselements erzeugt und jeder Lie-Gruppenhomomorphismus $f$ ist durch sein Differential $T_e f$ am Einselement bereits eindeutig festgelegt.
    \begin{proof}
        Sei $\tilde U$ eine offene Umgebung von $e \in G$ und setze $U:= \tilde U \cap {\tilde U}^{-1}$.
        Sei $H := \bigcup_{n \in \N} U^n$.
        Dann ist $H$ offen in $G$.

        Auch die Teilmenge $V := \bigcup_{m \in G \setminus H} mH$ ist offen in $G$
        Außerdem gilt $G = H \cup V$ und $H \cap V = \emptyset$.
        Da $G$ zusammenhängend ist, muss $V = \emptyset$ ($H$ enthält $e$).
        Folglich $H = G$.

        Für $\tilde U$ hinreichend klein ist $f$ nach \ref{4.14} auf $\tilde U$ durch $f(\exp(X)) = \exp(T_e f(X))$ bestimmt, alle anderen Elemente $g$ lassen sich als (endliche) Produkte $g = g_1 \dotsc g_k$ mit $g_i \in \tilde U$ schreiben und somit ist $f(g) = f(g_1) \dotsc f(g_k)$ auch durch $T_e f$ bestimmt.
    \end{proof}
\end{st}

\begin{note}
    \begin{itemize}
        \item
            $\exp$ ist im Allgemeinen \emph{nicht} surjektiv, deswegen reicht 4.14 nicht aus aus, um den zweiten Teil der Behauptung von \ref{4.16} zu beweisen, vgl. Aufgabe 22.
        \item
            Es lässt sich aber (für $G$ zusammenhängend) zeigen, dass $\exp(\frk g)^2 = G$, d.h jedes Element von $G$ lässt sich als Produkt von zwei Elementen aus dem Bild von $\exp$ schreiben.
    \end{itemize}
\end{note}


\Timestamp{2015-12-08}


\begin{df} \label{4.17}
    Eine \emphdef{Darstellung} einer Gruppe $G$ auf ein Vektorraum $V$ ist ein Homomorphismus $\rho: G \to \Gl(V)$.
    \begin{note}
        \begin{itemize}
            \item
                Ist $V$ ein $\R$-Vektorraum, dann sagt man auch: $\rho$ ist eine \emphdef[Darstellung!reell]{reelle Darstellung}.
            \item
                Ist $\rho$ injektiv, dann spricht man von einer \emphdef[Darstellung!treu]{treuen Darstellung}.
            \item
                Ist in $V$ eine endliche Basis gegeben, kann man $\Gl(V)$ mit $\Gl(n, \K)$ identifizieren.
                Dann wird jedes Gruppenelement $g$ durch die Matrix $\rho(g)$ dargestellt und die Gruppenoperation auf $G$ entpsricht der Matrixmultiplikation.
        \end{itemize}
    \end{note}
\end{df}

\begin{ex}
    \begin{itemize}
        \item
            Für $G \subset \Gl(n, \K)$ ist eine Darstellung von $G$ auf $\K^n$ gegeben, diese nennt man \emphdef{Standarddarstellung}.
            Standarddarstellung von $\O(n)$ auf $\R^n$
        \item
            Aufgabe 19:
        \item
            Die \emphdef{triviale Darstellung} von $G$ auf $V$ ist gegeben durch $\rho(g) = \id_V$.
        \item
            Die \emphdef{adjungierte Darstellung}, siehe folgende Definition.
    \end{itemize}
\end{ex}

\begin{df} \label{4.18}
    Sei $G$ eine Lie-Gruppe und $\frk g = T_e G$ die Lie-Algebra.
    Für $g \in G$ sei die \emphdef{Konjugation mit $g$} gegeben als
    \begin{math}
        \iota_g : G &\to G, \\
        x &\mapsto gxg^{-1} = L_g \circ R_{g^{-1}}(x).
    \end{math}
    $\iota_g$ ist ein Gruppenautomorphismus.

    Definiere $\Ad_g := T_e \iota_g$, d.h. $\Ad_g$ ist die lineare Abbildung
    \begin{math}
        \Ad_g: \frk g &\to \frk g, \\
        X &\mapsto T_e(L_g \circ R_{g^{-1}})(X).
    \end{math}
    Definiere $\Ad: G \to \Gl(\frk g)$ durch $\Ad(g) := \Ad_g$.
    \begin{note}
        Wegen $\iota_{g^{-1}} = \iota_g^{-1}$ ist $\Ad_G$ tatsächlich ein Vektorraumisomorphismus von $\frk g$, denn $\Ad_g = T_e \iota_g \circ T_e \iota_{g^{-1}} = T_e \id_G = \id_{\frk g}$.
        Homomorphieeigeschaft siehe folgender Satz.
    \end{note}
\end{df}

\begin{st} \label{4.19}
    $\Ad_g$ und $\Ad$ erfüllt folgende Eigenschaften:
    \begin{enumerate}[(i)]
        \item
            $\Ad_g$ ist ein Lie-Algebren-Automorphismus.
        \item
            $\Ad$ ist tatsächlich eine Darstllung von $G$,
        \item
            $\Ad_g(X) = \ddx[t]|_{t=0} g \exp(tX)g^{-1}$.
    \end{enumerate}
    \begin{proof}
        \begin{enumerate}[(i)]
            \item
                Es gilt $\iota_g \circ \iota_h(x) = ghxh^{t1}g^{-1} = (gh)x(gh)^{-1} = \iota_{gh}(x)$.
                Mit Kettenregel folgt $\Ad_g \circ \Ad_h = T_e \iota_g \circ T_e \iota_h = T_e \iota_{gh} = \Ad_{gh}$, was (ii) zeigt.

                Es gilt außerdem, dass für linksinvariantes $X$ auch $(L_g \circ R_{g^{-1}})_*(X)$ wieder linksinvariant ist, somit folgt $[\Ad_g(X), \Ad_g(Y)] = \Ad_g([X,Y])$, vgl. Aufgabe 17, was (i) zeigt.

                (iii) folgt direkt aus der Kettenregel.
        \end{enumerate}
    \end{proof}
\end{st}

\begin{st} \label{4.20}
    Für $X, Y \in \frk g$ gilt
    \begin{math}
        [X, Y] = \ddx[s]|_{s=0} \Ad_{\exp(sX)}(Y)
        = \frac{\partial^2}{\partial s \partial t} |_{s=t=0} \exp(sX) \exp(tY) \exp(-sX).
    \end{math}
    Mit anderen Worten $T_e(\Ad_{\argdot}(Y))(X) = [X, Y]$.
    \begin{proof}
        \begin{math}
            [X, Y](e)
            &= L_X Y(e) \\
            &= \ddx[s]|_{s=0} {\Phi^{-X}_s}_* Y(e) \\
            &= \ddx[s]|_{s=0} T \Phi_s^{-X} (Y(\Phi_s^X(e))) \\
            &= \frac{\partial^2}{\partial s \partial t} |_{s=t=0} (\Phi_s^{-X} \circ \Phi_t^Y \circ \Phi_s^X(e)) \\
            &= \frac{\partial^2}{\partial s \partial t} |_{s=t=0} R_{\exp(sX)\exp(tX)\exp(-sX)}(e) \\
            &= \frac{\partial^2}{\partial s \partial t} |_{s=t=0} \exp(sX)\exp(tX)\exp(-sX),
        \end{math}
        dank $\Phi_t^X(g) = g \Phi_t^Y(g)$.
    \end{proof}
\end{st}

\begin{nt*}
    Wir kennen nun Lie-Klammern auf Mannigfaltigkeiten und auf linksinvariante Vektorfelder.

    Auf Mannigfaltigeiten hatten wir die Interpretation als Derivationen und als Lie-Ableitung von Vektorfeldern gesehen (außerdem Fluss + Wurzeln).

    Für linksinvariante Vektorfelder haben wir ihre Interpretation als Ableitung von $\Ad$ gesehen.
\end{nt*}

\begin{ex*}
    Für $G = \Gl(n, \R)$, $X, Y \in \R^{n\times n} = T_e \Gl(n,\R)$.
    (linksinvariante Fortsetzungen beachten!)
    \begin{math}
        [X, Y] = \ddx[s]|_{s=0} e^{sX} Y e^{-sX}
        = XY - YX.
    \end{math}
    D.h. für Matrix-Lie-Gruppen ist die Lie-Klammer durch den Kommutator von Matrix gegeben.

    $[\tilde X, \tilde Y] = \widetilde{[X, Y]}$ (Lieklammer auf Vektorfelder und Kommutator auf Matrizen).
\end{ex*}

Erinnerung:
\begin{math}
    \begin{tikzcd}
        \frk g \ar[r,"T_e f"] \ar[d,"\exp_G"] & \frk h \ar[d,"\exp_H"] \\
        G \ar[r,"f"] & H
    \end{tikzcd}
\end{math}

\begin{kor} \label{4.21}
    Mit \ref{4.14} folgt erhalten wir
    \begin{math}
        \ad := T_e \Ad_{\argdot} : \frk g \to \End(\frk g),
    \end{math}
    d.h. $\ad(X) = [X,\argdot]$ gilt:
    \begin{math}
        \Ad_{\exp_G(X)} = \exp_{\Gl(\frk g)}(\ad(X)).
    \end{math}
\end{kor}

