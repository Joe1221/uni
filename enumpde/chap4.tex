\chapter{Approximation für parabolische Probleme} \label{chap:4}

\Timestamp{2015-01-20}

\paragraph{Motivation:}

Betrachte die instationäre Wärmeleitungsgleichung:
\begin{math}[numbered] \label{eq:4.1}
	\partial_t u - \Laplace u &= f &&\text{in $\Omega \times (0,T)$} \\
	u &= g && \text{auf $\partial \Omega \times (0,T)$} \\
	u(\argdot, 0) &= u_0.
\end{math}
Das Vorgehen zur FEM-basierten Approximation ist wie folgt:
\begin{itemize}
	\item
		Definiere geeignete „schwache Lösung“ für \eqref{4.1}.
	\item
		Führe die FEM Ortsdiskretisierung durch.
	\item
		Führe die FD Zeitdiskretisierung durch.
	\item
		Untersuche das Verfahren auf Stabilität, Konvergenz und Fehleraussagen.
\end{itemize}

Erweiterung auf allgemeinere parabolische Probleme ist analog möglich, wird hier jedoch nicht behandelt.

Zunächst einige Definitionen und Aussagen ohne Beweis (in diesen Fällen siehe Evans, §5.9.2 und Appendix E)

\begin{df}[Bochner-Räume] \label{4.1}
	Sei $X$ ein Banachraum mit Norm $\|\argdot\|_X$ und $T > 0$.
	Wir definieren für alle messbaren Funktionen $u: [0,T] \to X$
	\begin{math}
		\|u\|_{L^p([0,T], X)} &:= \Big( \int_0^T \|u(t)\|_X^p \di[t] \Big)^{\frac{1}{p}}, && 1 \le p < \infty, \\
		\|u\|_{L^\infty([0,T], X)} &:= \esssup_{t\in[0,T]} \|u(t)\|_{X}.
	\end{math}
	Für $1 \le p \le \infty$ definieren wir die \emphdef{Bochner-Räume} als
	\begin{math}
		L^p([0,1], X) := \Set{ u:[0,T] \to X \text{ messbar} & \|u\|_{L^p([0,1], X)} < \infty }.
	\end{math}
	\begin{note}
		\begin{itemize}
			\item
				Für eine sinnvolle Definition von $\|u\|_{L^p([0,T], X)}$, dem sogenannten \emphdef{Bochner-Integral}, ist genauer die \emphdef{Bochner-Messbarkeit} erforderlich.
		\end{itemize}
	\end{note}
\end{df}

\begin{ex*}
	\begin{itemize}
		\item
			Für $\Omega \subset \R^d$ offen und beschränkt gilt
			\begin{math}
				L^2([0,T), L^2(\Omega)) &= L^2([0,T] \times \Omega), \\
				L^2([0,T], H^1(\Omega)) &= \Set{ u \in L^2([0,T] \times \Omega) & \Nabla_x u \in L^2([0,T] \times \Omega) }.
			\end{math}
		\item
			$L^2([0,T], X)$ ist Hilbertraum, falls $X$ Hilbertraum via kanonischem Skalarprodukt:
			\begin{math}
				\<u, v\>_{L^2([0,T], X)} := \int_0^T \< u(t), v(t) \>_{X}.
			\end{math}
	\end{itemize}
\end{ex*}

% FIXME: \begin{df}[Raum $C([0,T], X)$] \label{4.2}
\begin{df} \label{4.2}
	Sei $X$ ein Banachraum mit Norm $\|\argdot\|_{X}$.
	Wir definieren für stetiges $u: [0,T] \to X$
	\begin{math}
		\|u\|_{C([0,T], X)} := \max_{0 \le t \le T} \|u(t)\|_{X}
	\end{math}
	und damit
	\begin{math}
		C([0,T], X) := \Set{u : [0,T] \to X \text{ stetig} & \|u\|_{C([0,T], X)} < \infty }.
	\end{math}
	\begin{note}
		\begin{itemize}
			\item
				Analog definiert man den Raum der stetig differenzierbaren Funktionen $C^1([0,T], X)$.
		\end{itemize}
	\end{note}
\end{df}

\begin{df}[Schwache Zeitableitung] \label{4.3}
	Sei $u \in L^1([0,T], X)$.
	Wir nennen $v \in L^1([0,T], X)$ \emphdef{schwache Zeitableitung} von $u$, falls
	\begin{math}
		\int_0^T u(t) \phi'(t) \di[t]
		&= - \int_0^T v(t) \phi(t) \di[t],
		&& \forall \phi \in C_0^\infty([0,T]).
	\end{math}
	Wir schreiben dann $u'$ statt $v$.
\end{df}

\begin{st}[Einbettung] \label{4.4}
	Sei $u \in L^2([0,T], H_0^1(\Omega))$ mit schwacher Ableitung $u \in L^2([0,T], H^{-1}(\Omega))$.
	Dann gilt
	\begin{enumerate}[i)]
		\item
			$u \in C([0,T], H_0^1(\Omega))$, d.h. es existiert ein stetiger Repräsentant.
		\item
			die Abbildung $t \mapsto \|u(t)\|_{L^2(\Omega)}^2$ ist absolut stetig und es gilt
			\begin{math}
				\ddx[t] \|u(t)\|_{L^2(\Omega)}^2
				= 2 \< u'(t), u(t) \>_{L^2(\Omega)}
			\end{math}
			für fast alle $t \in [0,T]$.
		\item
			Es gilt die Abschätzung
			\begin{math}
				\|u\|_{C([0,T], L^2(\Omega))}
				\le C \Big( \|u\|_{C([0,T], H_0^1(\Omega)} + \|u'\|_{L^2}([0,T], H^{-1}(\Omega)) \Big).
			\end{math}
			Dabei hängt die Konstante $C$ nur von $T$ ab.
	\end{enumerate}
	\begin{proof}
		Siehe Evans, Thm. 3 in §5.9.
	\end{proof}
	\begin{note}
		\begin{itemize}
			\item
				Ist eine Verallgemeinerung der Einsicht aus §3, dass $v \in H^1(\R)$, d.h. eine Funktion mit schwacher Ableitung global stetig ist.
			\item
				Es wird statt $u \in L^2([0,T], H_0^1(\Omega))$ hier allgemeiner $u' \in L^2([0,T], H^{-1}(\Omega))$ erlaubt.
				Erinnerung aus \ref{chap:3} zu Bemerkung nach Satz \ref{3.22} (Stetigkeit von $l$):
				es gilt $\int_\Omega fv < \infty$, falls $f, v \in L^2(\Omega)$, aber für $v \in H_0^1(\Omega)$ sind auch Funktionen $f \not\in L^2$ erlaubt, also
				\begin{math}
					„H^{-1}(\Omega)“ \supsetneq L^2(\Omega) \supsetneq H_0^1(\Omega).
				\end{math}
				Die rechte Inklusion ist klar, die linke ist eine etwas laxe Notation für eine geeignete Einbettung
				\begin{math}
					f &\in L^2
					&& \leadsto &
					\tilde f(v) &:= \int_\Omega fv
				\end{math}
				und dann $f \in H^{-1}(\Omega)$.

				Ähnlich ist es hier: $u'(t) \in H^{-1}(\Omega)$ ist in dem Sinne zu verstehen, dass $u'(t)$ kein Funktional, sondern eine Funktion auf $\Omega$ ist und
				\begin{math}
					l(v) := \int_\Omega u'(t,x) v(x) \di[x]
				\end{math}
				erfüllt $l \in H^{-1}(\Omega)$.
				Dann ist insbesondere $\int_\Omega u'(t,x) v(x) \di[x] < \infty$ für $v \in H_0^1(\Omega)$ sinnvoll definiert und kann in Definition der schwachen Lösung verwendet werden.
		\end{itemize}
	\end{note}
\end{st}

\begin{df}[schwache Lösung für Wärmeleitung] \label{4.5}
	Sei $\Omega \subset \R^d$ Lipschitz-Gebiet, $T > 0$.
	Die Funktion $u: \Omega \times [0,T] \to \R$ heißt \emphdef{schwache Lösung} des ARWP der Wärmeleitungsgleichung mit Dirichlet-Nullrandwerten zu $v_0 \in L^2(\Omega)$ und $f \in L^2([0,T], H^{-1}(\Omega))$ falls
	\begin{enumerate}[i)]
		\item
			\emph{Regularität}: $u \in L^2([0,T], H_0^1(\Omega))$ und $u' \in L^2([0,T], H^{-1}(\Omega))$,
		\item
			\emph{PDE}:
			Für alle $\phi \in H_0^1(\Omega)$ und fast alle $t \in (0,T)$ gilt
			\begin{math}[numbered] \label{eq:4.1.5}
				\<u'(t), \phi\>_{L^2(\Omega)} + \< \Nabla u(t), \Nabla \phi \>_{L^2(\Omega)}
				= \<f(t), \phi\>_{L^2(\Omega)},
			\end{math}
		\item
			\emph{Anfangswerte}: $u(\argdot, 0) = u_0$.
	\end{enumerate}
	\begin{note}
		\begin{itemize}
			\item
				Die Definiton der Anfangswerte ergibt Sinn, da mit i) und \ref{4.4} i) $u$ stetig ist.
			\item
				Inhomogene Dirichlet-Randwerte können wie in \ref{chap:3} behandelt werden (Lösen eines homogenen Problems, Addition der Randwerte).
		\end{itemize}
	\end{note}
\end{df}

\begin{st}[Existenz und Eindeutigkeit] \label{4.6}
	Für ein ARWP in schwacher Form gemäß \ref{4.5} existiert eine eindeutige schwache Lösung
	\begin{proof}
		Siehe z.B. G. Dziuk: Theorie und Numerik partieller Differentialgleichungen,  De Gruyter Verlag, Berlin, 2000.
	\end{proof}
\end{st}

Es folgt die Diskretisierung im Ort, was auf ein Anfangswertproblem eines Systems gewöhnlicher Differentialgleichungen für die DOFs führt.
Wähle einen endlichdimensionalen FEM-Approximationsraum $V_h \subset H_0^1(\Omega)$ und führe Galerkin-Projektion der schwachen Form \eqref{eq:4.1.5} auf $V_h$ durch.

\begin{df}[FEM Semidiskretisierung] \label{4.7}
	Sei $\scr T_h$ zulässige Triangulierung, $V_h := \P_{k,0}(\scr T_h)$.
	Wir nennen $u_h \in C^1([0,T], V_h)$ Lösung der semidiskreten Wärmeleitung, falls
	\begin{math}[numbered] \label{eq:4.2}
		\<u_h'(t), \phi\>_{L^2(\Omega)} + \<\Nabla u_h(t), \Nabla \phi\>_{L^2(\Omega)} &= \<f(t), \phi\>_{L^2(\Omega)}, && \forall \phi \in V_h, t \in (0,T), \\
		u_h(0) &= u_h^0,
	\end{math}
	mit geeignetem $u_h^0 \in V_h$.
\end{df}

\begin{kor}[AWP für Koeffizientenvektor] \label{4.8}
	Sei $\Set{\phi_i}_{i=1}^n$ eine nodale Basis von $V_h$ zu Knoten $\Set{x_i}_{i=1}^n \subset \_\Omega$,
	\begin{math}[numbered] \label{eq:4.3}
		u_h(t) = \sum_{i=1}^n u_{h,i}(t) \phi_i
	\end{math}
	mit DOF-Vetor $\underbar{u}_h = (u_{h,i})_{i=1}^n$.
	Dann ist $u_h$ Lösung des semidiskreten Problems \eqref{eq:4.2} genau dann, wenn $\underbar{u}_h: [0,T] \to \R^n$ Lösung des AWP
	\begin{math}[numbered] \label{eq:4.4}
		M_h \underbar{u}_h'(t) + A_h \underbar{u}_h(t) &= f_h(t), \\
		\underbar{u}_h(0) = u_h^0
	\end{math}
	mit Massematrix $M_h = (\<\phi_i, \phi_j\>_{L^2(\Omega)})_{i,j = 1}^n$, Steifigkeitsmatrix $A_h = (\<\Nabla \phi_i, \Nabla \phi_j\>_{L^2(\Omega)})_{i,j=1}^n$ und rechte Seite $f_h(t) = (\int_\Omega f(t) \phi_i)_{i=1}^n$.
	\begin{proof}
		Klar durch Einsetzen von Ansatz \eqref{4.4} in die PDE in semidiskreter Form.
	\end{proof}
	\begin{note}[Wahl der $u_0$-Approximation]
		\begin{itemize}
			\item
				Man kann z.B. $u_h^0$ als orthogonale $L^2$-Projektion $u_h^0 := P_{V_h} u_0$ wählen.
				Dies lässt sich als LGS schreiben:
				\begin{math}
					u_h^0 &= P_{V_h} u_0
					&&\iff & \int_\Omega ( u_h^0 - u_0) \phi_i &= 0, \forall i \\
					& &&\iff & \int_\Omega \underbrace{u_h^0}_{=\sum_{j} u_{h,j}^0 \phi_j} \phi_i &= \underbrace{\int_\Omega u_0 \phi_i}_{u_{0,i}}, \forall i \\
					& &&\iff & M_h \underbar{u}_h^0 &= \underbar{u}_0
				\end{math}
				mit $\underbar{u}_0 := (\int_\Omega u_0 \phi_i )_{i=1}^n$.
\Timestamp{2015-01-27}
			\item
				Alternativ kann auch eine Polynominterpolation erfolgen, falls $u_0$ Punktauswertung erlaubt.
				Wegen nodaler Basis erreicht man dies durch $\underbar{u}_h^0 := \Vector{u_0(x_1) & \dots & u_0(x_n)} \in \R^n$.
			\item
				Problem \eqref{eq:4.2}, bzw. \eqref{eq:4.4} ist kontinuierlich in $t$, aber diskret bezüglich $x$, daher„semidiskret“.
			\item
				Anschaulich: wegen nodaler Basis ist
				\begin{math}
					(\underbar{u}_h(t))_i = u_h(x_i, t) \approx u(x_i, t).
				\end{math}
				Man sucht also Lösung entlang von Linien im $\_\Omega \times (0,T)$-Zylinder, daher heißt \eqref{eq:4.2} bzw. \eqref{eq:4.4} auch „Linienmethode“ (engl. “method of lines”).
		\end{itemize}
	\end{note}
\end{kor}

\subsection*{Diskretisierung in der Zeit}

Das ODE-System \eqref{eq:4.4} kann mit Diskretisierungsverfahren aus der Numerik 2 behandelt werden.
Sei $0 = t^0 < t^1 < \dotsb < t^k = T$, $\Delta t^k := t^{k+1} - t^k$ (Superskript $k$ bezeichnet im Folgenden einen Zeitindex, keine Potenz).

\begin{itemize}
	\item
		Finde $\underbar{u}_h^k = (u_{h,i}^k)_{i=1}^n \in \R^n$ für $k = 0, \dotsc, K$, sodass $\underbar{u}_h^k \approx \underbar{u}_h(t^k)$, wobei $\underbar{u}_h(t)$ DOF-Vektor als Lösung von $\eqref{eq:4.4}$.
		Dann ist bei nodaler Basis das Ziel
		\begin{math}
			u_{h,i}^k &\approx u(x_i, t^k) && \forall i,k
		\end{math}
		mit $u$ Lösung von \eqref{eq:4.1}, falls es Punktauswertungen erlaubt.
	\item
		Betrachte das implizite Euler-Verfahren:
		\begin{math}
			\underbar{u}_h'(t^{k+1}) &\approx \frac{\underbar{u}_h^{k+1} - \underbar{u}_h^k}{\Delta t^k}, &
			f_h^k &:= f_h(t^k),
		\end{math}
		also
		\begin{math}
			(M_h + \Delta t^k A_h)\underbar{u}_h^{k+1}
			= M_h \underbar{u}_h^k + \Delta t^k f_h^{k+1}.
		\end{math}
		Folglich ist zu jedem Zeitpunkt das Lösen eines LGS erforderlich.
	\item
		Explizites Euler-Verfahren:
		\begin{math}
			\underbar{u}_h'(t^k) := \frac{\underbar{u}_h^{k+1} - \underbar{u}_h^k}{\Delta t^k}
		\end{math}
		also
		\begin{math}
			M_h \underbar{u}_h^{k+1} = (M_h - \Delta t^k A_h) \underbar{u}_h^k + \Delta t^k f_h^k.
		\end{math}
		Zu $f_h^k$:
		falls $f_h$ keine Punktauswertung erlaubt oder $f_h$ zeitabhängig ist, ist Verbesserung durch Zeit-Mittelung erreichbar: $f_h^k := \frac{1}{\Delta t^k} \int_{t^k}^{t^{k+1}} f_h(t) \di[t]$.
	\item
		Die Konvergenz und Fehleranalysis aus Numerik 2 reicht nicht aus, weil
		\begin{enumerate}[1.]
			\item
				ODE-Dimension $n$ war als fest angenommen worden (in unserem Fall abhängig von Gitterweite $h$) und nur der Effekt von $\Delta t^k \to 0$ untersucht wird, also Approximation des semidiskreten Problems \eqref{eq:4.4}.
				Bei uns ist simultanes $\Delta t \to 0, h \to 0$ erforderlich, um Konvergenz gegen Lösung von \eqref{eq:4.1} zu erhalten.
			\item
				Wenn wir Standard-Fehleraussagen aus Numerik 2 verwenden würden zur Lösung des entsprechenden gewöhnlichen LGS
				\begin{math}
					\underbar{u}_h'(t) = \underbrace{M_h^{-1} A_h}_{B_h} \underbar{u}_h(t) + M_h^{-1} f_h(t),
				\end{math}
				taucht eine Lipschitz-Konstante der rechten Seite auf, also $\|B_h\|$, welche von negativen $h$-Potenzen abhängen kann.
				Folglich erhalten wir keine Konvergenzaussagen für $h \to 0$.
		\end{enumerate}
		Daher ist eine detailliertere Fehleranalysis erforderlich.
\end{itemize}

Sich ergebende Fragen wären
\begin{itemize}
	\item
		Welche Zeit-Integration ist zu empfehlen?
	\item
		Wie werden $\Delta t$ und $h$ sinnvoll gewählt?
\end{itemize}

\begin{df}[$\theta$-Verfahren für Wärmeleitung] \label{4.9}
	Sei $\theta \in [0,1], K \in \N, 0 = t^0 < \dotsb < t^K = T, \Delta t^k := t^{k+1} - t^k, V_h \subset H_0^1(\Omega)$ FEM-Raum mit Basis $\Set{\phi_i}_{i=1}^n$.
	Wir nennen $\Set{u_h^k}_{k=0}^K \subset V_h$ Lösung des $\theta$-Verfahrens, falls $u_h^k = \sum_{i=1}^n u_{h,i}^k \phi_i$ für DOF-Vektor $\underbar{u}_h^k = (u_{h,i}^k)_{i=1}^n$ und für $k=0, \dotsc, K$ gilt
	\begin{math}
		\big(M_h + \Delta t^k \theta A_h\big) \underbar{u}_h^{k+1} = \big(M_h - \Delta t^k (1-\theta) A_h ) \underbar{u}_h^k + \Delta t^k (1 - \theta) f_h^k + \Delta t^k \theta f_h^{k+1},
	\end{math}
	wobei $\underbar{u}_h^0$ geeignet gewählt.
	\begin{note}
		\begin{itemize}
			\item
				Wohldefiniertheit folgt, da jedes LGS eindeutig lösbar, denn $M_h + \Delta t A_h$ ist positiv definit und symmetrisch.
			\item
				Man erhält spezielle Verfahren durch Wahl von $\theta$:
				\begin{itemize}
					\item
						$\theta = 0$: explizites Euler-Verfahren,
					\item
						$\theta = 1$: implizites Euler-Verfahren,
					\item
						$\theta = \frac{1}{2}$: Trapez-Methode / Crank-Nicolson-Verfahren.
				\end{itemize}
			\item
				Man kann zeigen: $\theta$-Verfahren ist konsisten mit Ordnung $1$ bezüglich $t$, für $\theta = \frac{1}{2}$ sogar konsistent mit Ordnung 2.
				Idealerweise sehen wir diese Ordnungen in der zu behandelten Fehleranalysis.
		\end{itemize}
	\end{note}
\end{df}

\subsection*{Stabilität}

Wir können nun Stabilität des $\theta$-Verfahrens in verschiedenen Normen zeigen, z.B. zunächst einer diskreten „Raum-Zeit-Norm“:
\begin{math}
	\||\underbar{u}_h\||^2 := \|u_h^k\|_{L^2(\Omega)}^2 + \sum_{k=0}^{K-1} \Delta t^k |u_h^{k+\theta}|_{H^1}^2,
\end{math}
wobei $\underbar{u}_h = \Set{u_h^k}_{k=0}^K$ und $u_h^{k+\theta} := \theta u_h^{k+1} + (1- \theta) u_h^k$.

\begin{st}[Stabilität des $\theta$-Verfahrens in der Raum-Zeit-Norm] \label{4.10}
	Für $\frac{1}{2} \le \theta \le 1$ und $\theta$-Verfahren aus \ref{4.9} gilt
	\begin{math}[numbered] \label{eq:4.5}
		&\|u_h^k\|_{L^2}^2 + \sum_{k=0}^{K-1} \Delta t^k |u_h^{k+\theta}|_{H^1}^2 + (2\theta - 1) \|u_h^{k+1} - u_h^k\|_{L^2}^2 \\
		&\quad\le \|u_h^0\|_{L^2}^2 + C \sum_{k=0}^{K-1} \Delta t^k \|f^{k+\theta}\|_{L^2}^2,
	\end{math}
	wobei $u_h^{k+\theta} := \theta u_h^{k+1} - (1-\theta) u_{h}^{k}$, $f^{k+\theta} := (1-\theta) f^k + \theta f^{k+1}$ und $C := s^2$ mit $s := \diam(\Omega)$.

	Also gilt Stabilität bezüglich der Raum-Zeit-Norm
	\begin{math}[numbered] \label{eq:4.6}
		\|| \underbar{u}_h \||^2 \le \|u_h^0\|_{L^2}^2 + C T \max_{k=0,\dotsc, k} \|f^k\|_{L^2}^2,
	\end{math}
	falls $f \in L^\infty ([0,T], L^2)$.
	\begin{proof}
		In schwacher Form lautet das $\theta$-Verfahren:
		\begin{math}
			\<u_h^{k+1}0- u_h^k, \phi\>_{L^2} + \Delta t^k \<\Nabla u_h^{k+\theta}, \Nabla \phi\>_{L^2} &= \Delta t^k \< f^{k+\theta}, \phi\>, && \forall \phi \in V_h
		\end{math}
		Einsetzen von $u_h^{k+\theta}$ als Testfunktion ergibt
		\begin{math}[numbered] \label{eq:4.7}
			\<u_h^{k+1} - u_h^k, u_h^{k+\theta}\>_{L^2} + \Delta t^k \underbrace{\<\Nabla u_h^{k+\theta}, \Nabla u_h^{k+\theta}\>_{L^2}}_{=|u_h^{k+\theta}|_{H^1}^2} = \Delta t^k \<f^{k+\theta}, u_h^{k+2}\>_{L^2}.
		\end{math}
		Für den ersten Term gilt
		\begin{math}[numbered] \label{eq:4.8}
			\<u_h^{k+1} - u_h^k, u_h^{k+\theta}\>_{L^2}
			&= \<u_h^{k+1} - u_h^k, \theta u_h^{k+1} + (1-\theta) u_h^k\> \\
			&= \<u_h^{k+1} - u_h^k, \frac{1}{2} u_h^{k+1} + \frac{1}{2} u_h^k + (\theta - \frac{1}{2})(u_h^{k+1} - u_h^k) \>_{L^2} \\
			&= \frac{1}{2} \|u_h^{k+1}\|_{L^2}^2 - \frac{1}{2} \|u_h^k\|_{L^2}^2 + (\theta - \frac{1}{2})\|u_h^{k+1} - u_h^k\|_{L^2}^2.
		\end{math}
		Dies in \eqref{eq:4.7} ergibt mit Cauchy-Schwarz und Poincaré-Friedrich \ref{3.12} und Young-Ungleichung $ab \le \frac{a^2}{2\eps} + \frac{b^2}{2} \eps$, mit $\eps = 1$.
		\begin{math}
			&\|u_h^{k+1}\|_{L^2}^2 - \|u_h^k\|_{L^2}^2 + (2\theta - 1) \|u_h^{k+1} - u_h^k\|_{L^2}^2 + 2 \Delta t^k |u_h^{k+\theta}|_{H^1}^2 \\
			&\quad= 2\Delta t^k \< f^{k+\theta}, u_h^{k + \theta} \> \\
			&\quad\le 2 \Delta t^k \|f^{k+\theta}\|_{L^2} \|u_h^{k+\theta}\|_{L^2} \\
			&\quad\le 2 \Delta t^k \|f^{k+\theta}\|_{L^2} S|u_h^{k+\theta}|_{H^1} \\
			&\quad\le \Delta t^k s^2 \|f^{k+\theta}\|_{L^2}^2 + \Delta t^k |u_h^{k+\theta}|_{H^1}^2.
		\end{math}
		Also ist
		\begin{math}
			&\|u_h^{k+1}\|_{L^2}^2 - \|u_h^k\|_{L^2}^2 + (2\theta - 1) \|u_h^{k+1} - u_h^k\|_{L^2}^2 + \Delta t^k |u_h^{k+\theta}|_{H^1}^2 \\
			&\quad\le \Delta t^k s^2 \|f^{k+\theta}\|_{L^2}^2.
		\end{math}
		Summation über $k$ (Teleskopsumme) liefert die Behauptung \eqref{eq:4.5} mit $c := s^2$.
		Weil
		\begin{math}
			\|f^{k+\theta}\| &= \|\theta f^{k+1} - (1-\theta) f^k\| \\
			&\le \theta \|f^{k+1}\| + (1-\theta) \|f^k\| \\
			&\le \theta \max_{k'} \|f^{k'}\| + (1-\theta) \max_{k'} \|f^{k'}\| \\
			&= \max_{k'} \|f^{k'}\|
		\end{math}
		folgt \eqref{eq:4.6}.
	\end{proof}
	\begin{note}
		\begin{itemize}
			\item
				Für $\theta \in [\frac{1}{2}, 1]$ bleibt die diskrete Raum-Zeit-Norm also für $\Delta t^k \to 0, h \to 0$ beschränkt, Fehler werden nicht exponentiell verstärkt.
			\item
				Für $\theta < \frac{1}{2}$ ist das Verfahren nur stabil, falls $\Delta t$ hinreichend klein ist, ähnlich wie im nächsten Satz formuliert.
		\end{itemize}
	\end{note}
\end{st}

% fixme
%\begin{st}[Stabilität in $L^\infty([0,T], L^2)$] \label{4.11}
\begin{st} \label{4.11}
	Für $\theta \in [0, 1]$.
	Sei $\Set{u_h^k}_{k=0}^K$ Lösung des $\theta$-Verfahrens aus \ref{4.9} und $f \in L^\infty([0,T], L^2)$.
	Falls $\theta < \frac{1}{2}$, so gelte zusätzlich die Bedingung $(1 - \theta) \le \frac{h_{\text{min}}^2}{c^2}$, wobei $h_{\text{min}} = \min_{T \in \scr T_h} h_T$ und $c$ die optimale Konstante der inversen Abschätzung
	\begin{math}
		\|\Nabla v_h\|_{L^2(\Omega)} &\le \frac{c}{h_{\text{min}}} \|v_h\|_{L^2(\Omega)}, && \forall v_h \in V_h.
	\end{math}

	Dann ist das $\theta$-Verfahren stabil bezüglich $L^\infty([0,T], L^2(\Omega))$, d.h. $\forall k=0, \dotsc, K$ gilt
	\begin{math}
		\|u_h^k\|_{L^2} \le \|u_h^0\|_{L^2(\Omega)}  + C \|f\|_{L^\infty([0,T], L^2(\Omega))}
	\end{math}
	mit
	\begin{math}
		C := \begin{cases}
			\sqrt{\frac{s^2 T}{2}} & \theta \in [\frac{1}{2}, 1] \\
			\sqrt{\frac{s^2 T}{1 + \theta}} & \theta \in [0, \frac{1}{2})
		\end{cases},
	\end{math}
	wobei $s = \diam(\Omega)$.
	\begin{proof}
\Timestamp{2015-01-30}
		Wie in \ref{4.10} (Raum-Zeit-Norm-Stabilität) testen wir mit $u_h^{k+\theta}$ in schwacher Form und erhalten $\frac{1}{\Delta t} \cdot \eqref{4.7}$.
		\begin{math}
			\underbrace{\<\frac{u_h^{k+1} - u_h^k}{\Delta t^k}, u_h^{k+\theta}\>_{L^2} }_{T_1}+ \underbrace{|u_h^{k+\theta}|_{H^1}^2}_{T_2}
			= \underbrace{\<f^{k+\theta}, u_h^{k+\theta}\>_{L^2}}_{T_3}.
		\end{math}
		Für $T_1$ gilt $\frac{1}{\Delta t^k} \cdot \eqref{4.8}$
		\begin{math}
			T_1 = \frac{1}{2\delta t^k} \big(\|u_h^{k+1}\|_{L^2}^2 - \|u_h^k\|_{L^2}^2\big) + \frac{\theta - \frac{1}{2}}{\Delta t^k} \|u_h^{k+1} - u_h^k\|_{L^2}^2.
		\end{math}
		Für $T_3$ folgt mit Young $ab \le \frac{1}{2\eps} a^2 + \frac{\eps}{2} b^2$ für $\eps = 2$ (statt $\eps = 1$ wie in \ref{4.10}):
		\begin{math}
			T_3
			&\le \|f^{k+\theta}\|_{L^2} \|u_h^{k+\theta}\|_{L^2} \\
			&\le \|f^{k+\theta}\|_{L^2} s |u_h^{k+\theta}|_{H^1} \\
			&\le \frac{s^2}{4} \|f^{k+\theta}\|_{L^2}^2 + \underbrace{|u_h^{k+\theta}|_{H^1}^2}_{=T_2},
		\end{math}
		also insgesamt
		\begin{math}
			\frac{1}{2\Delta t^k} \big( \|u_h^{k+1}\|_{L^2}^2 - \|u_h^k\|^2\big) + \frac{\theta - \frac{1}{2}}{\Delta t^k} \|u_h^{k+1} - u_h^k\|_{L^2}^2
			\le \frac{s^2}{4} \|f^{k+\theta}\|_{L^2}^2.
		\end{math}
		\begin{seg}{$\theta \ge \frac{1}{2}$}
			In diesem Fall ist der zweite Term nichtnegativ, Multiplikation mit $2\Delta t^k$ liefert dann
			\begin{math}
				\|u_h^{k+1}\|_{L^2}^2 - \|u_h\|_{L^2}^2
				\le \frac{s^2\Delta t^k}{2} \|f^{k+\theta}\|_{L^2}^2
			\end{math}
			Wie im Beweis von \ref{4.10} gilt $\|f^{k+\theta}\|_{L^2} \le \esssup_{t \in [0,T]} \|f(\argdot, t)\|_{L^2} = \|f\|_{L^\infty([0,T],L^2(\Omega)}$ und mit der Teleskopsumme
			\begin{math}
				\underbrace{\|u_h^k\|_{L^2}^2}_{a^2}
				\le \underbrace{\|u_h^0\|_{L^2}^2}_{b^2} + \underbrace{\Big( \sum_{k'=0}^{k-1} \frac{s^2\Delta t^k}{2} \Big)}_{=\frac{s^2}{2}t^k \le \frac{s^2}{2}T = C^2} \underbrace{\|f\|_{L^\infty([0,T], L^2)}}_{d}.
			\end{math}
			Dann gilt die Behauptung auch „ohne Quadrate“, da für $a,b,c,C \ge 0$ gilt
			\begin{math}
				&& a^2 &\le b^2 + C^2 d^2 \\
				&\implies& a^2 &\le b^2 + C^2 d^2 + 2bdC = (b + Cd)^2 \\
				&\implies& a &\le b + Cd
			\end{math}
		\end{seg}
		\begin{seg}{$\theta < \frac{1}{2}$}
			Testen mit $\phi = u_h^{k+1}$ liefert
			\begin{math}
				\<u_h^{k+1} - u_h^k, u^{k+1}\> - \Delta t^k \<\underbrace{\Nabla u_h^{k+\theta}}_{\theta u_h^{k+1} + (1-\theta)u_h^k}, \Nabla u_h^{k+1}\>_{L^2}
				= \Delta t^k \<f^{k+\theta}, u_h^{k+1}\>
			\end{math}
			Mit $(a-b) a = \frac{1}{2} (a^2 + (a-b)^2 - b^2)$ und $ab = -\frac{1}{2}((a-b)^2 - a^2 - b^2)$ gilt
			\begin{math}
				&\frac{1}{2} \big( \|u_h^{k+1}\|^2 + \|u_h^{k+1} - u_h^k\|^2 - \|u_h^k\|^2\big)
				+ \Delta t^k \theta \|\Nabla u_h^{k+1}\|^2 \\
				&\qquad- \frac{1}{2} \Delta t^k (1-\theta) \big( \|\Nabla u_h^k - \Nabla u_h^{k+1}\|^2 - \| \Nabla u_h^k\|^2 - \|\Nabla u_h^{k+1}\|^2 \\
				&\quad\le \Delta t^k \|f^{k+\theta}\| \|u_h^{k+1}\| \\
				&\quad\le \frac{\Delta t^k s^2}{2\eps} \|f^{k+\theta}\|^2 + \frac{\Delta t^k}{2} \eps \|\Nabla u_h^{k+1}\|^2
			\end{math}
			Multipliziere mit $2$ und nutze $\|\Nabla (u_h^{k+1} - u_h^k)\|^2 \le \frac{c^2}{\|h_{\text{min}}^2\|} \|u_h^{k+1} - u_h^k\|^2$ und $\theta + \frac{1}{2} (1 - \theta) - \frac{\eps}{2} = \frac{1 + \theta - \eps}{2}$:
			\begin{math}
				&(\|u_h^{k+1}\|^2 - \|u_h^k\|^2) + (1 - \Delta t^k (1-\theta) \frac{c^2}{h_{\text{min}}^2}) \|u_h^{k+1} - u_h^k\|^2 \\
				&\qquad+ \Delta t^k (1 + \theta - \eps) \|\Nabla u_h^{k+1}\|^2 + \Delta t^k (1-\theta) \|\Nabla u_h^k\|^2 \\
				&\quad \le \frac{\Delta t^k}{\eps} s^2 \|f^{k+\theta}\|^2.
			\end{math}
			Mit $\eps := 1 + \theta$ verschwindet der dritte Term.
			Wegen der Voraussetzung an $\Delta t$ sind der zweite und vierte Term nichtnegativ, also durch $0$ abschätzbar.
			Summation liefert
			\begin{math}
				\|u_h^k\|^2
				\le \|u_h^0\|^2 + \frac{Ts^2}{1 + \theta} \|f\|_{L^\infty([0,T), L^2)}.
			\end{math}
			Die Behauptung gilt analog zum obigen Fall auch „ohne Quadrate“.
		\end{seg}
	\end{proof}
	\begin{note}
		\begin{itemize}
			\item
				Die Bedingung $\Delta t \le C_{\text{CFL}} h_{\text{min}}^2$ nennt man auch \emphdef{CFL-Bedingung} nach Courant, Friedrichs, Lewy und impliziert, dass man $\Delta t$ nicht beliebig groß wählen kann, sondern $\Delta t$ mit $h_{\text{min}}^2$ skalieren muss.
			\item
				Für $\theta \approx 0$ und kleine lokale Gitterweite $h_{\text{min}}$, dann muss $\Delta t$ extrem klein gewählt werden.
				Daher kann das zugehörige Verfahren impraktikabel sein, weil zu viele Zeitschritte erforderlich sind, um $T$ zu erreichen.
			\item
				Die Konstante in der inversen Abschätzung hängt von $\sigma_{\text{max}}$ ab, wobei $\sigma_T \le \sigma_{\text{max}}$ für alle $T \in \scr T_h$.
				Somit sollte die Gitterverfeinerung derart durchgeführt werden, dass Simplizes nicht degenerieren, d.h. $\sigma_T \le \sigma_{\text{max}}$ auch für verfeinerte Gitter.
		\end{itemize}
	\end{note}
\end{st}

\begin{st}[A-priori-Fehlerabschätzung für $\theta$-Verfahren] \label{4.12}
	Sei $\Omega \subset \R^d$ ein Lipschitz-Gebiet, sodass das Poisson-RWP $H^{m+1}(\Omega)$-regulär ist für ein $m \in \N$ und die Lösung $u$ des Wärmeleitungsgleichungs-ARWP genügend regulär ist, $V_h := \P_{m,0}(\scr T_h)$ für zulässige Triangulierung $\scr T_h$ mit $\sigma_{\text{max}} > 0, M > 0$, sodass $\sigma_T \le \sigma_{\text{max}}$, $\frac{h}{h_T} \le M$ für alle $T \in \scr T_h$.
	Sei $\Set{u_h^k}_{k=0}^K \subset V_h$ Lösung des $\theta$-Verfahrens für $\theta \in [0,1], \Delta t^k = \Delta t$.
	Falls $\theta \in [0,\frac{1}{2})$, gelte zusätzlich (wie in \ref{4.11}) $(1-\theta)\Delta t \le \frac{h_{\text{min}}^2}{c^2}$.
	Dann gilt
	\begin{math}
		&\|u_h^k - u(t^k)\|_{L^2(\Omega)} \\
		&\quad\le \|u_0 - u_h^0\|_{L^2}
		+ \frac{|2\theta - 1|}{2} \Delta t \|\partial_t^2 u\|_{L^\infty([0,T], L^2} \\
		&\qquad + \frac{2-\theta}{2} \Delta t^2 \max_{k'=0, \dotsc, k-1} \frac{1}{\Delta t} \int_{t^{k'}}^{t^{k' + 1}} \|\partial_t^3 u(s)\|_{L^2(\Omega)} \di[s] \\
		&\qquad+ C h^{m+1} \bigg( \|u_0\|_{H^{m+1(\Omega)}}
		+ \max_{k'=0,\dotsc, K-1} \frac{1}{\Delta t} \int_{t^{k'}}^{t^{k'+1}} \|\partial_t u(s)\|_{H^{m+1}(\Omega)} \bigg)
	\end{math}
	\begin{proof}
		Sei $P_h: H_0^1(\Omega) \to V_h$ Galerkin-Projektion, d.h. $\<\Nabla P_h u(t), \Nabla \phi\>_{L^2(\Omega)} = \<\Nabla u(t), \nabla \phi\>_{L^2(\Omega)}$ für alle $\phi \in V_h, t \in [0,T]$.
		Wir definieren $u^{k+\theta} := \theta u(t^{k+1}) + (1-\theta) u(t^k)$,  den Fehler $e_h^k = u_h^k - P_h u(t^k)$ und $f^{k+\theta} := \theta f(t^{k+1}) - (1-\theta)f(t^k)$.
		Die Definition des $\theta$-Verfahrens lautete
		\begin{math}[numbered] \label{eq:4.9}
			\frac{1}{\Delta t} \< u_h^{k+1} - u_h^k, \phi\>_{L^2} + \<\Nabla u_h^{k+\theta}, \Nabla \phi\>_{L^2} = \<f^{k+\theta}, \phi\>_{L^2}.
		\end{math}
		Mit der Definition der Galerkin-Projektion:
		\begin{math}[numbered] \label{eq:4.10}
			&\frac{1}{\Delta t} \< P_h u(t^{k+1}) - P_h u(t^k), \phi\>_{L^2} + \<\Nabla P_h u^{k+\theta}, \Nabla \phi\> \\
			&\quad= \frac{1}{\Delta t} \< P_h u(t^{k+1}) - P_h u(t^k), \phi\>_{L^2} + \<\Nabla u^{k+\theta}, \Nabla \phi\>_{L^2}.
		\end{math}
		Subtraktion von \eqref{eq:4.9} liefert ein $\theta$-Verfahren für $e_h^k$ mit $\eps_h^k$ als rechte Seite:
		\begin{math}
			\frac{1}{\Delta t} \<e_h^{k+1} - e_h^k, \phi\>_{L^2} + \<\Nabla e_h^{k+\theta}, \Nabla \phi\>_{L^2}
			= \<\eps_h^k, \phi\>_{L^2},
		\end{math}
		mit
		\begin{math}
			\<\eps_h^k, \phi\>
			= \<f^{k+\theta}, \phi\>_{L^2} - \frac{1}{\Delta t} \<P_h u(t^{k+1}) - P_h u(t^k), \phi\> - \< \Nabla u^{k+\theta}, \Nabla \phi\>_{L^2}.
		\end{math}
		Mit der schwachen Form der Wärmeleitungsgleichung gilt
		\begin{math}
			\<f^{k+\theta},\phi\>_{L^2} - \<\Nabla u^{k+\theta}, \Nabla \phi\>_{L^2}
			= \< \partial_t u^{k+\theta}, \phi\>_{L^2}
		\end{math}
		und wir erhalten
		\begin{math}[numbered] \label{eq:4.11}
			\eps_h^k = \partial_t u^{k+\theta} - \frac{1}{\Delta t} (P_hh u(t^{k+1}) - P_h u(t^k)).
		\end{math}
		Mit dem Stabilitätssatz folgt also
		\begin{math}
			\|e_h^k\|_{L^2(\Omega)}
			\le \|e_h^0\|_{L^2(\Omega)} + C \max_{k'=0,\dotsc, K} \|\eps_h^{k'}\|_{L^2(\Omega)}
		\end{math}
		Für den Anfangsfehler gilt mit $\Delta$-Ungleichung und Aubin-Nitsche
		\begin{math}
			\|e_h^0\|_{L^2(\Omega)}
			= \|u_h^0 - P_h u(t^0)\|_{L^2}
			\le \|u_h^0 - u_0\| + \underbrace{\|u_0 - P_h u(t^0)\|}_{\le C h^{m+1} \|u_0\|_{H^{m+1}(\Omega)}}.
		\end{math}
		Es verbleibt eine Schranke für $\|\eps_h^k\|$ zu bestimmen.
		Aus \eqref{eq:4.11} folgt durch Addition von $0$:
		\begin{math}
			\eps_h^k = \underbrace{\partial_t u^{k+\theta} - \frac{1}{\Delta t}(u(t^{k+1}) - u(t^k))}_{\eps_{h,1}^k} + \underbrace{\frac{1}{\Delta t} (\Id - P_h)(u(t^{k+1}) - u(t^k))}_{\eps_{h,2}^k}.
		\end{math}
		Für $\eps_{h,2}^k$ folgt mit Aubin-Nitsche:
		\begin{math}
			\|\eps_{h,2}^k\|_{L^2}
			&= \Big\|(\Id - P) \frac{1}{\Delta t} \int_{t^k}{t^{k+1}} \int_{t^k}^{t^{k+1}} \partial_t u(s) \di[s] \Big\|_{L^2} \\
			&\le \frac{1}{\Delta t} \int_{t^k}^{t^{k+1}} \|(\Id - P_h) \partial_t u(s) \|_{L^2} \di[s] \\
			&\le C h^{m+1}\frac{1}{\Delta t} \int_{t^k}^{t^{k+1}} \|\partial_t u(s) \|_{H^{m+1}} \di[s].
		\end{math}
		Für $\eps_{h,1}^k$ betrachte $2$ Taylor-Entwicklungen für $u(t)$:
		\begin{math}
			u(t^{k+1}) &= u(t^k) + \Delta t \partial_t u(t^k) + \frac{1}{2} \Delta t^2 \partial_t^2 u(t^k) + \frac{1}{2} \int_{t^k}^{t^{k+1}} (t^{k+1} - s)^2 \partial_t^3 u(s) \di[s] \\
			u(t^{k}) &= u(t^{k+1}) - \Delta t \partial_t u(t^{k+1}) + \frac{1}{2} \Delta t^2 \partial_t^2 u(t^{k+1}) - \frac{1}{2} \int_{t^k}^{t^{k+1}} (t^{k} - s)^2 \partial_t^3 u(s) \di[s]
		\end{math}
		Hieraus erhalten wir
		\begin{math}
			\frac{1}{\Delta t}(u(t^{k+1}) - u(t^k))
	 		&= \partial_t u(t^k) + \frac{1}{2} \Delta t \partial_t^2 u(t^k) + \frac{1}{2\Delta t}  \int_{t^k}^{t^{k+1}} (t^{k+1} -s)^2 \partial_t^3 u(s) \di[s] \\
			\frac{1}{\Delta t}(u(t^{k+1}) - u(t^k))
			&= \partial_t u(t^{k+1}) - \frac{1}{2} \Delta t \partial_t^2 u(t^{k+1}) + \frac{1}{2\Delta t}  \int_{t^k}^{t^{k+1}} (t^{k} -s)^2 \partial_t^3 u(s) \di[s].
		\end{math}
		Multiplikation der ersten Gleichung mit $(1-\theta)$ und der zweiten Gleichung mit $\theta$ und Addition liefert für $\eps_{h,1}^k$:d
		\begin{math}
			\eps_{h,1}^k &= \partial_t u^{k+\theta} + \frac{1}{\Delta t} (u(t^{k+1}) - u(t^k)) \\
			&= \partial_t u^{k+\theta} - \partial_t u^{k+\theta} - \frac{\Delta t}{2}\big( (1-\theta) \partial_t^2 u(t^k) - \theta \partial_t^2 u(t^{k+1}) \big) \\
			&\quad\underbrace{- \frac{1-\theta}{2\Delta t} \int_{t^k}^{t^{k+1}} \underbrace{(t^{k+1} - s)^2}_{\le \Delta t^2} \partial_t^3 u(s) \di[s] - \frac{\theta}{2\Delta t} \int_{t^k}^{t^{k+1}} \underbrace{(t^k - s)^2}_{\le \Delta t^2} \partial_t^3 u(s) \di[s]}_{=R^k}
		\end{math}
		Mit $\partial_t^2 u(t^k) = \partial_t^2 u(t^{k+1}) - \int_{t^k}^{t^{k+1}} \partial_t^3 u(s) \di[s]$ folgt
		\begin{math}
			\eps_{h, 1}^k
			&= - \frac{\Delta t^k}{2} \Big( (1-\theta - \theta) \partial_t^2 u(t^{k+1}) - (1-\theta) \int_{t^k}^{t^{k+1}} \partial_t^3 u(s) \di[s] \Big) + R^k \\
			&= \frac{\Delta t}{2} (2\theta - 1) \partial_t^2 u(t^{k+1}) + \frac{\Delta t}{2} (1- \theta) \int_{t^k}^{t^{k+1}} \partial_t^3 u(s) \di[s] + R^k
		\end{math}
		Es folgt mit der Definition von $R^k$ für die Norm:
		\begin{math}
			\|\eps_{h,1}^k\|
			\le \frac{|2\theta - 1|}{2} \Delta t \|\partial_t^2 u(t^k)\|_{L^2} + \underbrace{\big( \frac{1-\theta}{2} + \frac{1-\theta}{2} + \frac{\theta}{2}\big)}_{=\frac{2-\theta}{2}} \Delta t^2 \frac{1}{\Delta t} \int_{t^k}^{t^{k+1}} \|\partial_t^3 u(s)\| \di[s].
		\end{math}
		Die Gesamtfehlerschranke folgt aus der Dreieckungleichung:
		\begin{math}
			\|u(t^k) - u_h^k\|
			\le \|u(t^k) - P_h u(t^k)\| + \underbrace{\|P_h u(t^k) - u_h^t\|}_{= e_h^k}.
		\end{math}
		Der letzte Term ist bereits abgeschätzt, für den ersten Term gilt mit Aubin-Nitsche und $u(t^k) = u(0) + \int_{0}^{t^k} \partial_t u(s) \di[s]$.
		\begin{math}
			\|u(t^k) - P_h u(t^k)\|
			&\le C h^{m+1} \| u(t^k\|_{H^{m+1}} \\
			&\le C h^{m+1} \bigg( \| u(0)\|_{H^{m+1}} + \int_0^{t^k} \|\partial_t u(s)|_{H^{m+1}} \bigg).
		\end{math}
	\end{proof}
\end{st}





