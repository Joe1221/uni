\chapter{Approximation für parabolische Probleme} \label{chap:4}

\Timestamp{2014-01-20}

\paragraph{Motivation:}

Betrachte die instationäre Wärmeleitungsgleichung:
\begin{math}[numbered] \label{eq:4.1}
	\partial_t u - \Laplace u &= f &&\text{in $\Omega \times (0,T)$} \\
	u &= g && \text{auf $\partial \Omega \times (0,T)$} \\
	u(\argdot, 0) &= u_0.
\end{math}
Das Vorgehen zur FEM-basierten Approximation ist wie folgt:
\begin{itemize}
	\item
		Definiere geeignete „schwache Lösung“ für \eqref{4.1}.
	\item
		Führe die FEM Ortsdiskretisierung durch.
	\item
		Führe die FD Zeitdiskretisierung durch.
	\item
		Untersuche das Verfahren auf Stabilität, Konvergenz und Fehleraussagen.
\end{itemize}

Erweiterung auf allgemeinere parabolische Probleme ist analog möglich, wird hier jedoch nicht behandelt.

Zunächst einige Definitionen und Aussagen ohne Beweis (in diesen Fällen siehe Evans, §5.9.2 und Appendix E)

\begin{df}[Bochner-Räume] \label{4.1}
	Sei $X$ ein Banachraum mit Norm $\|\argdot\|_X$ und $T > 0$.
	Wir definieren für alle messbaren Funktionen $u: [0,T] \to X$
	\begin{math}
		\|u\|_{L^p([0,T], X)} &:= \Big( \int_0^T \|u(t)\|_X^p \di[t] \Big)^{\frac{1}{p}}, && 1 \le p < \infty, \\
		\|u\|_{L^\infty([0,T], X)} &:= \esssup_{t\in[0,T]} \|u(t)\|_{X}.
	\end{math}
	Für $1 \le p \le \infty$ definieren wir die \emphdef{Bochner-Räume} als
	\begin{math}
		L^p([0,1], X) := \Set{ u:[0,T] \to X \text{ messbar} & \|u\|_{L^p([0,1], X)} < \infty }.
	\end{math}
	\begin{note}
		\begin{itemize}
			\item
				Für eine sinnvolle Definition von $\|u\|_{L^p([0,T], X)}$, dem sogenannten \emphdef{Bochner-Integral}, ist genauer die \emphdef{Bochner-Messbarkeit} erforderlich.
		\end{itemize}
	\end{note}
\end{df}

\begin{ex*}
	\begin{itemize}
		\item
			Für $\Omega \subset \R^d$ offen und beschränkt gilt
			\begin{math}
				L^2([0,T), L^2(\Omega)) &= L^2([0,T] \times \Omega), \\
				L^2([0,T], H^1(\Omega)) &= \Set{ u \in L^2([0,T] \times \Omega) & \Nabla_x u \in L^2([0,T] \times \Omega) }.
			\end{math}
		\item
			$L^2([0,T], X)$ ist Hilbertraum, falls $X$ Hilbertraum via kanonischem Skalarprodukt:
			\begin{math}
				\<u, v\>_{L^2([0,T], X)} := \int_0^T \< u(t), v(t) \>_{X}.
			\end{math}
	\end{itemize}
\end{ex*}

% FIXME: \begin{df}[Raum $C([0,T], X)$] \label{4.2}
\begin{df} \label{4.2}
	Sei $X$ ein Banachraum mit Norm $\|\argdot\|_{X}$.
	Wir definieren für stetiges $u: [0,T] \to X$
	\begin{math}
		\|u\|_{C([0,T], X)} := \max_{0 \le t \le T} \|u(t)\|_{X}
	\end{math}
	und damit
	\begin{math}
		C([0,T], X) := \Set{u : [0,T] \to X \text{ stetig} & \|u\|_{C([0,T], X)} < \infty }.
	\end{math}
	\begin{note}
		\begin{itemize}
			\item
				Analog definiert man den Raum der stetig differenzierbaren Funktionen $C^1([0,T], X)$.
		\end{itemize}
	\end{note}
\end{df}

\begin{df}[Schwache Zeitableitung] \label{4.3}
	Sei $u \in L^1([0,T], X)$.
	Wir nennen $v \in L^1([0,T], X)$ \emphdef{schwache Zeitableitung} von $u$, falls
	\begin{math}
		\int_0^T u(t) \phi'(t) \di[t]
		&= - \int_0^T v(t) \phi(t) \di[t],
		&& \forall \phi \in C_0^\infty([0,T]).
	\end{math}
	Wir schreiben dann $u'$ statt $v$.
\end{df}

\begin{st}[Einbettung] \label{4.4}
	Sei $u \in L^2([0,T], H_0^1(\Omega))$ mit schwacher Ableitung $u \in L^2([0,T], H^{-1}(\Omega))$.
	Dann gilt
	\begin{enumerate}[i)]
		\item
			$u \in C([0,T], H_0^1(\Omega))$, d.h. es existiert ein stetiger Repräsentant.
		\item
			die Abbildung $t \mapsto \|u(t)\|_{L^2(\Omega)}^2$ ist absolut stetig und es gilt
			\begin{math}
				\ddx[t] \|u(t)\|_{L^2(\Omega)}^2
				= 2 \< u'(t), u(t) \>_{L^2(\Omega)}
			\end{math}
			für fast alle $t \in [0,T]$.
		\item
			Es gilt die Abschätzung
			\begin{math}
				\|u\|_{C([0,T], L^2(\Omega))}
				\le C \Big( \|u\|_{C([0,T], H_0^1(\Omega)} + \|u'\|_{L^2}([0,T], H^{-1}(\Omega)) \Big).
			\end{math}
			Dabei hängt die Konstante $C$ nur von $T$ ab.
	\end{enumerate}
	\begin{proof}
		Siehe Evans, Thm. 3 in §5.9.
	\end{proof}
	\begin{note}
		\begin{itemize}
			\item
				Ist eine Verallgemeinerung der Einsicht aus §3, dass $v \in H^1(\R)$, d.h. eine Funktion mit schwacher Ableitung global stetig ist.
			\item
				Es wird statt $u \in L^2([0,T], H_0^1(\Omega))$ hier allgemeiner $u' \in L^2([0,T], H^{-1}(\Omega))$ erlaubt.
				Erinnerung aus \ref{chap:3} zu Bemerkung nach Satz \ref{3.22} (Stetigkeit von $l$):
				es gilt $\int_\Omega fv < \infty$, falls $f, v \in L^2(\Omega)$, aber für $v \in H_0^1(\Omega)$ sind auch Funktionen $f \not\in L^2$ erlaubt, also
				\begin{math}
					„H^{-1}(\Omega)“ \supsetneq L^2(\Omega) \supsetneq H_0^1(\Omega).
				\end{math}
				Die rechte Inklusion ist klar, die linke ist eine etwas laxe Notation für eine geeignete Einbettung
				\begin{math}
					f &\in L^2
					&& \leadsto &
					\tilde f(v) &:= \int_\Omega fv
				\end{math}
				und dann $f \in H^{-1}(\Omega)$.

				Ähnlich ist es hier: $u'(t) \in H^{-1}(\Omega)$ ist in dem Sinne zu verstehen, dass $u'(t)$ kein Funktional, sondern eine Funktion auf $\Omega$ ist und
				\begin{math}
					l(v) := \int_\Omega u'(t,x) v(x) \di[x]
				\end{math}
				erfüllt $l \in H^{-1}(\Omega)$.
				Dann ist insbesondere $\int_\Omega u'(t,x) v(x) \di[x] < \infty$ für $v \in H_0^1(\Omega)$ sinnvoll definiert und kann in Definition der schwachen Lösung verwendet werden.
		\end{itemize}
	\end{note}
\end{st}

\begin{df}[schwache Lösung für Wärmeleitung] \label{4.5}
	Sei $\Omega \subset \R^d$ Lipschitz-Gebiet, $T > 0$.
	Die Funktion $u: \Omega \times [0,T] \to \R$ heißt \emphdef{schwache Lösung} des ARWP der Wärmeleitungsgleichung mit Dirichlet-Nullrandwerten zu $v_0 \in L^2(\Omega)$ und $f \in L^2([0,T], H^{-1}(\Omega))$ falls
	\begin{enumerate}[i)]
		\item
			\emph{Regularität}: $u \in L^2([0,T], H_0^1(\Omega))$ und $u' \in L^2([0,T], H^{-1}(\Omega))$,
		\item
			\emph{PDE}:
			Für alle $\phi \in H_0^1(\Omega)$ und fast alle $t \in (0,T)$ gilt
			\begin{math}[numbered] \label{eq:4.2}
				\<u'(t), \phi\>_{L^2(\Omega)} + \< \Nabla u(t), \Nabla \phi \>_{L^2(\Omega)}
				= \<f(t), \phi\>_{L^2(\Omega)},
			\end{math}
		\item
			\emph{Anfangswerte}: $u(\argdot, 0) = u_0$.
	\end{enumerate}
	\begin{note}
		\begin{itemize}
			\item
				Die Definiton der Anfangswerte ergibt Sinn, da mit i) und \ref{4.4} i) $u$ stetig ist.
			\item
				Inhomogene Dirichlet-Randwerte können wie in \ref{chap:3} behandelt werden (Lösen eines homogenen Problems, Addition der Randwerte).
		\end{itemize}
	\end{note}
\end{df}

\begin{st}[Existenz und Eindeutigkeit] \label{4.6}
	Für ein ARWP in schwacher Form gemäß \ref{4.5} existiert eine eindeutige schwache Lösung
	\begin{proof}
		Siehe z.B. G. Dziuk: Theorie und Numerik partieller Differentialgleichungen,  De Gruyter Verlag, Berlin, 2000.
	\end{proof}
\end{st}

Es folgt die Diskretisierung im Ort, was auf ein Anfangswertproblem eines Systems gewöhnlicher Differentialgleichungen für die DOFs führt.
Wähle einen endlichdimensionalen FEM-Approximationsraum $V_h \subset H_0^1(\Omega)$ und führe Galerkin-Projektion der schwachen Form \eqref{eq:4.2} auf $V_h$ durch.

\begin{df}[FEM Semidiskretisierung] \label{4.7}
	Sei $\scr T_h$ zulässige Triangulierung, $V_h := \P_{k,0}(\scr T_h)$.
	Wir nennen $u_h \in C^1([0,T], V_h)$ Lösung der semidiskreten Wärmeleitung, falls
	\begin{math}[numbered] \label{eq:4.3}
		\<u_h'(t), \phi\>_{L^2(\Omega)} + \<\Nabla u_h(t), \Nabla \phi\>_{L^2(\Omega)} &= \<f(t), \phi\>_{L^2(\Omega)}, && \forall \phi \in V_h, t \in (0,T), \\
		u_h(0) &= u_h^0,
	\end{math}
	mit geeignetem $u_h^0 \in V_h$.
\end{df}

\begin{kor}[AWP für Koeffizientenvektor] \label{4.8}
	Sei $\Set{\phi_i}_{i=1}^n$ eine nodale Basis von $V_h$ zu Knoten $\Set{x_i}_{i=1}^n \subset \_\Omega$,
	\begin{math}[numbered] \label{eq:4.4}
		u_h(t) = \sum_{i=1}^n u_{h,i}(t) \phi_i
	\end{math}
	mit DOF-Vetor $\underbar{u}_h = (u_{h,i})_{i=1}^n$.
	Dann ist $u_h$ Lösung des semidiskreten Problems \eqref{eq:4.3} genau dann, wenn $\underbar{u}_h: [0,T] \to \R^n$ Lösung des AWP
	\begin{math}[numbered] \label{eq:4.5}
		M_h \underbar{u}_h'(t) + A_h \underbar{u}_h(t) &= f_h(t), \\
		\underbar{u}_h(0) = u_h^0
	\end{math}
	mit Massematrix $M_h = (\<\phi_i, \phi_j\>_{L^2(\Omega)})_{i,j = 1}^n$, Steifigkeitsmatrix $A_h = (\<\Nabla \phi_i, \Nabla \phi_j\>_{L^2(\Omega)})_{i,j=1}^n$ und rechte Seite $f_h(t) = (\int_\Omega f(t) \phi_i)_{i=1}^n$.
	\begin{proof}
		Klar durch Einsetzen von Ansatz \eqref{4.4} in die PDE in semidiskreter Form.
	\end{proof}
	\begin{note}[Wahl der $u_0$-Approximation]
		\begin{itemize}
			\item
				Man kann z.B. $u_h^0$ als orthogonale $L^2$-Projektion $u_h^0 := P_{V_h} u_0$ wählen.
				Dies lässt sich als LGS schreiben:
				\begin{math}
					u_h^0 &= P_{V_h} u_0
					&&\iff & \int_\Omega ( u_h^0 - u_0) \phi_i &= 0, \forall i \\
					& &&\iff & \int_\Omega \underbrace{u_h^0}_{=\sum_{j} u_{h,j}^0 \phi_j} \phi_i &= \underbrace{\int_\Omega u_0 \phi_i}_{u_{0,i}}, \forall i \\
					& &&\iff & M_h \underbar{u}_h^0 &= \underbar{u}_0
				\end{math}
				mit $\underbar{u}_0 := (\int_\Omega u_0 \phi_i )_{i=1}^n$.
		\end{itemize}
	\end{note}
\end{kor}
