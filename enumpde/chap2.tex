\chapter{Finite Differenzen Verfahren für elliptische Probleme} \label{chap:2}



\section{Finite Differenzen für Poisson-Gleichung}


\begin{df}[Finite Differenz] \label{2.1}
	Sei $h \in \R$, $e_j \in \R^d$ Einheitsvektor für $j = 1, \dotsc, d$ und $u: \Set{x, x \pm e_j h & j = 1, \dotsc, d} \to \R$, dann definieren wir \emphdef[Vorwärtsdifferenz]{Vorwärts-} oder \emphdef{rechtsseitige Differenz} durch
	\[
		(\partial_{x_j}^{+h}u)(x) := \frac{u(x+he_j) -u(x)}{h}
	\]
	analog \emphdef[Rückwärtsdifferenz]{Rückwärts-} oder \emphdef{linksseitige Differenz} durch
	\[
		(\partial_{x_j}^{-h}u)(x) := \frac{u(x) - u(x-he_j)}{h}
	\]
	und \emphdef[symmetrische Differenz]{symmetrische} oder \emphdef{zentrale Differenz} durch
	\[
		(\partial_{x_j}^{ch} u)(x)
		= \f 12 (\partial_{x_j}^{+h} u(x) + \partial_{x_j}^{-h} u(x))
		= \frac{u(x+he_j) - u(x-he_j)}{2h}
	\]
\end{df}


\begin{st}[Approximationsgüte] \label{2.2}
	Sei $u: \Omega \to \R, x \in \Omega \subset \R^d, r \in \N^+$ mit $B_r(x) \subset \Omega$.	
	Für $h < r$ gilt dann
	\begin{enumerate}[i)]
		\item
			$|\partial_{x_j} u(x) - \partial_{x_j}^{\pm h} u(x)| \le \f h2 \|\partial_{x_j}^2 u\|_\infty$ für $u \in C^2(\_\Omega)$
		\item
			$|\partial_{x_j} u(x) - \partial_{x_j}^{c h} u(x)| \le \f {h^2}6 \|\partial_{x_j}^3 u\|_\infty$ für $u \in C^3(\_\Omega)$
		\item
			$|\partial_{x_j}^2 u(x) - \partial_{x_j}^{-h} \partial_{x_j}^{+h} u(x)| \le \f {h^2}{12} \|\partial_{x_j}^4 u\|_\infty$ für $u \in C^4(\_\Omega)$
	\end{enumerate}
	\begin{proof}
		Es genügt dies für $d = 1$ zu zeigen, denn  sei $u \in C^k(\_\Omega), v(t) = u(x + th e_j)$.
		Dann ist $\|\ddx[t^k] v(t)\|_\infty \le \|\partial_{x_j}^k u\|_{C^0(\_\Omega)}$.
		\begin{enumerate}[i)]
			\item
				Taylor liefert für ein $\xi \in (x, x + h)$
				\[
					u(x+h) = u(x) + hu'(x) + \f {h^2}2 u''(\xi)
				\]
				Es folgt
				\[
					\partial_x u(x) - \partial_x^{+h} u(x)
					= u'(x) - \f{u(x+h) - u(x)}{h}
					= - \f h2 u''(\xi).
				\]
				Analog für die linksseitige Differenz.
			\item
				Subtraktion von
				\begin{align*}
					u(x+h) &= u(x) + hu'(x) + \f {h^2}2 u''(x) + \f {h^3}6 u'''(\xi) \\
					u(x-h) &= u(x) - hu'(x) + \f {h^2}2 u''(x) - \f {h^3}6 u'''(\_\xi) \\
				\end{align*}
				mit $\xi \in (x,x+h), \_\xi \in (x-h,x)$ liefert
				\[
					u(x+h) - u(x-h) = 2h u'(x) + \f{h^3}6 \big(u'''(\xi) + u'''(\_\xi)\big),
				\]
				also
				\[
					u'(x) - \f{u(x+h) - u(x-h)}{2h}
					= - \f{h^2}{12} \big( u'''(\xi) + u'''(\_\xi) \big)
					\le \f{h^2}6 \|u'''\|_\infty.
				\]
			\item
				Addition von
				\begin{align*}
					u(x+h) &= u(x) + hu'(x) + \f{h^2}2 u''(x) + \f {h^3}6 u'''(x) + \f{h^4}{24} u''''(\xi) \\
					-2u(x) &= -2u(x) \\
					u(x+h) &= u(x) - hu'(x) + \f{h^2}2 u''(x) - \f {h^3}6 u'''(x) + \f{h^4}{24} u''''(\_\xi) \\
				\end{align*}
				mit $\xi \in (x,x+h), \_\xi \in (x-h, x)$ liefert
				\begin{align*}
					\partial_x^{-h} \partial_x^{+h} u(x)
					&= \dfrac{\frac{u(x+h)-u(x)}{h}-\frac{u(x)-u(x-h)}{h}}{h} \\
					&= \frac{u(x+h) - 2u(x) +  u(x+h)}{h^2} \\
					&= \f 1{h^2} \big( \f {h^2}2 + \f {h^2}2 \big) u''(x) + \f 1{h^2} \f {h^4}{24} \big( u''''(\xi) + u''''(\_\xi) \big)
				\end{align*}
		\end{enumerate}
	\end{proof}
	\begin{note}
		\begin{itemize}
			\item
				Die Approximation in iii) ist also eine zweite zentrale Differenz
				\[
					\partial_{x_j}^{-h} \partial_{x_j}^{+h} u(x)
					= \partial_{x_j}^{+h} \partial_{x_j}^{-h} u(x)
					= \partial_{x_j}^{c, \f h2} \partial_{x_j}^{c, \f h2} u(x)
					= \f{u(x+h) - 2u(x) + u(x-h)}{h^2}.
				\]
			\item
				Aus dem Beweis folgt, dass $\partial_x^{-h} \partial_{x}^{+h} u(x) = u''(x)$ falls $u^{(4)} = 0$, z.B. für $u \in \P_3$.
			\item
				Man kann zentrale Differenzen für höhere Ableitungen verallgemeinern:
				\[
					\partial_{x_j}^{h,m} u(x)
					:= (\partial_{x_{j}}^{c, \f h2})^m u(x)
				\]
				falls $u: \Set{u+(k-\f m2) h e_j & k = 0, \dotsc, m } \to \R$.
				Dann ist
				\[
					\partial_{x_j}^{h,m} u(x)
					= \f 1{h^m} \sum_{k=0}^m \binom{m}{k} (-1)^{k+m} u\big( u + (k-\f m2) h e_j \big).
				\]
		\end{itemize}
	\end{note}
\end{st}

\begin{kor}[FD-Approximation für Laplace] \label{2.3} 
	Sei $u : \Set{x, x \pm h e_j} \to \R$.
	Dann definieren
	\begin{equation} \label{eq:2.1}
		\Laplace_h u(x) :=
		\Big(\sum_{i=1}^d \partial_{x_j}^{-h} \partial_{x_j}^{+h} u \Big) (x)
	\end{equation}
	und es gilt unter Voraussetzungen von \ref{2.2}
	\[
		|\Laplace u(x) - \Laplace u(x)| \le C h^2
	\]
	für $u \in C^4(\_\Omega)$.
	\begin{proof}
		Dreiecksungleichung und \ref{2.2} iii) liefert
		\begin{align*}
			|\Laplace u(x) - \Laplace_h u(x)|
			&= \Big| \sum_{i=1}^d \partial_{x_j}^2 u(x) - \partial_{x_j}^{-h} \partial_{x_j}^{+h} u(x) \Big| \\
			&\le \sum_{i=1}^d | \partial_{x_j}^2 u(x) - \partial_{x_j}^{-h} \partial_{x_j}^{+h} u(x) \Big| \\
			&\le \sum_{j=1}^d \f {h^2}{12} \|\partial_{x_j}^{(4)} u\|_\infty \\
			&\le d \f {h^2}{12} \|u\|_{C^4(\_\Omega)}.
		\end{align*}
	\end{proof}
	\begin{note}
		\begin{itemize}
			\item
				Für $p(x) := \prod_{i=1}^d p_i(x_i)$ mit $p_i \in \P_3$ ist $\Laplace_h$ exakt, d.h. $\Laplace_h p(x) = \Laplace p(x)$.
		\end{itemize}
	\end{note}
\end{kor}


\begin{df}[Würfelgebiet] \label{2.4}
	Sei $\Omega \subset \R^d$ offen, beschränkt.
	$\Omega$ heißt \emphdef{Würfelgebiet} zu $h \in \R^+$, falls $Z \subset \Z^d$ sodass $\Omega = W \setminus \Boundary W =: \mathring W$ mit $W := \bigcup_{z \in Z} W(z)$ und $W(z) := [z_1h , (z_1+1)h] \times \dotsc \times [z_d h, (z_d+1)h] \subset \R^d$.
	\begin{note}
		\begin{itemize}
			\item
				Ist $\Omega$ ein Würfelgebiet zu $h$, dann ist $\Omega$ auch ein Würfelgebiet zu $\f hn$ für alle $n \in \N$.
		\end{itemize}
	\end{note}
\end{df}

Wir wollen uns im Folgenden nur mit Würfelgebieten beschäftigen

\begin{df}[FD-Gitter] \label{2.5}
	Sei $\Omega \subset \R^d$ ein Würfelgebiet zu $h \in \R^+$, $\Gamma := \Boundary \Omega$, also $\_\Omega = \Omega \cup \Gamma$.
	Wir definieren das \emphdef{Gitter} $\_\Omega_h$ durch \emphdef{innere Punkte} $\Omega_h$ und \emphdef{Randpunkt} $\Gamma_h$, wobei
	\begin{align*}
		\Omega_h &:= \Set{ x \in \Omega & \exists z \in \Z^d : x = hz } \\
		\Gamma_h &:= \Set{ x \in \Gamma & \exists z \in \Z^d : x = hz } \\
		\_\Omega_h &:= \Omega_h \cup \Gamma_h.
	\end{align*}
	\begin{note}
		\begin{itemize}
			\item
				Jeder innere Punkt hat genau $2d$ Nachbarn im Abstand von $h$ in $\_\Omega_h$
			\item
				Erweiterung für allgemeine Gebiete später.
		\end{itemize}
	\end{note}
\end{df}

\begin{df}[Gitterfunktionen] \label{2.6}
	Zu einem Gitter $\_\Omega_h$ definieren wir den Raum der \emphdef{Gitterfunktionen} $X_h := \Set{ v : \_\Omega_h \to \R }$
	und den Teilraum der Funktionen mit Nullrandwerten $X_h^0 := \Set{ v \in X_h & \forall x \in \Gamma_h : v(x) = 0 } \subset X_h$
	und den Raum der Funktionen auf inneren Punkten $Y_h := \Set{v: \Omega_h \to \R}$ mit Maximumsnorm $\|v\|_{\_\Omega_h} := \max_{x\in \_\Omega_h} |v(x)|$ und $\|v\|_{\Omega_h} := \max_{x\in\Omega_h} |v(x)|$.
\end{df}

\begin{nt*}[Nebenbemerkungen]
	\begin{itemize}
		\item
			Also sind $(X_h, \|\argdot\|_{\_\Omega_h}), (X_h^0, \|\argdot\|_{\Omega_h}), (X_h^ , \|\argdot\|_{\_\Omega_h}), (Y_h, \|\argdot\|_{\Omega_h})$ Banachräume, weil endlichdimensional und damit vollständig.
		\item
			Man kann auch $X_h$ mit einer Hilbertraumstruktur versehen, indem man das \emphdef[diskretes $l_2$-Skalarprodukt]{diskrete $L_2$-Skalarprodukt} definiert:
			\[
				\<u,v\>_{l_2} := h^d \sum_{x\in\_\Omega_h} u(x) v(x),
			\]
			welches die Norm $\|u\|_{l_,} := \sqrt{h^d \sum_{x\in\_\Omega} u(x)^2}$ induziert.
			Dann ist $X_h$ auch vollständig bezüglich $\|\argdot\|_{l_2}$, weiter gilt: $\lim_{h\to 0} \|u\|_{l_2} = \|u\|_{L^2(\Omega)}$ für $u \in C^0(\Omega)$.
		\item
			Man kann $X_h$ auch mit einer Seminorm versehen, welche auch die Ableitungen miteinbezieht
			\[
				|u|_{h_1}
				:= \Big( h^d \sum_{x\in\Omega} \sum_{j=1}^d  \big(\partial_{x_j}^{+h} u(x)\big)^2 \Big)^{\f 12},
			\]
			die \emphdef{diskrete $h_1$ Seminorm}.
			Dies ist eine Norm auf $X_h^0$, aber nicht auf $X_h$.
			Damit erhält man durch Kombination mit der diskreten $l_2$-Norm eine Norm auf $X_h$ („diskrete $h_1$ Norm“):
			\[
				\|u\|_{h_1}
				:= \sqrt{\|u\|_{l_2}^2 + |u|_{h_1}^2},
			\]
			bezüglich welcher $X_h$ ein Hilbertraum ist.
	\end{itemize}
\end{nt*}

Für $v \in X$ ist mit \eqref{eq:2.1} der Operator $-\Laplace_h v(x)$ für $x \in \Omega_h$ wohldefiniert, d.h. wir können $\Laplace_h : X_h \to Y_h$ als linearen Operator sehen.

\begin{df}[FD-Approximation für Poisson-RWP] \label{2.7}
	Sei ein Gitter $\_\Omega_h$ gegeben.
	Dann nennen wir $u_h \in X_h$ \emphdef{Finite Differenzen Lösung} des Poisson-RWPs aus \ref{1.23}, falls
	\begin{align} \label{eq:2.2}
		\Laplace_h u_h(x) &= f(x) && \text{$x \in \Omega_h$} \\
		u_h(x) &= g(x) && \text{$x\in \Gamma_h$}. \notag
	\end{align}
\end{df}

\begin{nt*}[Berechnung via LGS]
	\begin{itemize}
		\item
			Lege eine Aufzählung $\Set{x_1, \dotsc, x_n} = \Omega_h$ fest.
			Dann ist \eqref{eq:2.2} äquivalent zu einem LGS für Unbekannte $\underbar{u}_h = (u_i)_{i=1}^n$ mit $u_i = u_h(x_i)$ für $i=1,\dotsc, n$, denn $u_h(x)$ für $x \in \Gamma$ ist schon festgelet durch $g$.
		\item
			Sei FD-Operator in $x_i \in \Omega_h$ gegeben durch
			\[
				\Laplace_h u(x_i) = \sum_{j=1}^n \alpha_{ij} u(x_j) + \sum_{x\in \Gamma_h} \beta_{ix} u(x).
			\]
			Dann ist das LGS gegeben durch $A_h \underbar{u}_h = b_h$ mit $(A_h)_{ij} = \alpha_{ij}$ und $(b_h)_i = f(x_i) - \sum_{x\in \Gamma_h} \beta_{ix} g(x)$.
		\item
			$A_h$ ist dünn besetzt (sparse), da sie nur sehr wenige nichtnull-Einträge pro Zeile enthält. 
	\end{itemize}
\end{nt*}

\Timestamp{2014-10-31}

\begin{ex} \label{2.8}
	Sei $d=1, \Omega = (0,1), n \in \N, h := \f 1{n+1}, x_i := i h, i \in \Set{0, \dotsc, n+1}$.
	Dann ist $\Omega_h := \Set{x_1, \dotsc, x_n}, \Gamma_h := \Set{x_0, x_{n+1}}$.
	Betrachte die Poisson-Gleichung
	\begin{align*}
		-u''(x) &= f(x) && \text{in $\Omega$} \\
		u(0) &= \alpha \\
		u(1) &= \beta
	\end{align*}
	Sei $u_i \approx u(x_i)$ für $i=1, \dotsc, n$, $u_0 = \alpha, u_{n+1} = \beta$.
	Diskretisierung ergibt
	\[
		- \f{u_{i+1} - 2u_i + 2u_{i-1}}{h^2} = f(x_i)
	\]
	für $i = 1, \dotsc, n$.
	Das LGS ergibt sich als
	\[
		\overbrace{\f 1{h^2} \underbrace{\Matrix{2 & -1 & & \\ -1 & \ddots & \ddots & \\ & \ddots & \ddots & -1 \\ & & -1 & 2}}_{\tilde A_n}}^{A_h}
		\Vector{u_1 & \dots & u_n}
		= \Vector{f(x_1) + \f{\alpha}{h^2} & f(x_2) & \dots & f(x_{n-1} & f(x_n) + \f{\beta}{h^2})}.
	\]
	\begin{note}
		\begin{itemize}
			\item
				$A_h$ ist tridiagonal, symmetrisch, $A_h = \f 1{h^2} \tilde A_n$
			\item
				$A_h$ ist regulär, denn per Induktion folgt $\det \tilde A_{n} = n + 1$:
				\begin{proof}
					Für $n = 1$ ist $\det(\tilde A_n) = \det(2) = n+1$.
					Die Aussage gelte für $n-1$, es gilt
					\begin{align*}
						\det \tilde A_n
						&= \det \Matrix{\tilde A_{n-1} & & \\ & & -1 \\ & -1 & 2} \\
						&= 2 \det \tilde A_{n-1} - (-1) \det \Matrix{\tilde A_{n-2} & & \\ & & & \\ & -1 & -1} \\
						&= 2 n + (-1) \det A_{n-2}
						= 2n - (n-1)
						= n+1
					\end{align*}
				\end{proof}
			\item
				$A_h$ ist positiv definit, denn Gerschgorin liefert $\lambda_i(\tilde A_n) \in [0,4]$ und wegen Regularität $\lambda_i(\tilde A_n) \neq 0$.
				Es folgt also $\lambda_i > 0$ für $i = 1, \dotsc, n$.
			\item
				Also existiert eine eindeutige FD-Lösung für das Poisson-RWP in einer Dimension.
			\item
				Wegen $A_h$ symmetrisch, positiv definit kann das CG oder das PCG Verfahren zum iterativen Lösen des LGS verwendet werden.
			\item
				Man kann hier auch direkt stetige Abhängigkeit von den Daten beweisen:
				Seien $u, u_h$ Lösungen zu $f, \alpha, \beta$, bzw. $\_f, \_\alpha, \_\beta$.
				Dann existiert $C > 0$ unabhängig von $f, \_f, \alpha, \_\alpha, \beta, \_\beta$ sodass
				\[
					\|u_h - \_u_h\| \le C \Big( \|f-\_f\| + |\alpha - \_\alpha| + |\beta - \_\beta| \Big).
				\]
		\end{itemize}
	\end{note}
\end{ex}

\begin{ex} \label{2.9}
	Sei $d = 2, \Omega = (0,1)^2$ und betrachte das Poisson-RWP mit $g(x) = 0$, $m \in \N, h:= \f 1{m+1}, n := m^2$.
	Statt Einzelindex ist ein Doppelindex übersichtlicher, setze dazu
	\begin{align*}
		x_{ij} = (ih, jh) \quad \text{für $0 \le i,j \le m+1$}, &
		u_{ij} \approx u(x_{ij}).
	\end{align*}
	Die Diskretisierung des RWP liefert für jeden inneren Punkt eine Gleichung
	\[
		\f 1{h^2} \Big( 4 u_{ij} - u_{i-1,j} - u_{i+1,j} - u_{i,j-1} - u_{i,j+1}\Big)
		= f(ih, jh).
	\]
	Am Rand gilt
	\[
		u_{0,j} = u_{m+1,j} = u_{j,0} = u_{j,m+1}
	\]
	für $0 < i,j < m$.
	Für jede beliebige Wahl einer Aufzählung der $\Set{u_{ij}}$ erhält man ein System $A_h u_h = b_h$ mit $A_h$ symmetrisch, $\f 4{h^2}$ auf der Diagonalen und $-\f 1{h^2}$ an bis zu $4$ Einträgen pro Zeile/Spalte.

	Falls eine lexikographische Aufzahlung gewählt wird:
	\[
		\_u_h = \Big(u_{11}, u_{12}, \dotsc, u_{1m}, u_{21}, \dotsc, u_{2m}, \dotsc, u_{mm} \Big)
	\]
	so erhält $A_h$ eine Bandstruktur, jedoch nicht mehr tridiagonal wie in \ref{2.8}, sondern „Block-tridiagonal“:
	\begin{align*}
		A_h &= \f 1{h^2} \Matrix{B & C & & \\C & \ddots & \ddots & \\ & \ddots & \ddots & C \\ & & C & B}, &
		B &= \Matrix{4 & -1 & & \\ -1 & \ddots & \ddots & \\ & \ddots & \ddots & -1 \\ & & -1 & 4}, &
		C &= \Matrix{-1 & & \\ & \ddots & \\ & & -1},
	\end{align*}
	\begin{note}[Beliebige Gebiete]
		\begin{itemize}
			\item
				Falls $\Omega$ beschränkt ist, aber kein Würfelgebiet, muss das Gitter modifiziert werden, indem Schnittpunkte von $\Gamma$ mit den $h \Z^d$ Würfelkanten hinzugenommen werden:
				\[
					\Gamma_h := \Set{ x \in \Gamma & \exists z\in \Z^d, j =1,\dotsc, d : x \in z + \R e_j }
				\]
				Dann wird wie gehabt $\_\Omega_h := \Omega_h \cup \Gamma_h$ mit $\Omega_h$ aus \ref{2.5} definiert.
			\item
				% Rand in Süd/West richtung
				Koeffizienten der FD-Diskretisierung werden angepasst.
				Die Taylor-Entwicklung liefert
				\begin{align*}
					u''(x) &= \f{2}{h_W(h_O + h_W} u(x-h_W) - \f 2{h_Oh_W} u(x) + \f 2{h_O(h_O+h_W)} u(x+h_0) \\
					&\quad + \LandauO(h) \\
					\Laplace(u) &= \f{2}{h_W(h_O+h_W)} u(x-e_1h_W) - \f 2{h_Oh_N} u(x) + \f 2{h_O(h_O+h_W)} u(x+h_0 e_1) \\
					&\quad + \f 2{h_S(h_N+h_S)} u(x-e_2 h_S) - \f 2{h_S h_N} u(x) + \f 2{h_N(h_S+h_N)} u(x+ e_2 h_N) \\
					&\quad + \LandauO(h)
				\end{align*}
				„Shortley-Weller Approximation“.
		\end{itemize}
	\end{note}
	\begin{note}[Andere Randbedingungen]
		Neben Dirichlet- können auch andere Randebedingungen realisiert werden, z.B. Neumann-Randbedingungen:
		\begin{align*}
			(\Nabla u) \cdot h &= g_N
			&& \text{auf $\Gamma_N \subset \Gamma$}.
		\end{align*}
		Wir nennen $\Gamma_N$ \emphdef{Neumann-Randteil}.
		Wir nehmen an, dass $x$ auf einer Kante liegt und nicht auf einer Ecke von $\Gamma$ (sonst ist keine Normale $n$ definiert).
		Sei $n = \pm e_j$ (da Würfelgebiet) äußerer Normalenvektor für ein $j = 1, \dotsc, d$.
		Wir Approximieren $\Nabla u$ durch
		\[
			(\Nabla_h u)_i :=
			\begin{cases}
				\partial_{x_i}^{c,h} u & i \neq j \\
				\partial_{x_j}^{-h} u & i = j
			\end{cases}.
		\]
		Nun sind $u_h(x), x \in \Gamma_h \cap \Gamma_N$ Unbekannte.
		Es fällt eine weitere Gleichung für das LGS an:
		\[
			(\Nabla_h u_h(x)) \cdot n = g_N(x)
		\]
		für $x \in \Gamma_h \cap \Gamma_N$.
	\end{note}
\end{ex}


\section{Allgemeine Elliptische PDEs zweiter Ordnung}

\begin{df}[Allgemeine Elliptische RWP] \label{2.10}
	Zu $\Omega \subset \R^d$ beschränkt, $f \in C^0(\Omega), g \in C^0(\Boundary \Omega)$ sei
	\begin{equation} \label{eq:2.3}
		(\scr L u)(x)
		= - \sum_{i,j=1}^d a_{ij}(x) \partial_{x_j} \partial_{x_i} u(x)
		+ \sum_{i=1}^d b_i(x) \partial_{x_i} u(x) + c(x) u(x)
	\end{equation}
	gleichmäßig elliptisch und $a_{ij}, b_i, c \in C^0(\_\Omega)$.
	Gesucht ist $u \in C^2(\Omega) \cap C^0(\_\Omega)$.
	\begin{align*}
		\scr L u(x) &= f(x) && \text{$x \in \Omega$}, \\
		u(x) &= g(x) && \text{$x\in\Gamma$}.
	\end{align*}
\end{df}

\begin{df}[FD-Approximation] \label{2.11}
	Für $\scr L$ aus \eqref{eq:2.3}, $x \in \Omega$ mit $\_B_h(x)  \subset \_\Omega$ definiere
	\begin{align*}
		\scr L_h u(x)
		&:= - \sum_{i=1}^d a_{ii} (x) \partial_{x_i}^{-h} \partial_{x_i}^{+h} u(x) \\
		&= - \sum_{\substack{i,j=1 \\ i\neq j}} a_{ij} (x) \partial_{x_i}^{c,h} \partial_{x_j}^{c,h} u(x)
		+ \sum_{i=1}^d b_i(x) \partial_{x_i}^{c,h} u(x)
		+ c(x) u(x).
	\end{align*}
	\begin{note}
		Wir werden sehen, dass
		\begin{itemize}
			\item
				\ref{2.11} nur unter weiteren Annahmen an $a_{ij}, b_i, c, h$ eine „stabile Diskretisierung“ ergibt,
			\item
				Eiene etwas sorgfältigere Diskretisierung des Hauptteils eine erweiterte Klasse von Problemen stabil diskretisiert.
		\end{itemize}
	\end{note}
\end{df}

\begin{st}[FD-Approximationsfehler für $\scr L_h$] \label{2.12}
	Sei $u \in C^4(\_\Omega), x \in \Omega$ sodass $x + \sum_{i=1}^d \sigma_i e_i h \in \_\Omega$ für $\sigma_i = \Set{0,+1, -1}$, $i = 1, \dotsc, d$.
	Dann existiert ein $C$ (unabhängig von $x,h$) sodass
	\[
		\big|\scr L u(x) - \scr L_h u(x)\big| \le C h^2.
	\]
	\begin{proof}
		Taylor analog zu \ref{2.2} und \ref{2.3}.
	\end{proof}
\end{st}

\begin{note}[FD-Stern]
	\begin{itemize}
		\item
			Die Diskretisierung $\scr L_h$ kann man anschaulischer notieren: z.B. für $d = 2$.
			Falls $\scr L_h u(x_1, u_2) = \f 1{h^2} \sum_{i,j=-m}^m \alpha_{ij} u(x_1 + ih, x_2 + jh)$ dann ist
			\begin{equation} \label{eq:2.4}
				\Matrix[{
					\alpha_{-m,m} & \hdots & \alpha_{0,m} & \hdots & \alpha_{m,m} \\
					\vdots && \vdots && \vdots \\
					\alpha_{-m, 0} & \hdots & \alpha_{0,1} & \hdots & \alpha_{m,0} \\
					\vdots && \vdots && \vdots \\
					\alpha_{-m,-m} & \hdots & \alpha_{0,-m} & \hdots & \alpha_{m,-m}
				}_*
			\end{equation}
			Für $\scr L_h = -\Laplace_h$ aus \ref{2.3} ergibt sich für $m = 1$ der \emphdef{5-Punkt-FD-Stern}
			\[
				\Matrix[{ 0 & -1 & 0 \\ -1 & 4 & -1 \\ 0 & -1 & 0 }_*
			\]
			Für $\scr L_h$ aus \ref{2.11} (ohne Einschräkung $a_{12} = a_{21}$):
			\[
				\f 12 \Matrix[{ a_{12}(x) & - 2 a_{22}(x) & - a_{12}(x) \\ -2 a_{11}(x) & 4(a_{11}(x) + a_{22}(x)) & -2a_{11}(x) \\ -a_{12}(x) & -2 a_{22}(x) & a_{12}(x)}_*
				+ \f h2 \Matrix[{ 0 & b_2(x) & 0 \\ -b_1(x) & 0 & b_1(x) \\ 0 & -b_2(x) & 0 }_*
				+ h^2 \Matrix[{0 & 0 & 0 \\ 0 & c(x) & 0 \\ 0 & 0 & 0}_*.
			\]
		\item
			Für $m = 1$ (also $3\times 3$ Sterne) ist höchstens Approximationsordnung 2 erreichbar, wie in \ref{2.12} und \ref{2.3} für spezielle $\scr L_h$ gesehen
		\item
			Für $m > 1$ sind bessere Approximationsordnungen erreichbar
	\end{itemize}
\end{note}

\begin{df}[FD-Approximation für elliptisches RWP] \label{2.13}
	Sei $\Omega$ Würfelgebiet zu $h \in \R^+$, $\_\Omega_h$ zugehöriges Gitter.
	Dann ist $u_h \in X_h$ FD-Lösung falls
	\begin{align*}
		\scr L_h u_h(x) &= f(x) && \text{$x \in \Omega_h$}, \\
		u_h(x) &= g(x) && \text{$x\in \Gamma_h$}.
	\end{align*}
\end{df}

\begin{df}[Diskreter Zusammenhang] \label{2.14}
	Ein Gitter $\_\Omega_h$ heißt \emphdef{diskret zusammenhängend} falls es für alle $x,y \in \Omega_h$ (innere Punkte) eine Punktfolge $z_i \in \Omega_h$, $i = 0, \dotsc, k$ gibt mit $z_0 = x, z_k = y$ und $|z_i - z_{i-1}| = h$ für $i = 1, \dotsc, k$.
	\begin{note}
		Wenn zu grob gesampled wird, kann es vorkommen, dass $\Omega$ zwar zusammenhängend ist, $\Omega_h$ jedoch nicht.
		Falls $\Omega$ zusammenhängend, ist jedoch für hinreichend kleines $h$ auch $\_\Omega_h$ diskret zusammenhängend.
	\end{note}
\end{df}

\begin{lem}[Sternlemma] \label{2.15}
	Sei $k \ge 1$, $\Set{\alpha_i}_{i=0}^k \subset \R$ und $\Set{p_i}_{i=0}^n \subset \R$ gegeben und es gelte
	\begin{enumerate}[i)]
		\item
			$\alpha_0 > 0$ und $\alpha_i < 0$ für $i = 1, \dotsc, k$2
		\item
			$\sum_{i=0}^k \alpha_i = 0$,
		\item
			$\sum_{i=0}^k \alpha_i p_i \le 0$.
	\end{enumerate}
	Dann folgt aus $p_0 \ge \max_{i=1,\dotsc, k} p$ die Gleichheit $p_0 = p_1 = \dotsb = p_k$.
	\begin{proof}
		Es gilt
		\[
			\sum_{i=1}^k \alpha_i(p_i-p_0)
			= \sum_{i=0}^k \alpha_i (p_i-p_0)
			= \underbrace{\sum_{i=0}^k \alpha_i p_i}_{\le 0} - \underbrace{p_0 \sum_{i=0}^k \alpha_i}_{=0}
			\le 0.
		\]
		Wegen $p_0 \ge p_i, \alpha_i < 0$ für $i = 1, \dotsc, k$ sind die Summanden links nicht-negativ, also null und es folgt $p_i = p_0$.
	\end{proof}
\end{lem}

\begin{st}[Diskretes Maximumsprinzip] \label{2.16}
	Sei $u_h \in X_h$ eine FD-Lösung zu dem RWP \ref{2.13} mit $f(x) \le 0$ für $x \in \Omega_h$.
	Der Differenzenstern \eqref{eq:2.4} zu $m = 1$ (also ein $3\times 3$ Stern) erfülle in allen Punkten
	\begin{enumerate}[i)]
		\item
			$\sum_{i,j=-1}^1 \alpha_{i,j} = 0$,
		\item
			$\alpha_{0,0} > 0$,
		\item
			$\forall (i,j) \neq (0,0) : \alpha_{i,j} \le 0$,
		\item
			$\alpha_{1,0}, \alpha_{0,1}, \alpha_{0,-1}, \alpha_{-1,0} < 0$, „Koeffizienten in Hauprichtungen negativ“.
	\end{enumerate}
	Dann gilt
	\[
		\max_{x\in\_\Omega_h} u_h(x) = \max_{x\in \Gamma_h} u_h(x).
	\]
\end{st}
