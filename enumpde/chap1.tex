\chapter{Grundlagen}

\section{Definition / Notationen}

\begin{df}[Multiindex und partielle Ableitung] \label{1.1}
	Sei $u: \R^d \to \R$ genügend oft differenzierbar.
	Wir nennen $\beta = (\beta_1, \dotsc, \beta_d)^T \in \N_0^d$ mit $|\beta| := \sum_{i=1}^d \beta_i = k$ einen \emphdef{Multiindex der Ordnung $k$}.
	Wir definieren
	\[
		\partial^\beta u := (\pddx[x_1])^{\beta_1} (\pddx[x_d])^{\beta_d} u,
	\]
	die \emphdef{partielle Ableitungen} von $u$ zum Index $\beta$.
	Sei $\mathbb{B}_k := \Set{\beta \in \N_0^d & |\beta| = k}$ die Menge der Multiindizes der Ordnung $k$ und
	\[
		\D^k u := (\partial^\beta u)_{\beta \in \mathbb{B}_k}
	\]
	der Vektor aller partieller Ableitungen der Ordnung $k$ (mit beliebiger Reihenfolge).
\end{df}

\begin{df}[Ableitungsoperatoren]
	Für $u: \R^d \to \R$ hinreichend oft differenzierbar definieren wir den \emphdef[Gradient]{Gradienten}
	\[
		\grad u(x) := \nabla u(x) := \Vector{ \partial_{x_1} u(x) & \dots & \partial_{x_d} u(x) }
	\]
	für $x = (x_1, \dotsc, x_d)$ und $\partial_{x_i} := \pddx[x_i], i = 1, \dotsc, d$.
	Für ein hinreichend oft differenzierbares Vektorfeld $v: \R^d \to \R^d$ definiere die \emphdef{Divergenz}
	\[
		\div v(x) = \nabla \cdot v(x) = \sum_{i=1}^d \partial_{x_i} v_i (x)
	\]
	und für $d = 3$ zusätzlich die \emphdef{Rotation}
	\[
		\rot v(x) = \nabla \times v(x) = \Vector{ \partial_{x_2} v_3 - \partial_{x_3} v_2 & \partial_{x_3} v_1 - \partial_{x_1} v_3 & \partial_{x_1} v_2 - \partial_{x_2} v_1 }.
	\]
	Mit Abkürzung $\partial_{x_i}^2 := (\partial_{x_i})^2$ definiere den \emphdef{Laplace-Operator}
	\[
		\Laplace u(x) := \nabla \cdot (\nabla u) = \div( \grad  u(x) ) = \sum_{i=1}^d \partial_{x_i}^2 u(x).
	\]
	Skalare Operatoren werden für vektorielle Funktionen komponentenweise definiert:
	\[
		\Laplace v(x) := \Vector{\Laplace v_1(x) & \dots & \Laplace v_d(x)},
	\]
	und für $b \in \R^d$
	\[
		(b \cdot \nabla) v := (\sum_{i=1}^d b_i \partial_{x_i}) v 
		= \Vector { \sum_{i=1}^d b_i \partial_{x_i} v_1 & \dots & \sum_{i=1}^d b_i \partial_{x_i} v_d }.
	\]
\end{df}

\begin{df}[Räume stetig differenzierbarer Funktionen] \label{1.3}
	Sei $\Omega \subset \R^d$ offen, beschränkt.
	Wir bezeichnen mit $C^m(\_\Omega, \R^n)$ den Raum der $m$-mal stetig differenzierbaren Funktionen (sodass die $m$-ten Ableitungen stetig auf $\_\Omega$ fortsetzbar sind).
	Für $n = 1$ schreiben wir $C^m(\_\Omega) = C^m(\_\Omega, \R^1)$ und definiere für $u \in C^0(\_\Omega)$ die \emphdef{Supremumsnorm}
	\[
		\|u\|_\infty = \sup_{x\in \_\Omega} = |u(x)|
	\]
	und damit die Norm auf $C^m(\_\Omega)$:
	\[
		\|u\|_{C^m(\_\Omega} := \sum_{|\beta| \le m} \|\partial^\beta u\|_\infty
	\]
	für $u \in C^m(\_\Omega)$.
	\begin{note}
		\begin{itemize}
			\item
				$C^m(\_\Omega)$ ist ein Banachraum, d.h. ein vollständiger, normierter Raum (Alt, Lemma 1.8)
			\item
				Man kann auch $C^m(\Omega)$ für offenes oder potentiell unbeschränktes $\Omega$ und auch $m = \infty$ definieren.
				Statt einer Norm wird dann eine Metrik (Frechét-Metrik) eingeführt, bzgl. $C^m(\Omega)$ immernoch vollständig ist (Alt, Abschnitt 1.6).
		\end{itemize}		
	\end{note}
\end{df}

\begin{df}[$L^p$-Räume] \label{1.4}
	Für $p \in [1, \infty)$ definieren wir
	\[
		\tilde L^p(\Omega) := \Set{ u: \Omega \to \R \text{ Lebesgue-messbar} & \Big(\int_\Omega |u|^p \Big)^{\f 1p} < \infty}
	\]
	mit Seminorm
	\[
		\|u\|_p := \|u\|_{L^p(\Omega)}  := \Big(\int_\Omega |u|^p \Big)^{\f 1p}.
	\]
	Für $p = \infty$ definieren wir
	\[
		\tilde L^\infty(\Omega) := \Set{ u: \Omega \to \R \text{ Lebesgue-messbar} & \esssup_{x\in \Omega} |u(x)| < \infty }
	\]
	mit $\|u\|_\infty := \|u\|_{L^\infty(\Omega)} := \esssup_{x\in \Omega} |u(x)|$.
	Sei $\sim$ die Äquivalenzrelation auf $\tilde L^p(\Omega)$ via
	\[
		u \sim v :\iff \exists N \subset \Omega \text{ Nullmenge} : \forall x \in \Omega \setminus \N : u(x) = v(x).
	\]
	Dann definieren wir $L^p(\Omega)$ als die Menge der Äquivalenzklassen
	\[
		L^p(\Omega) := \tilde L^p / \sim.
	\]
	$\|u\|_p$ ist auf jeder einzelnen Äquivalenzklasse konstant und ist daher auch auf $L^p(\Omega)$ erweiterbar.
	Wir nennen $(L^p(\Omega), \|\argdot\|_p)$ \emphdef{normierter $L^p$-Raum}.
	\begin{note}
		\begin{itemize}
			\item
				$L^p(\Omega)$ ist vollständig bezüglich $\|\argdot\|_p$, also ein Banachraum (Alt, Lemma 1.1, Satz 1.14),
			\item
				Elemente von $L^p(\Omega)$ sind also Äquivalenzklassen von Funktionen, die sich nur auf einer Nullmenge unterscheiden.
				Konsequente Unterscheidung zwischen Funktionen und Äquivalenzklassen wäre mühsam.
				Praktische Konvention: $u \in L^p(\Omega)$ soll heißen $u \in U \in L^p(\Omega)$ für geeignete Äquivalenzklasse $U$ mit $u: \Omega \to \R$ als Repräsentant von $U$.
			\item
				Wir nennen $L^p(\Omega)$ trotzdem \emph{Funktionenraum}.
				Beim Arbeiten mit $L^p$-Räumen muss jedoch immer bedacht/hinterfragt werden, ob Operationen sinnvoll definiert sind, d.h. unabhängig vom Repräsentanten.
				Beispielsweise ist die Punktauswertung $u(x)$ nicht wohldefiniert.
		\end{itemize}
	\end{note}
\end{df}


