\chapter{Grundlagen}

\section{Definitionen und Notationen}

\begin{df}[Multiindex und partielle Ableitung] \label{1.1}
	Sei $u: \R^d \to \R$ hinreichend oft differenzierbar.
	Wir nennen $\beta = (\beta_1, \dotsc, \beta_d)^T \in \N_0^d$ mit $k := |\beta| := \sum_{i=1}^d \beta_i$ einen \emphdef[Multiindex]{Multiindex der Ordnung $k$}.

	Wir definieren
	\[
		\partial^\beta u := (\pddx[x_1])^{\beta_1} \dotsb (\pddx[x_d])^{\beta_d} u,
	\]
	die \emphdef[partielle Ableitung]{partielle Ableitung von $u$ zum Index $\beta$}.

	Sei $\mathbb{B}_k := \Set{\beta \in \N_0^d & |\beta| = k}$ die Menge aller Multiindizes der Ordnung $k$ und
	\[
		\D^k u := (\partial^\beta u)_{\beta \in \mathbb{B}_k}
	\]
	der Vektor aller partieller Ableitungen der Ordnung $k$ (mit beliebiger Reihenfolge).
\end{df}

\begin{df}[Ableitungsoperatoren] \label{1.2}
	Für $u: \R^d \to \R$ hinreichend oft differenzierbar definieren wir den \emphdef[Gradient]{Gradienten}
	\[
		\grad u(x) := \nabla u(x) := \Vector{ \partial_{x_1} u(x) & \dots & \partial_{x_d} u(x) },
	\]
	wobei $x = (x_1, \dotsc, x_d)$ und $\partial_{x_i} := \pddx[x_i]$ für $i = 1, \dotsc, d$.

	Für ein hinreichend oft differenzierbares Vektorfeld $v: \R^d \to \R^d$ definieren wir die \emphdef{Divergenz} durch
	\[
		\div v(x) := \nabla \cdot v(x) = \sum_{i=1}^d \partial_{x_i} v_i (x)
	\]
	und im Fall $d = 3$ zusätzlich die \emphdef{Rotation} durch
	\[
		\rot v(x) := \nabla \times v(x) = \Vector{ \partial_{x_2} v_3 - \partial_{x_3} v_2 & \partial_{x_3} v_1 - \partial_{x_1} v_3 & \partial_{x_1} v_2 - \partial_{x_2} v_1 }.
	\]
	Wir nutzen die Abkürzung $\partial_{x_i}^2 := (\partial_{x_i})^2$ und definieren den \emphdef{Laplace-Operator} durch
	\[
		\Laplace u(x) := \nabla \cdot (\nabla u) = \div( \grad  u(x) ) = \sum_{i=1}^d \partial_{x_i}^2 u(x).
	\]
	Skalare Operatoren werden für vektorielle Funktionen komponentenweise definiert, z.B.
	\[
		\Laplace v(x) := \Vector{\Laplace v_1(x) & \dots & \Laplace v_d(x)},
	\]
	und für $b \in \R^d$
	\[
		(b \cdot \nabla) v := \Big(\sum_{i=1}^d b_i \partial_{x_i}\Big) v
		= \Vector { \sum_{i=1}^d b_i \partial_{x_i} v_1 & \dots & \sum_{i=1}^d b_i \partial_{x_i} v_d }.
	\]
\end{df}

\begin{df}[Räume stetig differenzierbarer Funktionen] \label{1.3}
	Sei $\Omega \subset \R^d$ offen und beschränkt.
	Wir bezeichnen mit $C^m(\_\Omega, \R^n)$ den Raum der $m$-mal stetig differenzierbaren Funktionen (differenzierbar auf $\Omega$, sodass die $m$-ten Ableitungen stetig auf $\_\Omega$ fortsetzbar sind) von $\_\Omega$ nach $\R^n$.

	Für $n = 1$ schreiben wir auch kurz $C^m(\_\Omega) = C^m(\_\Omega, \R^1)$ und definieren hier für $u \in C^0(\_\Omega)$ die \emphdef{Supremumsnorm}
	\[
		\|u\|_\infty := \sup_{x\in \_\Omega} = |u(x)|.
	\]
	Auf $C^m(\_\Omega)$ definieren wir damit eine Norm:
	\[
		\|u\|_{C^m(\_\Omega)} := \sum_{|\beta| \le m} \|\partial^\beta u\|_\infty,
	\]
	wobei $u \in C^m(\_\Omega)$.
	\begin{note}
		\begin{itemize}
			\item
				$C^m(\_\Omega)$ ist ein Banachraum, d.h. ein vollständiger, normierter Raum (Alt, Lemma 1.8)
			\item
				Man kann auch $C^m(\Omega)$ für offenes oder potentiell unbeschränktes $\Omega$ und auch $m = \infty$ definieren.
				Statt einer Norm wird dann eine Metrik (Frechét-Metrik) eingeführt, bzgl. der $C^m(\Omega)$ immernoch vollständig ist (Alt, Abschnitt 1.6).
		\end{itemize}
	\end{note}
\end{df}

\begin{df}[$L^p$-Räume] \label{1.4}
	Für $p \in [1, \infty)$ definieren wir
	\[
		\tilde L^p(\Omega) := \Set{ u: \Omega \to \R \text{ Lebesgue-messbar} & \big(\mathsmaller{\int}_\Omega |u|^p \big)^{\f 1p} < \infty}
	\]
	mit Seminorm
	\[
		\|u\|_p := \|u\|_{L^p(\Omega)}  := \Big(\int_\Omega |u|^p \Big)^{\f 1p}.
	\]
	Für $p = \infty$ definieren wir
	\[
		\tilde L^\infty(\Omega) := \Set{ u: \Omega \to \R \text{ Lebesgue-messbar} & \esssup_{x\in \Omega} |u(x)| < \infty }
	\]
	mit $\|u\|_\infty := \|u\|_{L^\infty(\Omega)} := \esssup_{x\in \Omega} |u(x)|$.

	Sei $\sim$ die Äquivalenzrelation auf $\tilde L^p(\Omega)$ via
	\[
		u \sim v \defiff \exists N \subset \Omega \text{ Nullmenge} : \forall x \in \Omega \setminus \N : u(x) = v(x).
	\]
	Dann definieren wir $L^p(\Omega)$ als die Menge der Äquivalenzklassen
	\[
		L^p(\Omega) := \tilde L^p(\Omega) / \sim.
	\]
	$\|u\|_p$ ist auf jeder einzelnen Äquivalenzklasse konstant und ist daher ohne weiteres auf $L^p(\Omega)$ erweiterbar.
	Wir nennen $(L^p(\Omega), \|\argdot\|_p)$ \emphdef[$L^p$-Raum]{normierter $L^p$-Raum}.
	\begin{note}
		\begin{itemize}
			\item
				$L^p(\Omega)$ ist vollständig bezüglich $\|\argdot\|_p$, also ein Banachraum (Alt, Lemma 1.1, Satz 1.14),
			\item
				Elemente von $L^p(\Omega)$ sind also Äquivalenzklassen von Funktionen, die sich nur auf einer Nullmenge unterscheiden.
				Konsequente Unterscheidung zwischen Funktionen und Äquivalenzklassen wäre mühsam.
				Daher folgende praktische Konvention: $u \in L^p(\Omega)$ soll heißen $u \in U \in L^p(\Omega)$ für eine geeignete Äquivalenzklasse $U$ mit $u: \Omega \to \R$ als Repräsentant von $U$.
			\item
				Wir nennen $L^p(\Omega)$ trotz der Äquivalenzklassen einen \emph{Funktionenraum}.
				Beim Arbeiten mit $L^p$-Räumen muss jedoch immer bedacht/hinterfragt werden, ob die betrachteten Operationen sinnvoll definiert sind, d.h. unabhängig vom Repräsentanten sind.
				Beispielsweise ist die Punktauswertung $u(x)$ nicht wohldefiniert, eine Mittelung über alle Funktionswerte jedoch schon.
\Timestamp{2014-10-17}
			\item
				$(u,v) := \<u,v\>_{L^(\Omega)} := \int_\Omega uv$ ist ein Skalarprodukt auf $L^2(\Omega)$ und $\|u\|_2 = \sqrt{(u,v)}$, also $L^2(\Omega)$ ist vollständig bezüglich einer aus einem Skalarprodukt induzierten Norm, also ein sogenannter \emphdef{Hilbertraum}.
			\item
				Zu einem Banachraum $V$ ist der \emph{Dualraum} $V'$
				\[
					V' := \Set{ \phi: V \to \R & \phi \text{ linear und stetig} }
				\]
				mit der induzierten Norm
				\[
					\|\phi\|_{V'} := \sup_{u\in V\setminus \Set 0} \frac{\phi((u)}{\|u\|_V}
				\]
				wieder ein Banachraum.
			\item
				Für $1 < p,q < \infty$ mit $\f 1p + pq = 1$ ist $L^p(\Omega)$ ist isomorph zu $(L^q(\Omega))'$.
			\item
				$L^2(\Omega)$ ist alsowegen $\f 12 + \f 12 = 1$ isomorph zu $L^2(\Omega)'$.
		\end{itemize}
	\end{note}
\end{df}

\begin{df}[lokal integrierbare Funktionen] \label{1.5}
	Wir definieren
	\[
		L^1_{\text{loc}}(\Omega) :=
		\Set{ u : \Omega \to \R \text{ lebesgue-messbar} & \forall  K \subset \Omega \text{ kompakt } : \int_k |u(x)| \di < \infty }
	\]
\end{df}

\begin{ex*}
	\begin{itemize}
		\item
			$L^1(\Omega) \subset L^1_{\text{loc}}(\Omega)$.
			Aus $u(x) = 1, x \in \Omega := \R$ folgt $u \not\in L^1$, aber $u \in L^1_{\text{loc}}(\Omega)$.
	\end{itemize}
\end{ex*}

\begin{df}[Funktionen mit kompaktem Träger] \label{1.6}
	Wir definieren für $\Omega \subset \R^d$ offen (möglicherweise nubeschränkt), $m \in \N_0 \cup \Set \infty$.
	\[
		C_0^m(\Omega)
		:= \Set{ u \in C^m(\Omega) & \supp(u) \subset \Omega \text{ ist beschränkt} },
	\]
	wobei $\supp(u) := \_{\Set{x \in \Omega & u(x) \neq 0}}$ (also abgeschlossen) den \emphdef{Träger} (engl. “support”) von $u$ bezeichnet.
\end{df}

\begin{st}[Fundamentalsatz der Variatonsrechnung] \label{1.7}
	Sei $u \in L^1_{\text{loc}}, \Omega \in \R^d$ offen. Dann sind äquivalent
	\begin{enumerate}[i)]
		\item
			$\forall v \in C_0^\infty(\Omega) : \int_\Omega uv = 0$
		\item
			$u = 0$ fast überall in $\Omega$.			
	\end{enumerate}
	\begin{proof}
		2.11 in Alt.
	\end{proof}
\end{st}

\begin{df}[Skalare PDE] \label{1.8}
	Sei $F: \R^{|\mathbb{B}_k} \times \R^{|B_{k-1}} \times \dotsb \times \R^d \times \R \times \Omega \to \R$ gegeben.
	Dann ist
	\[ \label{eq:1.1}
		F(\D^k u(x), D^{k-1} u(x), \dotsc, D^1 u(x), u(x), x) = 0,
		\qquad x \in \Omega
	\]
	eine \emphdef[partielle Differentialgleichung!skalare]{skalare partielle Differentialgleichung der Ordnung $k$} für eine unbekannte Lösung $u: \Omega \to \R$.
\end{df}

\begin{df}[Lineare/Nichtlineare PDE] \label{1.9}
	Die PDE \eqref{1.1} ist
	\begin{enumerate}[i)]
		\item
			\emphdef{linear}, falls sie die Form $\sum_{|\beta| \le k} a_\beta(x) \partial^\beta u(x) = f(x)$ für Multiindex $\beta \in \N_0^d$ und gegebenen Funktionen $a_\beta, f$ besitzt.
			Die PDE heißt \emphdef{homogen}, falls $f(x) = 0$, sonst \emphdef{inhomogen}.
		\item
			\emphdef{semilinear}, falls sie die Form
			\[
				\sum_{|\beta| = k} a_\beta(x) \partial^\beta u(x) + a (D^{k-1} u(x), \dotsc, D^1 u(x), u, x) = 0
			\]
			besitzt.
		\item
			\emphdef{quasilinear}, falls sie die Form
			\[
				\sum_{|\beta| = k} a_\beta(D^{k-1}u, \dotsc, D^1 u, u, x) \partial^\beta uu(x) + a(D^{k-1} u(x), \dotsc, u, x) = 0
			\]
			besitzt.
		\item
			\emphdef{voll nichtlinear} falls sie die nichtlinear von $D^k$ abhängt.
	\end{enumerate}
	\begin{note}[Systeme]
		Ein System von PDEs ist eine Sammliung mehrerer skalarer PDEs für mehrere unbekannte Funktionen $u = (u_1, \dotsc, u_n)^T$.
		Typischerweise sind die einzelnen PDEs miteinander gekoppelt und die Anzahl der Gleichungen und der Unbekannten stimmen überein. 
	\end{note}
	\begin{note}[zeitäbhängige Probleme]
		Alle Notationen und Definitionen für $\Omega \subset \R^d$ mit $x = (x_1, \dotsc, x_d)^T \in \Omega$ erweitern wir auf Orts-Zeit-Zylinder $\Omega_T := \Omega \times (0, T) \subset \R^d \times \R$ mit $(x,t) \in \Omega_T$, $T \in \R^+ \cup \Set \infty$.
		Ortsvariable $x$, Zeitvariable $t$.
		Insbesondere $\partial_t := \pddx[t], \partial_t^2 := \pddx[t^2]$.
		Dann bezeichnet für $u \in C^1(\Omega_t)$
		\begin{align*}
			\nabla_x u(x) &:= \Vector{ \partial_{x_1} u(x) & \dots & \partial_{x_d} u(x) }, \\
			\Laplace_x u(x) &:= \sum_{i=1}^d \partial_{x_i}^2 u(x)
		\end{align*}
		Falls keine Verwechslungsgefahr besteht, lässt man $x$ bei $\nabla_x, \Laplace_x$ auch weg.
	\end{note}
\end{df}

\begin{ex}[Lineare PDEs] \label{1.10}
	Einige häufig begegneten PDEs (Koeffizienten der Einfachheit halber $1$).
	\begin{itemize}
		\item
			\emphdef{Laplace-Gleichung}: $-\Laplace u = 0$,
		\item
			\emphdef{Poisson-Gleichung}: $-\Laplace u = f$,
		\item
			\emphdef{Helmholtz-Gleichung}: $-\Laplace u - \lambda u = 0$ für $\lambda > 0$,
		\item
			\emphdef{Advektions-Gleichung}: $\partial_t u + b \cdot \nabla u = 0$ für $b \in \R^d$
		\item
			\emphdef{Wärmeleitungs-Gleichung} oder \emphdef{Diffusionsgleichung}: $\partial_t u - \Laplace u = 0$.
		\item
			\emphdef{Schrödinger-Gleichung}: $i \partial_t u + \Laplace u = 0$, wobei $i = \sqrt{-1} \in \C$.
		\item
			\emphdef{Wellengleichung}: $\partial_t^2 u - \Laplace u = 0$,
		\item
			\emphdef{Airy's-Gleichung}: $\partial_t u + \partial_x^3 u = 0$
		\item
			\emphdef{Balken-Gleichung}: $\partial_t u + \partial_x^4 u = 0$.
		\item
			\emphdef{Allgemeine Diffusions-Advektions-Reaktions-Gleichung}:
			\[
				- \div \cdot (A \nabla u)0+ b \cdot \nabla u + c u = f,
			\]
			wobei $A \in \R^{d\times d}, b \in \R^d, c \in \R$.
			% fixme: bezeichnung : Diffusion, Advektion, Reaktions, Quellterm
	\end{itemize}
\end{ex}

\begin{ex}[Nichtlineare PDEs] \label{1.11}
	\begin{itemize}
		\item
			\emphdef{Nichtlineare Poission-Gleichung}: $-\Laplace u = f(u)$,
		\item
			\emphdef{$p$-Laplace-Gleichung}: $\div (|\nabla u|^{p-2} \nabla u) = 0$,
		\item
			\emphdef{Minimalflächen-Gleichung}: $\div ( \f{\nabla u}{\sqrt{1 + |\nabla u|^2}} ) = 0$,
		\item
			\emphdef{Hamilton-Jacobi-Gleichung}: $\partial_t u + H(\nabla u, u) = 0$,
		\item
			\emphdef{Burgers-Gleichung}: $\partial_t u + \partial_x(\f 12 u^2) = 0$,
		\item
			\emphdef{skalare Erhaltungsgleichung}: $\partial_t u + \nabla \cdot(f(u, \nabla u)) = 0$,
		\item
			\emphdef{Korteweg de Vries Gleichung (KdV)}: $\partial_t u + u \partial_x u + \partial_x^3 u = 0$,
		\item
			\emphdef{allgemeine Transport-Reaktions-Gleichung}: $\partial_t u + \div(f(u, \nabla u)) = g(u)$.
	\end{itemize}
\end{ex}

\begin{ex}[Lineare Systeme] \label{1.12}
	\begin{itemize}
		\item
			\emphdef{Maxwell-Gleichungen}:
			\begin{align*}
				\partial_t E &= \rot B, \\
				\partial_t B &= - \rot E, \\
				0 &= \div B = \div E.
			\end{align*}
		\item
			\emphdef{Oseen-Gleichungen}:
			\begin{align*}
				(b \cdot \nabla) u - \mu \Laplace u + \nabla p &= 0, \\
				\div u &= 0.
			\end{align*}
			Für $b = 0$ sind dies die \emphdef{Stokes-Gleichungen}.
		\item
			\emphdef[Poission-Gleichung!gemischte Formulierung]{gemischte Formulierung der Poission-Gleichung}:
			\begin{align*}
				\div v &= f, \\
				v + \nabla u &= 0
			\end{align*}
	\end{itemize}
\end{ex}

\begin{ex}[Nichtlineare Systeme] \label{1.13}
	\begin{itemize}
		\item
			System von Erhaltungsgleichungen
			\[
				\partial_t u + \div(F(u)) = 0
			\]
			für $F: \R^d \to \R^d$
		\item
			\emphdef{Navier-Stokes-Gleichungen}
			\begin{align*}
				\partial_t u + (u\cdot \nabla) u - \mu \Laplace u + \nabla p &= 0
				\div u &= 0
			\end{align*}
			Für $\mu = 0$ ergeben sich die \emphdef{Euler-Gleichungen} für ein nichtviskoses, inkompressibles Fluid.
	\end{itemize}
\end{ex}


\section{Klassifikation linearer PDEs zweiter Ordnung}


\begin{df}[linearer Differentialoperator zweiter Ordnung] \label{1.14}
	Sei $\Omega \subset \R^d$ offen, $A = (a_{ij})_{i,j = 1}^d \in C^0 (\Omega, \R^{d\times d}), b = (b_i)_{i=1}^d \in C^0(\Omega)^d, c \in C^0(\Omega)$.
	Dann nennen wir $\scr L: C^2(\Omega) \to C^0(\Omega)$ mit
	\[ \label{eq:1.2}
		(\scr L u)(x) := - \sum_{i,j=1}^d a_{ij}(x) \partial_{x_i} \partial_{x_j} u(x) + \sum_{i=1}^d b_i \partial_{x_i} u(x) + c(x) u(x)
	\]
	\emphdef[Differentialoperator!allgemein, linear, zweiter Ordnung]{allgemeiner linearer Differentialoperator zweiter Ordnung}.
	\begin{note}
		\begin{itemize}
			\item
				$\scr L$ erfasst die Differential-Operatoren in Laplace-, Poisson-, Helmholtz, Wärmeleitungs-, Diffusions  und allgeimener Diffusions-Advektions-Reaktionsgleichung aus \ref{1.10}.
			\item
				Zu $f \in C^0(\Omega)$ ergibt sich eine entsprechende PDE
				\[ \label{eq:1.3}
					\scr L u(x) = f(x), x \in \Omega.
				\]
			\item
				Wir nennen $-\sum_{i,j=1}^d a_{ij}(x) \partial_{x_i} \partial_{x_j} u$ \emphdef{Hauptteil} von $\scr L$.
			\item
				Ohne Einschränkung kann $A$ als symmetrisch vorausgesetzt werden, denn $\partial_{x_i} \partial_{x_j} u = \partial_{x_j} \partial_{x_i} u$.
				Falls $A$ nicht symmetrisch ist, so ergibt $A_s := \f 12 (A + A^T)$ identisches $\scr L$. \Exercise
				$A$ hat somit ohne Einschränkung nur reelle Eigenwerte.
			\item
				Auch geläufig ist die sogennante \emphdef{Divergenzform}, welche man bei differenzierbaren $a_{ij}, b_i$ leicht in obige Form bringen kann.
				Sei
				\begin{align*}
					(\_{\scr L} u)(x) &:= - \nabla \cdot (A(x) \cdot \nabla u) + \nabla \cdot (b \cdot u) + c u \\
					&= - \sum_{i,j=1}^d \partial_{x_i} (a_{ij} \partial_{x_j} u) + \sum_{i=1}^d \partial_{x_i} (b_i u) + cu \\
					&= -\sum_{i,j=1}^d a_{ij} \partial_{x_i} \partial_{x_j} u - \sum_{j=1}^d \sum_{i=1}^d (\partial_{x_i} a_{ij}) \partial_{x_j} + \sum_{i=1}^d b_i \partial_{x_i} u + \sum_{i=1}^d (\partial_{x_i} b_i) \cdot u + cu \\
					&= %fixme
				\end{align*}
				mit der Wahl
				\[
					\tilde b_i := b_i - \sum_{j=1}^d (\partial_{x_j} a_{ji})
					%fixme: \tilde c_i
				\]
		\end{itemize}
	\end{note}
\end{df}

\begin{df}[Klassifikation] \label{1.15}
	Der Operator $\scr L$ aus \eqref{eq:1.2} ist
	\begin{itemize}
		\item
			\emphdef{elliptisch} in $x$, falls alle Eigenwerte von $A(x)$ positiv,
		\item
			\emphdef{parabolisch} in $x$, falls $d-1$ Eigenwerte von $A(x)$ positiv, ein Eigenwert Null ist, aber $\rg([A(x),b(x)]) = d$.
		\item
			\emphdef{hyperbolisch} in $x$, falls $d-1$ Eigenwerte von $A(x)$ positiv und ein Eigenwert negativ ist.
	\end{itemize}
	$\scr L$ \emphdef{elliptisch}, \emphdef{parabolisch}, bzw. \emphdef{hyperbolisch}, wenn er es in jedem $x \in \Omega$ ist.
	Die PDE \eqref{eq:1.3} ist \emphdef{elliptisch}, \emphdef{parabolisch}, bzw. \emphdef{hyperbolisch}, wenn $\scr L$ dies ist.
	\begin{note}
		\begin{itemize}
			\item
				Die Begriffe sind motiviert aus Kegelschnitten/Quadriken:
				\[
					z^T A(x) z = 1
				\]
				beschreibt unter den genannten Voraussetzungen ein Ellipsoid/Paraboloid/Hyperboloid.
		\end{itemize}
	\end{note}
\end{df}


