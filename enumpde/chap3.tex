\chapter{FEM für koerzive und inf-sup-stabile Probleme} \label{chap:3}


\section{Sobolev-Räume}


\begin{df}[Schwache Ableitung] \label{3.1}
	Sei $\beta \in \N_0^d$ ein Multiindex, $u \in L_{\text{loc}}^1, \Omega \subset \R^d$.
	Wir nennen $v^\beta \in L_{\text{loc}}^1(\Omega)$ \emphdef{schwache Ableitung} von $u$ falls gilt
	\[
		\int_\Omega u \partial^\beta \phi \di[x]
		= (-1)^{|\beta|} \int_\Omega v^\beta \phi \di[x]
	\]
	für alle $\phi \in C_0^\infty(\Omega)$.
\end{df}

\begin{ex*}
	Sei $\Omega = (-1, 1), u(x) := |x|$.
	Dann ist $\sign(x)$ die schwache Ableitung von $u$.
	Denn sei $\phi \in C_0^\infty(\Omega)$, dann ist
	\begin{align*}
		\int_{-1}^1 u(x) \phi'(x) \di[x]
		&= \int_{-1}^0 u(x) \phi'(x) \di[x]
		+ \int_0^1 u(x) \phi'(x) \di[x] \\
		&= \int_{-1}^0 -x \phi'(x) \di[x]
		+ \int_0^1 x \phi'(x) \di[x] \\
		&= - \int_{-1}^0 (-1) \phi(x) \di[x] + \underbrace{[-x \phi(x)]_{-1}^0}_{=0}
		- \int_0^1 \phi(x) \di[x] + \underbrace{[x \phi(x)]_0^1}_{=0} \\
		&= - \int_{-1}^1 \sign(x) \phi(x) \di[x]
	\end{align*}
\end{ex*}

\begin{st}[Eindeutigkeit] \label{3.2}
	Sei $u \in L^1_{\text{loc}}(\Omega)$, dann existiert höchstens eine schwache Ableitung zu $\beta \in \N_0^d$.
	\begin{proof}
		Seien $v^\beta, \_v^\beta$ zwei schwache Ableitungen.
		Dann ist
		\[
			(-1)^{|\beta|} \int_\Omega \_v^\beta \phi \di[x]
			= \int_\Omega u \partial^\beta \phi \di[x]
			= (-1)^{|\beta|} \int_\Omega v^\beta \phi \di[x],
		\]
		also für alle $\phi \in C_0^\infty(\Omega)$
		\[
			\int_\Omega (v^\beta - \_v^\beta) \phi \di[x] = 0.
		\]
		Nach dem Fundamentalsatz der Variationsrechnung \ref{1.7} also $v^\beta = \_v^\beta$ fast überall, also $v^\beta = \_v^\beta$ im $L^1_{\text{loc}}$-Sinn.
	\end{proof}
\end{st}

\begin{st}[klassische Ableitung ist schwache Ableitung] \label{3.3}
	Sei $u \in C^m(\Omega)$, $v^\beta$ für $|\beta| \le m$ die schwache Ableitung und $\partial^\beta u \in C^{m-|\beta|}(\Omega)$ die klassische Ableitung.
	Dann gilt $v^\beta = \partial^\beta u$ punktweise, also im $C^0$-Sinn.
	\begin{proof}
		Mit partieller Integration gilt
		\[
			\int_\Omega u \partial^\beta \phi \di[x]
			= (-1)^{|\beta|} \int_\Omega (\partial^\beta u) \phi \di[x]
		\]
		für alle $\phi \in C_0^\infty(\Omega)$ und die klassische Ableitung ist eine schwache Ableitung.
		Wegen der Eindeutigkeit \ref{3.2} ist $\partial^\beta u$ die schwache Ableitung.
	\end{proof}
	\begin{note}
		\begin{itemize}
			\item
				Die schwache Ableitung umfassen die klassischen Ableitungen, daher schreiben wir im Folgenden einfach $\partial^\beta u$ statt $v^\beta$.
			\item
				\ref{3.3} gilt auch auf Teilgebieten, d.h. falls $u$ stückweise klassisch differenzierbar und global schwach differenzierbar, so ist die schwache Ableitung gerade die stückweise klassische Ableitung.
		\end{itemize}
	\end{note}
\end{st}

\begin{ex*}
	$u(x) := \sign(x)$ ist nicht schwach differenzierbar.
	\begin{proof}
		Falls $u$ schwach differenzierbar wäre, so müsste
		\[
			\partial_x u = \begin{cases}
				0 & x < 0 \\
				0 & x > 0
			\end{cases}
		\]
		gemäß der stückweisen klassichen Ableitung.
		Es folgt $(-1)\int_\Omega (\partial_x u) \phi \di[x] = 0$, aber für $\phi(0) \neq 0, \supp \phi \subset (-1,1)$ nicht leer gilt
		\begin{align*}
			\int_\Omega u \partial_x \phi \di[x]
			&= \int_{-1}^0 (-1) \partial_x \phi \di[x] + \int_0^1 1 \partial_x \phi \di[x] \\
			&= \underbrace{\phi(1)}_{=0} \underbrace{- \phi(0) - \phi(0)}_{= 2 \phi(0)} + \underbrace{\phi(-1) }_{=0}
			\neq 0
		\end{align*}
	\end{proof}
\end{ex*}

\Timestamp{2014-11-14}

\begin{df}[Sobolev-Räume] \label{3.4}
	Sei $m \in \N_0, p \in \N \cup \Set \infty, u \in L_{\text{loc}}^1(\Omega), \Omega\subset \R^d$ offen.
	Falls alle schwachen Ableitungen $\partial^\beta u, |\beta| \le m$ existieren, definieren wir die Sobolev-Norm durch
	\begin{align*}
		\|u\|_{H^{m,p}(\Omega)}
		&:= \Big( \sum_{|\beta| \le m} \|\partial^\beta u\|_{L^p(\Omega)}^p \Big)^{\f 1p} \qquad 1 \le p \le \infty, \\
		\|u\|_{H^{m,\infty}(\Omega)}
		&:= \max_{|\beta|\le m} \|\partial^\beta u\|_{L^\infty(\Omega)}.
	\end{align*}
	Hiermit definieren wir die Sobolev-Räume $H^{m,p}(\Omega)$ als:
	\[
		H^{m,p}(\Omega)
		:= \Set{ u \in L_{\text{loc}}^1(\Omega) & \|u\|_{H^{m,p}} < \infty}.
	\]
	Für $p = 2$ schreiben wir auch $H^m(\Omega) := H^{m,2}(\Omega)$.
	Schließlich definieren wir die Sobolev-Seminorm durch
	\begin{align*}
		|u|_{H^{m,p}(\Omega)}
		&:= \Big( \sum_{|\beta| = m} \|\partial^\beta u\|_{L^p(\Omega)}^p \Big)^{\f 1p} \qquad 1 \le p \le \infty, \\
		|u|_{H^{m,\infty}(\Omega)}
		&:= \max_{|\beta| = m} \|\partial^\beta u\|_{L^\infty(\Omega)}.
	\end{align*}
	\begin{note}
		\begin{itemize}
			\item
				Anstelle von $H^{m,p}(\Omega)$ wird in der Literatur auch oft $W^{m,p}(\Omega)$ verwendet.
			\item
				Aus \ref{3.3} folgt $C_0^m(\Omega) \subset H^{m,p}(\Omega)$.
				Falls $\Omega$ beschränkt, gilt auch $C^m(\_\Omega) \subset H^{m,p}(\_\Omega)$.
		\end{itemize}
	\end{note}
\end{df}

\begin{st}[Vollständigkeit von $H^{m,p}(\Omega)$] \label{3.5}
	Sei $\Omega \subset \R^d$ offen, $1 \le p \le \infty, m \in \N_0$.
	Dann ist $H^{m,p}(\Omega)$ vollständig, also ein Banachraum.
	Insbesondere ist $H^m(\Omega)$ ein Hilbertraum mit dem entsprechenden Skalarprodukt $\<u,v\>_{H^m} := \sum_{|\beta| \le m} \<\partial^\beta u, \partial^\beta u\>_{L^2(\Omega)}$.
	\begin{note}
		\begin{itemize}
			\item
				Dies ist eine sehr praktische Eigenschaft im Gegensatz zu klassischen Funktionenräumen, z.B. ist $C^m(\_\Omega)$ nicht vollständig bezüglich $\|\argdot\|_{H^m,p}(\Omega)$.
				% fixme: gegenbeispiel
			\item
				Für $m = 0$ gilt $H^{m,p}(\Omega) = L^p(\Omega)$.
			\item
				Alternativ kann man Sobolev-Räume auch durch Vervollständigung von $C^m$ bezüglich $\|\argdot\|_{H^{m,p}}$ definieren.
				\[
					H^{m,p}(\Omega) = \_{C^m(\Omega)}^{\|\argdot\|_{H^{m,p}}}.
				\]
				Siehe dazu [Alt, 1.15].
		\end{itemize}
	\end{note}
	\begin{proof}[für $p=2$, allgemeiner in [Alt, 1.15]]
		Sei $(v_n)_{n\in\N}$ Cauchyfolge in $H^m(\Omega)$.
		Dann ist insbesondere $(\partial^\beta v_n)_{n\in\N}$ Cauchyfolge in $L^2(\Omega)$.
		Wegen Vollständigkeit von $L^2(\Omega)$ existiert $v^\beta \in L^2(\Omega)$ mit $\|\partial^\beta v_n - v^\beta\|_{L^\beta(\Omega)} \to 0$ für $n \to \infty$.
		Für eine Testfunktion $\phi \in C_0^\infty(\Omega)$ und alle $\beta \in \N_0^d$ gilt mit Cauchy-Schwartz
		\begin{align*}
			\<\partial^\beta v_n - v^\beta, \phi\>_{L^2(\Omega)}
			\le \|\partial^\beta v_n - v^\beta\|_{L^2}\|\phi\|_{L^2}.
		\end{align*}
		Es folgt
		\begin{align*}
			\<v^\beta, \phi\>_{L^2(\Omega)}
			&= \lim_{n\to\infty} \<\partial^\beta v_n, \phi\>_{L^2(\Omega)} \\
			&= \lim_{n\to\infty} (-1)^{|\beta|} \<v_n, \partial^\beta \phi\>_{L^2(\Omega)}
			= (-1)^{|\beta|} \<v^0, \partial^\beta \phi\>_{L^2}
		\end{align*}
		mit $v_0 := \lim_{n\to\infty} v_n$ via $\beta = 0$.
		Damit ist $v^\beta$ schwache Ableitung von $v_0$.
		Also $\|v_n - v^0\|_{H^m(\Omega)} \to 0$ und $v^0 \in H^m(\Omega)$.
	\end{proof}
\end{st}

\begin{st}[Approximierbarkeit durch $C^\infty$-Funktionen] \label{3.6}
	Für $1 \le p < \infty$ gilt $H^{m,p}(\Omega) \cap C^\infty(\Omega)$ ist dicht in $H^{m,p}(\Omega)$, d.h. für $f \in H^{m,p}$ existiert $f_j \in H^{m,p}(\Omega) \cap C^\infty(\Omega), j \in \N$ mit $\|f - f_j\|_{H^{m,p}(\Omega)} \to 0$.
	\begin{proof}
		Siehe [Alt, Satz 1.16].
	\end{proof}
	\begin{note}
		Aufgrund von \ref{3.6} sieht man leicht, dass Regeln zum Umgang mit Ableitungen von klassischen Ableitungen auf schwache Ableitungen übertragen werden können,
		darunter Linearität, partielle Integration, Gäußscher Integralsatz, Produkt- und Kettenregel, etc.
	\end{note}
	\begin{note}[Randwerte]
		Da $L^p(\Omega)$-Funktionen auf Nullmengen undefiniert sind, bzw. beliebig abgeändert werden können, ist unklar, was man unter Randwerten in $H^{m,p}(\Omega)$ verstehen soll.
		Tatsächlich hilft zusätzliche Regularität ($m \ge 1$), sogenannte „\emphdef{schwache Randwerte}“ zu definieren, welche mit dem „\emphdef{Spuroperator}“ extrahiert werden können.
	\end{note}
\end{st}

\begin{df}[Sobolev-Funktionen mit schwachen Nullrandwerten] \label{3.7}
	Für $1 \le p < \infty$ und $m \in \N$ definieren wir Sobolev-Räume mit (schwachen) Nullrandwerten:
	\[
		H^{m,p}_0(\Omega)
		:= \_{C_0^m(\Omega)}^{\|\argdot\|_{H^{m,p}(\Omega)}}.
	\]
	\begin{note}
		\begin{itemize}
			\item
				$\Omega$ darf auch unbeschränkt sein.
			\item
				In der Literatur findet man auch $W_0^{m,p}(\Omega), \mathring H^{m,p}(\Omega)$ als Notationen.
		\end{itemize}
	\end{note}
\end{df}

\begin{st}[Vollständigkeit von $H^{m,p}_0(\Omega)$] \label{3.8}
	Für $1 \le p < \infty, m \in \N$ ist $H^{m,p}_0(\Omega)$ abgeschlossener Unterraum von $H^{m,p}(\Omega)$, insbesondere ein Banachraum und $\|\argdot\|_{H^{m,p}}$ überträgt sich auf $H_0^{m,p}(\Omega)$.
	\begin{proof}
		Abgeschlossenheit ist klar nach Konstruktion.
		$H^{m,p}_0(\Omega) \subset H^{m,p}(\Omega)$ wegen $C_0^m(\Omega) \subset H^{m,p}(\Omega)$ und Abgeschlossenheit von $H^{m,p}(\Omega)$.
	\end{proof}
\end{st}

% fixme
%\begin{note}
%	Es gilt
%	\begin{align*}
%		L^p(\Omega) = H^{0,p}(\Omega) \supset H^{1,p}(\Omega) \supset
%		H_0^{0,p}(\Omega) &\supset H_0^{1,p}(\Omega) &\supset \dots
%		C_0^0(\Omega) &\supset C_0^1(\Omega)
%	\end{align*}
%\end{note}

\begin{df}[Lipschitz-Gebiet]
	Sei $\Omega \subset \R^d$ offen und beschränkt.
	$\Omega$ heißt \emphdef{Lipschitz-Gebiet}, wenn endlich viele offene Mengen $U_i \subset \R^d, i = 1, \dotsc, n$ existieren, sodass $\partial \Omega \subset \bigcup_{i=1}^n U_i$ und $\Boundary \Omega \cap U_i$ lässt sich als Graph einer Lipschitz-stetigen Funktionen schreiben und $\Omega$ liegt auf einer Seite des Graphen.
	\begin{note}
		Für Lipschitz-Gebiete gelten insbesondere (siehe [Alt, 5.9]):
		\begin{itemize}
			\item
				Satz von Gauß:
				Sei $v \in (\_\Omega, \R^d) \cap C^1(\Omega, \R)$ mit $\div v \in L^1(\Omega)$, dann gilt
				\[
					\int_\Omega \div v(x) \di[x]
					= \int_{\Boundary \Omega} v(x) \cdot n(x) \di[\sigma(x)].
				\]
			\item
				partielle Integration:
				Für $u \in C^1(\_\Omega), v \in C^1(\_\Omega, \R^d)$ gilt
				\[
					\int_\Omega \Nabla u(x) \cdot v(x) \di[x]
					= - \int_\Omega u(x) \div v(x) \di[x] + \int_{\Boundary \Omega}) u(x) v(x) \cdot n(x) \di[\sigma(x)].
				\]
		\end{itemize}
	\end{note}
\end{df}

\begin{st}[Spursatz] \label{3.10}
	Sei $\Omega \subset \R^d$ Lipschitz-Gebiet, $1 \le p < \infty$.
	Dann existiert ein linearer, stetiger \emphdef{Spuroperator} $\gamma: H^{1,p(\Omega)} \to L^p(\Boundary \Omega)$ sodass
	\[
		\gamma(u) = u|_{\Boundary \Omega}
	\]
	für alle $u \in H^{1,p}(\Omega) \cap C^0(\_\Omega)$.

	Insbesondere gilt für $u \in H_0^{1,p}(\Omega)$ dann $\gamma(u) = 0$.
	Wegen Stetigkeit existiert also $C_\gamma > 0$, sodass
	\[
		\|\gamma(u)\|_{L^p(\Boundary \Omega)} \le C_\gamma \|u\|_{H^{1,p}(\Omega)}
	\]
	für alle $u \in H^{1,p}(\Omega)$.
	\begin{proof}
		Für $d = 2, p = 2$ siehe PDEMAS 13/14, allgemeiner in [Alt, 5.7].
	\end{proof}
\end{st}

\begin{df}[Sobolev Dualräume] \label{3.11}
	Für $1 \le p, q \le \infty, \f 1p + \f 1q = 1$ bezeichnen wir $H^{-m,q}(\Omega) := (H_0^{m,p}(\Omega))'$.
	Wir schreiben $H^{-m}(\Omega) := H^{-m, 2}(\Omega)$.
	\begin{note}
		Damit werden wir Differentialgleichungen betrachten können, deren rechte Seite (Quellterm) Funktionale statt Funktionen sind.
	\end{note}
\end{df}

\begin{note}[Stetigkeit für $d = 1$]
	\begin{itemize}
		\item
			Für $\Omega \subset \R$ ist $u \in H^1{\Omega}$ stetig (d.h. es existiert ein stetiger Repräsentant in der Äquivalenzklasse von $u$).
		\item
			Für $d > 1$ ist dies falsch: $H^1(\Omega)$ für $\Omega \subset \R^d$ enthält Funktionen mit Punktsingularitäten, z.b.
			\begin{align*}
				d = 2: \qquad u(x) &:= \log \log(\f 2{|x|}) \in H^1(B_1(0)), \\
				d \ge 3: \qquad u(x) &:= |x|^{-\beta} \in H^1(B_1(0)), \beta < \f{d-2}2.
			\end{align*}
	\end{itemize}
\end{note}

\begin{st}[Poincaré-Friedrich Ungleichung]
	Sei $\Omega \subset \R^d$ offen und beschränkt, $s := \diam(\Omega)$.
	Dann gilt
	\[
		\|v\|_{L^2(\Omega)} \le s |v|_{H^1(\Omega)}
	\]
	für alle $v \in H_0^1(\Omega)$.
	\begin{proof}
		Weil $C_0^\infty(\Omega)$ dicht in $H_0^1(\Omega)$ genügt es, die Ungleichung für $v \in C_0^\infty(\Omega)$ zu zeigen.
		ObdA sei $\Omega \subset [0,s]^d =: R$.
		Dann ist $v \in C_0^\infty(\Omega)$ durch $0$ fortsetzbar, also $v \in C_0^\infty(R)$.
		\[
			v(x_1, \dotsc, x_d) = \underbrace{v(0, x_2, \dotsc, x_d)}_{=0} +  \int_0^{x_1} \partial_{x_1} v(t, x_2, \dotsc, x_d) \di[t].
		\]
		Mit $x = (x_1, \dotsc, x_d)$ ist
		\begin{align*}
			|v(x)|^2
			&\le \Big(\int_0^{x_1} 1^2 \di[t] \Big) \Big( \int_0^{x_1} |\partial_{x_1} v(t,x_2,\dotsc, x_d)|^2 \di[t] \Big) \\
			&\le s \int_0^s |\partial_{x_1} v(t, x_2, \dotsc, x_d)|^2 \di[t].
		\end{align*}
		Die rechte Seite ist unabhängig von $x_1$, integriere daher bezüglich $x_1$:
		\[
			\int_0^s |v(x)|^2 \di[x_1]
			\le s^2 \int_0^s |\partial_{x_1} v(x)|^2 \di[x_1].
		\]
		Integration über andere Koordinaten liefert
		\[
			\int_R |v(x)|^2 \di[x]
			\le \int_R (\partial_{x_1} v(x))^2 \di[x]
			\le s^2 |v|_{H^1(\Omega)}.
		\]
	\end{proof}
	\begin{note}
		\begin{itemize}
			\item
				Die Ungleichung gilt bereits, wenn Nullrandwerte nur auf einem Teil des Randes vorliegen.
		\end{itemize}
	\end{note}
\end{st}

\begin{st}[Norm-Äquivalenz auf $H_0^{m}(\Omega)$] \label{3.13}
	Sei $\Omega \subset \R^d$ offen, beschränkt mit $\diam(\Omega) \le s$.
	Dann sind in $H_0^m(\Omega)$ die Norm $\|\argdot\|_{H^m(\Omega)}$ und die Seminorm $|\argdot|_{H^m(\Omega)}$ äquivalent.
	Insbesondere ist die Seminorm $|\argdot|_{H^m(\Omega)}$ eine Norm auf $H_0^m(\Omega)$.
	Genauer gilt
	\[
		|v|_{H^m(\Omega)}
		\le \|v\|_{H^m(\Omega)}
		\le (1 + s)^m |v|_{H^m(\Omega)}
	\]
	für $v \in H_0^m(\Omega)$.
	\begin{proof}
		Siehe Übung.
	\end{proof}
\end{st}

