\chapter{FEM für koerzive und inf-sup-stabile Probleme} \label{chap:3}


\section{Sobolev-Räume} \label{sec:3.1}


\begin{df}[Schwache Ableitung] \label{3.1}
	Sei $\Omega \subset \R^d, u \in L_{\text{loc}}^1(\Omega)$ und $\beta \in \N_0^d$ ein Multiindex.
	Wir nennen $v^\beta \in L_{\text{loc}}^1(\Omega)$ \emphdef{schwache Ableitung} von $u$ falls gilt
	\[
		\int_\Omega u \partial^\beta \phi \di[x]
		= (-1)^{|\beta|} \int_\Omega v^\beta \phi \di[x]
	\]
	für alle $\phi \in C_0^\infty(\Omega)$.
\end{df}

\begin{ex*}
	Sei $\Omega = (-1, 1), u: \Omega \to \R, u(x) := |x|$.
	Dann ist $\sign(x)$ die schwache Ableitung von $u$.
	\begin{proof}
		Denn sei $\phi \in C_0^\infty(\Omega)$, dann ist
		\begin{align*}
			\int_{-1}^1 u(x) \phi'(x) \di[x]
			&= \int_{-1}^0 u(x) \phi'(x) \di[x]
			+ \int_0^1 u(x) \phi'(x) \di[x] \\
			&= \int_{-1}^0 -x \phi'(x) \di[x]
			+ \int_0^1 x \phi'(x) \di[x] \\
			&= - \int_{-1}^0 (-1) \phi(x) \di[x] + \underbrace{[-x \phi(x)]_{-1}^0}_{=0}
			- \int_0^1 \phi(x) \di[x] + \underbrace{[x \phi(x)]_0^1}_{=0} \\
			&= - \int_{-1}^1 \sign(x) \phi(x) \di[x].
		\end{align*}
	\end{proof}
\end{ex*}

\begin{st}[Eindeutigkeit] \label{3.2}
	Sei $u \in L^1_{\text{loc}}(\Omega)$, dann existiert höchstens eine schwache Ableitung zu $\beta \in \N_0^d$.
	\begin{proof}
		Seien $v^\beta, \_v^\beta$ zwei schwache Ableitungen.
		Dann ist
		\[
			(-1)^{|\beta|} \int_\Omega \_v^\beta \phi \di[x]
			= \int_\Omega u \partial^\beta \phi \di[x]
			= (-1)^{|\beta|} \int_\Omega v^\beta \phi \di[x],
		\]
		also für alle $\phi \in C_0^\infty(\Omega)$
		\[
			\int_\Omega (v^\beta - \_v^\beta) \phi \di[x] = 0.
		\]
		Nach dem Fundamentalsatz der Variationsrechnung \ref{1.7} also $v^\beta = \_v^\beta$ fast überall, also $v^\beta = \_v^\beta$ im $L^1_{\text{loc}}$-Sinn.
	\end{proof}
\end{st}

\begin{st}[klassische Ableitung ist schwache Ableitung] \label{3.3}
	Sei $u \in C^m(\Omega)$, $v^\beta \in L^1_{\text{loc}}$ für $|\beta| \le m$ die schwache Ableitung von $u$ und $\partial^\beta u \in C^{m-|\beta|}(\Omega)$ die klassische Ableitung von $u$.
	Dann gilt $v^\beta = \partial^\beta u$ in $L^1_{\text{loc}}$.
	\begin{proof}
		Mit partieller Integration gilt
		\[
			\int_\Omega u \partial^\beta \phi \di[x]
			= (-1)^{|\beta|} \int_\Omega (\partial^\beta u) \phi \di[x]
		\]
		für alle $\phi \in C_0^\infty(\Omega)$ und die klassische Ableitung ist eine schwache Ableitung.
		Wegen der Eindeutigkeit \ref{3.2} ist $\partial^\beta u$ in $L^1_{\text{loc}}$ die schwache Ableitung.
	\end{proof}
	\begin{note}
		\begin{itemize}
			\item
				Der Begriff der schwachen Ableitung umfasst den der klassischen Ableitung, daher schreiben wir im Folgenden einfach $\partial^\beta u$ statt $v^\beta$ für die schwache Ableitung.
			\item
				\ref{3.3} gilt auch auf Teilgebieten, z.B. falls $u$ stückweise klassisch differenzierbar und global schwach differenzierbar, so ist die schwache Ableitung gerade die stückweise klassische Ableitung.
		\end{itemize}
	\end{note}
\end{st}

\begin{ex*}
	$u(x) := \sign(x)$ ist nicht schwach differenzierbar.
	\begin{proof}
		Falls $u$ schwach differenzierbar wäre, so müsste
		\[
			\partial_x u = \begin{cases}
				0 & x < 0 \\
				0 & x > 0
			\end{cases}
		\]
		gemäß der stückweisen klassichen Ableitung.
		Es folgt $(-1)\int_\Omega (\partial_x u) \phi \di[x] = 0$.
		Jedoch ist für $\phi \in C_0^\infty(\Omega), \phi(0) \neq 0$:
		\begin{align*}
			\int_\Omega u \partial_x \phi \di[x]
			&= \int_{-1}^0 (-1) \partial_x \phi \di[x] + \int_0^1 1 \partial_x \phi \di[x] \\
			&= \underbrace{\phi(-1)}_{=0} \underbrace{- \phi(0) - \phi(0)}_{= -2 \phi(0)} + \underbrace{\phi(1) }_{=0}
			\neq 0.
		\end{align*}
	\end{proof}
\end{ex*}

\Timestamp{2014-11-14}

\begin{df}[Sobolev-Räume] \label{3.4}
	Sei $m \in \N_0, p \in \N \cup \Set \infty, u \in L_{\text{loc}}^1(\Omega), \Omega\subset \R^d$ offen.
	Falls alle schwachen Ableitungen $\partial^\beta u, |\beta| \le m$ existieren, definieren wir die \emphdef{Sobolev-Norm} durch
	\begin{align*}
		\|u\|_{H^{m,p}(\Omega)}
		&:= \Big( \sum_{|\beta| \le m} \|\partial^\beta u\|_{L^p(\Omega)}^p \Big)^{\f 1p} \qquad 1 \le p < \infty, \\
		\|u\|_{H^{m,\infty}(\Omega)}
		&:= \max_{|\beta|\le m} \|\partial^\beta u\|_{L^\infty(\Omega)}.
	\end{align*}
	Hiermit definieren wir den \emphdef{Sobolev-Raum} $H^{m,p}(\Omega)$ als
	\[
		H^{m,p}(\Omega)
		:= \Set{ u \in L_{\text{loc}}^1(\Omega) & \|u\|_{H^{m,p}} < \infty}
	\]
	mit der entsprechenden Sobolev-Norm.

	Für $p = 2$ schreiben wir auch $H^m(\Omega) := H^{m,2}(\Omega)$.
	Schließlich definieren wir die \emphdef{Sobolev-Seminorm} durch
	\begin{align*}
		|u|_{H^{m,p}(\Omega)}
		&:= \Big( \sum_{|\beta| = m} \|\partial^\beta u\|_{L^p(\Omega)}^p \Big)^{\f 1p} \qquad 1 \le p < \infty, \\
		|u|_{H^{m,\infty}(\Omega)}
		&:= \max_{|\beta| = m} \|\partial^\beta u\|_{L^\infty(\Omega)}.
	\end{align*}
	\begin{note}
		\begin{itemize}
			\item
				Anstelle von $H^{m,p}(\Omega)$ wird in der Literatur auch oft $W^{m,p}(\Omega)$ verwendet.
			\item
				Aus \ref{3.3} folgt $C_0^m(\Omega) \subset H^{m,p}(\Omega)$.
				Falls $\Omega$ beschränkt, gilt auch $C^m(\_\Omega) \subset H^{m,p}(\_\Omega)$.
		\end{itemize}
	\end{note}
\end{df}

\begin{st}[Vollständigkeit von $H^{m,p}(\Omega)$] \label{3.5}
	Sei $\Omega \subset \R^d$ offen, $1 \le p \le \infty, m \in \N_0$.
	Dann ist $H^{m,p}(\Omega)$ vollständig, also ein Banachraum.

	Weiter ist $H^m(\Omega)$ ein Hilbertraum mit dem entsprechenden Skalarprodukt
	\begin{math}
		\<u,v\>_{H^m} := \sum_{|\beta| \le m} \<\partial^\beta u, \partial^\beta u\>_{L^2(\Omega)}.
	\end{math}
	\begin{note}
		\begin{itemize}
			\item
				Dies ist eine sehr praktische Eigenschaft im Gegensatz zu klassischen Funktionenräumen, z.B. ist $C^m(\_\Omega)$ nicht vollständig bezüglich $\|\argdot\|_{H^m,p}(\Omega)$.
				% fixme: gegenbeispiel
			\item
				Für $m = 0$ gilt $H^{m,p}(\Omega) = L^p(\Omega)$.
			\item
				Alternativ kann man Sobolev-Räume auch durch Vervollständigung von $C^m$ bezüglich $\|\argdot\|_{H^{m,p}}$ definieren.
				\[
					H^{m,p}(\Omega) = \_{C^m(\Omega)}^{\|\argdot\|_{H^{m,p}}}.
				\]
				Siehe dazu [Alt, 1.15].
		\end{itemize}
	\end{note}
	\begin{proof}[für $p=2$, allgemeiner in Alt, 1.15]
		Sei $(v_n)_{n\in\N}$ Cauchyfolge in $H^m(\Omega)$.
		Dann ist insbesondere $(\partial^\beta v_n)_{n\in\N}$ Cauchyfolge in $L^2(\Omega)$.
		Wegen Vollständigkeit von $L^2(\Omega)$ existiert $v^\beta \in L^2(\Omega)$ mit $\|\partial^\beta v_n - v^\beta\|_{L^\beta(\Omega)} \to 0$ für $n \to \infty$.
		Für eine Testfunktion $\phi \in C_0^\infty(\Omega)$ und alle $\beta \in \N_0^d$ gilt mit Cauchy-Schwartz
		\begin{align*}
			\<\partial^\beta v_n - v^\beta, \phi\>_{L^2(\Omega)}
			\le \|\partial^\beta v_n - v^\beta\|_{L^2}\|\phi\|_{L^2}.
		\end{align*}
		Es folgt
		\begin{align*}
			\<v^\beta, \phi\>_{L^2(\Omega)}
			&= \lim_{n\to\infty} \<\partial^\beta v_n, \phi\>_{L^2(\Omega)} \\
			&= \lim_{n\to\infty} (-1)^{|\beta|} \<v_n, \partial^\beta \phi\>_{L^2(\Omega)}
			= (-1)^{|\beta|} \<v^0, \partial^\beta \phi\>_{L^2}
		\end{align*}
		mit $v^0 := \lim_{n\to\infty} v_n$ via $\beta = 0$.
		Damit ist $v^\beta$ schwache Ableitung von $v^0$.
		Also $\|v_n - v^0\|_{H^m(\Omega)} \to 0$ und $v^0 \in H^m(\Omega)$.
	\end{proof}
\end{st}

\begin{st}[Approximierbarkeit durch $C^\infty$-Funktionen] \label{3.6}
	Für $1 \le p < \infty$ ist $H^{m,p}(\Omega) \cap C^\infty(\Omega)$ dicht in $H^{m,p}(\Omega)$, d.h. für $f \in H^{m,p}(\Omega)$ existiert $f_j \in H^{m,p}(\Omega) \cap C^\infty(\Omega), j \in \N$ mit $\|f - f_j\|_{H^{m,p}(\Omega)} \to 0$.
	\begin{proof}
		Siehe [Alt, Satz 1.16].
	\end{proof}
	\begin{note}
		Aufgrund von \ref{3.6} sieht man leicht, dass Regeln zum Umgang mit Ableitungen von klassischen Ableitungen auf schwache Ableitungen übertragen werden können,
		darunter Linearität, partielle Integration, Gaußscher Integralsatz, Produkt- und Kettenregel, etc.
	\end{note}
\end{st}

\begin{note}[Randwerte]
	Da $L^p(\Omega)$-Funktionen auf Nullmengen undefiniert sind, bzw. beliebig abgeändert werden können, ist unklar, was man unter Randwerten in $H^{m,p}(\Omega)$ verstehen soll.
	Tatsächlich hilft zusätzliche Regularität ($m \ge 1$), sogenannte „\emphdef{schwache Randwerte}“ zu definieren, welche mit dem „\emphdef{Spuroperator}“ extrahiert werden können.
\end{note}

\begin{df}[Sobolev-Funktionen mit schwachen Nullrandwerten] \label{3.7}
	Für $1 \le p < \infty$ und $m \in \N$ definieren wir Sobolev-Räume mit (schwachen) Nullrandwerten:
	\[
		H^{m,p}_0(\Omega)
		:= \_{C_0^m(\Omega)}^{\|\argdot\|_{H^{m,p}(\Omega)}}.
	\]
	\begin{note}
		\begin{itemize}
			\item
				$\Omega$ darf auch unbeschränkt sein.
			\item
				In der Literatur findet man auch $W_0^{m,p}(\Omega)$ oder $\mathring H^{m,p}(\Omega)$ als Notationen.
		\end{itemize}
	\end{note}
\end{df}

\begin{st}[Vollständigkeit von $H^{m,p}_0(\Omega)$] \label{3.8}
	Für $1 \le p < \infty, m \in \N$ ist $H^{m,p}_0(\Omega)$ abgeschlossener Unterraum von $H^{m,p}(\Omega)$, insbesondere ein Banachraum und die Sobolev-Norm $\|\argdot\|_{H^{m,p}}$ überträgt sich auf $H_0^{m,p}(\Omega)$.
	\begin{proof}
		Abgeschlossenheit ist klar nach Konstruktion.
		$H^{m,p}_0(\Omega) \subset H^{m,p}(\Omega)$ folgt aus $C_0^m(\Omega) \subset H^{m,p}(\Omega)$ und Abgeschlossenheit von $H^{m,p}(\Omega)$.
	\end{proof}
\end{st}

% fixme
%\begin{note}
%	Es gilt
%	\begin{align*}
%		L^p(\Omega) = H^{0,p}(\Omega) \supset H^{1,p}(\Omega) \supset
%		H_0^{0,p}(\Omega) &\supset H_0^{1,p}(\Omega) &\supset \dots
%		C_0^0(\Omega) &\supset C_0^1(\Omega)
%	\end{align*}
%\end{note}

\begin{df}[Lipschitz-Gebiet]
	Sei $\Omega \subset \R^d$ offen und beschränkt.
	$\Omega$ heißt \emphdef{Lipschitz-Gebiet}, wenn endlich viele offene Mengen $U_i \subset \R^d, i = 1, \dotsc, n$ existieren, sodass
	\begin{enumerate}[i)]
		\item
			$\partial \Omega \subset \bigcup_{i=1}^n U_i$,
		\item
			$\Boundary \Omega \cap U_i$ lassen sich als Graphen Lipschitz-stetiger Funktionen schreiben,
		\item
			$\Omega$ liegt auf einer Seite des Graphen.
	\end{enumerate}
	\begin{note}
		Für Lipschitz-Gebiete gelten insbesondere (siehe [Alt, 5.9]):
		\begin{itemize}
			\item
				Satz von Gauß:
				Sei $v \in (\_\Omega, \R^d) \cap C^1(\Omega, \R)$ mit $\div v \in L^1(\Omega)$, dann gilt
				\[
					\int_\Omega \div v(x) \di[x]
					= \int_{\Boundary \Omega} v(x) \cdot n(x) \di[\sigma(x)].
				\]
			\item
				partielle Integration:
				Für $u \in C^1(\_\Omega), v \in C^1(\_\Omega, \R^d)$ gilt
				\[
					\int_\Omega \Nabla u(x) \cdot v(x) \di[x]
					= - \int_\Omega u(x) \div v(x) \di[x] + \int_{\Boundary \Omega} u(x) v(x) \cdot n(x) \di[\sigma(x)].
				\]
		\end{itemize}
	\end{note}
\end{df}

\begin{st}[Spursatz] \label{3.10}
	Sei $\Omega \subset \R^d$ Lipschitz-Gebiet, $1 \le p < \infty$.
	Dann existiert ein linearer, stetiger \emphdef{Spuroperator} $\gamma: H^{1,p}(\Omega) \to L^p(\Boundary \Omega)$ sodass
	\[
		\gamma(u) = u|_{\Boundary \Omega}
	\]
	für alle $u \in H^{1,p}(\Omega) \cap C^0(\_\Omega)$.

	Insbesondere gilt für $u \in H_0^{1,p}(\Omega)$ dann $\gamma(u) = 0$.
	Wegen Stetigkeit existiert also $C_\gamma > 0$, sodass
	\[
		\|\gamma(u)\|_{L^p(\Boundary \Omega)} \le C_\gamma \|u\|_{H^{1,p}(\Omega)}
	\]
	für alle $u \in H^{1,p}(\Omega)$.
	\begin{proof}
		Für $d = 2, p = 2$ siehe PDEMAS 13/14, allgemeiner in [Alt, 5.7].
	\end{proof}
\end{st}

\begin{df}[Sobolev Dualräume] \label{3.11}
	Für $1 \le p, q \le \infty, \f 1p + \f 1q = 1$ bezeichnen wir $H^{-m,q}(\Omega) := (H_0^{m,p}(\Omega))'$.
	Wir schreiben $H^{-m}(\Omega) := H^{-m, 2}(\Omega)$.
	\begin{note}
		Damit werden wir Differentialgleichungen betrachten können, deren rechte Seite (Quellterm) Funktionale statt Funktionen sind.
	\end{note}
\end{df}

\begin{note}[Stetigkeit für $d = 1$]
	\begin{itemize}
		\item
			Für $\Omega \subset \R$ ist $u \in H^1{\Omega}$ stetig (d.h. es existiert ein stetiger Repräsentant in der Äquivalenzklasse von $u$).
		\item
			Für $d > 1$ ist dies falsch: $H^1(\Omega)$ für $\Omega \subset \R^d$ enthält Funktionen mit Punktsingularitäten, z.b.
			\begin{align*}
				d = 2: \qquad u(x) &:= \log \log(\f 2{|x|}) \in H^1(B_1(0)), \\
				d \ge 3: \qquad u(x) &:= |x|^{-\beta} \in H^1(B_1(0)), \beta < \f{d-2}2.
			\end{align*}
	\end{itemize}
\end{note}

\begin{st}[Poincaré-Friedrich Ungleichung]
	Sei $\Omega \subset \R^d$ offen und beschränkt, $s := \diam(\Omega)$.
	Dann gilt
	\[
		\|v\|_{L^2(\Omega)} \le s |v|_{H^1(\Omega)}
	\]
	für alle $v \in H_0^1(\Omega)$.
	\begin{proof}
		Weil $C_0^\infty(\Omega)$ dicht in $H_0^1(\Omega)$ genügt es, die Ungleichung für $v \in C_0^\infty(\Omega)$ zu zeigen.
		ObdA sei $\Omega \subset [0,s]^d =: R$.
		Dann ist $v \in C_0^\infty(\Omega)$ durch $0$ fortsetzbar, also $v \in C_0^\infty(R)$.
		\[
			v(x_1, \dotsc, x_d) = \underbrace{v(0, x_2, \dotsc, x_d)}_{=0} +  \int_0^{x_1} \partial_{x_1} v(t, x_2, \dotsc, x_d) \di[t].
		\]
		Mit $x = (x_1, \dotsc, x_d)$ ist
		\begin{align*}
			|v(x)|^2
			&\le \Big(\int_0^{x_1} 1^2 \di[t] \Big) \Big( \int_0^{x_1} |\partial_{x_1} v(t,x_2,\dotsc, x_d)|^2 \di[t] \Big) \\
			&\le s \int_0^s |\partial_{x_1} v(t, x_2, \dotsc, x_d)|^2 \di[t].
		\end{align*}
		Die rechte Seite ist unabhängig von $x_1$, integriere daher bezüglich $x_1$:
		\[
			\int_0^s |v(x)|^2 \di[x_1]
			\le s^2 \int_0^s |\partial_{x_1} v(x)|^2 \di[x_1].
		\]
		Integration über andere Koordinaten liefert
		\[
			\int_R |v(x)|^2 \di[x]
			\le \int_R (\partial_{x_1} v(x))^2 \di[x]
			\le s^2 |v|_{H^1(\Omega)}.
		\]
	\end{proof}
	\begin{note}
		\begin{itemize}
			\item
				Die Ungleichung gilt bereits, wenn Nullrandwerte nur auf einem Teil des Randes vorliegen.
		\end{itemize}
	\end{note}
\end{st}

\begin{st}[Norm-Äquivalenz auf $H_0^{m}(\Omega)$] \label{3.13}
	Sei $\Omega \subset \R^d$ offen, beschränkt mit $\diam(\Omega) \le s$.
	Dann sind in $H_0^m(\Omega)$ die Norm $\|\argdot\|_{H^m(\Omega)}$ und die Seminorm $|\argdot|_{H^m(\Omega)}$ äquivalent.
	Insbesondere ist die Seminorm $|\argdot|_{H^m(\Omega)}$ eine Norm auf $H_0^m(\Omega)$.
	Genauer gilt
	\[
		|v|_{H^m(\Omega)}
		\le \|v\|_{H^m(\Omega)}
		\le (1 + s)^m |v|_{H^m(\Omega)}
	\]
	für $v \in H_0^m(\Omega)$.
	\begin{proof}
		Siehe Übung.
	\end{proof}
\end{st}


\Timestamp{2014-11-18}

\section{Schwache Lösungen für PDEs} \label{sec:3.2}

Sei das Poisson-Problem mit Nullrandwerten gegeben: $-\Laplace u = f$ in $\Omega$, $u = 0$ auf $\Gamma$.
Sei $u \in C^2(\Omega) \cap C^0(\_\Omega)$ klassische Lösung.
Multiplikation mit Testfunktion $v \in C_0^1(\Omega)$ und Integration ergibt
\begin{align*}
	\int_\Omega f(x) v(x) \di[x] &= - \int_\Omega \Laplace u(x) v(x) \di[x] \\
	&= \int_\Omega \Nabla u(x) \cdot \Nabla v(x) \di[x] - \int_{\Boundary \Omega} (\nabla u(x) \cdot n) \underbrace{v(x)}_{=0} \di[x].
\end{align*}
Also löst die klassische Lösung die „schwache Form“ der PDE
\begin{equation} \label{eq:3.1}
	\int_\Omega \Nabla u(x) \cdot \Nabla v(x) \di[x]
	= \int_\Omega f(x) v(x) \di[x],
	\qquad \forall v \in C_0^1(\Omega).
\end{equation}
\begin{itemize}
	\item
		Die Terme in \eqref{eq:3.1} ergeben Sinn für $u\in C_0^1(\Omega)$ oder $u \in H_0^1(\Omega)$.
		Es liegt nahe, nach „schwachen Lösungen“ zu suchen, d.h. nach Lösungen von \eqref{eq:3.1}.
	\item
		Die klassiche Lösung ist insbesondere eine schwache Lösung.
	\item
		Für allgemeine $f$ (z.B. unstetig) kann es sein, dass keine klassische Lösung existiert, dafür aber eine schwache Lösung $u \in H_0^1(\Omega)$.
	\item
		\eqref{eq:3.1} kann durch die Bilinearform $a(u,v) := \int_\Omega \Nabla u \cdot \Nabla v \di[x]$ und die Linearform $l(v) := \int_\Omega fv \di[x]$ umgeschrieben werden:
		\begin{equation} \label{eq:3.2}
			a(u,v) = l(v) \qquad \forall v \in H_0^1(\Omega).
		\end{equation}
	\item
		Im Folgenden wird Lösungstheorie unter anderem für \eqref{eq:3.2} untersucht als Grundlage für FEM in \ref{sec:3.3}.
	\item
		In diesem Abschnitt seien $V, W$ stets Hilberträume mit Skalarprodukte $\<\argdot, \argdot\>_V, \<\argdot, \argdot\>_W$ und Normen $\|\argdot\|_V, \|\argdot\|_W$.
		Das Subkript wird auch weggelassen, wenn keine Verwechslungsgefahr besteht.
\end{itemize}

\begin{df}[Lineare Operatoren] \label{3.14}
	$A: V \to W$ ist \emphdef{linearer stetiger Operator}, falls $\|A\|_{L(V,W)} := \sup_{v \in V\setminus \Set 0} \f{\|Av\|_W}{\|u\|_V} < \infty$.
	Weiter ist
	\[
		L(V,W) := \Set{A: V \to W & \text{$A$ ist linear und stetig}}
	\]
	der \emphdef{Raum der stetigen Operatoren} mit Norm $\|\argdot\|_{L(V,W)}$.
	\begin{note}
		Also ist $V' = L(V, \R)$, d.h. der Dualraum ist der Raum der stetigen linearen Abbildungen nach $\R$.
	\end{note}
\end{df}

\begin{df}[Adjungierter Operator] \label{3.15}
	Zu $A \in L(V,W)$ definieren wir den (Hilbertraum)-adjungierten Operator $A^*: W \to V$ durch
	\[
		\<Av,w\>_W = \<v,A^*w\>_V
	\]
	für alle $v \in V, w \in W$.
	\begin{note}
		\begin{itemize}
			\item
				Es gilt $A^* \in L(W,V), \|A^*\|_{L(W,V)} = \|A\|_{L(V,W)}$.
			\item
				Falls $A$ invertierbar und $A^{-1}$ stetig, dann ist auch $\|(A^{-1})^*\|_{L(V,W)} = \|A^{-1}\|_{L(W,V)}$ (siehe auch [Alt 10.1]).
		\end{itemize}
	\end{note}
\end{df}

\begin{df}[stetige Bilinearform] \label{3.16}
	Eine Bilinearform $A \in V \times W \to \R$ heißt \emphdef{stetig} mit \emphdef{Stetigkeitskonstante} $\gamma \in \R$, wenn
	\[
		\gamma = \sup_{v \in V\setminus \Set 0} \sup_{w \in W \setminus \Set 0} \f{a(v,w)}{\|v\|\|w\|} < \infty.
	\]
	\begin{note}
		\begin{itemize}
			\item
				Also ist das Skalarprodukt $\<\argdot, \argdot\>_V: V \times V \to \R$ stetig mit $\gamma = 1$ wegen Cauchy-Schwartz $\<v, w\>_V \le \|v\|\|w\|$.
		\end{itemize}
	\end{note}
\end{df}

\begin{st}[Projektionssatz] \label{3.17}
	Sei $W \subset V$ abgeschlossener Unterraum.
	Dann existiert genau ein Projektionsoperator $P: V \to W$ mit
	\[
		\<v - Pv, w\> = 0
	\]
	für alle $v \in V, w \in W$.

	$P$ ist ein stetiger, linearer Operator, $P \in L(V,W)$, genannt \emphdef{orthogonale Projektion}.
	\begin{proof}
		siehe PDEMAS 13/14, oder Alt
	\end{proof}
	\begin{note}
		Die Orthogonale Projektion erfüllt die Optimalitätseigenschaft: $\|v - Pv\|^2 = \min_{w\in W} \|v - w\|^2$.
	\end{note}
\end{st}

\begin{st}[Rieszscher Darstellungssatz] \label{3.18}
	Sei $V$ ein Hilbertraum und $J: V \to V'$ definiert durch
	\begin{math}
		(J(v))(w) &:= \<v, w\>_V & \forall v, w \in V.
	\end{math}
	Dann ist $J$ eine stetige, lineare, bijektive Isometrie.

	Insbesondere existiert zu $l \in V'$ ein eindeutiger Riesz-Repräsentant $v_l := J^{-1}(l) \in V$ mit $l(\argdot) = \<v_l, \argdot\>_V$.
	\begin{proof}
		Siehe PDEMAS 13/14 oder Alt.
	\end{proof}
\end{st}

\begin{st}[Repräsentierender Operator] \label{3.19}
	Sei $a: V \times W \to \R$ eine stetige Bilinearform mit Stetigkeitskonstante $\gamma$.
	Dann existiert $A \in L(V,W)$ mit $\|A\|_{L(V,W)} = \gamma$ und
	\begin{equation} \label{eq:3.3}
		\<Av, w\>_W = a(v,w)
	\end{equation}
	für alle $(v,w) \in V \times W$.
	\begin{note}
		Die Umkehrung gilt ebenfalls.
	\end{note}
	\begin{proof}
		Es gilt $a(v, \argdot) \in W'$ für alle $v \in V$, also existiert mit Riesz \ref{3.18} ein $w_v \in W$ mit
		\[
			\<w_v, \argdot\>_W = a(v,\argdot).
		\]
		Definiere $A: V \to W$ durch $Av := w_v$ für alle $v \in V$.
		Dann ist \eqref{eq:3.3} klar.
		Linearität von $A$ folgt aus Linearität von $a(v,\argdot)$ bezüglich dem ersten Argument.
		Betracht die Stetigkeit, es gilt
		\[
			\|Av\|_W^2
			= \<Av, Av\>_W
			= \<w_v, Av\>
			= a(v, Av)
			\le \gamma \|v\| \|Av\|,
		\]
		also $\f{\|Av\|}{\|v\|} \le \gamma$ und insbesondere $\|A\| = \sup_{v \neq 0} \f{\|Av\|}{\|v\|} \le \gamma < \infty$.
		Angenommen $\|A\| < \gamma$.
		Dann existiert $v \neq 0, w \neq 0$ mit $\f{a(v,w)}{\|v\|\|w\|} \in (\|A\|, \gamma)$.
		Es folgt
		\[
			\|A\|\|v\|\|w\|
			\ge \<Av, w\>
			= a(v,w)
			> \|A\|\|v\|w\|,
		\]
		ein Widerspruch, also $\|A\| = \gamma$.
	\end{proof}
\end{st}

\begin{df}[Koerzivität] \label{3.20}
	Eine Bilinearform $a: V \times V  \to \R$ heißt \emphdef{koerziv} mit \emphdef{Koerzivitätskonstante} $\alpha$, wenn
	\[
		\alpha := \inf_{v \in V \setminus \Set 0} \f{a(v,v)}{\|v\|^2} > 0.
	\]
	\begin{note}
		\begin{itemize}
			\item
				Also ist das Skalarprodukt koerziv mit $\alpha = 1$:
				\[
					\inf_{v \neq 0} \f{\<v,v\>}{\|v\|^2}
					= \inf_{v \neq 0} 1
					= 1 > 0.
				\]
			\item
				Eine Bilinearform ist koerziv genau dann, wenn der symmetrische Anteil $a_S(V,W)$ koerziv ist:
				\[
					a_S(V,W) := \f 12 (a(v,w) + a(w,v)),
				\]
				denn $a(u,u) = a_S(u,u)$.
			\item
				$\gamma$ und $\alpha$ lassen sich durch geeignete Eigenwertprobleme/Singulärwertprobleme berechnen (siehe Übung).
			\item
				Es gilt stets $\alpha \le \gamma$.
		\end{itemize}
	\end{note}
\end{df}

\begin{df}[Bilinearform/Linearform für PDEs zweiter Ordnung] \label{3.21}
	Sei $\Omega \subset \R^d$ beschränkt und das Null-RWP für eine PDE zweiter Ordnung gegeben, d.h.
	\begin{math}
		-\Nabla (A \Nabla u) + \Nabla (b \cdot u) + c u &= f && \text{in $\Omega$}, \\
		u &= 0 && \text{auf $\Gamma$},
	\end{math}
	wobei $A = (a_{ij})_{ij=1}^d \in (L^\infty(\Omega))^{d\times d}, b = (b_i)_{i=1}^d \in (L^\infty(\Omega))^d, c \in L^\infty(\Omega), f \in L^2(\Omega)$.

	Dann definieren wir die Bilinearform und Linearform
	\begin{math}[numbered] \label{eq:3.4}
		% a(u,v) &:= \int_\Omega (Au)^T v + (b\nabla v) u + cuv
		a(u,v) &:= \int_\Omega (A \Nabla u) \cdot \Nabla v - (b \cdot \Nabla v) u + c u v \di[x], \\
		l(v) &:= \int_\Omega f v \di[x]
	\end{math}
	für $u, v \in H^1(\Omega)$.
\end{df}

\begin{st}[Stetigkeit und Koerzivität, $b=0$, $c = 0$] \label{3.22}
	Sei $b = 0, c = 0$, $A$ gleichmäßig elliptisch, d.h. für ein $\tilde \alpha > 0$ gilt $z^T A(x) z \ge \tilde \alpha \|z\|^2$ für alle $x \in \Omega, z \in \R^d$ und $A$ gleichmäßig beschränkt, d.h. für $C > 0$ (und beliebiger Matrixnorm) gilt $\|A(x)\| \le C$ für alle $x \in \Omega$.

	Dann ist die Bilinearform aus \ref{3.21} stetig auf $H^1(\Omega)$ und koerziv auf $H_0^1(\Omega)$.
	\begin{proof}
		\begin{math}
			a(u,v)
			&= \int_\Omega (A \Nabla u) \cdot \Nabla v \di[x] \\
			&\le \int_\Omega \underbrace{\|A(x) \Nabla u(x)\|}_{\le \|A\| \|\Nabla u\| \le C \|\Nabla u\|} \|\Nabla v(x)\| \di[x] \\
			&\le C \int_\Omega \|\Nabla u\| \|\Nabla v\| \di[x] \\
			&\le C \Big( \int_\Omega \|\Nabla u\|^2 \Big)^{\f 12} \Big(\int_\Omega \|\Nabla v\|^2 \Big)^{\f 12} \\
			&\le C \sqrt{\int_\Omega \|\Nabla u\|^2 + \|u\|^2} \sqrt{\int_\Omega \|\nabla v\|^2 + \|v\|^2} \\
			&= C \|u\|_{H^1} \|v\|_{H^1},
		\end{math}
		also ist $a(\argdot, \argdot)$ stetig auf $H^1(\Omega) \times H^1(\Omega)$.
		Da $A$ ohne Einschränkung symmetrisch und gleichmäßig elliptisch:
		\begin{align*}
			a(u,u) &= \int_\Omega (A \Nabla u)^T \Nabla u \di[x]
			= \int_\Omega (\Nabla u)^T A \nabla u \di[x] \\
			&\ge \int_\Omega \tilde \alpha \|\Nabla u\|^2
			= \tilde \alpha |u|_{H^1(\Omega)}^2
			\intertext{
				für $u \in H_0^1(\Omega)$ folgt mit Normäquivalenz \ref{3.13}
			}
			&\ge \f {\tilde \alpha}{(1+s)^2} \|u\|_{H^1}^2
		\end{align*}
		mit $\diam(\Omega) \le s$.
		Es folgt
		\[
			\alpha := \inf_{u \neq 0} \frac{a(u,u)}{\|u\|_{H^1}^2}
			\ge \f{\tilde \alpha}{(1+s)^2} \frac{\|u\|_{H^1}^2}{\|u\|_{H_1}^2}
			>0,
		\]
		also ist $a(\argdot, \argdot)$ koerziv auf $H_0^1(\Omega)$.
	\end{proof}
	\begin{note}
		\begin{itemize}
			\item
				$a(\argdot, \argdot)$ ist nicht koerziv auf $H^1(\Omega)$:
				Wähle $u \in H^1(\Omega)$ konstant $u = k \neq 0$, dann ist
				\[
					\f{a(u,u)}{\|u\|_{H^1(\Omega)}^2}
					= \f{0}{\|u\|_{H^1(\Omega)}^2}
					= 0.
				\]
			\item
				$a(\argdot, \argdot)$ ist stetig auf $H_0^1(\Omega)$, Stetigkeit vererbt sich auf Teilräume:
				\begin{align*}
					\sup_{u \in H_0^1(\Omega)} \sup_{v \in H_0^1(\Omega)} \f{a(u,v)}{\|u\|\|v\|}
					\le \sup_{u \in H^1(\Omega)} \sup_{v \in H^1(\Omega)} \f{a(u,v)}{\|u\|\|v\|}
					= \gamma < \infty
				\end{align*}
			\item
				$a(\argdot, \argdot)$ ist koerziv auf Teilräumen von $H_0^1(\Omega)$.
				Koerzivität vererbt sich auf Teilräume:
				Sei $W \subset H_0^1$, dann gilt
				\[
					\inf_{u \in W} \f{a(u,u)}{\|u\|^2}
					\ge \inf_{u \in H_0^1(\Omega)} \f{a(u,u)}{\|u\|^2}
					= \alpha
					> 0.
				\]
			\item
				Falls $A$ symmetrisch (ist ohne Einschränkung stets der Fall), so ist $a(\argdot, \argdot)$ symmetrische Bilinearform.
			\item
				Ähnliche Aussage wie \ref{3.22} gilt falls $b \neq 0$ und $c > 0$ hinreichend groß.
		\end{itemize}
	\end{note}
\end{st}

\begin{note}[Stetigkeit von $l$]
	\begin{itemize}
		\item
			$l(\argdot)$ aus \eqref{eq:3.4} ist offensichtlich linear, $l: H^1 \to \R$.
		\item
			Ist $f \in L^2(\Omega)$, dass ist
			\[
				l(v) = \int_\Omega fv \di[x]
				= \<f, v \>_{L^2}
				\le \|f\|_{L^2} \|v\|_{L^2}
				\le \|f\|_{H^1}\|v\|_{H^1},
			\]
			also ist $l(\argdot)$ stetig auf $H^1$ und $H_0^1(\Omega)$.
		\item
			Falls $v \in H_0^1(\Omega)$, sind sogar manche $f \not\in L^2(\Omega)$, solange $l \in H^{-1}(\Omega)$.
	\end{itemize}
\end{note}

\Timestamp{2014-11-21}

\begin{df}[Energie-Skalarprodukt] \label{3.23}
	Sei $a$ eine stetige Bilinearform und koerziv.
	Dann bildet der symmetrische Anteil ein Skalarprodukt, das sogenannte \emphdef{Energieskalarprodukt}
	\[
		\<u,v\>_a :=  \f 12\big(a(u,v) + a(v,u)\big).
	\]
	\begin{proof}
		Bilinearität und Symmetrie sind offensichtlich.
		Positivität folgt aus Koerzivität:
		Für $u \neq 0$ gilt
		\[
			\<u,u\>_a
			= \f 12 (a(u,u) + a(u,u))
			= a(u,u)
			\ge \underbrace{\alpha}_{>0} \|u\|^2
			> 0.
		\]
	\end{proof}
	\begin{note}
		Es ergibt sich hiermit die für Fehleranalyse nützliche \emphdef{Energienorm} $\|u\|_a := \sqrt{\<u,u\>_a}$.
	\end{note}
\end{df}

\begin{df}[Schwache Lösung] \label{3.24}
	Seien $a(\argdot, \argdot), l(\argdot)$ wie in \eqref{eq:3.4}.
	Wir nennen $u \in H_0^1(\Omega)$ \emphdef{schwache Lösung} des RWP, falls
	\begin{equation} \label{eq:3.5}
		a(u,v) = l(v)
	\end{equation}
	für alle $v \in H_0^1$.
\end{df}

\begin{st} \label{3.25}
	Sei $u \in C^2(\Omega) \cap C^0(\_\Omega)$ klassische Lösung des RWP und $f \in C^0$.
	Dann ist $u$ auch schwache Lösung.
	\begin{proof}
		Wie zu Beginn des Abschnitts:
		Multiplizieren mit Testfunktion, Integration, partielle Integration.
	\end{proof}
\end{st}

\begin{st}[Wohlgestelltes schwaches Poisson-RWP] \label{3.26}
	Betrachte \eqref{eq:3.5} für $A(x) = I, c(x) = 0, b(x) = 0$, also
	\begin{align*}
		a(u,v) &= \int_\Omega \nabla u \cdot \nabla v \di[x], &
		l(v) &= \int_\Omega f v \di[x].
	\end{align*}
	Dann existiert für $f \in L^2(\Omega)$ eine eindeutige schwache Lösung $u \in H_0^1(\Omega)$, beschränkt durch die rechte Seite:
	\[
		\|u\|_{H^1(\Omega)} \le \f 1{\alpha} \|f\|_{L^2}.
	\]
	\begin{note}
		Dies ist eine Stabilitätsaussage, daher nennt man $\alpha$ auch \emphdef{Stabilitätskonstante}.
	\end{note}
	\begin{proof}
		$A(x)$ ist gleichmäßig elliptisch mit Elliptizitätskonstante $\tilde \alpha = 1$ und gleichmäßig beschränkt.
		Nach \ref{3.22} ist $a(u,v)$ koerziv und stetig auf $H_0^1(\Omega)$ mit Koerzivitätskonstante $\alpha = \f 1{(1+s)^2}$.
		Da $a(\argdot, \argdot)$ auch symmetrisch, ist $a(\argdot, \argdot)$ ein Energieskalarprodukt und
		\[
			a(u,u)
			= \<u,u\>_a
			= \int_\Omega \nabla u \cdot \nabla u \di[x]
			= |u|_{H^1(\Omega)}^2.
		\]
		Mit der Normäquivalenz \ref{3.13} gilt
		\begin{equation} \label{eq:3.6}
			\|v\|_{H^1(\Omega)}
			\le (1+s) |v|_{H^1(\Omega)}
		\end{equation}
		für alle $v \in H_0^1$.
		Also ist $(H_0^1(\Omega), \|\argdot\|_{H^1})$ ein Hilbertraum genau dann, wenn $(H_0^1(\Omega), |\argdot|_{H^1})$ Hilbertraum ist.
		Nach Riesz existiert eine (bijektive) Isometrie $J: H_0^1 \to (H_0^1)' = H^{-1}$ bezüglich $|\argdot|_{H^1}$.
		Mit $f \in L^2(\Omega)$ ist $l \in H^{-1}$ also mit \ref{3.18} $u := J^{-1} l$ ein Element mit
		\[
			a(u,v)
			= \<u,v\>_a
			= \int_\Omega \nabla u \cdot \nabla v \di[x]
			= \<J^{-1}l, v\>_a
			= l(v)
		\]
		und wegen Isometrie
		\begin{equation} \label{eq:3.7}
			|u|_{H^1} = \|l\|_{H^{-1}}
			= \sup_{v} \f{l(v)}{|v|_{H^1}}.
		\end{equation}
		Für die $H^1$-Norm folgt dann
		\begin{align*}
			\|u\|_{H^1}
			&\stack{\ref{3.6}}\le (1+s)|u|_{H^1} \\
			&\stack{\ref{3.7}}= (1+s) \sup_v \f{l(v)}{|v|_{H^1}} \\
			&\stack{\ref{3.6}}\le \sup_{v} \f{l(v)}{\|v\|_{H^1} \f 1{(1+s)}}
			= (1+s)^2 \sup_v \f{l(v)}{\|v\|_{H^1}}
			= \f 1{\alpha} \|l\|_{H^{-1}}.
		\end{align*}
	\end{proof}
\end{st}

\begin{note}
	\begin{itemize}
		\item
			Die „schwache Form“ für $-\Laplace u$ ist also $a: H_0^1 \to H^{-1}, u \mapsto a(u,\argdot), u \in H_0^1$.
		\item
			Die Riesz-Abbildung $J^{-1}$ (bezüglich des Energieskalarproduktes) ist also der Lösungsoperator für das Poisson-RWP.
		\item
			Analoge Argumentation liefert Wohlgestelltheit für das RWP \ref{3.24} mit $A(x)$ symmetrisch, gleichmäßig elliptisch und gleichmäßig beschränkt, $b = 0$, $c \ge 0$.
			Dann ist $a(\argdot, \argdot)$ wieder ein Skalarprodukt, also das Energieskalarprodukt.
	\end{itemize}
\end{note}

\begin{kor}[Stetige Abhängigkeit von Daten] \label{3.27}
	Für eine PDE in schwacher Form mit $a(\argdot, \argdot)$ bilinear, $l \in V'$, $a(u,v) = l(v)$ für $v \in V$ existiere für alle $l \in V'$ eine eindeutige Lösung $u \in V$ mit Schranke $\|u\| \le \f 1{\alpha} \|l\|_{V'}$.

	Dann ist $u$ stetig von der rechten Seite abhängig, d.h. für $\_u$ zu Daten $\_l \in V'$ gilt
	\[
		\|u-\_u\| \le \f 1{\alpha} \|l - l'\|.
	\]
	\begin{proof}
		$w := u - \_u$ löst $a(u,v) = l(v) - \_l(v)$.
		Die Behauptung folgt aus der Annahme.
	\end{proof}
\end{kor}

\begin{st}[Schwache Form als Minimierungsproblem] \label{3.28}
	Sei $a: V \times V \to \R$ bilinear, stetig, koerziv und symmetrisch.
	Dann ist die schwache Form \eqref{eq:3.5} äquivalent zu einem Minimierungsproblem, d.h.
	\begin{math}
		& & &\text{$u$ löst $a(u,v) = l(v)$ für alle $v \in V$} \\
		&\iff&
		&\text{$u = \argmin_{v\in V} \big( \f 12 a(v,v) - l(v)$} \big).
	\end{math}
	\begin{proof}
		Siehe Übung.
	\end{proof}
	\begin{note}
		\begin{itemize}
			\item
				Für nichtsymmetrisches $a(\argdot, \argdot)$ existiert keine solche Aussage.
			\item
				Mit diesem Satz wird nocheinmal klar, dass Vollständigkeit von $V$ essentiell für die Existenz einer Lösung ist:
				Über $C_0^1$ wird das Funktional im Allgemeinen kein Minimum annehmen, obwohl Minimalfolgen existieren.
		\end{itemize}
	\end{note}
\end{st}

\begin{note}
	Allgemeine koerzive Probleme (z.B. nicht-symmetrische) erlauben eine Wohlgestellheitsaussage mit „Satz von Lax-Milgram“.
	Wir überspringen diesen hier, weil er aus dem allgemeineren Nečas-Theorem folgt.
\end{note}

\begin{note}[Verallgemeinerte Randbedingungen]
	\begin{enumerate}[i)]
		\item
			Inhomogenes Dirichlet-RWP:
			\begin{math}[numbered] \label{eq:3.8}
				-\Laplace u &= f && \text{in $\Omega$}, \\
				u &= g && \text{auf $\Gamma$}.
			\end{math}
			Sei $g$ derart, dass $\_g \in C^2(\Omega) \cap C^0(\_\Omega)$ existiert mit $\_g = g$ auf $\Gamma$.
			Dann gilt
			\begin{math}
				&\text{$u$ löst \eqref{eq:3.8}} &
				&\iff &
				&\text{$\_u := u - \_g$ löst \eqref{eq:3.9}},
			\end{math}
			wobei
			\begin{math}[numbered] \label{eq:3.9}
				-\Laplace \_u &= - \Laplace (u - \_g) = f - \Laplace \_g && \text{in $\Omega$}, \\
				\_u &= 0 && \text{auf $\Gamma$}.
			\end{math}
			Löse also das homogene Problem für $\_u$ und es folgt, dass $u := \_u + \_g$ das ursprüngliche RWP löst.
			Insbesondere folgt Existenz und Eindeuktigkeit schwacher Lösungen für inhomogene Dirichlet-RWP.
		\item
			Gemischte Dirichlet- und Neumann-Bedingungen:
			Sei $\Gamma = \Gamma_N \cup \Gamma_D$, betrachte
			\begin{math}
				-\Laplace u &= f && \text{in $\Omega$}, \\
				u &= 0 && \text{auf $\Gamma_D$}, \\
				\nabla u \cdot n &= g_N && \text{auf $\Gamma_N$}.
			\end{math}
			Der Lösungs-/Testraum ist
			\[
				V := H_{\Gamma_D}^1
				:= \Set{v \in H^1 & v|_{\Gamma_D} = 0},
			\]
			also $H_0^1(\Omega) \subset H_{\Gamma_D}^1(\Omega) \subset H^1(\Omega)$.
			Multipliziere mit Testfunktion, integriere und nutze partielle Integration:
			\begin{align*}
				\int_\Omega fv
				&= \int_\Omega \nabla u \cdot \nabla v - \int_{\Boundary \Omega} (\nabla u \cdot n) v \\
				&= \int_\Omega \nabla u \cdot \nabla v - \int_{\Gamma_D} (\nabla u \cdot n) \underbrace{v}_{=0} - \int_{\Gamma_N} \underbrace{(\nabla u \cdot n) v}_{= g_N}.
			\end{align*}
			Damit ist die schwache Form durch
			\[
				\int_\Omega \nabla u \cdot \nabla v
				= \int_\Omega fv + \int_{\Gamma_N} g_Nv
			\]
			gegeben.
			Dirichlet-Randbedingungen werden also in die Konstruktion von $V$ eingebaut („wesentliche“, oder „essentielle“ Randbedingungen).
			Neumann-Randebedingungen werden über Zusatzterme in die schwache Form eingebaut („natürliche Randbedingungen“).
	\end{enumerate}
\end{note}

Unter welchen Bedingungen ist $u \in H^m(\Omega), m > 1$, oder gar $u \in C^\infty(\Omega)$?
Dies führt auf die sogenannte \emphdef{Regularitätstheorie}.

\begin{df}[$H^s(\Omega)$-Regularität] \label{3.29}
	Sei $H_0^1(\Omega) \subset V \subset H^1(\Omega)$.
	Eine PDE in schwacher Form
	\[
		a(u,v) = \<f, v\>_{L^2(\Omega)},
		\qquad \forall v \in V
	\]
	mit $a(\argdot, \argdot)$ bilinear, stetig und koerziv auf $V$ heißt \emphdef{$H^s$-regulär}, wenn es eine Konstante $C_R \ge 0$ sodass zu jedem $f \in H^{s-2}(\Omega)$ eine Lösung $u \in H^s(\Omega)$ existiert mit
	\[
		\|u\|_{H^s}
		\le C_R \|f\|_{H^{s-2}(\Omega)}.
	\]
\end{df}

\begin{ex*}
	\begin{itemize}
		\item
			Sei $d = 2$, $\Omega = \Set{ x \in \R^2 & 1 \le \|x\| \le 2 }$.
			$u(x) = \ln(\|x\|)$ ist klassische Lösung des inhomogonen RWP
			\begin{equation*}
				\begin{aligned}
					-\Laplace u &= 0 & \ &\text{in $\Omega$}, \\
					u &= 0 & \ &\text{auf $\Boundary B_1(0)$}, \\
					u &= \ln 2 & \ &\text{auf $\Boundary B_2(0)$}
				\end{aligned}
			\end{equation*}
			Es gilt $u \in C^\infty(\_\Omega)$.
			Mit \ref{3.25}, \ref{3.26} (genauer: die analogen Aussagen für inhomogene RWP) folgt, dass $u$ eindeutige schwache Lösung ist, insbesondere $u \in H^m(\Omega), m \in \N$ wegen Beschränktheit von $\Omega$ also alle $\partial^\beta u \in L^2(\Omega)$.
			Das Problem ist damit $H^s(\Omega)$-regulär für alle $s \in \N$.
		\item
			Sei $\alpha \in (0,2), \Omega := \Set{ x = \Vector{r \cos \phi & r \sin \phi} & r \in (0,1), \phi \in (0,\alpha\pi) }$.
			\begin{math}
				u(x) &= \|x\|^{\f 1\alpha} \sin(\f {\phi(x)}{\alpha}), &
				\phi(x) &= \arctan \f{x_2}{x_1}
			\end{math}
			ist klassische Lösung von
			\begin{math}
				-\Laplace u &= 0 && \text{in $\Omega$}, \\
				u &= \sin(\frac{\phi(x)}{\alpha}) && \text{auf $\Gamma_2$}, \\
				u &= 0 && \text{auf $\Gamma_1 \cup \Gamma_3$}.
			\end{math}
			Für $\alpha \le 1$ ist $u \in H^2(\Omega)$, für $\alpha > 1$ ist $u \not\in H^2(\Omega)$.

			Regulärität häng also von der Geometrie ab.
	\end{itemize}
\end{ex*}

\begin{st}[Satz von Friedrich] \label{3.30}
	Sei $\Omega \subset \R^d$ offen, beschränkt mit glattem Rand (mindestens $C^2$) oder ein konvexes Lipschitzgebiet, $f \in L^2(\Omega)$.
	Dann ist das Poisson-Problem $H^2(\Omega)$-regulär.
	\begin{proof}
		Für glatte Gebiete z.B. in [Alt, 10.17, 10.18].
	\end{proof}
	\begin{proof}
		Verallgemeinerung:
		Falls $f \in H^{s-2}(\Omega)$, $\Omega$ hat einen $C^s$-Rand, dann ist $u \in H^s(\Omega)$.
	\end{proof}
\end{st}

\begin{df}[Inf-sup-Stabilität] \label{3.31}
	Eine stetige Bilinearform $a: V \times W \to \R$ heißt \emphdef{inf-sup stabil} mit inf-sup-Konstante $\beta$, falls
	\[
		\beta := \inf_{v \in V \setminus \Set 0} \sup_{w \in W \setminus \Set 0} \f{a(v,w)}{\|v\|\|w\|} > 0.
	\]
\end{df}

\begin{st}[Charakterisierung von inf-sup-Stabilität] \label{3.32}
	Sei $a: V \times W \to \R$ stetige, inf-sup stabile Bilinearform mit inf-sup Konstante $\beta$ und $A: V \to W$ repräsentierender Operator nach \ref{3.19}.
	Dann gilt
	\begin{enumerate}[i)]
		\item
			$A$ ist \emphdef{suprimierender Operator}:
			\[
				\sup_{v \in W\setminus \Set 0} \f{a(v,w)}{\|w\|}
				= \f{a(v,Av)}{\|Av\|}
			\]
			für alle $v \in V \setminus \Set 0$.
		\item
			$\beta = \int_{v \neq 0} \f {a(v,Av)}{\|v\| \|Av\|} = \inf_{v \neq 0} \f{\|Av\|}{\|v\|}$.
		\item
			$A$ ist injektiv, d.h. $\forall v \neq 0 : Av \neq 0$.
		\item
			$\forall v \in V \setminus \Set 0 \exists w \in W \setminus \Set 0 : \beta \|v\|\|w\| \le a(v,w)$.
	\end{enumerate}
\Timestamp{2014-11-25}
	\begin{note}
		Analog zu ii) gilt für die Stetigkeitskonstante $\gamma = \sup_{v \in V \setminus \Set 0} \f{\|Av\|}{\|v\|}$.
	\end{note}
	\begin{proof}
		\begin{enumerate}[i)]
			\item
				$\ge$ ist klar wegen Supremum, zeige $\le$.
				Sei $w \in W, v \in V, w \neq 0 \neq v$.
				Dann ist
				\begin{math}
					\f {a(v,w)}{\|w\|}
					= \f{\<Av, w\>}{\|w\|}
					\le \f{\|Av\|\|w\|}{\|w\|}
					= \f{\|Av\|\|Av\|}{\|Av\|}
					= \f{\<Av, Av\>}{\|Av\|}
					= \f{a(v,Av)}{\|Av\|}.
				\end{math}
				Gleiches gilt für $\sup_{w\in W\setminus \Set 0} \f{a(v,w)}{\|w\|} = \f {a(v,Av)}{\|Av\|}$.
			\item
				\begin{math}
					\beta
					= \inf_{v\neq 0} \sup_{u\neq 0} \f{a(v,w)}{\|v\|\|w\|}
					\stack{i)}= \inf_{v\neq 0} \f{a(v,Av)}{\|v\|\|Av\|}
					= \inf \f{\<Av,Av\>}{\|v\|\|Av\|}.
				\end{math}
			\item
				Angenommen $A$ ist nicht injektiv, d.h. $\exists v \in V \setminus \Set 0$ mit $Av = 0$.
				Dann ist
				\begin{math}
					0
					= \<Av, w\>
					= a(v,w)
					\ge \beta \|v\|\|w\|
					> 0
				\end{math}
				für alle $w \in W \setminus \Set 0$.
				Ein Widerspruch, also ist $A$ injektiv.
			\item
				nach ii) gilt $\beta = \inf_{v\in V \setminus \Set 0} \f{a(v,Av)}{\|v\|\|Av\|}$, also für alle $v \in V \setminus \Set 0$:
				\begin{math}
					\beta \|v\|\|Av\| \le a(v,Av),
				\end{math}
				also $w = Av \neq 0$.
		\end{enumerate}
	\end{proof}
\end{st}

\begin{st}[Beziehung von $\alpha$ und $\beta$] \label{3.33}
	Sei $a: V \times V \to \R$ stetig, bilinear.
	Falls $a$ koerziv mit Konstante $\alpha$, dann ist $\alpha$ inf-sup-stabil mit Konstante $\beta$ sodass $\beta \ge \alpha$.
	\begin{proof}
		\begin{math}
			\beta
			= \inf_{v\neq 0} \sup_{w\neq 0} \f{a(v,w)}{\|v\|\|w\|}
			\ge \inf \f{a(v,v)}{\|v\|^2}
			= \alpha
			> 0.
		\end{math}
	\end{proof}
\end{st}

\begin{note}
	\begin{itemize}
		\item
			Man kann zeigen:
			Falls $a(\argdot, \argdot)$ koerziv, symmetrisch, dann ist $a$ inf-sup-stabil mit $\beta = \alpha$.
			\begin{proof}
				Siehe IANS Report 4/2011, B. Haasdonk.
			\end{proof}
		\item
			$\beta$ ist immer die „bessere“ Konstante, da $\beta \ge \alpha$.
			Es ergeben sich schärfere Beschränktheitsaussagen für schwache Lösungen von PDEs.
		\item
			inf-sup-Stabilität übertragt sich nicht auf Teilräume.
			Dies macht Schwierigkeiten für FEM-Diskretisierungen (siehe Übung).
	\end{itemize}
\end{note}

\begin{st} \label{3.34}
	Sei $a(v,w)$ stetige Bilinearform mit zugehörigem Operator $A: V \to W$ gemäß \ref{3.19}.
	Dann sind äquivalent
	\begin{enumerate}[i)]
		\item
			$a(v,w)$ erfüllt $\inf_{v\neq 0} \sup_{w\neq 0} \frac{a(v,w)}{\|v\|\|w\|} \ge \beta$ und $\forall w \in W \setminus \Set 0$ existiert ein $v \in V$ sodass $a(v,w) \neq 0$.
		\item
			$A$ ist invertierbar mit stetiger Inversen
			\begin{math}
				\|A^{-1}\|_{L(W,V)} \le \frac{1}{\beta}.
			\end{math}
	\end{enumerate}
	\begin{proof}
		\begin{seg}{\ProofImplication[1][2]}
			Zeige, dass $A$ bijektiv ist.
			$A$ ist bereits injektiv nach \ref{3.32} ii).
			Zeige, dass $\im A$ vollständig ist und nutze den Projektionssatz für die Surjektivität.

			Sei $w_n = Av_n$ eine Cauchy-Folge in $\im A$.
			$W$ ist vollständig, also existiert $w \in W$ mit $\lim_{n\to \infty} w_n = w$.
			Es verbleibt $w \in \im A$ zu zeigen.
			Mit der inf-sup-Eigenschaft \ref{3.32} ii) folgt
			\begin{math}
				\beta &= \inf_{v\neq 0} \frac{\|Av\|}{\|v\|}
				&&\implies&
				\forall v \in V: \beta \|v\| \le \|Av\|,
			\end{math}
			also
			\begin{math}
				\beta \|v_n - v_m\| \le A(v_n - v_m)\|
				= \|w_n - w_m\|
				\to 0
			\end{math}
			für $n,m \to \infty$.
			Somit ist $v_n$ Cauchy-Folge in $V$ und es existiert $v \in V$ mit $v = \lim_{n\to \infty} v_n$.
			Stetigkeit von $A$ ergibt
			\begin{math}
				Av
				= \lim_{n\to\infty} Av_n
				= \lim_{n\to\infty} w_n
				= w
			\end{math}
			und somit $w \in \im A$ und $\im A$ abgeschlossen.

			Nun zur Surjektivität.
			Angenommen $\_w \in W$ mit $\_w \not \in \im A$.
			Sei $p: W \to \im A$ orthogonale Projektion gemäß \ref{3.17}, d.h. $\_{\_w} = \_w - P\_w \neq 0$ und $\<\_{\_w}, Av\> = 0$ für alle $w \in \im A$.
			Also für alle $v \in V$
			\begin{math}
				0
				= \<\_{\_w}, Av\>
				= \<Av, \_{\_w}\>
				= a(v,\_{\_w}),
			\end{math}
			ein Widerspruch zur zweiten Annahme in i).
			Somit ist $A$ surjektiv.

			Da $A$ bijektiv, existiert die Inverse $A^{-1}$ und für $v \in V$ folgt mit $\beta \|v\| \le \|Av\|$ auch $\beta \|A^{-1}w\| \le \|w\|$.
			Selbiges gilt für das Supremum
			\begin{math}
				\|A^{-1}\|
				= \sup_{w \in W \setminus \Set 0}  \f{\|A^{-1}\|\|w\|}{\|w\|}
				\le \f 1{\beta}.
			\end{math}
		\end{seg}
		\begin{seg}{\ProofImplication[2][1]}
			Es gilt
			\setcounter{equation}{10}
			\begin{math}[numbered] \label{eq:3.11}
				\inf_{v\neq 0} \sup_{w\neq 0} \frac{a(v,w)}{\|v\|\|w\|}
				&\stackrel{\ref{3.32} ii)}= \inf_{v\neq 0} \frac{\|Av\|}{\|v\|}
				= \inf_{w\neq  0} \frac{\|w\|}{\|A^{-1}w\|} \\
				&= \frac{1}{\sup_{w\neq 0} \frac{\|A^{-1}w\|}{\|w\|}}
				= \frac{1}{\|A^{-1}\|}
				\ge \frac{1}{\frac{1}{\beta}}
				= \beta
			\end{math}
			Sei $w \in W \setminus \Set 0$, $\exists v \in V \setminus \Set 0$ mit $w = Av$, dann ist
			\begin{math}
				a(v,w)
				= \<Av,w\>
				= \<Av,Av\>
				= \|Av\|^2
				> 0.
			\end{math}
		\end{seg}
	\end{proof}
\end{st}

\begin{note}[Nicht-Koerzivität für Advektion]
	\begin{itemize}
		\item
			Falls in \eqref{eq:3.4} $A = 0, c = 0$ (oder $A \neq 0, c \neq 0$ und $b$ hinreichend groß), dann ist $a(v,w)$ im Allgemeinen nicht koerziv auf $H_0^1(\Omega)$.
		\item
			Sei $\Omega = [0,1], b = 1, A = c = 0$.
			\begin{math}
				a(v,w)
				= \int_0^1 v \partial_x w \di[x].
			\end{math}
			Betrachte
			\begin{math}
				u(x) &= \sin(2\pi x) \in L^2(\Omega) \in L^2(\Omega), \\
				\partial_x u(x) &= 2\pi \cos(2\pi x) \in L^2(\Omega), \\
				u(0) = u(1) &= 0.
			\end{math}
			Dann ist $u \in H_0^1(\Omega)$, insbesondere $\|u\|_{H^1(\Omega)} \neq 0$, aber
			\begin{math}
				\inf_{v\neq 0} \frac{a(v,v)}{\|v\|^2}
				\le \frac{a(u,u)}{\|v\|^2}
				= \frac{\int_0^1 u \partial_x u \di[x]}{\|u\|^2}
				= 0.
			\end{math}
			Somit ist $a$ nicht positiv und insbesondere nicht koerziv.
		\item
			Knobelaufgabe:
			Ist $a(\argdot, \argdot)$ inf-sup-stabil auf $H_0^1(\Omega)$?
		\item
			Durch geeignete Wahl von Räumen kann obiges $a$ als inf-sup-stabil nachgewiesen wreden mit folgendem Hilssatz \ref{3.35}.
	\end{itemize}
\end{note}

\begin{st}[Lösbarkeit der Divergenzgleichung] \label{3.35}
	Sei $\Omega$ ein Lipschitzgebiet, $q \in L_0^2(\Omega) = \Set{ p \in L^2(\Omega) & \int_\Omega p = 0 }$.
	Dann existiert $\tilde v \in H_0^1(\Omega)^d$ so dass $\div \tilde v = q$ und es gilt
	\begin{math}[numbered] \label{eq:3.12}
		|\tilde v|_{H^1(\Omega)^d}
		\le C(\Omega) \|q\|_{L^2(\Omega)}.
	\end{math}
	\begin{proof}
		Für $d = 1$ siehe Übung.
		Allgemeiner siehe Babuska und Aziz 1972, “Survey Lectures on the mathematical foundations of the FEM” in “The Mathematical Foundations of the FEM with applications to PDEs”.
	\end{proof}
\end{st}

\begin{kor}[Inf-sup-Stabilität des Divergenzoperators] \label{3.36}
	Sei $\Omega$ ein Lipschitzgebiet und beschränkt.
	Für ein $q \in Q := L_0^2(\Omega)$ und $v \in V = H_0^1(\Omega)^d$.
	Setze $b(q,v) := \int_\Omega q \div v \di[x]$.

	Dann ist $b$ inf-sup-stabil auf $Q \times V$.
	\begin{proof}
		Sei $q \in Q$ und $\tilde v \in V$ aus \ref{3.35}, d.h. $q = \div \tilde v$.
		\begin{math}
			b(q,\tilde v)
			= \int_\Omega q \div \tilde v \di[x]
			= \int_\Omega q^2
			= \|q\|^2_{L^2(\Omega)}.
		\end{math}
		Wegen \eqref{eq:3.12} und der Normäquivalenz aus \ref{3.13} folgt
		\begin{math}
			\|\tilde v\|_{H^1(\Omega)^d}
			\le (1+s) |\tilde v|_{H^1(\Omega)^d}
			\le (1+s) C(\Omega) \|q\|_{L^2(\Omega)}^2.
		\end{math}
		Betrachte
		\begin{math}
			\inf_{q} \sup_{v} \frac{b(q,v)}{\|q\|\|v\|}
			\ge \inf_{q} \frac{b(q,\tilde v)}{\|q\|\|\tilde v\|}
			= \inf_{q} \frac{\|q\|_{L^2(\Omega)}^2}{C'(\Omega) \|q\|\|q\|}
			= \frac{1}{C'(\Omega)}
			> 0
		\end{math}
		wobei $C'(\Omega) = (1 + \diam(\Omega)) C(\Omega)$.
	\end{proof}
\end{kor}

\begin{note}
	\begin{itemize}
		\item
			Die Notation $b, Q, V$ (statt $a, v, w$) ist üblich für das Stokes-Problem
		\item
			Der folgende Satz garantiert Wohlgestelltheit für lineare PDEs mit inf-sup-stabiler Bilinearform.
	\end{itemize}
\end{note}

\begin{st}[Nečas Theorem] \label{3.37}
	Sei $a: V \times W \to \R$ stetige Bilinearform, $l \in W'$.
	Dann hat die Gleichung
	\begin{math}[numbered] \label{eq:3.13}
		a(u,w) = l(w)
	\end{math}
	für alle $w \in W$ eine eindeutige Lösung $u \in V$ und ist stetig abhängig von den Daten via
	\begin{math}[numbered] \label{eq:3.14}
		\|u\|_V
		\le \frac{1}{\beta}\|L\|_{W'}
	\end{math}
	für ein $\beta > 0$ unabhängig von $L$ genau dann, wenn eine der folgenden äquivalenten Bedingungen erfüllt ist.
	\begin{enumerate}[i)]
		\item
			$a$ erfüllt $\inf_{v} \sup_{w} \frac{a(v,w)}{\|v\|\|w\|} \ge \beta > 0$ und $\forall w \in W \setminus \Set 0 \exists v \in V: a(v,w) \neq 0$.
		\item
			$a$ erfüllt $\inf_{w} \sup_{v} \frac{a(v,w)}{\|v\|\|w\|} \ge \beta > 0$ und $\forall v \in V \setminus \Set 0 \exists w \in W: a(v,w) \neq 0$.
	\end{enumerate}
	\begin{proof}
		Es gilt $\<Au,w\> = a(u,w) = l(w) = \<w_l, w\>$ genau dann, wenn $Au = w_l$, wobei $w_l$ der Riesz-Repräsentant von $L$ ist.
		Wir führen den Beweis mit Hilfe von \ref{3.34}.
		\begin{seg}{i) $\implies$ \eqref{eq:3.13}, \eqref{eq:3.14}}
			Mittels \ref{3.34} ist $A$ invertierbar und $\|A^{-1}\| \le \f 1\beta$.
			$u := A^{-1} w_l$ ist die eindeutige Lösung von \eqref{eq:3.13}.
			Weiter ist
			\begin{math}
				\|u\|_V
				\le \|A^{-1}w\|
				\le \|A^{-1}\|_{L(W,V)} \underbrace{\|w_l\|_{W'}}_{\|L\|_{W'}}
				\le \frac{1}{\beta}\|L\|_{W'}.
			\end{math}
		\end{seg}
		\begin{seg}{\eqref{eq:3.13}, \eqref{eq:3.14} $\implies$ i)}
			\eqref{eq:3.13} hat eine eindeutige Lösung für alle $l \in w'$, also existiert $u \in V$ mit $Au = w_l$ für alle $l \in W'$ und $A$ invertierbar.
			Es gilt
			\begin{math}
				\|A^{-1} w_l\|
				\stack{\eqref{3.14}}\le \frac{1}{\beta}\|w_l\|,
			\end{math}
			also $\|A^{-1}\| \le \frac{1}{\beta}$.
			Aus \ref{3.34} folgt die Bedingung i).
		\end{seg}
		\begin{seg}{ii) $\implies$ \eqref{eq:3.13}, \eqref{eq:3.14}}
			Hier betrachten wir $A^*: W \to V$.
			\eqref{eq:3.11} aus dem Beweis von \ref{3.34} besagte für $a$ inf-sup-stabil:
			\begin{math}
				\inf_{v\neq 0} \sup_{w\neq 0} \frac{a(v,w)}{\|v\|\|w\|}
				= \|A^{-1}\|^{-1}.
			\end{math}
			Falls ii) gilt, so folgt analog mit der Existenz von $A^*$
			\begin{math}
				\beta \le \inf_{w\neq 0} \sup_{v\neq 0} \frac{a(v,w)}{\|v\|\|w\|}
				= \inf_{w\neq 0} \sup_{v\neq 0}  \frac{\<Av, w\>}{\|v\|\|w\|}
				= \inf_{w\neq 0} \sup_{v\neq 0}  \frac{\<v, A^*w\>}{\|v\|\|w\|}.
			\end{math}
			Damit ist $A^*$ invertierbar mit $\|(A^*)^{-1}\| \le \frac{1}{\beta}$.
			Es folgt
			\begin{math}[numbered] \label{eq:3.15}
				= \inf_{w\neq 0} \sup_{v\neq 0}  \frac{\<A^*,w\>}{\|v\|\|w\|}.
				= \|(A^*)^{-1}\|^{-1}.
			\end{math}
			$A^*$ invertierbar, stetig und $A$ stetig, also auch $A$ invertierbar, $\|A\| = \|A^*\|, \|A^{-1}\| = \|(A^*)^{-1}\|$.
			Es existiert ein eindeutiges $u = A^{-1} w_l$ als Lösung von \eqref{eq:3.13}.
			\eqref{3.14} folgt durch Einsetzen.
		\end{seg}
		\begin{seg}{\eqref{eq:3.13}, \eqref{eq:3.14} $\implies$ ii)}
			Analog zum Fall i) kann man zeigen, dass $A$ invertierbar ist $\|A^{-1}\| \le \frac{1}{\beta}$.
			Es folgt, dass $A^*$ invertierbar ist mit $\|(A^*)^{-1}\| \le \frac{1}{\beta}$.
			Mit \eqref{eq:3.15} folgt ii).
		\end{seg}
	\end{proof}
\Timestamp{2014-11-28}
	\begin{note}
		\begin{itemize}
			\item
				Der Satz besagt, dass inf-sup-Stabilität der „allgemeinste“ Stabilitätsbegriff für Wohlgestelltheit eines Problems der Form \eqref{eq:3.13} ist.
			\item
				Für koerzive, aber möglicherwiese unsymmetrische Bilinearform im Fall $V = W$ folgt Existenz, Eindeutigkeit und Stabilität als Korollar.
		\end{itemize}
	\end{note}
\end{st}

\begin{kor}[Lax-Milgram] \label{3.38}
	Sei $a: V \times V \to \R$ stetige, koerzive Bilinearform mit Koerzivitätskonstante $\alpha > 0$.
	Dann hat der Operator $A \in L(V,V)$ mit $\<Au, v\> = a(u,v)$ aus \ref{3.19} eine stetige Inverse mit
	\begin{math}
		\|A^{-1}\| \le \frac{1}{\alpha}
	\end{math}
	\begin{proof}
		Mit \ref{3.33} ist $a(\argdot, \argdot)$ also inf-sup-stabil mit $\beta \ge \alpha > 0$.
		Aus dem Beweis von \ref{3.37} folgt
		\begin{math}
			\alpha
			\le \inf_{v} \sup_{w} \frac{a(v,w)}{\|v\|\|w\|}
			= \|A^{-1}\|^{-1}
		\end{math}
	\end{proof}
	\begin{note}
		\begin{itemize}
			\item
				Mit \ref{3.37} folgt insbesondere Wohlgestelltheit
				\begin{math}[numbered] \label{eq:3.16}
					\inf_{v\neq 0} \sup_{w \neq 0} \frac{a(v,w)}{\|v\|\|w\|}
					= \beta
					= \inf_{w\neq 0} \sup_{v\neq 0} \frac{a(v,w)}{\|v\|\|w\|},
				\end{math}
				also selbes $\beta$ für beide inf-sup-Bedingungen.
			\item
				Die Umgekehrte Richtung in \eqref{eq:3.16} gilt auch:
				\begin{math}
					\inf_{w\neq 0} \sup_{v\neq 0} \frac{a(v,w)}{\|v\|\|w\|}
					&& \implies &&
					\forall w \neq 0 \exists v: a(v,w) \neq 0.
				\end{math}
				Dies zusammen mit der analogen anderen Seite ist gerade Bedingung ii) aus \ref{3.37}, also äquivalent zur Wohlgestelltheit.
			\item
				\ref{3.37} ii) impliziert, dass auch das sogenannte \emphdef[duales Problem]{duale Problem} zu $f \in V'$:
				\begin{math}[numbered] \label{eq:3.17}
					a(v,w) = f(v) && \forall v \in V
				\end{math}
				eine eindeutige Lösung $w \in W$ besitzt mit $\|w\| \le \frac{1}{\beta} \|f\|_{V'}$.
		\end{itemize}
	\end{note}
\end{kor}

\begin{kor}[Wohlgestelltheit für allgemeine elliptische RWP] \label{3.39}
	Sei ein homogenes Dirichlet-RWP für allgemeine PDE zweiter Ordnung in schwacher Form gegeben wie in \ref{3.21}, d.h.
	\begin{math}
		a(u,v) &= \int_\Omega (A \Nabla u) \cdot \Nabla v - (b \cdot \Nabla v) u + c uv \di[x] \\
		l(v) &= \int_\Omega fv \di[x]
	\end{math}
	für $u,v \in H_0^1(\Omega)$.

	Sei $A$ gleichmäßig elliptisch, $A, b, c$ gleichmäßig beschränkt, $c \ge 0$ hinreichend groß, sodass $a(\argdot, \argdot)$ koerziv auf $H_0^1(\Omega)$ mit Koerzivitätskonstante $\alpha > 0$ ist.

	Dann existiert für alle $f \in L^2(\Omega)$ eine eindeutige schwache Lösung $u \in H_0^1(\Omega)$ mit $a(u,v) = l(v)$ für alle $v \in H_0^1(\Omega)$ und diese ist beschränkt durch
	\begin{math}
		\|u\|_{H^1(\Omega)}
		\le \frac{1}{\alpha}\|f\|_{H^{-1}(\Omega)}
	\end{math}
	\begin{proof}
		Analog zu \ref{3.22} sieht man Stetigkeit und Koerzivität von $a$.
		Nach Lax-Milgram \ref{3.38} ist $A$ invertierbar mit $\|A^{-1}\| \le \frac{1}{\alpha}$, also ist $u := A^{-1} v_L$ eindeutige Lösung mit
		\begin{math}
			\|u\|
			\le \|A^{-1}\|\|v_L\|
			\le \frac{1}{\alpha} \|l\|_{H^{-1}(\Omega)}.
		\end{math}
	\end{proof}
\end{kor}

\begin{df}[Sattelpunktproblem] \label{3.40}
	Seien $V, Q$ Hilberträume, $a: V \times V \to \R$, $b: Q \times V \to \R$ stetige Bilinearformen, $f \in V', g \in Q'$.
	Gesucht ist die Lösung $(u,p) \in V \times Q$, welche erfüllt:
	\begin{math}[numbered] \label{eq:3.18}
		a(u,v) + b(p,v) &= f(v) &&\forall v \in V,\\
		b(q,u) &= g(q) && \forall q \in Q.
	\end{math}
	\begin{note}
		\begin{itemize}
			\item
				\eqref{eq:3.18} kann auch in bisheriger Form geschrieben werden:
				\begin{math}
					\_a(\_u,\_v) = \_l(\_v) && \forall \_v \in \_V
				\end{math}
				mit $\_u = (u,p) \in \_V := V \times Q$, $\_v = (v,q)$ und
				\begin{math}
					\_a(\_u,\_v)
					&:= a(u,v) + b(p,v) + b(q,u), \\
					\_l(\_v)
					&:= f(v) + g(q).
				\end{math}
				Die inf-sup-Stabilität von $\_a$ ist schwer zu zeigen.
				Daher untersuchen wir das Sattelpunktproblem in der Formulierung \eqref{eq:3.18}.
				Dann folgt inf-sup-Stabilität von $\_a$ aus der Wohlgestelltheit von \eqref{eq:3.18} (Nečas-Theorem).
			\item
				Falls $a$ koerziv und symmetrisch, ist $u \in V$ eindeutiger Minimierer des bedingten Optimierungsproblem
				\begin{math}
					\min_{v\in V} \frac{1}{2} a(v,v) - f(v)
					&\udN&
					b(\argdot, u) = g(\argdot) \text{ in $Q$}.
				\end{math}
				Hier ist $p$ der Lagrange-Multiplikator der Nebenbedingung und $(u,p)$ eindeutiger Sattelpunkt der Lagrange-Funktion
				\begin{math}
					L(v,q)
					= \frac{1}{2} a(v,v) - f(v) + b(q,v) - g(q),
				\end{math}
				mit $v \in V, q\in Q$.
			\item
				Man kann das Stokes-Problem
				\begin{math}
					-\mu \Laplace u + \Nabla p &= r && \text{in $\Omega$} \\
					\div u &= 0 && \text{in $\Omega$} \\
					u &= 0 && \text{auf $\Gamma$}
				\end{math}
				in schwacher Form als Sattelpunktproblem schreiben.
		\end{itemize}
	\end{note}
\end{df}

\begin{st}[Stokes-Sattelpunktproblem] \label{3.41}
	Sei $\Omega \subset \R^d$ offen und beschränktes Lipschitz-Gebiet, $r \in (L^2(\Omega))^d$, $\mu > 0$.
	Wähle $V := (H_0^1(\Omega))^d, Q := \Set{q \in L^2(\Omega) & \int_\Omega q \di[x] = 0}$ und definiere $a: V \times V \to \R, b \in Q \times V \to \R$ und $f \in V', g \in Q'$ durch
	\begin{math}
		a(v,w) &:= \mu \int_\Omega \Nabla u : \Nabla w \di[x], &
		f(v) &:= \int_\Omega rv \di[x], \\
		b(q,v) &:= -\int_\Omega q \div(v) \di[x], &
		g(q) &:= 0,
	\end{math}
	wobei $\Nabla v : \Nabla w := \sum_{i=1}^d \Nabla v_i \cdot \Nabla w_i$ für $v = (v_i)_{i=1}^d, w= (w_i)_{i=1}^d$.

	Dann ist $a(\argdot, \argdot$ stetig, symmetrisch und koreziv, $f, g$ stetig und $b(\argdot, \argdot)$ stetig und inf-sup-stabil.
	\begin{proof}
		Stetigkeit und Koerzivität von $a$ wie im elliptischen Fall in \ref{3.22} wegen $\mu > 0$.
		Stetigkeit von $g$ trivial, von $f$ folgt wie im elliptischen Fall (siehe Bemerkung vor \ref{3.23}).
		Stetigkeit von $b$ folgt mit Cauchy-Schwartz und Normäquivalenz auf $\R^d$:
		\begin{math}
			|b(q,v)|
			&\le \|q\|_{L^2} \|\div v\|_{L^2} \\
			\|\div v\|_{L^2}
			&\le \sum_{i=1}^d \|\partial_{x_i} v_i\|_{L^2} \\
			&\le \underbrace{\sum_{i=1}^d \|v_i\|_{H^1}}_{=\|(\|v_i\|_{H^1})_{i=1}^d\|_1}
			\le C \Big(\sum_{i=1}^d \|v_i\|_{H^1}^2\Big)^{\frac{1}{2}}
			= C \|v\|_{(H^1(\Omega))^d}.
		\end{math}
		Die inf-sup-Stabilität von $b(\argdot, \argdot)$ wurde in \ref{3.36} gezeigt.
	\end{proof}
\end{st}

\begin{st}[Orthogonales Komplement] \label{3.42}
	Sei $V_0 \subset V$ abgeschlossener Teilraum eines Hilbertraums.
	Dann existiert das \emphdef{orthogonale Komplement}
	\[
		V_\orth
		:= \Set{ v \in V & \forall v_0 \in V_0 : \<v,v_0\> = 0 }
	\]
	mit $V = V_0 \oplus V_\orth$ und $V_\orth$ ist abgeschlossener, linearer Teilraum, also Hilbertraum.
	\begin{proof}
		$V_\orth$ linearer Teilraum ist klar.
		Für die direkte Summe sei $v \in V_0 \cap V_\orth$, dann ist $\<v,v\> = 0$, d.h. $v = 0$, also $V_0 \cap V_\orth = \Set 0$.
		%fixme:
		Für die Abgeschlossenheit sei $(v_i)_{i\in\N}$ Cauchy-Folge in $V_\orth$.
		Da $V$ vollständig, existiert $\_v := \lim_{i\to\infty} v_i \in V$.
		Sei $v_0 \in V_0$, dann ist
		\begin{math}
			\<\_v, v_0\>
			= \<\lim_{i\to\infty} v_i, v_0\>
			= \lim_{i\to\infty} \<v_i, v_0\>
			= 0
		\end{math}
		und somit $\_v \in V_\orth$.
	\end{proof}
\end{st}

\begin{st}[Brezzi-Theorem, Existenz und Eindeutigkeit des SPP] \label{3.43}
	Das SPP \eqref{eq:3.18} besitzt für alle $f \in V', g \in Q'$ eine eindeutige Lösung $(u,p) \in V \times Q$ und diese ist beschränkt durch die Daten, d.h. es existiert $\beta > 0$ mit
	\begin{math}[numbered] \label{eq:3.19}
		(\|u\|_V^2 + \|p\|_Q^2)^{\frac{1}{2}}
		\le \frac{1}{\beta} (\|f\|_{V'}^2 + \|g\|_{Q'}^2)^{\frac{1}{2}}
	\end{math}
	genau dann, wenn Konstanten $\beta_a, \beta_b > 0$ existieren, sodass
	\begin{math}[numbered] \label{eq:3.20}
		\inf_{v\in V_0} \sup_{w \in V_0} \frac{a(v,w)}{\|v\|\|w\|}
		= \inf_{w\in V_0} \sup_{v\in V_0} \frac{a(v,w)}{\|v\|\|w\|}
		= \beta_a > 0
	\end{math}
	und
	\begin{math}[numbered] \label{eq:3.21}
		\inf_{q\in Q} \sup_{v \in V} \frac{b(q,v)}{\|q\|\|v\|}
		= \beta_b > 0,
	\end{math}
	wobei $V_0 := \Set{v \in V & \forall q \in Q : b(q,v) = 0}$.
\Timestamp{2014-12-02}
	\begin{proof}[anglehnt an K. Siebert, NUMPDE 12/13]
		\begin{seg}[\ProofImplication*]
			$b(\argdot, \argdot)$ ist stetig, also $V_0$ abgeschlossener Unterraum, also existiert mit \ref{3.42} ein orthogonales Komplement $V_\orth$ mit $v = V_0 \oplus V_\orth$.
			Für $v \in V$ sei $v = v_0 + v_\orth$ mit $v_0 \in V_0, v_\orth \in V_\orth$.

			Wegen $b(q,v) = b(q,v_0) + b(q,v_\orth) = b(q,v_\orth)$ folgt aus \eqref{eq:3.21}, dass
			\begin{math}
				\beta_b
				= \inf_{q\in Q} \sup_{v\in V} \frac{b(q,v)}{\|q\|\|v\|}
				= \inf_{q\in Q} \sup_{v\in V_\orth} \frac{b(q,v)}{\|q\|\|v\|}.
			\end{math}
			Nach Definition von $v_0$ gilt $\forall v \in V_\orth \exists q \in Q: b(q,v) \neq 0$.
			Also sind für $b(\argdot, \argdot)$ die Bedingungen des Nečas-Theorems \ref{3.37} ii) erfüllt, daher existiert für alle $g \in Q'$ eine eindeutige Lösung $u_\orth \in V_\orth$ von
			\begin{math}
				b(q,v_\orth) &= g(q) & \forall q \in Q.
			\end{math}
			Weiter existiert mit \eqref{eq:3.20} eine eindeutige Lösung $u_0 \in V_0$ von
			\begin{math}
				a(u_0,v) &= f(v) - a(u_\orth, v) & \forall v \in V_0,
			\end{math}
			denn $f(\argdot) - a(u_\orth, \argdot) \in V_0'$.
			Wir definieren $u := u_\orth + u_0$ und $f_\orth \in V'$ via
			\begin{math}[numbered] \label{eq:3.22:1}
				f_\orth(v) &:= f(v) - a(u,v) &\forall v \in V.
			\end{math}
			Dann ist auch $f_\orth \in (V_\orth)'$, also hat mit \eqref{3.37} i) und der Bemerkung nach \eqref{3.39} das duale Problem eine eindeutige Lösung $p \in Q$ mit
			\begin{math}
				b(p,v) &= f_\orth(v) & \forall v \in V_\orth.
			\end{math}

			Zeige nun: $(u,p)$ löst Sattelpunktproblem.
			Für $v \in V_0$ gilt
			\begin{math}
				a(u,v) + b(p,v)
				= \underbrace{a(u_0,v)}_{=f(v) - a(u_\orth, v)} + a(u_\orth, v ) + \underbrace{b(p,v)}_{=0}
				= f(v).
			\end{math}
			Für $v \in V_\orth$ gilt mit \eqref{eq:3.22:1}
			\begin{math}
				a(u,v) + b(p,v)
				= f(v) - f_\orth(v) + \underbrace{b(p,v)}_{=f_\orth(v)}
				= f(v).
			\end{math}
			Damit ist die erste Sattelpunktsgleichung erfüllt.
			Für $q \in Q$ gilt
			\begin{math}
				b(q,u)
				= \underbrace{b(q, u_0)}_{=0} + b(q, u_\orth)
				= g(q),
			\end{math}
			d.h. die zweite Sattelpunktsgleichung ist auch erfüllt.
			Somit ist Existenz und Eindeutigkeit gezeigt.

			Stabilität folgt aus Stabilität der Teilprobleme:
			es gilt $\|u_\orth\| \le \frac{1}{\beta}\|g\|_{Q'}, \|u_0\| \le \frac{1}{\beta_a} \|f-a(u_\orth, \argdot)\|_{v'}$.
			Wegen $\|\argdot\|_{V'} \le \|\argdot\|_{V}$ und mit Stetigkeitskonstante $\gamma$ von $a(\argdot, \argdot)$ folgt
			\begin{math}
				\|u_0\| \le \frac{1}{\beta_a}(\|f\|_{V'} + \gamma \|u_\orth\|)
				\le \frac{1}{\beta_a}(\|f\|_{V'} + \frac{\gamma}{\beta_b}\|g\|_{Q'}).
			\end{math}
			Für die Summe also
			\begin{math}
				\|u\|
				\le \|u_L\| + \|u_0\|
				\le \frac{1}{\beta_a} + \frac{1}{\beta_b}(\frac{\gamma}{\beta_a} + 1) \|g\|
				\le C_u(\|f\| + \|g\|),
			\end{math}
			wobei $C_u := \max \Set{\frac{1}{\beta_a}, \frac{1}{\beta_b}(\frac{\gamma}{\beta_a} + 1)}$.
			Mit der Youngschen Ungleichung $2 \_b \_c \le b^2 + c^2$ für $\_b, \_c \ge 0$ folgt für $\_a \le \_b + \_c$
			\begin{math}
				\_a^2
				\le (\_b + \_c)^2
				= \_b^2 + 2\_b\_c + \_c^2
				\le 2(\_b^2 + \_c^2).
			\end{math}
			Mit $\_a := \|u\|, \_b := C_u\|f\|, \_c := C_u\|g\|$ folgt
			\begin{math}
				\|u\|^2 \le 2 C_u^2 (\|f\|^2 + \|g\|^2).
			\end{math}
			Für $p$ folgt analog:
			\begin{math}
				\|p\|
				&\le \frac{1}{\beta_b} \|f_\orth\|
				= \frac{1}{\beta_b}\|f - a(u,\argdot)\|_{(V_\orth)'}
				&\le \frac{1}{\beta_b}\|f-a(u,\argdot)\|_{V'}
				&\le \frac{1}{\beta_b}(\|f\| + \gamma \|u\|)
				&\le \frac{1}{\beta_b}(1 + \gamma C_u) \|f\| + \frac{1}{\beta_b} \gamma C_u \|g\| \\
				&\le C_p (\|f\| + \|g\|).
			\end{math}
			Mit $C_p := \max \Set{ \frac{1}{\beta_b}(1+\gamma C_u), \frac{\gamma}{\beta_b} C_u }$ und mit der Young-Ungleichung wie oben
			\begin{math}
				\|p\|^2 \le 2 C_p^2 (\|f\|^2 + \|g\|^2).
			\end{math}
			Summation ergibt also
			\begin{math}
				(\|u\|^2 + \|p\|^2)^{\frac{1}{2}}
				\le \underbrace{\sqrt{2(C_u^2 + C_p^2)}}_{=\beta^{-1}} (\|f\|^2 + \|g\|^2)^{\frac 12}.
			\end{math}
		\end{seg}
		\begin{seg}[\ProofImplication]
			Angenommen für eine der Konstanten gilt $\beta_a = 0$ oder $\beta_b = 0$.
			Es ergibt sich Mehrdeutigkeit der Lösung eines Teilproblems aus der Rückrichtung des Beweises und somit auch eine mehrdeutige Lösung des Sattelpunktproblems.
		\end{seg}
	\end{proof}
\end{st}

\begin{kor}[Wohlgestelltheit Stokes] \label{3.44}
	Das Stokes-Sattelpunktproblem, d.h. \ref{3.40} mit $a(\argdot, \argdot), b(\argdot), f(\argdot), g(\argdot)$ gemäß \ref{3.41} besitzt eine eindeutige Lösung, welche beschränkt ist durch die Daten $f, g$.
	\begin{proof}
		Klar mit \ref{3.41} und \ref{3.43}.
	\end{proof}
\end{kor}


\section{Petrov-Galerkin-Verfahren} \label{sec:3.3}


Für eine gegebene schwache Form einer PDE suchen wir $u \in V$ mit
\begin{math}
	a(u,w) &= l(w) & \forall w \in W.
\end{math}

Wir schränken dazu das Problem auf endlichdimensionale Teilräume ein, um eine berechenbare Approximation zu erhalten.

\begin{df}[Diskretes Problem, Lösung] \label{3.45}
	Sei $a: V \times W \to \R$ stetige Bilinearform, $l \in W'$.
	Seien $V_h \subset V, W_h \subset W$ endlichdimensionale Teilräume mit $\dim V_h = \dim W_h = n$.
	Dann ist $u_h \in v_h$ gesucht als \emphdef{diskrete Lösung} von
	\begin{math}
		a(u_h, w) &= l(w) & \forall w \in W_h.
	\end{math}
	\begin{note}
		\begin{itemize}
			\item
				Wir nennen $V_h$ den \emphdef{Ansatzraum} und $W_h$ \emphdef{Testraum}.
			\item
				Fall $V \neq W$ oder $V_h \neq W_h$, dann heißt diese Einschränkung \emphdef{Petrov-Galerkin-Projektion}.
			\item
				Falls $V = W$ und $V_h = W_h$ heißt es \emphdef{Ritz-Galerkin-Projektion} oder \emphdef{Galerkin-Projektion}.
			\item
				Der Begriff der „Projektion“ legt nahe, dass man sich einen Projektionsoperator $P_h: V \to V_h$ definieren kann mit $P_h u = u_h$ und $P_h^2 = P_h$.
		\end{itemize}
	\end{note}
\end{df}

Wegen $\dim V_h = \dim W_h$ vereinfacht sich die Charakterisierung von inf-sup-Stabilität.

\begin{st}[Diskrete inf-sup-Bedingungen] \label{3.46}
	Für eine Bilinearform $a: V_h \times W_h \to \R$ sind äquivalent
	\begin{enumerate}[i)]
		\item
			$\inf_{v\in V_h} \sup_{w\in W_h} \frac{a(v,w)}{\|v\|\|w\|} = \beta_h > 0$,
		\item
			$\inf_{w\in W_h} \sup_{v\in V_h} \frac{a(v,w)}{\|v\|\|w\|} = \beta_h > 0$,
		\item
			$\forall w \in W_h \setminus \Set 0 \exists v \in V_h : a(v,w) \neq 0$ (d.h. $A$ ist surjektiv),
		\item
			$\forall v \in V_h \setminus \Set 0 \exists w \in W_h : a(v,w) \neq 0$ (d.h. $A$ ist injektiv),
	\end{enumerate}
	\begin{note}
		Wir nennen $\beta_h$ auch \emphdef{diskrete inf-sup-Konstante}.
	\end{note}
	\begin{proof}
		\begin{seg}[\ProofImplication)[1][3]]
			$A$ ist injektiv nach \ref{3.32} ii).
			Wegen $\dim V_h = \dim W_h$ folgt $A$ surjektiv, also $\forall w \in W, w \neq 0 \exists v \in V: a(v,w) \neq 0$ (wähle z.b. $v := A^T w$, dann ist in Matrixdarstellung $a(v,w) = w^TA A^T w = \|A^T w\|^2 \neq 0$ wegen Injektivität).
		\end{seg}
		\begin{seg}[\ProofImplication)[3][1]]
			$A$ ist surjektiv, also auch injektiv und es existiert eine stetige Inverse.
			Setze $\beta := \|A^{-1}\| > 0$.
			Mit \ref{3.34} ist
			\begin{math}
				\inf_{v\in V_h} \sup_{w\in W_h} \frac{a(v,w)}{\|v\|\|w\|}
				\ge \beta
			\end{math}
			und es folgt i).
		\end{seg}
		\begin{seg}[\ProofImplication)[2][4], \ProofImplication)[4][2]]
			analog zu i) $\iff$ iii).
		\end{seg}
		\begin{seg}[\ProofImplication)[1][2]]
			Mit i) und iii) folgt nach \ref{3.34}, dass $A$ invertierbar ist mit $\|A^{-1}\| \le \frac{1}{\beta}$.
			Damit ist $A^*$ invertierbar mit $\|(A^*)^{-1}\| \le \frac{1}{\beta}$.
			Nach \ref{3.34} erfüllt $a(\argdot, \argdot)$
			\begin{math}
				\inf_{w\neq 0} \sup_{v\neq 0} \frac{a(v,w)}{\|v\|\|w\|}
				\ge \beta,
			\end{math}
			also folgt ii).
		\end{seg}
	\end{proof}
\end{st}

\begin{st}[Wohlgestelltheit] \label{3.47}
	$a(\argdot, \argdot)$ erfülle eine der Bedingungen aus \ref{3.46}.
	Dann hat das diskrete Problem \ref{3.45} eine eindeutige Lösung $u_h \in V_h$ mit
	\begin{math}
		\|u_h\| \le \frac{1}{\beta_h} \|l\|_{W'}.
	\end{math}
	\begin{proof}
		klar mit Nečas \ref{3.37}.
	\end{proof}
	\begin{note}
		\begin{itemize}
			\item
				Für koerzive Probleme ist Wohlgestelltheit trivial:
				Koerzivität vererbt sich auf Teilräume, $\beta_h = \alpha$ ist gültige inf-sup-Konstante.
			\item
				Für (nicht-koerzive) inf-sup-stabile Probleme muss diskrete Stabilität bei der Wahl der Teilräume $V_h, W_h$ berücksichtigt werden, siehe auch Beispiel \ref{3.53}.
		\end{itemize}
	\end{note}
\end{st}

\begin{lem}[Galerkin-Orthogonalität] \label{3.48}
	Seien $u \in V, u_h \in V_h$ Lösung der PDE, bzw. der Petrov-Galerkin-Projektion.
	Dann gilt
	\begin{math}
		\forall w \in W_h : a(u - u_h, w) = 0,
	\end{math}
	sogenannte \emphdef{Galerkin-Orthogonalität}.
	\begin{proof}
		Es gilt für alle $w \in W_h$
		\begin{math}
			a(u-u_h, W)
			= \underbrace{a(u, w)}_{l(w)} - \underbrace{a(u_h, w)}_{l(w), w \in W_h}
			= l(w) - l(w)
			= 0.
		\end{math}
	\end{proof}
	\begin{note}
		\begin{itemize}
			\item
				Falls $a(\argdot, \argdot)$ symmetrisch und koerziv, $W = V, W_h = V_h$, dann bedeutet diese Aussage gerade Orthogonalität des Projektionsfehlers bei der orthogonalen Projektion bezüglich des Energie-Skalarprodukts.

				Definiere $P_a: V \to V_h$ durch $\<v - P_a v, w\>_a = 0$ für alle $w \in V_h, v \in V$.
				Dann ist $P_a u = u_h$, denn $\<u - u_h, w\>_a = a(u - u_h, w) = 0$ mittels Galerkin-Orthogonalität \ref{3.48}.
				Die Galerkin-Projektion ist also genau die orthogonale Projektion bezüglich $\<\argdot, \argdot\>_a$.
			\item
				Falls $a(\argdot, \argdot)$ nicht symmetrisch, nicht koerziv oder $V \neq W$, so existiert keine solche Interpretation als orthogonale Projektion.
				Wir sprechen aber trotzdem von Galerkin-Orthogonalität.
		\end{itemize}
	\end{note}
\end{lem}

\Timestamp{2014-12-05}

\begin{kor}[Reproduktion von Lösungen] \label{3.49}
	Sei $a(\argdot, \argdot)$ inf-sup-stabil auf $V \times W$ und $V_h \times W_h$, $u \in V$ Lösung der (schwachen) PDE und $u_h \in V_h$ die Petrov-Galerkin-Projektion.
	Falls $u \in V_h$, dann ist $u_h = u$.
	\begin{proof}
		Für $u \in V_h$, dann ist auch $u - u_h \in Vh$.
		Angenommen $u - u_h \neq 0$, dann existiert $w \in W_h, w \neq 0$ mit
		\begin{math}
			a(u-u_h, w) \neq 0.
		\end{math}
		Dies ist ein Widerspruch zur Galerkin-Orthogonalität, also $u = u_h$.
	\end{proof}
	\begin{note}
		Also ist z.B. $V_h = \Span{u}$ optimaler 1D-Lösungsraum.
		Dieser ist natürlich derart „teuer“ zu berechnen, wie $u$ und daher inpraktikabel.
	\end{note}
\end{kor}

\begin{st}[Diskretes Problem als LGS] \label{3.50}
	Sei $V_h \subset V$ mit Basis $\Set{\phi_j}_{j=1}^n$ und $W_h \subset W$ mit Basis $\Set{\psi_i}_{i=1}^n$.
	Definiere die Systemmatrix $A_h = (a_{ij})_{i,j=1}^n$ und die „rechte Seite“ $b_h = (b_i)_{i=1}^n$ via
	\begin{math}
		a_{ij} &:= a(\phi_j, \psi_i), &
		b_i &:= l(\psi_i)
	\end{math}
	für $i,j = 1, \dotsc, n$.
	Dann ist $u_h = \sum_{j=1}^n u_j \phi_j$ mit DOF-Vektor $\underbar{u}_h = (u_i)_{i=1}^n$ Lösung des LGS
	\begin{math}[numbered] \setcounter{equation}{21} \label{eq:3.22}
		A_h \underbar{u}_h = b_h.
	\end{math}
	\begin{proof}
		Zeige: $\sum_{j=1}^{n} u_j \phi_j$ löst die schwache Form der PDE.
		Für eine Testfunktion $w = \psi_i$ gilt
		\begin{math}
			a\Big(\sum_{j=1}^{n} u_j \phi_j, \psi_i \Big)
			= \sum_{j=1}^{n} a(\phi_j, \psi_i)
			= (A_h \underbar{u}_h)_i
			\stack{\eqref{3.22}}= (b_h)_i
			= b_i
			= l(\psi_i).
		\end{math}
		Für $w \in W_h$ existiert $\underbar{w} \in \R^n$, $w = (w_i)_{i=1}^n$ mit $w = \sum_{i=1}^n w_i \psi_i$ und
		\begin{math}
			a\Big(\sum_{j=1}^{n} u_j \phi_j, w \Big)
			&= \sum_{i=1}^{n} w_i \underbrace{a\Big(\sum_{j=1}^{n} u_j \phi_j, \psi_i\Big)}_{=l(\psi_i)} \\
			&= \sum_{i=1}^{n} w_i l(\psi_i)
			= l\Big(\sum_{i=1}^{n} w_i \psi_i\Big)
			= l(w).
		\end{math}
	\end{proof}
	\begin{note}
		\begin{itemize}
			\item
				Falls $a(\argdot, \argdot)$ symmetrisch und $V_h = W_h$ mit identischer Basis, dann ist auch $A_h$ symmetrisch.
			\item
				Falls $a$ koerziv, dann ist $A_h$ positiv definit
				\begin{proof}
					\begin{math}
						\underbar{v}^T A_h \underbar{v}
						&= \sum_{i,j}^{} v_i v_j (A_{ij})
						= \sum_{i,j}^{} a\Big(\sum_{j}^{} v_j \phi_j, \sum_{i}^{} v_i \phi_i \Big) \\
						&= a(v,v)
						> \alpha\|v\|^2
					\end{math}
					und somit $\underbar{v}^T A_h \underbar{v} > 0 \iff v \neq 0$.
				\end{proof}
			\item
				Falls $a$ koerziv und symmetrisch, wird $A_h$ auch „Steifigkeitsmatrix“ genannt wegen physikalischer Anschauch im Kontext der linearen Elastizität.
		\end{itemize}
	\end{note}
\end{st}

\begin{ex} \label{3.51}
	$V_h$ aus Eigenfunktionen des Differentialoperators im Fall $V = W$.
	Die optimale Basis bei unbekannter/variabler rechte Seite $l$:

	Definiere die „schwache Form“ des Eigenwertproblems als:
	Finde Eigenfunktion $v \in V$ und Eigenwert $\Lambda$ mit
	\begin{math}
		a(v,w)
		&= \lambda \<v, w\>_{L^2}
		& \forall w \in V.
	\end{math}
	Unter gewissen Voraussetzungen, z.B. $a(\argdot, \argdot)$ symmetrisch, $\Omega$ zusammenhängend, siehe [Evans, §6.5] kann man zeigen:
	\begin{itemize}
		\item
			Es existieren nur reelle positive Eigenwerte: $\lambda \in (0, \infty)$.
		\item
			Es existiert eine unbeschränkte Folge an Eigenwerten, d.h. $(\lambda_j)_{j\in\N}$ mit
			\begin{math}
				0 < \lambda_1 \le \lambda_2 \le \dotsb
			\end{math}
			und $\lim_{j\to \infty}\lambda_j = \infty$.
		\item
			Es existiert eine $L_2$-orthonormale Menge $\Set{v_j}_{j\in\N}$ aus Eigenfunktionen $v_j$ zu Eigenwerten $\lambda_j$.
			Ansatz: Wähle $n \in \N$, setze $h := \f 1n$, $V_h := \Span{\phi_1, \dotsc, \phi_n} \subset V$ mit $\phi_i := v_i$.
			Für die Steifigkeitsmatrix der Galerkin-Projektion folgt:
			\begin{math}
				a(\phi_j, \phi_i)
				= a(v_j, v_i)
				= \lambda_j \underbrace{\<v_j, v_i\>_{L_2}}_{\delta_{ij}}
				= \lambda_j \delta_{ij},
			\end{math}
			also ergibt sich das LGS
			\begin{math}
				\Matrix{\lambda_1 & & 0 \\ & \ddots & \\ 0 & & \lambda_n}
				\Vector{u_1 & \dots & u_n}
				= \Vector{l(v_1) & \dots & l(v_n)}
			\end{math}
			mit Lösung $u_j := \frac{l(v_j)}{\lambda_j}$.

			Dies ergibt eine optimale Struktur des LGS, da $A_h$ diagonal ist.
			Im Allgemeinen ist das Verfahren jedoch inpraktikabel, da die Eigenfunktionen/Eigenwerte $v_i, \lambda_i$ selen bekannt sind.
	\end{itemize}
\end{ex}

\begin{ex} \label{3.52}
	$V_h$ als polynomieller Ansatzraum ($V = W, V_h = W_h$)
	Sei $d = 1, \Omega = (0,1), a(u,v) := \int_\Omega u' v'$ auf $H_0^1(\Omega)$.
	Ein Polynom mit Nullrandwerten hat die Gestalt $p(x) = x(1-x)q(x)$ für $q \in \P_m$.
	Wähle $n \in \N$, $h := \frac{1}{n}$, $V_h := \Span{\phi_1, \dotsc, \phi_n} \subset V$ mit $\phi_j(x) := x(1-x) x^{j-1} = x^j (1-x)$ für $j = 1, \dotsc, n$.
	Es ergibt sich
	\begin{math}
		a(\phi_j, \phi_i)
		= \int_\Omega \phi_j' \phi_i' \di[x],
	\end{math}
	im Allgemeinen ergibt sich $a(\phi_j, \phi_i) \neq 0$.
	Die Steifigkeitsmatrix $A_h$ ist somit dicht besetzt und bei großen Matrizen ergeben sich Speicherprobleme.
	Folglich ist das Verfahren für großes $n$ impraktikabel.

	Bei mäßigem $n$ (z.B. $n \approx 25$) findet dies unter dem Namen \emphdef{Spektralverfahren} Verwendung.
\end{ex}

\begin{ex}[Notwendigkeit von Petrov-Galerkin bei nicht-koerziven Problemen] \label{3.53}
	Falls $a(\argdot, \argdot)$ nicht koerziv ist, kann die (Ritz-)Galerkin-Projektion aus einem wohlgestellten Problem in $V$ ein singuläres System in $V_h$ liefern.

	Sei $V := \R^2, a(u,v) = u^T \Matrix{1 & 0 \\ 0 & -1} v$ für $u, v \in \R^2$ und $l(v) := \Vector{1 & 1}^T v$.

	$a(\argdot, \argdot)$ ist nicht koerziv, da $a(\Vector{0 & 1}, \Vector{0 & 1}) = -1 < \alpha \|\Vector{0 & 1}\|^2$ für alle $\alpha > 0$.
	Das System in $V$ ist wohlgestellt:
	\begin{math}
		a(u,v) = l(v)
		\iff u^T \Matrix{1 & 0 \\ 0 & -1} = \Vector{1 & 1}^T
		\iff u = \Vector{1 & -1}.
	\end{math}
	Übung: inf-sup-Konstante ist $1$.

	Mit $V_h = \Span{\Vector{1 & 1}}$, $W_h = V_h$ entsteht nun ein singuläres System:
	mit dem Ansatz $u_h = u_1 \phi_1$, wobei $\phi_1 = \Vector{1 & 1}$, $u_1 \in \R$ ergibt sich als LGS:
	\begin{math}
		A_h &= (a_{11}) = a(\phi_1, \phi_1) = \Vector{1 & 1}^T \Matrix{1 & 0 \\ 0 & -1} \Vector{1 & 1} = 0, \\
		b_h &= l(\phi_1) = \Vector{1 & 1}^T \Vector{1 & 1} = 2,
	\end{math}
	aber $0 \cdot u_h = 2$ ist nicht lösbar.

	Ein Ausweg bietet Petrov-Galerkin:
	Wähle $\phi_1 \in \R^2 \setminus \Span{\Vector{1 & 1}}$ beliebig, $\psi_1 := \Matrix{1 & 0 \\ 0 & -1} \phi_1$, $V_h := \Span{\phi_1}, W_h := \Span{\psi_1}$.
	Es ergibt sich
	\begin{math}
		A_h &= (a(\phi_1, \psi_1))
		= \phi_1^T \Matrix{1 & 0 \\ 0 & -1} \underbrace{\Matrix{1 & 0 \\ 0 & -1} \phi_1}_{=\psi_1}
		= \|\phi_1\|^2 > 0, \\
		b_h &= l(\psi_1),
	\end{math}
	Also ist $A_h \underbar{u}_h = b_h$ eindeutig lösbar.
\end{ex}

Wir wenden uns jetzt Fehleranalysen für Petrov-Galerkin zu:
Zentral ist die Beziehung zur Bestapproximation.

Gegeben $V_h$, wie gut kommt $u_h \in V_h$ an das Minimum von $\inf_{v\in V_h} \|u - v\|$ heran?.
Zunächst ein Hilfssatz

\begin{lem} \label{3.54}
	Sei $P \in L(V, V)$ idempotent, d.h. $P^2 = P$, nichttrivial, d.h. $0 \neq P \neq \Id$.
	Dann ist
	\begin{math}
		\|P\|_{L(V,V)}
		= \|\Id - P\|_{V,V}.
	\end{math}
	\begin{proof}
		Siehe Übung.
	\end{proof}
\end{lem}

\begin{st}[Beziehung zur Bestapproximation] \label{3.55}
	Sei $a(\argdot, \argdot)$ inf-sup-stabil auf $V_h \times W_h$ mit inf-sup-Konstante $\beta_h > 0$ und stetig auf $V \times W$ mit Stetigkeitskonstante $\gamma > 0$.
	Dann gilt für den Fehler der Petrov-Galerkin-Projektion
	\begin{math}
		\|u - u_h\| \le \frac{\gamma}{\beta_h} \inf_{v\in V_h} \|u - v\|.
	\end{math}
	\begin{proof}
		Definieren $P_h: V \to V_h$:
		Zu $v \in V$ sei $P_h v \in V_h$ Lösung von
		\begin{math}[numbered] \label{eq:3.23}
			a(P_h v, w) &= a(v,w) & \forall w \in W_h.
		\end{math}
		Wegen $a(v, \argdot) \in W'$ existiert eine eindeutige Lösung $P_h v$, also ist $P_h$ wohldefiniert.
		$P_h$ ist weiter linear (trivial) und eine Projektion $P_h^2 = P_h$, denn für $v \in V_h$ erfüllt $P_h v = v$ die Gleichung \eqref{eq:3.23}.
		Damit gilt
		\begin{math}
			(\Id - P_h)v
			= v - P_h v
			= 0
		\end{math}
		für alle $v \in V_h$.
		Weiter gilt $u_h = P_h u$, denn $a(P_h u, w) \stack{\eqref{eq:3.23}}= a(u,w) = l(w)$ für alle $w \in W_h$, was gerade dem diskreten Problem \ref{3.45} entspricht.
		Für die Norm schließen wir
		\begin{math}
			\|P_h v\|
			\le \frac{\|a(v,\argdot)\|_{W'}}{\beta_h}
			\le \frac{\gamma \|v\|}{\beta_h},
		\end{math}
		also
		\begin{math}
			\|P_h\|
			= \inf_{v\neq 0} \frac{\|P_h v\|}{\|v\|}
			\le \frac{\gamma}{\beta_h}.
		\end{math}
		Für $v \in V_h$ gilt die Behauptung mittels \ref{3.54} falls $P_h$ nichttrivial ist:
		\begin{math}
			\|u - u_h\|
			&= \|(\Id - P_h)u\|
			= \|(\Id - P_h u) - (\Id - P_h)v\| \\
			&= \|\Id - P_h(u-v)\| \\
			&\le \|\Id - P_h\| \|u-v\|
			\stack{\ref{3.54}}= \|P_h\|\|u-v\| \\
			&\le \frac{\gamma}{\beta} \|u-v\|
		\end{math}
		für alle $v \in V_h$, insbesondere also
		\begin{math}
			\|u - u_h\| \le \frac{\gamma}{\beta_h} \inf_{v\in V_h} \|u - v\|.
		\end{math}
		Falls $P_h = \Id$ ist $V = V_h$, $\|u - u_h\| = 0$, also ist die Behauptung klar.
		Falls $P_h = 0$, dann ist $V_h = \Set{0}$ (sonst finde $v \in V_h \setminus \Set 0$, also $0 = P_h v = v \neq 0$, ein Widerspruch) und insbesondere $\dim V_h = 0$, also $a(\argdot, \argdot)$ nicht inf-sup-stabil auf $V_h \times W_h$.
	\end{proof}
\end{st}

\begin{kor}[Lemma von Céa] \label{3.56}
	Falls $a(\argdot, \argdot)$ stetig und koerziv mit Koerzivitätskonstante $\alpha$, so gilt
	\begin{math}
		\|u - u_h\| \le \frac{\gamma}{\alpha} \inf_{v\in V_h} \|u - v\|.
	\end{math}
	\begin{proof}
		Klar mit \ref{3.55} und $\alpha$-$\beta$-Beziehung in \ref{3.33}.
	\end{proof}
\end{kor}

\begin{note}
	\begin{itemize}
		\item
			$\|u - u_h\|$ ist bei Galerkin-Projektion für $a(\argdot, \argdot)$ koerziv, also höchstens um einen $h$-unabhängigen Faktor $\frac{\gamma}{\alpha}$ schlechter als die Bestapproximation.
			Dies nennt man \emphdef{Quasi-Optimalität} der Galerkin-Projektion.
		\item
			Ein offensichtlich gutes Konstruktionsprinzip für $V_h$ wäre:
			Wähle $V_h$ so, dass $\inf_{v \in V_h} \|u - v\|$ klein, dann ist der Galerkin-Approximationsfehler garantiert klein.
		\item
			Formal impliziert \ref{3.56} Konvergenz der Galerkin-Projektion, falls die Bestapproximation konvergiert:
			\begin{math}[numbered] \label{eq:3.24}
				\inf_{v\in V_h} \|u - v\| &\le C h^p
				&&\implies&
				\|u - u_h\| &\le C' h^p
			\end{math}
			mit $C' := C \frac{\gamma}{\alpha}$.
		\item
			Konvergenz und Fehleranalysis ist also reduziert auf die PDE-unabhängige Frage:
			Wie gut kann $V_h$ Funktionen $u \in V$ gut approximieren?
	\end{itemize}
\end{note}


