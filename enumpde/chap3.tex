\chapter{FEM für koerzive und inf-sup-stabile Probleme} \label{chap:3}


\section{Sobolev-Räume}


\begin{df}[Schwache Ableitung] \label{3.1}
	Sei $\beta \in \N_0^d$ ein Multiindex, $u \in L_{\text{loc}}^1, \Omega \subset \R^d$.
	Wir nennen $v^\beta \in L_{\text{loc}}^1(\Omega)$ \emphdef{schwache Ableitung} von $u$ falls gilt
	\[
		\int_\Omega u \partial^\beta \phi \di[x]
		= (-1)^{|\beta|} \int_\Omega v^\beta \phi \di[x]
	\]
	für alle $\phi \in C_0^\infty(\Omega)$.
\end{df}

\begin{ex*}
	Sei $\Omega = (-1, 1), u(x) := |x|$.
	Dann ist $\sign(x)$ die schwache Ableitung von $u$.
	Denn sei $\phi \in C_0^\infty(\Omega)$, dann ist
	\begin{align*}
		\int_{-1}^1 u(x) \phi'(x) \di[x]
		&= \int_{-1}^0 u(x) \phi'(x) \di[x]
		+ \int_0^1 u(x) \phi'(x) \di[x] \\
		&= \int_{-1}^0 -x \phi'(x) \di[x]
		+ \int_0^1 x \phi'(x) \di[x] \\
		&= - \int_{-1}^0 (-1) \phi(x) \di[x] + \underbrace{[-x \phi(x)]_{-1}^0}_{=0}
		- \int_0^1 \phi(x) \di[x] + \underbrace{[x \phi(x)]_0^1}_{=0} \\
		&= - \int_{-1}^1 \sign(x) \phi(x) \di[x]
	\end{align*}
\end{ex*}

\begin{st}[Eindeutigkeit] \label{3.2}
	Sei $u \in L^1_{\text{loc}}(\Omega)$, dann existiert höchstens eine schwache Ableitung zu $\beta \in \N_0^d$.
	\begin{proof}
		Seien $v^\beta, \_v^\beta$ zwei schwache Ableitungen.
		Dann ist
		\[
			(-1)^{|\beta|} \int_\Omega \_v^\beta \phi \di[x]
			= \int_\Omega u \partial^\beta \phi \di[x]
			= (-1)^{|\beta|} \int_\Omega v^\beta \phi \di[x],
		\]
		also für alle $\phi \in C_0^\infty(\Omega)$
		\[
			\int_\Omega (v^\beta - \_v^\beta) \phi \di[x] = 0.
		\]
		Nach dem Fundamentalsatz der Variationsrechnung \ref{1.7} also $v^\beta = \_v^\beta$ fast überall, also $v^\beta = \_v^\beta$ im $L^1_{\text{loc}}$-Sinn.
	\end{proof}
\end{st}

\begin{st}[klassische Ableitung ist schwache Ableitung] \label{3.3}
	Sei $u \in C^m(\Omega)$, $v^\beta$ für $|\beta| \le m$ die schwache Ableitung und $\partial^\beta u \in C^{m-|\beta|}(\Omega)$ die klassische Ableitung.
	Dann gilt $v^\beta = \partial^\beta u$ punktweise, also im $C^0$-Sinn.
	\begin{proof}
		Mit partieller Integration gilt
		\[
			\int_\Omega u \partial^\beta \phi \di[x]
			= (-1)^{|\beta|} \int_\Omega (\partial^\beta u) \phi \di[x]
		\]
		für alle $\phi \in C_0^\infty(\Omega)$ und die klassische Ableitung ist eine schwache Ableitung.
		Wegen der Eindeutigkeit \ref{3.2} ist $\partial^\beta u$ die schwache Ableitung.
	\end{proof}
	\begin{note}
		\begin{itemize}
			\item
				Die schwache Ableitung umfassen die klassischen Ableitungen, daher schreiben wir im Folgenden einfach $\partial^\beta u$ statt $v^\beta$.
			\item
				\ref{3.3} gilt auch auf Teilgebieten, d.h. falls $u$ stückweise klassisch differenzierbar und global schwach differenzierbar, so ist die schwache Ableitung gerade die stückweise klassische Ableitung.
		\end{itemize}
	\end{note}
\end{st}

\begin{ex*}
	$u(x) := \sign(x)$ ist nicht schwach differenzierbar.
	\begin{proof}
		Falls $u$ schwach differenzierbar wäre, so müsste
		\[
			\partial_x u = \begin{cases}
				0 & x < 0 \\
				0 & x > 0
			\end{cases}
		\]
		gemäß der stückweisen klassichen Ableitung.
		Es folgt $(-1)\int_\Omega (\partial_x u) \phi \di[x] = 0$, aber für $\phi(0) \neq 0, \supp \phi \subset (-1,1)$ nicht leer gilt
		\begin{align*}
			\int_\Omega u \partial_x \phi \di[x]
			&= \int_{-1}^0 (-1) \partial_x \phi \di[x] + \int_0^1 1 \partial_x \phi \di[x] \\
			&= \underbrace{\phi(1)}_{=0} \underbrace{- \phi(0) - \phi(0)}_{= 2 \phi(0)} + \underbrace{\phi(-1) }_{=0}
			\neq 0
		\end{align*}
	\end{proof}
\end{ex*}

