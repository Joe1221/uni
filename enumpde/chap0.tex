\chapter{Einführung}

\Timestamp{2014-10-14}

Eine partielle Differentialgleichung (PDE) ist eine Gleichung, welche verschiedene Ableitungen einer Funktion in Beziehung setzt.
Gesucht ist eine unbekannte Funktion als Lösung der partiellen Differentialgleichung.

\begin{ex*}
	\begin{itemize}
		\item
			Poisson Gleichung
			\[
				- \pddx[x_1^2] u(x_1,x_2) - \pddx[x_2^2] u(x_1, x_2) = f(x_1, x_2),
			\]
			wobei $x = (x_1, x_2) \in \Omega$, $\Omega \subset \R^2$ offen.
		\item
			Instationäre Wärmegleichung
			\[
				\pddx[t] u(x,t) - \pddx[x^2] u(x,t) = f(x, t),
			\]
			wobei $(x, t) \in \Omega_T \subset \R \times \R^+$.
		\item
			Wellengleichung
			\[
				\pddx[t^2] u(x, t) = c \pddx[x^2] u(x,t),
			\]
			wobei $(x,t) \in \Omega_T$.
	\end{itemize}
\end{ex*}

Für komplexere partielle Differentialgleichungen existieren keine Lösungsformeln, was die Numerik erforderlich macht.

\paragraph{Fragen, welche in der Vorlesung behandelt werden}

\begin{itemize}
	\item
		Durch welche numerische Verfahren kann die PDE-Lösung approximiert werden?
	\item
		Konvergiert das numerische Verfahren gegen die Lösung der PDE?
	\item
		Mit welcher Konvergenzrate?
	\item
		Kann der Fehler der numerischen Lösung quantifiziert werden? (A priori vs a posteriori)
	\item
		Wie können numerische Verfahren effizient implementiert werden?
	\item
		Ist die numerische Approximation stabil, d.h. existiert eine stetige Abhängigkeit von den Daten?
	\item
		Diskretes Analogon von physikalischer Eigenschaft (z.B. Erhaltung, Nichtnegativität, Beschränktheit)
\end{itemize}

\paragraph{Fragen, die in dieser Vorlesung nicht, oder nur marginal behandelt werden}

\begin{itemize}
	\item
		Modellierung: Herleitung von PDE aus physikalischen Prinzipien (siehe Vorlesung PDEMAS).
	\item
		Lösungstheorie von PDEs: Existenz- und Eindeutigkeitssätze, Regulärität (nur insofern wie sinnvoll für die Numerik)
\end{itemize}

\paragraph{Gliederung}

\begin{enumerate}[1.,start=0]
	\item
		Organisation
	\item
		Einführung und Grundlagen
	\item
		Finite Differenzen für elliptische Probleme
	\item
		Finite Elemente für elliptische und inf-sup stabile Probleme
	\item
		Parabolische Probleme
	\item
		Finite Volumen Verfahrene für Erhaltungsgleichungen
\end{enumerate}

eventuelle weitere Verfahren:
\begin{itemize}
	\item
		Spektralmethoden,
	\item
		Discontinuous Galerkin,
	\item
		Kollation,
	\item
		Reduzierte Basis Methoden
\end{itemize}

