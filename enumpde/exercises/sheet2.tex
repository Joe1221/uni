\documentclass{myexercise}

\begin{document}

\begin{exercise}[Aufgabe 1]
	Setze
	\begin{align*}
		u(x,t)
		&:= \f 12\big( u_{0}(x-ct) + u_{0}(x+ct) \big) \\
		u_\eps(x,t)
		&:= \f 12\big( u_{0,\eps}(x-ct) + u_{0,\eps}(x+ct) \big)
	\end{align*}
	Zu zeigen ist $\int_\Omega u \partial_t^2 \phi - c^2 u \partial_x^2 \phi \di[(x,t)] = 0$.

	Nach [Alt, Satz 2.14] gilt $\lim_{\eps \to 0} \|u_{0,\epsilon} - u_0\|_{L^1(\R)} = 0$, also für alle $t \in (0,\infty)$ und alle $\phi \in C_0^\infty(\Omega_T)$ insbesondere
	\[
		0 = \lim_{\epsilon\to 0}\underbrace{\int_\R \big(u_\epsilon(x,t) - u(x,t)\big) \phi(x,t) \di[x]}_{\supp(\dotsc) \text{ kompakt}}
	\]
	und somit
	\[
		\lim_{\epsilon \to 0} \int_\Omega (u_\epsilon - u)\phi \di[(x,t)]
		= \int_0^\infty \lim_{\epsilon\to 0} \int_\R \big( u_\epsilon(x,t) - u(x,t)\big) \phi(x,t) \di[x] \di[t]
		= 0.
	\]
	Weiter erfüllt $u_\epsilon(x,t)$ nach [Satz 1.22] die Gleichung $\partial_t^2 v(x,t) - c^2 \partial_x^2 v(x,t) = 0$ in $\Omega_T$.
	Bei partieller Integration verschwinden die Randterme da $\partial^\beta \phi \in C_0^\infty$ und es gilt
	\[
		\int_\Omega u_\epsilon \partial_t^2 \phi - c^2 u_\epsilon \partial_x^2 \phi  \di[(x,t)] = 0
	\]
	Es gilt jetzt
	\begin{align*}
		\int_\Omega u \partial_t^2 \phi - c^2 u \partial_x^2 \phi \di[(x,t)]
		= \int_\Omega (u - u_\epsilon) \partial_t^2 \phi \di[(x,t)] - c^2 \int_\Omega (u-u_\epsilon) \partial_x^2 \phi \di[(x,t)]
		\to 0
	\end{align*}
	für $\eps \to 0$.

\end{exercise}

\newpage
\begin{exercise}[Aufgabe 2]
	\begin{enumerate}[a)]
		\item
			$u_1 - u_2$ löst
			\begin{align*}
				-\Laplace u &= f_1 - f_2 & &\text{in $\Omega$}\\
				u &= g_1 - g_2 & &\text{auf $\Boundary \Omega$}
			\end{align*}
			Wegen $f_1 \le f_2$ liefert das Maximumsprinzip
			\[
				\max_{x\in\_\Omega} u_1 - u_2
				= \max_{x\in\Boundary\Omega} u_1 - u_2
				= \max_{x\in\Boundary\Omega} \underbrace{g_1 - g_2}_{\le 0}
				\le 0
			\]
			und folglich $u_1 \le u_2$ auf ganz $\_\Omega$.
		\item
			Wegen $f_1 \le f_2$ liefert das Maximumsprinzip angewendet auf $u_1 - u_2$
			\[
				\max_{x\in\_\Omega} u_1 - u_2
				= \max_{x\in\Boundary\Omega} g_1 - g_2
				\le \|g_1 - g_2\|_{C^0(\Boundary\Omega)}.
			\]
			Analog für $u_2 - u_1$ gilt $\max_{x\in\_\Omega} u_2 - u_1 \le \|g_1 - g_2\|_{C^0(\Boundary\Omega)}$ und somit da $\_\Omega$ kompakt
			\[
				\|u_1 - u_2\|_{C^0(\_\Omega)}
				= \max_{x\in\_\Omega} |u_1 - u_2|
				\le \|g_1 - g_2\|_{C^0(\Boundary\Omega)}.
			\]
		\item
			Sei $r > 0$ so, dass $B_r(0) \supset \_\Omega$.
			Definiere
			\[
				v(x) := (r^2 - \f 12 x_1^2)\|f_1 - f_2\|_\infty.
			\]
			Wir beobachten $0 \le r^2 - \f 12 x_1^2 \le r^2$.
			Es gilt offenbar
			\begin{align*}
				-\Laplace v &= -\partial_{x_1}^2 v = \|f_1 - f_2\|_\infty \ge \begin{cases}
					f_1 - f_2 \\
					f_2 - f_1
				\end{cases} && \text{in $\Omega$} \\
				v &\ge 0 &&\text{auf $\Boundary \Omega$}
			\end{align*}
			Wendet man Teil a) auf $v$ und $u_1 - u_2$, bzw. $u_2 - u_1$ an, folgt sofort $v \ge |u_1 - u_2|$.
			Wir haben schließlich
			\[
				\|u_1 - u_2\|_\infty
				\le \|v\|_\infty
				\le r^2 \|f_1 - f_2\|_\infty.
			\]
		\item
			Falls eine Lösung $\tilde u$ für
			\begin{align*}
				-\Laplace \tilde u &= f_1 && \text{in $\Omega$}, \\
				\tilde u &= g_2 && \text{auf $\Boundary \Omega$}
			\end{align*}
			existiert, so ergibt sich mit b) und c)
			\begin{align*}
				\|u_1 - u_2\|_\infty &\le \underbrace{\|u_1 - \tilde u\|_\infty}_{\le c_1 \|g_1 - g_2\|_\infty} + \underbrace{\|\tilde u - u_2\|_\infty}_{\le c_2 \|f_1 - f_2\|_\infty} \\
				&\le C\Big(\|g_1 - g_2\|_\infty + \|f_1 - f_2\|_\infty \Big).
			\end{align*}
			Alternativ lässt sich auch die Methode aus c) anwenden mit der Wahl
			\[
				v(x) := \|g_1 - g_2\|_\infty + (r^2 - \f 12 x_1^2) \|f_1 - f_2\|_\infty.
			\]
	\end{enumerate}
\end{exercise}

\newpage
\begin{exercise}[Aufgabe 3]
	\begin{enumerate}[a)]
		\item
			Gemäß Beispiel 1.26 aus der Vorlesung ist jede klassische Lösung entlang der charakteristischen Linien konstant.
			Diese sind bestimmt durch $\gamma(0) = x_0, \gamma'(t) = f'(u(\gamma(t),t))$.
			Damit ergibt sich
			\begin{align*}
				\gamma'(t) &= u(\gamma(t),t) = \const = u(\gamma(0),0) = u_0(x_0), \\
				\gamma(t) &= \int_0^t u_0(x_0) \di[s] + x_0 = t u_0(x_0) + x_0 \\
				&= \begin{cases}
					x_0 & x_0 \le 0\\
					t x_0^2(x_0 - 2)^2 + x_0 & 0 < x_0 < 1 \\
					t + x_0 & 1 \le x_0
				\end{cases}
			\end{align*}
			Ziel ist es, obigen Ausdruck nach $x_0$ aufzulösen und in $u_0$ einzusetzen, um eine explizite Parametrisierung zu erhalten.

			Die quartische Gleichung im Fall $0 < x_0 < 1$ besitzt eine explizite Lösungsformel, wir zeigen also lediglich, dass innerhalb von $(0,1)$ genau eine Lösung $x_0$ existiert.

			$g(x_0) = tx_0^2(x_0-2)^2 + x_0$ ist für jedes feste $t \in (0,\infty)$ innerhalb von $x_0 \in (0,1)$ streng monoton steigend, da
			\begin{align*}
				g'(x_0) &= 2tx_0(x_0-2)^2 + 2tx_0^2(x_0-2) + 1 \\
				&= 2t x_0 \underbrace{(x_0-2)}_{<0}\underbrace{(2x_0-2)}_{<0} + 1 \\
				&> 0.
			\end{align*}
			Damit $g$ injektiv auf $(0,1)$ und wegen $\im g = (0,1+t) \subset \R$ existiert für jedes feste $t \in (0,\infty)$ und $x \in (0,1+t)$ eine eindeutige Lösung $x_0 \in (0,1)$ für obige quartische Gleichung.
			Sei $x_0(x,t)$ diese Lösung, so ergibt sich durch elementares Umformen die explizite Lösung
			\[
				u(x,t) = \begin{cases}
					0 & x \le 0 \\
					\frac{x - x_0(x,t)}{t} & 0 < x < 1 + t \\
					1 & 1 + t < x
				\end{cases}
			\]
		\item
			Geht man vor wie in der a), so erhält man für eine klassische Lösung die notwendige Bedingung
			\[
				u(x,t) = \begin{cases}
					1 & x \le t \\
					0 & x \ge 1
				\end{cases}
			\]
			Diese ist für $t \ge 1$ nicht erfüllbar, es ergibt sich eine obere Schranke für die Endzeit: $T = 1$.
	\end{enumerate}
\end{exercise}

\end{document}
