\documentclass{myexercise}

\setcounter{MaxMatrixCols}{20}

\begin{document}

\begin{exercise}[Programmieraufgabe]
  \begin{enumerate}[a)]
    \item
      Siehe \Code{FDSquaresSimple.buildGrid()}.
      Zunächst wird das Gebiet als Maske erzeugt (\Code{FDSquaresSimple.mask}), anschließend die Nicht-Null-Einträge mittels \Code{find()} ermittelt und die Indizierung dieses Vektors als Nummerierung für die Gitterpunkte verwendet.
      \begin{figure}[ht]
        \centering
        \[
          \begin{matrix}
       &     &     &     & 129 & 130 & 131 & 132 & 133 & 134 & 135 & 136 & 137 \\
       &     &     &     & 120 & 121 & 122 & 123 & 124 & 125 & 126 & 127 & 128 \\
       &     &     &     & 111 & 112 & 113 & 114 & 115 & 116 & 117 & 118 & 119 \\
       &     &     &     & 102 & 103 & 104 & 105 & 106 & 107 & 108 & 109 & 110 \\
    89 &  90 &  91 &  92 &  93 &  94 &  95 &  96 &  97 &  98 &  99 & 100 & 101 \\
    76 &  77 &  78 &  79 &  80 &  81 &  82 &  83 &  84 &  85 &  86 &  87 &  88 \\
    63 &  64 &  65 &  66 &  67 &  68 &  69 &  70 &  71 &  72 &  73 &  74 &  75 \\
    50 &  51 &  52 &  53 &  54 &  55 &  56 &  57 &  58 &  59 &  60 &  61 &  62 \\
    37 &  38 &  39 &  40 &  41 &  42 &  43 &  44 &  45 &  46 &  47 &  48 &  49 \\
    28 &  29 &  30 &  31 &  32 &  33 &  34 &  35 &  36 &     &     &     &     \\
    19 &  20 &  21 &  22 &  23 &  24 &  25 &  26 &  27 &     &     &     &     \\
    10 &  11 &  12 &  13 &  14 &  15 &  16 &  17 &  18 &     &     &     &     \\
     1 &   2 &   3 &   4 &   5 &   6 &   7 &   8 &   9 &     &     &     &    
          \end{matrix}
        \]
        \caption{Nummerierung der Gitterpunkte für $h = 2^{-2}$, $1$ entspricht $(0,0)$, $9$ entspricht $(2,0)$}
      \end{figure}
    \item
      no-op
    \item
      Siehe \Code{FDSquaresSimple.initialize()}.
    \item
      Siehe \Code{FiniteDifferenceSolver.solve()}.
    \item
      Siehe \Code{FDSquaresSimple.plot()}.
    \item
      Siehe \Code{FDSquaresSimple.computeError()}.
    \item
      Siehe Tabelle in \ref{fig:err} (Halbierung von $h$ ergibt ungefähr Viertelung von $\|u_h - u\|$) und Abbildung \ref{fig:err} ($\lim_{h\to 0} \frac{\|u_h - u\|_\infty}{h^2} = C$ mit $0 < C < \infty$ deutet auf quadratische Konvergenzrate hin).
      \begin{table}[ht]
        \begin{subtable}[c]{0.5\textwidth}
          \centering
          \begin{tabular}{cc}
            $h$ & $\|u_h - u\|_\infty$ \\ \hline
     0.250000000000000 & 0.192771116152572 \\
     0.125000000000000 & 0.044761850924581 \\
     0.062500000000000 & 0.010989314920693 \\
     0.031250000000000 & 0.002734954832517 \\
     0.015625000000000 & 0.000682968393772 \\
     0.007812500000000 & 0.000170694001378 \\
     0.003906250000000 & 0.000042670495021 \\
     0.001953125000000 & 0.000010667435937 \\
     0.000976562500000 & 0.000002666847241
          \end{tabular}
          \caption{Approximationsfehler für $h = 2^{-k}$, $2 \le k \le 10$}
        \end{subtable}
        \begin{subfigure}[c]{0.5\textwidth}
	   \includegraphics[width=\textwidth]{err.png}
          \caption{$\frac{\|u_h - u\|_\infty}{h^2}$ über $h$ geplottet}
        \end{subfigure}
        \caption{Approximationsfehler}
        \label{fig:err}
      \end{table}
    \item
      Siehe Abbildung \ref{fig:vis}.
      \begin{figure}[ht]
        \begin{subfigure}[t]{0.5\textwidth}
	   \includegraphics[width=\textwidth]{vis1.png}
	   \caption{$h = 2^{-2}$}
        \end{subfigure}
        \begin{subfigure}[t]{0.5\textwidth}
	   \includegraphics[width=\textwidth]{vis2.png}
	   \caption{$h = 2^{-4}$}
        \end{subfigure}
        \begin{subfigure}[t]{0.5\textwidth}
	   \includegraphics[width=\textwidth]{vis3.png}
	   \caption{$h = 2^{-8}$, ohne Gitterlinien}
        \end{subfigure}
	\caption{Visualisierungen für verschiedene $h$}
        \label{fig:vis}
      \end{figure}
  \end{enumerate}
\end{exercise}

\end{document}
