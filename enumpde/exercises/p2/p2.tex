\documentclass{myexercise}

\usepackage{pgfplots}
\pgfplotsset{compat=newest}

\begin{document}

\begin{exercise}[Programmieraufgabe]
  \begin{enumerate}[a)]
    \item
      Siehe \Code{semicircleg.m} und \Code{sectorg.m}.
    \item
      Siehe \Code{mypdeplot.m}.
    \item
      Siehe \Code{bisect.m} für eine Implementierung des “longest-edge bisection” Algorithmus nach Rivara.
      Hier werden Dreiecke stets an der längsten Seite halbiert.
      Um Konsistenz zu wahren, müssen in der Regel mehr Bisektionen durchgeführt werden, als angefordert.

      \Code{bisect2.m} ist eine etwas naïvere Implementierung, arbeitet allerdings vollständig vektoriell.
      Auch hier werden Dreiecke stets an der längsten Seite halbiert.
      Es werden nur solche Dreiecke halbiert, welche ihre längste Kante am Rand haben, oder paarweise eine gemeinsame längste Seite besitzen.
    \item
      Siehe \Code{visualize.m} und Abbildung \ref{fig:vis}.

      \newlength\fheight
      \newlength\fwidth
      \begin{figure}[ht]
        \setlength\fheight{0.14\textwidth}
        \setlength\fwidth{0.28\textwidth}
        \begin{subfigure}[t]{0.32\textwidth}
	   % This file was created by matlab2tikz v0.5.0 running on MATLAB 8.3.
% Copyright (c) 2008--2014, Nico Schlömer <nico.schloemer@gmail.com>
% All rights reserved.
% Minimal pgfplots version: 1.3
% 
% The latest updates can be retrieved from
%   http://www.mathworks.com/matlabcentral/fileexchange/22022-matlab2tikz
% where you can also make suggestions and rate matlab2tikz.
% 
\begin{tikzpicture}

\begin{axis}[%
width=0.950920245398773\fwidth,
height=\fheight,
at={(0\fwidth,0\fheight)},
scale only axis,
xmin=-1,
xmax=1,
ymin=0,
ymax=1,
axis x line*=bottom,
axis y line*=left
]

\addplot[area legend,solid,fill=white,draw=blue,forget plot]
table[row sep=crcr] {%
x	y\\
-0.923879532511287	0.38268343236509\\
-1	1.22464679914735e-16\\
-0.685468610907825	0.306663883939896\\
}--cycle;


\addplot[area legend,solid,fill=white,draw=blue,forget plot]
table[row sep=crcr] {%
x	y\\
0.38268343236509	0.923879532511287\\
6.12323399573677e-17	1\\
-0.00927134816825834	0.745786196144667\\
}--cycle;


\addplot[area legend,solid,fill=white,draw=blue,forget plot]
table[row sep=crcr] {%
x	y\\
1	0\\
0.923879532511287	0.38268343236509\\
0.5	0\\
}--cycle;


\addplot[area legend,solid,fill=white,draw=blue,forget plot]
table[row sep=crcr] {%
x	y\\
0.5	0\\
0.923879532511287	0.38268343236509\\
0.595246578424459	0.38185823077614\\
}--cycle;


\addplot[area legend,solid,fill=white,draw=blue,forget plot]
table[row sep=crcr] {%
x	y\\
0.24797149880479	0.454954002717384\\
0.38268343236509	0.923879532511287\\
-0.00927134816825834	0.745786196144667\\
}--cycle;


\addplot[area legend,solid,fill=white,draw=blue,forget plot]
table[row sep=crcr] {%
x	y\\
6.12323399573677e-17	1\\
-0.38268343236509	0.923879532511287\\
-0.00927134816825834	0.745786196144667\\
}--cycle;


\addplot[area legend,solid,fill=white,draw=blue,forget plot]
table[row sep=crcr] {%
x	y\\
0.923879532511287	0.38268343236509\\
0.707106781186548	0.707106781186547\\
0.595246578424459	0.38185823077614\\
}--cycle;


\addplot[area legend,solid,fill=white,draw=blue,forget plot]
table[row sep=crcr] {%
x	y\\
0.707106781186548	0.707106781186547\\
0.38268343236509	0.923879532511287\\
0.24797149880479	0.454954002717384\\
}--cycle;


\addplot[area legend,solid,fill=white,draw=blue,forget plot]
table[row sep=crcr] {%
x	y\\
-0.38268343236509	0.923879532511287\\
-0.707106781186547	0.707106781186548\\
-0.297626820073946	0.443191615206783\\
}--cycle;


\addplot[area legend,solid,fill=white,draw=blue,forget plot]
table[row sep=crcr] {%
x	y\\
-0.297626820073946	0.443191615206783\\
-0.707106781186547	0.707106781186548\\
-0.685468610907825	0.306663883939896\\
}--cycle;


\addplot[area legend,solid,fill=white,draw=blue,forget plot]
table[row sep=crcr] {%
x	y\\
-0.707106781186547	0.707106781186548\\
-0.923879532511287	0.38268343236509\\
-0.685468610907825	0.306663883939896\\
}--cycle;


\addplot[area legend,solid,fill=white,draw=blue,forget plot]
table[row sep=crcr] {%
x	y\\
0	0\\
0.5	0\\
0.24797149880479	0.454954002717384\\
}--cycle;


\addplot[area legend,solid,fill=white,draw=blue,forget plot]
table[row sep=crcr] {%
x	y\\
-0.5	0\\
0	0\\
-0.297626820073946	0.443191615206783\\
}--cycle;


\addplot[area legend,solid,fill=white,draw=blue,forget plot]
table[row sep=crcr] {%
x	y\\
0	0\\
0.24797149880479	0.454954002717384\\
-0.297626820073946	0.443191615206783\\
}--cycle;


\addplot[area legend,solid,fill=white,draw=blue,forget plot]
table[row sep=crcr] {%
x	y\\
0.24797149880479	0.454954002717384\\
0.5	0\\
0.595246578424459	0.38185823077614\\
}--cycle;


\addplot[area legend,solid,fill=white,draw=blue,forget plot]
table[row sep=crcr] {%
x	y\\
0.707106781186548	0.707106781186547\\
0.24797149880479	0.454954002717384\\
0.595246578424459	0.38185823077614\\
}--cycle;


\addplot[area legend,solid,fill=white,draw=blue,forget plot]
table[row sep=crcr] {%
x	y\\
-1	1.22464679914735e-16\\
-0.5	0\\
-0.685468610907825	0.306663883939896\\
}--cycle;


\addplot[area legend,solid,fill=white,draw=blue,forget plot]
table[row sep=crcr] {%
x	y\\
-0.5	0\\
-0.297626820073946	0.443191615206783\\
-0.685468610907825	0.306663883939896\\
}--cycle;


\addplot[area legend,solid,fill=white,draw=blue,forget plot]
table[row sep=crcr] {%
x	y\\
-0.297626820073946	0.443191615206783\\
0.24797149880479	0.454954002717384\\
-0.00927134816825834	0.745786196144667\\
}--cycle;


\addplot[area legend,solid,fill=white,draw=blue,forget plot]
table[row sep=crcr] {%
x	y\\
-0.38268343236509	0.923879532511287\\
-0.297626820073946	0.443191615206783\\
-0.00927134816825834	0.745786196144667\\
}--cycle;

\addplot [color=red,solid,forget plot]
  table[row sep=crcr]{%
1	0\\
0.923879532511287	0.38268343236509\\
};
\addplot [color=red,solid,forget plot]
  table[row sep=crcr]{%
0.923879532511287	0.38268343236509\\
0.707106781186548	0.707106781186547\\
};
\addplot [color=red,solid,forget plot]
  table[row sep=crcr]{%
0.707106781186548	0.707106781186547\\
0.38268343236509	0.923879532511287\\
};
\addplot [color=red,solid,forget plot]
  table[row sep=crcr]{%
0.38268343236509	0.923879532511287\\
6.12323399573677e-17	1\\
};
\addplot [color=red,solid,forget plot]
  table[row sep=crcr]{%
6.12323399573677e-17	1\\
-0.38268343236509	0.923879532511287\\
};
\addplot [color=red,solid,forget plot]
  table[row sep=crcr]{%
-0.38268343236509	0.923879532511287\\
-0.707106781186547	0.707106781186548\\
};
\addplot [color=red,solid,forget plot]
  table[row sep=crcr]{%
-0.707106781186547	0.707106781186548\\
-0.923879532511287	0.38268343236509\\
};
\addplot [color=red,solid,forget plot]
  table[row sep=crcr]{%
-0.923879532511287	0.38268343236509\\
-1	1.22464679914735e-16\\
};
\addplot [color=red,solid,forget plot]
  table[row sep=crcr]{%
-1	1.22464679914735e-16\\
-0.5	0\\
};
\addplot [color=red,solid,forget plot]
  table[row sep=crcr]{%
-0.5	0\\
0	0\\
};
\addplot [color=red,solid,forget plot]
  table[row sep=crcr]{%
0	0\\
0.5	0\\
};
\addplot [color=red,solid,forget plot]
  table[row sep=crcr]{%
0.5	0\\
1	0\\
};
\end{axis}
\end{tikzpicture}%
	   \caption{\Code{semicircleg}, Initialtriangulierung}
        \end{subfigure}
        \begin{subfigure}[t]{0.32\textwidth}
	   % This file was created by matlab2tikz v0.5.0 running on MATLAB 8.3.
% Copyright (c) 2008--2014, Nico Schlömer <nico.schloemer@gmail.com>
% All rights reserved.
% Minimal pgfplots version: 1.3
% 
% The latest updates can be retrieved from
%   http://www.mathworks.com/matlabcentral/fileexchange/22022-matlab2tikz
% where you can also make suggestions and rate matlab2tikz.
% 
\begin{tikzpicture}

\begin{axis}[%
width=0.950920245398773\fwidth,
height=\fheight,
at={(0\fwidth,0\fheight)},
scale only axis,
xmin=-1,
xmax=1,
ymin=0,
ymax=1,
axis x line*=bottom,
axis y line*=left
]

\addplot[area legend,solid,fill=white,draw=blue,forget plot]
table[row sep=crcr] {%
x	y\\
-0.75	0\\
-0.5	0\\
-0.592734305453913	0.153331941969948\\
}--cycle;


\addplot[area legend,solid,fill=white,draw=blue,forget plot]
table[row sep=crcr] {%
x	y\\
-0.398813410036973	0.221595807603391\\
-0.685468610907825	0.306663883939896\\
-0.592734305453913	0.153331941969948\\
}--cycle;


\addplot[area legend,solid,fill=white,draw=blue,forget plot]
table[row sep=crcr] {%
x	y\\
-0.398813410036973	0.221595807603391\\
-0.592734305453913	0.153331941969948\\
-0.5	0\\
}--cycle;


\addplot[area legend,solid,fill=white,draw=blue,forget plot]
table[row sep=crcr] {%
x	y\\
-0.842734305453913	0.153331941969948\\
-0.923879532511287	0.38268343236509\\
-0.98078528040323	0.195090322016129\\
}--cycle;


\addplot[area legend,solid,fill=white,draw=blue,forget plot]
table[row sep=crcr] {%
x	y\\
-0.842734305453913	0.153331941969948\\
-0.98078528040323	0.195090322016129\\
-1	1.22464679914735e-16\\
}--cycle;


\addplot[area legend,solid,fill=white,draw=blue,forget plot]
table[row sep=crcr] {%
x	y\\
-0.696287696047186	0.506885332563222\\
-0.707106781186547	0.707106781186548\\
-0.831469612302545	0.555570233019602\\
}--cycle;


\addplot[area legend,solid,fill=white,draw=blue,forget plot]
table[row sep=crcr] {%
x	y\\
-0.685468610907825	0.306663883939896\\
-0.696287696047186	0.506885332563222\\
-0.810083614279236	0.444784382464156\\
}--cycle;


\addplot[area legend,solid,fill=white,draw=blue,forget plot]
table[row sep=crcr] {%
x	y\\
-0.831469612302545	0.555570233019602\\
-0.923879532511287	0.38268343236509\\
-0.810083614279236	0.444784382464156\\
}--cycle;


\addplot[area legend,solid,fill=white,draw=blue,forget plot]
table[row sep=crcr] {%
x	y\\
-0.831469612302545	0.555570233019602\\
-0.810083614279236	0.444784382464156\\
-0.696287696047186	0.506885332563222\\
}--cycle;


\addplot[area legend,solid,fill=white,draw=blue,forget plot]
table[row sep=crcr] {%
x	y\\
-0.842734305453913	0.153331941969948\\
-0.685468610907825	0.306663883939896\\
-0.804674071709556	0.344673658152493\\
}--cycle;


\addplot[area legend,solid,fill=white,draw=blue,forget plot]
table[row sep=crcr] {%
x	y\\
-0.842734305453913	0.153331941969948\\
-0.804674071709556	0.344673658152493\\
-0.923879532511287	0.38268343236509\\
}--cycle;


\addplot[area legend,solid,fill=white,draw=blue,forget plot]
table[row sep=crcr] {%
x	y\\
-0.810083614279236	0.444784382464156\\
-0.923879532511287	0.38268343236509\\
-0.804674071709556	0.344673658152493\\
}--cycle;


\addplot[area legend,solid,fill=white,draw=blue,forget plot]
table[row sep=crcr] {%
x	y\\
-0.810083614279236	0.444784382464156\\
-0.804674071709556	0.344673658152493\\
-0.685468610907825	0.306663883939896\\
}--cycle;


\addplot[area legend,solid,fill=white,draw=blue,forget plot]
table[row sep=crcr] {%
x	y\\
-0.842734305453913	0.153331941969948\\
-0.75	0\\
-0.717734305453913	0.153331941969948\\
}--cycle;


\addplot[area legend,solid,fill=white,draw=blue,forget plot]
table[row sep=crcr] {%
x	y\\
-0.842734305453913	0.153331941969948\\
-0.717734305453913	0.153331941969948\\
-0.685468610907825	0.306663883939896\\
}--cycle;


\addplot[area legend,solid,fill=white,draw=blue,forget plot]
table[row sep=crcr] {%
x	y\\
-0.592734305453913	0.153331941969948\\
-0.685468610907825	0.306663883939896\\
-0.717734305453913	0.153331941969948\\
}--cycle;


\addplot[area legend,solid,fill=white,draw=blue,forget plot]
table[row sep=crcr] {%
x	y\\
-0.592734305453913	0.153331941969948\\
-0.717734305453913	0.153331941969948\\
-0.75	0\\
}--cycle;


\addplot[area legend,solid,fill=white,draw=blue,forget plot]
table[row sep=crcr] {%
x	y\\
-0.842734305453913	0.153331941969948\\
-1	1.22464679914735e-16\\
-0.875	0\\
}--cycle;


\addplot[area legend,solid,fill=white,draw=blue,forget plot]
table[row sep=crcr] {%
x	y\\
-0.842734305453913	0.153331941969948\\
-0.875	0\\
-0.75	0\\
}--cycle;


\addplot[area legend,solid,fill=white,draw=blue,forget plot]
table[row sep=crcr] {%
x	y\\
0.477539139995669	0.581030391951966\\
0.707106781186548	0.707106781186547\\
0.555570233019602	0.831469612302545\\
}--cycle;


\addplot[area legend,solid,fill=white,draw=blue,forget plot]
table[row sep=crcr] {%
x	y\\
0.477539139995669	0.581030391951966\\
0.595246578424459	0.38185823077614\\
0.651176679805503	0.544482505981344\\
}--cycle;


\addplot[area legend,solid,fill=white,draw=blue,forget plot]
table[row sep=crcr] {%
x	y\\
0.477539139995669	0.581030391951966\\
0.651176679805503	0.544482505981344\\
0.707106781186548	0.707106781186547\\
}--cycle;


\addplot[area legend,solid,fill=white,draw=blue,forget plot]
table[row sep=crcr] {%
x	y\\
0.831469612302545	0.555570233019602\\
0.707106781186548	0.707106781186547\\
0.651176679805503	0.544482505981344\\
}--cycle;


\addplot[area legend,solid,fill=white,draw=blue,forget plot]
table[row sep=crcr] {%
x	y\\
0.831469612302545	0.555570233019602\\
0.651176679805503	0.544482505981344\\
0.595246578424459	0.38185823077614\\
}--cycle;


\addplot[area legend,solid,fill=white,draw=blue,forget plot]
table[row sep=crcr] {%
x	y\\
0.477539139995669	0.581030391951966\\
0.421609038614624	0.418406116746762\\
0.595246578424459	0.38185823077614\\
}--cycle;


\addplot[area legend,solid,fill=white,draw=blue,forget plot]
table[row sep=crcr] {%
x	y\\
0.373985749402395	0.227477001358692\\
0.595246578424459	0.38185823077614\\
0.421609038614624	0.418406116746762\\
}--cycle;


\addplot[area legend,solid,fill=white,draw=blue,forget plot]
table[row sep=crcr] {%
x	y\\
0.373985749402395	0.227477001358692\\
0.421609038614624	0.418406116746762\\
0.24797149880479	0.454954002717384\\
}--cycle;


\addplot[area legend,solid,fill=white,draw=blue,forget plot]
table[row sep=crcr] {%
x	y\\
0.31532746558494	0.689416767614335\\
0.119350075318266	0.600370099431026\\
0.24797149880479	0.454954002717384\\
}--cycle;


\addplot[area legend,solid,fill=white,draw=blue,forget plot]
table[row sep=crcr] {%
x	y\\
-0.0248276606345778	0.449072808962083\\
0.24797149880479	0.454954002717384\\
0.119350075318266	0.600370099431026\\
}--cycle;


\addplot[area legend,solid,fill=white,draw=blue,forget plot]
table[row sep=crcr] {%
x	y\\
-0.0248276606345778	0.449072808962083\\
0.119350075318266	0.600370099431026\\
-0.00927134816825834	0.745786196144667\\
}--cycle;


\addplot[area legend,solid,fill=white,draw=blue,forget plot]
table[row sep=crcr] {%
x	y\\
0.31532746558494	0.689416767614335\\
0.477539139995669	0.581030391951966\\
0.430111286180379	0.752454962231626\\
}--cycle;


\addplot[area legend,solid,fill=white,draw=blue,forget plot]
table[row sep=crcr] {%
x	y\\
0.555570233019602	0.831469612302545\\
0.38268343236509	0.923879532511287\\
0.430111286180379	0.752454962231626\\
}--cycle;


\addplot[area legend,solid,fill=white,draw=blue,forget plot]
table[row sep=crcr] {%
x	y\\
0.555570233019602	0.831469612302545\\
0.430111286180379	0.752454962231626\\
0.477539139995669	0.581030391951966\\
}--cycle;


\addplot[area legend,solid,fill=white,draw=blue,forget plot]
table[row sep=crcr] {%
x	y\\
0.31532746558494	0.689416767614335\\
0.24797149880479	0.454954002717384\\
0.36275531940023	0.517992197334675\\
}--cycle;


\addplot[area legend,solid,fill=white,draw=blue,forget plot]
table[row sep=crcr] {%
x	y\\
0.31532746558494	0.689416767614335\\
0.36275531940023	0.517992197334675\\
0.477539139995669	0.581030391951966\\
}--cycle;


\addplot[area legend,solid,fill=white,draw=blue,forget plot]
table[row sep=crcr] {%
x	y\\
0.421609038614624	0.418406116746762\\
0.477539139995669	0.581030391951966\\
0.36275531940023	0.517992197334675\\
}--cycle;


\addplot[area legend,solid,fill=white,draw=blue,forget plot]
table[row sep=crcr] {%
x	y\\
0.421609038614624	0.418406116746762\\
0.36275531940023	0.517992197334675\\
0.24797149880479	0.454954002717384\\
}--cycle;


\addplot[area legend,solid,fill=white,draw=blue,forget plot]
table[row sep=crcr] {%
x	y\\
-0.195977390266674	0.834832864327977\\
-0.195090322016128	0.98078528040323\\
-0.38268343236509	0.923879532511287\\
}--cycle;


\addplot[area legend,solid,fill=white,draw=blue,forget plot]
table[row sep=crcr] {%
x	y\\
-0.00927134816825834	0.745786196144667\\
-0.097988695133337	0.917416432163989\\
-0.195977390266674	0.834832864327977\\
}--cycle;


\addplot[area legend,solid,fill=white,draw=blue,forget plot]
table[row sep=crcr] {%
x	y\\
-0.195090322016128	0.98078528040323\\
-0.195977390266674	0.834832864327977\\
-0.097988695133337	0.917416432163989\\
}--cycle;


\addplot[area legend,solid,fill=white,draw=blue,forget plot]
table[row sep=crcr] {%
x	y\\
-0.195090322016128	0.98078528040323\\
-0.097988695133337	0.917416432163989\\
6.12323399573677e-17	1\\
}--cycle;


\addplot[area legend,solid,fill=white,draw=blue,forget plot]
table[row sep=crcr] {%
x	y\\
0.186706042098416	0.834832864327977\\
6.12323399573677e-17	1\\
-0.00463567408412914	0.872893098072334\\
}--cycle;


\addplot[area legend,solid,fill=white,draw=blue,forget plot]
table[row sep=crcr] {%
x	y\\
0.186706042098416	0.834832864327977\\
-0.00463567408412914	0.872893098072334\\
-0.00927134816825834	0.745786196144667\\
}--cycle;


\addplot[area legend,solid,fill=white,draw=blue,forget plot]
table[row sep=crcr] {%
x	y\\
-0.097988695133337	0.917416432163989\\
-0.00927134816825834	0.745786196144667\\
-0.00463567408412914	0.872893098072334\\
}--cycle;


\addplot[area legend,solid,fill=white,draw=blue,forget plot]
table[row sep=crcr] {%
x	y\\
-0.097988695133337	0.917416432163989\\
-0.00463567408412914	0.872893098072334\\
6.12323399573677e-17	1\\
}--cycle;


\addplot[area legend,solid,fill=white,draw=blue,forget plot]
table[row sep=crcr] {%
x	y\\
0.186706042098416	0.834832864327977\\
0.38268343236509	0.923879532511287\\
0.195090322016128	0.98078528040323\\
}--cycle;


\addplot[area legend,solid,fill=white,draw=blue,forget plot]
table[row sep=crcr] {%
x	y\\
0.186706042098416	0.834832864327977\\
0.195090322016128	0.98078528040323\\
6.12323399573677e-17	1\\
}--cycle;


\addplot[area legend,solid,fill=white,draw=blue,forget plot]
table[row sep=crcr] {%
x	y\\
0.186706042098416	0.834832864327977\\
0.31532746558494	0.689416767614335\\
0.349005448975015	0.806648150062811\\
}--cycle;


\addplot[area legend,solid,fill=white,draw=blue,forget plot]
table[row sep=crcr] {%
x	y\\
0.186706042098416	0.834832864327977\\
0.349005448975015	0.806648150062811\\
0.38268343236509	0.923879532511287\\
}--cycle;


\addplot[area legend,solid,fill=white,draw=blue,forget plot]
table[row sep=crcr] {%
x	y\\
0.430111286180379	0.752454962231626\\
0.38268343236509	0.923879532511287\\
0.349005448975015	0.806648150062811\\
}--cycle;


\addplot[area legend,solid,fill=white,draw=blue,forget plot]
table[row sep=crcr] {%
x	y\\
0.430111286180379	0.752454962231626\\
0.349005448975015	0.806648150062811\\
0.31532746558494	0.689416767614335\\
}--cycle;


\addplot[area legend,solid,fill=white,draw=blue,forget plot]
table[row sep=crcr] {%
x	y\\
0.186706042098416	0.834832864327977\\
-0.00927134816825834	0.745786196144667\\
0.153028058708341	0.717601481879501\\
}--cycle;


\addplot[area legend,solid,fill=white,draw=blue,forget plot]
table[row sep=crcr] {%
x	y\\
0.186706042098416	0.834832864327977\\
0.153028058708341	0.717601481879501\\
0.31532746558494	0.689416767614335\\
}--cycle;


\addplot[area legend,solid,fill=white,draw=blue,forget plot]
table[row sep=crcr] {%
x	y\\
0.119350075318266	0.600370099431026\\
0.31532746558494	0.689416767614335\\
0.153028058708341	0.717601481879501\\
}--cycle;


\addplot[area legend,solid,fill=white,draw=blue,forget plot]
table[row sep=crcr] {%
x	y\\
0.119350075318266	0.600370099431026\\
0.153028058708341	0.717601481879501\\
-0.00927134816825834	0.745786196144667\\
}--cycle;


\addplot[area legend,solid,fill=white,draw=blue,forget plot]
table[row sep=crcr] {%
x	y\\
0.711939766255643	0.191341716182545\\
1	0\\
0.98078528040323	0.195090322016128\\
}--cycle;


\addplot[area legend,solid,fill=white,draw=blue,forget plot]
table[row sep=crcr] {%
x	y\\
0.711939766255643	0.191341716182545\\
0.98078528040323	0.195090322016128\\
0.923879532511287	0.38268343236509\\
}--cycle;


\addplot[area legend,solid,fill=white,draw=blue,forget plot]
table[row sep=crcr] {%
x	y\\
0.711939766255643	0.191341716182545\\
0.5	0\\
0.75	0\\
}--cycle;


\addplot[area legend,solid,fill=white,draw=blue,forget plot]
table[row sep=crcr] {%
x	y\\
0.711939766255643	0.191341716182545\\
0.75	0\\
1	0\\
}--cycle;


\addplot[area legend,solid,fill=white,draw=blue,forget plot]
table[row sep=crcr] {%
x	y\\
0.711939766255643	0.191341716182545\\
0.595246578424459	0.38185823077614\\
0.547623289212229	0.19092911538807\\
}--cycle;


\addplot[area legend,solid,fill=white,draw=blue,forget plot]
table[row sep=crcr] {%
x	y\\
0.711939766255643	0.191341716182545\\
0.547623289212229	0.19092911538807\\
0.5	0\\
}--cycle;


\addplot[area legend,solid,fill=white,draw=blue,forget plot]
table[row sep=crcr] {%
x	y\\
0.373985749402395	0.227477001358692\\
0.5	0\\
0.547623289212229	0.19092911538807\\
}--cycle;


\addplot[area legend,solid,fill=white,draw=blue,forget plot]
table[row sep=crcr] {%
x	y\\
0.373985749402395	0.227477001358692\\
0.547623289212229	0.19092911538807\\
0.595246578424459	0.38185823077614\\
}--cycle;


\addplot[area legend,solid,fill=white,draw=blue,forget plot]
table[row sep=crcr] {%
x	y\\
0.711939766255643	0.191341716182545\\
0.923879532511287	0.38268343236509\\
0.759563055467873	0.382270831570615\\
}--cycle;


\addplot[area legend,solid,fill=white,draw=blue,forget plot]
table[row sep=crcr] {%
x	y\\
0.711939766255643	0.191341716182545\\
0.759563055467873	0.382270831570615\\
0.595246578424459	0.38185823077614\\
}--cycle;


\addplot[area legend,solid,fill=white,draw=blue,forget plot]
table[row sep=crcr] {%
x	y\\
0.831469612302545	0.555570233019602\\
0.595246578424459	0.38185823077614\\
0.759563055467873	0.382270831570615\\
}--cycle;


\addplot[area legend,solid,fill=white,draw=blue,forget plot]
table[row sep=crcr] {%
x	y\\
0.831469612302545	0.555570233019602\\
0.759563055467873	0.382270831570615\\
0.923879532511287	0.38268343236509\\
}--cycle;


\addplot[area legend,solid,fill=white,draw=blue,forget plot]
table[row sep=crcr] {%
x	y\\
-0.0248276606345778	0.449072808962083\\
-0.00927134816825834	0.745786196144667\\
-0.153449084121102	0.594488905675725\\
}--cycle;


\addplot[area legend,solid,fill=white,draw=blue,forget plot]
table[row sep=crcr] {%
x	y\\
-0.195977390266674	0.834832864327977\\
-0.340155126219518	0.683535573859035\\
-0.174713237193888	0.714660885001851\\
}--cycle;


\addplot[area legend,solid,fill=white,draw=blue,forget plot]
table[row sep=crcr] {%
x	y\\
-0.195977390266674	0.834832864327977\\
-0.174713237193888	0.714660885001851\\
-0.00927134816825834	0.745786196144667\\
}--cycle;


\addplot[area legend,solid,fill=white,draw=blue,forget plot]
table[row sep=crcr] {%
x	y\\
-0.153449084121102	0.594488905675725\\
-0.00927134816825834	0.745786196144667\\
-0.174713237193888	0.714660885001851\\
}--cycle;


\addplot[area legend,solid,fill=white,draw=blue,forget plot]
table[row sep=crcr] {%
x	y\\
-0.153449084121102	0.594488905675725\\
-0.174713237193888	0.714660885001851\\
-0.340155126219518	0.683535573859035\\
}--cycle;


\addplot[area legend,solid,fill=white,draw=blue,forget plot]
table[row sep=crcr] {%
x	y\\
-0.555570233019602	0.831469612302545\\
-0.707106781186547	0.707106781186548\\
-0.523630953703033	0.695321177522791\\
}--cycle;


\addplot[area legend,solid,fill=white,draw=blue,forget plot]
table[row sep=crcr] {%
x	y\\
-0.502366800630247	0.575149198196665\\
-0.340155126219518	0.683535573859035\\
-0.523630953703033	0.695321177522791\\
}--cycle;


\addplot[area legend,solid,fill=white,draw=blue,forget plot]
table[row sep=crcr] {%
x	y\\
-0.38268343236509	0.923879532511287\\
-0.555570233019602	0.831469612302545\\
-0.44786267961956	0.75750259308079\\
}--cycle;


\addplot[area legend,solid,fill=white,draw=blue,forget plot]
table[row sep=crcr] {%
x	y\\
-0.523630953703033	0.695321177522791\\
-0.340155126219518	0.683535573859035\\
-0.44786267961956	0.75750259308079\\
}--cycle;


\addplot[area legend,solid,fill=white,draw=blue,forget plot]
table[row sep=crcr] {%
x	y\\
-0.523630953703033	0.695321177522791\\
-0.44786267961956	0.75750259308079\\
-0.555570233019602	0.831469612302545\\
}--cycle;


\addplot[area legend,solid,fill=white,draw=blue,forget plot]
table[row sep=crcr] {%
x	y\\
-0.195977390266674	0.834832864327977\\
-0.38268343236509	0.923879532511287\\
-0.361419279292304	0.803707553185161\\
}--cycle;


\addplot[area legend,solid,fill=white,draw=blue,forget plot]
table[row sep=crcr] {%
x	y\\
-0.195977390266674	0.834832864327977\\
-0.361419279292304	0.803707553185161\\
-0.340155126219518	0.683535573859035\\
}--cycle;


\addplot[area legend,solid,fill=white,draw=blue,forget plot]
table[row sep=crcr] {%
x	y\\
-0.44786267961956	0.75750259308079\\
-0.340155126219518	0.683535573859035\\
-0.361419279292304	0.803707553185161\\
}--cycle;


\addplot[area legend,solid,fill=white,draw=blue,forget plot]
table[row sep=crcr] {%
x	y\\
-0.44786267961956	0.75750259308079\\
-0.361419279292304	0.803707553185161\\
-0.38268343236509	0.923879532511287\\
}--cycle;


\addplot[area legend,solid,fill=white,draw=blue,forget plot]
table[row sep=crcr] {%
x	y\\
-0.502366800630247	0.575149198196665\\
-0.491547715490885	0.374927749573339\\
-0.297626820073946	0.443191615206783\\
}--cycle;


\addplot[area legend,solid,fill=white,draw=blue,forget plot]
table[row sep=crcr] {%
x	y\\
-0.398813410036973	0.221595807603391\\
-0.297626820073946	0.443191615206783\\
-0.491547715490885	0.374927749573339\\
}--cycle;


\addplot[area legend,solid,fill=white,draw=blue,forget plot]
table[row sep=crcr] {%
x	y\\
-0.398813410036973	0.221595807603391\\
-0.491547715490885	0.374927749573339\\
-0.685468610907825	0.306663883939896\\
}--cycle;


\addplot[area legend,solid,fill=white,draw=blue,forget plot]
table[row sep=crcr] {%
x	y\\
-0.502366800630247	0.575149198196665\\
-0.297626820073946	0.443191615206783\\
-0.318890973146732	0.563363594532909\\
}--cycle;


\addplot[area legend,solid,fill=white,draw=blue,forget plot]
table[row sep=crcr] {%
x	y\\
-0.502366800630247	0.575149198196665\\
-0.318890973146732	0.563363594532909\\
-0.340155126219518	0.683535573859035\\
}--cycle;


\addplot[area legend,solid,fill=white,draw=blue,forget plot]
table[row sep=crcr] {%
x	y\\
-0.153449084121102	0.594488905675725\\
-0.340155126219518	0.683535573859035\\
-0.318890973146732	0.563363594532909\\
}--cycle;


\addplot[area legend,solid,fill=white,draw=blue,forget plot]
table[row sep=crcr] {%
x	y\\
-0.153449084121102	0.594488905675725\\
-0.318890973146732	0.563363594532909\\
-0.297626820073946	0.443191615206783\\
}--cycle;


\addplot[area legend,solid,fill=white,draw=blue,forget plot]
table[row sep=crcr] {%
x	y\\
-0.696287696047186	0.506885332563222\\
-0.502366800630247	0.575149198196665\\
-0.604736790908397	0.641127989691606\\
}--cycle;


\addplot[area legend,solid,fill=white,draw=blue,forget plot]
table[row sep=crcr] {%
x	y\\
-0.696287696047186	0.506885332563222\\
-0.604736790908397	0.641127989691606\\
-0.707106781186547	0.707106781186548\\
}--cycle;


\addplot[area legend,solid,fill=white,draw=blue,forget plot]
table[row sep=crcr] {%
x	y\\
-0.523630953703033	0.695321177522791\\
-0.707106781186547	0.707106781186548\\
-0.604736790908397	0.641127989691606\\
}--cycle;


\addplot[area legend,solid,fill=white,draw=blue,forget plot]
table[row sep=crcr] {%
x	y\\
-0.523630953703033	0.695321177522791\\
-0.604736790908397	0.641127989691606\\
-0.502366800630247	0.575149198196665\\
}--cycle;


\addplot[area legend,solid,fill=white,draw=blue,forget plot]
table[row sep=crcr] {%
x	y\\
-0.696287696047186	0.506885332563222\\
-0.685468610907825	0.306663883939896\\
-0.593917705769036	0.44090654106828\\
}--cycle;


\addplot[area legend,solid,fill=white,draw=blue,forget plot]
table[row sep=crcr] {%
x	y\\
-0.696287696047186	0.506885332563222\\
-0.593917705769036	0.44090654106828\\
-0.502366800630247	0.575149198196665\\
}--cycle;


\addplot[area legend,solid,fill=white,draw=blue,forget plot]
table[row sep=crcr] {%
x	y\\
-0.491547715490885	0.374927749573339\\
-0.502366800630247	0.575149198196665\\
-0.593917705769036	0.44090654106828\\
}--cycle;


\addplot[area legend,solid,fill=white,draw=blue,forget plot]
table[row sep=crcr] {%
x	y\\
-0.491547715490885	0.374927749573339\\
-0.593917705769036	0.44090654106828\\
-0.685468610907825	0.306663883939896\\
}--cycle;


\addplot[area legend,solid,fill=white,draw=blue,forget plot]
table[row sep=crcr] {%
x	y\\
0.373985749402395	0.227477001358692\\
0	0\\
0.25	0\\
}--cycle;


\addplot[area legend,solid,fill=white,draw=blue,forget plot]
table[row sep=crcr] {%
x	y\\
0.373985749402395	0.227477001358692\\
0.25	0\\
0.5	0\\
}--cycle;


\addplot[area legend,solid,fill=white,draw=blue,forget plot]
table[row sep=crcr] {%
x	y\\
0.373985749402395	0.227477001358692\\
0.24797149880479	0.454954002717384\\
0.123985749402395	0.227477001358692\\
}--cycle;


\addplot[area legend,solid,fill=white,draw=blue,forget plot]
table[row sep=crcr] {%
x	y\\
0.373985749402395	0.227477001358692\\
0.123985749402395	0.227477001358692\\
0	0\\
}--cycle;


\addplot[area legend,solid,fill=white,draw=blue,forget plot]
table[row sep=crcr] {%
x	y\\
-0.0248276606345778	0.449072808962083\\
0.123985749402395	0.227477001358692\\
0.24797149880479	0.454954002717384\\
}--cycle;


\addplot[area legend,solid,fill=white,draw=blue,forget plot]
table[row sep=crcr] {%
x	y\\
-0.148813410036973	0.221595807603391\\
-0.25	0\\
0	0\\
}--cycle;


\addplot[area legend,solid,fill=white,draw=blue,forget plot]
table[row sep=crcr] {%
x	y\\
-0.148813410036973	0.221595807603391\\
-0.0248276606345778	0.449072808962083\\
-0.161227240354262	0.446132212084433\\
}--cycle;


\addplot[area legend,solid,fill=white,draw=blue,forget plot]
table[row sep=crcr] {%
x	y\\
-0.153449084121102	0.594488905675725\\
-0.297626820073946	0.443191615206783\\
-0.161227240354262	0.446132212084433\\
}--cycle;


\addplot[area legend,solid,fill=white,draw=blue,forget plot]
table[row sep=crcr] {%
x	y\\
-0.153449084121102	0.594488905675725\\
-0.161227240354262	0.446132212084433\\
-0.0248276606345778	0.449072808962083\\
}--cycle;


\addplot[area legend,solid,fill=white,draw=blue,forget plot]
table[row sep=crcr] {%
x	y\\
-0.148813410036973	0.221595807603391\\
0	0\\
-0.0124138303172889	0.224536404481042\\
}--cycle;


\addplot[area legend,solid,fill=white,draw=blue,forget plot]
table[row sep=crcr] {%
x	y\\
-0.148813410036973	0.221595807603391\\
-0.0124138303172889	0.224536404481042\\
-0.0248276606345778	0.449072808962083\\
}--cycle;


\addplot[area legend,solid,fill=white,draw=blue,forget plot]
table[row sep=crcr] {%
x	y\\
0.123985749402395	0.227477001358692\\
-0.0248276606345778	0.449072808962083\\
-0.0124138303172889	0.224536404481042\\
}--cycle;


\addplot[area legend,solid,fill=white,draw=blue,forget plot]
table[row sep=crcr] {%
x	y\\
0.123985749402395	0.227477001358692\\
-0.0124138303172889	0.224536404481042\\
0	0\\
}--cycle;


\addplot[area legend,solid,fill=white,draw=blue,forget plot]
table[row sep=crcr] {%
x	y\\
-0.398813410036973	0.221595807603391\\
-0.148813410036973	0.221595807603391\\
-0.223220115055459	0.332393711405087\\
}--cycle;


\addplot[area legend,solid,fill=white,draw=blue,forget plot]
table[row sep=crcr] {%
x	y\\
-0.398813410036973	0.221595807603391\\
-0.223220115055459	0.332393711405087\\
-0.297626820073946	0.443191615206783\\
}--cycle;


\addplot[area legend,solid,fill=white,draw=blue,forget plot]
table[row sep=crcr] {%
x	y\\
-0.161227240354262	0.446132212084433\\
-0.297626820073946	0.443191615206783\\
-0.223220115055459	0.332393711405087\\
}--cycle;


\addplot[area legend,solid,fill=white,draw=blue,forget plot]
table[row sep=crcr] {%
x	y\\
-0.161227240354262	0.446132212084433\\
-0.223220115055459	0.332393711405087\\
-0.148813410036973	0.221595807603391\\
}--cycle;


\addplot[area legend,solid,fill=white,draw=blue,forget plot]
table[row sep=crcr] {%
x	y\\
-0.398813410036973	0.221595807603391\\
-0.5	0\\
-0.324406705018486	0.110797903801696\\
}--cycle;


\addplot[area legend,solid,fill=white,draw=blue,forget plot]
table[row sep=crcr] {%
x	y\\
-0.398813410036973	0.221595807603391\\
-0.324406705018486	0.110797903801696\\
-0.148813410036973	0.221595807603391\\
}--cycle;


\addplot[area legend,solid,fill=white,draw=blue,forget plot]
table[row sep=crcr] {%
x	y\\
-0.25	0\\
-0.148813410036973	0.221595807603391\\
-0.324406705018486	0.110797903801696\\
}--cycle;


\addplot[area legend,solid,fill=white,draw=blue,forget plot]
table[row sep=crcr] {%
x	y\\
-0.25	0\\
-0.324406705018486	0.110797903801696\\
-0.5	0\\
}--cycle;

\addplot [color=red,solid,forget plot]
  table[row sep=crcr]{%
-0.75	0\\
-0.5	0\\
};
\addplot [color=red,solid,forget plot]
  table[row sep=crcr]{%
0.923879532511287	0.38268343236509\\
0.831469612302545	0.555570233019602\\
};
\addplot [color=red,solid,forget plot]
  table[row sep=crcr]{%
0.831469612302545	0.555570233019602\\
0.707106781186548	0.707106781186547\\
};
\addplot [color=red,solid,forget plot]
  table[row sep=crcr]{%
-0.923879532511287	0.38268343236509\\
-0.98078528040323	0.195090322016129\\
};
\addplot [color=red,solid,forget plot]
  table[row sep=crcr]{%
-0.98078528040323	0.195090322016129\\
-1	1.22464679914735e-16\\
};
\addplot [color=red,solid,forget plot]
  table[row sep=crcr]{%
-0.707106781186547	0.707106781186548\\
-0.831469612302545	0.555570233019602\\
};
\addplot [color=red,solid,forget plot]
  table[row sep=crcr]{%
-0.831469612302545	0.555570233019602\\
-0.923879532511287	0.38268343236509\\
};
\addplot [color=red,solid,forget plot]
  table[row sep=crcr]{%
-1	1.22464679914735e-16\\
-0.875	0\\
};
\addplot [color=red,solid,forget plot]
  table[row sep=crcr]{%
-0.875	0\\
-0.75	0\\
};
\addplot [color=red,solid,forget plot]
  table[row sep=crcr]{%
0.707106781186548	0.707106781186547\\
0.555570233019602	0.831469612302545\\
};
\addplot [color=red,solid,forget plot]
  table[row sep=crcr]{%
0.555570233019602	0.831469612302545\\
0.38268343236509	0.923879532511287\\
};
\addplot [color=red,solid,forget plot]
  table[row sep=crcr]{%
6.12323399573677e-17	1\\
-0.195090322016128	0.98078528040323\\
};
\addplot [color=red,solid,forget plot]
  table[row sep=crcr]{%
-0.195090322016128	0.98078528040323\\
-0.38268343236509	0.923879532511287\\
};
\addplot [color=red,solid,forget plot]
  table[row sep=crcr]{%
0.38268343236509	0.923879532511287\\
0.195090322016128	0.98078528040323\\
};
\addplot [color=red,solid,forget plot]
  table[row sep=crcr]{%
0.195090322016128	0.98078528040323\\
6.12323399573677e-17	1\\
};
\addplot [color=red,solid,forget plot]
  table[row sep=crcr]{%
1	0\\
0.98078528040323	0.195090322016128\\
};
\addplot [color=red,solid,forget plot]
  table[row sep=crcr]{%
0.98078528040323	0.195090322016128\\
0.923879532511287	0.38268343236509\\
};
\addplot [color=red,solid,forget plot]
  table[row sep=crcr]{%
0.5	0\\
0.75	0\\
};
\addplot [color=red,solid,forget plot]
  table[row sep=crcr]{%
0.75	0\\
1	0\\
};
\addplot [color=red,solid,forget plot]
  table[row sep=crcr]{%
-0.38268343236509	0.923879532511287\\
-0.555570233019602	0.831469612302545\\
};
\addplot [color=red,solid,forget plot]
  table[row sep=crcr]{%
-0.555570233019602	0.831469612302545\\
-0.707106781186547	0.707106781186548\\
};
\addplot [color=red,solid,forget plot]
  table[row sep=crcr]{%
0	0\\
0.25	0\\
};
\addplot [color=red,solid,forget plot]
  table[row sep=crcr]{%
0.25	0\\
0.5	0\\
};
\addplot [color=red,solid,forget plot]
  table[row sep=crcr]{%
-0.5	0\\
-0.25	0\\
};
\addplot [color=red,solid,forget plot]
  table[row sep=crcr]{%
-0.25	0\\
0	0\\
};
\end{axis}
\end{tikzpicture}%
	   \caption{\Code{semicircleg}, zweifache globale Verfeinerung}
        \end{subfigure}
        \begin{subfigure}[t]{0.32\textwidth}
	   % This file was created by matlab2tikz v0.5.0 running on MATLAB 8.3.
% Copyright (c) 2008--2014, Nico Schlömer <nico.schloemer@gmail.com>
% All rights reserved.
% Minimal pgfplots version: 1.3
% 
% The latest updates can be retrieved from
%   http://www.mathworks.com/matlabcentral/fileexchange/22022-matlab2tikz
% where you can also make suggestions and rate matlab2tikz.
% 
\begin{tikzpicture}

\begin{axis}[%
width=0.950920245398773\fwidth,
height=\fheight,
at={(0\fwidth,0\fheight)},
scale only axis,
xmin=-1,
xmax=1,
ymin=0,
ymax=1,
axis x line*=bottom,
axis y line*=left
]

\addplot[area legend,solid,fill=white,draw=blue,forget plot]
table[row sep=crcr] {%
x	y\\
-0.923879532511287	0.38268343236509\\
-1	1.22464679914735e-16\\
-0.685468610907825	0.306663883939896\\
}--cycle;


\addplot[area legend,solid,fill=white,draw=blue,forget plot]
table[row sep=crcr] {%
x	y\\
0.38268343236509	0.923879532511287\\
6.12323399573677e-17	1\\
-0.00927134816825834	0.745786196144667\\
}--cycle;


\addplot[area legend,solid,fill=white,draw=blue,forget plot]
table[row sep=crcr] {%
x	y\\
1	0\\
0.923879532511287	0.38268343236509\\
0.5	0\\
}--cycle;


\addplot[area legend,solid,fill=white,draw=blue,forget plot]
table[row sep=crcr] {%
x	y\\
0.5	0\\
0.923879532511287	0.38268343236509\\
0.595246578424459	0.38185823077614\\
}--cycle;


\addplot[area legend,solid,fill=white,draw=blue,forget plot]
table[row sep=crcr] {%
x	y\\
0.24797149880479	0.454954002717384\\
0.38268343236509	0.923879532511287\\
-0.00927134816825834	0.745786196144667\\
}--cycle;


\addplot[area legend,solid,fill=white,draw=blue,forget plot]
table[row sep=crcr] {%
x	y\\
6.12323399573677e-17	1\\
-0.38268343236509	0.923879532511287\\
-0.00927134816825834	0.745786196144667\\
}--cycle;


\addplot[area legend,solid,fill=white,draw=blue,forget plot]
table[row sep=crcr] {%
x	y\\
0.923879532511287	0.38268343236509\\
0.707106781186548	0.707106781186547\\
0.595246578424459	0.38185823077614\\
}--cycle;


\addplot[area legend,solid,fill=white,draw=blue,forget plot]
table[row sep=crcr] {%
x	y\\
0.707106781186548	0.707106781186547\\
0.38268343236509	0.923879532511287\\
0.24797149880479	0.454954002717384\\
}--cycle;


\addplot[area legend,solid,fill=white,draw=blue,forget plot]
table[row sep=crcr] {%
x	y\\
-0.707106781186547	0.707106781186548\\
-0.923879532511287	0.38268343236509\\
-0.685468610907825	0.306663883939896\\
}--cycle;


\addplot[area legend,solid,fill=white,draw=blue,forget plot]
table[row sep=crcr] {%
x	y\\
0.707106781186548	0.707106781186547\\
0.24797149880479	0.454954002717384\\
0.595246578424459	0.38185823077614\\
}--cycle;


\addplot[area legend,solid,fill=white,draw=blue,forget plot]
table[row sep=crcr] {%
x	y\\
-0.00927134816825834	0.745786196144667\\
-0.0248276606345778	0.449072808962083\\
0.24797149880479	0.454954002717384\\
}--cycle;


\addplot[area legend,solid,fill=white,draw=blue,forget plot]
table[row sep=crcr] {%
x	y\\
-0.685468610907825	0.306663883939896\\
-1	1.22464679914735e-16\\
-0.75	0\\
}--cycle;


\addplot[area legend,solid,fill=white,draw=blue,forget plot]
table[row sep=crcr] {%
x	y\\
-0.75	0\\
-0.5	0\\
-0.592734305453913	0.153331941969948\\
}--cycle;


\addplot[area legend,solid,fill=white,draw=blue,forget plot]
table[row sep=crcr] {%
x	y\\
-0.75	0\\
-0.592734305453913	0.153331941969948\\
-0.685468610907825	0.306663883939896\\
}--cycle;


\addplot[area legend,solid,fill=white,draw=blue,forget plot]
table[row sep=crcr] {%
x	y\\
-0.707106781186547	0.707106781186548\\
-0.340155126219518	0.683535573859035\\
-0.38268343236509	0.923879532511287\\
}--cycle;


\addplot[area legend,solid,fill=white,draw=blue,forget plot]
table[row sep=crcr] {%
x	y\\
-0.00927134816825834	0.745786196144667\\
-0.38268343236509	0.923879532511287\\
-0.340155126219518	0.683535573859035\\
}--cycle;


\addplot[area legend,solid,fill=white,draw=blue,forget plot]
table[row sep=crcr] {%
x	y\\
-0.685468610907825	0.306663883939896\\
-0.502366800630247	0.575149198196665\\
-0.707106781186547	0.707106781186548\\
}--cycle;


\addplot[area legend,solid,fill=white,draw=blue,forget plot]
table[row sep=crcr] {%
x	y\\
-0.340155126219518	0.683535573859035\\
-0.707106781186547	0.707106781186548\\
-0.502366800630247	0.575149198196665\\
}--cycle;


\addplot[area legend,solid,fill=white,draw=blue,forget plot]
table[row sep=crcr] {%
x	y\\
-0.340155126219518	0.683535573859035\\
-0.502366800630247	0.575149198196665\\
-0.297626820073946	0.443191615206783\\
}--cycle;


\addplot[area legend,solid,fill=white,draw=blue,forget plot]
table[row sep=crcr] {%
x	y\\
-0.502366800630247	0.575149198196665\\
-0.685468610907825	0.306663883939896\\
-0.491547715490885	0.374927749573339\\
}--cycle;


\addplot[area legend,solid,fill=white,draw=blue,forget plot]
table[row sep=crcr] {%
x	y\\
-0.502366800630247	0.575149198196665\\
-0.491547715490885	0.374927749573339\\
-0.297626820073946	0.443191615206783\\
}--cycle;


\addplot[area legend,solid,fill=white,draw=blue,forget plot]
table[row sep=crcr] {%
x	y\\
-0.0248276606345778	0.449072808962083\\
-0.00927134816825834	0.745786196144667\\
-0.153449084121102	0.594488905675725\\
}--cycle;


\addplot[area legend,solid,fill=white,draw=blue,forget plot]
table[row sep=crcr] {%
x	y\\
-0.340155126219518	0.683535573859035\\
-0.297626820073946	0.443191615206783\\
-0.153449084121102	0.594488905675725\\
}--cycle;


\addplot[area legend,solid,fill=white,draw=blue,forget plot]
table[row sep=crcr] {%
x	y\\
-0.340155126219518	0.683535573859035\\
-0.153449084121102	0.594488905675725\\
-0.00927134816825834	0.745786196144667\\
}--cycle;


\addplot[area legend,solid,fill=white,draw=blue,forget plot]
table[row sep=crcr] {%
x	y\\
-0.148813410036973	0.221595807603391\\
-0.0248276606345778	0.449072808962083\\
-0.161227240354262	0.446132212084433\\
}--cycle;


\addplot[area legend,solid,fill=white,draw=blue,forget plot]
table[row sep=crcr] {%
x	y\\
-0.153449084121102	0.594488905675725\\
-0.297626820073946	0.443191615206783\\
-0.161227240354262	0.446132212084433\\
}--cycle;


\addplot[area legend,solid,fill=white,draw=blue,forget plot]
table[row sep=crcr] {%
x	y\\
-0.153449084121102	0.594488905675725\\
-0.161227240354262	0.446132212084433\\
-0.0248276606345778	0.449072808962083\\
}--cycle;


\addplot[area legend,solid,fill=white,draw=blue,forget plot]
table[row sep=crcr] {%
x	y\\
-0.161227240354262	0.446132212084433\\
-0.297626820073946	0.443191615206783\\
-0.223220115055459	0.332393711405087\\
}--cycle;


\addplot[area legend,solid,fill=white,draw=blue,forget plot]
table[row sep=crcr] {%
x	y\\
-0.161227240354262	0.446132212084433\\
-0.223220115055459	0.332393711405087\\
-0.148813410036973	0.221595807603391\\
}--cycle;


\addplot[area legend,solid,fill=white,draw=blue,forget plot]
table[row sep=crcr] {%
x	y\\
-0.592734305453913	0.153331941969948\\
-0.542141010472399	0.264129845771643\\
-0.685468610907825	0.306663883939896\\
}--cycle;


\addplot[area legend,solid,fill=white,draw=blue,forget plot]
table[row sep=crcr] {%
x	y\\
-0.491547715490885	0.374927749573339\\
-0.685468610907825	0.306663883939896\\
-0.542141010472399	0.264129845771643\\
}--cycle;


\addplot[area legend,solid,fill=white,draw=blue,forget plot]
table[row sep=crcr] {%
x	y\\
-0.592734305453913	0.153331941969948\\
-0.5	0\\
-0.449406705018486	0.110797903801696\\
}--cycle;


\addplot[area legend,solid,fill=white,draw=blue,forget plot]
table[row sep=crcr] {%
x	y\\
-0.491547715490885	0.374927749573339\\
-0.348220115055459	0.332393711405087\\
-0.297626820073946	0.443191615206783\\
}--cycle;


\addplot[area legend,solid,fill=white,draw=blue,forget plot]
table[row sep=crcr] {%
x	y\\
-0.223220115055459	0.332393711405087\\
-0.297626820073946	0.443191615206783\\
-0.348220115055459	0.332393711405087\\
}--cycle;


\addplot[area legend,solid,fill=white,draw=blue,forget plot]
table[row sep=crcr] {%
x	y\\
-0.223220115055459	0.332393711405087\\
-0.273813410036973	0.221595807603391\\
-0.148813410036973	0.221595807603391\\
}--cycle;


\addplot[area legend,solid,fill=white,draw=blue,forget plot]
table[row sep=crcr] {%
x	y\\
-0.542141010472399	0.264129845771643\\
-0.592734305453913	0.153331941969948\\
-0.495773857745443	0.18746387478667\\
}--cycle;


\addplot[area legend,solid,fill=white,draw=blue,forget plot]
table[row sep=crcr] {%
x	y\\
-0.449406705018486	0.110797903801696\\
-0.495773857745443	0.18746387478667\\
-0.592734305453913	0.153331941969948\\
}--cycle;


\addplot[area legend,solid,fill=white,draw=blue,forget plot]
table[row sep=crcr] {%
x	y\\
-0.542141010472399	0.264129845771643\\
-0.445180562763929	0.298261778588365\\
-0.491547715490885	0.374927749573339\\
}--cycle;


\addplot[area legend,solid,fill=white,draw=blue,forget plot]
table[row sep=crcr] {%
x	y\\
-0.348220115055459	0.332393711405087\\
-0.491547715490885	0.374927749573339\\
-0.445180562763929	0.298261778588365\\
}--cycle;


\addplot[area legend,solid,fill=white,draw=blue,forget plot]
table[row sep=crcr] {%
x	y\\
-0.348220115055459	0.332393711405087\\
-0.311016762546216	0.276994759504239\\
-0.223220115055459	0.332393711405087\\
}--cycle;


\addplot[area legend,solid,fill=white,draw=blue,forget plot]
table[row sep=crcr] {%
x	y\\
-0.273813410036973	0.221595807603391\\
-0.223220115055459	0.332393711405087\\
-0.311016762546216	0.276994759504239\\
}--cycle;


\addplot[area legend,solid,fill=white,draw=blue,forget plot]
table[row sep=crcr] {%
x	y\\
-0.495773857745443	0.18746387478667\\
-0.470477210254686	0.242862826687517\\
-0.542141010472399	0.264129845771643\\
}--cycle;


\addplot[area legend,solid,fill=white,draw=blue,forget plot]
table[row sep=crcr] {%
x	y\\
-0.445180562763929	0.298261778588365\\
-0.542141010472399	0.264129845771643\\
-0.470477210254686	0.242862826687517\\
}--cycle;


\addplot[area legend,solid,fill=white,draw=blue,forget plot]
table[row sep=crcr] {%
x	y\\
-0.495773857745443	0.18746387478667\\
-0.449406705018486	0.110797903801696\\
-0.42411005752773	0.166196855702543\\
}--cycle;


\addplot[area legend,solid,fill=white,draw=blue,forget plot]
table[row sep=crcr] {%
x	y\\
-0.445180562763929	0.298261778588365\\
-0.373516762546216	0.276994759504239\\
-0.348220115055459	0.332393711405087\\
}--cycle;


\addplot[area legend,solid,fill=white,draw=blue,forget plot]
table[row sep=crcr] {%
x	y\\
-0.311016762546216	0.276994759504239\\
-0.348220115055459	0.332393711405087\\
-0.373516762546216	0.276994759504239\\
}--cycle;


\addplot[area legend,solid,fill=white,draw=blue,forget plot]
table[row sep=crcr] {%
x	y\\
-0.311016762546216	0.276994759504239\\
-0.336313410036973	0.221595807603391\\
-0.273813410036973	0.221595807603391\\
}--cycle;


\addplot[area legend,solid,fill=white,draw=blue,forget plot]
table[row sep=crcr] {%
x	y\\
-0.470477210254686	0.242862826687517\\
-0.495773857745443	0.18746387478667\\
-0.447293633891208	0.20452984119503\\
}--cycle;


\addplot[area legend,solid,fill=white,draw=blue,forget plot]
table[row sep=crcr] {%
x	y\\
-0.470477210254686	0.242862826687517\\
-0.447293633891208	0.20452984119503\\
-0.398813410036973	0.221595807603391\\
}--cycle;


\addplot[area legend,solid,fill=white,draw=blue,forget plot]
table[row sep=crcr] {%
x	y\\
-0.42411005752773	0.166196855702543\\
-0.447293633891208	0.20452984119503\\
-0.495773857745443	0.18746387478667\\
}--cycle;


\addplot[area legend,solid,fill=white,draw=blue,forget plot]
table[row sep=crcr] {%
x	y\\
-0.470477210254686	0.242862826687517\\
-0.398813410036973	0.221595807603391\\
-0.421996986400451	0.259928793095878\\
}--cycle;


\addplot[area legend,solid,fill=white,draw=blue,forget plot]
table[row sep=crcr] {%
x	y\\
-0.470477210254686	0.242862826687517\\
-0.421996986400451	0.259928793095878\\
-0.445180562763929	0.298261778588365\\
}--cycle;


\addplot[area legend,solid,fill=white,draw=blue,forget plot]
table[row sep=crcr] {%
x	y\\
-0.373516762546216	0.276994759504239\\
-0.445180562763929	0.298261778588365\\
-0.421996986400451	0.259928793095878\\
}--cycle;


\addplot[area legend,solid,fill=white,draw=blue,forget plot]
table[row sep=crcr] {%
x	y\\
-0.373516762546216	0.276994759504239\\
-0.421996986400451	0.259928793095878\\
-0.398813410036973	0.221595807603391\\
}--cycle;


\addplot[area legend,solid,fill=white,draw=blue,forget plot]
table[row sep=crcr] {%
x	y\\
-0.373516762546216	0.276994759504239\\
-0.398813410036973	0.221595807603391\\
-0.354915086291595	0.249295283553815\\
}--cycle;


\addplot[area legend,solid,fill=white,draw=blue,forget plot]
table[row sep=crcr] {%
x	y\\
-0.373516762546216	0.276994759504239\\
-0.354915086291595	0.249295283553815\\
-0.311016762546216	0.276994759504239\\
}--cycle;


\addplot[area legend,solid,fill=white,draw=blue,forget plot]
table[row sep=crcr] {%
x	y\\
-0.336313410036973	0.221595807603391\\
-0.311016762546216	0.276994759504239\\
-0.354915086291595	0.249295283553815\\
}--cycle;


\addplot[area legend,solid,fill=white,draw=blue,forget plot]
table[row sep=crcr] {%
x	y\\
-0.324406705018486	0.110797903801696\\
-0.375	0\\
-0.25	0\\
}--cycle;


\addplot[area legend,solid,fill=white,draw=blue,forget plot]
table[row sep=crcr] {%
x	y\\
-0.449406705018486	0.110797903801696\\
-0.5	0\\
-0.412203352509243	0.0553989519008478\\
}--cycle;


\addplot[area legend,solid,fill=white,draw=blue,forget plot]
table[row sep=crcr] {%
x	y\\
-0.375	0\\
-0.324406705018486	0.110797903801696\\
-0.412203352509243	0.0553989519008478\\
}--cycle;


\addplot[area legend,solid,fill=white,draw=blue,forget plot]
table[row sep=crcr] {%
x	y\\
-0.375	0\\
-0.412203352509243	0.0553989519008478\\
-0.5	0\\
}--cycle;


\addplot[area legend,solid,fill=white,draw=blue,forget plot]
table[row sep=crcr] {%
x	y\\
-0.36161005752773	0.166196855702543\\
-0.386906705018486	0.110797903801696\\
-0.324406705018486	0.110797903801696\\
}--cycle;


\addplot[area legend,solid,fill=white,draw=blue,forget plot]
table[row sep=crcr] {%
x	y\\
-0.412203352509243	0.0553989519008478\\
-0.324406705018486	0.110797903801696\\
-0.386906705018486	0.110797903801696\\
}--cycle;


\addplot[area legend,solid,fill=white,draw=blue,forget plot]
table[row sep=crcr] {%
x	y\\
-0.412203352509243	0.0553989519008478\\
-0.386906705018486	0.110797903801696\\
-0.449406705018486	0.110797903801696\\
}--cycle;


\addplot[area legend,solid,fill=white,draw=blue,forget plot]
table[row sep=crcr] {%
x	y\\
-0.42411005752773	0.166196855702543\\
-0.449406705018486	0.110797903801696\\
-0.405508381273108	0.13849737975212\\
}--cycle;


\addplot[area legend,solid,fill=white,draw=blue,forget plot]
table[row sep=crcr] {%
x	y\\
-0.386906705018486	0.110797903801696\\
-0.36161005752773	0.166196855702543\\
-0.405508381273108	0.13849737975212\\
}--cycle;


\addplot[area legend,solid,fill=white,draw=blue,forget plot]
table[row sep=crcr] {%
x	y\\
-0.386906705018486	0.110797903801696\\
-0.405508381273108	0.13849737975212\\
-0.449406705018486	0.110797903801696\\
}--cycle;


\addplot[area legend,solid,fill=white,draw=blue,forget plot]
table[row sep=crcr] {%
x	y\\
-0.380211733782351	0.193896331652967\\
-0.42411005752773	0.166196855702543\\
-0.39286005752773	0.166196855702543\\
}--cycle;


\addplot[area legend,solid,fill=white,draw=blue,forget plot]
table[row sep=crcr] {%
x	y\\
-0.380211733782351	0.193896331652967\\
-0.39286005752773	0.166196855702543\\
-0.36161005752773	0.166196855702543\\
}--cycle;


\addplot[area legend,solid,fill=white,draw=blue,forget plot]
table[row sep=crcr] {%
x	y\\
-0.405508381273108	0.13849737975212\\
-0.36161005752773	0.166196855702543\\
-0.39286005752773	0.166196855702543\\
}--cycle;


\addplot[area legend,solid,fill=white,draw=blue,forget plot]
table[row sep=crcr] {%
x	y\\
-0.405508381273108	0.13849737975212\\
-0.39286005752773	0.166196855702543\\
-0.42411005752773	0.166196855702543\\
}--cycle;


\addplot[area legend,solid,fill=white,draw=blue,forget plot]
table[row sep=crcr] {%
x	y\\
-0.380211733782351	0.193896331652967\\
-0.398813410036973	0.221595807603391\\
-0.411461733782351	0.193896331652967\\
}--cycle;


\addplot[area legend,solid,fill=white,draw=blue,forget plot]
table[row sep=crcr] {%
x	y\\
-0.380211733782351	0.193896331652967\\
-0.411461733782351	0.193896331652967\\
-0.42411005752773	0.166196855702543\\
}--cycle;


\addplot[area legend,solid,fill=white,draw=blue,forget plot]
table[row sep=crcr] {%
x	y\\
-0.447293633891208	0.20452984119503\\
-0.42411005752773	0.166196855702543\\
-0.411461733782351	0.193896331652967\\
}--cycle;


\addplot[area legend,solid,fill=white,draw=blue,forget plot]
table[row sep=crcr] {%
x	y\\
-0.447293633891208	0.20452984119503\\
-0.411461733782351	0.193896331652967\\
-0.398813410036973	0.221595807603391\\
}--cycle;


\addplot[area legend,solid,fill=white,draw=blue,forget plot]
table[row sep=crcr] {%
x	y\\
-0.380211733782351	0.193896331652967\\
-0.336313410036973	0.221595807603391\\
-0.367563410036973	0.221595807603391\\
}--cycle;


\addplot[area legend,solid,fill=white,draw=blue,forget plot]
table[row sep=crcr] {%
x	y\\
-0.380211733782351	0.193896331652967\\
-0.367563410036973	0.221595807603391\\
-0.398813410036973	0.221595807603391\\
}--cycle;


\addplot[area legend,solid,fill=white,draw=blue,forget plot]
table[row sep=crcr] {%
x	y\\
-0.354915086291595	0.249295283553815\\
-0.398813410036973	0.221595807603391\\
-0.367563410036973	0.221595807603391\\
}--cycle;


\addplot[area legend,solid,fill=white,draw=blue,forget plot]
table[row sep=crcr] {%
x	y\\
-0.354915086291595	0.249295283553815\\
-0.367563410036973	0.221595807603391\\
-0.336313410036973	0.221595807603391\\
}--cycle;


\addplot[area legend,solid,fill=white,draw=blue,forget plot]
table[row sep=crcr] {%
x	y\\
0	0\\
0.5	0\\
0.373985749402395	0.227477001358692\\
}--cycle;


\addplot[area legend,solid,fill=white,draw=blue,forget plot]
table[row sep=crcr] {%
x	y\\
0.595246578424459	0.38185823077614\\
0.24797149880479	0.454954002717384\\
0.373985749402395	0.227477001358692\\
}--cycle;


\addplot[area legend,solid,fill=white,draw=blue,forget plot]
table[row sep=crcr] {%
x	y\\
0.595246578424459	0.38185823077614\\
0.373985749402395	0.227477001358692\\
0.5	0\\
}--cycle;


\addplot[area legend,solid,fill=white,draw=blue,forget plot]
table[row sep=crcr] {%
x	y\\
-0.0248276606345778	0.449072808962083\\
0.123985749402395	0.227477001358692\\
0.24797149880479	0.454954002717384\\
}--cycle;


\addplot[area legend,solid,fill=white,draw=blue,forget plot]
table[row sep=crcr] {%
x	y\\
0.373985749402395	0.227477001358692\\
0.24797149880479	0.454954002717384\\
0.123985749402395	0.227477001358692\\
}--cycle;


\addplot[area legend,solid,fill=white,draw=blue,forget plot]
table[row sep=crcr] {%
x	y\\
0.373985749402395	0.227477001358692\\
0.123985749402395	0.227477001358692\\
0	0\\
}--cycle;


\addplot[area legend,solid,fill=white,draw=blue,forget plot]
table[row sep=crcr] {%
x	y\\
-0.148813410036973	0.221595807603391\\
-0.0124138303172889	0.224536404481042\\
-0.0248276606345778	0.449072808962083\\
}--cycle;


\addplot[area legend,solid,fill=white,draw=blue,forget plot]
table[row sep=crcr] {%
x	y\\
0.123985749402395	0.227477001358692\\
-0.0248276606345778	0.449072808962083\\
-0.0124138303172889	0.224536404481042\\
}--cycle;


\addplot[area legend,solid,fill=white,draw=blue,forget plot]
table[row sep=crcr] {%
x	y\\
0.123985749402395	0.227477001358692\\
-0.0124138303172889	0.224536404481042\\
0	0\\
}--cycle;


\addplot[area legend,solid,fill=white,draw=blue,forget plot]
table[row sep=crcr] {%
x	y\\
-0.25	0\\
0	0\\
-0.0744067050184865	0.110797903801696\\
}--cycle;


\addplot[area legend,solid,fill=white,draw=blue,forget plot]
table[row sep=crcr] {%
x	y\\
-0.0124138303172889	0.224536404481042\\
-0.148813410036973	0.221595807603391\\
-0.0744067050184865	0.110797903801696\\
}--cycle;


\addplot[area legend,solid,fill=white,draw=blue,forget plot]
table[row sep=crcr] {%
x	y\\
-0.0124138303172889	0.224536404481042\\
-0.0744067050184865	0.110797903801696\\
0	0\\
}--cycle;


\addplot[area legend,solid,fill=white,draw=blue,forget plot]
table[row sep=crcr] {%
x	y\\
-0.324406705018486	0.110797903801696\\
-0.25	0\\
-0.199406705018486	0.110797903801696\\
}--cycle;


\addplot[area legend,solid,fill=white,draw=blue,forget plot]
table[row sep=crcr] {%
x	y\\
-0.0744067050184865	0.110797903801696\\
-0.148813410036973	0.221595807603391\\
-0.199406705018486	0.110797903801696\\
}--cycle;


\addplot[area legend,solid,fill=white,draw=blue,forget plot]
table[row sep=crcr] {%
x	y\\
-0.0744067050184865	0.110797903801696\\
-0.199406705018486	0.110797903801696\\
-0.25	0\\
}--cycle;


\addplot[area legend,solid,fill=white,draw=blue,forget plot]
table[row sep=crcr] {%
x	y\\
-0.273813410036973	0.221595807603391\\
-0.23661005752773	0.166196855702543\\
-0.148813410036973	0.221595807603391\\
}--cycle;


\addplot[area legend,solid,fill=white,draw=blue,forget plot]
table[row sep=crcr] {%
x	y\\
-0.199406705018486	0.110797903801696\\
-0.148813410036973	0.221595807603391\\
-0.23661005752773	0.166196855702543\\
}--cycle;


\addplot[area legend,solid,fill=white,draw=blue,forget plot]
table[row sep=crcr] {%
x	y\\
-0.199406705018486	0.110797903801696\\
-0.23661005752773	0.166196855702543\\
-0.324406705018486	0.110797903801696\\
}--cycle;


\addplot[area legend,solid,fill=white,draw=blue,forget plot]
table[row sep=crcr] {%
x	y\\
-0.36161005752773	0.166196855702543\\
-0.324406705018486	0.110797903801696\\
-0.29911005752773	0.166196855702543\\
}--cycle;


\addplot[area legend,solid,fill=white,draw=blue,forget plot]
table[row sep=crcr] {%
x	y\\
-0.23661005752773	0.166196855702543\\
-0.273813410036973	0.221595807603391\\
-0.29911005752773	0.166196855702543\\
}--cycle;


\addplot[area legend,solid,fill=white,draw=blue,forget plot]
table[row sep=crcr] {%
x	y\\
-0.23661005752773	0.166196855702543\\
-0.29911005752773	0.166196855702543\\
-0.324406705018486	0.110797903801696\\
}--cycle;


\addplot[area legend,solid,fill=white,draw=blue,forget plot]
table[row sep=crcr] {%
x	y\\
-0.336313410036973	0.221595807603391\\
-0.317711733782351	0.193896331652967\\
-0.273813410036973	0.221595807603391\\
}--cycle;


\addplot[area legend,solid,fill=white,draw=blue,forget plot]
table[row sep=crcr] {%
x	y\\
-0.29911005752773	0.166196855702543\\
-0.273813410036973	0.221595807603391\\
-0.317711733782351	0.193896331652967\\
}--cycle;


\addplot[area legend,solid,fill=white,draw=blue,forget plot]
table[row sep=crcr] {%
x	y\\
-0.29911005752773	0.166196855702543\\
-0.317711733782351	0.193896331652967\\
-0.36161005752773	0.166196855702543\\
}--cycle;


\addplot[area legend,solid,fill=white,draw=blue,forget plot]
table[row sep=crcr] {%
x	y\\
-0.380211733782351	0.193896331652967\\
-0.36161005752773	0.166196855702543\\
-0.348961733782351	0.193896331652967\\
}--cycle;


\addplot[area legend,solid,fill=white,draw=blue,forget plot]
table[row sep=crcr] {%
x	y\\
-0.380211733782351	0.193896331652967\\
-0.348961733782351	0.193896331652967\\
-0.336313410036973	0.221595807603391\\
}--cycle;


\addplot[area legend,solid,fill=white,draw=blue,forget plot]
table[row sep=crcr] {%
x	y\\
-0.317711733782351	0.193896331652967\\
-0.336313410036973	0.221595807603391\\
-0.348961733782351	0.193896331652967\\
}--cycle;


\addplot[area legend,solid,fill=white,draw=blue,forget plot]
table[row sep=crcr] {%
x	y\\
-0.317711733782351	0.193896331652967\\
-0.348961733782351	0.193896331652967\\
-0.36161005752773	0.166196855702543\\
}--cycle;

\addplot [color=red,solid,forget plot]
  table[row sep=crcr]{%
1	0\\
0.923879532511287	0.38268343236509\\
};
\addplot [color=red,solid,forget plot]
  table[row sep=crcr]{%
0.923879532511287	0.38268343236509\\
0.707106781186548	0.707106781186547\\
};
\addplot [color=red,solid,forget plot]
  table[row sep=crcr]{%
0.707106781186548	0.707106781186547\\
0.38268343236509	0.923879532511287\\
};
\addplot [color=red,solid,forget plot]
  table[row sep=crcr]{%
0.38268343236509	0.923879532511287\\
6.12323399573677e-17	1\\
};
\addplot [color=red,solid,forget plot]
  table[row sep=crcr]{%
6.12323399573677e-17	1\\
-0.38268343236509	0.923879532511287\\
};
\addplot [color=red,solid,forget plot]
  table[row sep=crcr]{%
-0.38268343236509	0.923879532511287\\
-0.707106781186547	0.707106781186548\\
};
\addplot [color=red,solid,forget plot]
  table[row sep=crcr]{%
-0.707106781186547	0.707106781186548\\
-0.923879532511287	0.38268343236509\\
};
\addplot [color=red,solid,forget plot]
  table[row sep=crcr]{%
-0.923879532511287	0.38268343236509\\
-1	1.22464679914735e-16\\
};
\addplot [color=red,solid,forget plot]
  table[row sep=crcr]{%
0	0\\
0.5	0\\
};
\addplot [color=red,solid,forget plot]
  table[row sep=crcr]{%
0.5	0\\
1	0\\
};
\addplot [color=red,solid,forget plot]
  table[row sep=crcr]{%
-1	1.22464679914735e-16\\
-0.75	0\\
};
\addplot [color=red,solid,forget plot]
  table[row sep=crcr]{%
-0.75	0\\
-0.5	0\\
};
\addplot [color=red,solid,forget plot]
  table[row sep=crcr]{%
-0.25	0\\
0	0\\
};
\addplot [color=red,solid,forget plot]
  table[row sep=crcr]{%
-0.5	0\\
-0.375	0\\
};
\addplot [color=red,solid,forget plot]
  table[row sep=crcr]{%
-0.375	0\\
-0.25	0\\
};
\end{axis}
\end{tikzpicture}%
	   \caption{\Code{semicircleg}, 7-fache lokale Verfeinerung um $(-0.4,0.2)$}
        \end{subfigure}

        \setlength\fheight{0.28\textwidth}
        \begin{subfigure}[t]{0.32\textwidth}
	   % This file was created by matlab2tikz v0.5.0 running on MATLAB 8.3.
% Copyright (c) 2008--2014, Nico Schlömer <nico.schloemer@gmail.com>
% All rights reserved.
% Minimal pgfplots version: 1.3
% 
% The latest updates can be retrieved from
%   http://www.mathworks.com/matlabcentral/fileexchange/22022-matlab2tikz
% where you can also make suggestions and rate matlab2tikz.
% 
\begin{tikzpicture}

\begin{axis}[%
width=0.950920245398773\fwidth,
height=\fheight,
at={(0\fwidth,0\fheight)},
scale only axis,
xmin=-1,
xmax=1,
ymin=-1,
ymax=1,
axis x line*=bottom,
axis y line*=left
]

\addplot[area legend,solid,fill=white,draw=blue,forget plot]
table[row sep=crcr] {%
x	y\\
0.25	-0.433012701892219\\
0	0\\
-0.263837842483558	-0.433813525677282\\
}--cycle;


\addplot[area legend,solid,fill=white,draw=blue,forget plot]
table[row sep=crcr] {%
x	y\\
0.5	0\\
1	0\\
0.90630778703665	0.422618261740699\\
}--cycle;


\addplot[area legend,solid,fill=white,draw=blue,forget plot]
table[row sep=crcr] {%
x	y\\
-0.3502012155183	-0.0436984564427255\\
0	0\\
-0.275219716646685	0.242071513609225\\
}--cycle;


\addplot[area legend,solid,fill=white,draw=blue,forget plot]
table[row sep=crcr] {%
x	y\\
-0.0363406393987691	0.530207755117238\\
0	0\\
0.379029583261423	0.447312993400064\\
}--cycle;


\addplot[area legend,solid,fill=white,draw=blue,forget plot]
table[row sep=crcr] {%
x	y\\
0	0\\
0.5	0\\
0.379029583261423	0.447312993400064\\
}--cycle;


\addplot[area legend,solid,fill=white,draw=blue,forget plot]
table[row sep=crcr] {%
x	y\\
-0.263837842483558	-0.433813525677282\\
0	0\\
-0.3502012155183	-0.0436984564427255\\
}--cycle;


\addplot[area legend,solid,fill=white,draw=blue,forget plot]
table[row sep=crcr] {%
x	y\\
0.5	0\\
0.90630778703665	0.422618261740699\\
0.379029583261423	0.447312993400064\\
}--cycle;


\addplot[area legend,solid,fill=white,draw=blue,forget plot]
table[row sep=crcr] {%
x	y\\
-0.573576436351046	0.819152044288992\\
-0.866025403784439	0.5\\
-0.416690163543255	0.543178699587941\\
}--cycle;


\addplot[area legend,solid,fill=white,draw=blue,forget plot]
table[row sep=crcr] {%
x	y\\
0.90630778703665	0.422618261740699\\
0.642787609686539	0.766044443118978\\
0.379029583261423	0.447312993400064\\
}--cycle;


\addplot[area legend,solid,fill=white,draw=blue,forget plot]
table[row sep=crcr] {%
x	y\\
-0.939692620785908	-0.342020143325669\\
-0.707106781186548	-0.707106781186547\\
-0.638996414189727	-0.209517673592198\\
}--cycle;


\addplot[area legend,solid,fill=white,draw=blue,forget plot]
table[row sep=crcr] {%
x	y\\
0.258819045102521	0.965925826289068\\
-0.17364817766693	0.984807753012208\\
-0.0363406393987691	0.530207755117238\\
}--cycle;


\addplot[area legend,solid,fill=white,draw=blue,forget plot]
table[row sep=crcr] {%
x	y\\
-0.866025403784439	0.5\\
-0.996194698091746	0.0871557427476582\\
-0.587832333246482	0.185786213081415\\
}--cycle;


\addplot[area legend,solid,fill=white,draw=blue,forget plot]
table[row sep=crcr] {%
x	y\\
0.5	-0.866025403784439\\
0.25	-0.433012701892219\\
0.0871557427476579	-0.996194698091746\\
}--cycle;


\addplot[area legend,solid,fill=white,draw=blue,forget plot]
table[row sep=crcr] {%
x	y\\
0.0871557427476579	-0.996194698091746\\
0.25	-0.433012701892219\\
-0.0666913610308223	-0.700759246464352\\
}--cycle;


\addplot[area legend,solid,fill=white,draw=blue,forget plot]
table[row sep=crcr] {%
x	y\\
-0.707106781186548	-0.707106781186547\\
-0.342020143325669	-0.939692620785908\\
-0.263837842483558	-0.433813525677282\\
}--cycle;


\addplot[area legend,solid,fill=white,draw=blue,forget plot]
table[row sep=crcr] {%
x	y\\
-0.263837842483558	-0.433813525677282\\
-0.342020143325669	-0.939692620785908\\
-0.0666913610308223	-0.700759246464352\\
}--cycle;


\addplot[area legend,solid,fill=white,draw=blue,forget plot]
table[row sep=crcr] {%
x	y\\
-0.587832333246482	0.185786213081415\\
-0.996194698091746	0.0871557427476582\\
-0.638996414189727	-0.209517673592198\\
}--cycle;


\addplot[area legend,solid,fill=white,draw=blue,forget plot]
table[row sep=crcr] {%
x	y\\
-0.996194698091746	0.0871557427476582\\
-0.939692620785908	-0.342020143325669\\
-0.638996414189727	-0.209517673592198\\
}--cycle;


\addplot[area legend,solid,fill=white,draw=blue,forget plot]
table[row sep=crcr] {%
x	y\\
-0.0363406393987691	0.530207755117238\\
-0.17364817766693	0.984807753012208\\
-0.416690163543255	0.543178699587941\\
}--cycle;


\addplot[area legend,solid,fill=white,draw=blue,forget plot]
table[row sep=crcr] {%
x	y\\
-0.17364817766693	0.984807753012208\\
-0.573576436351046	0.819152044288992\\
-0.416690163543255	0.543178699587941\\
}--cycle;


\addplot[area legend,solid,fill=white,draw=blue,forget plot]
table[row sep=crcr] {%
x	y\\
0.642787609686539	0.766044443118978\\
0.258819045102521	0.965925826289068\\
0.379029583261423	0.447312993400064\\
}--cycle;


\addplot[area legend,solid,fill=white,draw=blue,forget plot]
table[row sep=crcr] {%
x	y\\
0.258819045102521	0.965925826289068\\
-0.0363406393987691	0.530207755117238\\
0.379029583261423	0.447312993400064\\
}--cycle;


\addplot[area legend,solid,fill=white,draw=blue,forget plot]
table[row sep=crcr] {%
x	y\\
-0.866025403784439	0.5\\
-0.587832333246482	0.185786213081415\\
-0.416690163543255	0.543178699587941\\
}--cycle;


\addplot[area legend,solid,fill=white,draw=blue,forget plot]
table[row sep=crcr] {%
x	y\\
-0.416690163543255	0.543178699587941\\
-0.587832333246482	0.185786213081415\\
-0.275219716646685	0.242071513609225\\
}--cycle;


\addplot[area legend,solid,fill=white,draw=blue,forget plot]
table[row sep=crcr] {%
x	y\\
-0.707106781186548	-0.707106781186547\\
-0.263837842483558	-0.433813525677282\\
-0.638996414189727	-0.209517673592198\\
}--cycle;


\addplot[area legend,solid,fill=white,draw=blue,forget plot]
table[row sep=crcr] {%
x	y\\
-0.638996414189727	-0.209517673592198\\
-0.263837842483558	-0.433813525677282\\
-0.3502012155183	-0.0436984564427255\\
}--cycle;


\addplot[area legend,solid,fill=white,draw=blue,forget plot]
table[row sep=crcr] {%
x	y\\
-0.342020143325669	-0.939692620785908\\
0.0871557427476579	-0.996194698091746\\
-0.0666913610308223	-0.700759246464352\\
}--cycle;


\addplot[area legend,solid,fill=white,draw=blue,forget plot]
table[row sep=crcr] {%
x	y\\
0.25	-0.433012701892219\\
-0.263837842483558	-0.433813525677282\\
-0.0666913610308223	-0.700759246464352\\
}--cycle;


\addplot[area legend,solid,fill=white,draw=blue,forget plot]
table[row sep=crcr] {%
x	y\\
0	0\\
-0.0363406393987691	0.530207755117238\\
-0.275219716646685	0.242071513609225\\
}--cycle;


\addplot[area legend,solid,fill=white,draw=blue,forget plot]
table[row sep=crcr] {%
x	y\\
-0.587832333246482	0.185786213081415\\
-0.638996414189727	-0.209517673592198\\
-0.3502012155183	-0.0436984564427255\\
}--cycle;


\addplot[area legend,solid,fill=white,draw=blue,forget plot]
table[row sep=crcr] {%
x	y\\
-0.0363406393987691	0.530207755117238\\
-0.416690163543255	0.543178699587941\\
-0.275219716646685	0.242071513609225\\
}--cycle;


\addplot[area legend,solid,fill=white,draw=blue,forget plot]
table[row sep=crcr] {%
x	y\\
-0.587832333246482	0.185786213081415\\
-0.3502012155183	-0.0436984564427255\\
-0.275219716646685	0.242071513609225\\
}--cycle;

\addplot [color=red,solid,forget plot]
  table[row sep=crcr]{%
0	0\\
0.5	0\\
};
\addplot [color=red,solid,forget plot]
  table[row sep=crcr]{%
0.5	0\\
1	0\\
};
\addplot [color=red,solid,forget plot]
  table[row sep=crcr]{%
1	0\\
0.90630778703665	0.422618261740699\\
};
\addplot [color=red,solid,forget plot]
  table[row sep=crcr]{%
0.90630778703665	0.422618261740699\\
0.642787609686539	0.766044443118978\\
};
\addplot [color=red,solid,forget plot]
  table[row sep=crcr]{%
0.642787609686539	0.766044443118978\\
0.258819045102521	0.965925826289068\\
};
\addplot [color=red,solid,forget plot]
  table[row sep=crcr]{%
0.258819045102521	0.965925826289068\\
-0.17364817766693	0.984807753012208\\
};
\addplot [color=red,solid,forget plot]
  table[row sep=crcr]{%
-0.17364817766693	0.984807753012208\\
-0.573576436351046	0.819152044288992\\
};
\addplot [color=red,solid,forget plot]
  table[row sep=crcr]{%
-0.573576436351046	0.819152044288992\\
-0.866025403784439	0.5\\
};
\addplot [color=red,solid,forget plot]
  table[row sep=crcr]{%
-0.866025403784439	0.5\\
-0.996194698091746	0.0871557427476582\\
};
\addplot [color=red,solid,forget plot]
  table[row sep=crcr]{%
-0.996194698091746	0.0871557427476582\\
-0.939692620785908	-0.342020143325669\\
};
\addplot [color=red,solid,forget plot]
  table[row sep=crcr]{%
-0.939692620785908	-0.342020143325669\\
-0.707106781186548	-0.707106781186547\\
};
\addplot [color=red,solid,forget plot]
  table[row sep=crcr]{%
0.5	-0.866025403784439\\
0.25	-0.433012701892219\\
};
\addplot [color=red,solid,forget plot]
  table[row sep=crcr]{%
0.25	-0.433012701892219\\
0	0\\
};
\addplot [color=red,solid,forget plot]
  table[row sep=crcr]{%
-0.707106781186548	-0.707106781186547\\
-0.342020143325669	-0.939692620785908\\
};
\addplot [color=red,solid,forget plot]
  table[row sep=crcr]{%
-0.342020143325669	-0.939692620785908\\
0.0871557427476579	-0.996194698091746\\
};
\addplot [color=red,solid,forget plot]
  table[row sep=crcr]{%
0.0871557427476579	-0.996194698091746\\
0.5	-0.866025403784439\\
};
\end{axis}
\end{tikzpicture}%
	   \caption{\Code{sectorg}, Initialtriangulierung}
        \end{subfigure}
        \begin{subfigure}[t]{0.32\textwidth}
	   % This file was created by matlab2tikz v0.5.0 running on MATLAB 8.3.
% Copyright (c) 2008--2014, Nico Schlömer <nico.schloemer@gmail.com>
% All rights reserved.
% Minimal pgfplots version: 1.3
% 
% The latest updates can be retrieved from
%   http://www.mathworks.com/matlabcentral/fileexchange/22022-matlab2tikz
% where you can also make suggestions and rate matlab2tikz.
% 
\begin{tikzpicture}

\begin{axis}[%
width=0.950920245398773\fwidth,
height=\fheight,
at={(0\fwidth,0\fheight)},
scale only axis,
xmin=-1,
xmax=1,
ymin=-1,
ymax=1,
axis x line*=bottom,
axis y line*=left
]

\addplot[area legend,solid,fill=white,draw=blue,forget plot]
table[row sep=crcr] {%
x	y\\
-0.00691892124177887	-0.433413113784751\\
0.25	-0.433012701892219\\
0.125	-0.21650635094611\\
}--cycle;


\addplot[area legend,solid,fill=white,draw=blue,forget plot]
table[row sep=crcr] {%
x	y\\
-0.00691892124177887	-0.433413113784751\\
-0.0666913610308223	-0.700759246464352\\
0.0916543194845889	-0.566885974178286\\
}--cycle;


\addplot[area legend,solid,fill=white,draw=blue,forget plot]
table[row sep=crcr] {%
x	y\\
-0.00691892124177887	-0.433413113784751\\
0.0916543194845889	-0.566885974178286\\
0.25	-0.433012701892219\\
}--cycle;


\addplot[area legend,solid,fill=white,draw=blue,forget plot]
table[row sep=crcr] {%
x	y\\
0.168577871373829	-0.714603699991982\\
0.25	-0.433012701892219\\
0.0916543194845889	-0.566885974178286\\
}--cycle;


\addplot[area legend,solid,fill=white,draw=blue,forget plot]
table[row sep=crcr] {%
x	y\\
0.168577871373829	-0.714603699991982\\
0.0916543194845889	-0.566885974178286\\
-0.0666913610308223	-0.700759246464352\\
}--cycle;


\addplot[area legend,solid,fill=white,draw=blue,forget plot]
table[row sep=crcr] {%
x	y\\
-0.00691892124177887	-0.433413113784751\\
-0.16526460175719	-0.567286386070817\\
-0.0666913610308223	-0.700759246464352\\
}--cycle;


\addplot[area legend,solid,fill=white,draw=blue,forget plot]
table[row sep=crcr] {%
x	y\\
-0.302928992904613	-0.686753073231595\\
-0.0666913610308223	-0.700759246464352\\
-0.16526460175719	-0.567286386070817\\
}--cycle;


\addplot[area legend,solid,fill=white,draw=blue,forget plot]
table[row sep=crcr] {%
x	y\\
-0.302928992904613	-0.686753073231595\\
-0.16526460175719	-0.567286386070817\\
-0.263837842483558	-0.433813525677282\\
}--cycle;


\addplot[area legend,solid,fill=white,draw=blue,forget plot]
table[row sep=crcr] {%
x	y\\
0.703153893518325	0.21130913087035\\
1	0\\
0.976296007119933	0.216439613938103\\
}--cycle;


\addplot[area legend,solid,fill=white,draw=blue,forget plot]
table[row sep=crcr] {%
x	y\\
0.703153893518325	0.21130913087035\\
0.5	0\\
0.75	0\\
}--cycle;


\addplot[area legend,solid,fill=white,draw=blue,forget plot]
table[row sep=crcr] {%
x	y\\
0.703153893518325	0.21130913087035\\
0.75	0\\
1	0\\
}--cycle;


\addplot[area legend,solid,fill=white,draw=blue,forget plot]
table[row sep=crcr] {%
x	y\\
0.703153893518325	0.21130913087035\\
0.439514791630711	0.223656496700032\\
0.5	0\\
}--cycle;


\addplot[area legend,solid,fill=white,draw=blue,forget plot]
table[row sep=crcr] {%
x	y\\
0.189514791630711	0.223656496700032\\
0.5	0\\
0.439514791630711	0.223656496700032\\
}--cycle;


\addplot[area legend,solid,fill=white,draw=blue,forget plot]
table[row sep=crcr] {%
x	y\\
0.189514791630711	0.223656496700032\\
0.439514791630711	0.223656496700032\\
0.379029583261423	0.447312993400064\\
}--cycle;


\addplot[area legend,solid,fill=white,draw=blue,forget plot]
table[row sep=crcr] {%
x	y\\
0.189514791630711	0.223656496700032\\
0.379029583261423	0.447312993400064\\
0.171344471931327	0.488760374258651\\
}--cycle;


\addplot[area legend,solid,fill=white,draw=blue,forget plot]
table[row sep=crcr] {%
x	y\\
0.318924314181972	0.706619409844566\\
0.171344471931327	0.488760374258651\\
0.379029583261423	0.447312993400064\\
}--cycle;


\addplot[area legend,solid,fill=white,draw=blue,forget plot]
table[row sep=crcr] {%
x	y\\
0.189514791630711	0.223656496700032\\
0.25	0\\
0.5	0\\
}--cycle;


\addplot[area legend,solid,fill=white,draw=blue,forget plot]
table[row sep=crcr] {%
x	y\\
-0.0181703196993845	0.265103877558619\\
-0.0363406393987691	0.530207755117238\\
-0.155780178022727	0.386139634363231\\
}--cycle;


\addplot[area legend,solid,fill=white,draw=blue,forget plot]
table[row sep=crcr] {%
x	y\\
-0.226515401471012	0.53669322735259\\
-0.275219716646685	0.242071513609225\\
-0.155780178022727	0.386139634363231\\
}--cycle;


\addplot[area legend,solid,fill=white,draw=blue,forget plot]
table[row sep=crcr] {%
x	y\\
-0.226515401471012	0.53669322735259\\
-0.155780178022727	0.386139634363231\\
-0.0363406393987691	0.530207755117238\\
}--cycle;


\addplot[area legend,solid,fill=white,draw=blue,forget plot]
table[row sep=crcr] {%
x	y\\
-0.0181703196993845	0.265103877558619\\
0.189514791630711	0.223656496700032\\
0.0765870761159712	0.376932125908635\\
}--cycle;


\addplot[area legend,solid,fill=white,draw=blue,forget plot]
table[row sep=crcr] {%
x	y\\
-0.0181703196993845	0.265103877558619\\
0.0765870761159712	0.376932125908635\\
-0.0363406393987691	0.530207755117238\\
}--cycle;


\addplot[area legend,solid,fill=white,draw=blue,forget plot]
table[row sep=crcr] {%
x	y\\
0.171344471931327	0.488760374258651\\
-0.0363406393987691	0.530207755117238\\
0.0765870761159712	0.376932125908635\\
}--cycle;


\addplot[area legend,solid,fill=white,draw=blue,forget plot]
table[row sep=crcr] {%
x	y\\
0.171344471931327	0.488760374258651\\
0.0765870761159712	0.376932125908635\\
0.189514791630711	0.223656496700032\\
}--cycle;


\addplot[area legend,solid,fill=white,draw=blue,forget plot]
table[row sep=crcr] {%
x	y\\
-0.0181703196993845	0.265103877558619\\
0.0947573958153557	0.111828248350016\\
0.189514791630711	0.223656496700032\\
}--cycle;


\addplot[area legend,solid,fill=white,draw=blue,forget plot]
table[row sep=crcr] {%
x	y\\
0.25	0\\
0.189514791630711	0.223656496700032\\
0.0947573958153557	0.111828248350016\\
}--cycle;


\addplot[area legend,solid,fill=white,draw=blue,forget plot]
table[row sep=crcr] {%
x	y\\
0.25	0\\
0.0947573958153557	0.111828248350016\\
0	0\\
}--cycle;


\addplot[area legend,solid,fill=white,draw=blue,forget plot]
table[row sep=crcr] {%
x	y\\
-0.137609858323343	0.121035756804613\\
-0.3502012155183	-0.0436984564427255\\
-0.17510060775915	-0.0218492282213627\\
}--cycle;


\addplot[area legend,solid,fill=white,draw=blue,forget plot]
table[row sep=crcr] {%
x	y\\
-0.137609858323343	0.121035756804613\\
-0.17510060775915	-0.0218492282213627\\
0	0\\
}--cycle;


\addplot[area legend,solid,fill=white,draw=blue,forget plot]
table[row sep=crcr] {%
x	y\\
-0.131918921241779	-0.216906762838641\\
0	0\\
-0.17510060775915	-0.0218492282213627\\
}--cycle;


\addplot[area legend,solid,fill=white,draw=blue,forget plot]
table[row sep=crcr] {%
x	y\\
-0.131918921241779	-0.216906762838641\\
-0.17510060775915	-0.0218492282213627\\
-0.3502012155183	-0.0436984564427255\\
}--cycle;


\addplot[area legend,solid,fill=white,draw=blue,forget plot]
table[row sep=crcr] {%
x	y\\
-0.137609858323343	0.121035756804613\\
-0.275219716646685	0.242071513609225\\
-0.312710466082493	0.0991865285832498\\
}--cycle;


\addplot[area legend,solid,fill=white,draw=blue,forget plot]
table[row sep=crcr] {%
x	y\\
-0.137609858323343	0.121035756804613\\
-0.312710466082493	0.0991865285832498\\
-0.3502012155183	-0.0436984564427255\\
}--cycle;


\addplot[area legend,solid,fill=white,draw=blue,forget plot]
table[row sep=crcr] {%
x	y\\
-0.469016774382391	0.0710438783193448\\
-0.3502012155183	-0.0436984564427255\\
-0.312710466082493	0.0991865285832498\\
}--cycle;


\addplot[area legend,solid,fill=white,draw=blue,forget plot]
table[row sep=crcr] {%
x	y\\
-0.469016774382391	0.0710438783193448\\
-0.312710466082493	0.0991865285832498\\
-0.275219716646685	0.242071513609225\\
}--cycle;


\addplot[area legend,solid,fill=white,draw=blue,forget plot]
table[row sep=crcr] {%
x	y\\
-0.137609858323343	0.121035756804613\\
-0.0181703196993845	0.265103877558619\\
-0.146695018173035	0.253587695583922\\
}--cycle;


\addplot[area legend,solid,fill=white,draw=blue,forget plot]
table[row sep=crcr] {%
x	y\\
-0.137609858323343	0.121035756804613\\
-0.146695018173035	0.253587695583922\\
-0.275219716646685	0.242071513609225\\
}--cycle;


\addplot[area legend,solid,fill=white,draw=blue,forget plot]
table[row sep=crcr] {%
x	y\\
-0.155780178022727	0.386139634363231\\
-0.275219716646685	0.242071513609225\\
-0.146695018173035	0.253587695583922\\
}--cycle;


\addplot[area legend,solid,fill=white,draw=blue,forget plot]
table[row sep=crcr] {%
x	y\\
-0.155780178022727	0.386139634363231\\
-0.146695018173035	0.253587695583922\\
-0.0181703196993845	0.265103877558619\\
}--cycle;


\addplot[area legend,solid,fill=white,draw=blue,forget plot]
table[row sep=crcr] {%
x	y\\
-0.137609858323343	0.121035756804613\\
0	0\\
-0.00908515984969226	0.132551938779309\\
}--cycle;


\addplot[area legend,solid,fill=white,draw=blue,forget plot]
table[row sep=crcr] {%
x	y\\
-0.137609858323343	0.121035756804613\\
-0.00908515984969226	0.132551938779309\\
-0.0181703196993845	0.265103877558619\\
}--cycle;


\addplot[area legend,solid,fill=white,draw=blue,forget plot]
table[row sep=crcr] {%
x	y\\
0.0947573958153557	0.111828248350016\\
-0.0181703196993845	0.265103877558619\\
-0.00908515984969226	0.132551938779309\\
}--cycle;


\addplot[area legend,solid,fill=white,draw=blue,forget plot]
table[row sep=crcr] {%
x	y\\
0.0947573958153557	0.111828248350016\\
-0.00908515984969226	0.132551938779309\\
0	0\\
}--cycle;


\addplot[area legend,solid,fill=white,draw=blue,forget plot]
table[row sep=crcr] {%
x	y\\
-0.131918921241779	-0.216906762838641\\
-0.3502012155183	-0.0436984564427255\\
-0.307019529000929	-0.238755991060004\\
}--cycle;


\addplot[area legend,solid,fill=white,draw=blue,forget plot]
table[row sep=crcr] {%
x	y\\
-0.131918921241779	-0.216906762838641\\
-0.307019529000929	-0.238755991060004\\
-0.263837842483558	-0.433813525677282\\
}--cycle;


\addplot[area legend,solid,fill=white,draw=blue,forget plot]
table[row sep=crcr] {%
x	y\\
-0.451417128336643	-0.32166559963474\\
-0.263837842483558	-0.433813525677282\\
-0.307019529000929	-0.238755991060004\\
}--cycle;


\addplot[area legend,solid,fill=white,draw=blue,forget plot]
table[row sep=crcr] {%
x	y\\
-0.451417128336643	-0.32166559963474\\
-0.307019529000929	-0.238755991060004\\
-0.3502012155183	-0.0436984564427255\\
}--cycle;


\addplot[area legend,solid,fill=white,draw=blue,forget plot]
table[row sep=crcr] {%
x	y\\
-0.131918921241779	-0.216906762838641\\
-0.00691892124177887	-0.433413113784751\\
-0.00345946062088943	-0.216706556892375\\
}--cycle;


\addplot[area legend,solid,fill=white,draw=blue,forget plot]
table[row sep=crcr] {%
x	y\\
-0.131918921241779	-0.216906762838641\\
-0.00345946062088943	-0.216706556892375\\
0	0\\
}--cycle;


\addplot[area legend,solid,fill=white,draw=blue,forget plot]
table[row sep=crcr] {%
x	y\\
0.125	-0.21650635094611\\
0	0\\
-0.00345946062088943	-0.216706556892375\\
}--cycle;


\addplot[area legend,solid,fill=white,draw=blue,forget plot]
table[row sep=crcr] {%
x	y\\
0.125	-0.21650635094611\\
-0.00345946062088943	-0.216706556892375\\
-0.00691892124177887	-0.433413113784751\\
}--cycle;


\addplot[area legend,solid,fill=white,draw=blue,forget plot]
table[row sep=crcr] {%
x	y\\
-0.131918921241779	-0.216906762838641\\
-0.263837842483558	-0.433813525677282\\
-0.135378381862668	-0.433613319731016\\
}--cycle;


\addplot[area legend,solid,fill=white,draw=blue,forget plot]
table[row sep=crcr] {%
x	y\\
-0.131918921241779	-0.216906762838641\\
-0.135378381862668	-0.433613319731016\\
-0.00691892124177887	-0.433413113784751\\
}--cycle;


\addplot[area legend,solid,fill=white,draw=blue,forget plot]
table[row sep=crcr] {%
x	y\\
-0.16526460175719	-0.567286386070817\\
-0.00691892124177887	-0.433413113784751\\
-0.135378381862668	-0.433613319731016\\
}--cycle;


\addplot[area legend,solid,fill=white,draw=blue,forget plot]
table[row sep=crcr] {%
x	y\\
-0.16526460175719	-0.567286386070817\\
-0.135378381862668	-0.433613319731016\\
-0.263837842483558	-0.433813525677282\\
}--cycle;


\addplot[area legend,solid,fill=white,draw=blue,forget plot]
table[row sep=crcr] {%
x	y\\
-0.641357783663847	0.521589349793971\\
-0.573576436351046	0.819152044288992\\
-0.737277336810124	0.67559020761566\\
}--cycle;


\addplot[area legend,solid,fill=white,draw=blue,forget plot]
table[row sep=crcr] {%
x	y\\
-0.641357783663847	0.521589349793971\\
-0.737277336810124	0.67559020761566\\
-0.866025403784439	0.5\\
}--cycle;


\addplot[area legend,solid,fill=white,draw=blue,forget plot]
table[row sep=crcr] {%
x	y\\
-0.641357783663847	0.521589349793971\\
-0.495133299947151	0.681165371938466\\
-0.573576436351046	0.819152044288992\\
}--cycle;


\addplot[area legend,solid,fill=white,draw=blue,forget plot]
table[row sep=crcr] {%
x	y\\
-0.295169170605093	0.763993226300075\\
-0.573576436351046	0.819152044288992\\
-0.495133299947151	0.681165371938466\\
}--cycle;


\addplot[area legend,solid,fill=white,draw=blue,forget plot]
table[row sep=crcr] {%
x	y\\
-0.641357783663847	0.521589349793971\\
-0.866025403784439	0.5\\
-0.72692886851546	0.342893106540707\\
}--cycle;


\addplot[area legend,solid,fill=white,draw=blue,forget plot]
table[row sep=crcr] {%
x	y\\
-0.953716950748227	0.300705799504273\\
-0.587832333246482	0.185786213081415\\
-0.72692886851546	0.342893106540707\\
}--cycle;


\addplot[area legend,solid,fill=white,draw=blue,forget plot]
table[row sep=crcr] {%
x	y\\
-0.953716950748227	0.300705799504273\\
-0.72692886851546	0.342893106540707\\
-0.866025403784439	0.5\\
}--cycle;


\addplot[area legend,solid,fill=white,draw=blue,forget plot]
table[row sep=crcr] {%
x	y\\
0.642668685149036	0.434965627570382\\
0.642787609686539	0.766044443118978\\
0.510908596473981	0.606678718259521\\
}--cycle;


\addplot[area legend,solid,fill=white,draw=blue,forget plot]
table[row sep=crcr] {%
x	y\\
0.642668685149036	0.434965627570382\\
0.510908596473981	0.606678718259521\\
0.379029583261423	0.447312993400064\\
}--cycle;


\addplot[area legend,solid,fill=white,draw=blue,forget plot]
table[row sep=crcr] {%
x	y\\
0.318924314181972	0.706619409844566\\
0.379029583261423	0.447312993400064\\
0.510908596473981	0.606678718259521\\
}--cycle;


\addplot[area legend,solid,fill=white,draw=blue,forget plot]
table[row sep=crcr] {%
x	y\\
0.318924314181972	0.706619409844566\\
0.510908596473981	0.606678718259521\\
0.642787609686539	0.766044443118978\\
}--cycle;


\addplot[area legend,solid,fill=white,draw=blue,forget plot]
table[row sep=crcr] {%
x	y\\
0.642668685149036	0.434965627570382\\
0.90630778703665	0.422618261740699\\
0.793353340291235	0.608761429008721\\
}--cycle;


\addplot[area legend,solid,fill=white,draw=blue,forget plot]
table[row sep=crcr] {%
x	y\\
0.642668685149036	0.434965627570382\\
0.793353340291235	0.608761429008721\\
0.642787609686539	0.766044443118978\\
}--cycle;


\addplot[area legend,solid,fill=white,draw=blue,forget plot]
table[row sep=crcr] {%
x	y\\
0.642668685149036	0.434965627570382\\
0.703153893518325	0.21130913087035\\
0.804730840277488	0.316963696305525\\
}--cycle;


\addplot[area legend,solid,fill=white,draw=blue,forget plot]
table[row sep=crcr] {%
x	y\\
0.642668685149036	0.434965627570382\\
0.804730840277488	0.316963696305525\\
0.90630778703665	0.422618261740699\\
}--cycle;


\addplot[area legend,solid,fill=white,draw=blue,forget plot]
table[row sep=crcr] {%
x	y\\
0.976296007119933	0.216439613938103\\
0.90630778703665	0.422618261740699\\
0.804730840277488	0.316963696305525\\
}--cycle;


\addplot[area legend,solid,fill=white,draw=blue,forget plot]
table[row sep=crcr] {%
x	y\\
0.976296007119933	0.216439613938103\\
0.804730840277488	0.316963696305525\\
0.703153893518325	0.21130913087035\\
}--cycle;


\addplot[area legend,solid,fill=white,draw=blue,forget plot]
table[row sep=crcr] {%
x	y\\
0.642668685149036	0.434965627570382\\
0.379029583261423	0.447312993400064\\
0.541091738389874	0.329311062135207\\
}--cycle;


\addplot[area legend,solid,fill=white,draw=blue,forget plot]
table[row sep=crcr] {%
x	y\\
0.642668685149036	0.434965627570382\\
0.541091738389874	0.329311062135207\\
0.703153893518325	0.21130913087035\\
}--cycle;


\addplot[area legend,solid,fill=white,draw=blue,forget plot]
table[row sep=crcr] {%
x	y\\
0.439514791630711	0.223656496700032\\
0.703153893518325	0.21130913087035\\
0.541091738389874	0.329311062135207\\
}--cycle;


\addplot[area legend,solid,fill=white,draw=blue,forget plot]
table[row sep=crcr] {%
x	y\\
0.439514791630711	0.223656496700032\\
0.541091738389874	0.329311062135207\\
0.379029583261423	0.447312993400064\\
}--cycle;


\addplot[area legend,solid,fill=white,draw=blue,forget plot]
table[row sep=crcr] {%
x	y\\
-0.673051597688137	-0.458312227389373\\
-0.939692620785908	-0.342020143325669\\
-0.843391445812886	-0.537299608346824\\
}--cycle;


\addplot[area legend,solid,fill=white,draw=blue,forget plot]
table[row sep=crcr] {%
x	y\\
-0.673051597688137	-0.458312227389373\\
-0.843391445812886	-0.537299608346824\\
-0.707106781186548	-0.707106781186547\\
}--cycle;


\addplot[area legend,solid,fill=white,draw=blue,forget plot]
table[row sep=crcr] {%
x	y\\
-0.673051597688137	-0.458312227389373\\
-0.638996414189727	-0.209517673592198\\
-0.789344517487818	-0.275768908458933\\
}--cycle;


\addplot[area legend,solid,fill=white,draw=blue,forget plot]
table[row sep=crcr] {%
x	y\\
-0.673051597688137	-0.458312227389373\\
-0.789344517487818	-0.275768908458933\\
-0.939692620785908	-0.342020143325669\\
}--cycle;


\addplot[area legend,solid,fill=white,draw=blue,forget plot]
table[row sep=crcr] {%
x	y\\
-0.817595556140736	-0.0611809654222698\\
-0.939692620785908	-0.342020143325669\\
-0.789344517487818	-0.275768908458933\\
}--cycle;


\addplot[area legend,solid,fill=white,draw=blue,forget plot]
table[row sep=crcr] {%
x	y\\
-0.673051597688137	-0.458312227389373\\
-0.485472311835053	-0.570460153431915\\
-0.56223436301239	-0.389988913512056\\
}--cycle;


\addplot[area legend,solid,fill=white,draw=blue,forget plot]
table[row sep=crcr] {%
x	y\\
-0.673051597688137	-0.458312227389373\\
-0.56223436301239	-0.389988913512056\\
-0.638996414189727	-0.209517673592198\\
}--cycle;


\addplot[area legend,solid,fill=white,draw=blue,forget plot]
table[row sep=crcr] {%
x	y\\
-0.451417128336643	-0.32166559963474\\
-0.638996414189727	-0.209517673592198\\
-0.56223436301239	-0.389988913512056\\
}--cycle;


\addplot[area legend,solid,fill=white,draw=blue,forget plot]
table[row sep=crcr] {%
x	y\\
-0.451417128336643	-0.32166559963474\\
-0.56223436301239	-0.389988913512056\\
-0.485472311835053	-0.570460153431915\\
}--cycle;


\addplot[area legend,solid,fill=white,draw=blue,forget plot]
table[row sep=crcr] {%
x	y\\
-0.537299608346824	-0.843391445812885\\
-0.342020143325669	-0.939692620785908\\
-0.413746227580361	-0.755076387108912\\
}--cycle;


\addplot[area legend,solid,fill=white,draw=blue,forget plot]
table[row sep=crcr] {%
x	y\\
-0.302928992904613	-0.686753073231595\\
-0.485472311835053	-0.570460153431915\\
-0.413746227580361	-0.755076387108912\\
}--cycle;


\addplot[area legend,solid,fill=white,draw=blue,forget plot]
table[row sep=crcr] {%
x	y\\
-0.302928992904613	-0.686753073231595\\
-0.413746227580361	-0.755076387108912\\
-0.342020143325669	-0.939692620785908\\
}--cycle;


\addplot[area legend,solid,fill=white,draw=blue,forget plot]
table[row sep=crcr] {%
x	y\\
-0.707106781186548	-0.707106781186547\\
-0.537299608346824	-0.843391445812885\\
-0.511385960090939	-0.7069257996224\\
}--cycle;


\addplot[area legend,solid,fill=white,draw=blue,forget plot]
table[row sep=crcr] {%
x	y\\
-0.413746227580361	-0.755076387108912\\
-0.485472311835053	-0.570460153431915\\
-0.511385960090939	-0.7069257996224\\
}--cycle;


\addplot[area legend,solid,fill=white,draw=blue,forget plot]
table[row sep=crcr] {%
x	y\\
-0.413746227580361	-0.755076387108912\\
-0.511385960090939	-0.7069257996224\\
-0.537299608346824	-0.843391445812885\\
}--cycle;


\addplot[area legend,solid,fill=white,draw=blue,forget plot]
table[row sep=crcr] {%
x	y\\
-0.673051597688137	-0.458312227389373\\
-0.707106781186548	-0.707106781186547\\
-0.5962895465108	-0.638783467309231\\
}--cycle;


\addplot[area legend,solid,fill=white,draw=blue,forget plot]
table[row sep=crcr] {%
x	y\\
-0.673051597688137	-0.458312227389373\\
-0.5962895465108	-0.638783467309231\\
-0.485472311835053	-0.570460153431915\\
}--cycle;


\addplot[area legend,solid,fill=white,draw=blue,forget plot]
table[row sep=crcr] {%
x	y\\
-0.511385960090939	-0.7069257996224\\
-0.485472311835053	-0.570460153431915\\
-0.5962895465108	-0.638783467309231\\
}--cycle;


\addplot[area legend,solid,fill=white,draw=blue,forget plot]
table[row sep=crcr] {%
x	y\\
-0.511385960090939	-0.7069257996224\\
-0.5962895465108	-0.638783467309231\\
-0.707106781186548	-0.707106781186547\\
}--cycle;


\addplot[area legend,solid,fill=white,draw=blue,forget plot]
table[row sep=crcr] {%
x	y\\
0.318924314181972	0.706619409844566\\
0.642787609686539	0.766044443118978\\
0.461748613235034	0.887010833178222\\
}--cycle;


\addplot[area legend,solid,fill=white,draw=blue,forget plot]
table[row sep=crcr] {%
x	y\\
0.111239202851876	0.748066790703153\\
-0.17364817766693	0.984807753012208\\
-0.10499440853285	0.757507754064723\\
}--cycle;


\addplot[area legend,solid,fill=white,draw=blue,forget plot]
table[row sep=crcr] {%
x	y\\
0.111239202851876	0.748066790703153\\
-0.10499440853285	0.757507754064723\\
-0.0363406393987691	0.530207755117238\\
}--cycle;


\addplot[area legend,solid,fill=white,draw=blue,forget plot]
table[row sep=crcr] {%
x	y\\
-0.295169170605093	0.763993226300075\\
-0.10499440853285	0.757507754064723\\
-0.17364817766693	0.984807753012208\\
}--cycle;


\addplot[area legend,solid,fill=white,draw=blue,forget plot]
table[row sep=crcr] {%
x	y\\
0.111239202851876	0.748066790703153\\
0.258819045102521	0.965925826289068\\
0.043619387365336	0.999048221581858\\
}--cycle;


\addplot[area legend,solid,fill=white,draw=blue,forget plot]
table[row sep=crcr] {%
x	y\\
0.111239202851876	0.748066790703153\\
0.043619387365336	0.999048221581858\\
-0.17364817766693	0.984807753012208\\
}--cycle;


\addplot[area legend,solid,fill=white,draw=blue,forget plot]
table[row sep=crcr] {%
x	y\\
0.111239202851876	0.748066790703153\\
0.318924314181972	0.706619409844566\\
0.288871679642246	0.836272618066817\\
}--cycle;


\addplot[area legend,solid,fill=white,draw=blue,forget plot]
table[row sep=crcr] {%
x	y\\
0.111239202851876	0.748066790703153\\
0.288871679642246	0.836272618066817\\
0.258819045102521	0.965925826289068\\
}--cycle;


\addplot[area legend,solid,fill=white,draw=blue,forget plot]
table[row sep=crcr] {%
x	y\\
0.461748613235034	0.887010833178222\\
0.258819045102521	0.965925826289068\\
0.288871679642246	0.836272618066817\\
}--cycle;


\addplot[area legend,solid,fill=white,draw=blue,forget plot]
table[row sep=crcr] {%
x	y\\
0.461748613235034	0.887010833178222\\
0.288871679642246	0.836272618066817\\
0.318924314181972	0.706619409844566\\
}--cycle;


\addplot[area legend,solid,fill=white,draw=blue,forget plot]
table[row sep=crcr] {%
x	y\\
0.111239202851876	0.748066790703153\\
-0.0363406393987691	0.530207755117238\\
0.141291837391601	0.618413582480902\\
}--cycle;


\addplot[area legend,solid,fill=white,draw=blue,forget plot]
table[row sep=crcr] {%
x	y\\
0.111239202851876	0.748066790703153\\
0.141291837391601	0.618413582480902\\
0.318924314181972	0.706619409844566\\
}--cycle;


\addplot[area legend,solid,fill=white,draw=blue,forget plot]
table[row sep=crcr] {%
x	y\\
0.171344471931327	0.488760374258651\\
0.318924314181972	0.706619409844566\\
0.141291837391601	0.618413582480902\\
}--cycle;


\addplot[area legend,solid,fill=white,draw=blue,forget plot]
table[row sep=crcr] {%
x	y\\
0.171344471931327	0.488760374258651\\
0.141291837391601	0.618413582480902\\
-0.0363406393987691	0.530207755117238\\
}--cycle;


\addplot[area legend,solid,fill=white,draw=blue,forget plot]
table[row sep=crcr] {%
x	y\\
-0.953716950748227	0.300705799504273\\
-0.996194698091746	0.0871557427476582\\
-0.792013515669114	0.136470977914537\\
}--cycle;


\addplot[area legend,solid,fill=white,draw=blue,forget plot]
table[row sep=crcr] {%
x	y\\
-0.953716950748227	0.300705799504273\\
-0.792013515669114	0.136470977914537\\
-0.587832333246482	0.185786213081415\\
}--cycle;


\addplot[area legend,solid,fill=white,draw=blue,forget plot]
table[row sep=crcr] {%
x	y\\
-0.817595556140736	-0.0611809654222698\\
-0.792013515669114	0.136470977914537\\
-0.996194698091746	0.0871557427476582\\
}--cycle;


\addplot[area legend,solid,fill=white,draw=blue,forget plot]
table[row sep=crcr] {%
x	y\\
0.168577871373829	-0.714603699991982\\
0.5	-0.866025403784439\\
0.375	-0.649519052838329\\
}--cycle;


\addplot[area legend,solid,fill=white,draw=blue,forget plot]
table[row sep=crcr] {%
x	y\\
0.168577871373829	-0.714603699991982\\
0.375	-0.649519052838329\\
0.25	-0.433012701892219\\
}--cycle;


\addplot[area legend,solid,fill=white,draw=blue,forget plot]
table[row sep=crcr] {%
x	y\\
0.168577871373829	-0.714603699991982\\
0.0871557427476579	-0.996194698091746\\
0.300705799504273	-0.953716950748227\\
}--cycle;


\addplot[area legend,solid,fill=white,draw=blue,forget plot]
table[row sep=crcr] {%
x	y\\
0.168577871373829	-0.714603699991982\\
0.300705799504273	-0.953716950748227\\
0.5	-0.866025403784439\\
}--cycle;


\addplot[area legend,solid,fill=white,draw=blue,forget plot]
table[row sep=crcr] {%
x	y\\
0.168577871373829	-0.714603699991982\\
-0.0666913610308223	-0.700759246464352\\
0.0102321908584178	-0.848476972278049\\
}--cycle;


\addplot[area legend,solid,fill=white,draw=blue,forget plot]
table[row sep=crcr] {%
x	y\\
0.168577871373829	-0.714603699991982\\
0.0102321908584178	-0.848476972278049\\
0.0871557427476579	-0.996194698091746\\
}--cycle;


\addplot[area legend,solid,fill=white,draw=blue,forget plot]
table[row sep=crcr] {%
x	y\\
-0.130526192220052	-0.99144486137381\\
0.0871557427476579	-0.996194698091746\\
0.0102321908584178	-0.848476972278049\\
}--cycle;


\addplot[area legend,solid,fill=white,draw=blue,forget plot]
table[row sep=crcr] {%
x	y\\
-0.130526192220052	-0.99144486137381\\
0.0102321908584178	-0.848476972278049\\
-0.0666913610308223	-0.700759246464352\\
}--cycle;


\addplot[area legend,solid,fill=white,draw=blue,forget plot]
table[row sep=crcr] {%
x	y\\
-0.302928992904613	-0.686753073231595\\
-0.342020143325669	-0.939692620785908\\
-0.204355752178245	-0.82022593362513\\
}--cycle;


\addplot[area legend,solid,fill=white,draw=blue,forget plot]
table[row sep=crcr] {%
x	y\\
-0.302928992904613	-0.686753073231595\\
-0.204355752178245	-0.82022593362513\\
-0.0666913610308223	-0.700759246464352\\
}--cycle;


\addplot[area legend,solid,fill=white,draw=blue,forget plot]
table[row sep=crcr] {%
x	y\\
-0.130526192220052	-0.99144486137381\\
-0.0666913610308223	-0.700759246464352\\
-0.204355752178245	-0.82022593362513\\
}--cycle;


\addplot[area legend,solid,fill=white,draw=blue,forget plot]
table[row sep=crcr] {%
x	y\\
-0.130526192220052	-0.99144486137381\\
-0.204355752178245	-0.82022593362513\\
-0.342020143325669	-0.939692620785908\\
}--cycle;


\addplot[area legend,solid,fill=white,draw=blue,forget plot]
table[row sep=crcr] {%
x	y\\
-0.302928992904613	-0.686753073231595\\
-0.263837842483558	-0.433813525677282\\
-0.374655077159305	-0.502136839554598\\
}--cycle;


\addplot[area legend,solid,fill=white,draw=blue,forget plot]
table[row sep=crcr] {%
x	y\\
-0.302928992904613	-0.686753073231595\\
-0.374655077159305	-0.502136839554598\\
-0.485472311835053	-0.570460153431915\\
}--cycle;


\addplot[area legend,solid,fill=white,draw=blue,forget plot]
table[row sep=crcr] {%
x	y\\
-0.451417128336643	-0.32166559963474\\
-0.485472311835053	-0.570460153431915\\
-0.374655077159305	-0.502136839554598\\
}--cycle;


\addplot[area legend,solid,fill=white,draw=blue,forget plot]
table[row sep=crcr] {%
x	y\\
-0.451417128336643	-0.32166559963474\\
-0.374655077159305	-0.502136839554598\\
-0.263837842483558	-0.433813525677282\\
}--cycle;


\addplot[area legend,solid,fill=white,draw=blue,forget plot]
table[row sep=crcr] {%
x	y\\
-0.817595556140736	-0.0611809654222698\\
-0.996194698091746	0.0871557427476582\\
-0.99144486137381	-0.130526192220051\\
}--cycle;


\addplot[area legend,solid,fill=white,draw=blue,forget plot]
table[row sep=crcr] {%
x	y\\
-0.817595556140736	-0.0611809654222698\\
-0.99144486137381	-0.130526192220051\\
-0.939692620785908	-0.342020143325669\\
}--cycle;


\addplot[area legend,solid,fill=white,draw=blue,forget plot]
table[row sep=crcr] {%
x	y\\
-0.295169170605093	0.763993226300075\\
-0.17364817766693	0.984807753012208\\
-0.38268343236509	0.923879532511287\\
}--cycle;


\addplot[area legend,solid,fill=white,draw=blue,forget plot]
table[row sep=crcr] {%
x	y\\
-0.295169170605093	0.763993226300075\\
-0.38268343236509	0.923879532511287\\
-0.573576436351046	0.819152044288992\\
}--cycle;


\addplot[area legend,solid,fill=white,draw=blue,forget plot]
table[row sep=crcr] {%
x	y\\
-0.502261248394869	0.364482456334678\\
-0.275219716646685	0.242071513609225\\
-0.34595494009497	0.392625106598583\\
}--cycle;


\addplot[area legend,solid,fill=white,draw=blue,forget plot]
table[row sep=crcr] {%
x	y\\
-0.502261248394869	0.364482456334678\\
-0.34595494009497	0.392625106598583\\
-0.416690163543255	0.543178699587941\\
}--cycle;


\addplot[area legend,solid,fill=white,draw=blue,forget plot]
table[row sep=crcr] {%
x	y\\
-0.226515401471012	0.53669322735259\\
-0.416690163543255	0.543178699587941\\
-0.34595494009497	0.392625106598583\\
}--cycle;


\addplot[area legend,solid,fill=white,draw=blue,forget plot]
table[row sep=crcr] {%
x	y\\
-0.226515401471012	0.53669322735259\\
-0.34595494009497	0.392625106598583\\
-0.275219716646685	0.242071513609225\\
}--cycle;


\addplot[area legend,solid,fill=white,draw=blue,forget plot]
table[row sep=crcr] {%
x	y\\
-0.502261248394869	0.364482456334678\\
-0.587832333246482	0.185786213081415\\
-0.431526024946584	0.21392886334532\\
}--cycle;


\addplot[area legend,solid,fill=white,draw=blue,forget plot]
table[row sep=crcr] {%
x	y\\
-0.502261248394869	0.364482456334678\\
-0.431526024946584	0.21392886334532\\
-0.275219716646685	0.242071513609225\\
}--cycle;


\addplot[area legend,solid,fill=white,draw=blue,forget plot]
table[row sep=crcr] {%
x	y\\
-0.469016774382391	0.0710438783193448\\
-0.275219716646685	0.242071513609225\\
-0.431526024946584	0.21392886334532\\
}--cycle;


\addplot[area legend,solid,fill=white,draw=blue,forget plot]
table[row sep=crcr] {%
x	y\\
-0.469016774382391	0.0710438783193448\\
-0.431526024946584	0.21392886334532\\
-0.587832333246482	0.185786213081415\\
}--cycle;


\addplot[area legend,solid,fill=white,draw=blue,forget plot]
table[row sep=crcr] {%
x	y\\
-0.502261248394869	0.364482456334678\\
-0.641357783663847	0.521589349793971\\
-0.614595058455164	0.353687781437693\\
}--cycle;


\addplot[area legend,solid,fill=white,draw=blue,forget plot]
table[row sep=crcr] {%
x	y\\
-0.502261248394869	0.364482456334678\\
-0.614595058455164	0.353687781437693\\
-0.587832333246482	0.185786213081415\\
}--cycle;


\addplot[area legend,solid,fill=white,draw=blue,forget plot]
table[row sep=crcr] {%
x	y\\
-0.72692886851546	0.342893106540707\\
-0.587832333246482	0.185786213081415\\
-0.614595058455164	0.353687781437693\\
}--cycle;


\addplot[area legend,solid,fill=white,draw=blue,forget plot]
table[row sep=crcr] {%
x	y\\
-0.72692886851546	0.342893106540707\\
-0.614595058455164	0.353687781437693\\
-0.641357783663847	0.521589349793971\\
}--cycle;


\addplot[area legend,solid,fill=white,draw=blue,forget plot]
table[row sep=crcr] {%
x	y\\
-0.502261248394869	0.364482456334678\\
-0.416690163543255	0.543178699587941\\
-0.529023973603551	0.532384024690956\\
}--cycle;


\addplot[area legend,solid,fill=white,draw=blue,forget plot]
table[row sep=crcr] {%
x	y\\
-0.502261248394869	0.364482456334678\\
-0.529023973603551	0.532384024690956\\
-0.641357783663847	0.521589349793971\\
}--cycle;


\addplot[area legend,solid,fill=white,draw=blue,forget plot]
table[row sep=crcr] {%
x	y\\
-0.495133299947151	0.681165371938466\\
-0.641357783663847	0.521589349793971\\
-0.529023973603551	0.532384024690956\\
}--cycle;


\addplot[area legend,solid,fill=white,draw=blue,forget plot]
table[row sep=crcr] {%
x	y\\
-0.495133299947151	0.681165371938466\\
-0.529023973603551	0.532384024690956\\
-0.416690163543255	0.543178699587941\\
}--cycle;


\addplot[area legend,solid,fill=white,draw=blue,forget plot]
table[row sep=crcr] {%
x	y\\
-0.451417128336643	-0.32166559963474\\
-0.3502012155183	-0.0436984564427255\\
-0.494598814854014	-0.126608065017462\\
}--cycle;


\addplot[area legend,solid,fill=white,draw=blue,forget plot]
table[row sep=crcr] {%
x	y\\
-0.451417128336643	-0.32166559963474\\
-0.494598814854014	-0.126608065017462\\
-0.638996414189727	-0.209517673592198\\
}--cycle;


\addplot[area legend,solid,fill=white,draw=blue,forget plot]
table[row sep=crcr] {%
x	y\\
-0.613414373718105	-0.0118657302553914\\
-0.638996414189727	-0.209517673592198\\
-0.494598814854014	-0.126608065017462\\
}--cycle;


\addplot[area legend,solid,fill=white,draw=blue,forget plot]
table[row sep=crcr] {%
x	y\\
-0.613414373718105	-0.0118657302553914\\
-0.817595556140736	-0.0611809654222698\\
-0.728295985165232	-0.135349319507234\\
}--cycle;


\addplot[area legend,solid,fill=white,draw=blue,forget plot]
table[row sep=crcr] {%
x	y\\
-0.613414373718105	-0.0118657302553914\\
-0.728295985165232	-0.135349319507234\\
-0.638996414189727	-0.209517673592198\\
}--cycle;


\addplot[area legend,solid,fill=white,draw=blue,forget plot]
table[row sep=crcr] {%
x	y\\
-0.789344517487818	-0.275768908458933\\
-0.638996414189727	-0.209517673592198\\
-0.728295985165232	-0.135349319507234\\
}--cycle;


\addplot[area legend,solid,fill=white,draw=blue,forget plot]
table[row sep=crcr] {%
x	y\\
-0.789344517487818	-0.275768908458933\\
-0.728295985165232	-0.135349319507234\\
-0.817595556140736	-0.0611809654222698\\
}--cycle;


\addplot[area legend,solid,fill=white,draw=blue,forget plot]
table[row sep=crcr] {%
x	y\\
-0.613414373718105	-0.0118657302553914\\
-0.702713944693609	0.0623026238295726\\
-0.817595556140736	-0.0611809654222698\\
}--cycle;


\addplot[area legend,solid,fill=white,draw=blue,forget plot]
table[row sep=crcr] {%
x	y\\
-0.792013515669114	0.136470977914537\\
-0.817595556140736	-0.0611809654222698\\
-0.702713944693609	0.0623026238295726\\
}--cycle;


\addplot[area legend,solid,fill=white,draw=blue,forget plot]
table[row sep=crcr] {%
x	y\\
-0.792013515669114	0.136470977914537\\
-0.702713944693609	0.0623026238295726\\
-0.587832333246482	0.185786213081415\\
}--cycle;


\addplot[area legend,solid,fill=white,draw=blue,forget plot]
table[row sep=crcr] {%
x	y\\
-0.226515401471012	0.53669322735259\\
-0.295169170605093	0.763993226300075\\
-0.355929667074174	0.653585962944008\\
}--cycle;


\addplot[area legend,solid,fill=white,draw=blue,forget plot]
table[row sep=crcr] {%
x	y\\
-0.226515401471012	0.53669322735259\\
-0.355929667074174	0.653585962944008\\
-0.416690163543255	0.543178699587941\\
}--cycle;


\addplot[area legend,solid,fill=white,draw=blue,forget plot]
table[row sep=crcr] {%
x	y\\
-0.495133299947151	0.681165371938466\\
-0.416690163543255	0.543178699587941\\
-0.355929667074174	0.653585962944008\\
}--cycle;


\addplot[area legend,solid,fill=white,draw=blue,forget plot]
table[row sep=crcr] {%
x	y\\
-0.495133299947151	0.681165371938466\\
-0.355929667074174	0.653585962944008\\
-0.295169170605093	0.763993226300075\\
}--cycle;


\addplot[area legend,solid,fill=white,draw=blue,forget plot]
table[row sep=crcr] {%
x	y\\
-0.226515401471012	0.53669322735259\\
-0.0363406393987691	0.530207755117238\\
-0.165754905001931	0.647100490708656\\
}--cycle;


\addplot[area legend,solid,fill=white,draw=blue,forget plot]
table[row sep=crcr] {%
x	y\\
-0.226515401471012	0.53669322735259\\
-0.165754905001931	0.647100490708656\\
-0.295169170605093	0.763993226300075\\
}--cycle;


\addplot[area legend,solid,fill=white,draw=blue,forget plot]
table[row sep=crcr] {%
x	y\\
-0.10499440853285	0.757507754064723\\
-0.295169170605093	0.763993226300075\\
-0.165754905001931	0.647100490708656\\
}--cycle;


\addplot[area legend,solid,fill=white,draw=blue,forget plot]
table[row sep=crcr] {%
x	y\\
-0.10499440853285	0.757507754064723\\
-0.165754905001931	0.647100490708656\\
-0.0363406393987691	0.530207755117238\\
}--cycle;


\addplot[area legend,solid,fill=white,draw=blue,forget plot]
table[row sep=crcr] {%
x	y\\
-0.469016774382391	0.0710438783193448\\
-0.613414373718105	-0.0118657302553914\\
-0.481807794618202	-0.0277820933490584\\
}--cycle;


\addplot[area legend,solid,fill=white,draw=blue,forget plot]
table[row sep=crcr] {%
x	y\\
-0.469016774382391	0.0710438783193448\\
-0.481807794618202	-0.0277820933490584\\
-0.3502012155183	-0.0436984564427255\\
}--cycle;


\addplot[area legend,solid,fill=white,draw=blue,forget plot]
table[row sep=crcr] {%
x	y\\
-0.494598814854014	-0.126608065017462\\
-0.3502012155183	-0.0436984564427255\\
-0.481807794618202	-0.0277820933490584\\
}--cycle;


\addplot[area legend,solid,fill=white,draw=blue,forget plot]
table[row sep=crcr] {%
x	y\\
-0.494598814854014	-0.126608065017462\\
-0.481807794618202	-0.0277820933490584\\
-0.613414373718105	-0.0118657302553914\\
}--cycle;


\addplot[area legend,solid,fill=white,draw=blue,forget plot]
table[row sep=crcr] {%
x	y\\
-0.469016774382391	0.0710438783193448\\
-0.587832333246482	0.185786213081415\\
-0.600623353482293	0.0869602414130118\\
}--cycle;


\addplot[area legend,solid,fill=white,draw=blue,forget plot]
table[row sep=crcr] {%
x	y\\
-0.469016774382391	0.0710438783193448\\
-0.600623353482293	0.0869602414130118\\
-0.613414373718105	-0.0118657302553914\\
}--cycle;


\addplot[area legend,solid,fill=white,draw=blue,forget plot]
table[row sep=crcr] {%
x	y\\
-0.702713944693609	0.0623026238295726\\
-0.613414373718105	-0.0118657302553914\\
-0.600623353482293	0.0869602414130118\\
}--cycle;


\addplot[area legend,solid,fill=white,draw=blue,forget plot]
table[row sep=crcr] {%
x	y\\
-0.702713944693609	0.0623026238295726\\
-0.600623353482293	0.0869602414130118\\
-0.587832333246482	0.185786213081415\\
}--cycle;

\addplot [color=red,solid,forget plot]
  table[row sep=crcr]{%
-0.866025403784439	0.5\\
-0.953716950748227	0.300705799504273\\
};
\addplot [color=red,solid,forget plot]
  table[row sep=crcr]{%
-0.953716950748227	0.300705799504273\\
-0.996194698091746	0.0871557427476582\\
};
\addplot [color=red,solid,forget plot]
  table[row sep=crcr]{%
-0.342020143325669	-0.939692620785908\\
-0.130526192220052	-0.99144486137381\\
};
\addplot [color=red,solid,forget plot]
  table[row sep=crcr]{%
-0.130526192220052	-0.99144486137381\\
0.0871557427476579	-0.996194698091746\\
};
\addplot [color=red,solid,forget plot]
  table[row sep=crcr]{%
0.25	-0.433012701892219\\
0.125	-0.21650635094611\\
};
\addplot [color=red,solid,forget plot]
  table[row sep=crcr]{%
0.125	-0.21650635094611\\
0	0\\
};
\addplot [color=red,solid,forget plot]
  table[row sep=crcr]{%
1	0\\
0.976296007119933	0.216439613938103\\
};
\addplot [color=red,solid,forget plot]
  table[row sep=crcr]{%
0.976296007119933	0.216439613938103\\
0.90630778703665	0.422618261740699\\
};
\addplot [color=red,solid,forget plot]
  table[row sep=crcr]{%
0.5	0\\
0.75	0\\
};
\addplot [color=red,solid,forget plot]
  table[row sep=crcr]{%
0.75	0\\
1	0\\
};
\addplot [color=red,solid,forget plot]
  table[row sep=crcr]{%
0	0\\
0.25	0\\
};
\addplot [color=red,solid,forget plot]
  table[row sep=crcr]{%
0.25	0\\
0.5	0\\
};
\addplot [color=red,solid,forget plot]
  table[row sep=crcr]{%
-0.573576436351046	0.819152044288992\\
-0.737277336810124	0.67559020761566\\
};
\addplot [color=red,solid,forget plot]
  table[row sep=crcr]{%
-0.737277336810124	0.67559020761566\\
-0.866025403784439	0.5\\
};
\addplot [color=red,solid,forget plot]
  table[row sep=crcr]{%
0.90630778703665	0.422618261740699\\
0.793353340291235	0.608761429008721\\
};
\addplot [color=red,solid,forget plot]
  table[row sep=crcr]{%
0.793353340291235	0.608761429008721\\
0.642787609686539	0.766044443118978\\
};
\addplot [color=red,solid,forget plot]
  table[row sep=crcr]{%
-0.707106781186548	-0.707106781186547\\
-0.537299608346824	-0.843391445812885\\
};
\addplot [color=red,solid,forget plot]
  table[row sep=crcr]{%
-0.537299608346824	-0.843391445812885\\
-0.342020143325669	-0.939692620785908\\
};
\addplot [color=red,solid,forget plot]
  table[row sep=crcr]{%
-0.939692620785908	-0.342020143325669\\
-0.843391445812886	-0.537299608346824\\
};
\addplot [color=red,solid,forget plot]
  table[row sep=crcr]{%
-0.843391445812886	-0.537299608346824\\
-0.707106781186548	-0.707106781186547\\
};
\addplot [color=red,solid,forget plot]
  table[row sep=crcr]{%
0.642787609686539	0.766044443118978\\
0.461748613235034	0.887010833178222\\
};
\addplot [color=red,solid,forget plot]
  table[row sep=crcr]{%
0.461748613235034	0.887010833178222\\
0.258819045102521	0.965925826289068\\
};
\addplot [color=red,solid,forget plot]
  table[row sep=crcr]{%
0.258819045102521	0.965925826289068\\
0.043619387365336	0.999048221581858\\
};
\addplot [color=red,solid,forget plot]
  table[row sep=crcr]{%
0.043619387365336	0.999048221581858\\
-0.17364817766693	0.984807753012208\\
};
\addplot [color=red,solid,forget plot]
  table[row sep=crcr]{%
0.5	-0.866025403784439\\
0.375	-0.649519052838329\\
};
\addplot [color=red,solid,forget plot]
  table[row sep=crcr]{%
0.375	-0.649519052838329\\
0.25	-0.433012701892219\\
};
\addplot [color=red,solid,forget plot]
  table[row sep=crcr]{%
0.0871557427476579	-0.996194698091746\\
0.300705799504273	-0.953716950748227\\
};
\addplot [color=red,solid,forget plot]
  table[row sep=crcr]{%
0.300705799504273	-0.953716950748227\\
0.5	-0.866025403784439\\
};
\addplot [color=red,solid,forget plot]
  table[row sep=crcr]{%
-0.996194698091746	0.0871557427476582\\
-0.99144486137381	-0.130526192220051\\
};
\addplot [color=red,solid,forget plot]
  table[row sep=crcr]{%
-0.99144486137381	-0.130526192220051\\
-0.939692620785908	-0.342020143325669\\
};
\addplot [color=red,solid,forget plot]
  table[row sep=crcr]{%
-0.17364817766693	0.984807753012208\\
-0.38268343236509	0.923879532511287\\
};
\addplot [color=red,solid,forget plot]
  table[row sep=crcr]{%
-0.38268343236509	0.923879532511287\\
-0.573576436351046	0.819152044288992\\
};
\end{axis}
\end{tikzpicture}%
	   \caption{\Code{sectorg}, zweifache globale Verfeinerung}
        \end{subfigure}
        \begin{subfigure}[t]{0.32\textwidth}
	   % This file was created by matlab2tikz v0.5.0 running on MATLAB 8.3.
% Copyright (c) 2008--2014, Nico Schlömer <nico.schloemer@gmail.com>
% All rights reserved.
% Minimal pgfplots version: 1.3
% 
% The latest updates can be retrieved from
%   http://www.mathworks.com/matlabcentral/fileexchange/22022-matlab2tikz
% where you can also make suggestions and rate matlab2tikz.
% 
\begin{tikzpicture}

\begin{axis}[%
width=0.950920245398773\fwidth,
height=\fheight,
at={(0\fwidth,0\fheight)},
scale only axis,
xmin=-1,
xmax=1,
ymin=-1,
ymax=1,
axis x line*=bottom,
axis y line*=left
]

\addplot[area legend,solid,fill=white,draw=blue,forget plot]
table[row sep=crcr] {%
x	y\\
0.25	-0.433012701892219\\
0	0\\
-0.263837842483558	-0.433813525677282\\
}--cycle;


\addplot[area legend,solid,fill=white,draw=blue,forget plot]
table[row sep=crcr] {%
x	y\\
0.5	0\\
1	0\\
0.90630778703665	0.422618261740699\\
}--cycle;


\addplot[area legend,solid,fill=white,draw=blue,forget plot]
table[row sep=crcr] {%
x	y\\
-0.263837842483558	-0.433813525677282\\
0	0\\
-0.3502012155183	-0.0436984564427255\\
}--cycle;


\addplot[area legend,solid,fill=white,draw=blue,forget plot]
table[row sep=crcr] {%
x	y\\
0.5	0\\
0.90630778703665	0.422618261740699\\
0.379029583261423	0.447312993400064\\
}--cycle;


\addplot[area legend,solid,fill=white,draw=blue,forget plot]
table[row sep=crcr] {%
x	y\\
0.90630778703665	0.422618261740699\\
0.642787609686539	0.766044443118978\\
0.379029583261423	0.447312993400064\\
}--cycle;


\addplot[area legend,solid,fill=white,draw=blue,forget plot]
table[row sep=crcr] {%
x	y\\
-0.939692620785908	-0.342020143325669\\
-0.707106781186548	-0.707106781186547\\
-0.638996414189727	-0.209517673592198\\
}--cycle;


\addplot[area legend,solid,fill=white,draw=blue,forget plot]
table[row sep=crcr] {%
x	y\\
0.258819045102521	0.965925826289068\\
-0.17364817766693	0.984807753012208\\
-0.0363406393987691	0.530207755117238\\
}--cycle;


\addplot[area legend,solid,fill=white,draw=blue,forget plot]
table[row sep=crcr] {%
x	y\\
0.5	-0.866025403784439\\
0.25	-0.433012701892219\\
0.0871557427476579	-0.996194698091746\\
}--cycle;


\addplot[area legend,solid,fill=white,draw=blue,forget plot]
table[row sep=crcr] {%
x	y\\
0.0871557427476579	-0.996194698091746\\
0.25	-0.433012701892219\\
-0.0666913610308223	-0.700759246464352\\
}--cycle;


\addplot[area legend,solid,fill=white,draw=blue,forget plot]
table[row sep=crcr] {%
x	y\\
-0.263837842483558	-0.433813525677282\\
-0.342020143325669	-0.939692620785908\\
-0.0666913610308223	-0.700759246464352\\
}--cycle;


\addplot[area legend,solid,fill=white,draw=blue,forget plot]
table[row sep=crcr] {%
x	y\\
0.642787609686539	0.766044443118978\\
0.258819045102521	0.965925826289068\\
0.379029583261423	0.447312993400064\\
}--cycle;


\addplot[area legend,solid,fill=white,draw=blue,forget plot]
table[row sep=crcr] {%
x	y\\
0.258819045102521	0.965925826289068\\
-0.0363406393987691	0.530207755117238\\
0.379029583261423	0.447312993400064\\
}--cycle;


\addplot[area legend,solid,fill=white,draw=blue,forget plot]
table[row sep=crcr] {%
x	y\\
-0.342020143325669	-0.939692620785908\\
0.0871557427476579	-0.996194698091746\\
-0.0666913610308223	-0.700759246464352\\
}--cycle;


\addplot[area legend,solid,fill=white,draw=blue,forget plot]
table[row sep=crcr] {%
x	y\\
0.25	-0.433012701892219\\
-0.263837842483558	-0.433813525677282\\
-0.0666913610308223	-0.700759246464352\\
}--cycle;


\addplot[area legend,solid,fill=white,draw=blue,forget plot]
table[row sep=crcr] {%
x	y\\
-0.0363406393987691	0.530207755117238\\
0.189514791630711	0.223656496700032\\
0.379029583261423	0.447312993400064\\
}--cycle;


\addplot[area legend,solid,fill=white,draw=blue,forget plot]
table[row sep=crcr] {%
x	y\\
0.5	0\\
0.379029583261423	0.447312993400064\\
0.189514791630711	0.223656496700032\\
}--cycle;


\addplot[area legend,solid,fill=white,draw=blue,forget plot]
table[row sep=crcr] {%
x	y\\
0.5	0\\
0.189514791630711	0.223656496700032\\
0	0\\
}--cycle;


\addplot[area legend,solid,fill=white,draw=blue,forget plot]
table[row sep=crcr] {%
x	y\\
0.189514791630711	0.223656496700032\\
-0.0363406393987691	0.530207755117238\\
-0.0181703196993845	0.265103877558619\\
}--cycle;


\addplot[area legend,solid,fill=white,draw=blue,forget plot]
table[row sep=crcr] {%
x	y\\
0.189514791630711	0.223656496700032\\
-0.0181703196993845	0.265103877558619\\
0	0\\
}--cycle;


\addplot[area legend,solid,fill=white,draw=blue,forget plot]
table[row sep=crcr] {%
x	y\\
-0.3502012155183	-0.0436984564427255\\
0	0\\
-0.137609858323343	0.121035756804613\\
}--cycle;


\addplot[area legend,solid,fill=white,draw=blue,forget plot]
table[row sep=crcr] {%
x	y\\
-0.0181703196993845	0.265103877558619\\
-0.137609858323343	0.121035756804613\\
0	0\\
}--cycle;


\addplot[area legend,solid,fill=white,draw=blue,forget plot]
table[row sep=crcr] {%
x	y\\
-0.573576436351046	0.819152044288992\\
-0.866025403784439	0.5\\
-0.641357783663847	0.521589349793971\\
}--cycle;


\addplot[area legend,solid,fill=white,draw=blue,forget plot]
table[row sep=crcr] {%
x	y\\
-0.573576436351046	0.819152044288992\\
-0.641357783663847	0.521589349793971\\
-0.416690163543255	0.543178699587941\\
}--cycle;


\addplot[area legend,solid,fill=white,draw=blue,forget plot]
table[row sep=crcr] {%
x	y\\
-0.641357783663847	0.521589349793971\\
-0.502261248394869	0.364482456334678\\
-0.416690163543255	0.543178699587941\\
}--cycle;


\addplot[area legend,solid,fill=white,draw=blue,forget plot]
table[row sep=crcr] {%
x	y\\
-0.0363406393987691	0.530207755117238\\
-0.17364817766693	0.984807753012208\\
-0.295169170605093	0.763993226300075\\
}--cycle;


\addplot[area legend,solid,fill=white,draw=blue,forget plot]
table[row sep=crcr] {%
x	y\\
-0.573576436351046	0.819152044288992\\
-0.416690163543255	0.543178699587941\\
-0.295169170605093	0.763993226300075\\
}--cycle;


\addplot[area legend,solid,fill=white,draw=blue,forget plot]
table[row sep=crcr] {%
x	y\\
-0.573576436351046	0.819152044288992\\
-0.295169170605093	0.763993226300075\\
-0.17364817766693	0.984807753012208\\
}--cycle;


\addplot[area legend,solid,fill=white,draw=blue,forget plot]
table[row sep=crcr] {%
x	y\\
-0.295169170605093	0.763993226300075\\
-0.416690163543255	0.543178699587941\\
-0.226515401471012	0.53669322735259\\
}--cycle;


\addplot[area legend,solid,fill=white,draw=blue,forget plot]
table[row sep=crcr] {%
x	y\\
-0.295169170605093	0.763993226300075\\
-0.226515401471012	0.53669322735259\\
-0.0363406393987691	0.530207755117238\\
}--cycle;


\addplot[area legend,solid,fill=white,draw=blue,forget plot]
table[row sep=crcr] {%
x	y\\
-0.939692620785908	-0.342020143325669\\
-0.638996414189727	-0.209517673592198\\
-0.817595556140736	-0.0611809654222698\\
}--cycle;


\addplot[area legend,solid,fill=white,draw=blue,forget plot]
table[row sep=crcr] {%
x	y\\
-0.939692620785908	-0.342020143325669\\
-0.817595556140736	-0.0611809654222698\\
-0.996194698091746	0.0871557427476582\\
}--cycle;


\addplot[area legend,solid,fill=white,draw=blue,forget plot]
table[row sep=crcr] {%
x	y\\
-0.817595556140736	-0.0611809654222698\\
-0.638996414189727	-0.209517673592198\\
-0.613414373718105	-0.0118657302553914\\
}--cycle;


\addplot[area legend,solid,fill=white,draw=blue,forget plot]
table[row sep=crcr] {%
x	y\\
-0.0181703196993845	0.265103877558619\\
-0.0363406393987691	0.530207755117238\\
-0.155780178022727	0.386139634363231\\
}--cycle;


\addplot[area legend,solid,fill=white,draw=blue,forget plot]
table[row sep=crcr] {%
x	y\\
-0.226515401471012	0.53669322735259\\
-0.275219716646685	0.242071513609225\\
-0.155780178022727	0.386139634363231\\
}--cycle;


\addplot[area legend,solid,fill=white,draw=blue,forget plot]
table[row sep=crcr] {%
x	y\\
-0.226515401471012	0.53669322735259\\
-0.155780178022727	0.386139634363231\\
-0.0363406393987691	0.530207755117238\\
}--cycle;


\addplot[area legend,solid,fill=white,draw=blue,forget plot]
table[row sep=crcr] {%
x	y\\
-0.137609858323343	0.121035756804613\\
-0.312710466082493	0.0991865285832498\\
-0.3502012155183	-0.0436984564427255\\
}--cycle;


\addplot[area legend,solid,fill=white,draw=blue,forget plot]
table[row sep=crcr] {%
x	y\\
-0.137609858323343	0.121035756804613\\
-0.0181703196993845	0.265103877558619\\
-0.146695018173035	0.253587695583922\\
}--cycle;


\addplot[area legend,solid,fill=white,draw=blue,forget plot]
table[row sep=crcr] {%
x	y\\
-0.155780178022727	0.386139634363231\\
-0.275219716646685	0.242071513609225\\
-0.146695018173035	0.253587695583922\\
}--cycle;


\addplot[area legend,solid,fill=white,draw=blue,forget plot]
table[row sep=crcr] {%
x	y\\
-0.155780178022727	0.386139634363231\\
-0.146695018173035	0.253587695583922\\
-0.0181703196993845	0.265103877558619\\
}--cycle;


\addplot[area legend,solid,fill=white,draw=blue,forget plot]
table[row sep=crcr] {%
x	y\\
-0.502261248394869	0.364482456334678\\
-0.34595494009497	0.392625106598583\\
-0.416690163543255	0.543178699587941\\
}--cycle;


\addplot[area legend,solid,fill=white,draw=blue,forget plot]
table[row sep=crcr] {%
x	y\\
-0.226515401471012	0.53669322735259\\
-0.416690163543255	0.543178699587941\\
-0.34595494009497	0.392625106598583\\
}--cycle;


\addplot[area legend,solid,fill=white,draw=blue,forget plot]
table[row sep=crcr] {%
x	y\\
-0.226515401471012	0.53669322735259\\
-0.34595494009497	0.392625106598583\\
-0.275219716646685	0.242071513609225\\
}--cycle;


\addplot[area legend,solid,fill=white,draw=blue,forget plot]
table[row sep=crcr] {%
x	y\\
-0.641357783663847	0.521589349793971\\
-0.866025403784439	0.5\\
-0.72692886851546	0.342893106540707\\
}--cycle;


\addplot[area legend,solid,fill=white,draw=blue,forget plot]
table[row sep=crcr] {%
x	y\\
-0.953716950748227	0.300705799504273\\
-0.587832333246482	0.185786213081415\\
-0.72692886851546	0.342893106540707\\
}--cycle;


\addplot[area legend,solid,fill=white,draw=blue,forget plot]
table[row sep=crcr] {%
x	y\\
-0.953716950748227	0.300705799504273\\
-0.72692886851546	0.342893106540707\\
-0.866025403784439	0.5\\
}--cycle;


\addplot[area legend,solid,fill=white,draw=blue,forget plot]
table[row sep=crcr] {%
x	y\\
-0.502261248394869	0.364482456334678\\
-0.641357783663847	0.521589349793971\\
-0.614595058455164	0.353687781437693\\
}--cycle;


\addplot[area legend,solid,fill=white,draw=blue,forget plot]
table[row sep=crcr] {%
x	y\\
-0.72692886851546	0.342893106540707\\
-0.587832333246482	0.185786213081415\\
-0.614595058455164	0.353687781437693\\
}--cycle;


\addplot[area legend,solid,fill=white,draw=blue,forget plot]
table[row sep=crcr] {%
x	y\\
-0.72692886851546	0.342893106540707\\
-0.614595058455164	0.353687781437693\\
-0.641357783663847	0.521589349793971\\
}--cycle;


\addplot[area legend,solid,fill=white,draw=blue,forget plot]
table[row sep=crcr] {%
x	y\\
-0.614595058455164	0.353687781437693\\
-0.587832333246482	0.185786213081415\\
-0.545046790820675	0.275134334708047\\
}--cycle;


\addplot[area legend,solid,fill=white,draw=blue,forget plot]
table[row sep=crcr] {%
x	y\\
-0.614595058455164	0.353687781437693\\
-0.545046790820675	0.275134334708047\\
-0.502261248394869	0.364482456334678\\
}--cycle;


\addplot[area legend,solid,fill=white,draw=blue,forget plot]
table[row sep=crcr] {%
x	y\\
-0.34595494009497	0.392625106598583\\
-0.502261248394869	0.364482456334678\\
-0.388740482520777	0.303276984971952\\
}--cycle;


\addplot[area legend,solid,fill=white,draw=blue,forget plot]
table[row sep=crcr] {%
x	y\\
-0.34595494009497	0.392625106598583\\
-0.388740482520777	0.303276984971952\\
-0.275219716646685	0.242071513609225\\
}--cycle;


\addplot[area legend,solid,fill=white,draw=blue,forget plot]
table[row sep=crcr] {%
x	y\\
-0.817595556140736	-0.0611809654222698\\
-0.792013515669114	0.136470977914537\\
-0.996194698091746	0.0871557427476582\\
}--cycle;


\addplot[area legend,solid,fill=white,draw=blue,forget plot]
table[row sep=crcr] {%
x	y\\
-0.953716950748227	0.300705799504273\\
-0.996194698091746	0.0871557427476582\\
-0.792013515669114	0.136470977914537\\
}--cycle;


\addplot[area legend,solid,fill=white,draw=blue,forget plot]
table[row sep=crcr] {%
x	y\\
-0.953716950748227	0.300705799504273\\
-0.792013515669114	0.136470977914537\\
-0.587832333246482	0.185786213081415\\
}--cycle;


\addplot[area legend,solid,fill=white,draw=blue,forget plot]
table[row sep=crcr] {%
x	y\\
-0.613414373718105	-0.0118657302553914\\
-0.702713944693609	0.0623026238295726\\
-0.817595556140736	-0.0611809654222698\\
}--cycle;


\addplot[area legend,solid,fill=white,draw=blue,forget plot]
table[row sep=crcr] {%
x	y\\
-0.792013515669114	0.136470977914537\\
-0.817595556140736	-0.0611809654222698\\
-0.702713944693609	0.0623026238295726\\
}--cycle;


\addplot[area legend,solid,fill=white,draw=blue,forget plot]
table[row sep=crcr] {%
x	y\\
-0.792013515669114	0.136470977914537\\
-0.702713944693609	0.0623026238295726\\
-0.587832333246482	0.185786213081415\\
}--cycle;


\addplot[area legend,solid,fill=white,draw=blue,forget plot]
table[row sep=crcr] {%
x	y\\
-0.469016774382391	0.0710438783193448\\
-0.600623353482293	0.0869602414130118\\
-0.613414373718105	-0.0118657302553914\\
}--cycle;


\addplot[area legend,solid,fill=white,draw=blue,forget plot]
table[row sep=crcr] {%
x	y\\
-0.702713944693609	0.0623026238295726\\
-0.613414373718105	-0.0118657302553914\\
-0.600623353482293	0.0869602414130118\\
}--cycle;


\addplot[area legend,solid,fill=white,draw=blue,forget plot]
table[row sep=crcr] {%
x	y\\
-0.702713944693609	0.0623026238295726\\
-0.600623353482293	0.0869602414130118\\
-0.587832333246482	0.185786213081415\\
}--cycle;


\addplot[area legend,solid,fill=white,draw=blue,forget plot]
table[row sep=crcr] {%
x	y\\
-0.600623353482293	0.0869602414130118\\
-0.469016774382391	0.0710438783193448\\
-0.528424553814436	0.12841504570038\\
}--cycle;


\addplot[area legend,solid,fill=white,draw=blue,forget plot]
table[row sep=crcr] {%
x	y\\
-0.600623353482293	0.0869602414130118\\
-0.528424553814436	0.12841504570038\\
-0.587832333246482	0.185786213081415\\
}--cycle;


\addplot[area legend,solid,fill=white,draw=blue,forget plot]
table[row sep=crcr] {%
x	y\\
-0.545046790820675	0.275134334708047\\
-0.466893636670726	0.289205659839999\\
-0.502261248394869	0.364482456334678\\
}--cycle;


\addplot[area legend,solid,fill=white,draw=blue,forget plot]
table[row sep=crcr] {%
x	y\\
-0.388740482520777	0.303276984971952\\
-0.502261248394869	0.364482456334678\\
-0.466893636670726	0.289205659839999\\
}--cycle;


\addplot[area legend,solid,fill=white,draw=blue,forget plot]
table[row sep=crcr] {%
x	y\\
-0.545046790820675	0.275134334708047\\
-0.587832333246482	0.185786213081415\\
-0.509679179096533	0.199857538213367\\
}--cycle;


\addplot[area legend,solid,fill=white,draw=blue,forget plot]
table[row sep=crcr] {%
x	y\\
-0.528424553814436	0.12841504570038\\
-0.509679179096533	0.199857538213367\\
-0.587832333246482	0.185786213081415\\
}--cycle;


\addplot[area legend,solid,fill=white,draw=blue,forget plot]
table[row sep=crcr] {%
x	y\\
-0.388740482520777	0.303276984971952\\
-0.353372870796634	0.228000188477273\\
-0.275219716646685	0.242071513609225\\
}--cycle;


\addplot[area legend,solid,fill=white,draw=blue,forget plot]
table[row sep=crcr] {%
x	y\\
-0.528424553814436	0.12841504570038\\
-0.469016774382391	0.0710438783193448\\
-0.450271399664487	0.142486370832332\\
}--cycle;


\addplot[area legend,solid,fill=white,draw=blue,forget plot]
table[row sep=crcr] {%
x	y\\
-0.466893636670726	0.289205659839999\\
-0.545046790820675	0.275134334708047\\
-0.488286407883629	0.244531599026683\\
}--cycle;


\addplot[area legend,solid,fill=white,draw=blue,forget plot]
table[row sep=crcr] {%
x	y\\
-0.466893636670726	0.289205659839999\\
-0.488286407883629	0.244531599026683\\
-0.431526024946584	0.21392886334532\\
}--cycle;


\addplot[area legend,solid,fill=white,draw=blue,forget plot]
table[row sep=crcr] {%
x	y\\
-0.509679179096533	0.199857538213367\\
-0.431526024946584	0.21392886334532\\
-0.488286407883629	0.244531599026683\\
}--cycle;


\addplot[area legend,solid,fill=white,draw=blue,forget plot]
table[row sep=crcr] {%
x	y\\
-0.509679179096533	0.199857538213367\\
-0.488286407883629	0.244531599026683\\
-0.545046790820675	0.275134334708047\\
}--cycle;


\addplot[area legend,solid,fill=white,draw=blue,forget plot]
table[row sep=crcr] {%
x	y\\
-0.466893636670726	0.289205659839999\\
-0.431526024946584	0.21392886334532\\
-0.41013325373368	0.258602924158636\\
}--cycle;


\addplot[area legend,solid,fill=white,draw=blue,forget plot]
table[row sep=crcr] {%
x	y\\
-0.466893636670726	0.289205659839999\\
-0.41013325373368	0.258602924158636\\
-0.388740482520777	0.303276984971952\\
}--cycle;


\addplot[area legend,solid,fill=white,draw=blue,forget plot]
table[row sep=crcr] {%
x	y\\
-0.353372870796634	0.228000188477273\\
-0.388740482520777	0.303276984971952\\
-0.41013325373368	0.258602924158636\\
}--cycle;


\addplot[area legend,solid,fill=white,draw=blue,forget plot]
table[row sep=crcr] {%
x	y\\
-0.509679179096533	0.199857538213367\\
-0.528424553814436	0.12841504570038\\
-0.47997528938051	0.17117195452285\\
}--cycle;


\addplot[area legend,solid,fill=white,draw=blue,forget plot]
table[row sep=crcr] {%
x	y\\
-0.509679179096533	0.199857538213367\\
-0.47997528938051	0.17117195452285\\
-0.431526024946584	0.21392886334532\\
}--cycle;


\addplot[area legend,solid,fill=white,draw=blue,forget plot]
table[row sep=crcr] {%
x	y\\
-0.450271399664487	0.142486370832332\\
-0.47997528938051	0.17117195452285\\
-0.528424553814436	0.12841504570038\\
}--cycle;


\addplot[area legend,solid,fill=white,draw=blue,forget plot]
table[row sep=crcr] {%
x	y\\
-0.41013325373368	0.258602924158636\\
-0.431526024946584	0.21392886334532\\
-0.392449447871609	0.220964525911296\\
}--cycle;


\addplot[area legend,solid,fill=white,draw=blue,forget plot]
table[row sep=crcr] {%
x	y\\
-0.41013325373368	0.258602924158636\\
-0.392449447871609	0.220964525911296\\
-0.353372870796634	0.228000188477273\\
}--cycle;


\addplot[area legend,solid,fill=white,draw=blue,forget plot]
table[row sep=crcr] {%
x	y\\
-0.312710466082493	0.0991865285832498\\
-0.137609858323343	0.121035756804613\\
-0.206414787485014	0.181553635206919\\
}--cycle;


\addplot[area legend,solid,fill=white,draw=blue,forget plot]
table[row sep=crcr] {%
x	y\\
-0.146695018173035	0.253587695583922\\
-0.275219716646685	0.242071513609225\\
-0.206414787485014	0.181553635206919\\
}--cycle;


\addplot[area legend,solid,fill=white,draw=blue,forget plot]
table[row sep=crcr] {%
x	y\\
-0.146695018173035	0.253587695583922\\
-0.206414787485014	0.181553635206919\\
-0.137609858323343	0.121035756804613\\
}--cycle;


\addplot[area legend,solid,fill=white,draw=blue,forget plot]
table[row sep=crcr] {%
x	y\\
-0.372118245514538	0.156557695964285\\
-0.312710466082493	0.0991865285832498\\
-0.293965091364589	0.170629021096237\\
}--cycle;


\addplot[area legend,solid,fill=white,draw=blue,forget plot]
table[row sep=crcr] {%
x	y\\
-0.206414787485014	0.181553635206919\\
-0.275219716646685	0.242071513609225\\
-0.293965091364589	0.170629021096237\\
}--cycle;


\addplot[area legend,solid,fill=white,draw=blue,forget plot]
table[row sep=crcr] {%
x	y\\
-0.206414787485014	0.181553635206919\\
-0.293965091364589	0.170629021096237\\
-0.312710466082493	0.0991865285832498\\
}--cycle;


\addplot[area legend,solid,fill=white,draw=blue,forget plot]
table[row sep=crcr] {%
x	y\\
-0.353372870796634	0.228000188477273\\
-0.323668981080612	0.199314604786755\\
-0.275219716646685	0.242071513609225\\
}--cycle;


\addplot[area legend,solid,fill=white,draw=blue,forget plot]
table[row sep=crcr] {%
x	y\\
-0.293965091364589	0.170629021096237\\
-0.275219716646685	0.242071513609225\\
-0.323668981080612	0.199314604786755\\
}--cycle;


\addplot[area legend,solid,fill=white,draw=blue,forget plot]
table[row sep=crcr] {%
x	y\\
-0.293965091364589	0.170629021096237\\
-0.323668981080612	0.199314604786755\\
-0.372118245514538	0.156557695964285\\
}--cycle;


\addplot[area legend,solid,fill=white,draw=blue,forget plot]
table[row sep=crcr] {%
x	y\\
-0.323668981080612	0.199314604786755\\
-0.353372870796634	0.228000188477273\\
-0.362745558155586	0.192278942220779\\
}--cycle;


\addplot[area legend,solid,fill=white,draw=blue,forget plot]
table[row sep=crcr] {%
x	y\\
-0.323668981080612	0.199314604786755\\
-0.362745558155586	0.192278942220779\\
-0.372118245514538	0.156557695964285\\
}--cycle;


\addplot[area legend,solid,fill=white,draw=blue,forget plot]
table[row sep=crcr] {%
x	y\\
-0.638996414189727	-0.209517673592198\\
-0.707106781186548	-0.707106781186547\\
-0.485472311835053	-0.570460153431915\\
}--cycle;


\addplot[area legend,solid,fill=white,draw=blue,forget plot]
table[row sep=crcr] {%
x	y\\
-0.342020143325669	-0.939692620785908\\
-0.263837842483558	-0.433813525677282\\
-0.485472311835053	-0.570460153431915\\
}--cycle;


\addplot[area legend,solid,fill=white,draw=blue,forget plot]
table[row sep=crcr] {%
x	y\\
-0.342020143325669	-0.939692620785908\\
-0.485472311835053	-0.570460153431915\\
-0.707106781186548	-0.707106781186547\\
}--cycle;


\addplot[area legend,solid,fill=white,draw=blue,forget plot]
table[row sep=crcr] {%
x	y\\
-0.3502012155183	-0.0436984564427255\\
-0.451417128336643	-0.32166559963474\\
-0.263837842483558	-0.433813525677282\\
}--cycle;


\addplot[area legend,solid,fill=white,draw=blue,forget plot]
table[row sep=crcr] {%
x	y\\
-0.485472311835053	-0.570460153431915\\
-0.263837842483558	-0.433813525677282\\
-0.451417128336643	-0.32166559963474\\
}--cycle;


\addplot[area legend,solid,fill=white,draw=blue,forget plot]
table[row sep=crcr] {%
x	y\\
-0.485472311835053	-0.570460153431915\\
-0.451417128336643	-0.32166559963474\\
-0.638996414189727	-0.209517673592198\\
}--cycle;


\addplot[area legend,solid,fill=white,draw=blue,forget plot]
table[row sep=crcr] {%
x	y\\
-0.613414373718105	-0.0118657302553914\\
-0.638996414189727	-0.209517673592198\\
-0.494598814854014	-0.126608065017462\\
}--cycle;


\addplot[area legend,solid,fill=white,draw=blue,forget plot]
table[row sep=crcr] {%
x	y\\
-0.451417128336643	-0.32166559963474\\
-0.3502012155183	-0.0436984564427255\\
-0.494598814854014	-0.126608065017462\\
}--cycle;


\addplot[area legend,solid,fill=white,draw=blue,forget plot]
table[row sep=crcr] {%
x	y\\
-0.451417128336643	-0.32166559963474\\
-0.494598814854014	-0.126608065017462\\
-0.638996414189727	-0.209517673592198\\
}--cycle;


\addplot[area legend,solid,fill=white,draw=blue,forget plot]
table[row sep=crcr] {%
x	y\\
-0.469016774382391	0.0710438783193448\\
-0.613414373718105	-0.0118657302553914\\
-0.481807794618202	-0.0277820933490584\\
}--cycle;


\addplot[area legend,solid,fill=white,draw=blue,forget plot]
table[row sep=crcr] {%
x	y\\
-0.494598814854014	-0.126608065017462\\
-0.3502012155183	-0.0436984564427255\\
-0.481807794618202	-0.0277820933490584\\
}--cycle;


\addplot[area legend,solid,fill=white,draw=blue,forget plot]
table[row sep=crcr] {%
x	y\\
-0.494598814854014	-0.126608065017462\\
-0.481807794618202	-0.0277820933490584\\
-0.613414373718105	-0.0118657302553914\\
}--cycle;


\addplot[area legend,solid,fill=white,draw=blue,forget plot]
table[row sep=crcr] {%
x	y\\
-0.312710466082493	0.0991865285832498\\
-0.409608994950345	0.0136727109383097\\
-0.3502012155183	-0.0436984564427255\\
}--cycle;


\addplot[area legend,solid,fill=white,draw=blue,forget plot]
table[row sep=crcr] {%
x	y\\
-0.481807794618202	-0.0277820933490584\\
-0.3502012155183	-0.0436984564427255\\
-0.409608994950345	0.0136727109383097\\
}--cycle;


\addplot[area legend,solid,fill=white,draw=blue,forget plot]
table[row sep=crcr] {%
x	y\\
-0.481807794618202	-0.0277820933490584\\
-0.409608994950345	0.0136727109383097\\
-0.469016774382391	0.0710438783193448\\
}--cycle;


\addplot[area legend,solid,fill=white,draw=blue,forget plot]
table[row sep=crcr] {%
x	y\\
-0.372118245514538	0.156557695964285\\
-0.390863620232442	0.0851152034512973\\
-0.312710466082493	0.0991865285832498\\
}--cycle;


\addplot[area legend,solid,fill=white,draw=blue,forget plot]
table[row sep=crcr] {%
x	y\\
-0.409608994950345	0.0136727109383097\\
-0.312710466082493	0.0991865285832498\\
-0.390863620232442	0.0851152034512973\\
}--cycle;


\addplot[area legend,solid,fill=white,draw=blue,forget plot]
table[row sep=crcr] {%
x	y\\
-0.409608994950345	0.0136727109383097\\
-0.390863620232442	0.0851152034512973\\
-0.469016774382391	0.0710438783193448\\
}--cycle;


\addplot[area legend,solid,fill=white,draw=blue,forget plot]
table[row sep=crcr] {%
x	y\\
-0.450271399664487	0.142486370832332\\
-0.469016774382391	0.0710438783193448\\
-0.420567509948464	0.113800787141815\\
}--cycle;


\addplot[area legend,solid,fill=white,draw=blue,forget plot]
table[row sep=crcr] {%
x	y\\
-0.390863620232442	0.0851152034512973\\
-0.372118245514538	0.156557695964285\\
-0.420567509948464	0.113800787141815\\
}--cycle;


\addplot[area legend,solid,fill=white,draw=blue,forget plot]
table[row sep=crcr] {%
x	y\\
-0.390863620232442	0.0851152034512973\\
-0.420567509948464	0.113800787141815\\
-0.469016774382391	0.0710438783193448\\
}--cycle;


\addplot[area legend,solid,fill=white,draw=blue,forget plot]
table[row sep=crcr] {%
x	y\\
-0.420567509948464	0.113800787141815\\
-0.372118245514538	0.156557695964285\\
-0.411194822589513	0.149522033398309\\
}--cycle;


\addplot[area legend,solid,fill=white,draw=blue,forget plot]
table[row sep=crcr] {%
x	y\\
-0.420567509948464	0.113800787141815\\
-0.411194822589513	0.149522033398309\\
-0.450271399664487	0.142486370832332\\
}--cycle;


\addplot[area legend,solid,fill=white,draw=blue,forget plot]
table[row sep=crcr] {%
x	y\\
-0.47997528938051	0.17117195452285\\
-0.450271399664487	0.142486370832332\\
-0.440898712305535	0.178207617088826\\
}--cycle;


\addplot[area legend,solid,fill=white,draw=blue,forget plot]
table[row sep=crcr] {%
x	y\\
-0.47997528938051	0.17117195452285\\
-0.440898712305535	0.178207617088826\\
-0.431526024946584	0.21392886334532\\
}--cycle;


\addplot[area legend,solid,fill=white,draw=blue,forget plot]
table[row sep=crcr] {%
x	y\\
-0.392449447871609	0.220964525911296\\
-0.401822135230561	0.185243279654802\\
-0.377597503013598	0.206621734066038\\
}--cycle;


\addplot[area legend,solid,fill=white,draw=blue,forget plot]
table[row sep=crcr] {%
x	y\\
-0.392449447871609	0.220964525911296\\
-0.377597503013598	0.206621734066038\\
-0.353372870796634	0.228000188477273\\
}--cycle;


\addplot[area legend,solid,fill=white,draw=blue,forget plot]
table[row sep=crcr] {%
x	y\\
-0.362745558155586	0.192278942220779\\
-0.353372870796634	0.228000188477273\\
-0.377597503013598	0.206621734066038\\
}--cycle;


\addplot[area legend,solid,fill=white,draw=blue,forget plot]
table[row sep=crcr] {%
x	y\\
-0.362745558155586	0.192278942220779\\
-0.377597503013598	0.206621734066038\\
-0.401822135230561	0.185243279654802\\
}--cycle;


\addplot[area legend,solid,fill=white,draw=blue,forget plot]
table[row sep=crcr] {%
x	y\\
-0.392449447871609	0.220964525911296\\
-0.431526024946584	0.21392886334532\\
-0.416674080088572	0.199586071500061\\
}--cycle;


\addplot[area legend,solid,fill=white,draw=blue,forget plot]
table[row sep=crcr] {%
x	y\\
-0.392449447871609	0.220964525911296\\
-0.416674080088572	0.199586071500061\\
-0.401822135230561	0.185243279654802\\
}--cycle;


\addplot[area legend,solid,fill=white,draw=blue,forget plot]
table[row sep=crcr] {%
x	y\\
-0.440898712305535	0.178207617088826\\
-0.401822135230561	0.185243279654802\\
-0.416674080088572	0.199586071500061\\
}--cycle;


\addplot[area legend,solid,fill=white,draw=blue,forget plot]
table[row sep=crcr] {%
x	y\\
-0.440898712305535	0.178207617088826\\
-0.416674080088572	0.199586071500061\\
-0.431526024946584	0.21392886334532\\
}--cycle;


\addplot[area legend,solid,fill=white,draw=blue,forget plot]
table[row sep=crcr] {%
x	y\\
-0.362745558155586	0.192278942220779\\
-0.401822135230561	0.185243279654802\\
-0.386970190372549	0.170900487809544\\
}--cycle;


\addplot[area legend,solid,fill=white,draw=blue,forget plot]
table[row sep=crcr] {%
x	y\\
-0.362745558155586	0.192278942220779\\
-0.386970190372549	0.170900487809544\\
-0.372118245514538	0.156557695964285\\
}--cycle;


\addplot[area legend,solid,fill=white,draw=blue,forget plot]
table[row sep=crcr] {%
x	y\\
-0.411194822589513	0.149522033398309\\
-0.372118245514538	0.156557695964285\\
-0.386970190372549	0.170900487809544\\
}--cycle;


\addplot[area legend,solid,fill=white,draw=blue,forget plot]
table[row sep=crcr] {%
x	y\\
-0.411194822589513	0.149522033398309\\
-0.386970190372549	0.170900487809544\\
-0.401822135230561	0.185243279654802\\
}--cycle;


\addplot[area legend,solid,fill=white,draw=blue,forget plot]
table[row sep=crcr] {%
x	y\\
-0.411194822589513	0.149522033398309\\
-0.401822135230561	0.185243279654802\\
-0.426046767447524	0.163864825243567\\
}--cycle;


\addplot[area legend,solid,fill=white,draw=blue,forget plot]
table[row sep=crcr] {%
x	y\\
-0.411194822589513	0.149522033398309\\
-0.426046767447524	0.163864825243567\\
-0.450271399664487	0.142486370832332\\
}--cycle;


\addplot[area legend,solid,fill=white,draw=blue,forget plot]
table[row sep=crcr] {%
x	y\\
-0.440898712305535	0.178207617088826\\
-0.450271399664487	0.142486370832332\\
-0.426046767447524	0.163864825243567\\
}--cycle;


\addplot[area legend,solid,fill=white,draw=blue,forget plot]
table[row sep=crcr] {%
x	y\\
-0.440898712305535	0.178207617088826\\
-0.426046767447524	0.163864825243567\\
-0.401822135230561	0.185243279654802\\
}--cycle;

\addplot [color=red,solid,forget plot]
  table[row sep=crcr]{%
0	0\\
0.5	0\\
};
\addplot [color=red,solid,forget plot]
  table[row sep=crcr]{%
0.5	0\\
1	0\\
};
\addplot [color=red,solid,forget plot]
  table[row sep=crcr]{%
1	0\\
0.90630778703665	0.422618261740699\\
};
\addplot [color=red,solid,forget plot]
  table[row sep=crcr]{%
0.90630778703665	0.422618261740699\\
0.642787609686539	0.766044443118978\\
};
\addplot [color=red,solid,forget plot]
  table[row sep=crcr]{%
0.642787609686539	0.766044443118978\\
0.258819045102521	0.965925826289068\\
};
\addplot [color=red,solid,forget plot]
  table[row sep=crcr]{%
0.258819045102521	0.965925826289068\\
-0.17364817766693	0.984807753012208\\
};
\addplot [color=red,solid,forget plot]
  table[row sep=crcr]{%
-0.17364817766693	0.984807753012208\\
-0.573576436351046	0.819152044288992\\
};
\addplot [color=red,solid,forget plot]
  table[row sep=crcr]{%
-0.573576436351046	0.819152044288992\\
-0.866025403784439	0.5\\
};
\addplot [color=red,solid,forget plot]
  table[row sep=crcr]{%
-0.996194698091746	0.0871557427476582\\
-0.939692620785908	-0.342020143325669\\
};
\addplot [color=red,solid,forget plot]
  table[row sep=crcr]{%
-0.939692620785908	-0.342020143325669\\
-0.707106781186548	-0.707106781186547\\
};
\addplot [color=red,solid,forget plot]
  table[row sep=crcr]{%
0.5	-0.866025403784439\\
0.25	-0.433012701892219\\
};
\addplot [color=red,solid,forget plot]
  table[row sep=crcr]{%
0.25	-0.433012701892219\\
0	0\\
};
\addplot [color=red,solid,forget plot]
  table[row sep=crcr]{%
-0.707106781186548	-0.707106781186547\\
-0.342020143325669	-0.939692620785908\\
};
\addplot [color=red,solid,forget plot]
  table[row sep=crcr]{%
-0.342020143325669	-0.939692620785908\\
0.0871557427476579	-0.996194698091746\\
};
\addplot [color=red,solid,forget plot]
  table[row sep=crcr]{%
0.0871557427476579	-0.996194698091746\\
0.5	-0.866025403784439\\
};
\addplot [color=red,solid,forget plot]
  table[row sep=crcr]{%
-0.866025403784439	0.5\\
-0.953716950748227	0.300705799504273\\
};
\addplot [color=red,solid,forget plot]
  table[row sep=crcr]{%
-0.953716950748227	0.300705799504273\\
-0.996194698091746	0.0871557427476582\\
};
\end{axis}
\end{tikzpicture}%
	   \caption{\Code{sectorg}, 7-fache lokale Verfeinerung um $(-0.4,0.2)$}
        \end{subfigure}
        \caption{Visualisierungen des Bisektionsalgorithmus}
        \label{fig:vis}
      \end{figure}
  \end{enumerate}
\end{exercise}

\end{document}
