\documentclass{myexercise}

\begin{document}

\begin{exercise}[Aufgabe 1]
	Sei $v \in C_0^\infty \subset \scr F$ beliebig.
	Da $\scr F \subset C^1(\Omega)$ ein linearer Unterraum ist, gilt $u + \eps v \in \scr F$ für alle $\eps \in \R$.
	Als Komposition differenzierbarer Funktionen ist $L\big(\nabla(u+\eps v)(x), (u+\eps v)(x), x\big)$ differenzierbar in $\eps$, es gilt
	\[
		\ddx[\eps] L\big(\nabla(u+\eps v)(x), u+\eps v(x), x\big)
		= \partial_p L\big(\nabla u(x), u(x), x\big) \cdot \nabla v(x) + \partial_z L\big(\nabla u(x), u(x), x\big) \cdot v(x).
	\]
	Obiger Ausdruck ist offenbar stetig und da $\supp \nabla v \subset \supp v$ kompakt, insbesondere majorisiert integrierbar über $\Omega$.
	Damit ist auch das Funktional $I(u+\eps v)$ differenzierbar in $\eps$.
	Da $I$ nach Annahme ein Minimum in $u$ hat, folgt zwangsläufig $\ddx[\eps] I(u+\eps v)\big|_{\eps = 0} = 0$, also (obiges Argument rechtfertigt folgende Vertauschung von Integral und Ableitung)
	\begin{align*}
		0
		&= \ddx[\eps] \int_\Omega L\Big(\nabla(u+\eps v)(x), u+\eps v(x), x\Big) \di[x] \Big|_{\eps=0} \\
		&= \int_\Omega \ddx[\epsilon] L\Big(\nabla(u+\epsilon v)(x), (u+\epsilon v)(x), x\Big) \Big|_{\epsilon=0} \di[x] \\
		&= \int_\Omega \partial_p L\big(\nabla u(x), u(x), x\big) \cdot \nabla v(x) + \partial_z L\big(\nabla u(x), u(x), x\big) \cdot v(x) \di[x] \\
		&= \int_\Omega \sum_{k=1}^d \big(\partial_p L(\nabla u, u, x)\big) \big(\partial_{x_i} v\big) + \partial_z L(\nabla u, u, x) \cdot v \di[x] \\
		&= \int_\Omega \sum_{k=1}^d -\partial_{x_i} \big(\partial_{p_i} L(\nabla u, u, x)\big) + \partial_{x_i}\big( v \partial_{p_i} L(\nabla u, u, x)\big) + v  \partial_z L(\nabla u, u, x) \di[x] \\
		&= \int_\Omega \bigg( - \sum_{i=1}^d \partial_{x_i} \big( \partial_{p_i} L(\nabla u, u, x) \big) + \partial_z L(\nabla u, u, x) \bigg) v(x) \di[x] + \sum_{i=1}^d \underbrace{\int_\Omega \partial_{x_i} \big(v \partial_{p_i} L(\nabla u, u, x) \big)}_{=0 \text{ s.u.}} \di[x]
	\end{align*}
	Nach dem Hauptsatz der Variationsrechnung, erfüllt $u$ also die DGL
	\[
		- \sum_{i=1}^d \partial_{x_i} \big( \partial_{p_i} L(\nabla u, u, x) \big) + \partial_z L(\nabla u, u, x) = 0.
	\]
	Es verbleibt $\int_\Omega \partial_{x_i}(v \partial_{p_i} L(\nabla u, u, x)) \di[x] = 0$ zu zeigen.
	Nun ist $\supp \partial_{x_i} v \subset \supp v \subset \Omega$ kompakt, also können wir den Integrationsbereich auf $\R$ erweitern.
	Nach Tonelli, können wir die Integrationsreihenfolge (der $x_i$) vertauschen.
	Da $\supp v$ kompakt, gilt
	\[
		\int_\R \partial_{x_i}(v \partial_{p_i} L(\nabla u, u, x)) \di[x_i]
		= \Big[v \partial_{p_i} L(\nabla u, u, x) \Big]_{-\infty}^{\infty}
		= 0,
	\]
	womit die Behauptung erfüllt ist
\end{exercise}

\newpage
\begin{exercise}[Aufgabe 2]
	\begin{enumerate}[(a)]
		\item
			$\Phi \in C^\infty(\Omega_T)$ nach c), man berechne also
			\begin{align*}
				\partial_t \Phi(x,t) &= \frac{-(4\pi)^{\f d2} \f d2 t^{\f d2 - 1}}{(4\pi t)^d} e^{-\f{\|x\|^2}{4t}} + \f 1{(4\pi t)}^{\f d2} e^{-\f{\|x\|^2}{4t}} \f{4 \|x\|^2}{(4t)^2} \\
					&= \f 1{4t^2} \Phi(x,t) ( \|x\|^2 - 2dt), \\
				\partial_{x_i} \Phi(x,t) &= - \f 1{2t} \Phi(x,t) x_i, \\
				\partial_{x_i}^2 \Phi(x,t) &= \f 1{4t^2} \Phi(x,t) (x_i - 2t).
			\end{align*}
			Nun sieht man sofort, dass
			\[
				\partial_t \Phi = \sum_{i=1}^d \partial_{x_i}^2 \Phi(x,t) = \Laplace_x \Phi(x,t).
			\]
		\item
			$\Phi$ ist als Komposition von $C^\infty(\Omega_T)$-Funktionen ebenfalls in $C^\infty(\Omega_T)$.
		\item
			Zunächst ist mit $u_i = \f{x_i}{2\sqrt t}$
			\[
				\int_\R e^{\f {x_i^2}{4t}} \di[x_i]
				= 2\sqrt t \int_\R e^{-u_i^2} \di[u_i]
				= 2 \sqrt {\pi t}.
			\]
			Damit folgt
			\begin{align*}
				\int_{\R^d} \Phi(x,t) \di[x]
				&= \int_{\R} \dotsi \int_\R \f 1{(4\pi t)^{\f d2}} e^{-\f{\|x\|^2}{4t}} \di[x_1] \di[x_2] \dotsc \di[x_d] \\
				&= \prod_{i=1}^d \f 1{2\sqrt{\pi t}} \int_\R e^{- \f {x_i^2}{4t}} \di[x_i] \\
				&= 1.
			\end{align*}
		\item
			Wir zeigen $\sup(\partial^\beta \Phi) < \infty$.
			Es gilt
			\begin{align*}
				\Phi(x,t) &= \f 1{(4\pi t)^{\f d2}} e^{-\frac{\|x\|^2}{4t}}, \\
				\partial_t \Phi(x,t) &= \f 1{4t^2} \Phi(x,t) \big( \|x\|^2 - 2dt \big), \\
				\partial_{x_i} \Phi(x,t) &= - \f 1{2t} \Phi(x,t) x_i.
			\end{align*}
			Anhand der Produktregel, kann man sich die Struktur der höheren Ableitungen vorstellen (Summanden welche $\Phi(x,t)$ enthalten zusammen mit Vorfaktoren $\|x\|^i, \f 1{t^j}$).

			Betrachte zunächst $x \in B_1(0), t > 1$: hier ist die Aussage klar, da alle Ableitungen einfach nach oben durch eine Konstante mal $\Phi(x,t)$ abgeschätzt werden können und
			\[
				|\Phi(x,t)| \le \f e{(4\pi t)^{\f d2}} \to 0
			\]
			für $t \to \infty$.

			Sei jetzt also $\|x\| \ge 1, t \le 1$, d.h. auch $|x_i| \le \|x\| \le \|x\|^2$ für alle $i$.
			Wir können nun die Summanden in $\partial^\beta \Phi$ nach oben hin abschätzen (anschaulich: beim Ableiten kommt schlimmstenfalls ein Term der Ordnung $\f{\|x\|^2}{t^2}$ hinzu, beachte $\|x\| \ge 1, t \le 1$):
			\begin{align*}
				|\partial^\beta \Phi(x,t)|
				&\le \big(\f{\|x\|}{t}\big)^{2|\beta|} C \Phi(x,t) \\
				&= \f 1{(4\pi t)^{\f d2}}\big(\f{\|x\|}{t}\big)^{2|\beta|} Ce^{-\f 14 \f{\|x\|^2}{t}} \\
				&\le \f 1{(4\pi \delta)^{\f d2}} \big(\f{\|x\|}{\delta}\big)^{2|\beta|} C e^{-\f{\|x\|^2}4}
				\xrightarrow{\|x\|\to\infty} 0,
			\end{align*}
			für eine Konstante $C$ (nur von $\beta$ abhängig).
	\end{enumerate}
\end{exercise}

\newpage
\begin{exercise}[Aufgabe 3]
	\begin{enumerate}[a)]
		\item
			Sei $L(\nabla w, w, x) = \sqrt{1 + \|\nabla w(x)\|^2}$, dann ist
			\begin{align*}
				\partial_{p_i} L(\nabla w, w, x) &= \f {\partial_{x_i} w}{\sqrt{1 + \|\nabla w\|^2}} \\
				\partial_{z} L(\nabla w, w, x) &= 0
			\end{align*}
			Nach Aufgabe 1 erfüllt ein Minimum $u$ von $I$ also
			\begin{align*}
				0 &= -\sum_{i=1}^d \partial_{x_i} \big( \partial_{p_i} L(\nabla u, u, x) \big) \\
				&= \div\Big( \f{\nabla u(x)}{\sqrt{1 + \|\nabla w\|^2}} \Big)
			\end{align*}
		\item
			Zunächst ist die Randbedingung erfüllt:
			Sei dazu $x \in \Boundary B_c(0)$, dann ist $\|x\| = c$, also
			\[
				u(x) = c \arcosh(\f{\|x\|}c)
				= c \arcosh(1)
				= 0.
			\]
			Berechne nun
			\begin{align*}
				\partial_{x_i} u(x)
					&= \f{c x_i}{\|x\| \sqrt{\|x\|^2 - c^2}}, \\
				\|\nabla u(x)\|^2
					&= \sum_{i=1}^d \big(\partial_{x_i} u(x)\big)^2
					= \f{c^2}{\|x\|^2 - c^2}, \\
				\f {1}{\sqrt{1 + \|\nabla u(x)\|^2}}
					&= \f{\sqrt{\|x\|^2 - c^2}}{\|x\|}, \\
				\partial_{x_i} \Big( \f{\nabla u(x)}{\sqrt{m + \|\nabla u(x)\|^2}} \Big)
					&= \partial_{x_i} \f{c x_i}{\|x\|^2}
					= \f c{\|x\|^2} \big( 1 - \f {2x_i^2}{\|x\|^2} \big).
			\end{align*}
			Damit ist
			\[
				\div\Big( \f{\nabla u(x)}{\sqrt{1 + \|\nabla w\|^2}} \Big)
				= \sum_{i=1}^d \partial_{x_i} \Big( \f{\nabla u(x)}{\sqrt{m + \|\nabla u(x)\|^2}} \Big)
				= \f{c}{\|x\|^2} (d - 2)
				= 0
			\]
			für $d = 2$.
	\end{enumerate}
\end{exercise}


\end{document}
