\chapter{Finite Volumen Verfahren für Erhaltungsgleichungen} \label{chap:5}



\begin{itemize}
    \item
        Wir betrachten das folgende Cauchy-Problem (AWP) für allgemeine skalare Erhaltungsgleichungen in Divergenzform.
        \begin{math}[numbered] \label{eq:5.1}
            \partial_t u + \Nabla_x f(u) &= 0  && \text{in $\Omega \times (0,T)$}
        \end{math}
        mit $\Omega = \R^d$, $T = \infty$, Flussfunktion $f: \R \to \R$ und Anfangsdaten $u(\argdot, 0) = u_0$ in $\Omega$.
    \item
        Bereits in §1 haben wir für die Burgers-Gleichung gesehen:
        die klassische Lösung existiert im Allgemeinen trotz glatter Anfangsdaten nicht für unendliche Zeiten, da klassische Lösungen entlang Charakteristiken konstant sind, Charakteristiken Geraden sind und sich schneiden können.
    \item
        Verallgemeinerungen von \ref{eq:5.1} zu Systemen, Randbedingungen, Quellterme, etc. sind möglich, werden jedoch nicht behandelt, siehe dazu Kröner 97.
        % fixme: ref, Numerical methods of consvervation laws
    \item
        Motivation:
        Erhaltungseigenschaft:
        Sei $u$ klassische Lösung, dann gilt für $\hat \Omega \subset \Omega$ beschränktes Lipschitzgebiet
        \begin{math}
            \ddx[t] \int_{\hat \Omega} u(x,t) \di[x]
             = \int_{\hat \Omega} \partial_t u(x,t) \di[x]
             = \int_{\hat \Omega} - \Nabla_x f(u)
             = - \int_{\Boundary \hat \Omega} f(x) \cdot n.
        \end{math}
        Die Änderung der Größe $u$ besteht aus dem, was aus dem Gebiet herausfließt.
        Falls $\int_{\Boundary \hat \Omega} f(u) \cdot n = 0$ (z.B. falls $f(0) = 0$ und $u(\argdot, t) \in C_0^0(\Omega), \supp(u(\argdot, t)) \subset \hat \Omega$, so ändert sich $\int_{\hat \Omega} u(\argdot, t)$ nicht, bleibt also erhalten.
\end{itemize}

\section{Lösungstheorie in 1D}

Wir wählen den Begriff der Distributionslösung als „schwache Lösung“.

\begin{df}[schwache Lösung des AWP] \label{5.1}
    Sei $u_0 \in L^\infty(\R)$, $f \in C^1(\R)$.
    Dann heißt $u \in L^\infty(\R \times \R^+)$ \emphdef{schwache Lösung} von \eqref{eq:5.1} genau dann, wenn
    \begin{math}
        \int_{\Omega_h} \int_0^T u \partial_t \phi + f(u) \partial_x \phi + \int_{\Omega} u_0 \phi(\argdot, 0)
        &= 0, && \forall \phi \in C_0^\infty(\Omega \times [0, T)).
    \end{math}
    \begin{note}
        \begin{itemize}
            \item
                Der Raum für $\phi$ soll nicht-null-Anfangswerte erlauben, z.B. Definition via
                \begin{math}
                    C_0^\infty(\Omega \times [0,T)) := \Set{\phi|_{\Omega \times [0,T)} & \phi \in C_0^\infty(\Omega \times (-\eps, T) }
                \end{math}
            \item
                Wir werden sehen, dass im Allgeimeinen Mehrdeutigkeit bei schwachen Lösungen vorliegt.
                Für Eindeutigkeit im Lösungsbegriff ist mehr erforderlich.
            \item
                Wie in §1 gilt für Distributionslösungen:
                \begin{itemize}
                    \item
                        klassische Lösung des AWP ist auch schwache Lösung,
                    \item
                        Falls $u$ schwache Lösung und $u \in C^1(\Omega \times (0,T))$, so ist $u$ auch klassische Lösung.
                \end{itemize}
                Man kann charakterisieren, wann eine stückweise klassische Lösung eine schwache Lösung darstellt.
        \end{itemize}
    \end{note}
\end{df}

\begin{st}[Rankin-Hugoniot] \label{5.2}
    Sei $\Omega \times (0,T)$ für $\Omega = \R$ zerlegt durch eine offene Kurve $S = \Set{(\sigma(t), t) & t \in \R^+}$ in linken und rechten Teil $M_l$, bzw. $M_r$.
    Sei $u: \Omega \times [0,T) \to \R$ und $u_l := u|_{M_l}$ und $u_r := u|_{M_r} \in C^1(\_{M_r})$ mit $u_l, u_r$ klassische Lösungen auf $\_M_l$, bzw. $\_M_r$.

    Dann gilt ist $u$ eine schwache Lösung genau dann, wenn
    \begin{math}[numbered] \label{eq:5.3}
        (u_l - u_r) \sigma' &= f(u_l) - f(u_r), && \text{auf $S$}.
    \end{math}
    \begin{note}
        \begin{itemize}
            \item
                \eqref{eq:5.3} heißt „\emphdef{Sprungbedingung}“ oder „\emphdef{Rankin-Hugoniot-Bedingung}“.
            \item
                $s(t) := \sigma'(t)$ nennen wir Geschwindigkeit der Unstetigkeit.
        \end{itemize}
    \end{note}
\end{st}
