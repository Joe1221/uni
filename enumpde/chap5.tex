\chapter{Finite Volumen Verfahren für Erhaltungsgleichungen} \label{chap:5}



\begin{itemize}
    \item
        Wir betrachten das folgende Cauchy-Problem (AWP) für allgemeine skalare Erhaltungsgleichungen in Divergenzform.
        \begin{math}[numbered] \label{eq:5.1}
            \partial_t u + \Nabla_x f(u) &= 0  && \text{in $\Omega \times (0,T)$}
        \end{math}
        mit $\Omega = \R^d$, $T = \infty$, Flussfunktion $f: \R \to \R$ und Anfangsdaten $u(\argdot, 0) = u_0$ in $\Omega$.
    \item
        Bereits in §1 haben wir für die Burgers-Gleichung gesehen:
        die klassische Lösung existiert im Allgemeinen trotz glatter Anfangsdaten nicht für unendliche Zeiten, da klassische Lösungen entlang Charakteristiken konstant sind, Charakteristiken Geraden sind und sich schneiden können.
    \item
        Verallgemeinerungen von \ref{eq:5.1} zu Systemen, Randbedingungen, Quellterme, etc. sind möglich, werden jedoch nicht behandelt, siehe dazu Kröner 97.
        % fixme: ref, Numerical methods of consvervation laws
    \item
        Motivation:
        Erhaltungseigenschaft:
        Sei $u$ klassische Lösung, dann gilt für $\hat \Omega \subset \Omega$ beschränktes Lipschitzgebiet
        \begin{math}
            \ddx[t] \int_{\hat \Omega} u(x,t) \di[x]
             = \int_{\hat \Omega} \partial_t u(x,t) \di[x]
             = \int_{\hat \Omega} - \Nabla_x f(u)
             = - \int_{\Boundary \hat \Omega} f(x) \cdot n.
        \end{math}
        Die Änderung der Größe $u$ besteht aus dem, was aus dem Gebiet herausfließt.
        Falls $\int_{\Boundary \hat \Omega} f(u) \cdot n = 0$ (z.B. falls $f(0) = 0$ und $u(\argdot, t) \in C_0^0(\Omega), \supp(u(\argdot, t)) \subset \hat \Omega$, so ändert sich $\int_{\hat \Omega} u(\argdot, t)$ nicht, bleibt also erhalten.
\end{itemize}

\section{Lösungstheorie in 1D}

Wir wählen den Begriff der Distributionslösung als „schwache Lösung“.

\begin{df}[schwache Lösung des AWP] \label{5.1}
    Sei $u_0 \in L^\infty(\R)$, $f \in C^1(\R)$.
    Dann heißt $u \in L^\infty(\R \times \R^+)$ \emphdef{schwache Lösung} von \eqref{eq:5.1} genau dann, wenn
    \begin{math}[numbered] \label{eq:5.2}
        \int_{\Omega_h} \int_0^T u \partial_t \phi + f(u) \partial_x \phi + \int_{\Omega} u_0 \phi(\argdot, 0)
        &= 0, && \forall \phi \in C_0^\infty(\Omega \times [0, T)).
    \end{math}
    \begin{note}
        \begin{itemize}
            \item
                Der Raum für $\phi$ soll nicht-null-Anfangswerte erlauben, z.B. Definition via
                \begin{math}
                    C_0^\infty(\Omega \times [0,T)) := \Set{\phi|_{\Omega \times [0,T)} & \phi \in C_0^\infty(\Omega \times (-\eps, T) }
                \end{math}
            \item
                Wir werden sehen, dass im Allgeimeinen Mehrdeutigkeit bei schwachen Lösungen vorliegt.
                Für Eindeutigkeit im Lösungsbegriff ist mehr erforderlich.
            \item
                Wie in §1 gilt für Distributionslösungen:
                \begin{itemize}
                    \item
                        klassische Lösung des AWP ist auch schwache Lösung,
                    \item
                        Falls $u$ schwache Lösung und $u \in C^1(\Omega \times (0,T))$, so ist $u$ auch klassische Lösung.
                \end{itemize}
                Man kann charakterisieren, wann eine stückweise klassische Lösung eine schwache Lösung darstellt.
        \end{itemize}
    \end{note}
\end{df}

\begin{st}[Rankin-Hugoniot] \label{5.2}
    Sei $\Omega \times (0,T)$ für $\Omega = \R$ zerlegt durch eine offene Kurve $S = \Set{(\sigma(t), t) & t \in \R^+}$ in linken und rechten Teil $M_l$, bzw. $M_r$.
    Sei $u: \Omega \times [0,T) \to \R$ und $u_l := u|_{M_l}$ und $u_r := u|_{M_r} \in C^1(\_{M_r})$ mit $u_l, u_r$ klassische Lösungen auf $\_M_l$, bzw. $\_M_r$.

    Dann gilt ist $u$ eine schwache Lösung genau dann, wenn
    \begin{math}[numbered] \label{eq:5.3}
        (u_l - u_r) \sigma' &= f(u_l) - f(u_r), && \text{auf $S$}.
    \end{math}
    \begin{note}
        \begin{itemize}
            \item
                \eqref{eq:5.3} heißt „\emphdef{Sprungbedingung}“ oder „\emphdef{Rankin-Hugoniot-Bedingung}“.
            \item
                $s(t) := \sigma'(t)$ nennen wir Geschwindigkeit der Unstetigkeit.
        \end{itemize}
    \end{note}
\Timestamp{2015-02-06}
    \begin{proof}
        Sei $u$ schwache Lösung von \eqref{eq:5.2}, d.h. für alle $\phi \in C_0^\infty(\Omega)$
        \begin{math}
            0 &= \int_\Omega \int_0^T u \partial_t \phi + f(u) \partial_x \phi + \int_\Omega u_0 \phi(\argdot,0) \\
            \iff 0 &= \underbrace{\int_{M_l} u_l \partial_t \phi + f(u_l) \partial_x \phi}_{=\Vector{f(u_l)&u_l} \cdot \Vector{\partial_x \phi & \partial_t \phi}} \\
            &\quad+ \int_{M_r} u_r \partial_t \phi + f(u_r) \partial_x \phi + \int_\Omega u_0 \phi(\argdot, 0) \\
            \iff 0 &= - \int_{M_l} \Vector{\partial_x & \partial_t} \cdot \Vector{f(u_l) & u_l} \phi
            + \int_0^T \phi \Vector{f(u_l) & u_l} n(t) \\
            &\quad+ \int_{-\infty}^{\sigma(0)} \phi(\argdot, 0) \underbrace{\Vector{f(u_l) & u_l} n_0}_{=-u_l}
            - \int_{M_r} \Vector{\partial_x & \partial_t} \cdot \Vector{f(u_r) & u_r} \phi \\
            &\quad+ \int_0^T \phi \Vector{f(u_r) & u_r} (-n(t))
            + \int_{\sigma(0)}^{\infty} \phi(\argdot, 0) \underbrace{\Vector{f(u_r) & u_r} n_0}_{=-u_r} \\
            &\quad+ \int_\Omega u_0 \phi(\argdot, 0) \\
            \stackrel{\text{$u_r,u_l$ klass. Lös.}}\iff 0 &= \int_0^T \phi \Vector{f(u_l) - f(u_r) & u_l - u_r} \underbrace{n(t)}_{= \Vector{1 & -\sigma'(t)}} \\
            \iff 0 &= f(u_l) - f(u_r) - (u_l - u_r) \sigma'.
        \end{math}
    \end{proof}
\end{st}

\begin{ex*}[Fehlende Eindeutigkeit]
    Betrachte die Burgers-Gleichung
    \begin{math}
        \partial_t u + \partial_x (\frac{1}{2} u^2) &= 0 && \text{auf $\R \times (0, \infty)$} \\
        u(\argdot, 0) &= u_0 := \begin{cases}
            0 & x \le 0 \\
            1 & x > 0
        \end{cases}.
    \end{math}
    Setze
    \begin{math}
        u_1(x,t) &:= \begin{cases}
            0 & x < \frac{t}{2} \\
            1 & x \ge \frac{t}{2}.
        \end{cases}, &
        u_2(x,t) &:= \begin{cases}
            0 & x \le 0 \\
            \frac{x}{t} & 0 \le x < t \\
            1 & x \ge t
        \end{cases}.
    \end{math}
    Für $u_1$ folgt mit $\sigma(t) = \frac{t}{2}$, dass $u_1$ konstant auf $M_l, M_r$, also $\partial_t u + \partial_x f(u) = 0$.
    $u_1(x,0) = 0 = u_0$, analog für $u_r$, also sind $u_l, u_r$ klassische Lösungen.

    Die Rankin-Hugoniot-Sprungbedingung gilt:
    \begin{math}
        f(u_l) - f(u_r) &= \frac{1}{2} 0^2 - \frac{1}{2} 1^2 = -\frac{1}{2} \\
        (u_l - u_r) \sigma' &= (0-1) \frac{1}{2} = -\frac{1}{2},
    \end{math}
    also ist $u_1$ schwache Lösung mit \ref{5.2}.
    $u_2$ ist ebenso klassische Lösung auf drei Teilgebiete, z.B. im mittleren Teil:
    \begin{math}
        \partial_t(\frac{x}{t}) + \partial_x(\frac{1}{2} (\frac{x}{t})^2)
        = - \frac{x}{t^2} + \frac{1}{2t^2} 2x = 0.
    \end{math}
    Die Rankin-Hugoniot-Bedingung gilt, $\sigma(t) = 0$, denn $u_2$ ist stetig, analog für $\sigma(t) = t$, also ist $u_2$ auch schwache Lösung.
\end{ex*}

\begin{note}
    \begin{itemize}
        \item
            Man benötigt also noch weitere Bedingungen, um eine eindeutige Lösung zu erhalten.
            Idee: Bezeichne das zu lösende Problem mit $(P_0)$.
            Definiere ein modifiziertes Problem $(P_\epsilon)$ mit eindeutiger Lösung $u_\epsilon$ für $\epsilon > 0$ und setze $u := \lim_{\epsilon\to 0} u_\epsilon$.
        \item
            Motivation anhand eines LGS:
            $(P_0)$, gegeben durch
            \begin{math}
                x + y &= 0 \\
                x + y &= 0
            \end{math}
            hat unendlich viele Lösungen.
            Definiere $(P_\epsilon)$ durch
            \begin{math}
                x_\epsilon + y_\epsilon &= \epsilon \\
                x_\epsilon + (1-\epsilon) y_\epsilon &= \epsilon a.
            \end{math}
            Dieses Problem hat eine eindeutige Lösung:
            \begin{math}
                y_\epsilon &= 1 - a \\
                x_\epsilon &= \epsilon - 1 + a.
            \end{math}
            Durch Grenzwertbildung $(y_\epsilon, x_\epsilon) \to (1 - a, a - 1)$ erhalten wir eine ausgezeichnete Lösung von $(P_0)$.
        \item
            Nun ist ein physikalisch relevantes $(P_\epsilon)$ gefragt.
    \end{itemize}
\end{note}

\begin{st}[Viskositätslimes] \label{5.3}
    Sei $\Omega = \R^d, T = \infty$, $u_0 \in L^\infty(\Omega) \cap L^1(\Omega) \cap C^0(\Omega), f \in C^2(\R, \R^d)$, sodass $\|D^2 f\| \le M$.
    Dann existiert für jedes $\epsilon > 0$ eine eindeutige klasssische Lösung $u_\epsilon$ von
    \begin{math}
        \partial_t u_\epsilon + \div_x \cdot f(u) &= \epsilon \Laplace u_\epsilon && \text{in $\Omega \times (0,T)$}, \\
        u_\epsilon(\argdot, 0) &= u_0
    \end{math}
    und $u_\epsilon$ konvergiert fast überall gegen eine Funnktion $u: \Omega \times (0, T)$, welche schwache Lösung von
    \begin{math}
        \partial_t u + \div_x \cdot f(u) &= 0 && \text{in $\Omega \times (0,T)$}, \\
        u(\argdot, 0) &= u_0
    \end{math}
    ist.
    Wir nennen $u$ \emphdef{Viskositätslimes}.
    \begin{proof}
        Siehe Kroner, 97, Thm 2.1.7 und dortige Referenz.
    \end{proof}
    \begin{note}
        \begin{itemize}
            \item
                Wegen Eindeutigkeit des Grenzwertes ist der Viskositätslimes eindeutig.
                Gesucht ist eine „nachprüfbare“ Eigenschaft, welche $u$ charakterisiert.
        \end{itemize}
    \end{note}
\end{st}

\begin{df}[Entropiepaar] \label{5.4}
    Sei $\Omega = \R$, $U, F \in C^2(\R)$, sodass $F' = U' f'$ und $U$ konvex.
    Dann nennen wir $(U, F)$ ein \emphdef{Entropiepaar} mit \emphdef{Entropie} $U$ und \emphdef{Entropiefluss} $F$ für die Gleichung $\partial_t u + \partial_x f(u) = 0$
\end{df}

\begin{ex*}
    \begin{itemize}
        \item
            Sei $f$ konvex.
            Dann ist $U(s) := f(s), F(s) := \int_0^s (f'(r))^2 \di[r]$ ein Entropiepaar, denn $U$ ist konvex und
            \begin{math}
                U' f' = f'(s) f'(s) = F'(s).
            \end{math}
        \item
            Allgemeiner: Sei $U$ eine konvexe Funktion, $F(s) := \int_0^s u'(r) f'(r) \di[r]$.
            Dann ist $(U, F)$ ein Entropiepaar.
    \end{itemize}
\end{ex*}

\paragraph{Motivation: Entropiebedingung}

Sei $u_\epsilon \in C^2(\R \times \R^+)$ Lösung von $\partial_t u_\epsilon + \partial_x f(u_\epsilon) = \epsilon \partial_x^2 u_\epsilon$.
Dann folgt mit der Definition von $u_\epsilon$ und $(U, F)$
\begin{math}
    &\partial_t U(u_\epsilon) + \partial_x F(u_\epsilon) - \epsilon \partial_x^2 U(u_\epsilon) \\
    &\quad= U'(u_\epsilon) \partial_t u_\epsilon + \underbrace{F'(u_\epsilon) \partial_x u_\epsilon}_{=U'(a_\epsilon) f'(u_\epsilon \partial_x u_\epsilon} - \underbrace{\epsilon \partial_x( U'(u_\epsilon) \partial_x u_\epsilon)}_{=U''(u_\epsilon) (\partial_x u_\epsilon)^2 + U'(u_\epsilon) \partial_x^2 u_\epsilon} \\
    &\quad= U'(u_\epsilon)\underbrace{( \partial_t u_\epsilon + \partial_x f(u_\epsilon) - \epsilon \partial_x^2 u_\epsilon)}_{=0} - \underbrace{\epsilon}_{>0} \underbrace{U''(u_\epsilon)}_{\ge0}\underbrace{(\partial_x u_\epsilon)^2}_{\ge 0} \\
    &\quad\le 0
\end{math}
Für alle $\phi \in C_0^\infty(\R \times (0,T)), \phi \ge 0$ folgt
\begin{math}
    \int_\R \int_{\R^+} (-U(u_\epsilon) \partial_t \phi - F(u_\epsilon) \partial_x \phi - \epsilon u(u_\epsilon) \partial_x^2 \phi
    \le 0
\end{math}
Falls (ist beweisbar) $u_\epsilon \to u, U(u_\epsilon) \to U(u), F(u_\epsilon) \to F(u)$ in $L^1$ und $|U(u_\epsilon)| \le \const$ erhalten wir für $\epsilon \to 0$
\begin{math}
    \int_\R \int_{\R^+} U(u) \partial_t \phi + F(u) \partial_x \phi \ge 0
\end{math}
für alle $\phi \in C_0^\infty(\Omega \times (0,T)), \phi \ge 0$.
Dies nennen wir „$\partial_t U(u) + \partial_x F(u) \le 0$“ im distributionellen Sinne, oder \emphdef{Entropieungleichung}.

\begin{df}[Entropielösung] \label{5.5}
    Eine schwache Lösung $u$ von \ref{5.2} heißt \emphdef{Entropielösung}, falls für alle Entropiepaare $(U, F)$ gilt:
    \begin{math}[numbered] \label{eq:5.4}
        \int_\R \int_{\R^+} U(u) \partial_t \phi + F(u) \partial_x \phi + \int_\R U(u_0) \phi(\argdot, 0) \ge 0
    \end{math}
    für alle $\phi \in C_0^\infty(\R \times [0,T)), \phi \ge 0$.
    \begin{note}
        \begin{itemize}
            \item
                Man kann zeigen, dass die Entropielösung eindeutig ist.
            \item
                Für schwache Lösungen, welche entlang einer Kurve $S$ unstetig sint, stellt die Lax-Entropiebedingung ein konkret nachprüfbares Kriterium dar.
                Es ist für $f$ konvex sogar äquivalent zu \ref{eq:5.4}.
        \end{itemize}
    \end{note}
\end{df}

\begin{st}[Lax-Entropiebedingung] \label{5.6}
    Sei $f'' \ge 0$ (also $f$ konvex), $u$ schwache Lösung mit Unstetigkeit entlang einer Kurve $S = \Set{(\sigma(t), t) & t \in (0,T) }$ und klassische Lösung auf den Teilgebieten.
    Sei $(U, F)$ ein Entropiepaar und $u$ erfülle \eqref{eq:5.4}.
    Dann gilt entlang $S$
    \begin{math}
        f'(u_l) &> s > f'(u_r), && s := \sigma'(t).
    \end{math}
    Wir nennen $S$ einen \emphdef{Schock} und $s := \frac{f(u_l) - f(u_r)}{u_l - u_r}$ die \emphdef{Schockgeschwindigkeit}.
    \begin{note}
        Siehe Thm 2.1.12 in Kroner. % fixme: ref
    \end{note}
    \begin{note}[Anschauung]
        \begin{itemize}
            \item
                Charakteristiken müssen in eine Unstetigkeit hineinlaufen, nicht hinaus.
                % fixme: Bild
            \item
                Für konvexe Flussfunktion $f$.
                % fixme: Bild
                $u_l < u_r$ erfüllt nicht die Lax-Entropie-Bedingung, für $u_r < u_l$ schon.
                Eine Lösunug kann über einen Schock also nur abnehmen.

                Betrachte das Beispiel zur fehlenden Eindeutigkeit, $f(u) = \frac{1}{2} u^2$.
                $u_1$ ist schwache Lösung, aber keine Entropielösung.

                Bei geänderte Anfangsbedingung $u_0 = \Ind_{\R_{\le 0}}$ ist die Lösung $u_3(x,t) = \Ind_{x < \frac{t}{2}}$.
                Die Rankin-Hugoniot-Sprungbedingung ist erfüllt: $f(u_l) - f(u_r) = \frac{1}{2}, (u_l - u_r) \sigma' = \frac{1}{2}$ also ist $u_3$ schwache Lösung.
                Wegen
                \begin{math}
                    f'(u_l) = 1 > \frac{1}{2} = s > 0 = f'(u_r)
                \end{math}
                ist die Lax-Entropiebedungung erfüllt und $u_3$ ist Entropielösung.
        \end{itemize}
    \end{note}
\end{st}


\section{Finite-Volumenverfahren in 1D}

\begin{itemize}
    \item
        Sei $\Omega = \R, \Delta x \in \R^+, x_i := i \Delta x, 2i \in \Z, h := \Delta X$.
        Sei $\scr T_h := \Set{(x_{i-\frac{1}{2}}, x_{i + \frac{1}{2}}) & i \in \Z}$ uniforme Zerlegung von $\Omega$.
    \item
        Ziel ist eine Approximation der Entropielösung $u(x,t)$ von $\eqref{eq:5.1}$ in einem Orts-Ansatzraum stückweise konstantere Funktionen.
        \begin{math}
            V_h := \Set{v: \Omega \to \R & v = \sum_{i\in\Z} v_i \Ind_{(x_{i-\frac{1}{2}}, x_{i+\frac{1}{2}})}, v_i \in \R}
        \end{math}
        mit Indikatorfunktion $\Ind_A$ von $A \subset \R$ zu diskreten Zeiten $0 = t^0 < t^1 < \dotsb < t^K = T$ mit $t^k = \Delta t k$.
    \item
        Gesucht ist $u_h = \Set{u_h^k}_{k=0}^K \in V_h^{k+1}$ mit $u_i^k := u_h^k(x_i) \approx(x_i, t^k)$.
    \item
        Die Idee des Verfahrens beruht auf das Erhaltungsprinzip.
        Betrachet das Rechteck (oder Referenzvolumen, Finite Volumen)
        \begin{math}
            V := (x_{i-\frac{1}{2}}, x_{i+\frac{1}{2}}) \times (t^k, t^{k+1}).
        \end{math}
        Integration der PDE über $V$:
        \begin{math}
            0 &= \int_V \partial_t u + \partial_x f(u)
            = \int_V \Vector{\partial_x & \partial_t} \Vector{f(u) & u}
            = \int_{\partial V} \Vector{f(u) & u} \cdot n \\
            &= \int_{t^k}^{t^{k+1}} f(u(x_{i+\frac{1}{2}}, t)) \di[t] -
\int_{t^k}^{t^{k+1}} f(u(x_{i-\frac{1}{2}}, t)) \di[t] \\
&\quad + \underbrace{\int_{x_{i-\frac{1}{2}}}^{x_{i+\frac{1}{2}}} f(u(x, t^{k+1})) \di[t]}_{\approx \Delta x u_i^{k+1}} -
\underbrace{\int_{x_{i-\frac{1}{2}}}^{x_{i+\frac{1}{2}}} f(u(x, t^{k})) \di[t]}_{\approx \Delta x u_i^k}
        \end{math}
        Der Fluss $f(u(x_{i+\frac{1}{2}},t))$ wird approximiert durch eine \emphdef{numerische Flussfunktion} $g: \R \times \R \to \R$, welche nur von den $u$-Werten links und rechts von $x_{i+\frac{1}{2}}$ abhängen:
        \begin{math}
            f(u(x_{i+\frac{1}{2}},t)) \approx g(\underbrace{u(x_i,t)}_{\approx u_i^k}, \underbrace{u(x_{i+1}, t)}_{\approx u_{i+1}^k}).
        \end{math}
        Damit approximieren wir z.B. mittels Vorwärts-Rechtecksintegration:
        \begin{math}
            \int_{t^k}{t^{k+1}} f(u(x_{i+\frac{1}{2}}, t)) \approx \Delta t g(u_i^k, u_{i+1}^k) =: g_{i+\frac{1}{2}}^k,
        \end{math}
        analog für das zweite Integral.
        \begin{math}
            0 = \Delta t (g_{i+\frac{1}{2}}^k - g_{i-\frac{1}{2}}^k) + \Delta x(u_i^{k+1} - u_i^k).
        \end{math}
\end{itemize}

\Timestamp{2015-02-10}

\begin{df}[FV-Verfahren, 1D] \label{5.7}
    Sei $u_0 \in L^1(\R), g \in C^0(\R^2, \R)$ und $\scr T_h$ Zerlegung von $\Omega = \R, t^k := \Delta t k$, $k= 0, \dotsc, K$.
    Dann ist $u_h^k = \sum_{i\in\Z} u_i^k \Ind_{(x_{i-\f 12}, x_{i+\f 12})}$ mit
    \begin{math}
        u_i^0 &:= \frac{1}{\Delta x} \int_{x_{i-\frac{1}{2}}}^{x_{i+\frac{1}{2}}} u_0(x) \di[x],
        u_i^{k+1} &:= u_i^k - \frac{\Delta t}{\Delta x} (g_{i+\frac{1}{2}} - g_{i-\frac{1}{2}}^k)
    \end{math}
    für $g_{i+\frac{1}{2}}^k := g(u_i^k, u_{i+1}^k)$ die \emphdef{FV-Approximation} von \eqref{5.1}.
    \begin{note}
        \begin{itemize}
            \item
                Obige Definition ist ein explizites FV-Verfahren.
            \item
                Falls bei Fluss-Approximation „Rückwärts-Rechtecks-Regel“ gewählt wird:
                \begin{math}
                    \int_{t^k}^{t^{k+1}} f(u(x_{i+\frac{1}{2},t})) \di[t]
                    \approx \Delta t g(u_i^{k+1}, u_{i+1}^{k+1}),
                \end{math}
                so erhalten wir ein implizites FV-Verfahren.
            \item
                Erweiterung auf nicht-uniforme Gitter, nichtkonstante Zeitschrittweiten oder unstruckturierte Gitter in 2D oder 3D ist einfach (ganz analog, im Gegensatz zu FD).
        \end{itemize}
    \end{note}
\end{df}

\begin{df}[Spezielle Numerische Flüsse Ⅰ] \label{5.8}
    Wir definieren
    \begin{itemize}
        \item
            $g(u,v) = f(u)$, \emphdef{Rückwärts-Differenz}, \emphdef{Upwind-Fluss},
        \item
            $g(u,v) = f(v)$, \emphdef{Vorwärts-Differenz}, \emphdef{Downwind-Fluss},
        \item
            $g(u,v) = \frac{1}{2}(f(u) + f(v))$, \emphdef{zentrale Differenz}.
    \end{itemize}
    \begin{note}
        \begin{itemize}
            \item
                Das FV-Verfahren für die Rückwärts-Differenz lautet
                \begin{math}
                    u_i^{k+1} = u_i^k - \frac{\Delta t}{\Delta x} (f(u_i^k) - f(u_{i-1}^k)),
                \end{math}
                entspricht also tatsächlich einer FD-Approximation
                \begin{math}
                    \partial_x f(u)(x_i) \approx \frac{f(u_i^k) - f(u_{i-1}^k}{\Delta x}.
                \end{math}
            \item
                Solche einfachen Flüsse sind nur bedingt sinnvoll:
                In \ref{chap:2} haben wir bereits gesehen: zentrale Differenzen mit Advektion ist nicht stabil.
            \item
                Am Beispiel der Burgers-Gleichung kann man sehen, dass zentrale Differenzen, auch falls stabil, nicht unbedingt die Entropie-Lösung approximieren, sondern eine andere schwache Lösung.

                Sei $f(u) = \frac{1}{2} u^2, g(u,v) := \frac{1}{2}(f(u) + f(v))$,
                \begin{math}
                    u_0(x) = \begin{cases}
                        -1 & x < - \frac{\Delta x}{2} \\
                        1 & x \ge - \frac{\Delta x}{2}
                    \end{cases}.
                \end{math}
                Es folgt damit
                \begin{math}
                    u_i^0 = \begin{cases}
                        -1 & i < 0 \\
                        1 & i \ge 0
                    \end{cases}.
                \end{math}
                Es gilt
                \begin{math}
                    g(1,1) &= \frac{1}{2}(f(1) + f(1)) = \frac{1}{2}(\frac{1}{2} + \frac{1}{2}) = \frac{1}{2}, \\
                    g(-1,-1) &= \frac{1}{2}, \\
                    g(-1,1) &= \frac{1}{2},
                \end{math}
                also
                \begin{math}
                    u_i^{k+1} = u_i^k - \frac{\Delta t}{\Delta x} (\underbrace{g(u_i^k, u_{i+1}^k)}_{=\frac{1}{2}} - \underbrace{g(u_{i-1}^k, u_i^k)}_{=\frac{1}{2}}) = u_i^k.
                \end{math}
                Damit ist $u_i^k = u_i^0$ für $k = 0, \dotsc, K$ ist FV-Approxiamiton.
                Wir wissen, dass
                \begin{math}
                    u(x,t) = \begin{cases}
                        -1 & x < -\frac{\Delta x}{2} \\
                        1 & x \ge -\frac{\Delta x}{2}
                    \end{cases}.
                \end{math}
                schwache Lösung ist (Runkin-Hugoniot ist erfüllt), aber keine Entropielösung, denn die Lax-Entropiebedingunge ist veletzt.
            \item
                Bei anderen Problemen kann die Entropielösung exakt approximiert werden:
                betrachte dazu den Upwind-Fluss mit Advektion:
                sei $f(u) = au$, $a > 0$ mit PDE $\partial_t u + a \partial_x u = 0$.
                Wir wissen aus \ref{chap:1}:
                zu $u_0 \in C^1(\R)$ ist
                \begin{math}
                    u(x,t) = u_0(x-at)
                \end{math}
                eindeutige klassische Lösung, also auch Entropielösung.
                Hier ist der Upwind-Fluss das Richtige.
                Sei $g(u,v) = f(u) = au, \Delta t = \frac{1}{a} \Delta x$, dann ergibt sich
                \begin{math}
                    u_i^{k+1} &= u_i^k - \frac{\Delta t}{\Delta x} (f(u_i^k) - f(u_{i-1}^k)) \\
                    &= u_i^k - \frac{1}{a} (a u_i^k - a u_{i-1}^k) \\
                    &= u_i^k - u_i^k + u_{i-1}^k \\
                    &= u_{i-1}^k.
                \end{math}
                Falls $u_0(x_i) = u_i^0$ (z.B. falls $u_0$ stückweise konstant) folgt per Induktion
                \begin{math}
                    u_i^k
                    = u_0(u_i - at^k)
                    = u(x_i, t^k),
                \end{math}
                also eine exakte Approximation.
        \end{itemize}
    \end{note}
\end{df}

\begin{df}[Spezielle Numerische Flüsse, Ⅱ] \label{5.9}
    Wir definieren den \emphdef{Lax-Friedrichs-Fluss}
    \begin{math}
        g(u,v)
        = \frac{1}{2}(f(u) + f(v)) + \frac{1}{2\lambda}(u-v), && \lambda = \frac{\Delta t}{\Delta x},
    \end{math}
    den \emphdef{Engquist-Osher-Fluss}
    \begin{math}
        g(u,v) &:= f^+(u) + f^-(v), \\
        f^+(y) &:= f(0) + \int_0^y \max \{f'(s), 0\} \di[s], \\
        f^-(y) &:= \int_0^y \min \{f'(s), 0\} \di[s].
    \end{math}
    \begin{note}
        \begin{itemize}
            \item
                Für den Lax-Friedrich-Fluss lautet das FV-Verfahren
                \begin{math}
                    u_i^{k+1} = u_i^k - \frac{\Delta t}{2\Delta x} (f(u_{i+1}^k) - f(u_{i-1}^k)) + \frac{1}{2}(u_{i+1}^k - 2u_i^k + u_{i-1}).
                \end{math}
                Der Ausdruck besteht aus einer zentralen Differenz als Approximation von $\partial_x f(u)$ und einen zusätzlichen Term, den wirr als (skalierte) Approximation von $\partial_x^2 u$ kennen (vgl. \ref{chap:2}).
                Diesen letzten Term nennen wir \emphdef{numerische Viskosität} oder \emphdef{numerische Dämpfung}, er spielet dieselbe Rolle, wie $\epsilon \partial_x^2 u$ im Viskositätslimes und garantiert, dass $u_h$ gegen die Entropielösung konvergiert, Bewies folgt.
            \item
                Zum Engquist-Osher-Fluss: es wird je nach Richtung des Transportes ein Upwind- oder ein Downwind-Fluss gewählt:
                Betrachte die Burgers-Gleichung, $f(u) = \frac{1}{2} u^2, f'(u) = u$.
                Falls $u,v > 0$:
                \begin{math}
                    f^+(u) &= 0 + \int_0^u s \di[s] = \frac{1}{2} u^2 = f(u), &
                    f^-(u) &= \int_0^v 0 \di[s] = 0,
                \end{math}
                also $g(u,v) = f^+(u) + f^-(u) = f(u)$, ein Upwind-Fluss.
                Falls $u,v < 0$ ergibt sich
                \begin{math}
                    f^+(u) &= 0, &
                    f^-(u) &= \int_0^v s \di[s] = \frac{1}{2} v^2 = f(v),
                \end{math}
                also $g(u,v) = f(v)$, ein Downwind-Fluss.
            \item
                Wir nennen $g(\argdot, \argdot)$ einen \emphdef[konsistenter Fluss]{konsistenten Fluss}, falls $g(u,u) = f(u)$ für alle $u$.
                Alle Flüsse aus \ref{5.8}, \ref{5.9} sind konsistent.
        \end{itemize}
    \end{note}
\end{df}


\begin{st}[Erhaltungseigenschaft] \label{5.10}
    Das FV-Verfahren ist erhaltend, d.h. für $u_0 \in L^1(\R)$ gilt
    \begin{math}
        \int_\R u_h^k(x) \di[x] = \int_\R u_0(x) \di[x]
    \end{math}
    für $k = 0, \dotsc, K$.
    \begin{proof}
        Für $k = 0, \dotsc, K$
        \begin{math}
            \int_\R u_h^k(u) \di[x]
            &= \int_\R \sum_{i\in\Z} u_i^k \Ind_{(x_{i-\frac{1}{2}}, x_{i+\frac{1}{2}})}(x) \di[x] \\
            &= \sum_{i\in\Z} u_i^k \underbrace{\int_\R \Ind_{(x_{i-\frac{1}{2}}, x_{i+\frac{1}{2}})}(x) \di[x]}_{=\Delta x} \\
            &= \Delta x \sum_{i\in \Z} u_i^k
        \end{math}
        Für $k = 0$ gilt nach Definition
        \begin{math}
            \int_\R u_h^0 = \Delta x \sum_{i\in\Z} u_i^0
            = \sum_{i\in\Z} \int_{x_{i-\frac{1}{2}}}^{x_{i+\frac{1}{2}}} u_0(x)
            = \int_\R u_0(x) \di[x].
        \end{math}
        Für $k = 0, \dotsc, K -1$ gilt
        \begin{math}
            \int_\R u_h^{k+1}(x) \di[x]
            &= \Delta x \sum_{i\in\Z} u_i^{k+1} \\
            &= \Delta x \sum_{i\in\Z} u_i^k \underbrace{- \Delta t \sum_{i\in\Z} g(u_i^k, u_{i+1}^k) + \Delta t \sum_{i\in\Z} g(u_{i-1}^k, u_i^k)}_{=0} \\
            &= \int_\R u_h^k(x) \di[x].
        \end{math}
        Somit folgt die Behauptung per Induktion.
    \end{proof}
\end{st}

\begin{st}[Konsistenz] \label{5.11}
    Sei $g$ konsistenter numerischer Fluss, $g \in C^2$ mit $\|\partial^\beta g\| \le C$, $u \in C^2$ klassische Lösung von \eqref{5.1} mit $\|\partial^\beta u\| \le C$ für $0 \le |\beta| \le 2$.
    Sei $Q_i^k: C^2(\R \times \R^+) \to \R$ definiert durch
    \begin{math}
        Q_i^k v := v(x_i, t^k) - \frac{\Delta t}{\Delta x} \Big(g\big(v(x_i,t^k), v(x_{i+1}, t^k)\big) - g\big(v(x_{i-1}, t^k), v(x_i, t^k)\big)\Big).
    \end{math}
    Sei $\Delta t \le \alpha \Delta x$ für ein $\alpha > 0$.
    Dann gilt für den lokalen Abschneidefehler $L_i^k := \frac{1}{\Delta t} (u(x_i, t^{k+1}) - Q_i^k u)$
    \begin{math}
        |L_i^k| \le C (\Delta x + \Delta t),
    \end{math}
    das Verfahren ist also konsisten von Ordnung 1.
    \begin{proof}
        Analog zu \ref{chap:2}, oder siehe Kröner, Lemma 2.2.4.
        %fixme: ref
    \end{proof}
    \begin{note}
        \begin{itemize}
            \item
                Man kann zeigen: falls $u$ glatte Lösung von
                \begin{math}
                    \partial_t u + \partial_x f(u) = \Delta x \partial_x (b(u) \partial_x u)
                \end{math}
                mit $b(u) = \frac{1}{2} (\partial_1 g(u,u) - \partial_2 g(u,u) - \frac{\Delta t}{\Delta x} f'(u)^2)$, dann ist das FV-Verfahren konsisten mit Ordnung 2,
                \begin{math}
                    |L_i^k| \le C(\Delta x^2 + \Delta t^2),
                \end{math}
                d.h. falls $b(u) \ge 0$, dann kann die rechte Seite als Diffusion aufgefasst werden.
                Mit $\Delta x \to 0$ verschwindet die rechte Seite, somit imitiert das FV-Verfahren den Viskositätslimes und es kann Konvergenz gegen Entropielösung erwartet werden.
        \end{itemize}
    \end{note}
\end{st}

Wir wollen jetzt Konvergenz gegen eine schwache Lösung, bzw. gegen eine Entropielösung untersuchen.
Dazu seien $(h_m)_{m\in\N}, (k_m)_{m\in\N}$ Nullfolgen mit $\frac{h_m}{k_m} = \const$.
Für $m \in \N$ setze $\Delta x := h_m$, $\Delta t := k_m$ und setze die FV-Approximation stückweise konstant fort auf $\R \times \R^+$ via
\begin{math}
    u_m(x,t) := \sum_{i,k} u_i^k \Ind_{[x_{i-\frac{1}{2}}, x_{i+\frac{1}{2}})}(x) \Ind_{[t^k, t^{k+1})}(t)
\end{math}

\begin{df}[Diskrete Entropiebedingung] \label{5.12}
    Sei $(U, F)$ ein Entropiepaar und $G \in C^0(\R^2)$ konsistenter numerischer Entropiefluss, d.h. $G(u,u) = F(u)$.
    Die FV-Approximation erfüllt die \emphdef{diskrete Entropiebedingung} genau dann, wenn
    \begin{math}[numbered] \label{eq:5.6}
        U(u_i^{k+1}) - U(u_i^k)
        \le -\frac{\Delta t}{\Delta x} (G_{i+\frac{1}{2}}^k - G_{i-\frac{1}{2}}^k),
    \end{math}
    für $G_{i+\frac{1}{2}}^k := G(u_i^k, u_{i+1}^k)$.

    Das Verfahren heißt \emphdef{konsistent} mit der Entropiebedingung falls \eqref{eq:5.6} für alle $(U,F)$ und $\Delta t, \Delta x \to 0$ gilt.
\end{df}

\begin{df}[Totalvariation] \label{5.13}
    Für $f: [a,b] \to \R$ ist die \emphdef{Totalvariation} definiert durch
    \begin{math}
        \TV_{[a,b]}(f) := \sup_{\substack{a=z_0 < \dotsc < z_n = b \\ n \in \N}} \sum_{i=0}^{n-1} |f(z_{i+1} - f(z_i)|.
    \end{math}
    \begin{note}
        \begin{itemize}
            \item
                Für $f$ monoton wachsend gilt da $|f(z_{i+1} - f(z_i)| = f(z_{i+1} - f(z_i)$
                \begin{math}
                    \TV_{[a,b]}(f) = f(b) - f(a).
                \end{math}
            \item
                für diskrete Fuktion $u_h^k$ (stückweise konstant) ist
                \begin{math}
                    \TV_{[a,b]}(u_h^k) = \sum_{i\in I} |u_{i+1}^k - u_i^k|
                \end{math}
                mit
                \begin{math}
                    I := \Set{i \in \Z & x_i \in (a+\frac{\Delta x}{2}, b - \frac{3}{2} \Delta x)}.
                \end{math}
        \end{itemize}
    \end{note}
\end{df}

\begin{st}[Hinreichende Bedingung zur Konvergenz] \label{5.14}
    Sei $u_0 \in L^1(\R) \cap L^\infty(\R)$, $(u_m)_{m\in \N}$ eine Folge von stückweise konstanten FV-Approximationen zu $\Delta x = h_m, \Delta t = k_m$.
    Sei $T > 0$ und $[a,b]$ so dass $\supp u_0 \subset [a,b]$ und $\supp u_h^k \subset [a,b]$ für alle $k = 0, \dotsc, K$, wobei $K := \floor{\frac{T}{\Delta t}}$.

    Falls
    \begin{enumerate}[i)]
        \item
            $g \in C^0(\R^2)$ konsisten und Lipschitz-stetig,
        \item
            $\sup_{x\in [a,b]} |u_m(x,t^k)| \le M_0$ für alle $m, k$,
        \item
            $\TV_{[a,b]} |u_m(\argdot, t)| \le M_1$ für alle $m, k$,
    \end{enumerate}
    dann existiert eine Teilfolge $(u_m')_{m\in\N}$ und ein $u \in L^1(\R \times [0,T])$ mit $u_m \to u$ in $L^1_{\text{locc}}(\R \times [0,T])$ und $u$ ist schwache Lösung.

    Ist das Verfahren weiter konsistent mit Entropiebedingung, so ist $u$ eindeutige Entropielösung.
    \begin{proof}
        Siehe Kröner 2.3.9.
    \end{proof}
\end{st}

\Timestamp{2015-02-13}

\begin{df}[Monotone Verfahren] \label{5.15}
    Das Verfahren
    \begin{math}
        u_i^{k+1}
        := H(\dotsc, u_{i-1}^k, u_i^k, u_{i+1}^k, \dotsc)
    \end{math}
    heißt \emphdef{monoton}, falls $H$ monoton steigend in allen Argumenten ist.
\end{df}

\begin{st}[Monotonie für Lax Friedrich und Engquist Osher] \label{5.16}
    Ein FV-Verfahren mit $g \in C^1(\R^2)$ ist monoton genau dann, wenn
    \begin{enumerate}[i)]
        \item
            $g(u,v)$ monoton wachsend bezüglich $u$ und monoton fallend bezüglich $v$, d.h. für alle $u,v$
            \begin{math}
                \partial_1 g(u,v) &\ge 0,
                \partial_2 g(u,v) &\le 0.
            \end{math}
        \item
            Für alle $u,v,w$ gilt $1 - \frac{\Delta t}{\Delta x} (\partial_1 g(u,v) - \partial_2 g(v,w)) \ge 0$.
    \end{enumerate}
    Insbesondere liefern der Lax-Friedrich- und Engquist-Osher-Fluss ein monotones Verfahren, falls
    \begin{math}
        \Delta t \le C_{\text{CFL}} \Delta x,
    \end{math}
    wobei $C_{\text{CFL}} := \frac{1}{2 L}$ mit Lipschitzkonstante $L$ von $g$.
    \begin{proof}
        %Es gilt für die Iteration
        %\begin{math}
        %    u_i^{k+1}
        %    &= u_i^k - \frac{\Delta t}{\Delta x}(g(u_i^k, u_{i+1}^k) - g(u_{i-1}^k, u_i^k)) \\
        %    &= u_i^k - \frac{\Delta t}{\Delta x}(g(u_i^k, u_i^k) + (u_{i+1}^k - u_i^k) \partial_2 g(u_i^k, \xi) - g(u_i^k, u_i^k - (u_{i-1}^k - u_i^k) \partial_1 g(\_\xi, u_i^k))) \\
        %    &= \Big(1 + \frac{\Delta t}{\Delta x} \partial_2 g(u_i^k, \xi) - \frac{\Delta t}{\Delta x} \partial_1 g(\_\xi, u_i^k)\Big) u_i^k - \frac{\Delta t}{\Delta x} \partial_2 g(u_i^k, \xi) u_{i+1}^k + \frac{\Delta t}{\Delta x} \partial_1 g(\_\xi, u_i^k) u_{i-1}^k \\
        %    &=:H(u_{i-1}^k, u_i^k, u_{i+1}^k)
        %\end{math}
        %Folglich ist $H$ monoton wachsend, wenn die geforderten Bedingungen erfüllt sind.
        %Andere Richtung via Gegenannahme: gilt i) oder ii) nicht, finde $u_i^k, u_{i+1}^k, u_{i-1}^k$, sodass $H$ nicht monoton in $(u_{i-1}^k, u_i^k, u_{i-1}^k)$ ist.

        Alternativ (vermutlich klarer): zeige $\pddx[u_i^k] H \ge 0$ mit $H$ aus dem FV-Verfahren.

        Zu Lax-Friedrich:
        \begin{math}
            g(u,v) &:= \frac{1}{2}(f(u) + f(v)) + \frac{1}{2 \lambda} (u - v), &
            \lambda &= \frac{\Delta t}{\Delta x}.
        \end{math}
        Sei $\Delta t \le C_{\text{CFL}} \Delta x$, d.h. $\lambda \le C_{\text{CFL}}$ mit $C_{\text{CFL}} = \frac{1}{2L}$.
        Wegen $f'(w) = \ddx[w]f(w) = \ddx[w] g(w,w) = \partial_1 g(w,w) + \partial_2 g(w,w)$ gilt
        \begin{math}
            |f'(w)| \le |\partial_1 g(w,w)| + |\partial_2 g(w,w)|
            \le 2 L.
        \end{math}
        Damit gilt
        \begin{math}
            \partial_1 g(u,v) &= \frac{1}{2} f'(u) + \frac{1}{2\lambda} \\
            &\ge - \frac{1}{2}|f'(u)| + \frac{1}{2\lambda} \\
            &\ge - \frac{1}{2}\max_{w}|f'(w)| + \frac{1}{2\lambda} \\
            &\ge - L + \frac{1}{2\lambda} \\
            &\ge - L + \frac{1}{2C_{\text{CFL}}} \\
            &\ge 0.
        \end{math}
        Analog zeigt man $\partial_2 g(u,v) \le 0$.
        Weiter ist
        \begin{math}
            1 - \frac{\Delta t}{\Delta x} (\partial_1 g(u,v) - \partial_2 g(v,w))
            &= 1 - \lambda \partial_1 g(u,v) + \lambda \partial_2 g(v,w) \\
            &\ge 1 - \lambda \big( |\partial_1 g(u,v)| + |\partial_2 g(u,v)| \big) \\
            &\ge 1 - 2 \lambda L \\
            &\ge 0.
        \end{math}
        Die Bedingungen i) und ii) sind also für Lax-Friedrich erfüllt.

        Die Bedingungen lassen sich auch für Engquist-Osher zeigen.
    \end{proof}
    \begin{note}
        Falls $g$ nicht differenzierbar, so lässt sich Monotonie auch über FD-Quotienten charakterisieren, d.h. das Verfahren ist monoton genau dann, wenn
        \begin{enumerate}[i)]
            \item
                für alle $u,v$
                \begin{math}
                    \frac{g(u,v) - g(\_u, v)}{u - \_u} &\ge 0, &
                    \frac{g(u,v) - g(u, \_v)}{v - \_v} &\le 0
                \end{math}
            \item
                für alle $u,v,w$
                \begin{math}
                    1 - \frac{\Delta t}{\Delta x} \Big( \frac{g(u,v) - g(\_u, v)}{u - \_u} - \frac{g(v,w) - g(v,\_w)}{w - \_w} \Big) \ge 0
                \end{math}
        \end{enumerate}
    \end{note}
\end{st}

\begin{st}[$L^\infty$-Stabilität] \label{5.17}
    Für monotone FV-Verfahren und $u_0 \in L^\infty(\R)$ gilt mit $\Delta t \le C_{\text{CFL}} \Delta x$, $C_{\text{CFL}} = \frac{1}{2L}$.
    \begin{math}
        \|u_h^k\|_{L^\infty}
        \le \|u_0\|_{\infty}.
    \end{math}
    \begin{proof}
        Nutze Induktion:
        Für $k = 0$ gilt
        \begin{math}
            |u_i^0| = \Big| \frac{1}{\Delta x} \int_{x_{i-\frac{1}{2}}}^{x_{i+\frac{1}{2}}} u_0 \Big|
            \le \frac{1}{\Delta x} \Delta x \|u_0\|_{L^\infty(x_{i-\frac{1}{2}}, x_{i+\frac{1}{2}})}
            \le \|u_0\|_{L^\infty(\R)}.
        \end{math}
        Induktionsschritt:
        Zeige ein lokales Maximumsprinzip, d.h.
        \begin{math}[numbered] \label{eq:5.7}
            \min\Set{u_{i-1}^k, u_i^k, u_{i+1}^k}
            \le u_i^k
            \le \max\Set{u_{i-1}^k, u_i^k, u_{i+1}^k}.
        \end{math}
        Dann folgt
        \begin{math}
            \|u_h^{k+1}\|
            = \max_{i} |u_i^{k+1}|
            \le \max_i |u_i^k|
            = \|u_h^k\|_{L^\infty}
            \le \|u_0\|_{L^\infty}.
        \end{math}
        Für \eqref{eq:5.7} genügt es zu zeigen, dass $u_i^{k+1}$ Konvexkombination von $u_{i-1}^k, u_i^k, u_{i+1}^k$ ist, d.h.
        \begin{math}
            u_i^{k+1} = \lambda_{i-1} u_{i-1}^k + \lambda_i u_i^k + \lambda_{i+1} u_{i+1}^k
        \end{math}
        mit $\lambda_{i-1}, \lambda_i, \lambda_{i+1} \in [0,1], \lambda_{i-1} + \lambda_i + \lambda_{i+1} = 1$.
        Dann folgt \eqref{eq:5.7}, denn sei $m := \max \Set{\lambda_{i-1}, \lambda_i, \lambda_{i+1}}$, dann ist
        \begin{math}
            u_i^{k+1}
            \le \lambda_{i-1} m + \lambda_i m + \lambda_{i+1} m
            = m
        \end{math}
        und analog für die untere Schranke in \eqref{eq:5.7}.

        $u_i^{k+1}$ ist Konvexkombination, da
        \begin{math}
            u_i^{k+1}
            &= u_i^k - \frac{\Delta t}{\Delta x} (g(u_i^k, u_{i+1}k) - g(u_i^k, u_i^k) + g(u_i^k, u_i^k) - g(u_{i-1}^k, u_i^k)) \\
            &= u_i^k - \underbrace{\frac{\Delta t}{\Delta x} \frac{g(u_i^k, u_{i+1}^k) - g(u_i^k, u_i^k)}{u_{i+1}^k - u_i^k}}_{c_{i+\frac{1}{2}}^k} (u_{i+1}^k - u_i^k) - \underbrace{\frac{\Delta t}{\Delta x} \frac{g(u_i^k, u_i^k) - g(u_{i-1}^k, u_i^k)}{u_i^k - u_{i-1}^k}}_{=d_{i-\frac{1}{2}}^k} (u_i^k - u_{i-1}^k) \\
            &= u_i^k + c_{i+\frac{1}{2}}^k (u_{i+1}^k - u_i^k) - d_{i-\frac{1}{2}} (u_i^k - u_{i-1}^k)
            &= (1 - c_{i+\frac{1}{2}}^k - d_{i-\frac{1}{2}}^k) u_i^k + c_{i+\frac{1}{2}}^k u_{i+1}^k + d_{i-\frac{1}{2}}^k u_{i-1}^k.
        \end{math}
        Es genügt nun zu zeigen, dass $c_{i+\frac{1}{2}}^k, d_{i-\frac{1}{2}}^k \in [0, \frac{1}{2}]$, dann ist dies eine Konvexkombination.
        Wegen $g$ monoton fallend im zweiten Argument folgt
        \begin{math}
            \frac{g(u_i^k, u_{i+1}^k) - g(u_i^k, u_i^k)}{u_{i+1}^k - u_i^k} \le 0,
        \end{math}
        also $c_{i+\frac{1}{2}}^k \ge 0$.
        Wegen der CFL-Bedingung ist
        \begin{math}
            c_{i+\frac{1}{2}}^k
            \le |c_{i+\frac{1}{2}}^k|
            \le \frac{\Delta t}{\Delta x} L
            \le \frac{\frac{1}{2L} \Delta x}{\Delta x} L
            = \frac{1}{2},
        \end{math}
        analog für $d_{i-\frac{1}{2}}^k$.
    \end{proof}
\end{st}

\begin{st}[TVD-Eigenschaft] \label{5.18}
    Ein monotones Verfahren mit CFL-Bedingung $\Delta t \le C_{\text{CFL}} \Delta x, C_{\text{CFL}} = \frac{1}{2L}$ ist \emphdef{total variation diminishing (TVD)}, d.h.
    \begin{math}
        \TV_{[a,b]}(u_h^{k+1}) \le \TV_{[a,b]}(u_h^k),
    \end{math}
    falls $\supp u_h^k \subset [a,b]$.
    \begin{proof}
        Wie im vorigen Beweis gilt für geeignete $c_{i+\frac{1}{2}}^k, d_{i-\frac{1}{2}}^k \in [0, \frac{1}{2}]$
        \begin{math}
            u_i^{k+1} = u_i^k + c_{i+\frac{1}{2}} (u_{i+1}^k - u_i^k) - d_{i-\frac{1}{2}}^k (u_i^k - u_{i-1}^k).
        \end{math}
        Damit folgt
        \begin{math}
            \TV_{[a,b]}(u_h^{k+1})
            &\le \sum_{i \in \Z} |u_i^{k+1} - u_{i-1}^{k+1}| \\
            &= \sum_{i\in\Z} \Big| u_i^k - u_{i-1}^k + c_{i+\frac{1}{2}}^k (u_{i+1}^k - u_i^k) - c_{i -\frac{1}{2}}^k (u_i^k - u_{i-1}^k) \\
            &\qquad - d_{i-\frac{1}{2}} (u_i^k - u_{i-1}^k) + d_{i -\frac{3}{2}} (u_{i-1}^k - u_{i-2}^k) \Big| \\
            &= \sum_{i\in\Z} \Big| \underbrace{(1 - c_{i-\frac{1}{2}}^k - d_{i-\frac{1}{2}}^k)}_{\ge 0} (u_i^k - u_{i-1}^k) \\
            &\qquad + \underbrace{c_{i+\frac{1}{2}}}_{\ge 0} (u_{i+1}^k - u_i^k) + \underbrace{d_{i-\frac{3}{2}}}_{\ge 0} (u_{i-1}^k - u_{i-2}^k) \Big| \\
            &\le \sum_{i\in\Z} (1 - c_{i+\frac{1}{2}}^k - d_{i-\frac{1}{2}}^k ) |u_i^k - u_{i-1}^k| \\
            &\qquad + \sum_{i-\frac{1}{2}} c_{i+\frac{1}{2}}^k |u_{i+1}^k - u_i^k| + \sum_{i\in\Z} d_{i-\frac{3}{2}} |u_{i-1}^k - u_{i-2}^k| \\
            &= \sum_{i\in\Z} |u_i^k - u_{i-1}^k| - \sum_{i\in\Z} c_{i-\frac{1}{2}}^k |u_i^k - u_{i-1}^k| - \sum_{i\in\Z} d_{i-\frac{1}{2}} |u_i^k - u_{i-1}^k| \\
            &\qquad+ \sum_{i\in\Z} c_{i+\frac{1}{2}} |u_{i+1}^k - u_i^k| + \sum_{i\in\Z} d_{i-\frac{3}{2}} |u_{i-1}^k - u_{i-2}^k| \\
            &= \sum_{i\in\Z} |u_i^k  - u_{i-1}^k| \\
            &= \TV_{[a,b]}(u_h^k).
        \end{math}
    \end{proof}
\end{st}

\begin{st}[Konsistenz mit E.B.] \label{5.19}
    Ein monotones FV-Verfahren ist konsistent mit der Entropiebedingung.
    \begin{proof}
        Siehe Skript, zeige
        \begin{math}
            U(u_i^{k+1}) - U(u_i^k)
            \le - \frac{\Delta t}{\Delta x} (G_{i+\frac{1}{2}}^k - G_{i-\frac{1}{2}}^k).
        \end{math}
    \end{proof}
\end{st}

\begin{kor}[Konvergenz] \label{5.20}
    Für monotone FV-Verfahren gilt Konvergenz in $L^1_{\text{loc}}$ gegen die eindeutige Entropielösung, falls $u_0 \in L^1(\R) \cap L^\infty(\R)$ mit $\supp u_0 \subset [a,b]$ und $\TV_{[a,b]}(u_0) < \infty$ und $\Delta t := k_m \le C_{\text{CFL}} h_m$ für $C_{\text{CFL}} = \frac{1}{2L}$ mit Lipschitz-konstante $L$ von $g$.
    \begin{proof}
        Die hinreichende Bedingungen zur Konvergenz in \ref{5.14} sind erfüllt.
    \end{proof}
\end{kor}

\begin{note}
    \begin{itemize}
        \item
            Man kann a-priori-Fehleraussagen der Form
            \begin{math}
                \|u(\argdot, t) - u_h(\argdot, t)\|_{L^1} \le C h^{\frac{1}{2}}
            \end{math}
            zeigen.
        \item
            Man kann zeigen, dass die Konsistenzordnung von monotonen Verfahren höchstens 1 ist.
        \item
            Man kann Verfahren höherer Ordnung konstruieren, indem man lokal um einen Schock ein monotones Verfahren erster Ordnung verwendet und in glatten Bereichen FD-Flüsse höherer Ordnung definiert.
    \end{itemize}
\end{note}

\begin{ex*}[Numerisches Beispiel, \Code{demo\textunderscore burgers(3)}]
    Sei $f(u) = \frac{1}{2} u^2$.
    Betrachte ein FV-Verfahren mit Engquist-Osher-Fluss.

    % Hütchen von -1 bis 1 der Höhe 3

    Geschwindigkeit der Spitze ist $\gamma' = f'(u) = u = 3$.
    \begin{itemize}
        \item
            Massenerhaltung
        \item
            $u(x,t) \in [0,3]$ wegen diskretem Maximumsprinzip
        \item
            R.H. Bedingung ist erfüllt:
            \begin{math}
                (u_l - u_r) \sigma' = (u_l - 0) p'(t),
            \end{math}
            $f(u_l) - f(u_r) = \frac{1}{2} u_l^2$, es folgt $p'(t) = \frac{1}{2} u_l$, etwa halbierte Geschwindigkeit.
        \item
            Lax-Entropie-Bedingung gilt wegen $u_l > 0$.
    \end{itemize}
\end{ex*}

\begin{note}[Sommer-Vorlesung]
    \begin{itemize}
        \item
            Weiterführende Aspekte,
        \item
            Reduzierte-Basis-Methoden,
        \item
            kein Aufbau auf diese Vorlesung.
    \end{itemize}
\end{note}
