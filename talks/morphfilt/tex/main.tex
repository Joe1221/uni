\documentclass{beamer}
\usepackage{fontspec}
\usepackage{xparse}
\usepackage{mymath}

\usetheme{Luebeck}

\DeclareDocumentCommand{\dilate}{}{\oplus}
\DeclareDocumentCommand{\erode}{}{\ominus}

% http://tex.stackexchange.com/questions/117117/plus-minus-sign-in-a-circle
\newcommand{\opm}{
  \mathbin{
    \mathchoice
      {\buildcirclepm{\displaystyle     }{0.14ex}{0.95}{0.05ex}{.7}}
      {\buildcirclepm{\textstyle        }{0.14ex}{0.95}{0.05ex}{.7}}
      {\buildcirclepm{\scriptstyle      }{0.13ex}{0.955}{0.04ex}{.55}}
      {\buildcirclepm{\scriptscriptstyle}{0.08ex}{0.95}{0.03ex}{.45}}
  }
}
\newcommand{\omp}{
  \mathbin{
    \mathchoice
      {\buildcirclemp{\displaystyle     }{0.14ex}{0.95}{0.05ex}{.7}}
      {\buildcirclemp{\textstyle        }{0.14ex}{0.95}{0.05ex}{.7}}
      {\buildcirclemp{\scriptstyle      }{0.13ex}{0.955}{0.04ex}{.55}}
      {\buildcirclemp{\scriptscriptstyle}{0.08ex}{0.95}{0.03ex}{.45}}
  }
}
\newcommand\buildcirclepm[5]{%
  \begin{tikzpicture}[baseline=(X.base), inner sep=-#5, outer sep=-.65]
    \node[draw,circle,line width=#4] (X)  {\footnotesize\raisebox{#2}{\scalebox{#3}{$#1\pm$}}};
  \end{tikzpicture}%
}
\newcommand\buildcirclemp[5]{%
  \begin{tikzpicture}[baseline=(X.base), inner sep=-#5, outer sep=-.65]
    \node[draw,circle,line width=#4] (X)  {\footnotesize\raisebox{#2}{\scalebox{#3}{$#1\mp$}}};
  \end{tikzpicture}%
}

\DeclareDocumentCommand{\dilateerode}{}{\opm}
\DeclareDocumentCommand{\erodedilate}{}{\omp}


\begin{document}


\section{Einführung}

\subsection{Beispiele und Motivation}

\begin{frame}
    \frametitle{Test!}
    Hallo, das ist ein Test.
\end{frame}


\subsection{Definitionen und Notationen}

\begin{frame}
    \frametitle{Bilder}
    \begin{definition}
        Ein \emph{Bild} ist eine Abbildung $u: \Omega \to F$. \pause
        Dabei sei
        \begin{enumerate}[1.]
            \item
                die \emph{Trägermenge} $\Omega$ entweder $\R^d$ oder $\Z^d$ \pause und
            \item
                der \emph{Farbraum} $F$ bestehend aus Grauwerten, speziell: $F = [0,1]$.
        \end{enumerate}
    \end{definition}
\end{frame}

\begin{frame}
    \frametitle{Erweiterung der Trägermenge für endliche Bilder}
\end{frame}

\begin{frame}
    \frametitle{Strukturelemente}
    \begin{definition}
        Ein \emph{Strukturelement} ist eine Teilmenge $B \subset \Omega$. \\ \pause
        Wir visualisieren $B$ als Binärbild $\Ind_B: \Omega \to F$.
    \end{definition}
\end{frame}

\begin{frame}
    \frametitle{Elementare Operationen und Relationen auf Bildern}
    \begin{definition}
        Für ein Bild $u: \Omega \to F$ definiere
        \begin{itemize}
            \item \pause
                die \emph{Translation} $u^p(x) := u(x-p)$ um $p \in \Omega$ \pause und
            \item
                das \emph{Komplement} $(\complement u)(x) := 1 - u(x)$.
        \end{itemize}
        \pause
        Für eine Menge $U$ aus Bildern $\Omega \to F$ definiere
        \begin{itemize}
            \item \pause
                die \emph{Vereinigung} $(\bigvee_{u \in U} u)(x) := \sup \Set{u(x) & u \in U}$ \pause und
            \item
                den \emph{Schnitt} $(\bigwedge_{u \in U} u)(x) := \inf \Set{u(x) & u \in U}$.
        \end{itemize}
        \pause
        Für zwei Bilder $u, v: \Omega \to F$ schreiben wir $u \le v$, wenn $u(x) \le v(x)$ an jeder Stelle $x \in \Omega$ gilt.
    \end{definition}
\end{frame}

\begin{frame}
    \frametitle{Elementare Operationen auf Strukturelementen}
    \begin{definition}
        Für Strukturelemente $B, C \subset \Omega$ definiere
        \begin{itemize}
            \item \pause
                die \emph{Spiegelung} $-B := \Set{-b & b \in B}$\pause,
            \item
                die \emph{Translation} $B + p := \Set{b + p & b \in B}$ \pause und
            \item
                die \emph{Addition} $B + C := \Set{b + c & b \in B, c \in C}$.
        \end{itemize}
    \end{definition}
\end{frame}

\section{Erosion / Dilatation}

\subsection{Definition}

\begin{frame}
    \frametitle{Erosion}
    \begin{definition}
        Für ein Bild $u$ und ein Strukturelement $B$ sei die \emph{Erosion}
        \begin{math}
            u \erode B := \bigvee_{b \in B} u^{-b}.
        \end{math}
    \end{definition}
\end{frame}

\begin{frame}
    \frametitle{Dilatation}
    \begin{definition}
        Für ein Bild $u$ und ein Strukturelement $B$ sei die \emph{Dilatation}
        \begin{math}
            u \dilate B := \bigwedge_{b \in B} u^{-b}.
        \end{math}
    \end{definition}
\end{frame}

\subsection{Eigenschaften}

\begin{frame}
    \frametitle{Eigenschaften von Erosion und Dilatation}
    \begin{lemma}
        Erosion und Dilatation genügen den Eigenschaften
        \begin{itemize}
            \item
                \emph{Dualität}: $u \dilate B := \complement (\complement u \erode B)$,
            \item
                \emph{Translationsinvarianz}: $u^p \dilateerode B = (u \dilateerode B)^p$ für $p \in \Omega$,
            \item
                \emph{Monotonie}: $u \le v \implies u \dilateerode B \le v \dilateerode B$,

        \end{itemize}
    \end{lemma}
\end{frame}


\section{Zusammengesetzte Operationen}

\subsection{Öffnen / Schließen}

\subsection{Top-hat Filter}

\subsection{Hit-or-Miss Filter}

\subsection{Morphologische Gradienten}


\section{Anwendungen und Beispiele}





\end{document}
