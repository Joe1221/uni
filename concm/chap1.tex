\Timestamp{2015-04-13}

\section{Binomialkoeffizienten}

\begin{itemize}
    \item
        Für $x \in \C, k \in \Z$ ist
        \begin{math}
            \binom{x}{k} = 
            \begin{cases}
                \frac{x(x-1)\dotsb(x-k+1)}{k!} & x \in \C, k \in \N, \\
                0  & k < 0.
            \end{cases}
        \end{math}
        Für $x \in \Z, k \in \Z$ gilt $\binom{x}{k} \in \Z$, sogar $x \in \N, k \in \Z \implies \binom{x}{k} \in \N$.
    \item
        Additionstheorem:
        \begin{math}
            \binom{x}{k} = \binom{x-1}{k} + \binom{x-1}{k-1}
        \end{math}
        \begin{proof}
            Die Fälle $k < 0$ und $k = 0$ sind klar.
            Für $k > 0$ gilt
            \begin{math}
                \binom{x}{k} &= \frac{x}{k} \binom{x-1}{k-1} \\
                &= \frac{x-k}{k} \binom{x-1}{k-1} + \frac{k}{k} \binom{x-1}{k-1} \\
                &= \binom{x-1}{k} + \binom{x-1}{k-1}.
            \end{math}
        \end{proof}
    \item
        Binomialsatz:
        Für $n \in \N$ gilt
        \begin{math}
            (x + y)^n = \sum_{k\in\Z} \binom{n}{k} x^k y^{n-k}.
        \end{math}
        \begin{proof}
            Der Fall $n = 0$ ist klar.
            Für $n > 0$ ist per Induktion
            \begin{math}
                (x + y)^n
                &= (x + y)(x + y)^{n-1}
                = (x + y) \sum_{k} \binom{n-1}{k} x^k y^{n-1-k} \\
                &= \sum_{k} \binom{n-1}{k} x^{k+1} y^{n-1-k} + \sum_{k} \binom{n-1}{k} x^k y^{n-k} \\
                &= \sum_{k} \binom{n-1}{k-1} x^k y^{n-k} +\sum_{k} \binom{n-1}{k} x^k y^{n-k} \\
                &= \sum_{k} \big( \binom{n-1}{k-1} + \binom{n-1}{k} \big) x^k y^{n-k} \\
                &= \sum_{k} \binom{n}{k} x^k y^{n-k}
            \end{math}
        \end{proof}
    \item
        Es gilt
        \begin{math}
            2^n = (1+1)^n
            = \sum_{k} \binom{n}{k}.
        \end{math}
        Für alle $k \in \Z$ gilt also $\binom{x}{k} \le 2^n$.
    \item
        Es gilt
        \begin{math}
            k \binom{n}{k}
            = (n-k+1) \binom{n}{k-1}
        \end{math}
    \item
        \begin{math}
            1 = \binom{n}{0} < \binom{n}{1} < \dotsb < \binom{n}{\floor{\frac{n}{2}}}
            = \binom{n}{\ceil{\frac{n}{2}}} > \dotsb > \binom{n}{n-1} > \binom{n}{n} = 1.
        \end{math}
    \item
        Für $n \ge 2$ gilt
        \begin{math}
            \binom{n}{\floor{\frac{n}{2}}} \ge \frac{2^n}{n}.
        \end{math}
    \item
        \begin{math}
            \frac{4^n}{2n} \le \binom{2n}{n} < \binom{2n + 1}{n} = \binom{2n}{n-1} + \binom{2n}{n} < \sum_k \binom{2n}{k} = 4^n
        \end{math}
\end{itemize}


\section{Das kleinste gemeinsame Vielfache}

Wir schreiben
\begin{math}
    \kgV(n) &:= \kgV(\Set{1, \dotsc, n}),\\
    \kgV(n) &:= \prod_{\substack{p \le n \\ \text{$p$ prim}}} p^{\floor{\log_p n}}.
\end{math}i

\begin{itemize}
    \item
        Wegen
        \begin{math}
            (\frac{n}{2})^{\frac{n}{2}} \le n! \le n^n
        \end{math}
        ist
        \begin{math}
            n! \in 2^{\LandauTheta(n \log n)}.
        \end{math}
\end{itemize}

\begin{lem}[Nair, 1982]
    Für alle natürlichen Zahlen $1 \le m \le n$ gilt
    \begin{math}
        m \binom{n}{m} \divs \kgV(n).
    \end{math}
    \begin{proof}
        Betrachte
        \begin{math}
            I := \int_0^1 x^{m-1} (1-x)^{n-m} \di[x] > 0.
        \end{math}
        Nun ist $(1-x)^{n-m} = \sum_k (-1)^k \binom{n-m}{k} x^k$ und
        \begin{math}
            x^{m-1} (1-x)^{n-1} = \sum_k (-1)^k \binom{n-m}{k} x^{m-1+k}.
        \end{math}
        Also ist
        \begin{math}
            I = \sum_{k} (-1)^k \binom{n-m}{k} \int_0^1 x^{m-1+k} \di[x]
            = \sum_k (-1)^k \frac{n-m}{k} \frac{1}{m+k}.
        \end{math}
        Es folgt $\kgV(n) I  \in \N$.
        Außerdem ist $I = \frac{1}{m \binom{n}{m}}$, denn per Induktion nach $n-m$ ist für $m = n$
        \begin{math}
            I = \int_0^1 x^{m-1}(1-x)^{n-n} \di[x] = \int_0^1 x^{m-1} \di[x] = \frac{1}{m} = \frac{1}{m \binom{n}{m}}
        \end{math}
        und für $1 \le m < n$ mit partieller Integration (Randterme verschwinden)
        \begin{math}
            I &= \int_0^1 x^{m-1} (1-x)^{n-m} \di[x] \\
            &= \int_0^1 \frac{1}{m} x^m (n-m)(1-x)^{n-(m+1)} \\
            &= \frac{n-m}{m} \int_0^1 x^m (1-x)^{n-(m+1)} \\
            &= \frac{n-m}{m} \frac{1}{(m+1) \binom{n}{m+1}} \\
            &= \frac{1}{m \binom{n}{m}}.
        \end{math}
        Also $m \binom{n}{m} \divs \kgV(n)$.
    \end{proof}
\end{lem}

Es gilt mit obigem Lemma
\begin{math}
    \kgV(n) \ge \ceil{\frac{n}{2}} \binom{n}{\ceil{\frac{n}{2}}}
    \ge \frac{n}{2} \binom{n}{\frac{n}{2}}
    \ge 2^{n-1}
\end{math}
für $n \ge 1$.

\begin{st}
    Für $n \ge 7$ gilt
    \begin{math}
        2^n < \kgV(n)
    \end{math}
    \begin{proof}
        Es gilt
        \begin{math}
            (2n+1) \binom{2n}{n} = (n+1) \binom{2n+1}{n+1} \divs \kgV(2n+1)
        \end{math}
        und
        \begin{math}
            n\binom{2n}{n} \divs \kgV(2n) \divs \kgV(2n+1).
        \end{math}
        Nun sind $n, 2n + 1$ teilerfremd und somit $2 (2n+1) \binom{2n}{n} \divs \kgV(2n+1)$.
        Außerdem
        \begin{math}
            n 4^n
            < n (2n+1) \binom{2n}{n}
            \le \kgV(2n+1).
        \end{math}
        Für $n \ge 4$ ist also
        \begin{math}
            2^{2n+2} \le n 4^n < \kgV(2n+1) \le \kgV(2n+2).
        \end{math}
        Die Behauptung gilt also für $n \ge 9$.
        Für $n = 7$ ist
        \begin{math}
            2^7 = 128 < 420 = \kgV(7)
        \end{math}
        und für $n = 8$ ist
        \begin{math}
            2^8 = 256 < 840 = \kgV(8).
        \end{math}
    \end{proof}
    \begin{note}
        Für $n = 6$ gilt die Aussage nicht mehr:
        \begin{math}
            2^6 = 64 \not< 60 = \kgV(6).
        \end{math}
    \end{note}
\end{st}

