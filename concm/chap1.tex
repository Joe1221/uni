\Timestamp{2015-04-13}

\section{Binomialkoeffizienten}

\begin{itemize}
    \item
        Für $x \in \C, k \in \Z$ ist
        \begin{math}
            \binom{x}{k} = 
            \begin{cases}
                \frac{x(x-1)\dotsb(x-k+1)}{k!} & x \in \C, k \in \N, \\
                0  & k < 0.
            \end{cases}
        \end{math}
        Für $x \in \Z, k \in \Z$ gilt $\binom{x}{k} \in \Z$, sogar $x \in \N, k \in \Z \implies \binom{x}{k} \in \N$.
    \item
        Additionstheorem:
        \begin{math}
            \binom{x}{k} = \binom{x-1}{k} + \binom{x-1}{k-1}
        \end{math}
        \begin{proof}
            Die Fälle $k < 0$ und $k = 0$ sind klar.
            Für $k > 0$ gilt
            \begin{math}
                \binom{x}{k} &= \frac{x}{k} \binom{x-1}{k-1} \\
                &= \frac{x-k}{k} \binom{x-1}{k-1} + \frac{k}{k} \binom{x-1}{k-1} \\
                &= \binom{x-1}{k} + \binom{x-1}{k-1}.
            \end{math}
        \end{proof}
    \item
        Binomialsatz:
        Für $n \in \N$ gilt
        \begin{math}
            (x + y)^n = \sum_{k\in\Z} \binom{n}{k} x^k y^{n-k}.
        \end{math}
        \begin{proof}
            Der Fall $n = 0$ ist klar.
            Für $n > 0$ ist per Induktion
            \begin{math}
                (x + y)^n
                &= (x + y)(x + y)^{n-1}
                = (x + y) \sum_{k} \binom{n-1}{k} x^k y^{n-1-k} \\
                &= \sum_{k} \binom{n-1}{k} x^{k+1} y^{n-1-k} + \sum_{k} \binom{n-1}{k} x^k y^{n-k} \\
                &= \sum_{k} \binom{n-1}{k-1} x^k y^{n-k} +\sum_{k} \binom{n-1}{k} x^k y^{n-k} \\
                &= \sum_{k} \big( \binom{n-1}{k-1} + \binom{n-1}{k} \big) x^k y^{n-k} \\
                &= \sum_{k} \binom{n}{k} x^k y^{n-k}
            \end{math}
        \end{proof}
    \item
        Es gilt
        \begin{math}
            2^n = (1+1)^n
            = \sum_{k} \binom{n}{k}.
        \end{math}
        Für alle $k \in \Z$ gilt also $\binom{x}{k} \le 2^n$.
    \item
        Es gilt
        \begin{math}
            k \binom{n}{k}
            = (n-k+1) \binom{n}{k-1}
        \end{math}
    \item
        \begin{math}
            1 = \binom{n}{0} < \binom{n}{1} < \dotsb < \binom{n}{\floor{\frac{n}{2}}}
            = \binom{n}{\ceil{\frac{n}{2}}} > \dotsb > \binom{n}{n-1} > \binom{n}{n} = 1.
        \end{math}
    \item
        Für $n \ge 2$ gilt
        \begin{math}
            \binom{n}{\floor{\frac{n}{2}}} \ge \frac{2^n}{n}.
        \end{math}
    \item
        \begin{math}
            \frac{4^n}{2n} \le \binom{2n}{n} < \binom{2n + 1}{n} = \binom{2n}{n-1} + \binom{2n}{n} < \sum_k \binom{2n}{k} = 4^n
        \end{math}
\end{itemize}


\section{Das kleinste gemeinsame Vielfache}

Wir schreiben
\begin{math}
    \kgV(n) &:= \kgV(\Set{1, \dotsc, n}),\\
    \kgV(n) &:= \prod_{\substack{p \le n \\ \text{$p$ prim}}} p^{\floor{\log_p n}}.
\end{math}i

\begin{itemize}
    \item
        Wegen
        \begin{math}
            (\frac{n}{2})^{\frac{n}{2}} \le n! \le n^n
        \end{math}
        ist
        \begin{math}
            n! \in 2^{\LandauTheta(n \log n)}.
        \end{math}
\end{itemize}

\begin{lem}[Nair, 1982]
    Für alle natürlichen Zahlen $1 \le m \le n$ gilt
    \begin{math}
        m \binom{n}{m} \divs \kgV(n).
    \end{math}
    \begin{proof}
        Betrachte
        \begin{math}
            I := \int_0^1 x^{m-1} (1-x)^{n-m} \di[x] > 0.
        \end{math}
        Nun ist $(1-x)^{n-m} = \sum_k (-1)^k \binom{n-m}{k} x^k$ und
        \begin{math}
            x^{m-1} (1-x)^{n-1} = \sum_k (-1)^k \binom{n-m}{k} x^{m-1+k}.
        \end{math}
        Also ist
        \begin{math}
            I = \sum_{k} (-1)^k \binom{n-m}{k} \int_0^1 x^{m-1+k} \di[x]
            = \sum_k (-1)^k \frac{n-m}{k} \frac{1}{m+k}.
        \end{math}
        Es folgt $\kgV(n) I  \in \N$.
        Außerdem ist $I = \frac{1}{m \binom{n}{m}}$, denn per Induktion nach $n-m$ ist für $m = n$
        \begin{math}
            I = \int_0^1 x^{m-1}(1-x)^{n-n} \di[x] = \int_0^1 x^{m-1} \di[x] = \frac{1}{m} = \frac{1}{m \binom{n}{m}}
        \end{math}
        und für $1 \le m < n$ mit partieller Integration (Randterme verschwinden)
        \begin{math}
            I &= \int_0^1 x^{m-1} (1-x)^{n-m} \di[x] \\
            &= \int_0^1 \frac{1}{m} x^m (n-m)(1-x)^{n-(m+1)} \\
            &= \frac{n-m}{m} \int_0^1 x^m (1-x)^{n-(m+1)} \\
            &= \frac{n-m}{m} \frac{1}{(m+1) \binom{n}{m+1}} \\
            &= \frac{1}{m \binom{n}{m}}.
        \end{math}
        Also $m \binom{n}{m} \divs \kgV(n)$.
    \end{proof}
\end{lem}

Es gilt mit obigem Lemma
\begin{math}
    \kgV(n) \ge \ceil{\frac{n}{2}} \binom{n}{\ceil{\frac{n}{2}}}
    \ge \frac{n}{2} \binom{n}{\frac{n}{2}}
    \ge 2^{n-1}
\end{math}
für $n \ge 1$.

\begin{st}
    Für $n \ge 7$ gilt
    \begin{math}
        2^n < \kgV(n)
    \end{math}
    \begin{proof}
        Es gilt
        \begin{math}
            (2n+1) \binom{2n}{n} = (n+1) \binom{2n+1}{n+1} \divs \kgV(2n+1)
        \end{math}
        und
        \begin{math}
            n\binom{2n}{n} \divs \kgV(2n) \divs \kgV(2n+1).
        \end{math}
        Nun sind $n, 2n + 1$ teilerfremd und somit $2 (2n+1) \binom{2n}{n} \divs \kgV(2n+1)$.
        Außerdem
        \begin{math}
            n 4^n
            < n (2n+1) \binom{2n}{n}
            \le \kgV(2n+1).
        \end{math}
        Für $n \ge 4$ ist also
        \begin{math}
            2^{2n+2} \le n 4^n < \kgV(2n+1) \le \kgV(2n+2).
        \end{math}
        Die Behauptung gilt also für $n \ge 9$.
        Für $n = 7$ ist
        \begin{math}
            2^7 = 128 < 420 = \kgV(7)
        \end{math}
        und für $n = 8$ ist
        \begin{math}
            2^8 = 256 < 840 = \kgV(8).
        \end{math}
    \end{proof}
    \begin{note}
        Für $n = 6$ gilt die Aussage nicht mehr:
        \begin{math}
            2^6 = 64 \not< 60 = \kgV(6).
        \end{math}
    \end{note}
\end{st}

\Timestamp{2015-04-14}

\begin{lem}
    Für $\frac{n}{2} < m \le n$ gilt $\kgV(n) \divs \kgV(m) \binom{n}{m}$.
    \begin{proof}
        Es gilt
        \begin{math}
            k \binom{n}{k} \binom{n-k}{m-k} = k \binom{m}{k} \binom{n}{m} \divs \kgV(m) \binom{n}{m},
        \end{math}
        denn
        \begin{math}
            \frac{n \dotsb (n-k+1)}{k!} \frac{(n-k) \dotsb (n-m+1)}{(m-k)!}
            \frac{n \dotsb (m-k+1)}{k!} \frac{n \dotsb (n-m+1)}{m!}.
        \end{math}
        Es gilt also $k \binom{n}{k} \divs \kgV(m) \binom{n}{m}$ für $k \le \frac{n}{2}$.
        Wegen $k n \binom{n}{k} = (n-k+1)\binom{n}{n-k+1}$ gilt dies auch für alle $k \le n$.
        Folglich insbesondere $\kgV(n) \divs \kgV(m) \binom{n}{m}$.
    \end{proof}
\end{lem}

\begin{st}
    Für alle $n \ge 1$ gilt $\kgV(n) \le 4^{n-1}$.
    \begin{proof}
        Der Fall $n = 1$ ist klar, den Rest zeigen wir induktiv.
        Für $n = 2m > 1$ ist mit vorigem Lemma
        \begin{math}
            \kgV(2m) \le \kgV(m) \binom{2m}{m}
            \le 4^{m-1} 4^m
            = 4^{2m-1}.
        \end{math}
        Für $n = 2m + 1 > 1$ ist
        \begin{math}
            \kgV(2m + 1) \le \kgV(m+1) \binom{2m+1}{m+1}
            \le 4^m \binom{2m + 1}{m}
            \le 4^m  \cdot 4^m
            = 4^{2m}.
        \end{math}
    \end{proof}
\end{st}

\section{Aussagen zur Primzahldichte}


Wir betrachten die Funktion
\begin{math}
    \pi(x) = \big| \Set{p \le x & \text{$p$ prim}} \big|
\end{math}
für $x \in \R$.
Im folgenden sei $p$ stets eine Primzahl.

Zusammen mit dem letzten Abschnitt gilt für $n \ge 7$
\begin{math}
    2^n < \kgV(n) = \prod_{p\le n} p^{\floor{\log_p n}} \le \prod_{p \le n} n = n^{\pi(n)},
\end{math}
Für $n \ge 4$ kann man hieraus folgen
\begin{math}
    \frac{n}{\log n} \le \pi(n).
\end{math}
Eine mögliche Interpretation dieser Aussage wäre: In den Zahlen mit $\le k$ Bits ist etwa jede $k$-te Zahl prim.

Kommen wir nun zu einer oberen Abschätzung für $\pi(n)$.
Für $1 < t \le n$ gilt
\begin{math}
    t^{\pi(n) - \pi(t)} \le \prod_{t<p\le n} p
    \le kgV(n)
    \le 4^n
\end{math}
Daraus folgt
\begin{math}
    \pi(n) \le \frac{2n}{\log t} + t,
\end{math}
da $\pi(t) \le t$.
Setzen wir nun $t := \frac{n}{\log^2 n}$, so ist
\begin{math}
    \pi(n) \le \frac{2n}{\log n} \cdot \underbrace{\frac{\log n}{\log n - 2 \log\log n}}_{\to 0 \text{ für } (n \to \infty)} + \frac{n}{\log^2 n}.
\end{math}
Wir erhalten also $\forall \epsilon > 0 \exists n_\epsilon \forall n > n_\epsilon$:
\begin{math}
    \pi(n) \le \frac{(2+\epsilon)n}{\log n}.
\end{math}
Der Primzahlsatz besagt sogar
\begin{math}
    \pi(x) \sim \frac{x}{\ln x}.
\end{math}


\section{Das Betrand'sche Postulat}


\begin{st}
    Für $n \ge 1$ existiert $p$ prim mit $n < p \le 2n$.
    \begin{proof}
        Für $n \le 4048$ gilt die Aussage mit den Primzahlen
        \begin{math}
            2, 3, 5, 7, 13, 23, 43, 83, 163, 317, 631, 1259, 2503, 4049.
        \end{math}
        Wir schreiben
        \begin{math}
            e_p(n) := \max\Set{a \in \N & p^a \divs n}.
        \end{math}
        Nun ist
        \begin{itemize}
            \item
                $n = \prod_{p} p^{e_p(n)}$,
            \item
                $n \binom{2n}{n} \divs \kgV(2n)$,
            \item
                $e_p(\kgV(2n)) = \floor{\log_p 2n}$.
        \end{itemize}
        Es folgt $e_p(\binom{2n}{n}) \le \log_p 2n$, also
        \begin{math}
            p^{e_p(\binom{2n}{n})} \le 2n
        \end{math}
        und für $p > \sqrt{2n}$ ist $e_p(\binom{2n}{n}) \le 1$.

        Für $\frac{2}{3} n < p \le n$ und $n \ge 3$ ist
        \begin{math}
            e_p(\binom{2n}{n}) = e_p(\frac{(2n)!}{n!n!}
            = 0.
        \end{math}
        Wir erhalten
        \begin{math}
            \frac{4^n}{2n}
            \le \binom{2n}{n}
            &\le \prod_{p \le \sqrt{2n}} 2n \cdot \prod_{\sqrt{2n} < p < \frac{2}{3}n} p \cdot \prod_{n < p \le 2n} p \\
            &\le (2n)^{\sqrt{2n} - 1} \kgV(\frac{2}{3} n) \prod_{n < p \le 2n} p \\
        \end{math}
        und daraus
        \begin{math}
            4^n  \le (2n)^{\sqrt{2n}} 4^{\frac{2}{3}} \prod_{n \le p \le 2n} p.
        \end{math}
        Für $m = 2n > 853$ gilt
        \begin{math}
            1 < 2^{\frac{1}{3}m - \sqrt{m} \log m} \le \prod_{n < p \le 2m} p.
        \end{math}
    \end{proof}
\end{st}

\begin{st}
    Für $n \ge 2^{12}$ gilt
    \begin{math}
        \big| \Set{p & n < p < 2n} \big|
        \ge \frac{n}{3 \log n}.
    \end{math}
    \begin{proof}
        Mit $m = 2n$ gilt
        \begin{math}
            (2n)^{|\Set{p & n < p < 2n}|}
            \ge 2^{\frac{1}{3} m - \sqrt{m} \log m},
        \end{math}
        also
        \begin{math}
            |\Set{p & n < p < 2n}|
            \ge \frac{m}{3 \log_2 m} - \sqrt{m}
            \ge \frac{n}{3 \log_2 n}.
        \end{math}
    \end{proof}
\end{st}


\section{Das Wachstum der Fakultät}

Wir hatten bereits gesehen: $\log n! \in \LandauTheta(n \log n)$.
Es ist
\begin{math}
    \ln n! = \ln 2 + \ln 3 + \dotsb + \ln n.
\end{math}
und
\begin{math}
    \ln (n-1)! < \int_1^n \ln x \di[x] < \ln n!.
\end{math}
Es gilt mit $\int \ln x \di[x] = x \ln x - x + c$
\begin{math}
    \ln (n-1)! < n \ln n - n + 1 < \ln n!.
\end{math}
Exponenzieren ergibt
\begin{math}
    (n-1)! < \frac{n^n}{e^n} \cdot e = e (\frac{n}{e})^n < n!.
\end{math}
Wir erhalten schließlich
\begin{math}
    e(\frac{n}{e})^n
    < n!
    < ne (\frac{n}{e})^n.
\end{math}

Die sogenannte \emph{Stirling'sche Formel} liefert genauer
\begin{math}
    n! \sim \sqrt{2\pi n} (\frac{n}{e})^n.
\end{math}

\section{Anwendung}

Wir hatten gezeigt
\begin{math}
    \frac{m}{\log m} \le \pi(m) \le \frac{3m}{\log m}.
\end{math}
für $m = p_n$, d.h. die $n$-te Primzahl gilt dann
\begin{math}
    \frac{p_n}{\log p_n} \le n \le \frac{3p_n}{\log p_n}
\end{math}
und wir erhalten
\begin{math}
    p_n &\le n \log p_n, \\
    p_n &\ge n \log n.
\end{math}

