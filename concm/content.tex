\documentclass{mywork}


\DeclareDocumentCommand{\kgV}{}{\operatorname{kgV}}

\begin{document}

\pagestyle{plain}

\section*{Konkrete Mathematik}


\begin{description}
    \item[13.04.2015]
        Binomialkoeffizienten, Additionstheorem, Binomialsatz, Wachstum der Binomialkoeffizienten, Teilbarkeitsrelation, das kleinste gemeinsame Vielfache $\kgV(n)$ der Zahlen $1,\dotsc,n$, Binomialkoeffizienten als Teiler von $\kgV(n)$, untere Schranke $2^n$ für $\kgV(n)$.
    \item[14.04.2015]
        obere Schranke $4^{n-1}$ für $\kgV(n)$, obere und untere Schranke für die Primzahldichte $π(n)$, Bertrand'sches Postulat, untere Schranke für die Anzahl der Primzahlen zwischen $n$ und $2n$, Größe der $n$-ten Primzahl, Wachstum der Fakultätsfunktion.
    \item[20.04.2015]
        Wachstum der Harmonischen Zahlen Hn: untere Schranke $\frac{1}{n}+\ln(n)$, obere Schranke $1+\ln(n)$; bijektive Beweise und kombinatorische Interpretation; Anzahl der Abbildungen von $A$ nach $B$; Anzahl der bijektiven/injektiven Abbildungen von $A$ nach $B$; Größe der Potenzmenge; kombinatorische Interpretation von Binomialkoeffizienten; Polynommethode; bijektiver Beweis des Additionstheorems; trinomiale Revision.
    \item[27.04.2015]
        Binomialinversion, obere Summation von Binomialkoeffizienten, parallele Summation von Binomialkoeffizienten, Vandermonde'sche Konvolution, Auswahl mit Wiederholung, allgemeiner Binomialsatz.
    \item[28.04.2015]
        Geometrische Reihe als Folgerung aus dem allgemeinen Binomialsatz, Multinomialkoeffizienten und Multinomialsatz, Durchschnittsanalyse von Bubblesort, Inklusion und Exklusion, Siebformel von Sylvester, Rencontres-Zahlen.
    \item[04.05.2015]
        Stirling'sche Zahlen zweiter Art, einfache Spezialfälle der Stirling'schen Zahlen zweiter Art, Additionstheorem für Stirling'sche Zahlen zweiter Art, Anzahl der Surjektionen, Potenz als Summen von fallenden Faktoriellen, Zykel einer Permutation, Zykelschreibweise für Permutationen, Stirling'sche Zahlen erster Art, Anzahl der Permutationen als Summen von Stirling'schen Zahlen erster Art, Abschätzungen für Stirling'sche Zahlen zweiter Art, einfache Spezialfälle der Stirling'schen Zahlen erster Art, Additionstheorem für Stirling'sche Zahlen erster Art, Anzahl Permutationen mit zwei Zykeln, Dualität zwischen Stirling'schen Zahlen erster und zweiter Art, Stirling'scher Schmetterling, Bell-Zahlen, Rekursionsformel für Bell-Zahlen.
    \item[11.05.2015]
        Bäume, Catalan-Zahlen, Dyck-Wörter als kombinatorische Interpretation, Klammergebirge, (saturierte) Binärbäume, Rekursionsformel für Catalan-Zahlen, Cayley-Formel für Bäume, Prüfer-Codes.
    \item[12.05.2015]
        Verknüpfung, Operation, assoziative, kommutativ, abelsch, neutral, Einselement und Eindeutigkeit, Nullelement und Eindeutigkeit, Inverses und Eindeutigkeit, Halbgruppe, Monoid, Gruppe, Unterstruktur, Homomorphismus, Isomorphismus, bijektive Homomorphismen, Links- und Rechtsnebenklasse, $G/H$, $H \setminus G$, Eigenschaften von Nebenklassen, Gruppenordnung (kurz: Ordnung), Elementordnung (kurz ebenfalls: Ordnung), Index einer Untergruppe, Satz von Lagrange, Eigenschaften der Elementordnung, zyklische Gruppen, erzeugendes Element (kurz: Erzeuger), Gruppen von Primzahlordnung, Untergruppen zyklischer Gruppen.
    \item[18.05.2015]
        Elementordnung bei Produkten, Satz von Cauchy: Zu jedem Primteiler der Gruppenordnung gibt es ein Element diese Ordnung, symmetrische Gruppen, Anzahl der Fehlstellungen einer Permutation, Vorzeichen (Signum), Transpositionen und Vorzeichen.
    \item[19.05.2015]
        Kern und Bild eines Homomorphismus, Normalteiler, Charakterisierung von Normalteilern, Faktorgruppe, Untergruppen von Index 2 sind Normalteiler, Homomorphiesatz der Gruppentheorie, Ringe, kommutative Ringe, Nullring, Einheitengruppe/multiplikative Gruppe, Schiefkörper, Körper, Unterringe, Unterkörper, Ringhomomorphismus, Ideal, Restklassen, Restklassenring, Homomorphiesatz der Ringtheorie.
    \item[01.06.2015]
        Ringhomomorphismen von einem Körper in einen Ring sind injektiv; maximale Ideale; $R/M$ ist genau dann dann ein Körper, wenn $M$ maximal ist; Nullteiler; nullteilerfrei; Charakteristik; in nullteilerfreien Ringen ist die Charakteristik 0 oder eine Primzahl; in einem Ring mit Primzahlcharakteristik $p$ definiert hoch-$p$ einen Ringhomomorphismus; kleiner Satz von Fermat für Ringe; größter gemeinsamer Teiler; teilerfremd; euklidischer Algorithmus; in den ganzen Zahlen wird jedes Ideal von nur einem Element erzeugt.
    \item[08.06.2015]
        Lemma von Bézout, erweiterter euklidischer Algorithmus, Rekursionstiefe und Laufzeit des erweiterten euklidischen Algorithmus, invertieren modulo $n$, multiplikative Gruppe modulo $n$, Ideal in den ganzen Zahlen, der chinesische Restsatz, die eulersche phi-Funktion, Berechnung der eulerschen phi-Funktion bei bekannter Primzahlfaktorisierung, der Satz von Euler, von $X$ erzeugtes Ideal, Hauptideal, formale Potenzreihen, Polynome, Nullpolynom, Grad eines Polynoms, Leitkoeffizient, Unbestimmte/Variable.
    \item[09.06.2015]
        Schnelle modulare Exponentiation, das RSA-Kryptosystem, Korrektheit von RSA, Polynomring, Ring der formalen Potenzreihen, Addition und Multiplikation von formalen Potenzreihen/Polynomen, Nullteilerfreiheit überträgt sich auf formale Potenzreihen, Invertierbarkeit formaler Potenzreihen, Gradformel, Ableitung, Ableitungsregeln für Addition und Multiplikation, Modulooperation für Polynome (Existenz und Eindeutigkeit), Polynomringe über Körpern sind Hauptidealringe, ggT für Polynome, teilerfremde Polynome.
    \item[22.06.2015]
        Nullstelle $s$ und Teilbarkeit durch das Polynom $X-s$, Vielfachheit einer Nullstelle, einfache Nullstellen, in nullteilerfreien Ringen haben Polynome vom Grad $d$ höchstens $d$ Nullstellen (auch wenn man die Vielfachheiten beachtet), Einfachheit einer Nullstelle $s$ und die Ableitung bei $s$; in einem Polynom $f$ über einem Körper sind genau dann alle Nullstellen einfach, wenn $f$ und $f'$ teilerfremd sind; irreduzible Polynome; $K[X]/f$ ist genau dann ein Körper, wenn $f$ irreduzibel ist; für $f$ ungleich 0 ist die additive Gruppe $K[X]/f$ isomorph zu $K^{\deg(f)}$, Primkörper, endliche Untergruppen der multiplikativen Gruppe eines Körpers sind zyklisch, die Summe der Elemente einer nichttrivialen endlichen Untergruppe der multiplikativen Gruppe eines Körpers ergibt 0.
    \item[23.06.2015]
        Erweiterungskörper, Zerfällungskörper; zu dem Polynom $f$ über einem Körper $K$ existiert ein Erweiterungskörper $E$, über dem $f$ in Linearfaktoren zerfällt (und wenn $K$ endlich ist, dann auch $E$); algebraischer Abschluss, die Anzahl der Elemente in einem endlichen Körper ist eine Primzahlpotenz, zu jeder Primzahlpotenz $q$ gibt es einen Körper mit $q$ Elementen, Quadrate, Euler-Kriterium, Wurzelziehen in endlichen Körpern nach Cipolla, Korrektheit des Algorithmus von Cipolla.
    \item[29.06.2015]
        Erfolgswahrscheinlichkeit des Algorithmus von Cipolla, Wurzelziehen in Gruppen ungerader Ordnung, Primzahlerkennung in deterministischer Polynomialzeit (Teil 1).
    \item[30.06.2015]
        Primzahlerkennung in deterministischer Polynomialzeit (Teil 2).
    \item[06.07.2015]
        Jacobi-Symbol, Zolotarevs Lemma, Legendre-Symbol, das quadratische Reziprozitätsgesetz.
    \item[13.07.2015]
        Berechnung des Jacobi-Symbols, der Solovay-Strassen-Primzahltest, Fehlerwahrscheinlichkeit von höchstens 50\% beim Solovay-Strassen-Primzahltest, der Miller-Rabin-Primzahltest, Fehlerwahrscheinlichkeit von höchstens 25\% beim Miller-Rabin-Primzahltest (Teil 1).
    \item[14.07.2015]
        Fehlerwahrscheinlichkeit von höchstens 25\% beim Miller-Rabin-Primzahltest (Teil 2); Faktorisierung von $n$, wenn $\phi(n)$ nur kleine Primteiler hat; Sicherheit des geheimen Schlüssels beim RSA-Verfahren; die Einheitengruppe modulo $p^e$ ist zyklisch, wenn $p$ eine ungerade Primzahl ist.
    \item[20.07.2015]
        Gewöhnliche erzeugende Funktionen, Konvergenzradius und Wachstum, einfache Beispiele erzeugender Funktionen, die erzeugende Funktion der Fibonacci-Zahlen, Lösen von Rekursionsgleichungen, die erzeugende Funktion der Stirling-Zahlen zweiter Art, exponentielle erzeugende Funktionen, Konvergenzradius und Wachstum bei exponentiellen erzeugenden Funktionen, die exponentielle erzeugende Funktion der Stirling-Zahlen zweiter Art, Ableitung und Erwartungswert bei erzeugenden Funktionen.
    \item[21.07.2015]
        Das Lovasz-Local-Lemma und die Erfüllbarkeit boolescher Formeln.
\end{description}

\end{document}
