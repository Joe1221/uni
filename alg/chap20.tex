\chapter{Die allgemeine Gleichung vom Grad \texorpdfstring{$n \in \N$}{nϵℕ}}

\paragraph{$n = 2$, Quadratische Gleichung}

Sei $K$ ein Körper.
Für $n = 2$ ist die quadratische Gleichung $0 = f(x) = x^2 + ax + b$, wohlbekannt.
Die Mitternachtsformel liefert die Lösungen
\[
	x_{1,2} = \f{-a}2 \pm \sqrt{\f{a^2}4 - b}.
\]
Sind $a, b \in K$, so sind $\f{a^2}4 - b$ und $-\f a2$ auch in $K$ enthalten und $f(x) = 0$ lässt sich lösen, indem wir $\sqrt{\f {a^2}4 - b}$ an $K$ adjungieren (falls diese Wurzel nicht schon in $K$ enthalten ist).
Das geht völlig unabhängig von einer speziellen Wahl von $a, b \in K$, wir betrachten $a$ und $b$ als \emph{Variable}.

Setzen wir als $F := K(a,b) = Q(K[a,b])$, so ist $f(x) \in F[x]$ und der Zerfällungkörper von $f \in F[x]$ über $F$ ergibt sich als $E = F(\sqrt{\f{a^2}4 - b})$ mit $[E : F] = 2$.

Setzen wir $\alpha := \f {\alpha^2}4 - b \in F$, so ist $E = F(\sqrt {\alpha})$, d.h. $\mu_{\alpha, F} = x^2 - \alpha$ und wir haben $E \isomorphic F[x] / (x^2 - \alpha) F[x]$.
$G(E / F)$ ist zyklisch der Ordnung $2$, d.h. $G(E / F) \isomorphic C_2$.

\paragraph{$n = 3$, Kubische Gleichung}
Betrachte nun $x^3 + ax^2 + bx + c = 0$.
Durch geeignete Substitution $x = \alpha z + \beta$ kann man die kubische Gleichung auf die Form
\[
	x^3 + px + q = 0
\]
reduzieren, d.h. der quadratische Term lässt sich wegdiskutieren.
Lösungen dieser Gleichung ergeben sich als
\begin{align*}
	x_1 &= u + v, &
	x_{2,3} &= -\f{u+v}2 \pm \f{v-u}2 + \f{u-v}2 i \sqrt 3,
\end{align*}
wobei
\begin{align*}
	u &= \sqrt[3]{-\f q2 + \sqrt D}, &
	v &= \f p{3u}, &
	D &= (\f p3)^3 + (\f q2)^2.
\end{align*}
Sei $A := \sqrt{D} - \f q2$.
Nun ist $D \in F := K(p,q)$, $A \in F(\sqrt D)$ und $u,v \in F(\sqrt D)(\sqrt[3]{A})$.
Somit ist $F(\sqrt D)(\sqrt[3]{A})$ der Zerfällungskörper von $x^3 + px + q \in F[x]$.

Für $n = 4$ existieren ähnliche Formeln, aber etwas komplizierter.

\paragraph{Allgemeines $n \in \N$}

Wir betrachten
\[
	f(x) = a_0 + a_1 x + \dotsb + a_{n-1} x^{n-1} + x^n \in F[x],
\]
wobei $F = K(a_0, \dotsc, a_{n-1})$ der Körper der rationalen Funkitonen in den Unbestimmten $a_0, a_1, \dotsc, a_{n-1}$ ist.
Sei $E$ der Zerfällungskörper von $f \in F[x] \subset \_F[x]$, d.h.
\[
	f = (x-\alpha_1) \dotsb (x-\alpha_n) \in E[x] \subset \_F[x].
\]
Die Idee ist nun, die Lösungen $\alpha_1, \dotsc, \alpha_n \in E = F(\alpha_0, \dotsc, \alpha_n)$ durch fortgesetzte Iteration
\[
	F = L_0 \subset \underbrace{L_0 (\beta_1)}_{=:L_1} \subset \dotsb \subset L_i \subset \underbrace{L_i(\beta_{i+1})}_{=:L_{i+1}} \subset \dotsb \subset E
\]
zu konstruieren, wobei sich $\beta_{i+1}$ als eine $k$-te Wurzeln eines rationalen Ausdrucks in $\beta_i$ mit Koeffizienten in $L_i$ ergibt (der deshalb in $L_i$ enthalten ist).

Wir werden sehen, dass dies impliziert, dass $G(E / F)$ eine auflösbare Gruppe ist, d.h. es gibt einen Turm von Untergruppen
\[
	(1) = H_0 \le H_1 \le H_2 \le \dotsb \le H_k = G
\]
mit $H_i \Idealof H_{i+1}$ und $H_{i+1} / H_i$ abelsch für $i = 1, \dotsc, k-1$.
Direkt zeigen wir dann, dass $G(E / F) = O_n$ ist mit $n = \deg f$ und dass $O_n$ für $n \ge 5$ nicht auflösbar ist.
Also funktioniert das Verfahren für $n \ge 5$ nicht mehr und es folgt, dass die allgemeine Gleichung vom Grad $n \ge 5$ nicht durch Radikale aufgelöst werden kann.

Betrachtet man den Schritt von $L_i$ zu $L_{i+1} = L_i(\beta_{i+1})$, so erhält man $\beta_{i+1}$ als $k$-te Wurzel eines rationalen Ausdrucks $\alpha$ in $\beta_i$ mit Koeffizienten in $L_i$, d.h. als Zerfällungskörper einer Gleichung $x^k - \alpha \in L_i[x]$.
Wir werden daher als erstes die Galoisgruppe einer solchen Körpererweiterung untersuchen.

Der Einfachheit halber nehmen wir $\Char K = 0$ an.
Alle Aussagen funktionieren auch für beliebige Körper $K$, $F = K(a_0, \dotsc, a_{n-1})$, wird aber komplizierter, wenn man $p^i$-te Wurzeln ziehen muss.
Sei also in diesem Kapitel $K$ ein Körper mit $\Char K = 0$.
Dann ist insbesondere $K$ vollkommen nach \ref{19.1-6}.


\section{Kreisteilungskörper}

Wir beginnen mit dem Spezialfall, dass $E$ Zerfällungskörper von $x^n - 1 \in K[x]$ ist.

\begin{df} \label{20.1-1}
	Sei $n \in \N$.
	$\zeta \in \_K$ heißt \emphdef[Einheitswurzel]{$n$-te Einheitswurzel}, falls $\zeta^n = 1$ ist und \emphdef[Einheitswurzel!primitiv]{primitive $n$-te Einheitswurzel}, falls dazuhin $\zeta^m \neq 1$ für alle $m < n \in \N$ gilt.
	\begin{note}
		Mit $\Char K = 0$ ist $e^{\f{2\pi i}n} = \zeta \in \_K \cap \C$ primitive $n$-te Einheitswurzel, daher der Name „Kreisteilungskörper“.
	\end{note}
\end{df}

Sei $K = \Q, n \in \N$ und $E$ der Zerfällungskörper von $x^n - 1 \in K[x]$ über $K$.
Sei $\scr En = \Set{ \beta \in E & \beta^n = 1 }$, so ist $\scr E_n$ gerade die Menge der Nullstellen von $x^n - 1$ in $\_\Q$ und daher $E = \Q(\scr E_n)$.
Wegen $(\zeta \xi)^n = \zeta^n \xi^n = 1 \cdot 1$ für $\zeta, \xi \in \scr E_n$ und $(\zeta^{-1})^n = (\zeta^n)^{-1} = 1^{-1} = 1$ ist $\scr E_n$ Untergruppe von $E^* = E \setminus \Set 0$ und daher zyklisch nach \ref{19.2-1}.
Sei $\zeta$ ein Erzeuger von $\scr E_n$.
Dann ist $\scr E_n = \Set{ \zeta^i & i \in \N }$.
Wegen $\ddx (x^n - 1) = nx^{n-1}$ und $\ggT(nx^{n-1}, x^n - 1) = 1$ ist $x^n - 1 \in \Q[x]$ separabel (siehe \ref{19.1-6}) und so ist $|\scr E_n| = |\zeta| = n$, d.h. $\zeta$ ist \emph{primitive} $n$-te Einheitswurzel.
Dann ist $E = \Q(\zeta) \supset \scr E_n$, da mit $\zeta$ auch $\zeta^i \in \Q(\zeta)$ ist für alle $i \in \N$.
\begin{note}
	Auch hier ist wieder
	\[
		\scr E_n = \Set{ e^{\f{2\pi k i}n} & 0 \le k \le n-1 } \subset \_Q \subset \C.
	\]
\end{note}
Wir haben nun gezeigt:

\begin{st} \label{20.1-2}
	Sei $\Char K = 0, n \in \N$.
	Dann ist $x^n - 1 \in K[x]$ separabel über $K$ und daher $\scr E_n = \Set{ \zeta \in \_k & \zeta^n = 1 }$ zyklische Gruppe der Ordnung $n$ unter Multiplikation.

	Ist $\zeta \in \scr E_n$, so ist $\zeta$ primitive $n$-te Einheitswurzel genau dann, wenn $\zeta$ ein Erzeuger von $\scr E_n$ ist, d.h. $\<\zeta\> = \scr E_n$.
	Dann ist $\scr E_n = \Set{ \zeta^k & k = 0, 1, \dotsc, n-1 }$.
	Der Zerfällungskörper $E$ von $x^n - 1 \in K[x]$ über $K$ ist dann die einfache Körpererweiterung $E = K(\zeta$ nudedaher ist $K(\zeta)$ galoissch über $K$ für jede primitive $n$-te Einheitswurzel in $\_K$.
\end{st}

\begin{nt} \label{20.1-3}
	Sei $G = \<\zeta\>$ zyklische, multiplikative Gruppe der Ordnung $n$, d.h. $G = \Set{ 1 = \zeta^0, \zeta, \dotsc, \zeta^{n-1} }$.

	Sei $d \divs n, \omega = \zeta^d$.
	Dann ist $|\omega| = k = \f nd$ und die Abbildung $H \mapsto |H|$ ist eine Bijektion von der Menge der Untergruppen von $G$ in die Menge der Teiler von $n$.

	Ist $H \le G$, $|H| = k \divs n, n = kd$, so ist $H = \<\omega\>$ mit $\omega = \zeta^d$.

	Sind $H, U \le G$, $H = \<\zeta^d\>, U = \<\zeta^e\>$ mit $d,e \divs n$, so ist $H \subset U$ genau dann, wenn $e \divs d$.
\end{nt}



\begin{lem} \label{20.1-4}
	Sei $G = \<\zeta\>$, wie in \ref{20.1-3} und sei $d \in \Z$.
	Dann ist $|\zeta^d| = \f n{\ggT(d,n)}$.
	\begin{proof}
		Sei $g = \ggT(n, d)$, $n = ag$, $d = bg$ mit $\ggT(a,b) = 1$.
		So ist $d \f ng = da = bg a = bn$.
		Insbesondere ist $(\zeta^d)^a = \zeta^{nb} = 1^b = 1$ und daher ist $c = |\zeta^d|$ ein Teiler von $a = \f n{\ggT(d,n)}$.
		$|\zeta^d| = c$ impliziert $(\zeta^d)^c = 1 = \zeta^{dc}$, also ist $n$ Teiler von $dc$,
		d.h. $ag$ ist Teiler von $bgc$ und daher ist $a$ Teiler von $bc$.
		Wegen $\ggT(a,b)$ impliziert dies: $a$ ist Teiler von $c$.
		Also ist $a = c$.
	\end{proof}
\end{lem}

\begin{df} \label{20.1-5}
	Die \emphdef{Eulersche $\phi$-Funktion} $\phi: \N \to \N$ ist definiert durch
	\[
		\phi(n) = | \{d \in \N : d \le n-1, \ggT(n,d) = 1 \} |.
	\]
	So ist $\phi(n)$ die Anzahl der zu $n$ teilerfremden natürlichen Zahlen $\le n -1$.
\end{df}

\begin{kor} \label{20.1-6}
	Sei $G = \<\zeta\>$ zyklisch der Ordnung $n$.
	Dann besitzt $G$ genau $\phi(n)$ viele verschiedene Erzeuger.

	Ist $K$ ein Körper mit $\Char K = 0$ und $\zeta \in K$ eine primitive $n$-te Einheitswurzel, so enthält $K$ genau $\phi(n)$ viele verschiedene primitive $n$-te Einheitswurzeln.
	\begin{proof}
		Nach \ref{20.1-4} ist für $1 \le d \le n-1$: $|\zeta^d| = \f n{\ggT(n,d)} = n$ genau dann, wenn $\ggT(n,d) = 1$ ist.
	\end{proof}
\end{kor}

\begin{st} \label{20.1-7}
	Sei $K = \Q$, $\zeta \in \C$ primitive $n$-te Einheitswurzel, $n \in \N$ (etwa $\zeta = e^{\f{2\pi i}n})$.
	Dann ist $\Q(\zeta)$ galoissch über $\Q$ und es ist $[\Q(\zeta) : \Q] = \phi(n)$.
	Die Galoisgruppe $G = G(\Q(\zeta) / \Q)$ von $\Q(\zeta)$ über $\Q$ ist isomorph zur Gruppe $U(\Z / n\Z)$ der Einheiten im Ring $\Z / n \Z$ und ist daher insbesondere abelsch der Ordnung $\phi(n)$.

	Körper, die aus $\Q$ durch Adjunktion einer $n$-ten Einheitswurzel entstehen, heißen \emphdef{Kreisteilungskörper}.
	\begin{proof}
		$\zeta = e^{\f{2\pi i}n}$ ist primitive $n$-te Einheitswurzel.
		Nach \ref{20.1-2} ist $\Q(\zeta) = E$ der Zerfällungskörper von $x^n - 1 \in \Q[x]$ und ist daher galoissch über $\Q$.
		Sei $f = \mu_{\zeta,\Q} \in \Q[x]$.
		Dann teilt $f$ das Polynom $x^n - 1$, da $\zeta$ Nullstelle von $x^n - 1$ ist.
		Nach \ref{15.3-6} ist $f \in \Z[x]$ und wir finden $h \in \Z[x]$ (normiert) so, dass $x^n - 1 = fh$ ist.
		Sei $1 \le d \le n - 1$, $\ggT(d,n) = 1$.
		Dann ist $\omega = \zeta^d \in E$ ebenfalls primitive $n$-te Einheitswurzel.
		Wir zeigen: $\omega$ ist ebenfalls Nullstelle von $f$.
		Sei $1 \le p \le n - 1$ Primzahl mit $p \ndivs n$, d.h. $\ggT(p,n) = 1$ (\oBdA $n \ge \zeta$).
		Wir zeigen die Behauptung für $p = d$:
		Angenommen $f(\omega) = f(\zeta^p) \neq 0$, dann muss $h(\omega) = 0$ sein, da $x^n - 1 = fh$ und $\omega^n - 1 = 0$ ist.
		Betrachte den Epimorphismus $\_{} : \Z \to \Z / p \Z = \F_p : z \mapsto \_z = z + p \Z$ und die Forsetzung $\_{}: \Z[x] \to \F_p[x]: g = \sum_{i=0}^n \beta_i x^i \mapsto \sum_{i=0}^n \_{\beta_i} x^i = \_g \in \F_p[x]$.
		Nun ist $0 = h(\omega) = h(\zeta^p)$, d.h. $\zeta$ ist Nullstelle von $h(x^p) \in \Z[x]$.
		Sei $h(x^p) = f(x)g(x)$ in $\Z[x]$, so ist $\_h(x^p) = \_f(x)\_g(x)$ in $\F_p[x]$.
		Nach \ref{19.1-16} (Frobeniushomomorphismus ausgedehnt auf Polynomringe) ist $\_h(x^p) = (\_h(x))^p$.
		Nun ist $\_{x^n - 1} = x^n - \_1 \in \F_p[x]$ separabel über $\F_p$.
		Wegen $\ddx x^n - \_1 = \_n x^{n-1} \neq 0$ wegen $\ggT(n, p) = 1$.
		Also besitzt $x^n - \_1 \in \F_p[x]$ nur algebraische Nullstellen in $\_{\F_p}$ und daher hat der Teiler $\_f$ von $x^n - \_1 \in \F_p[x]$ ebenfalls nur einfache Nullstellen wegen $x^n - \_1 = \_f  \_h$.
		Andererseits impliziert $\_f \_g = \_h(x^p) = (\_h(x))^p$ in $\F_p[x]$.
		Insbesondere haben $\_f$ und $\_h$ gemeinsame Nullstellen in $\_F_p$ und daher hat $x^n - \_1 = \_f \_h$ mehrfache Nullstellen, ein Widerspruch.
		Also ist $f(\zeta^p) = f(\omega) = 0$ und damit $\mu_{\omega, \Q} = f = \mu_{\zeta, \Q}$.
		Sei $1 \le d \le n-1$, $\ggT(d,n) = 1$ und $d = p_1 \dotsc p_k$ die Faktorisierung von $d$ in Primelemente.
		$\zeta^{p_1p_k} = \omega^{p_2}$ wie oben folgt $\omega^{p_2} = \zeta^{p_1 p_2}$ ist Nullstelle von $\mu_{\omega, \Q} = \mu_{\zeta, \Q}$ usw.
		Iterativ ist $\zeta^d$ Nullstelle von $f = \mu_{\zeta, \Q}$.
		Also mit \ref{20.1-6} sofort:
		alle primitiven $n$-ten Einheitswurzeln in $E = \Q(\zeta)$ sind Nullstellen von $f = \mu_{\zeta, \Q}$.
		Also ist $\deg f \ge \phi(n)$.
		Sei $\sigma \in G(E / \Q)$.
		Dann ist $(\sigma(\zeta))^n = \sigma(\zeta^n) = \sigma(1) = 1$, d.h. $\sigma(\zeta)$ ist wieder $n$-te Einheitswurzel und primitiv.
		Definiere
		\[
			g(x) = \prod_{\sigma \in G} (x - \sigma(\zeta)) \in \_\Q[x].
		\]
		Dann ist offensichtlich $\tau(g(x)) = \prod_{\sigma \in G} (x - \tau\sigma(\zeta)) = \prod_{\sigma \in G} (x - \sigma(\zeta)) = g(x)$.
		Es folgt $g(x) \in E^G[x] = \Q[x]$.
		Wegen $(x - 1_G(\zeta))$ Faktor von $G$ ist $g(\zeta) = 0$ und $f$ ist Teiler von $g$.
		Also ist $\phi(n) \le \deg f \le \deg g \le \phi(n)$ und daraus folgt $\deg f = \deg g = \phi(n) \stack{HS}= |G| = [E: \Q]$.
		Ist $\_0 \neq \_k = k + n \Z \in \Z / n \Z$ (für $1 \le k \le n-1$), so ist $\_k \in U(\Z / n\Z)$ genau dann, wenn gilt: $\exists l \in Z : \_l\_k = \_1$ und daher $\{z\_k : z \in \Z\} = \Z / n \Z$.
		Dies bedeutet aber, dass $\_k$ die additive Gruppe zyklische Gruppe $(\Z / n \Z, +)$ erzeugt, was äquivalent ist zu $\ggT(k,n) = 1$. 
		Für $\sigma \in G, \sigma(\zeta) = \zeta^k$ für ein $k$ mit $1 \le k \le n-1$ und $\ggT(k, n) = 1$.
		Sind $1 \le i,j \le n-1$, $\ggT(n,i) = 1 = \ggT(n,j)$, $\sigma_i(\zeta) = \zeta^i, \sigma_j(\zeta) = \zeta^j$, dann ist
		\[
			\sigma_i \circ \sigma_j (\zeta) = \sigma_i(\zeta^j) = (\sigma_i(\zeta))^j = (\zeta^i)^j = \zeta^{ij},
		\]
		also $\sigma_i \circ \sigma_j = \sigma_{ij}$.
		Also $G \isomorphic U(\Z / n\Z)$.
	\end{proof}
\end{st}

\begin{kor} \label{20.1-8}
	Sei $\Char K = 0$ und sei $\zeta \in \_K$ primitive $n$-te Einheitswurzel.
	Dann ist $K(\zeta)$ galoissch über $K$ und $[K(\zeta) : K] \le \phi(n)$.
	Die Galoisgruppe $G = G(K(\zeta) / K)$ ist isomorph zu einer Untergruppe von $U(\Z / n\Z)$ und daher insbesondere abelsch.
	\begin{proof}
		Wegen $\Char K = 0$ haben wir $\Q \subset K$ und daher ist $\mu_{\zeta,K}$ Teiler von $\mu_{\zeta, \Q}$.
		Insbesondere sind alle Nullstellen von $\mu_{\zeta, K}$ primitive $n$-te Einheitswurzeln und daher eine Potenz $\zeta^k$ von $\zeta$ mit $\ggT(k,n) = 1$, $1 \le k \le n-1$.
		Also ist jede Nullstelle von $\mu_{\zeta, K}$ Potenz von $\zeta$ und daher in $K(\zeta)$ enthalten.
		Also ist $K(\zeta)$ galoissch über $K$.
		Für $\sigma \in G = G(K(\zeta) / K)$ finden wir $1 \le k \le n-1$, $\ggT(n,k) = 1$ mit $\sigma(\zeta) = \zeta^k$ und wie in \ref{20.1-7} definiert $\sigma \mapsto k + n \Z = \_k$ einen injektiven Gruppenhomomorphismus von $G$ in $U(\Z / n\Z)$.
	\end{proof}
	\begin{note}
		Sind $\zeta, \omega$ primitive $n$-te Einheitswurzeln, so kann durchaus $\mu_{\zeta, K} \neq \mu_{\omega, K}$ passieren.
		Dann ist trotzdem $K(\zeta) = K(\omega)$, aber es gibt keinen $K$-Automorphismus $\sigma$ von $G = G(K(\zeta) / K)$ mit $\sigma(\zeta) = \omega$.
	\end{note}
\end{kor}

\begin{df} \label{20.1-9}
	Sei $n \in \N$ und $\zeta \in \C$ primitive $n$-te Einheitswurzeln (\oBdA $\zeta = e^{\f{2\pi i}n}$).
	Das Minimalpolynom $\mu_{\zeta, \Q} \in \Z[x] \subset \Q[x]$ wird \emphdef[Kreisteilungspolynom]{$n$-tes Kreisteilungspolynom} genannt und mit $\Phi_n(x)$ bezeichnet.
\end{df}

Wie finden wir diese Polynome $\Phi_n(x)$?

\begin{st} \label{20.1-10}
	Sei $n \in \N$.
	Dann lässt sich $\Phi_n(x) \in \Z[x]$ rekursiv aus folgender Formel bestimmen:
	\[
		x^n = 1 = \prod_{d \divs n} \Phi_d(x).
	\]
	\begin{note}
		Die Formel ist eine Zerlegung von $x^n - 1$ in irreduzible Faktoren.
	\end{note}
\end{st}

\begin{ex} \label{20.1-11}
	\begin{enumerate}[1)]
		\item
			Ist $p \in \N$ Primzahl, so ist $x^p - 1 = (x-1)(1 + x + x^2 + \dotsb + x^{p-1})$ und $\Phi_p(x) = 1 + x + \dotsb + x^{p-1}$ ist irreduzibel.
			Für $m \in \N$ gilt
			\[
				\Phi_{p^m}(x) = 1 + x^{p^{m-1}} + x^{2p^{m-1}} + \dotsb + x^{(p-1)p^{m-1}} = \Phi_p(x^{p^{m-1}}).
			\]
			Insbesondere ist $\Phi(p^m) = (p-1)p^{m-m} = p^m - p^{m-1}$ (Übung).
		\item
			$\Phi_n(x) \in \Z[x]$.
			Also kann man die Koeffizienten von $\Phi_n(x)$ modulo einer Primzahl $p$ reduzieren und erhält $\_{\Phi_n(x)} \in \F_p[x]$.
			Im allgemeinen ist $\_{\Phi_n} \in \F_p[x]$ jedoch \emph{nicht} irreduzibel.
		\item
			Ist $n$ ungerade in $\N$, so ist $(x^{2n} - 1) = (x^n - 1)(x^n + 1) = -(x^n-1)((-x)^n-1)$ und daher ist $\Phi_{2n}(x) = - \Phi_n(-x)$.
		\item
			Sei $n = p_1^{\nu_1} \dotsb p_s^{\nu_s}$ die Primfaktorzerlegung von $n \in \N$.
			Dann ist
			\[
				\Phi_n(x) = \Phi_{p_1\dotsb p_s} \Big( x^{p_1^{\nu_1 -1 }p_2^{\nu_2 - 1} \dotsb p_s^{\nu_s - 1}} \Big).
			\]
		\item
			Sei $K$ ein beliebiger Körper, $\Char K \ge 0$.
			Sei $\zeta \in \_K$ $n$-te Einheitswurzel ungleich 1.
			Dann ist $1 + \zeta + \dotsb + \zeta^{n-1} = 0$.
			Ist insbesondere $K = \F_q$ mit einer Primzahlpotenz $q$, so sind alle Elemente ungleich 0 von $K$ $(q-1)$-te Einheitswurzeln über $\F_p$, wobei $p := \Char K$.
			Also ist $\sum_{\alpha \in K} \alpha = 0$.

			Somit haben wir die Aussage:
			Ist $p$ Primzahl, $K = \F_p$, so gilt
			\[
				\sum_{1 \le l \le p-1} l \equiv 0 \mod p.
			\]
	\end{enumerate}
\end{ex}

\section{Radikalerweiterungen}

Im Folgenden sei $K$ ein Körper mit $\Char K = 0$.

\begin{df}
	Sei $n \in \N, \alpha \in K$.
	Dann heißt $x^n - \alpha \in K[x]$ reines Polynom.
	Eine Körpererweiterung $K \subset E$ heißt \emphdef{Radikalerweiterung}, wenn es eine Kette
	\[
		K = L_0 \subset L_1 \subset \dotsb \subset L_m = E
	\]
	von Zwischenkörper von $K \subset E$ gibt, so dass für jedes $i = 0, 1, \dotsc, m-1$ gilt: $L_{i+1} = L_i(\alpha_i)$, wobei $\alpha_i$ Nullstelle eines reinen Polynoms über $L_i$ ist.

	Sei $f \in K[x]$.
	Dann heißt die Gleichung $f(x) = 0$ über $K$ \emphdef[auflösbar]{durch Radikale auflösbar}, falls es eine Radikalerweiterung $E$ von $K$ gibt, so dass $f$ über $E$ in Linearfaktoren zerfällt.
\end{df}

\begin{nt} \label{20.2-2}
	$x^n - 1 \in K[x]$ ist ein Spezialfall von \ref{20.1-1} und daher ist $K(\zeta)$ für eine primitive Einheitswurzel $\zeta \in \_K$ eine Radikalerweiterung von $K$.
\end{nt}

\begin{st} \label{20.2-3}
	Sei $n \in \N$ und $K$ enthalte eine primtive $n$-te Einheitswurzel (und damit alle $n$-ten Einheitswurzeln).
	Dann gelten die folgenden Aussagen.
	\begin{enumerate}[i)]
		\item
			Sei $\alpha \in K$ und $\_\beta \in \_K$ mit $\beta^n = \alpha$.
			Dann ist $K(\beta)$ galoissch über $K$ mit zyklischer Galoisgruppe $(\Z / d \Z, +)$, wobei $d \divs n$.
		\item
			Ist $E$ Galoiserweiterung von $K$ vom Grad $n$ mit zyklischer Galoisgruppe, so gibt es ein $\beta \in u_K$ mit $\beta^n = \alpha \in K$ und $E = K(\beta)$ ist Zerfällungskörper von $x^n - \alpha \in K[x]$ über $K$.
	\end{enumerate}
	\begin{proof}
		\begin{enumerate}[i)]
			\item
				Sei $\zeta \in K$ eine primitive $n$-te Einheitswurzel.
				Sei $\beta \in \_K$ irgendeine Wurzel (Nullstelle) von $x^n - \alpha$, so ist $\beta^n = \alpha$ und daher $(\zeta^i \beta)^n = (\zeta^n)^i \beta^n = \beta^n = \alpha$, d.h. $\zeta^i \beta$ ist ebenfalls Nullstelle von $x^n - \alpha$ ($0 \le i \le n-1$).
				Also ist $x^n - \alpha = \prod_{0\le i \le n-1} (x - \zeta^i \beta) \in \_K[x]$.
				Da $\zeta^i \in K$ für alle $0\le i \le n-1$, sind die $\zeta^i \beta \in K(\beta)$.
				Also ist $K(\beta)$ Zerfällungskörper von $x^n - \alpha$ und daher normal über $K$ (separabel wegen $\Char K = $) und daher galoissch über $K$.
				Also sind alle Wurzeln von $\mu_{K, \beta}$ in $K(\beta)$ von der Form $\zeta^i \beta$ für ein $0 \le i \le n -1$.
				Nach \ref{18.4-4} induziert die Wurzel $\zeta^i \beta$ für $0 \le i \le n-1$ für $\zeta^i \beta$ Nullstelle von $\mu_{K, \beta}$ genau einen $K$-Homomorphismus $\sigma_i: K(\beta) \to \_K$ mit $\sigma_i(\beta) = \zeta^i \beta$ und $\im \sigma_i = K(\beta)$, da $K(\beta)$ normal über $K$ ist.
				Also ist $\sigma_i \in G(K(\beta) / K)$.
				Die Abbildung $\sigma_a \mapsto i + n \Z = \_i \in (\Z / n\Z, +)$ ist eine injektive Abbildung von $G = G(K(\beta), K)$ in $(\Z / n\Z, + )$.
				Sind $\sigma_i, \sigma_j \in G$, d.h. $\zeta^i \beta, \zeta^j$ sind Nullstellen von $\mu_{\beta,K}$ und $\sigma_i(\beta) = \zeta^i \beta$, $\sigma_j(\beta) = \zeta^j \beta$, so ist $\sigma_j \circ \sigma_i(\beta) = \sigma_j(\zeta^i \beta) = \zeta^i \sigma_j(\beta) = \zeta^i(\zeta^j \beta) = \zeta^{\_{i+j}}(\beta) = \sigma_{\_{i+j}}(\beta)$.
				Also ist $\sigma_i \mapsto \_i \in (\Z / n \Z, +)$ ein Gruppenhomomorphismus und daher ist $G(K(\beta) / K)$ zyklisch der Ordnung $d$, wobei $d \divs n$.
			\item
				später
		\end{enumerate}
	\end{proof}
\end{st}

\begin{kor} \label{20.2-4}
	Sei $n \in \N, \alpha \in K$ und sei $E$ der Zerfällungskörper von $x^n - \alpha \in K[x]$.
	Sei $\beta \in \_K$ Nullstelle von $x^n - \alpha$ und sei $\zeta \in \_K$ eine primitive $n$-te Einheitswurzeln.
	Dann gilt
	\begin{enumerate}[i)]
		\item
			$E = K(\zeta, \alpha)$ ist Radikalerweiterung von $K$ mit Zwischenkörperkette $K \subset K(\zeta) \subset K(\zeta, \alpha) = E$.
		\item
			$E$ ist galoissch über $K$ und über $K(\zeta), K(\zeta)$ ist galoissch über $K$, $G(E / K(\zeta))$ ist zyklsicher Normalteiler von $G(E / K)$ und $G(K(\zeta) / K) \isomorphic G(E / K) / \allowbreak G(E / K(\zeta))$ ist abelsch.
	\end{enumerate}
	\begin{proof}
		Es gilt $\beta, \zeta \in \_K$, $x^n - a = \prod_{i=0}^{n-1} (x - \zeta^i \beta)$ in $\_K[x]$ wie in \ref{20.1-3}.
		Also ist $K(\zeta, \beta) = E = K(\Set{\zeta^i \beta & i = 0, \dotsc, n-1 })$.
		Also ist $\zeta = \f {\zeta \beta}{\beta} \in E$ und daher ist $K(\zeta) = E$.
		Andererseits ist $\zeta^i \beta \in K(\beta, \zeta)$ für $i = 0, \dotsc, n-1$ und $E = K(\zeta, \beta)$.
		Also ist $K \subset K(\zeta) \subset K(\zeta, \alpha) = E$, wobei $\zeta$ Nullstelle von $x^n - 1$ und $\beta$ Nullstelle von $x^n  - \alpha$.
		Nach \ref{20.1-2} ist $K(\zeta)$ galoissch über $K$ und $E$ ist algebraisch über $K$ und normal über $K$.
		Mit dem Hauptsatz \ref{19.4-21} ist $G(E / K(\zeta)) \Idealof G(E / K)$ und $G(E / K) / G(E / K(\zeta)) \isomorphic G(K(\zeta) / K)$.
		Rest folgt.
	\end{proof}
\end{kor}

\begin{lem} \label{20.2-5}
	Sei $E$ Radikalerweiterung von $K$ (und daher insbesondere endlich über $K$).
	Dann gibt es eine endliche Körpererweiterung $E'$ von $E$, die galoissch über $K$ und Radikalerweiterung von $K$ ist.
	\begin{proof}
		$[E : K] < \infty$, da $E$ Radikalerweiterung ist.
		Nutze Induktion über $[E : K]$.
		Für $[E : K] = 1$ ist nichts zu zeigen.
		Sei $[E : K] \ge 2$.
		Nach Voraussetzung gibt es ein $K \subset L \subset E = L(\beta)$ mit $\alpha \not\in L$ und $L$ ist Radikalerweiterung von $K$.
		Also gibt es eine galoissche Radikalerweiterung $L'$ von $K$ mit $L \subset L', [L' : L] < \infty$.
		Also ist $L'$ Zerfällungskörper über ein Polynom aus $K[x]$.
		Sei $G' = G(L' / K) \ni \sigma$, ($\alpha \in L' \implies \sigma(\alpha) \in L')$.
		\[
			g(x) = \prod_{\sigma \in G'} (x^n - \underbrace{\sigma(\alpha)}_{=\sigma(\beta^n)})
			\in L'[x].
		\]
		Wegen $\sigma g = g$ für alle $g \in G$ ist $g \in K[x]$.
		$x^n - \alpha$ ist Faktor von $g(x)$, also $\beta \in E'$.
		Setze $E'$ als Zerfällungskörper.
		Also ist $E$ auch Zerfällungskörper von $fg \in K[x]$.
		Also ist $E'$ galoissch über $K$ und $E = L(\beta) \subset E'$.
		Da jede Nullstelle von $g \in K[x] \subset L'[x]$ eine Nullstelle eines Polynoms $x^n - \sigma(\alpha)$ ist für $\sigma \in G'$ und damit $n$-te Wurzel von $\sigma(\alpha) \in L'$ ist, ist $E'$ Radikalerweiterung von $L'$.
		$L' \subset L(\beta_1) \subset (L(\beta_1))(\beta_2) \subset L(\beta_1, \dotsc, \beta_m) = E$ mit Nullstellen $\beta_i$ für $1\le i \le m$.
		Da $L'$ Radikalerweiterung von $K$ ist und $E'$ Radikalerweiterung von $L'$ ist $E'$ Radikalerweiterung von $K$
		(durch  Hintereinanderfügen der Zwischenkörperketten von $L'$ über $K$ und $E'$ über $L'$).
	\end{proof}
\end{lem}

\begin{st} \label{20.2-6}
	Sei $E$ Radikalerweiterung von $K$.
	Dann gibt es eine endliche, galoissche Radikalerweiterung $E'$ über $K$ mit Zwischenkörperkette $K = L_0' \subset L_1' \subset \dotsb \subset L_m' = E'$, so dass gilt:
	\begin{enumerate}[i)]
		\item
			$L_{i+1}' = L_i'(\alpha_{i+1})$ mit $\alpha_{i+1}^{k_{i+1}} =: \beta_i \in L_i'$ für ein $k_{i+1} \in \N$,
		\item
			$L_{i+1}'$ ist galoissch über $L_i'$ für $i = 1, \dotsc, m$.
	\end{enumerate}
	\begin{proof}
		Nach \ref{20.2-5} können wir annehmen, dass $E$ galoissch über $K$ ist.
		Es existiert also eine Kette von Zwischenkörpern $K = L_1 \subset L_2 \subset \dotsb \subset L_m = E$ mit $\alpha_{i+1} \in L_{i+1}$ und $\alpha_{i+1}^{k_{i+1}} = \beta_i \in L_i$, $L_{i+1} = L_i(\alpha_{i+1})$ für $i=1, \dotsc, m$.
		Sei $k = k_2 \dotsb k_m$ und sei $\zeta \in \_K$ eine primitive $k$-te Einheitswurzel.
		Dann ist $L_1' = K(\zeta)$ galoissch über $K$ mit abelscher Galoisgruppe $G(L_1' / K) \le U(\Z / n\Z)$.
		Sei $L_i' = L_i(\zeta)$ für $i = 2, \dotsc, m-1$ und $L_m' = E' = E(\zeta)$.
		Da $E$ endlich gaoissch über $K$, ist $E$ Zerfällungkörper eines Polynoms $f \in K[x]$ und daher ist $E' = E(\zeta)$ Zerfällungskörper von $f \cdot (x^n - 1) \in K[x]$.
		Also ist $E'$ galoissch über $K$.
		\[
			K \subset K(\zeta) = L_1' = K(\zeta) \subset L_2' = L_2(\zeta) \subset \dotsb \subset L_m' = L_m(\zeta) = E(\zeta) = E'.
		\]
		Nun ist $L_{i+1}' = L_{i+1}(\zeta) = L_i (\zeta, \alpha_{i+1}) = L_i'(\alpha_{i+1})$ mit $\alpha_{i+1}^{k_{i+1}} = \beta_i \in L_i \subset L_i'$.
		Mit der $k$-ten primtiven Einheitswurzel $\zeta$ enthält $K(\zeta)$ alle $k_i$-ten Einheitswurzeln.
		Nach \ref{20.2-4} ist $L_{i+1}'$ galoissch über $L_i'$ mit zyklischer Galoisgruppe.
		Da $E' = E(\zeta)$ endlicher Gaoiserweiterung von $E$ ist, ist $E'$ endliche Galoiserweiterung von $K$ und auch Radikalerweiterung von $K$.
	\end{proof}
\end{st}

\begin{nt} \label{20.2-7}
	In der Situation von \ref{20.2-6} (wobei wir „$'$“ hier weglassen) impliziert der Hauptsatz \ref{19.4-21} der Galoistheorie nun:
	Sei $H_i = G(E / L_i)$, dann ist
	\[
		(1) = H_m \le H_{m-1} \le \dotso \le H_1 \le H_0 = G(E / K)
	\]
	eine Kette von Untergruppen von $G(E / K)$ mit $H_{i+1} \Idealof H_i$ und $H_i / H_{i+1} = G(E / L_i) / G(E / L_{i+1} \isomorphic G(L_{i+1} / L_i)$, da jetzt $L_{i+1}$ galoissch über $L_i$ ist.
	Entsteht $L_{i+1}$ aus $L_i$ durch Adjunktion einer Einheitswurzel, wenden wir \ref{20.1-6} und sonst \ref{20.2-3} an und schließen in allen Fällen: $H_i / H_{i+1}$ ist abelsch.

	Endliche Gruppen mit einer solchen Kette von Untergruppen heißen \emphdef{auflösbar}.
\end{nt}


\section{Auflösbare Gruppen}
