\chapter{Die allgemeine Gleichung vom Grad \texorpdfstring{$n \in \N$}{nϵℕ}}

\paragraph{$n = 2$, Quadratische Gleichung}

Sei $K$ ein Körper.
Für $n = 2$ ist die quadratische Gleichung $0 = f(x) = x^2 + ax + b$, wohlbekannt.
Die Mitternachtsformel liefert die Lösungen
\[
	x_{1,2} = \f{-a}2 \pm \sqrt{\f{a^2}4 - b}.
\]
Sind $a, b \in K$, so sind $\f{a^2}4 - b$ und $-\f a2$ auch in $K$ enthalten und $f(x) = 0$ lässt sich lösen, indem wir $\sqrt{\f {a^2}4 - b}$ an $K$ adjungieren (falls diese Wurzel nicht schon in $K$ enthalten ist).
Das geht völlig unabhängig von einer speziellen Wahl von $a, b \in K$, wir betrachten $a$ und $b$ als \emph{Variable}.

Setzen wir als $F := K(a,b) = Q(K[a,b])$, so ist $f(x) \in F[x]$ und der Zerfällungkörper von $f \in F[x]$ über $F$ ergibt sich als $E = F(\sqrt{\f{a^2}4 - b})$ mit $[E : F] = 2$.

Setzen wir $\alpha := \f {\alpha^2}4 - b \in F$, so ist $E = F(\sqrt {\alpha})$, d.h. $\mu_{\alpha, F} = x^2 - \alpha$ und wir haben $E \isomorphic F[x] / (x^2 - \alpha) F[x]$.
$G(E / F)$ ist zyklisch der Ordnung $2$, d.h. $G(E / F) \isomorphic C_2$.

\paragraph{$n = 3$, Kubische Gleichung}
Betrachte nun $x^3 + ax^2 + bx + c = 0$.
Durch geeignete Substitution $x = \alpha z + \beta$ kann man die kubische Gleichung auf die Form
\[
	x^3 + px + q = 0
\]
reduzieren, d.h. der quadratische Term lässt sich wegdiskutieren.
Lösungen dieser Gleichung ergeben sich als
\begin{align*}
	x_1 &= u + v, &
	x_{2,3} &= -\f{u+v}2 \pm \f{v-u}2 + \f{u-v}2 i \sqrt 3,
\end{align*}
wobei
\begin{align*}
	u &= \sqrt[3]{-\f q2 + \sqrt D}, &
	v &= \f p{3u}, &
	D &= (\f p3)^3 + (\f q2)^2.
\end{align*}
Sei $A := \sqrt{D} - \f q2$.
Nun ist $D \in F := K(p,q)$, $A \in F(\sqrt D)$ und $u,v \in F(\sqrt D)(\sqrt[3]{A})$.
Somit ist $F(\sqrt D)(\sqrt[3]{A})$ der Zerfällungskörper von $x^3 + px + q \in F[x]$.

Für $n = 4$ existieren ähnliche Formeln, aber etwas komplizierter.

\paragraph{Allgemeines $n \in \N$}

Wir betrachten
\[
	f(x) = a_0 + a_1 x + \dotsb + a_{n-1} x^{n-1} + x^n \in F[x],
\]
wobei $F = K(a_0, \dotsc, a_{n-1})$ der Körper der rationalen Funkitonen in den Unbestimmten $a_0, a_1, \dotsc, a_{n-1}$ ist.
Sei $E$ der Zerfällungskörper von $f \in F[x] \subset \_F[x]$, d.h.
\[
	f = (x-\alpha_1) \dotsb (x-\alpha_n) \in E[x] \subset \_F[x].
\]
Die Idee ist nun, die Lösungen $\alpha_1, \dotsc, \alpha_n \in E = F(\alpha_0, \dotsc, \alpha_n)$ durch fortgesetzte Iteration
\[
	F = L_0 \subset \underbrace{L_0 (\beta_1)}_{=:L_1} \subset \dotsb \subset L_i \subset \underbrace{L_i(\beta_{i+1})}_{=:L_{i+1}} \subset \dotsb \subset E
\]
zu konstruieren, wobei sich $\beta_{i+1}$ als eine $k$-te Wurzeln eines rationalen Ausdrucks in $\beta_i$ mit Koeffizienten in $L_i$ ergibt (der deshalb in $L_i$ enthalten ist).

Wir werden sehen, dass dies impliziert, dass $G(E / F)$ eine auflösbare Gruppe ist, d.h. es gibt einen Turm von Untergruppen
\[
	(1) = H_0 \le H_1 \le H_2 \le \dotsb \le H_k = G
\]
mit $H_i \Idealof H_{i+1}$ und $H_{i+1} / H_i$ abelsch für $i = 1, \dotsc, k-1$.
Direkt zeigen wir dann, dass $G(E / F) = \S_n$ ist mit $n = \deg f$ und dass $\S_n$ für $n \ge 5$ nicht auflösbar ist.
Also funktioniert das Verfahren für $n \ge 5$ nicht mehr und es folgt, dass die allgemeine Gleichung vom Grad $n \ge 5$ nicht durch Radikale aufgelöst werden kann.

Betrachtet man den Schritt von $L_i$ zu $L_{i+1} = L_i(\beta_{i+1})$, so erhält man $\beta_{i+1}$ als $k$-te Wurzel eines rationalen Ausdrucks $\alpha$ in $\beta_i$ mit Koeffizienten in $L_i$, d.h. als Zerfällungskörper einer Gleichung $x^k - \alpha \in L_i[x]$.
Wir werden daher als erstes die Galoisgruppe einer solchen Körpererweiterung untersuchen.

Der Einfachheit halber nehmen wir $\Char K = 0$ an.
Alle Aussagen funktionieren auch für beliebige Körper $K$, $F = K(a_0, \dotsc, a_{n-1})$, wird aber komplizierter, wenn man $p^i$-te Wurzeln ziehen muss.
Sei also in diesem Kapitel $K$ ein Körper mit $\Char K = 0$.
Dann ist insbesondere $K$ vollkommen nach \ref{19.1-6}.


\section{Kreisteilungskörper}

Wir beginnen mit dem Spezialfall, dass $E$ Zerfällungskörper von $x^n - 1 \in K[x]$ ist.

\begin{df} \label{20.1-1}
	Sei $n \in \N$.
	$\zeta \in \_K$ heißt \emphdef[Einheitswurzel]{$n$-te Einheitswurzel}, falls $\zeta^n = 1$ ist und \emphdef[Einheitswurzel!primitiv]{primitive $n$-te Einheitswurzel}, falls dazuhin $\zeta^m \neq 1$ für alle $m < n \in \N$ gilt.
	\begin{note}
		Mit $\Char K = 0$ ist $e^{\f{2\pi i}n} = \zeta \in \_K \cap \C$ primitive $n$-te Einheitswurzel, daher der Name „Kreisteilungskörper“.
	\end{note}
\end{df}

Sei $K = \Q, n \in \N$ und $E$ der Zerfällungskörper von $x^n - 1 \in K[x]$ über $K$.
Sei $\scr En = \Set{ \beta \in E & \beta^n = 1 }$, so ist $\scr E_n$ gerade die Menge der Nullstellen von $x^n - 1$ in $\_\Q$ und daher $E = \Q(\scr E_n)$.
Wegen $(\zeta \xi)^n = \zeta^n \xi^n = 1 \cdot 1$ für $\zeta, \xi \in \scr E_n$ und $(\zeta^{-1})^n = (\zeta^n)^{-1} = 1^{-1} = 1$ ist $\scr E_n$ Untergruppe von $E^* = E \setminus \Set 0$ und daher zyklisch nach \ref{19.2-1}.
Sei $\zeta$ ein Erzeuger von $\scr E_n$.
Dann ist $\scr E_n = \Set{ \zeta^i & i \in \N }$.
Wegen $\ddx (x^n - 1) = nx^{n-1}$ und $\ggT(nx^{n-1}, x^n - 1) = 1$ ist $x^n - 1 \in \Q[x]$ separabel (siehe \ref{19.1-6}) und so ist $|\scr E_n| = |\zeta| = n$, d.h. $\zeta$ ist \emph{primitive} $n$-te Einheitswurzel.
Dann ist $E = \Q(\zeta) \supset \scr E_n$, da mit $\zeta$ auch $\zeta^i \in \Q(\zeta)$ ist für alle $i \in \N$.
\begin{note}
	Auch hier ist wieder
	\[
		\scr E_n = \Set{ e^{\f{2\pi k i}n} & 0 \le k \le n-1 } \subset \_Q \subset \C.
	\]
\end{note}
Wir haben nun gezeigt:

\begin{st} \label{20.1-2}
	Sei $\Char K = 0, n \in \N$.
	Dann ist $x^n - 1 \in K[x]$ separabel über $K$ und daher $\scr E_n = \Set{ \zeta \in \_k & \zeta^n = 1 }$ zyklische Gruppe der Ordnung $n$ unter Multiplikation.

	Ist $\zeta \in \scr E_n$, so ist $\zeta$ primitive $n$-te Einheitswurzel genau dann, wenn $\zeta$ ein Erzeuger von $\scr E_n$ ist, d.h. $\<\zeta\> = \scr E_n$.
	Dann ist $\scr E_n = \Set{ \zeta^k & k = 0, 1, \dotsc, n-1 }$.
	Der Zerfällungskörper $E$ von $x^n - 1 \in K[x]$ über $K$ ist dann die einfache Körpererweiterung $E = K(\zeta)$ und daher ist $K(\zeta)$ galoissch über $K$ für jede primitive $n$-te Einheitswurzel in $\_K$.
\end{st}

\begin{nt} \label{20.1-3}
	Sei $G = \<\zeta\>$ zyklische, multiplikative Gruppe der Ordnung $n$, d.h. $G = \Set{ 1 = \zeta^0, \zeta, \dotsc, \zeta^{n-1} }$.

	Sei $d \divs n, \omega = \zeta^d$.
	Dann ist $|\omega| = k = \f nd$ und die Abbildung $H \mapsto |H|$ ist eine Bijektion von der Menge der Untergruppen von $G$ in die Menge der Teiler von $n$.

	Ist $H \le G$, $|H| = k \divs n, n = kd$, so ist $H = \<\omega\>$ mit $\omega = \zeta^d$.

	Sind $H, U \le G$, $H = \<\zeta^d\>, U = \<\zeta^e\>$ mit $d,e \divs n$, so ist $H \subset U$ genau dann, wenn $e \divs d$.
\end{nt}



\begin{lem} \label{20.1-4}
	Sei $G = \<\zeta\>$, wie in \ref{20.1-3} und sei $d \in \Z$.
	Dann ist $|\zeta^d| = \f n{\ggT(d,n)}$.
	\begin{proof}
		Sei $g = \ggT(n, d)$, $n = ag$, $d = bg$ mit $\ggT(a,b) = 1$.
		So ist $d \f ng = da = bg a = bn$.
		Insbesondere ist $(\zeta^d)^a = \zeta^{nb} = 1^b = 1$ und daher ist $c = |\zeta^d|$ ein Teiler von $a = \f n{\ggT(d,n)}$.
		$|\zeta^d| = c$ impliziert $(\zeta^d)^c = 1 = \zeta^{dc}$, also ist $n$ Teiler von $dc$,
		d.h. $ag$ ist Teiler von $bgc$ und daher ist $a$ Teiler von $bc$.
		Wegen $\ggT(a,b)$ impliziert dies: $a$ ist Teiler von $c$.
		Also ist $a = c$.
	\end{proof}
\end{lem}

\begin{df} \label{20.1-5}
	Die \emphdef{Eulersche $\phi$-Funktion} $\phi: \N \to \N$ ist definiert durch
	\[
		\phi(n) = | \{d \in \N : d \le n-1, \ggT(n,d) = 1 \} |.
	\]
	So ist $\phi(n)$ die Anzahl der zu $n$ teilerfremden natürlichen Zahlen $\le n -1$.
\end{df}

\begin{kor} \label{20.1-6}
	Sei $G = \<\zeta\>$ zyklisch der Ordnung $n$.
	Dann besitzt $G$ genau $\phi(n)$ viele verschiedene Erzeuger.

	Ist $K$ ein Körper mit $\Char K = 0$ und $\zeta \in K$ eine primitive $n$-te Einheitswurzel, so enthält $K$ genau $\phi(n)$ viele verschiedene primitive $n$-te Einheitswurzeln.
	\begin{proof}
		Nach \ref{20.1-4} ist für $1 \le d \le n-1$: $|\zeta^d| = \f n{\ggT(n,d)} = n$ genau dann, wenn $\ggT(n,d) = 1$ ist.
	\end{proof}
\end{kor}

\begin{st} \label{20.1-7}
	Sei $K = \Q$, $\zeta \in \C$ primitive $n$-te Einheitswurzel, $n \in \N$ (etwa $\zeta = e^{\f{2\pi i}n})$.
	Dann ist $\Q(\zeta)$ galoissch über $\Q$ und es ist $[\Q(\zeta) : \Q] = \phi(n)$.
	Die Galoisgruppe $G = G(\Q(\zeta) / \Q)$ von $\Q(\zeta)$ über $\Q$ ist isomorph zur Gruppe $U(\Z / n\Z)$ der Einheiten im Ring $\Z / n \Z$ und ist daher insbesondere abelsch der Ordnung $\phi(n)$.

	Körper, die aus $\Q$ durch Adjunktion einer $n$-ten Einheitswurzel entstehen, heißen \emphdef{Kreisteilungskörper}.
	\begin{proof}
		$\zeta = e^{\f{2\pi i}n}$ ist primitive $n$-te Einheitswurzel.
		Nach \ref{20.1-2} ist $\Q(\zeta) = E$ der Zerfällungskörper von $x^n - 1 \in \Q[x]$ und ist daher galoissch über $\Q$.
		Sei $f = \mu_{\zeta,\Q} \in \Q[x]$.
		Dann teilt $f$ das Polynom $x^n - 1$, da $\zeta$ Nullstelle von $x^n - 1$ ist.
		Nach \ref{15.3-6} ist $f \in \Z[x]$ und wir finden $h \in \Z[x]$ (normiert) so, dass $x^n - 1 = fh$ ist.
		Sei $1 \le d \le n - 1$, $\ggT(d,n) = 1$.
		Dann ist $\omega = \zeta^d \in E$ ebenfalls primitive $n$-te Einheitswurzel.
		Wir zeigen: $\omega$ ist ebenfalls Nullstelle von $f$.
		Sei $1 \le p \le n - 1$ Primzahl mit $p \ndivs n$, d.h. $\ggT(p,n) = 1$ (\oBdA $n \ge \zeta$).
		Wir zeigen die Behauptung für $p = d$:
		Angenommen $f(\omega) = f(\zeta^p) \neq 0$, dann muss $h(\omega) = 0$ sein, da $x^n - 1 = fh$ und $\omega^n - 1 = 0$ ist.
		Betrachte den Epimorphismus $\_{} : \Z \to \Z / p \Z = \F_p : z \mapsto \_z = z + p \Z$ und die Forsetzung $\_{}: \Z[x] \to \F_p[x]: g = \sum_{i=0}^n \beta_i x^i \mapsto \sum_{i=0}^n \_{\beta_i} x^i = \_g \in \F_p[x]$.
		Nun ist $0 = h(\omega) = h(\zeta^p)$, d.h. $\zeta$ ist Nullstelle von $h(x^p) \in \Z[x]$.
		Sei $h(x^p) = f(x)g(x)$ in $\Z[x]$, so ist $\_h(x^p) = \_f(x)\_g(x)$ in $\F_p[x]$.
		Nach \ref{19.1-16} (Frobeniushomomorphismus ausgedehnt auf Polynomringe) ist $\_h(x^p) = (\_h(x))^p$.
		Nun ist $\_{x^n - 1} = x^n - \_1 \in \F_p[x]$ separabel über $\F_p$.
		Wegen $\ddx x^n - \_1 = \_n x^{n-1} \neq 0$ wegen $\ggT(n, p) = 1$.
		Also besitzt $x^n - \_1 \in \F_p[x]$ nur algebraische Nullstellen in $\_{\F_p}$ und daher hat der Teiler $\_f$ von $x^n - \_1 \in \F_p[x]$ ebenfalls nur einfache Nullstellen wegen $x^n - \_1 = \_f  \_h$.
		Andererseits impliziert $\_f \_g = \_h(x^p) = (\_h(x))^p$ in $\F_p[x]$.
		Insbesondere haben $\_f$ und $\_h$ gemeinsame Nullstellen in $\_F_p$ und daher hat $x^n - \_1 = \_f \_h$ mehrfache Nullstellen, ein Widerspruch.
		Also ist $f(\zeta^p) = f(\omega) = 0$ und damit $\mu_{\omega, \Q} = f = \mu_{\zeta, \Q}$.
		Sei $1 \le d \le n-1$, $\ggT(d,n) = 1$ und $d = p_1 \dotsc p_k$ die Faktorisierung von $d$ in Primelemente.
		$\zeta^{p_1p_k} = \omega^{p_2}$ wie oben folgt $\omega^{p_2} = \zeta^{p_1 p_2}$ ist Nullstelle von $\mu_{\omega, \Q} = \mu_{\zeta, \Q}$ usw.
		Iterativ ist $\zeta^d$ Nullstelle von $f = \mu_{\zeta, \Q}$.
		Also mit \ref{20.1-6} sofort:
		alle primitiven $n$-ten Einheitswurzeln in $E = \Q(\zeta)$ sind Nullstellen von $f = \mu_{\zeta, \Q}$.
		Also ist $\deg f \ge \phi(n)$.
		Sei $\sigma \in G(E / \Q)$.
		Dann ist $(\sigma(\zeta))^n = \sigma(\zeta^n) = \sigma(1) = 1$, d.h. $\sigma(\zeta)$ ist wieder $n$-te Einheitswurzel und primitiv.
		Definiere
		\[
			g(x) = \prod_{\sigma \in G} (x - \sigma(\zeta)) \in \_\Q[x].
		\]
		Dann ist offensichtlich $\tau(g(x)) = \prod_{\sigma \in G} (x - \tau\sigma(\zeta)) = \prod_{\sigma \in G} (x - \sigma(\zeta)) = g(x)$.
		Es folgt $g(x) \in E^G[x] = \Q[x]$.
		Wegen $(x - 1_G(\zeta))$ Faktor von $G$ ist $g(\zeta) = 0$ und $f$ ist Teiler von $g$.
		Also ist $\phi(n) \le \deg f \le \deg g \le \phi(n)$ und daraus folgt $\deg f = \deg g = \phi(n) \stack{HS}= |G| = [E: \Q]$.
		Ist $\_0 \neq \_k = k + n \Z \in \Z / n \Z$ (für $1 \le k \le n-1$), so ist $\_k \in U(\Z / n\Z)$ genau dann, wenn gilt: $\exists l \in Z : \_l\_k = \_1$ und daher $\{z\_k : z \in \Z\} = \Z / n \Z$.
		Dies bedeutet aber, dass $\_k$ die additive Gruppe zyklische Gruppe $(\Z / n \Z, +)$ erzeugt, was äquivalent ist zu $\ggT(k,n) = 1$. 
		Für $\sigma \in G, \sigma(\zeta) = \zeta^k$ für ein $k$ mit $1 \le k \le n-1$ und $\ggT(k, n) = 1$.
		Sind $1 \le i,j \le n-1$, $\ggT(n,i) = 1 = \ggT(n,j)$, $\sigma_i(\zeta) = \zeta^i, \sigma_j(\zeta) = \zeta^j$, dann ist
		\[
			\sigma_i \circ \sigma_j (\zeta) = \sigma_i(\zeta^j) = (\sigma_i(\zeta))^j = (\zeta^i)^j = \zeta^{ij},
		\]
		also $\sigma_i \circ \sigma_j = \sigma_{ij}$.
		Also $G \isomorphic U(\Z / n\Z)$.
	\end{proof}
\end{st}

\begin{kor} \label{20.1-8}
	Sei $\Char K = 0$ und sei $\zeta \in \_K$ primitive $n$-te Einheitswurzel.
	Dann ist $K(\zeta)$ galoissch über $K$ und $[K(\zeta) : K] \le \phi(n)$.
	Die Galoisgruppe $G = G(K(\zeta) / K)$ ist isomorph zu einer Untergruppe von $U(\Z / n\Z)$ und daher insbesondere abelsch.
	\begin{proof}
		Wegen $\Char K = 0$ haben wir $\Q \subset K$ und daher ist $\mu_{\zeta,K}$ Teiler von $\mu_{\zeta, \Q}$.
		Insbesondere sind alle Nullstellen von $\mu_{\zeta, K}$ primitive $n$-te Einheitswurzeln und daher eine Potenz $\zeta^k$ von $\zeta$ mit $\ggT(k,n) = 1$, $1 \le k \le n-1$.
		Also ist jede Nullstelle von $\mu_{\zeta, K}$ Potenz von $\zeta$ und daher in $K(\zeta)$ enthalten.
		Also ist $K(\zeta)$ galoissch über $K$.
		Für $\sigma \in G = G(K(\zeta) / K)$ finden wir $1 \le k \le n-1$, $\ggT(n,k) = 1$ mit $\sigma(\zeta) = \zeta^k$ und wie in \ref{20.1-7} definiert $\sigma \mapsto k + n \Z = \_k$ einen injektiven Gruppenhomomorphismus von $G$ in $U(\Z / n\Z)$.
	\end{proof}
	\begin{note}
		Sind $\zeta, \omega$ primitive $n$-te Einheitswurzeln, so kann durchaus $\mu_{\zeta, K} \neq \mu_{\omega, K}$ passieren.
		Dann ist trotzdem $K(\zeta) = K(\omega)$, aber es gibt keinen $K$-Automorphismus $\sigma$ von $G = G(K(\zeta) / K)$ mit $\sigma(\zeta) = \omega$.
	\end{note}
\end{kor}

\begin{df} \label{20.1-9}
	Sei $n \in \N$ und $\zeta \in \C$ primitive $n$-te Einheitswurzeln (\oBdA $\zeta = e^{\f{2\pi i}n}$).
	Das Minimalpolynom $\mu_{\zeta, \Q} \in \Z[x] \subset \Q[x]$ wird \emphdef[Kreisteilungspolynom]{$n$-tes Kreisteilungspolynom} genannt und mit $\Phi_n(x)$ bezeichnet.
\end{df}

Wie finden wir diese Polynome $\Phi_n(x)$?

\begin{st} \label{20.1-10}
	Sei $n \in \N$.
	Dann lässt sich $\Phi_n(x) \in \Z[x]$ rekursiv aus folgender Formel bestimmen:
	\[
		x^n = 1 = \prod_{d \divs n} \Phi_d(x).
	\]
	\begin{note}
		Die Formel ist eine Zerlegung von $x^n - 1$ in irreduzible Faktoren.
	\end{note}
\end{st}

\begin{ex} \label{20.1-11}
	\begin{enumerate}[1)]
		\item
			Ist $p \in \N$ Primzahl, so ist $x^p - 1 = (x-1)(1 + x + x^2 + \dotsb + x^{p-1})$ und $\Phi_p(x) = 1 + x + \dotsb + x^{p-1}$ ist irreduzibel.
			Für $m \in \N$ gilt
			\[
				\Phi_{p^m}(x) = 1 + x^{p^{m-1}} + x^{2p^{m-1}} + \dotsb + x^{(p-1)p^{m-1}} = \Phi_p(x^{p^{m-1}}).
			\]
			Insbesondere ist $\Phi(p^m) = (p-1)p^{m-m} = p^m - p^{m-1}$ (Übung).
		\item
			$\Phi_n(x) \in \Z[x]$.
			Also kann man die Koeffizienten von $\Phi_n(x)$ modulo einer Primzahl $p$ reduzieren und erhält $\_{\Phi_n(x)} \in \F_p[x]$.
			Im allgemeinen ist $\_{\Phi_n} \in \F_p[x]$ jedoch \emph{nicht} irreduzibel.
		\item
			Ist $n$ ungerade in $\N$, so ist $(x^{2n} - 1) = (x^n - 1)(x^n + 1) = -(x^n-1)((-x)^n-1)$ und daher ist $\Phi_{2n}(x) = - \Phi_n(-x)$.
		\item
			Sei $n = p_1^{\nu_1} \dotsb p_s^{\nu_s}$ die Primfaktorzerlegung von $n \in \N$.
			Dann ist
			\[
				\Phi_n(x) = \Phi_{p_1\dotsb p_s} \Big( x^{p_1^{\nu_1 -1 }p_2^{\nu_2 - 1} \dotsb p_s^{\nu_s - 1}} \Big).
			\]
		\item
			Sei $K$ ein beliebiger Körper, $\Char K \ge 0$.
			Sei $\zeta \in \_K$ $n$-te Einheitswurzel ungleich 1.
			Dann ist $1 + \zeta + \dotsb + \zeta^{n-1} = 0$.
			Ist insbesondere $K = \F_q$ mit einer Primzahlpotenz $q$, so sind alle Elemente ungleich 0 von $K$ $(q-1)$-te Einheitswurzeln über $\F_p$, wobei $p := \Char K$.
			Also ist $\sum_{\alpha \in K} \alpha = 0$.

			Somit haben wir die Aussage:
			Ist $p$ Primzahl, $K = \F_p$, so gilt
			\[
				\sum_{1 \le l \le p-1} l \equiv 0 \mod p.
			\]
	\end{enumerate}
\end{ex}

\section{Radikalerweiterungen}

Im Folgenden sei $K$ ein Körper mit $\Char K = 0$.

\begin{df}
	Sei $n \in \N, \alpha \in K$.
	Dann heißt $x^n - \alpha \in K[x]$ reines Polynom.
	Eine Körpererweiterung $K \subset E$ heißt \emphdef{Radikalerweiterung}, wenn es eine Kette
	\[
		K = L_0 \subset L_1 \subset \dotsb \subset L_m = E
	\]
	von Zwischenkörper von $K \subset E$ gibt, so dass für jedes $i = 0, 1, \dotsc, m-1$ gilt: $L_{i+1} = L_i(\alpha_i)$, wobei $\alpha_i$ Nullstelle eines reinen Polynoms über $L_i$ ist.

	Sei $f \in K[x]$.
	Dann heißt die Gleichung $f(x) = 0$ über $K$ \emphdef[auflösbar]{durch Radikale auflösbar}, falls es eine Radikalerweiterung $E$ von $K$ gibt, so dass $f$ über $E$ in Linearfaktoren zerfällt.
\end{df}

\begin{nt} \label{20.2-2}
	$x^n - 1 \in K[x]$ ist ein Spezialfall von \ref{20.1-1} und daher ist $K(\zeta)$ für eine primitive Einheitswurzel $\zeta \in \_K$ eine Radikalerweiterung von $K$.
\end{nt}

\begin{st} \label{20.2-3}
	Sei $n \in \N$ und $K$ enthalte eine primtive $n$-te Einheitswurzel (und damit alle $n$-ten Einheitswurzeln).
	Dann gelten die folgenden Aussagen.
	\begin{enumerate}[i)]
		\item
			Sei $\alpha \in K$ und $\_\beta \in \_K$ mit $\beta^n = \alpha$.
			Dann ist $K(\beta)$ galoissch über $K$ mit zyklischer Galoisgruppe $(\Z / d \Z, +)$, wobei $d \divs n$.
		\item
			Ist $E$ Galoiserweiterung von $K$ vom Grad $n$ mit zyklischer Galoisgruppe, so gibt es ein $\beta \in u_K$ mit $\beta^n = \alpha \in K$ und $E = K(\beta)$ ist Zerfällungskörper von $x^n - \alpha \in K[x]$ über $K$.
	\end{enumerate}
	\begin{proof}
		\begin{enumerate}[i)]
			\item
				Sei $\zeta \in K$ eine primitive $n$-te Einheitswurzel.
				Sei $\beta \in \_K$ irgendeine Wurzel (Nullstelle) von $x^n - \alpha$, so ist $\beta^n = \alpha$ und daher $(\zeta^i \beta)^n = (\zeta^n)^i \beta^n = \beta^n = \alpha$, d.h. $\zeta^i \beta$ ist ebenfalls Nullstelle von $x^n - \alpha$ ($0 \le i \le n-1$).
				Also ist $x^n - \alpha = \prod_{0\le i \le n-1} (x - \zeta^i \beta) \in \_K[x]$.
				Da $\zeta^i \in K$ für alle $0\le i \le n-1$, sind die $\zeta^i \beta \in K(\beta)$.
				Also ist $K(\beta)$ Zerfällungskörper von $x^n - \alpha$ und daher normal über $K$ (separabel wegen $\Char K = $) und daher galoissch über $K$.
				Also sind alle Wurzeln von $\mu_{K, \beta}$ in $K(\beta)$ von der Form $\zeta^i \beta$ für ein $0 \le i \le n -1$.
				Nach \ref{18.4-4} induziert die Wurzel $\zeta^i \beta$ für $0 \le i \le n-1$ für $\zeta^i \beta$ Nullstelle von $\mu_{K, \beta}$ genau einen $K$-Homomorphismus $\sigma_i: K(\beta) \to \_K$ mit $\sigma_i(\beta) = \zeta^i \beta$ und $\im \sigma_i = K(\beta)$, da $K(\beta)$ normal über $K$ ist.
				Also ist $\sigma_i \in G(K(\beta) / K)$.
				Die Abbildung $\sigma_a \mapsto i + n \Z = \_i \in (\Z / n\Z, +)$ ist eine injektive Abbildung von $G = G(K(\beta), K)$ in $(\Z / n\Z, + )$.
				Sind $\sigma_i, \sigma_j \in G$, d.h. $\zeta^i \beta, \zeta^j$ sind Nullstellen von $\mu_{\beta,K}$ und $\sigma_i(\beta) = \zeta^i \beta$, $\sigma_j(\beta) = \zeta^j \beta$, so ist $\sigma_j \circ \sigma_i(\beta) = \sigma_j(\zeta^i \beta) = \zeta^i \sigma_j(\beta) = \zeta^i(\zeta^j \beta) = \zeta^{\_{i+j}}(\beta) = \sigma_{\_{i+j}}(\beta)$.
				Also ist $\sigma_i \mapsto \_i \in (\Z / n \Z, +)$ ein Gruppenhomomorphismus und daher ist $G(K(\beta) / K)$ zyklisch der Ordnung $d$, wobei $d \divs n$.
			\item
				später
		\end{enumerate}
	\end{proof}
\end{st}

\begin{kor} \label{20.2-4}
	Sei $n \in \N, \alpha \in K$ und sei $E$ der Zerfällungskörper von $x^n - \alpha \in K[x]$.
	Sei $\beta \in \_K$ Nullstelle von $x^n - \alpha$ und sei $\zeta \in \_K$ eine primitive $n$-te Einheitswurzeln.
	Dann gilt
	\begin{enumerate}[i)]
		\item
			$E = K(\zeta, \alpha)$ ist Radikalerweiterung von $K$ mit Zwischenkörperkette $K \subset K(\zeta) \subset K(\zeta, \alpha) = E$.
		\item
			$E$ ist galoissch über $K$ und über $K(\zeta), K(\zeta)$ ist galoissch über $K$, $G(E / K(\zeta))$ ist zyklsicher Normalteiler von $G(E / K)$ und $G(K(\zeta) / K) \isomorphic G(E / K) / \allowbreak G(E / K(\zeta))$ ist abelsch.
	\end{enumerate}
	\begin{proof}
		Es gilt $\beta, \zeta \in \_K$, $x^n - a = \prod_{i=0}^{n-1} (x - \zeta^i \beta)$ in $\_K[x]$ wie in \ref{20.1-3}.
		Also ist $K(\zeta, \beta) = E = K(\Set{\zeta^i \beta & i = 0, \dotsc, n-1 })$.
		Also ist $\zeta = \f {\zeta \beta}{\beta} \in E$ und daher ist $K(\zeta) = E$.
		Andererseits ist $\zeta^i \beta \in K(\beta, \zeta)$ für $i = 0, \dotsc, n-1$ und $E = K(\zeta, \beta)$.
		Also ist $K \subset K(\zeta) \subset K(\zeta, \alpha) = E$, wobei $\zeta$ Nullstelle von $x^n - 1$ und $\beta$ Nullstelle von $x^n  - \alpha$.
		Nach \ref{20.1-2} ist $K(\zeta)$ galoissch über $K$ und $E$ ist algebraisch über $K$ und normal über $K$.
		Mit dem Hauptsatz \ref{19.4-21} ist $G(E / K(\zeta)) \Idealof G(E / K)$ und $G(E / K) / G(E / K(\zeta)) \isomorphic G(K(\zeta) / K)$.
		Rest folgt.
	\end{proof}
\end{kor}

\begin{lem} \label{20.2-5}
	Sei $E$ Radikalerweiterung von $K$ (und daher insbesondere endlich über $K$).
	Dann gibt es eine endliche Körpererweiterung $E'$ von $E$, die galoissch über $K$ und Radikalerweiterung von $K$ ist.
	\begin{proof}
		$[E : K] < \infty$, da $E$ Radikalerweiterung ist.
		Nutze Induktion über $[E : K]$.
		Für $[E : K] = 1$ ist nichts zu zeigen.
		Sei $[E : K] \ge 2$.
		Nach Voraussetzung gibt es ein $K \subset L \subset E = L(\beta)$ mit $\alpha \not\in L$ und $L$ ist Radikalerweiterung von $K$.
		Also gibt es eine galoissche Radikalerweiterung $L'$ von $K$ mit $L \subset L', [L' : L] < \infty$.
		Also ist $L'$ Zerfällungskörper über ein Polynom aus $K[x]$.
		Sei $G' = G(L' / K) \ni \sigma$, ($\alpha \in L' \implies \sigma(\alpha) \in L')$.
		\[
			g(x) = \prod_{\sigma \in G'} (x^n - \underbrace{\sigma(\alpha)}_{=\sigma(\beta^n)})
			\in L'[x].
		\]
		Wegen $\sigma g = g$ für alle $g \in G$ ist $g \in K[x]$.
		$x^n - \alpha$ ist Faktor von $g(x)$, also $\beta \in E'$.
		Setze $E'$ als Zerfällungskörper.
		Also ist $E$ auch Zerfällungskörper von $fg \in K[x]$.
		Also ist $E'$ galoissch über $K$ und $E = L(\beta) \subset E'$.
		Da jede Nullstelle von $g \in K[x] \subset L'[x]$ eine Nullstelle eines Polynoms $x^n - \sigma(\alpha)$ ist für $\sigma \in G'$ und damit $n$-te Wurzel von $\sigma(\alpha) \in L'$ ist, ist $E'$ Radikalerweiterung von $L'$.
		$L' \subset L(\beta_1) \subset (L(\beta_1))(\beta_2) \subset L(\beta_1, \dotsc, \beta_m) = E$ mit Nullstellen $\beta_i$ für $1\le i \le m$.
		Da $L'$ Radikalerweiterung von $K$ ist und $E'$ Radikalerweiterung von $L'$ ist $E'$ Radikalerweiterung von $K$
		(durch  Hintereinanderfügen der Zwischenkörperketten von $L'$ über $K$ und $E'$ über $L'$).
	\end{proof}
\end{lem}

\begin{st} \label{20.2-6}
	Sei $E$ Radikalerweiterung von $K$.
	Dann gibt es eine endliche, galoissche Radikalerweiterung $E'$ über $K$ mit Zwischenkörperkette $K = L_0' \subset L_1' \subset \dotsb \subset L_m' = E'$, so dass gilt:
	\begin{enumerate}[i)]
		\item
			$L_{i+1}' = L_i'(\alpha_{i+1})$ mit $\alpha_{i+1}^{k_{i+1}} =: \beta_i \in L_i'$ für ein $k_{i+1} \in \N$,
		\item
			$L_{i+1}'$ ist galoissch über $L_i'$ für $i = 1, \dotsc, m$.
	\end{enumerate}
	\begin{proof}
		Nach \ref{20.2-5} können wir annehmen, dass $E$ galoissch über $K$ ist.
		Es existiert also eine Kette von Zwischenkörpern $K = L_1 \subset L_2 \subset \dotsb \subset L_m = E$ mit $\alpha_{i+1} \in L_{i+1}$ und $\alpha_{i+1}^{k_{i+1}} = \beta_i \in L_i$, $L_{i+1} = L_i(\alpha_{i+1})$ für $i=1, \dotsc, m$.
		Sei $k = k_2 \dotsb k_m$ und sei $\zeta \in \_K$ eine primitive $k$-te Einheitswurzel.
		Dann ist $L_1' = K(\zeta)$ galoissch über $K$ mit abelscher Galoisgruppe $G(L_1' / K) \le U(\Z / n\Z)$.
		Sei $L_i' = L_i(\zeta)$ für $i = 2, \dotsc, m-1$ und $L_m' = E' = E(\zeta)$.
		Da $E$ endlich gaoissch über $K$, ist $E$ Zerfällungkörper eines Polynoms $f \in K[x]$ und daher ist $E' = E(\zeta)$ Zerfällungskörper von $f \cdot (x^n - 1) \in K[x]$.
		Also ist $E'$ galoissch über $K$.
		\[
			K \subset K(\zeta) = L_1' = K(\zeta) \subset L_2' = L_2(\zeta) \subset \dotsb \subset L_m' = L_m(\zeta) = E(\zeta) = E'.
		\]
		Nun ist $L_{i+1}' = L_{i+1}(\zeta) = L_i (\zeta, \alpha_{i+1}) = L_i'(\alpha_{i+1})$ mit $\alpha_{i+1}^{k_{i+1}} = \beta_i \in L_i \subset L_i'$.
		Mit der $k$-ten primtiven Einheitswurzel $\zeta$ enthält $K(\zeta)$ alle $k_i$-ten Einheitswurzeln.
		Nach \ref{20.2-4} ist $L_{i+1}'$ galoissch über $L_i'$ mit zyklischer Galoisgruppe.
		Da $E' = E(\zeta)$ endlicher Gaoiserweiterung von $E$ ist, ist $E'$ endliche Galoiserweiterung von $K$ und auch Radikalerweiterung von $K$.
	\end{proof}
\end{st}

\begin{nt} \label{20.2-7}
	In der Situation von \ref{20.2-6} (wobei wir „$'$“ hier weglassen) impliziert der Hauptsatz \ref{19.4-21} der Galoistheorie nun:
	Sei $H_i = G(E / L_i)$, dann ist
	\[
		(1) = H_m \le H_{m-1} \le \dotso \le H_1 \le H_0 = G(E / K)
	\]
	eine Kette von Untergruppen von $G(E / K)$ mit $H_{i+1} \Idealof H_i$ und $H_i / H_{i+1} = G(E / L_i) / G(E / L_{i+1} \isomorphic G(L_{i+1} / L_i)$, da jetzt $L_{i+1}$ galoissch über $L_i$ ist.
	Entsteht $L_{i+1}$ aus $L_i$ durch Adjunktion einer Einheitswurzel, wenden wir \ref{20.1-6} und sonst \ref{20.2-3} an und schließen in allen Fällen: $H_i / H_{i+1}$ ist abelsch.

	Endliche Gruppen mit einer solchen Kette von Untergruppen heißen \emphdef{auflösbar}.
\end{nt}

\coursetimestamp{16}{06}{2014}

\section{Auflösbare Gruppen}

\begin{df} \label{20.3-1}
	Sei $G$ eine Gruppe.
	Eine Kette
	\[
		G = G_0 \ge G_1 \ge \dotsb \ge G_m = (1)
	\]
	mit $G_{i+1} \Idealof G_i$ heißt \emphdef{Subnormalbasis} von $G$.
	Besitzt $G$ eine Subnormalreihe mit abelschen Faktoren $G_i / G_{i+1}$, so heißt $G$ auflösbar.
\end{df}

Ein Satz von Feit-Thompson (1961) besagt:
Sei $G$ eine endliche Gruppe und sei $|G| \in \N$ ungerade, dann ist $G$ auflösbar.

\begin{df} \label{20.3-2}
	Sei $G$ eine Gruppe und seien $h,g \in G$.
	Dann heißt $[g, h] = ghg^{-1}h^{-1} \in G$ Kommutatior von $g$ und $h$ (da $[g,h]hg = ghg^{-1}h^{-1}hg = gh$ ist).
\end{df}

\begin{lem} \label{20.3-3}
	Sei $G$ eine Grupe, $g, h, f \in G$.
	Dann gilt
	\begin{enumerate}[i)]
		\item
			$[g, h]^{-1} = [h, g]$,
		\item
			$[g, h]^f = f^{-1} [g,h] f = [g^f, h^f] = [f^{-1}g f, f^{-1} h f]$,
		\item
			Sei $\sigma \in \Aut(G)$ Automorphismus von $G$.
			Dann ist $\sigma([g,h]) = [\sigma(g), \sigma(h)]$.
	\end{enumerate}
	\begin{proof}
		Direkte Rechnung.
	\end{proof}
\end{lem}

\begin{kor}
	Sei $G$ eine Gruppe.
	Dann heißt die von allen Kommutatoren $[g,h]$, $g, h \in G$ erzeugte Untergruppe von $G$ \emphdef{Kommutatoruntergruppe} von $G$ und wird mit $[G, G] = G'$ bezeichnet.
	Sie ist nicht nur ein Normalteiler von $G$, sondern sogar eine \emphdef{charakteristische Untergruppe}, wobei $H \le G$ charakteristische Untergruppe genannt wird, falls $\sigma(H) = H$ für alle $\sigma \in \Aut(G)$.
	Darüber hinaus gilt
	\begin{enumerate}[i)]
		\item
			$G / [G, G]$ ist abelsch,
		\item
			Ist $H \le G$ mit $[G, G] \subset H$, so ist $H \Idealof G$ und $G / H$ ist abelsch,
		\item
			$H \Idealof G, G / H$ abelsche impliziert $[G, G] \subset H$.
			So ist $[G, G]$ der kleinste Normalteiler von $G$ mit abelscher Faktorgruppe.
	\end{enumerate}
	\begin{proof}
		Sei $\scr K = \Set{ [g,h] \in G & g,h \in G }$.
		Dann ist $[G, G] = \bigcap_{\scr K \subset H \le G} H \le G$.
		Wegen $\scr K' = \scr K$ besteht aus allen endlichen Produkten von Elementen von $\scr K$.
		Wegen $\scr K^f = \Set{ [g,h]^f & g, h \in G } = \scr K$ ($f \in G$) ist $[G, G]$ invariant unter Konjugation, d.h. $[G, G] \NormalDivisor G$.
		Wegen $\sigma(\scr K) = \scr K$ ist $[G, G]$ sogar invariant unter allen Automorphismen $\sigma$ von $G$ und daher charakteristisch in $G$.

		Klar ist $[g,h] = 1 \iff ghg^{-1}h^{-1} = 1 \iff gh = hg$.
		Also ist $G$ abelsch genau dann, wenn $[G, G] = (1)$.

		\begin{enumerate}[i)]
			\item
				Sei $H = [G, G] \NormalDivisor G$, $g, h \in G$.
				Dann ist in $G / H$:
				\[
					[gH, hH]
					= gH hH (gH)^{-1}(hH)^{-1}
					= ghg^{-1}h^{-1} H
					= H
					= 1_{G / H}.
				\]
				Also ist $G / H$ abelsch.
			\item
				Ist $[G, G] \le H \le G$, so ist
				\[
					H / [G,G] = \Set{ h [G,G] \in G / [G, G] & h \in H } \le G / [G, G]
				\]
				und ist daher abelsch.
				Für $g, f \in G$ ist $gH fH = fH gH$ und daher $(fH)^{-1} (gH) fH = gH$, d.h. $H \NormalDivisor G$.

				(Die Isomorphiesätze liefern eine Korrespondenz zwischen den Untergruppen $H \le G$ mit $[G, G] \subset H$ und den Untergruppen von $G / [G,G]$)
			\item
				Sei $H \NormalDivisor G$, $g, f \in g$, $G / H$ abelsch.
				Dann ist
				\[
					gfg^{-1}f^{-1} H
					= gH fH (gH)^{-1} (fH)^{-1}
					= [gH, fH]
					= 1_H
					= H.
				\]
				Also ist $gf g^{-1}f^{-1} \in H$, $\scr K \subset H$, $[G, G] \subset H$.
		\end{enumerate}
	\end{proof}
\end{kor}

\begin{df} \label{20.3-5}
	Sei $G$ eine Gruppe.
	Die $i$-te iterierte Kommutatoruntergruppe $D_i(G)$, $i \in \N$ wird induktiv definiert als
	\begin{enumerate}[i)]
		\item
			$D_0(G) = G, D_1(G) = [G, G]$,
		\item
			Für $i \ge 1$: $D_i(G) := [D_{i-1}(G), D_{i-1}(G)] \NormalDivisor(G)$
	\end{enumerate}
	So ist $D_i(G) \le G, G = D_0(G) \ge D_1(G) \ge \dotsc \ge D_i(G)$ uund $D_i(G) \NormalDivisor D_{i-1}(G)$ mit $D_{i-1}(G) / D_i(G)$ abelsch.
\end{df}

\begin{st} \label{20.3-6}
	$G$ ist genau dann auflösbar, falls $D_n(G) = (1)$ ist für ein $n \in \N$.
	\begin{proof}
		\begin{segnb}{\ProofImplication*}
			Dann ist $G = D_0(G) \ge D_1(G) \ge \dotsc \ge D_{n-1}(G) \ge D_n(G) = (1)$ Subnormalreiche von $G$ mit abelschen Faktoren $D_i(G) / D_{i+1}(G)$ für $i = 0, \dotsc, n-1$ wegen $D_{i+1}(G) = [D_i(G), D_i(G)]$ (siehe \ref{20.3-4}).

		\end{segnb}
		\begin{segnb}{\ProofImplication}
			Sei $G = G_0 \ge G_1 \dotsc \ge G_{n-1} \ge G_n = (1)$ Subnormalreihe von $G$ mit abelschen Faktoren.
			Wir zeigen $D_i(G) \subset G_i$ (dann ist $D_n(G) \subset G_n = (1)$, also $D_n(G) = (1)$).
			Für $i = 0$ ist $D_0(G) = G \subset G = G_0$.
			Sei $i > 0$.
			Angenommen wir haben $D_i(G) \subset G_i$ schon gezeigt.
			Da nach Voraussetzung $G_i / G_{i+1}$ abelsch ist, ist nach \ref{20.3-4} iii) $[G_i, G_i] \subset G_{i+1}$.
			Wegen $D_i(G) \subset G_i$ ist daher $D_{i+1}(G) = [D_i(G), D_i(G)] \subset [G_i, G_i]\subset G_{i+1}$.
		\end{segnb}
	\end{proof}
\end{st}

\begin{st} \label{20.3-7}
	Untergruppen und epimorphe Bilder von auflösbaren Gruppen sind auflösbar.
	Ist $G$ Gruppe, $H \NormalDivisor G$, so ist $G$ auflösbar genau dann, wenn $H$ und $G / H$ auflösbar sind.
	% Kurze exakte Folge (1) -> H -> G -> G / H -> (1)
	\begin{proof}
		\begin{enumerate}[i)]
			\item
				Sei $H \le G$ und $G$ auflösbar.
				Dann ist $D_n(G) = (1)$ für ein $n \in \N$ nach \ref{20.3-6}.
				Klar ist $[H, H] \subset [G, G]$ und allgemeiner $D_i(H) \le D_i(G)$.
				Also ist $D_n(H) = (1)$, wenn $D_n(G) = 1$, d.h. $H$ ist auflösbar nach \ref{20.3-6}.
			\item
				Sei $N$ epimorphes Bild von $G$, d.h. sei $\pi: G \to N$ Gruppepimorphismus mit $\ker H \NormalDivisor G$ und sei $G$ auflösbar.
				Sei \oBdA $N = G / H$.
				Wegen $\pi([g,h]) = \pi(ghg^{-1}h^{-1}) = \pi(g)\pi(h)\pi(g)^{-1}\pi(h)^{-1} = [\pi(g), \pi(h)] \in [N, N]$ ist $\pi([G,G]) = [\pi(G), \pi(G)] = [N, N]$.
				Allgemein ist $\pi(D_i(G)) = D_i(\pi(G)) = D_i(N)$.
				Wegen $D_n(G) = (1)$ für ein $n \in \N$, folgt $D_n(N) = D_n(\pi(G)) = \pi(D_n(G)) = (\pi(1)) = (1_N)$ und daher $N$ auflösbar nach \ref{20.3-6}.
			\item
				Sei $H \NormalDivisor G$.
				Ist $G$ auflösbar, so auch $H$ nach i) und $G / H$ nach ii).
				Seien $H$ und $G / H$ auflösbar.
				Dann gibt es ein $n \in \N$ so, dass $D_n(G / H) = (1_{G/H})$ ist.
				Sei $\pi: G \to G /H = N$ die natürliche Projektion.
				Wie oben ist $\pi(D_i(G)) = D_i(G / H)$ und daher ist $D_n(G) \subset H$ ($\pi(D_n(G))) = D_n(\pi(G)) = D_n(G / H) = (1_{G/H})$.
				So ist $D_{n+1}(G) \subset [H, H]$ nd allgemein $D_{n+k}(G) \subset D_k(H)$.
				Wegen $D_m(H) = (1)$ existiert $m \in \N$ ist daher $D_{n+m}(G) \subset D_m(H) = (1)$ und damit ist $G$ auflösbar nach \ref{20.3-6}.
		\end{enumerate}

	\end{proof}
\end{st}

\begin{df} \label{20.3-8}
	Eine Gruppe $G$ ist \emphdef{einfach}, falls (1) und $G$ die einzigen Normalteiler von $G$ sind.
	Eine endliche abelsche Gruppe ist einfach genau dann, wenn ihre Mächtigkeit eine Primzahl $p$ ist.
	\begin{note}
		Einfache, nicht abelsche Gruppen sind \emph{nicht} auflösbar.
	\end{note}
\end{df}

Alle endlichen, einfachen Gruppen wurden klassifiziert (1982 / 2003).

Sei $K$ ein Körper mit $\Char K = 0$.
Eine Körpererweiterung $K \subset E$ heißt \emphdef{Radikalerweiterung}, wenn es eine Kette
\[
	K = L_0 \subset L_1 \subset \dotsb \subset L_m = E
\]
von Zwischenkörpern $L_i$ gibt, mit $L_{i+1} = L_i(\alpha_i)$, $\alpha_i^{k_i} \in L_i$ für ein $k_i \in \N$.
In \ref{20.2-6} sahen wir, dass sich die $L_i$ und $E$ durch Einheitswurzeln so erweitern lassen, dass $E$ galoissch über $K$ und $L_{i+1}$ galoissch über $L_i$ ist für $i = 0, \dotsc, m-1$.
Als Korollar haben wir in \ref{20.2-7} gesehen, dass dies zur Folge hat, dass $G(E / K)$ (endliche) aufösbare Gruppe ist.
Eine Gleichung $f(x) = 0$ mit $f \in K[x]$ heißt \emphdef{durch Radikale auflösbar}, falls es eine Radikalerweiterung $E$ von $K$ gibt, so dass $f$ in $E[x]$ in Linearfaktoren zerfällt.

\begin{st} \label{20.3-9}
	Sei $K$ ein Körper mit $\Char K = 0$ und sei $K \subset E \subset \_K$ eine Radikalerweiterung von $K$.
	Dann ist $G(E/K) = \Aut_K(E)$ eine auflösbare Gruppe.
	\begin{proof}
		Nach \ref{20.2-5} und \ref{20.2-6} finden wir einen Erweiterungskörper $E'$ von $E$ mit $[E' : E] < \infty$ und $E'$ ist normal über $K$ (und daher galoissch über $K$) und ist Radikalerweiterung mit Zwischenkörperkette $K = L_0 \le L_1 \le \dotsb \le L_m = E'$ so, dass $L_i$ galoissch über $K$ ist und daher ist nach \ref{20.2-7} $G(E' / K)$ auflösbar.
		Sei $\sigma \in \Aut_K(E)$, so ist $\sigma: E \to E \injto \_K$ ein $K$-Homomorphismus von $E$ in $\_K$.
		Nach dem Fortsetzungssatz \ref{18.4-7} existiert nun $\hat \sigma \in \Hom_K(E', \_K)$ mit $\hat \sigma|_E = \sigma$.
		Da $E'$ normal über $K$ ist $\im \hat \sigma = E' \subset \_K$, d.h. $\hat \sigma \in \Aut_K(E')$.
		Die Abbildung $\sigma \mapsto \hat \sigma$ von $\Aut_K(E)$ in $\Aut_K(E')$ ist injektiver Gruppenhomomorphismus.
		Also ist $\Aut_K(E)$ isomorph zu einer Untergruppe von $\Aut_K(E') = G(E / K)$ und ist daher auflösbar nach \ref{20.3-7}.
	\end{proof}
\end{st}

\begin{nt} \label{20.3-10}
	Ist $K \subset E \subset \_K$ mit $[E : K] < \infty$ und ist $G(E / K)$ auflösbar, so ist $E$ Radikalerweiterung von $K$.
	\begin{proof}
		Siehe Literatur.
	\end{proof}
\end{nt}

\begin{kor} \label{20.3-11}
	Sei $f \in K[x]$.
	Dann ist $f(x) = 0$ genau dann durch Radikale auflösbar, wenn der Zerfällungskörper $E$ von $f$ über $K$ auflösbare Galoisgruppe über $K$ hat, d.h. $G(E / K)$ ist auflösbar.
	\begin{proof}
		Sei \oBdA $f$ separabel über $K$.
		\begin{seg}{\ProofImplication}
			Sei $E'$ Radikalerweiterung so, dass $f$ in $E'[x]$ in Linearfaktoren zerfällt und sei $E$ der Zerfällungskörper von $f$ über $K$.
			Dann ist $E \subset E', E = K(\alpha_1, \dotsc, \alpha_n)$ mit $f = (x-\alpha_1) \dotsb (x-\alpha_n) \in \_K[x]$ und $\alpha_1, \dotsc, \alpha_n \in E'$ sein müssen.
			Nach \ref{20.2-5} können wir annehmen, dass $E'$ normal über $K$ ist und nach \ref{20.3-9} ist $G(E' / K)$ auflösbar.
			Da $E$ galoissch über $K$, ist $G(E' / E) \NormalDivisor G(E' / K)$ und daher ist $G(E / K) \isomorphic G(E' / K) / G(E' / E)$.
			Also ist $G(E / K)$ epimorphes Bild der auflösbaren Gruppe $G(E' / K)$ und daher auflösbar nach \ref{20.3-7}.
		\end{seg}
		\begin{seg}{\ProofImplication*}
			Siehe \ref{20.3-10}, wird hier ausgelassen.
		\end{seg}
	\end{proof}
\end{kor}

Zurück zur allgemeinen Glichung vom Grad $n$:
sei $K$ ein Körper mit $\Char K = 0$, $a_0, a_1, \dotsc, a_{n-1}$ sind Unbestimmte und $F = K(a_0, \dotsc, a_{n-1})$ ist der Körper der rationalen Funktionen über $K$ in den Unbestimmeten $a_0, \dotsc, a_{n-1}$.
Dann ist $f(x) = a_0 + a_1 x + \dotsb + a_{n-1} x^{n-1} + x^n$ und $f(x) = 0$ ist die allgemeine Gleichung vom Grad $n$ über $K$.
Sei $E$ der Zerfällungskörper von $f$ über $K$.
\ref{20.3-11} sagt nun: $f$ ist durch Radikale auflösbar impliziert, dass $G(E / F)$ auflösbare Gruppe ist.
Also zeigen wir $G(E / F) \isomorphic \S_n$ (symmetrische Gruppe) und dann $\S_n$ ist nicht auflösbar für $n \ge 5$.

Dann kann $f = 0$ für $n \ge 5$ nicht durch Radikale auflösbar sein.
Warum ist $G(E / F) \isomorphic \S_n$?

Ein emotionales Argument: Die Unbestimmten $a_0, \dotsc, a_{n-1}$ über $K$ sind alle „gleich gut“, permutiert man diese, so kommt doch wieder die allgemeine Gleichung vom Grad $n$ raus.

Im nächsten Abschnitt wollen wir das rigoros beweisen.


\section{Die Galoisgruppe der allgemeinen Gleichung}

Wir brauchen zunächst noch folgenden Verallgemeinerung des Satzes \ref{19.4-15} von Artin.

\begin{st} \label{20.4-1}
	Sei $K \subset E$ beliebige (nicht notwendigerweise algebraische) Körpererweiterung und sei $H$ \emph{endliche} Untergruppe von $\Aut_K(E) = G(E / K), L = E^H$.
	Dann ist $E$ algebraisch, endlich und galoissch über $L$ mit Galoisgruppe $G(E / L) = H$.
	Insbesondere ist $[E : L] = |H| < \infty$.
\coursetimestamp{20}{06}{2014}
	\begin{proof}
		Sei $\alpha \in E$ und sei $\sigma_1 = 1, \sigma_2, \dotsc, \sigma_k \in H$ so, dass $\sigma_i(\alpha) \neq \sigma_j(\alpha)$ für $1 \le i, j \le k, i \neq j$ und so, dass für alle $i \in H$ ein $1 \le i \le k$ existiert mit $\tau(\alpha) = \sigma_i(\alpha)$.
		So ist $\Set{ \sigma_i(\alpha) & i = 1, \dotsc, k } = \Set{ \tau(\alpha) & \tau \in H}$ und $\Set{ \sigma(\alpha_i) & i = 1, \dotsc, k } = \Set{ \tau \sigma_i(\alpha) & i = 1, \dotsc, k }$ für alle $\tau \in H$.
		Sei $f = \prod_{i=1}^k (x - \sigma_i(\alpha)) \in E[x]$ ($\sigma_i(\alpha) \in E$ für $i = 1, \dotsc, k$).
		Für $\tau \in H$ gilt
		\[
			\tau f = \prod_{i=1}^k (x - \tau \sigma_i(\alpha))
			= \prod_{i=1}^k (x - \sigma_i(\alpha))
		\]
		Also ist $f \in L[x], L = E^H$ und $f$ ist separabel über $L$.
		Wegen $\sigma_1(\alpha) = \alpha$ ist $f(\alpha) = 0$ und daher ist $\alpha \in E$ algebraisch über $L$ und separabel über $L$.
		Da $E$ sämtliche Nullstellen von $f$ enthält, enthält $E$ den Zerfällungskörper von $f$ über $L$;
		dies gilt für alle $\alpha \in E$.
		Daher ist $E$ normal über $L$ und daher galoissch über $L$.

		Klar ist $H \le G(E / L)$ nach Konstruktion und daher folgt die Behauptung $H = G(E / L)$ aus \ref{20.4-1}, wenn man dort $K$ durch $L$ ersetzt.
	\end{proof}
\end{st}

Sei $K$ ein Körper und seien $t_1, \dotsc, t_n$ Unbestimmte, $R = K[t_1, \dotsc, t_n]$ der Polynomring, $E = Q(R) = K(t_1, \dotsc, t_n)$ der Körper der rationalen Funktionen in den Unbestimmten $t_1, \dotsc, t_n$ über $K$.
Permutationen $\sigma \in \S_n$ induzieren $K$-Automorphismen von $R$, bzw. $E$ wie folgt.

\begin{df} \label{20.4-2}
	Sei $\sigma \in \S_n$.
	Dann wird durch $\sigma(t_i) = t_{\sigma(i)}$ ein $K$-Algebra-Automorphismus von $R$ und von $E$ induziert (universelle Eigenschaft von Polynomringen \ref{15.1-17} und Fortsetzung auf Quotientenkörper \ref{15.2-4}).
	Offensichtlich ist $\S_n$ auf diese Weise (endliche) Untergruppe von $\Aut_K(E) = G(E / K)$.
\end{df}

\begin{st} \label{20.4-3}
	Sei $K$ Körper, $E = K(t_1, \dotsc, t_n)$ und $\S_n \le \Aut_K(E)$, wie in \ref{20.4-2}.
	Sei $L = E^{\S_n}$ der Fixkörper von $E$ unter $\S_n$.
	Dann ist $E$ algebraisch und galoissch über $L$ mit Galoisgruppe $\S_n = G(E / L)$.
	\begin{proof}
		Klar mit \ref{20.4-1}.
	\end{proof}
\end{st}

Wir wollen jetzt den Fixkörper $L$ unter $\S_n$ näher beschreiben:
Eine rationale Funktion (Polynom) $f(t_1, \dotsc, t_n) \in E$ (bzw. $R$) heißt \emphdef{symmetrisch}, falls $\sigma f = f$ ist für alle $\sigma \in \S_n$ (d.h. falls $f \in L$ ist).
So ist
\[
	f(t_1, \dotsc, t_n) \in L
	\iff
	\forall \sigma \in \S_n : f(t_1, \dotsc, t_n) = f(t_{\sigma(1)}, \dotsc, t_{\sigma(n)}).
\]
Hier sind einige spezielle symmetrische Polynome:

\begin{ex} \label{20.4-4}
	Die elementar symmetrischen Polynome
	\begin{align*}
		s_1 &= t_1 + \dotsb + t_n \\
		s_2 &= \sum_{1 \le i < j \le n} t_i t_j  \\
		s_3 &= \sum_{1 \le i < j < k \le n} t_i t_j t_k \\
		\vdots\; &= \qquad \vdots \\
		s_{n-1} &= \sum_{1 \le i_1 < \dotsb < i_{n-1} \le n} t_{i_1} t_{i_2} \dotsb t_{i_{n-1}} \\
		s_n &= t_1 t_2 \dotsb t_n
	\end{align*}
	Offensichtlich erzeugen $s_1, \dotsc, s_n$ über $K$ eine $K$-Unteralgebra $K[s_1, \dotsc, s_n]$ von $K[t_1, \dotsc, t_n]$ und es gilt $K[s_1, \dotsc, s_n] \subset L = E^{\S_n}$.
\end{ex}

Klar ist: $K[s_1, \dotsc, s_n] \subset K[t_1, \dotsc, t_n]$.
Nach \ref{15.1-17} induziert $t_i \mapsto s_i$ einen $K$-Algebrahomomorphismus $\eps$ von $K[t_1, \dotsc, t_n]$ auf $K[s_1, \dotsc, s_n]$, der darin besteht, dass man in Polynomen $f(t_1, \dotsc, t_n) \in K[t_1, \dotsc, t_n]$ die Variablen $t_1, \dotsc, t_n$ durch die (symmetrischen) Polynome $s_1, \dotsc, s_n$ ersetzt.

\begin{st} \label{20.4-5}
	$\eps$ ist ein $K$-Algebra-Isomorphismus.

	Insbesondere ist $K[s_1, \dotsc, s_n]$ $K$-isomorph zu $K[t_1, \dotsc, t_n]$ und daher ist $K(s_1, \dotsc, s_n)$ $K$-isomorph zu $E = K(t_1, \dotsc, t_n)$.
	\begin{proof}
		Sei $f(t_1, \dotsc, t_n) \in \ker \eps$, dann ist $0 = f(s_1, \dotsc, s_n)$ wieder ein Polynom $g(t_1, \dotsc, t_n) \in K[t_1, \dotsc, t_n]$, $g = 0$, ein Widerspruch (durch Ausmultiplizieren).
		Also ist $\eps$ injektiv und surjektiv sowieso, also ein $K$-Algebra-Isomorphismus.
	\end{proof}
\end{st}

Mit \ref{15.2-4} lässt sich nun die Einbettung $K[s_1, \dotsc, s_n] \subset K[t_1, \dotsc, t_n]$ zu einer $K$-Einbettung $K(s_1, \dotsc, s_n) \subset K(t_1, \dotsc, t_n)$ fortsetzen, d.h. wir haben gezeigt:

\begin{kor} \label{20.4-6}
	Sei $E = K(t_1, \dotsc, t_n)$ der Körper der rationalen Funktionen über $K$ in den Unbestimmten $t_1, \dotsc, t_n$.
	Dann sind die elementarsymmetrischen Funktionen $s_1, \dotsc, s_n$ in den $t_i$ algebraisch unabhängig und erzeugen einen Unterkörper $K(s_1, \dotsc, s_n)$ von $E$, der in $L = E^{\S_n}$ enthalten ist: $K \subset K(s_1, \dotsc, s_n) \subset L \subset E$.
\end{kor}

Wir betrachten das Polynom $f(x) \in E[x]$, wobei $E = K(t_1, \dotsc, t_n)$, definiert durch
\[
	f(x) = (x-t_1)(x-t_2) \dotsb (x-t_n) \in E[x].
\]

\begin{lem} \label{20.4-7}
	Sei $E = K(t_1, \dotsc, t_n)$ wie oben und $f(x) = \prod_{i=1}^n (x - t_i) \in E[x]$.
	Dann ist
	\[
		f(x) = x^n - s_1 x^{n-1} + s_2 x^{n-2} - \dotsb + (-1)^{n-1} s_{n-1} x + (-1)^n s_n
	\]
	und ist daher in $K(s_1,\dotsc, s_n)[x]$ enthalten.
	So ist $E$ der Zerfällungskörper von $f(x) \in K(s_1, \dotsc, s_n)[x]$ über $K(s_1, \dotsc, s_n) \subset E$.
	\begin{proof}
		Um den Koeffizienten von $x^{n-i}$ von $f(x)$ zu berechnen, muss man alle Produkte
		\[
			(-t_{\nu_1})(-t_{\nu_2}) \dotsb (-t_{\nu_i})
			= (-1)^i t_{\nu_1} \dotsb t_{\nu_i}
		\]
		aufsummieren mit $1 \le \nu_1 < \nu_2 < \dotsb < \nu_i \le n$ und damit erhält man $(-1)^i s_i$.
		Offensichtlich gilt $K(t_1, \dotsc, t_n) \subset (K(s_1,\dotsc, s_n))(t_1, \dotsc, t_n) \subset E = K(t_1, \dotsc, t_n)$.
		Also ist $(K(s_1,\dotsc, s_n))(t_1, \dotsc, t_n) = E$ und $E$ entsteht aus $K(s_1,\dotsc, s_n)$ durch Adjunktion der Nullstellen $t_1, \dotsc, t_n$ von $f \in K(s_1,\dotsc, s_n)[x]$.
		Also ist $E$ galoissch über $K(s_1, \dotsc, s_n)$.
	\end{proof}
\end{lem}

\begin{kor} \label{20.4-8}
	Sei $E = K(t_1, \dotsc, t_n)$ und seien $s_1, \dotsc, s_n$ die elementarsymmetrischen Funktionen in den $t_1, \dotsc, t_n$ und sei $L$ der Körper der symmetrischen rationalen Funktionen in den $t_1, \dotsc, t_n$.
	So ist $L = E^{\S_n}$.
	Dann ist $L = K(s_1, \dotsc, s_n)$ und damit ist jede symmetrische rationale Funktion in den $t_1, \dotsc, t_n$ eindeutig gegeben als rationale funktion in den elementarsymmetrischen Funktionen $s_1, \dotsc, s_n$.
	Insbesondere ist $E$ algebraisch und galoissch über $K(s_1, \dotsc, s_n)$ mit Galoisgrupppe $\S_n$.
	\begin{proof}
		Nach \ref{19.5-1} ist $G(E / L_1) \le \S_n$ (als Permutationen der Nullstellen von $f$), $L_1 = K(s_1, \dotsc, s_n)$.
		Wir wissen bereits $G(E / L) = \S_n$.
		Wir haben daher wegen $K \subset L_1 \subset L \subset E$ mit dem Hauptsatz der Galoistheorie
		\[
			\S_n = G(E / L)
			\le G(E / L_1)
			\le \S_n
		\]
		Also ist $G(E / L) = G(E / L_1) = \S_n$ und nach dem Hauptsatz $L = L_1$.
	\end{proof}
\end{kor}

Zurück zur allgemeinen Gleichung vom Grad $n$ über $K$.
Seien $a_0, \dotsc, a_{n-1}$ Unbestimmte, $F = K(a_0, \dotsc, a_{n-1})$, $f = a_0 + a_1 x + \dotsb + a_{n-1} x^{n-1} + x^n \in F[x]$, $E$ der Zerfällungskörper von $f$ über $K$.
Seien $\alpha_1, \dotsc, \alpha_n \in E$ die Nullstellen von $f$ in $E$, so ist $f(x) = (x-\alpha_1)(x-\alpha_2) \dotsb (x- \alpha_n) \in E[x]$ und $E = F(\alpha_1, \dotsc, \alpha_n)$.

\begin{lem} \label{20.4-9}
	$F(\alpha_1, \dotsc, \alpha_n) = K(\alpha_1, \dotsc, \alpha_n)$ und $K(\alpha_1, \dotsc, \alpha_n)$ ist $K$-isomorph zum Körper $K(t_1, \dotsc, t_n)$ der rationalen Funktionen in den Unbestimmten $t_1, \dotsc, t_n$, induziert durch den Auswertungshomomorphismus $\eps: t_i \mapsto \alpha_i$.
	\begin{note}
		Daher ist insbesondere $\alpha_i \neq \alpha_j$ für $1 \le i, j \le n, i \neq j$, d.h. $f$ ist separabel.
	\end{note}
	\begin{proof}
		Offenbar ist $K(\alpha_1, \dotsc, \alpha_n) \subset F(\alpha_1, \dotsc, \alpha_n)$, da $K \subset F$.
		Nun ist $E = F(\alpha_1, \dotsc, \alpha_n) = (K(a_0, \dotsc, a_{n-1}))(\alpha_1, \dotsc, \alpha_n)$.
		Durch Ausmultiplizieren von $f = (x-\alpha_1) \dotsb (x-\alpha_n)$ ergibt sich $a_i = (-1)^{n-i} s_{n-i}(\alpha_1, \dotsc, \alpha_n)$.
		Daher ist $a_i \in K(\alpha_1, \dotsc, \alpha_n)$.
		Also $F(\alpha_1, \dotsc, \alpha_n) \subset K(a_0, \dotsc, a_{n-1})(\alpha_1, \dotsc, \alpha_n) \subset K(\alpha_1, \dotsc, \alpha_n)$ und $E = F(\alpha_1, \dotsc, \alpha_n) = K(\alpha_1, \dotsc, \alpha_n)$.
		Die Abbildung $\eps: K[t_1, \dotsc, t_n] \to K(\alpha_1, \dotsc, \alpha_n): h(t_1, \dotsc, t_n) \mapsto h(\alpha_1, \dotsc, \alpha_n)$ ist $K$-Algebra-Homomorphismus (siehe \ref{15.1-17}).

		Angenommen $\eps$ ist nicht injektiv.
		Sei $h(t_1, \dotsc, t_n) \in \ker \eps$, d.h. $h(\alpha_1, \dotsc, \alpha_n) = 0$.
		Setze
		\[
			H(t_1, \dotsc, t_n)
			:= \prod_{\sigma \in \S_n} h\Big(t_{\sigma(1)}, \dotsc, t_{\sigma(n)} \Big)
			= \prod_{\sigma \in \S_n} \sigma h .
		\]
		Dann ist $\tau H = H$ für alle $i \in \S_n$.
		Also lässt sich $H$ nach \ref{20.4-8} als rationale Funktion in den $s_1, \dotsc, s_n$ (welche Polynome in $t_1, \dotsc, t_n$ sind) schreiben, d.h. es existiert $N \in K[t_1, \dotsc, t_n]$ mit $H(t_1, \dotsc, t_n) = N(s_1, \dotsc, s_n)$.
		Beachte, dass $H(t_1, \dotsc, t_n) = h(t_1, \dotsc, t_n) \prod_{1 \neq \sigma \in \S_n} h(t_{\sigma(1)}, \dotsc, t_{\sigma(n)})$.
		Also ist
		\[
			H(\alpha_1, \dotsc \alpha_n)
			= h(\alpha_1, \dotsc, \alpha_n) \prod_{1 \neq \sigma \in \S_n} \sigma h(\alpha_1, \dotsc, \alpha_n)
			= 0.
		\]
		Andererseits wegen $s_i(\alpha_1, \dotsc, \alpha_n) = \pm a_i$ ist
		\[
			\eps(N(s_1, \dotsc, s_n)) = N(\pm a_{n-1}, \dotsc, \pm a_0) = \tilde N(a_0, \dotsc, a_{n-1}) = 0.
		\]
		Der Zähler $g$ (ein Polynom in $a_0, \dotsc, a_{n-1}$ mit Koeffizienten in $K$) muss ebenfalls $0$ sein, d.h. $a_0, \dotsc, a_{n-1}$ können nicht algebraisch unabhängig über $K$ sein.
		Also ist $\eps: K[t_1, \dotsc, t_n] \to K(\alpha_1, \dotsc, \alpha_n)$ injektiv und kann mit \ref{15.2-4} zu einem injektiven $K$-Homomorphismus $\eps: K(t_1, \dotsc, t_n) \to K(\alpha_1, \dotsc, \alpha_n)$ ausgedehnt werden ($K$-Isomorphismus nach \ref{18.3-7}).
	\end{proof}
\end{lem}

\begin{kor} \label{20.4-10}
	Sei $K$ ein Körper, $F = K(a_0, \dotsc, a_{n-1})$ der Körper der rationalen Funktionen in den Variablen $a_0, \dotsc, a_{n-1}, f = a_0 + a_1 x + \dotsb + a_{n-1} x^{n-1} + x^n \in F[x]$ und sei $E$ der Zerfällungskörper von $f$ über $F$.
	Dann ist $E$ galoissch über $F$ mit Galoisgruppe $G(E / F) \isomorphic \S_n$.
	\begin{proof}
		 Sei $f = (x-\alpha_1) \dotsb (x-\alpha_n) \in E[x]$.
		 Dann ist nach \ref{20.4-9} $E = F(\alpha_1, \dotsc, \alpha_n) = K(\alpha_1, \dotsc, \alpha_n) = K(t_1, \dotsc, t_n)$ durch den $K$-Isomorphismus $\eps: K(t_1, \dotsc, t_n) \to K(\alpha_1, \dotsc, \alpha_n), t_i \mapsto \alpha_i$.
		 Dabei gehen die elementarsymmetrischen Funktionen $s_i(t_1, \dotsc, t_n)$ bis aufs Vorzeichen in $\eps(s_i) = s_i(\alpha_1, \dotsc, \alpha_n) = \pm a_{n-i} \in F$ über und daher auch der Fixkörper $K(s_1, \dotsc, s_n)^L = K(t_1, \dotsc, t_n)^{\S_n}$ in den Körper $f = K(a_0, \dotsc, a_{n-1})$.
		 Also haben wir $K$-Isomorphismen $\eps: K(t_1, \dotsc, t_n) \to K(\alpha_1, \dotsc, \alpha_n) = E$ mit $\eps(L) = \eps(K(t_1, \dotsc, t_n)^{\S_n}) = F$.
		 Also ist $\Aut_F(E) = G(E / F) \isomorphic \Aut_L (K(t_1, \dotsc, t_n)) = \S_n$ (siehe \ref{20.4-8}).
	\end{proof}
\end{kor}

\begin{nt} \label{20.4-11}
	Seien die Voraussetzungen wie in \ref{20.4-10} erfüllt.
	Wegen $|\S_n| = n!$ folgt unmittelbar
	\begin{enumerate}[1.]
		\item
			Da $\deg f = n$ ist, ist $f$ irreduzibel und separabel über $F$.
		\item
			Seien $\alpha_1, \dotsc, \alpha_n$ die Nullstellen von $f$ in $E$ und $L_i := F(\alpha_1, \dotsc, \alpha_i) \subset E$ für $i = 1, \dotsc, n$.
			Dann ist $F = L_0 \subset L_1 \subset \dotsb \subset L_n = E_1$, $L_i = L_{i-1}(\alpha_i)$ mit $\mu_{\alpha_i, L_{i-1}} = (x-\alpha_i)\dotsb (x-\alpha_n) \in L_{i-1}[x]$ für $i = 1, \dotsc, n-1$.
	\end{enumerate}
	\begin{proof}
		Zeige den zweiten Teil, der erste folgt.
		Sei $\mu_i = \mu_{\alpha_i, L_{i-1}}$ von Grad $k_i$.
		Dann ist $[E : F] = [L_1 : L_0 ][L_2 : L_1] \dotsb [E : L_{n-1}] = k_1 \dotsb k_n$.
		Wir haben $[L_1 : L_0] \le n$, wegen $\mu_{\alpha_1, F} \divs f$.
		Weiter $[L_2 : L_1] \le n-1$, da $\mu_{\alpha_2, L_1} \divs (x-\alpha_2) \dotsb (x-\alpha_n)$, usw.
		Also $[L_i : L_{i-1}] \le n - i + 1$.
		Wegen $[E : F] = n!$ folgt die Behauptung.
	\end{proof}
\end{nt}

\coursetimestamp{23}{06}{2014}

\section{Die Einfachheit der alternierenden Gruppen \texorpdfstring{$A_n, n \ge 5$}{Aₙ, n≥5}}

$U(\Z) = \{1, -1\}$ ist eine (multiplikative) zylkische Gruppe mit Erzeuger $-1$ (wegen $(-1)^2 = 1$) und die Abbildung $\sign: \S_n \to U(\Z)$, die jeder Permutation $\sigma \in \S_n$ ihr Signum $\sign$ ist ein Gruppenepimorphismus mit Kern $A_n = \Set{ \sigma \in \S_n & \sign(\sigma) = 1}$ (alternierende Gruppe).
$A_n$ hat Index $2$ in $\S_n$ und ist daher Normalteiler von $S_n$.

Sei $K$ Körper, $f(x) \in K[x]$.
In \ref{20.3-11} sahen wir, dass $f(x)$ nur dann durch Radikale auflösbar ist, wenn die Galoisgruppe des Zerfällungskörpers von $f$ über $K$ eine auflösbare Gruppe ist.
\ref{20.4-10} besagt, dass die Galoisgruppe der \emph{allgemeinen} Gleichung $f(x) = 0$ (d.h. die Galoisgruppe des Zerfällungskörpers von $f$ über $K(a_0, \dotsc, a_{n-1})$) die symmetrische Gruppe $\S_n$ ist, wobei $n = \deg f$.
Also ist $f(x) = 0$ nur dann durch Radikale auflösbar, wenn $\S_n$ auflösbar ist.
$A_n$ ist nicht abelsch für $n \ge 4$ und (wie wir gleich zeigen werden) für $n \ge 5$ einfach, und daher für $n \ge 5$ nicht auflösbar.

\begin{st} \label{20.5-1}
	Sei $n \ge 5$, dann ist die allgemeine Gleichung nicht durch Radikale auflösbar.
\end{st}

Zum Beweis des Satzes fehlt also nur noch der Nachweis, dass $A_n$ für $n \ge 5$ nicht auflösbar ist.
Wir führen diesen, indem wir zeigen, dass $A_n$ für $n \ge 5$ eine einfache Gruppe ist, d.h. außer $(1)$ und $A_n$ keine weiteren Normalteiler besitzt.
Diese Aussage ist natürlich auch per se interessant, $\Set{ A_n & n \ge 5}$ ist eine Serie nicht-abelscher, einfacher Gruppen.


\begin{nt} \label{20.5-2}
	Betrachte die Ausnahmefälle $1 \le n \le 4$:
	\begin{enumerate}[label={$n=\arabic*$:},leftmargin=4em]
		\item
			$A_n = (1) = \S_n$,
		\item
			$A_n = (1) \le \S_n = \{1, (1,2) \}$.
		\item
			$\S_n = \S_3$ hat Ordnung $3! = 6$ und $A_3$ hat daher Ordnung $3$ und kann daher keine echten, nichttrivialen Untergruppen besitzen.
			Daher $A_3 \isomorphic C_3$, $\S_3 / A_3 \isomorphic c_2$, also ist $\S_3$ auflösbar.
		\item
			$|\S_4| = 4! = 24$, also $|A_4| = 12$.
			Es gilt $V_4 \NormalDivisor A_4 \NormalDivisor \S_4$ mit
			\[
				V_4 = \Set{ 1, (12)(34), (13)(24), (14)(23) } \isomorphic C_2 \times C_2,
			\]
			der \emphdef{kleinschen Vierergruppe}.
			Außerdem $A_4 / V_4 \isomorphic C_3, \S_4 / A_4 \isomorphic C_2$.
	\end{enumerate}
\end{nt}

\begin{df} \label{20.5-3}
	Sei $G$ eine Gruppe, $g, h \in G$.
	Dann heißen $g$ und $h$ \emphdef{konjugiert} in $G$, falls ein $x \in G$ existiert mit $g^x = x^{-1}gx = h$.
	Wir schreiben dann $g \sim_G h$.
	$\sim_G$ ist eine Äquivalenzrelation auf $G$ (leichte Übung)
	Die Äquivalenzklassen heißen \emphdef{Konjugationsklassen} von $G$.
	Die Konjugationsklasse, die $g \in G$ enthält, wird mit $g^G$ bezeichnet.

	\begin{note}
		Offenbar ist $g^G = \Set{x^{-1}gx & x \in G} \subset G$.
		Es gilt: $G$ abelsch genau dann, wenn $\forall g\in G : g^G = \Set g$.
	\end{note}
\end{df}

\begin{lem} \label{20.5-4}
	Sei $n \ge 3$.
	Dann wird $A_n$ durch die $3$-Zykeln erzeugt.
	\begin{proof}
		Sei $(a,b,c)$ ein $3$-Zykel, d.h. $a, b, c \in \{1, \dotsc, n\}$ sind paarweise verschieden.
		Dann ist $(a,b,c) = (bc, ac)$ Produkt zweier Transpositionen und daher gerade, d.h. enthalten in $A_n$ (siehe \ref{13.4-19}).
		Jedes Element von $A_n$ kann wieder als Produkt einer geraden Anzahl von Transpositionen geschrieben werden.
		Daher genügt es, zu zeigen, dass jedes Produkt zweier Transpositionen als Produkt von $3$-Zykeln geschrieben werden kann.

		Seien $a, b, c \in \{1, 2, \dotsc, n\}$.
		Wir unterscheiden folgende Fälle
		\begin{enumerate}[i)]
			\item
				Seien jeweils $2$ der $a, b, c, d$ gleich.
				\OBdA gilt $1 = (a,b)(a,b) = (1,2,3)^3$.
			\item
				Sei $a \neq c$.
				Dann ist $(a,b)(b,c) = (a, b, c)$.
			\item
				Seien $a, b, c, d$ paarweise verschieden.
				Dann ist
				\[
					(a,b)(c,d) = (a,b)(b,c)(b,c)(c,d) = (a,b,c)(b,c,d).
				\]
		\end{enumerate}
	\end{proof}
\end{lem}

\begin{lem} \label{20.5-5}
	Sei $n \ge 5$.
	Dann sind die $3$-Zykeln in $A_n$ nicht nur in $\S_n$, sodern schon in $A_n$ konjugiert.
	\begin{proof}
		Seien $\phi = (a_1, a_2, a_3), \psi = (b_1, b_2, b_.)$ zwei $3$-Zykeln in $A_n$.
		Sei $\sigma \in \S_n$ mit $\sigma(a_1) = b_1, \sigma(a_2) = b_2, \sigma(a_3) = b_3$.
		Dann ist nach \ref{13.4-26}
		\[
			\sigma \phi \sigma^{-1} = \sigma (a_1, a_2, a_3) \sigma^{-1} = (\sigma(a_1), \sigma(a_2), \sigma(a_3)) = (b_1, b_2, b_3) = \psi.
		\]
		Wegen $n \ge 5$ gibt es $c, d \in \Set{1, \dotsc, n} \setminus \Set{1, \dotsc, 3}$ mit $c \neq d$.
		Dann sind $\phi$ und $(c,d)$ disjunkte Zykeln und daher ist $\phi (c,d) = (c,d) \phi$ nach \ref{13.4-8}.
		Daher gilt mit $\tau := \sigma (c,d)$
		\[
			\tau \phi \tau^{-1}
			= \sigma (c,d) \phi (c,d)^{-1} \sigma^{-1}
			= \sigma \phi (c,d)(c,d)^{-1} \sigma^{-1}
			= \sigma \phi \sigma^{-1}
			= \psi.
		\]
		Nun ist $\sign \tau = - \sign \sigma$ und daher ist entweder $\sigma$, oder $\tau$ in $A_n$ enthalten.
	\end{proof}
\end{lem}

\begin{st} \label{20.5-6}
	Sei $n \ge 5$.
	Dann ist $A_n$ einfache Gruppe.
	\begin{proof}
		Sei $(1) \neq N \NormalDivisor A_n$.
		Wir zeigen $N = A_n$.
		Da $N$ normal in $A_n$ ist $\sigma \phi \sigma^{-1} \in N$ für alle $\phi \in N, \sigma \in A_n$.
		Also gilt: enthält $N$ einen $3$-Zykel, dann alle $3$-Zykeln wegen \ref{20.5-5} ($n \ge 5$) und daher ist $N = A_n$ nach \ref{20.5-4}.

		Es genügt also, zu zeigen, dass $N$ einen $3$-Zykel enthält.
		Beachte: $N$ kann keine Transpositionen enthalten, da $N \subset A_n$.

		Sei $1 \neq \phi \in N$.
		Ist $\phi$ $3$-Zykel, so sind wir fertig.
		Sei also $\phi$ kein $3$-Zykel.
		Dann kommt in der Zykelzerlegung von $\phi$ ein Zykel der Länge $\ge 4$, oder neben einem $3$-Zykel noch mindestens ein Zykel der Länge $\ge 2$, oder eine gerade Anzahl $\ge 2$ von Transpositionen vor.
		Wir haben also die folgenden Fälle:
		\begin{enumerate}[i)]
			\item
				$\phi = (a,b,c,d, \dotsc) \dotsc$,
			\item
				$\phi = (a,b,c)(d,e, \dotsc) \dotsc$,
			\item
				$\phi = (a,b)(c,d) \dotsc$,
		\end{enumerate}
		wobei $a, b, c, d \in \Set{1, \dotsc, n}$ paarweise verschieden.
		Beachte: Ist $\psi \in A_n$, so ist $\phi \psi \phi^{-1}\psi^{-1} \in N$.
		\begin{enumerate}[i)]
			\item
				Sei $\psi = (a,b,n) \in A_n$.
				Dann ist $\phi \psi \phi^{-1} = (b, c, d)$ und daher ist $\tau = \phi \psi \phi^{-1} \psi^{-1} = (b,c,d)(a,c,b) = (a,d,b) \in N$.
			\item
				Sei $\psi = (a,b,d) \in A_n$.
				Dann ist $\phi \psi \phi^{-1} = (b, c, e)$ und daher ist $\phi \psi \phi^{-1}\psi^{-1} = (b,c,e)(a,d,b) = (a, d, c, e, b) \in N$.
				Wegen i) folgt also, dass $N$ ein $3$-Zykel enthält.
			\item
				Es existiert $1 \le e \le n$ mit $e \not\in \Set{a, b, c, d}$.
				Setze $\psi = (a,c,e) \in A_n$.
				Dann ist $\phi \psi \phi^{-1} = (b, d, \phi(e))$.
				Unterscheide die Fälle:
				\begin{enumerate}[a)]
					\item
						Sei $\phi(e) = e$.
						Dann ist $\phi \psi \phi^{-1} = (b,d,e)$, also $\phi \psi \phi^{-1} \psi^{-1} = (b,d,e)(a,e,c) = (a,b,d,e,c)$.
						Mit i) enthält also $N$ einen $3$-Zykel.
					\item
						Sei $\phi(e) \neq e$.
						Dann ist
						\[
							N \ni \phi \psi \phi^{-1} \psi^{-1}
							= (b, d, \phi(e))(a, e, c).
						\]
						Mit ii) enthält also $N$ einen $3$-Zykel.
				\end{enumerate}
				Also enthält $N$ einen $3$-Zykel und daher alle $3$-Zykel und $A_n = N$.
		\end{enumerate}
	\end{proof}
\end{st}

