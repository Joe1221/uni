\chapter{Ringe und Moduln}


%fixme: def ring
%fixme: def körper
%fixme: def K-Algebra

\coursetimestamp{07}{04}{2014}

\section{Kommutative Ringe und $K$-Algebren}

\begin{conv}
	In diesem Abschnitt sei $R$, soweit nicht anders festgelegt, ein kommutativer Ring mit Einselement $1 = 1_R$.
	% ?: , bzw. eine $K$-Algebra für einen Körper $K$.
\end{conv}

Ein Unterring von $R$ ist eine nichtleere Teilmenge, die abgeschlossen ist bezüglich Addition, Multiplikation und Bildung additiver Inverse.

\begin{df}[Unterring]
	Sei $\emptyset \neq S \subset R$.
	$S$ ist ein \emphdef{Unterring} von $R$, wenn gilt
	\begin{enumerate}[1.]
		\item
			$\forall r,s \in S : r - s \in \S$, d.h. $(S, +)$ ist eine Untergruppe von $(R, +)$,
		\item
			$\forall r,s \in S : rs \in \S$.
	\end{enumerate}
\end{df}

\begin{df}[Ringhomomorphismus]
	Seien $S, R$ Ringe, $f: R \to S$ eine Abbildung.
	Dann heißt $f$ \emphdef[Homomorphismus!Ringhomomorphismus](Ring-)Homomorphismus, falls gilt
	\begin{enumerate}[1.]
		\item
			$f(a + b) = f(a) + f(b)$,
		\item
			$f(ab) = f(a)f(b)$.
	\end{enumerate}
	Ist $f(1_R) = 1_S$ so sagt man „$f$ erhält das Einselement“, oder $f$ ist \emphdef{eins-erhaltend}.

	Wir bezeichnen $\ker f := \{r \in \R : f(r) = 0\}$ als \emphdef{Kern} und $\im f := \{ f(r) : r \in R \} \subset S$ als \emphdef{Bild} von $f$.

	% fixme: Epi-, Mono-, Isomorphismen
\end{df}

\begin{lem} % 15.1-3
	Sei $f: R \to S$ ein Ringhomomorphismus, dann gilt
	\begin{enumerate}[1.]
		\item
			$\ker f$ ist Unterring von $R$,
		\item
			$\im f$ ist Unterring von $S$,
		\item
			Sei $r \in \ker f$ und $x \in \R$.
			Dann ist $rx = xr \in \ker f$.
	\end{enumerate}
	\begin{proof}
		\begin{enumerate}[1.]
			\item
				$r, s \in \ker f$, also $f(r) = 0 = f(s)$ und damit
				\begin{align*}
					f(r - s) = f(r) - f(s) = 0 - 0 = 0 \implies r- s \in \ker f \\
					f(rs) = f(r) f(s) = 0 \cdot 0 = 0 \implies rs = sr \in \ker f
				\end{align*}
			\item
				Seien $x, y \in \im f$, dann existieren $a, b \in \R$ mit $f(a) = x, f(b) = y$.
				Es gilt
				\begin{align*}
					f(a - b) = f(a) - f(b) = x - y \in \im f \\
					f(a b) = f(a) f(b) = x y \in \im f
				\end{align*}
			\item
				Sei $r \in \ker f$, also $f(r) = 0$ und damit
				\[
					f(r k) = f(r)f(x) = 0 f(x) = 0,
				\]
				also $rx = xr \in \ker f$.
		\end{enumerate}
	\end{proof}
\end{lem}

\begin{df}
	Ein Unterring $S$ von $R$ heißt \emphdef{Ideal} von $R$, falls $rs \in S$ gilt für alle $s \in S, r \in R$.
	Wir schreiben dann $S \idealof R$.
	\begin{note}
		Es gilt: $S$ ist Ideal genau dann, wenn $S \neq \emptyset$ und für alle $s_1, s_2 \in S, r \in R$ gilt: $s_1 - s_2 \in S$ und $sr \in S$.
	\end{note}
\end{df}

\begin{ex}
	Sei $S \idealof R$.
	Sei $s \in S$ und sei $s$ invertierbar, d.h. es existiert $s^{-1} \in R$ mit $ss^{-m} = 1_R$.
	Dann ist $S = R$
	\begin{proof}
		Wegen $s \in S, s \in R$ mit $ss^{-1} = 1_R \in S$ gilt % fixme
	\end{proof}
\end{ex}

% lem: 15.1-6 Faktorring ist Ring

% lem: 15.1-7 Faktorringe sind kerne eines Epimorphismus R \to R / I

% fak: 15.1-8
% \{0\} und R sind triviale Ideale
% ν


