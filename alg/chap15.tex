\chapter{Ringe und Moduln}


\setcounter{section}{2}
\section{Der Satz von Gauß}

Wir haben im letzten Abschnitt \ref{15.2} gesehen, dass Hauptidealringe faktoriell sind.
Der Satz von Gauß sagt unter anderem, dass es (viele) UFD's gibt, die keine HIR's sind.
Die Konstruktion hat darüber hineaus einige wichtige (und hübsche) Anwendungen für die späteren Kapitel zur Körpertheorie.

Im Folgenden sei $R$ ein kommutativer Integritätsbereich mit Einselement.
Mit $\Prim(R)$ bezeichnen wir ein System von Repräsentanten der Assoziierungsklassen von Primelementen in $R$.
So ist jedes Primelement von $R$ assoziiert zu genau einem Primelement in $\Prim(R)$, d.h. unterscheidet sich von diesem nur durch eine Einheit in $R$.

\begin{df} \label{15.3-1}
	Sei $R$ UFD, $K = Q(R)$ der Quotientenkörper von $R$.
	Sei $p \in \Prim(R), 0 \neq a \in R$.
	Da $R$ faktoriell ist, finden wir ein eindeutiges $b \in R$ mit $\ggT(b,p) = 1$ und ein eindeutiges $\nu_p(a) \in \N \cup \{0\}$ so, dass $a = p^{\nu_p(a)} b$ ist.
	Für $0 \neq \alpha = \f rs \in K$ ($r, s \in R, s \neq 0$) setzen wir $\nu_p(\alpha) := \nu_p(r) - \nu_p(s) \in \Z$.
	Beachte, dass $\nu_p(\alpha)$ nur von $\alpha$, nicht von der Darstellung von $\alpha$ als Bruch $\alpha = \f st$ abhängt.
	Wir definieren $\nu_p(0) = \infty$.
	Die Abbildung $\nu_p: K \to \Z \cup \{\infty\}$ heißt \emphdef[p-adsche Bewertung]{(exponentielle) $p$-adsche Bewertung von $K$} und $\nu_p(\alpha)$ der \emphdef{Wert} von $\alpha$ an der \emphdef{Stelle} $p\in \Prim(R)$ (genauer \emphdef{Primstelle} $p$).

	Klar ist wegen der eindeutigen Primfaktorzerlegung für Elemente von $R$:
	Sind $\alpha, \beta \in K = Q(R)$, dann ist
	\[
		\nu_p(\alpha \beta) = \nu_p(\alpha) + \nu_p(\beta)
	\]
	für alle $p \in \Prim(R)$.
	Darüber hinaus ist $\nu_p(\alpha) = 0$ für fast alle $p \in \Prim(R)$.
\end{df}

\begin{df} \label{15.3-2}
	Sei wie oben $K = Q(R)$ und sei
	\[
		f = f(x) = \alpha_0 + \alpha_1 x + \dotsb + \alpha_n x^n \in K[x]
	\]
	ein Element aus dem Polynomring.
	Wir setzen $\nu_p(f) = \infty$ für $f = 0$ und
	\[
		\nu_p(f) = \min_{\substack{0\le i \le n \\ \alpha_i \neq 0}} \nu_p(\alpha_i),
	\]
	wobei $p \in \Prim(R)$.
	$\nu_p(f)$ heißt \emphdef[p-Wert]{(exponentieller) $p$-Wert von $f$}.
\end{df}

\begin{nt} \label{15.3-3}
	Sei $\alpha_i = \f {r_i}{s_i}$ mit $r_i, s_i \in R$.
	Für $\alpha_i \neq 0$ können wir $\ggT(r_i, s_i) = 1$ annehmen.
	Ist $p$ Teiler von $r_i$ für alle $i$ mit $\alpha_i \neq 0$, so teilt $p$ kein $s_i$ mit $\alpha_i \neq 0$ und $\nu_p(f) = \nu_p(\ggT(r_i : \alpha_i \neq 0)) > 0$.
	Sonst gibt es ein $i$ mit $\alpha_i \neq 0$ und $\ggT(p, r_i) = 1$.
	Dann ist insbesondere $p$ Teiler von $s_i$ nur, wenn es kein Teiler von $r_i$ ist ($\alpha_i \neq 0$) und daher ist $\nu_p(f) = -\nu_p(\kgV(s_i : \alpha_i \neq 0)) < 0$, d.h. $-1$ mal der Exponent der höchsten Potenz von $p$, die in der Primfaktorzerlegung von den $s_i$ mit $\alpha_i \neq 0$ vorkommt.

	Insbesondere ist $\nu_p(f) \ge 0$ für alle $f \in R[x] \subset K[x]$.
\end{nt}

\begin{lem} \label{15.3-4}
	Sei $\lambda \in K = Q(R), f \in K[x]$ und $p \in \Prim(R)$.
	Dann ist $\nu_p(\lambda f) = \nu_p(\lambda) + \nu_p(f)$.
\end{lem}

\begin{kor} \label{15.3-5}
	Sei $f \in K[x]$, $K = Q(R)$, $R$ UFD und sei $p \in \Prim(R)$.
	Dann gibt es ein $f_1 \in R[x]$ mit $\nu_p(f_1) = 0$ und ein $\alpha \in K$ mit $\nu_p(\alpha) = \nu_p(f)$ so, dass $f = \alpha f_1$ ist.
\end{kor}

\begin{lem}[Gauß Lemma] \label{15.3-6}
	Sei $R$ ein UFD, $K = Q(R)$, $f,g \in K[x]$, $p\in \Prim(R)$.
	Dann gilt:
	\[
		\nu_p(fg) = \nu_p(f) + \nu_p(g).
	\]
\end{lem}

\begin{df} \label{15.3-7}
	Ein Polynom $f(x) \in K[x]$ für einen Körper $K$ heißt \emphdef{normiert}, falls der höchste Koeffizient ungleich Null in $f$ gleich $1$ ist.
	Ist $R$ ein UFD, $f \in R[x]$, so heißt $f$ \emphdef{primitiv}, falls $\ggT($Koeffizienten von $f)$ gleich $1$ ist.

	Klar ist: $f \in R[x]$ normiert ($R[x] \subset K[x]$) impliziert $f$ ist primitiv.
	Außerdem: $f \in R[x]$ primitiv genau dann, wenn $\nu_p(f) = 0$ für alle $p \in \Prim(R)$.
\end{df}

\begin{st}[Satz von Gauß] \label{15.3-8}
	Sei $R$ ein UFD, $K = Q(R)$.
	Dann ist $R[x]$ UFD und die Primelemente von $R[x]$ sind genau die Primelemente von $R$ (als konstante Polynome) und die primitiven Polynome in $R[x]$ vom Grad größer gleich 1, die als Polynome von $K[x] \supset R[x]$ irreduzibel sind.
\end{st}

\begin{kor} \label{15.3-9}
	Sei $R$ ein UFD.
	Dann ist $R[x_1, \dotsc, x_n]$ ($x_i$ für $1 \le i \le n$ Unbestimmte) ein UFD.
	Insbesondere ist der Polynomring $K[x_1, \dotsc, x_n]$ ein UFD für alle $n \in \N$ und jeden Körper $K$.
\end{kor}

\setcounter{thm}{10}
\begin{nt} \label{15.3-11}
	\begin{enumerate}[1)]
		\item
			Sei $K$ ein Körper, $2 \le n \in \N$.
			Dann ist $K[x_1, \dotsc, x_n]$ zwar UFD, aber kein HIR.
		\item
			Für $R = \Z, K = Q(R) = \Q$ folgt aus dem Satz von Gauß, dass man rationale Polynome mit ganzen Koeffizienten in $\Z[x]$ zerlegen kann, um eine Zerlegung in $\Q[x]$ zu erhalten.
			Insbesondere: testet man $f \in \Z[x]$ durch Einsetzen auf Nullstellen, so genügt es, ganze Zahlen in $f$ einzusetzen.
		\item
			Die Zerlegung von Polynomen in irreduzible Faktoren ist schwer.
			Schon die Frage, ob ein gegebenes Polynom irreduzibel ist, ist oft schwierig zu beantworten.
			\emph{Allerdings: $x + \alpha \in K[x], \alpha \in K$ für einen Körper $K$ ist immer irreduzibel!}
		\item
			Der Satz von Gauß sagt insbesondere, dass man die Zerlegung von Polynomen in $K[x]$ in Irreduzibel immer auf die Zerlegung in irreduzible in $R[x]$ zurückführen kann, wenn $K = Q(R)$ und $R$ UFD.
			Ist $f \in K[x]$, so lässt sich nämlich $f$ als $\alpha f_1$ mit $f_1 \in R[x], \alpha \in K$ schreiben, etwa indem man mit dem Hauptnenner der Koeffizienten von $f$ durchmultipliziert und beim resultierenden Polynom in $R[x]$ noch den $\ggT$ der Koeffizienten (aus $R$) ausklammert.
			Die irreduzibelen Polynome in $K[x]$ vom Grad größer gleich 1 sind also (bis auf skalare Vielfache ungleich Null) genau die irreduzibelen Polynome von $R[x]$.
	\end{enumerate}
\end{nt}

Sei $R$ ein UFD und $f \in R[x] \subset K[x]$ mit $K = Q(R)$.
Wie können wir herausfinden, ob $f$ irreduzibel ist oder nicht?

\begin{st}[Reduktionskriterium] \label{15.3-12}
	Seien $R$ und $S$ Integritätsbereiche und sei $\sigma: R \to S$ ein Ringhomomorphismus mit $\sigma(1_R) = 1_S$.
	Dann wird durch $\sigma: R[x] \to S[x] : \sum_{i=0}^n \alpha_i x^i \mapsto \sum_{i=0}^n \sigma(\alpha_i) x^i \in S[x]$ eine Fortsetzung von $\sigma$ auf $R[x]$ definiert, die ebenfalls ein Ringhomomorphismus ist.
	Sei $0 \neq f \in R[x]$ so, dass $\deg(\sigma(f)) = \deg(f)$ gilt.
	Sei $K = Q(R), L = Q(S)$.
	Ist $\sigma(f) \in L[x]$ irreduzibel, dann gibt es keine Faktorisierung $f = gh$ mit $g, h \in R[x], 1 \le \deg g, \deg h$.
	Ist $R$ zusätzlich UFD, so ist $f$ irreduzibel in $K[x]$ (und in $R[x]$).
\end{st}

\begin{ex} \label{15.3-13}
	Sei $R$ UFD und $m$ maximales ideal von $R$.
	Dann ist $R / m = L (= Q(L))$ ein Körper.
	Sei $\sigma: R \mapsto L: \alpha \mapsto \alpha + m = \_\alpha \in L$ die natürliche projektion.
	Sei $f = \sum_{i=0}^n \alpha_i x^i \in R[x] \subset K[x]$, $K = Q(R)$ mit $\alpha_i \in R$.
	Angenommen $\alpha_n \not\in m$, dann ist $\_{\alpha_n} = 0$ und daher $\deg(\sigma(f)) = n = \deg(f)$.
	Ist daher $\sigma(f) \in L[x]$ irreduzibel, dann auch $f$ in $R[x]$ und $K[x]$.
	\begin{ex*}
		Sei $f(x) = x^3 - 99x^2 - 28x - 1 \in \Z[x]$.
		Modulo 3 wird daraus $\_f(x) = x^3 - x - 1 \in \Z / 3\Z[x]$.
		Man rechnet leicht nach: $\_f(\_0), \_f(\_1), \_f(\_{-1}) \neq 0$ (wegen $\Z / 3\Z = \{\_0, \_1, \_{-1}\}$), d.h. $\_f$ hat keine Nullstelle in $\Z / 3\Z$ und ist daher irreduzibel.
		Also ist $f$ irreduzibel in $\Q[x]$.
	\end{ex*}
\end{ex}

\begin{st}[Eisenstein Kriterium] \label{15.3-14}
	Sei $R$ UFD, $K = Q(R)$ und sei $f(x) = \alpha_0 + \alpha_1 x + \dotsb + \alpha_n x^n \in R[x]$, $\alpha_n \neq 0$.
	Sei $p \in \Prim(R)$ so, dass gilt
	\begin{enumerate}[i)]
		\item
			$p$ teilt nicht $\alpha_n$ (z.B. $\alpha_n = 1$),
		\item
			$p$ teilt $\alpha_i$ für $0 \le i \le n - 1$,
		\item
			$p^2$ teilt nicht $\alpha_0$.
	\end{enumerate}
	Dann ist $f$ irreduzibel in $K[x]$.
\end{st}

\begin{ex} \label{15.3-15}
	\begin{itemize}
		\item
			$3x^5 - 15 \in \Z[x]$ ist irreduzibel mit $p = 5$,
		\item
			$2x^{10} - 21 \in \Z[x]$ ist irreduzibel mit $p = 3$ oder $p = 7$,
		\item
			$x^n - a \in \Z[x]$ ist irreduzibel, wenn $a \neq 0, \pm 1, p\divs a$ und $p^2 \ndivs a$ für eine Primzahl $p$.
	\end{itemize}
	Eisenstein ist also großartig, um irreduzible Polynome zu \emph{konstruieren}.
\end{ex}

