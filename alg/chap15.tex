\chapter{Ringe und Moduln}


\setcounter{section}{2}
\section{Der Satz von Gauß}

Wir haben im letzten Abschnitt \ref{15.2} gesehen, dass Hauptidealringe faktoriell sind.
Der Satz von Gauß sagt unter anderem, dass es (viele) UFD's gibt, die keine HIR's sind.
Die Konstruktion hat darüber hineaus einige wichtige (und hübsche) Anwendungen für die späteren Kapitel zur Körpertheorie.

Im Folgenden sei $R$ ein kommutativer Integritätsbereich mit Einselement.
Mit $\Prim(R)$ bezeichnen wir ein System von Repräsentanten der Assoziierungsklassen von Primelementen in $R$.
So ist jedes Primelement von $R$ assoziiert zu genau einem Primelement in $\Prim(R)$, d.h. unterscheidet sich von diesem nur durch eine Einheit in $R$.

\begin{df} \label{15.3-1}
	Sei $R$ UFD, $K = Q(R)$ der Quotientenkörper von $R$.
	Sei $p \in \Prim(R), 0 \neq a \in R$.
	Da $R$ faktoriell ist, finden wir eindeutiges $b \in R, \nu_p(a) \in \N_0$ mit $\ggT(b,p) = 1$ und $a = p^{\nu_p(a)} b$, d.h. $\nu_p(a)$ ist die maximale $p$-Potenz in der Primfaktorzerlegung von $a$.

	Für $0 \neq \alpha = \f rs \in K$ ($r, s \in R, s \neq 0$) setzen wir
	\[
		\nu_p(\alpha) := \nu_p(r) - \nu_p(s) \in \Z.
	\]
	Beachte, dass $\nu_p(\alpha)$ nur von $\alpha$, nicht von der Darstellung von $\alpha$ als Bruch $\alpha = \f st$ abhängt.
	Wir definieren $\nu_p(0) = \infty$.

	Die Abbildung $\nu_p: K \to \Z \cup \{\infty\}$ heißt \emphdef[p-adische Bewertung]{(exponentielle) $p$-adische Bewertung von $K$} und $\nu_p(\alpha)$ der \emphdef{Wert} von $\alpha$ an der \emphdef{Stelle} $p\in \Prim(R)$ (genauer \emphdef{Primstelle} $p$).

	\begin{note}
		\begin{itemize}
			\item
				Sind $\alpha, \beta \in K = Q(R)$, dann ist wegen der eindeutigen Primfaktorzerlegung
				\[
					\nu_p(\alpha \beta) = \nu_p(\alpha) + \nu_p(\beta)
				\]
				für alle $p \in \Prim(R)$.
			\item
				Ist $\alpha \in K$, so gilt $\nu_p(\alpha) = 0$ für fast alle $p \in \Prim(R)$.
			\item
				Ist $a \in R$ mit $\nu_p(a) = 0$ für alle $p \in \Prim(R)$, so ist $a \in U(R)$.
		\end{itemize}
	\end{note}
\end{df}

\begin{df} \label{15.3-2}
	Sei wie oben $K = Q(R)$ und sei $f(x) = \alpha_0 + \alpha_1 x + \dotsb + \alpha_n x^n \in K[x]$	ein Element aus dem Polynomring.
	Wir setzen $\nu_p(f) := \infty$ für $f = 0$ und
	\[
		\nu_p(f) := \min_{\substack{0\le i \le n \\ \alpha_i \neq 0}} \nu_p(\alpha_i),
	\]
	wobei $p \in \Prim(R)$.
	$\nu_p(f)$ heißt \emphdef[p-Wert]{(exponentieller) $p$-Wert von $f$}.
\end{df}

\begin{nt} \label{15.3-3}
	Betrachte $f \in K[x]$ wie in \ref{15.3-2}.
	Sei $\alpha_i = \f {r_i}{s_i}$ mit $r_i, s_i \in R$.
	Für $\alpha_i \neq 0$ können wir $\ggT(r_i, s_i) = 1$ annehmen.
	Ist $p$ Teiler von $r_i$ für alle $i$ mit $\alpha_i \neq 0$, so teilt $p$ kein $s_i$ mit $\alpha_i \neq 0$ und
	\[
		\nu_p(f) = \nu_p(\ggT \Set{r_i & \alpha_i \neq 0}) > 0.
	\]
	Sonst gibt es ein $i$ mit $\alpha_i \neq 0$ und $\ggT(p, r_i) = 1$.
	Dann ist insbesondere $p$ Teiler von $s_i$ nur, wenn es kein Teiler von $r_i$ ist ($\alpha_i \neq 0$) und daher ist
	\[
		\nu_p(f) = -\nu_p(\kgV \Set{s_i & \alpha_i \neq 0}) < 0,
	\]
	d.h. $-1$ mal der Exponent der höchsten Potenz von $p$, die in der Primfaktorzerlegung von den $s_i$ mit $\alpha_i \neq 0$ vorkommt.

	Insbesondere ist $\nu_p(f) \ge 0$ für alle $f \in R[x] \subset K[x]$.
\end{nt}

\begin{lem} \label{15.3-4}
	Sei $\lambda \in K = Q(R), f \in K[x]$ und $p \in \Prim(R)$.
	Dann ist $\nu_p(\lambda f) = \nu_p(\lambda) + \nu_p(f)$.
	\begin{proof}
		Sei $\lambda = \f rs$ für $r, s \in R, s \neq 0$ und $f = \sum_{i=0}^n \alpha_i x^i$ mit $\alpha_i = \f{r_i}{s_i}$ für $r_i, s_i \in R, s_i \neq 0$.
		Dann ist $\lambda f = \sum_{i=0}^n \f{rr_i}{ss_i} x^i$ und daher
		\begin{align*}
			\nu_p(\lambda f)
			&= \min_{\substack{0\le i \le n \\ \alpha_i \neq 0}} \big( \nu_p(r) + \nu_p(r_i) - \nu_p(s) - \nu_p(s_i) \big) \\
			&= \nu_p(r) - \nu_p(s) + \min_{\substack{0\le i \le n \\ \alpha_i \neq 0}} \big( \nu_p(r_i) - \nu_p(s_i) \big) \\
			&= \nu_p(\lambda) + \nu_p(f).
		\end{align*}
	\end{proof}
\end{lem}

\begin{kor} \label{15.3-5}
	Sei $R$ ein UFD, $K = Q(R)$, $p \in \Prim(R)$ und $f \in K[x]$.

	Dann existieren $f_1 \in R[x], \alpha \in K$ mit $f = \alpha f_1$ und $\nu_p(f_1) = 0$.
	Insbesondere ist dann $\nu_p(\alpha) = \nu_p(f)$.
	\begin{proof}
		Sei $f = \sum_{i=0}^n \alpha_i x^i$ mit $\alpha_i = \f{r_i}{s_i} \in K$.
		Erweitern mit $s = s_0 \dotsc s_n$ ergibt
		\[
			f = \f 1s \sum_{i=0}^n \underbrace{\f{sr_i}{s_i}}_{=:t_i} x^i = \f 1s g
		\]
		mit $g = \sum_{i=0}^n t_i x^i$.
		Setze $k := \nu_p(g) = \nu_p(\ggT\{t_0,\dotsc, t_n\})$.
		Dann ist $f_1 := \f 1{p^k} g \in R[x]$ mit $\nu_p(f_1) = \nu_p(\f 1{p^k}) \nu_p(g) = -k + k = 0$.
		Die Behauptung folgt mit $\alpha := \f {p^k}s$, denn dann ist $f = \f 1s g = \f {p^k}s f_1 = \alpha f_1$.
		Außerdem folgt $\nu_p(\alpha) = \nu_p(f)$ aus \ref{15.3-4}.
	\end{proof}
\end{kor}

\begin{lem}[Gauß Lemma] \label{15.3-6}
	Sei $R$ ein UFD, $K = Q(R)$, $f,g \in K[x]$, $p\in \Prim(R)$.
	Dann gilt:
	\[
		\nu_p(fg) = \nu_p(f) + \nu_p(g).
	\]
	\begin{proof}
		Gemäß \ref{15.3-5} seien $\alpha, \beta \in K, f_1, g_1 \in R[x]$ so gewählt, dass $f = \alpha f_1, g = \beta g_1$ mit $\nu_p(f) = \nu_p(\alpha), \nu_p(g) = \nu_p(\beta), \nu_p(f_1) = \nu_p(g_1) = 0$.
		Dann ist
		\begin{align*}
			\nu_p(fg)
			= \nu_p(\alpha \beta f_1 g_1)
			&= \nu_p(\alpha \beta) + \nu_p(f_1 g_1) \\
			&= \nu_p(\alpha) + \nu_p(\beta) + \nu_p(f_1g_1) \\
			&= \nu_p(f) + \nu_p(g) + \nu_p(f_1g_1)
		\end{align*}
		und es genügt $\nu_p(f_1g_1) = 0$ zu zeigen.

		Seien $f_1 = \sum_{i=0}^n \alpha_i x^i$, $g_1 = \sum_{j=0}^m \alpha_j x^j$ mit $\alpha_i, \beta_j \in R$, $\alpha_n, \beta_n \neq 0$.
		Seien $r, s \in \N$ maximal mit $\alpha_r, \beta_s \neq 0$ so, dass $\ggT(\alpha_r, p) = 1 = \ggT(\beta_s, p)$, d.h. $p$ ist kein Teiler von $\alpha_r$ und kein Teiler von $\beta_s$ ($r, s$ existieren wegen $\nu_p(f_1) = \nu_p(g_1) = 0$).
		Damit ist $p$ Teiler von $\alpha_{r+1}, \dotsc, \alpha_n, \beta_{s+1}, \dotsc, \beta_m$.
		Der Koeffizient von $x^{r+s}$ in $f_1g_1 \in R[x]$ ist dann von der Form
		\[
			\eps_{r + s}
			= \alpha_r\beta_s + \sum_{\substack{i < r, j > s \\ i + j = r + s}} \alpha_i \beta_j + \sum_{\substack{i > r, j < s \\ i + j = r + s}} \alpha_i \beta_j.
		\]
		Wegen $p \divs \beta_j$ für $j > s$ und $p \divs \alpha_i$ für $i > r$, teilt $p$ beide Summen, aber $p \ndivs \alpha_r \beta_s$, da weder $p \divs \alpha_r$, noch $p \divs \beta_s$ ($p$ prim).
		Wir schließen, dass $p$ kein Teiler von $\eps_{r+s}$ sein kann und daher $\nu_p(f_1g_1) = 0$.
	\end{proof}
\end{lem}

\begin{df} \label{15.3-7}
	Sei $K$ ein Körper.
	Ein Polynom $f = \sum_{i=0}^n \alpha_i x^i \in K[x]$ heißt \emphdef{normiert}, falls $\alpha_n = 1$ gilt.

	Sei $R$ ein UFD.
	Ein Polynom $f = \sum_{i=0}^n \alpha_i x^i \in R[x]$ heißt \emphdef{primitiv}, falls $\ggT(\Set{\alpha_i & i \in I}) = 1$ gilt.
	\begin{note}
		\begin{itemize}
			\item
				$f \in K[x]$ normiert impliziert $f \in R[x] \subset K[x]$ ist primitiv.
			\item
				$f \in R[x]$ ist primitiv genau dann, wenn $\nu_p(f) = 0$ für alle $p \in \Prim(R)$.
		\end{itemize}
	\end{note}
\end{df}

\begin{st}[Satz von Gauß] \label{15.3-8}
	Sei $R$ ein UFD, $K = Q(R)$.
	Dann ist $R[x]$ UFD und die Primelemente von $R[x]$ sind genau die Primelemente von $R$ (als konstante Polynome) und die primitiven Polynome in $R[x]$ vom Grad größer gleich 1, die als Polynome von $K[x] \supset R[x]$ irreduzibel sind.
	\begin{proof}
		Sei $f = \sum_{i=0}^n r_i x^i \in R[x], f \neq 0, r_i \in R$.
		Sei $r = \ggT(\Set{r_i : 0 \le i \le n})$, dann ist $f = r f_1$ mit $f_1 \in \R[x]$ primitiv.
		In $K[x]$ finden wir eine eindeutige Zerlegung $f_1 = \tilde p_1 \dotsb \tilde p_k$ mit $\tilde p_1, \dotsc, \tilde p_k \in K[x]$ irreduzibel, da $K[x]$ Hauptidealring und daher UFD ist (siehe \ref{15.2-25}).
		Sei $s_i \in R$ der Hauptnenner und $t_i$ der $\ggT$ der Koeffizienten von $\tilde p_i$, so lässt sich $\tilde p_i$ schreiben als $\tilde p_i = \f{t_i}{s_i} p_i$ mit $p_i \in R[x]$ primitiv, also $f = r \f ts p_1 \dotsb p_k$ mit primitiven $p_i \in R[x]$, $t = t_1 \dotsb t_k, s = s_1 \dotsb s_k \in R$.

		Sei $\alpha \in \Prim(R)$.
		Dann ist $\nu_\alpha(p_i) = 0$ für $i \in \Set{1, \dotsc, k}$ und $\nu_\alpha(f_1) = 0$.
		Nach dem Lemma von Gauß \ref{15.3-6} haben wir
		\begin{align*}
			0 = \nu_\alpha(f_1)
			= \nu_\alpha(\f ts p_1 \dotsb p_k)
			&= \nu_\alpha(t) - \nu_\alpha(s) + \sum_{i=1}^k \nu_\alpha(p_i) \\
			&= \nu_\alpha(t) - \nu_\alpha(s),
		\end{align*}
		d.h. $\nu_\alpha(t) = \nu_\alpha(s)$, also $\nu_\alpha(\f ts) = 0$ und somit ist, da $\alpha \in \Prim(R)$ beliebig gewählt war $\f ts \in U(R)$ eine Einheit.
		Sei \oBdA $\f ts = 1$ und daher $f_1 = p_1 \dotsb p_k$.

		Wir zeigen jetzt, dass $p_1, \dotsc, p_k$ auch irreduzibel in $R[x]$ sind.
		Sei $i \in \Set{1, \dotsc, k}$ und $p_i = gh$ mit $g, h \in R[x]$ eine Zerlegung von $p_i$.
		Dann ist dies auch eine Zerlegung in $K[x] \supset R[x]$ und daher \oBdA $g = \beta \in R$ eine Konstant.
		Wegen $0 = \nu_\alpha(p_i) = \nu_\alpha(gh) = \nu_\alpha(g) + \nu_\alpha(h)$ für alle $\alpha \in \Prim(R)$ nach \ref{15.3-6} schließen wir $\nu_\alpha(g) = 0 = \nu_\alpha(h)$, da $\nu_\alpha(g), \nu_\alpha(h) \ge 0$ wegen $g, h \in R[x]$.
		Damit haben wir $g \in U(R)$ gezeigt und $p_i$ ist irreduzibel in $R[x]$ für alle $i \in \Set{1,\dotsc, k}$.

		Wir zerlegen noch $r$ in Primfaktoren in $R$ und haben damit eine Zerlegung von $f = rp_1\dotsb p_k$ in irreduzible Elemente in $R[x]$ gefunden.
		Ist $f = s q_1 \dotsb q_l$ eine andere solche Zerlegung mit $s \in R$, so ist wegen der Eindeutigkeit der Primfaktorzerlegung in $K[x]$ $k = l$ und $q_i = \eps_i p_i$ für $0 \neq \eps_i \in  K, i \in \Set{1,\dotsc, k}$.
		So haben wir mit $\eps := \eps_1 \dotsb \eps_k$
		\[
			r p_1 \dotsb p_k
			= f
			= s q_1 \dotsb q_k
			= \eps s p_1 \dotsb p_k
		\]
		und daher $r = \eps s$.
		Wegen
		\[
			\nu_\alpha(f)
			= \nu_\alpha(r)
			= \nu_\alpha(\eps s)
			= \nu_\alpha(\eps) + \nu_\alpha(s)
			= \nu_\alpha(\eps) + \nu_\alpha(f)
		\]
		haben wir $\nu_\alpha(\eps) = 0$ für alle $\alpha \in \Prim(R)$ und daher $\eps \in U(R)$.
		Also ist die Zerlegung eindeutig.
		Der Rest folgt nun sofort.
	\end{proof}
\end{st}

\begin{kor} \label{15.3-9}
	Sei $R$ ein UFD, $n \in \N$.
	Dann ist $R[x_1, \dotsc, x_n]$ ein UFD.

	Insbesondere ist der Polynomring $K[x_1, \dotsc, x_n]$ ein UFD für jeden Körper $K$.
	\begin{proof}
		Funktioniert mit Satz von Gauß \ref{15.3-8} und Induktion über $n$, da $R[x_1, \dotsc, x_n] \isomorphic (R[x_1, \dotsc, x_{n-1}])[x_n]$.
	\end{proof}
\end{kor}

\setcounter{thm}{10}
\begin{nt} \label{15.3-11}
	\begin{enumerate}[1)]
		\item
			Sei $K$ ein Körper, $2 \le n \in \N$.
			Dann ist $K[x_1, \dotsc, x_n]$ zwar UFD, aber kein HIR.
		\item
			Für $R = \Z, K = Q(R) = \Q$ folgt aus dem Satz von Gauß, dass man rationale Polynome mit ganzen Koeffizienten in $\Z[x]$ zerlegen kann, um eine Zerlegung in $\Q[x]$ zu erhalten.
			Insbesondere: testet man $f \in \Z[x]$ durch Einsetzen auf Nullstellen, so genügt es, ganze Zahlen in $f$ einzusetzen.
		\item
			Die Zerlegung von Polynomen in irreduzible Faktoren ist schwer.
			Schon die Frage, ob ein gegebenes Polynom irreduzibel ist, ist oft schwierig zu beantworten.
			\emph{Allerdings: $x + \alpha \in K[x], \alpha \in K$ für einen Körper $K$ ist immer irreduzibel!}
		\item
			Der Satz von Gauß sagt insbesondere, dass man die Zerlegung von Polynomen in $K[x]$ in irreduzible Faktoren immer auf die Zerlegung in irreduzible in $R[x]$ zurückführen kann, wenn $K = Q(R)$ und $R$ UFD.
			Ist $f \in K[x]$, so lässt sich nämlich $f$ als $\alpha f_1$ mit $f_1 \in R[x], \alpha \in K$ schreiben, etwa indem man mit dem Hauptnenner der Koeffizienten von $f$ durchmultipliziert und beim resultierenden Polynom in $R[x]$ noch den $\ggT$ der Koeffizienten (aus $R$) ausklammert.
			Die irreduzibelen Polynome in $K[x]$ vom Grad größer gleich 1 sind also (bis auf skalare Vielfache ungleich Null) genau die irreduzibelen Polynome von $R[x]$.
	\end{enumerate}
\end{nt}

Sei $R$ ein UFD und $f \in R[x] \subset K[x]$ mit $K = Q(R)$.
Wie können wir herausfinden, ob $f$ irreduzibel ist oder nicht?

\begin{st}[Reduktionskriterium] \label{15.3-12}
	Seien $R$ und $S$ Integritätsbereiche und sei $\sigma: R \to S$ ein Ringhomomorphismus mit $\sigma(1_R) = 1_S$.
	Dann wird durch $\sigma: R[x] \to S[x] : \sum_{i=0}^n \alpha_i x^i \mapsto \sum_{i=0}^n \sigma(\alpha_i) x^i \in S[x]$ eine Fortsetzung von $\sigma$ auf $R[x]$ definiert, die ebenfalls ein Ringhomomorphismus ist.

	Sei $0 \neq f \in R[x]$ so, dass $\deg(\sigma(f)) = \deg(f)$ gilt.
	Sei $K = Q(R), L = Q(S)$.
	Ist $\sigma(f) \in L[x]$ irreduzibel, dann gibt es keine Faktorisierung $f = gh$ mit $g, h \in R[x], 1 \le \deg g, \deg h$.
	Ist $R$ zusätzlich UFD, so ist $f$ irreduzibel in $K[x]$ (und in $R[x]$).
	\begin{proof}
		Ist $f = gh$ in $R[x]$, so ist $\deg f = \deg g + \deg h$, da $R$ nullteilerfrei ist.
		Dann ist $\sigma(f) = \sigma(gh) = \sigma(g)\sigma(h)$ in $S[x]$ mit $\deg(\sigma(g)) \le \deg(g), \deg(\sigma(h)) \le \deg(h)$.
		Dann ist
		\begin{align*}
			\deg(g) + \deg(h)
			&= \deg(f)
			= \deg(\sigma(f))
			= \deg(\sigma(gh))
			= \deg(\sigma(g)\sigma(h)) \\
			&= \deg(\sigma(g)) + \deg(\sigma(h)) \\
			&\le \deg(g) + \deg(h),
		\end{align*}
		also gilt Gleichheit: $\deg(g) = \deg(\sigma(g)), \deg(h) = \deg(\sigma(h))$.

		Nun ist $\sigma(f)$ irreduzibel in $L[x]$ und $\sigma(f) = \sigma(g)\sigma(h)$ ist Zerlegung von $\sigma(f)$ in $L[x]$, also ist $\deg(\sigma(g)) = \deg(g) = 0$ oder $\deg(\sigma(h)) = \deg(h) = 0$.

		Ist $R$ UFD, so liefert der Satz von Gauß \ref{15.3-8}, dass $f$ irreduzibel in $K[x]$ ist.
	\end{proof}
\end{st}

\begin{ex} \label{15.3-13}
	Sei $R$ UFD und $m$ maximales Ideal von $R$.
	Dann ist $R / m = L (= Q(L))$ ein Körper.
	Sei $\sigma: R \mapsto L: \alpha \mapsto \alpha + m = \_\alpha \in L$ die natürliche Projektion.
	Sei $f = \sum_{i=0}^n \alpha_i x^i \in R[x] \subset K[x]$, $K = Q(R)$ mit $\alpha_i \in R$.
	Angenommen $\alpha_n \not\in m$, dann ist $\_{\alpha_n} = 0$ und daher $\deg(\sigma(f)) = n = \deg(f)$.
	Ist daher $\sigma(f) \in L[x]$ irreduzibel, dann auch $f$ in $R[x]$ und $K[x]$.
	\begin{ex*}
		Sei $f(x) = x^3 - 99x^2 - 28x - 1 \in \Z[x]$.
		Modulo 3 wird daraus $\_f(x) = x^3 - x - 1 \in \Z / 3\Z[x]$.
		Man rechnet leicht nach: $\_f(\_0), \_f(\_1), \_f(\_{-1}) \neq 0$ (wegen $\Z / 3\Z = \{\_0, \_1, \_{-1}\}$), d.h. $\_f$ hat keine Nullstelle in $\Z / 3\Z$ und ist daher irreduzibel.
		Also ist $f$ irreduzibel in $\Q[x]$.
	\end{ex*}
\end{ex}

\begin{st}[Eisenstein Kriterium] \label{15.3-14}
	Sei $R$ UFD, $K = Q(R)$ und sei $f(x) = \alpha_0 + \alpha_1 x + \dotsb + \alpha_n x^n \in R[x]$, $\alpha_n \neq 0$.
	Sei $p \in \Prim(R)$ so, dass gilt
	\begin{enumerate}[i)]
		\item
			$p$ teilt nicht $\alpha_n$ (z.B. $\alpha_n = 1$),
		\item
			$p$ teilt $\alpha_i$ für $0 \le i \le n - 1$,
		\item
			$p^2$ teilt nicht $\alpha_0$.
	\end{enumerate}
	Dann ist $f$ irreduzibel in $K[x]$.
	\begin{proof}
		Sei \oBdA $f$ primitiv, d.h. $\ggT(\Set{\alpha_i: 0 \le i \le n}) = 1$.
		Sei $f = gh$ eine Faktorisierung von $f$ in $K[x]$ mit $\deg g, \deg h \ge 1$.
		\OBdA ist $g, h \in R[x]$ nach dem Satz von Gauß \ref{15.3-8}.
		Sei
		\begin{align*}
			g(x) &= \sum_{i=0}^m \beta_i x^i, &
			h(x) &= \sum_{j=0}^{n-1} \gamma_j x^j,
		\end{align*}
		mit $\beta_m \neq 0 \neq \gamma_{n-m}$, $m, n-m \ge 1$.
		Da $\alpha_0 = \beta_0 \gamma_0$ durch $p$, aber nicht durch $p^2$ teilbar ist, ist \oBdA $p$ kein Teiler von $\beta_0$, aber $\gamma_0$ ist Vielfaches von $p$.

		Da $\alpha_n = \beta_n \gamma_{n-m}$ nicht durch $p$ teilbar ist, ist weder $\beta_m$, noch $\gamma_{n-m}$ durch $p$ teilbar.

		Sei $r \in \N$ minimal so, dass $p$ kein Teiler von $\gamma_r$ ist (existiert, da $p \ndivs \gamma_{n-m}$).
		Dann teilt $p$ alle $\gamma_{r-1}, \dotsc, \gamma_0$.
		Dann ist $r > 0$ und
		\[
			\alpha_r = \beta_0\gamma_r + \sum_{i = 1}^r \beta_i \gamma_{r-i}
		\]
		ist nicht durch $p$ teilbar, da die Summe durch $p$ teilbar ist, aber nicht $\beta_0\gamma_r$.
		Wegen $r \le n - m \le n - 1$ ist dies ein Widerspruch zur Voraussetzung ii), also ist $f$ irreduzibel.
	\end{proof}
\end{st}

\begin{ex} \label{15.3-15}
	\begin{itemize}
		\item
			$3x^5 - 15 \in \Z[x]$ ist irreduzibel mit $p = 5$,
		\item
			$2x^{10} - 21 \in \Z[x]$ ist irreduzibel mit $p = 3$ oder $p = 7$,
		\item
			$x^n - a \in \Z[x]$ ist irreduzibel, wenn $a \neq 0, \pm 1, p\divs a$ und $p^2 \ndivs a$ für eine Primzahl $p$.
	\end{itemize}
	Eisenstein ist also großartig, um irreduzible Polynome zu \emph{konstruieren}.
\end{ex}

