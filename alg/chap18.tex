\chapter{Körpererweiterungen}


\section{Polynomiale Gleichungen}

Wir betrachten polynomiale Gleichungen über einem Körper $K$, d.h. Gleichungen der Form $f(x) = 0$ mit $f \in K[x]$.
So ist etwa $2 \in \Z$ eine Lösung der Gleichung $x^2 - 4 = 0$.

Wir wissen: ist $\alpha \in K$ eine Lösung der polynomialen Gleichung $f(x) = 0$ mit $f \in K[x]$, so lässt sich $f$ schreiben als $f(x) = g(x)(x - \alpha)$ für ein $g \in K[x]$, wobei $\deg g = \deg f - 1$.

Es gibt aber durchaus polynomiale Gleichungen $f(x) = 0$, $f \in K[x]$, die keine Lösungen in $K$ besitzen für geeignete Körper $K$.
Beispielsweise besitzt $x^2 + 1 = 0$ in $\R$ keine Lösung, aber in $\C \supset \R$ schon, denn $\pm i$ sind Lösungen der Gleichung.
In der Tat wurde $\C$ gerade so gebastelt, dass diese Gleichung nun doch eine Lösung hat.
Wir sehen: Körpererweiterungen haben eine Menge mit dem Lösen polynomialer Gleichungen zu tun.
Wir werden sogar sehen, dass in der Tat jede polynomiale Gleichung Lösungen besitzt, sofern man sich auf Körpererweiterungen einlässt.
Darum geht es in der „Galoistheorie“ (nach Évariste Galois, 1811-1832), die wir nun untersuchen werden.

\begin{df} \label{18.1-1}
	Seien $K, E$ Körper mit $K \subset E$.
	Dann heißt $E$ \emphdef{Körpererweiterung} von $K$ und $K$ \emphdef{Unterkörper} von $E$.
	Klar ist, dass dann $E$ eine $K$-Algebra ist, also insbesondere ein $K$-Vektorraum.
	Die $K$-Dimension von $E$ wird \emphdef{Körpergrad} von $E$ über $K$ genannt und mit $[E : K]$ bezeichnet.
\end{df}

\begin{st} \label{18.1-2}
	Sei $K \subset E \subset L$ ein Turm von Körpererweiterungen.
	Dann ist $[L : K] = [L : E] [E : K]$.
	In der Tat gilt: ist $\scr L = \{l_j : j \in J\}$ eine $E$-Basis von $L$ und $\scr E = \{ e_i : i \in I \}$ eine $K$-Basis von $E$, so ist $\scr L \scr E = \{ l_j e_i : j \in J, i \in I \}$ eine $K$-Basis von $L$.
\end{st}

\begin{st} \label{18.1-3}
	Sei $K$ ein Körper.
	Dann gelten folgende Aussagen.
	\begin{enumerate}[i)]
		\item
			Der Durchschnitt von Unterkörpern von $K$ ist wieder ein Unterkörper von $K$.
		\item
			Sei $\emptyset \neq S \subset K$ eine Teilmenge von $K$.
			Dann ist
			\[
				\<S\>_{\text{Körper}} = \bigcap_{\substack{S \subset L \subset K \\ \text{$L$ Unterkörper}}} L
			\]
			ein Unterkörper von $K$.
			Er ist der kleinste Unterkörper von $K$, der $S$ enthält und wird der \emphdef[erzeugter Unterkörper]{von $S$ erzeugte Unterkörper von $K$} genannt.
		\item
			Sind $L_i, i \in I$ Unterkörper von $K$, so ist
			\[
				\Big\< \bigcup_{i \in I} L_i \Big\>
				= \prod_{i \in I} L_i
				:= \Set[\bigg]{ \prod_{i \in I} l_i & \text{$l_i \in L_i$ fast alle 1} }.
			\]
	\end{enumerate}
\end{st}

\begin{st} \label{18.1-4}
	Sei $K$ ein Körper, $f \in K[x]$ irreduzibel.
	Sei $E = K[x] / f K[x]$.
	Für $g \in K[x]$ sei $\_g = g + f K[x]$ die Nebenklasse von $g$ in $E$.
	Dann ist $E$ ein Erweiterungskörper von $K$ mit $K$-Basis $\{\_1, \_x, \_x^2, \dotsc, \_x^{k-1}\}$, wobei $k = \deg f$ und $\_x$ Lösung der polynomiellen Gleichung $f(x) = 0$ in $E$ ist.
\end{st}

\begin{kor} \label{18.1-5}
	Alle polynomialen Gleichungen über Körpern besitzen Lösungen (in entsprechenden Erweiterungskörpern).
\end{kor}

\begin{ex} \label{18.1-6}
	Hier folgt eine rein \emph{algebraische} Konstruktion von $\sqrt 2 \in \R$, ganz ohne analysis und Approximation.
	$x^2 - 2 \in \Q[x]$ ist irreduzibel, also ist $\sqrt 2 \not\in \Q$.
	Im Erweiterungskörper $E = \Q[x] / I$ mit $I = (x^2 - 2) \Q[x]$ ist $\sqrt 2 = x + I$ eine Quadratwurzel von $2$.

	Vorsicht ist jedoch geboten:
	Nicht alle reellen Zahlen lassen sich so als Restklassen von $x$ in einem Faktorring von $\Q[x]$ konstruieren: $\pi$ ist ein Beispiel.
	In der Tat ist die Menge der so „algebraisch konstruierbaren“ reellen Zahlen immernoch abzählbar, die der reellen Zahlen jedoch überabzählbar.
\end{ex}

\begin{ex*}
	Sei $E = \Q[x] / (x^2 - 2)\Q[x]$ wie in \ref{18.1-6}, d.h. nachdem man $x$ mit $\sqrt 2$ identifiziert $E = \Set{ r + s \sqrt 2 & r,s \in \Q }$.

	Wie sieht das Element $\f 1{\sqrt 2}$ in $E$ aus?
\end{ex*}


\section{Primkörper und Charakteristik} \label{sec:18.2}


Sei $K$ ein Körper.
In \ref{18.1-3} haben wir gesehen, dass der Durchschnitt aller Unterkörper von $K$ ein Unterkörper $\Pi$ von $K$ ist, der eindeutig bestimmte kleinste Unterkörper.
Er wird \emphdef{Primkörper} von $K$ genannt.
Wir werden diesen nun näher bestimmen.

Hierzu betrachten wir folgende Abbildung:
\begin{align*}
	\mu: \Z &\to K, &
	z &\mapsto z \cdot 1_K = \begin{cases}
		0 & z = 0 \\
		\underbrace{1_K + \dotsb + 1_K}_{\text{$z$ mal}} & z > 0 \\
		\underbrace{-1_K - \dotsb - 1_K}_{\text{$-z$ mal}} & z < 0
	\end{cases}.
\end{align*}
$\mu: \Z \to K$ ist Ringhomomorphismus $\mu(1_\Z) = 1_K \neq 0$ (leichte Übung).
Wir haben zwei Fälle:
\begin{seg}{1. Fall, $\mu$ ist injektiv}
	Da $\Z$ nullteilerfrei ist mit Quotientenkörper $\Q$, erweitert sich $\mu$ durch
	\[
		\mu: \Q \to K:
		\f ab \mapsto \f{\mu(a)}{\mu(b)} \in K
	\]
	für $a,b \in \Z, b \neq 0$ zu einem injektiven Körperhomomorphismus von $\Q$ in $K$.
	Dessen Bild ist ein Unterkörper von $K$, isomorph zu $\Q$.
	Sei $E \subset K$ irgendein Unterkörper von $K$.
	Dann ist $1_K \in E$ und daher auch $z \cdot 1_K \in E$ für $z \in \Z$, also auch $\f {a \cdot 1_K}{b \cdot 1_K} \in E$ für alle $a, b \in \Z, b \neq 0$ und somit $\Q \isomorphic \im \mu \subset E$.
	Also ist $\im \mu \isomorphic \Q$ der kleinste Unterköper von $K$ und damit der Primkörper $\Pi$ von $K$.
	Wir definieren in diesem Fall die \emphdef{Charakteristik} $\Char K$ von $K$ durch $\Char K = 0$.
\end{seg}
\begin{seg}{2. Fall, $\mu$ ist nicht injektiv}
	Sei $\ker \mu = p\Z$ mit $0 \neq p \in \Z$.
	Wegen $0 \neq 1_K = \mu(1_\Z)$ ist $p\Z \neq \Z$.
	Nun ist $\Z / p\Z \isomorphic \im \mu$ und $\im \mu \subset K$ ist ein nicht trivialer Unterring des nullteilerfreien Rings $K$ und daher ein Integritätsbereich.
	Nach \ref{15.2-16} ist $p\Z$ Primideal, d.h. $0 \neq p$ ist Primideal und $\Z / p \Z$ ist Körper isomorph zu $\im \mu$.
	Wie im ersten Fall schließen wir, dass dieser der Primkörper von $K$ ist und wir setzen $\Char K = p$.
\end{seg}


\section{Algebraische und Transzendente Körpererweiterungen}


Im Folgenden sei $K$ stets ein Körper.

\begin{df} \label{dup:18.3-1}
	Sei $E$ Erweiterungskörper von $K$.
	\begin{enumerate}[i)]
		\item
			Die Körpererweiterung $K \subset E$ heißt \emphdef[Körpererweiterung!endlich]{endlich}, falls $[E : K]$ endlich ist, sonst \emphdef[Körpererweiterung!unendlich]{unendlich}
		\item
			Ist $S \subset E$ so, dass $E = \<K \cup S\>_{\text{Körper}}$ gilt, so heißt $S$ \emphdef{Erzeugendensystem} von $E$ über $K$ und wir schreiben $E = K(S)$.
			Besteht $S$ aus genau einem Element $S = \{ \alpha \}$, $\alpha \in E$, so schreiben wir auch kürzer $E = K(\alpha) = K(\{\alpha\})$ und $E$ heißt \emphdef[Körpererweiterung!einfache]{einfache Körpererweiterung}.
			Das Element $\alpha \in E$ wird dann \emphdef{primitives Element} von $E$ über $K$ genannt.
		\item
			Ist $E = K(S), S = \{\alpha_1, \dotsc, \alpha_k\}$, so schreiben wir $K(\alpha_1, \dotsc, \alpha_k) = E$ anstatt $E = K(\{\alpha_1, \dotsc, \alpha_k\})$.
	\end{enumerate}
\end{df}

\begin{note}
	Ist $S \subset K$, so ist $K(S) = K$.
	Ist $S \subset E = K(S)$ und $\hat S = \{ \alpha \in S : \alpha \not\in K\}$, so ist $E = K(\hat S) = K(S)$.
	Immer gilt $E = K(S)$ mit $S = E$ (oder $S = E \setminus K$).
	Daher besitzt jede Körpererweiterung ein Erzeugendensystem.
\end{note}

\begin{ex*}
	\begin{enumerate}[1.)]
		\item
			Es ist $\C = \R(i), i^2 = -1$.
			Eine $\R$-Basis von $\C$ ist $\{1, i\}$ und daher ist $\C = \{ \alpha + \beta i : \alpha, \beta \in \R\}$.
			$\C \isomorphic \R[x] / (x^2+1) \R[x]$ und $i$ entspricht gerade der Nebenklasse $x + (x^2+1)\R[x]$.
		\item
			$E = \Q(\sqrt 2) = \{ a + b \sqrt 2 : a,b \in \Q \} \isomorphic \Q[x] / (x^2 - 2) \Q[2]$.
	\end{enumerate}
\end{ex*}

Der Trick von \ref{sec:18.2} geht nochmals, nun mit $K[x]$ (dies ist ein Hauptidealring) anstatt $\Z$ und der universellen Eigenschaft von $K[x]$:

Sei $K \subset E$ Körpererweiterung, $\alpha \in E$.
Dann folgt aus \ref{15.1-17}, dass Einsetzen von $\alpha$ in Polynome einen wohldefinierten $K$-Algebra Homomorphismus
\[
	\eps_\alpha : K[x] \to E : \sum_{i=0}^n \beta_i x^i \mapsto \sum_{i=0}^n \beta_i \alpha^i
\]
definiert.
Wir haben das folgende Lemma:

\setcounter{thm}{0}
\begin{lem} \label{18.3-1}
	Sei $K \subset E$ Körpererweiterung, $\alpha \in E$.
	Dann definiert
	\[
		\eps_\alpha: K[x] \to E: \sum_{i=0}^n \beta_i x^i \mapsto \sum_{i=0}^n \beta_i \alpha^i,
	\]
	mit $\beta_i \in K$, $0 \le i \le n$ einen $K$-Algebra Homomorphismus.
	Ist $\eps_\alpha$ injektiv, so lässt sich $\eps_\alpha$ zu einer Einbettung $\hat \eps_\alpha: Q(K[x]) = K(x) \to E$ ausdehnen und $\im \hat \eps_\alpha$ ist der Unterkörper $K(\alpha)$ von $E$.
	Ist $\eps_\alpha$ nicht injektiv, so ist $\ker \eps_\alpha = pK[x]$ ein Primideal ungleich $(0)$ von $K[x]$ und induziert einen $K$-linearen Körperisomorphismus von $K[x] / pK[x]$ auf $K(\alpha) \subset E$.
	In diesem Fall ist das Polynom $p \in K[x]$ irreduzibel und bis auf Skalare das eindeutig bestimmte Polynom kleinsten Grades mit Nullstelle $\alpha$ ($p(\alpha) = 0$).
\end{lem}

\begin{df} \label{18.3-2}
	Sei $K \subset E$ Körpererweiterung, $\alpha \in E$.
	\begin{enumerate}[i)]
		\item
			Ist $\eps_\alpha : K[x] \to E$ in \ref{18.3-1} injektiv, so heißt $\alpha$ \emphdef{transzendent} über $K$ und $K(\alpha) = \im \hat \eps_\alpha$ ($\isomorphic K(x) = Q(K[x])$) heißt \emphdef[Körpererweiterung!transzendente]{transzendente einfache Körpererweiterung}.
			Wegen $\dim_K K[x] = \infty$ sind $\dim_K K(x)$ und daher auch $\dim_K K(\alpha)$ und $\dim_K E$ unendlich.
			Die Elemente von $K(\alpha)$ ergeben sich dann als
			\[
				K(\alpha) = \Set{ \f{f(\alpha)}{g(\alpha)} & f,g \in K[x], g \neq 0 }
			\]
			(Jedes Element von $K(\alpha)$ lässt sich so eindeutig mit teilerfremden $f$ und $g$ in $K[x]$ schreiben).
		\item
			Ist $\eps_\alpha: K[x] \to E$ nicht injektiv, so ist $\ker \eps_\alpha \neq (0)$ und $K(\alpha) = \im \eps_\alpha \isomorphic K[x] / pK[x]$ ist endlich dimensional mit $K$-Basis $\{1, \alpha, \alpha^2, \dotsc, \alpha^{k-1}\}$, $k = \deg p$.
			Dann heißt $\alpha$ \emphdef{algebraisch} über $K$ und $K(\alpha)$ \emphdef{algebraische Erweiterung} von $K$.
			Ohne Einschränkung können wir $p$ als das eindeutig bestimmte normierte Polynom mit $p(\alpha) = 0$ nehmen.
			Dieses heißt \emphdef{Minimalpolynom} von $\alpha$ (über $K$) und wird mit $\mu_{K,\alpha} \in K[x]$ bezeichnet.
			Dann ist $0 \neq \mu_{K, \alpha}$ irreduzibel in $K[x]$.
			Jedes $\beta \in K(\alpha)$ (also auch z.B. $\alpha^{-1} \in K(\alpha)$ für $\alpha \neq 0$) besitzt dann eine eindeutige Darstellung $\beta = \lambda_0 + \lambda_1 \alpha + \dotsb + \lambda_{k-1} \alpha^{k-1}$ mit $\lambda_i \in K$ für $i = 0, \dotsc, k-1$.
			So ist
			\[
				K(\alpha) = \Set[\big]{ f(\alpha) & f \in K[x], \deg f \le k - 1 }.
			\]
	\end{enumerate}
	\begin{proof}
		Alles ist klar bis auf die $K$-Basis in ii).
		In \ref{18.1-4} hatten wir gesehen, dass $\Set{ \_1, \_x, \dotsc, \_x^{k-1} }$ mit $\_x = x + p K[x]$ eine $K$-Basis von $K[x] / p K[x]$ ist für $\deg p = k$ und $p$ irreduzibel.
		Wegen $\eps_\alpha(x) = \alpha$ und $p K[x] = \ker \eps_\alpha$ folgt, dass $\Set{ 1, \alpha, \dotsc, \alpha_{k-1} }$ eine $K$-Basis von $K[x] / pK[x] \isomorphic \im \eps_\alpha = K(\alpha)$ ist.
	\end{proof}
\end{df}

\begin{kor} \label{18.3-3}
	Eine einfache Körpererweiterung $K \subset E = K(\alpha)$, $\alpha \in E$ ist algebraisch genau dann, wenn $[E : K] < \infty$ ist.
	Insbesondere sind dann alle $\beta \in E$ algebraisch über $K$, falls $\alpha \in E = K(\alpha)$ es ist.
	Darüber hinaus ist $[E : K] = \deg(\mu_{K, \alpha})$.
\end{kor}

\begin{ex} \label{18.3-4}
	Sei $E = K(\alpha)$ einfache algebraische Körpererweiterung.
	Wir haben gesehen, dass sich dann jedes Element als Polynom $f(\alpha) = \lambda_0 + \lambda_1 \alpha + \dotsb + \lambda_{k-1} \alpha^{k-1}$ mit $k = [E : K], \lambda_i \in K$ schreiben lässt und diese Darstellung eindeutig ist.
	Für $0 \neq \beta = f(\alpha) \in E$, $f \in K[x], \deg f \le k-1$ beschreibe man einen Algorithmus, um die polynomiale Darstellung von $\beta^{-1}$ zu konstruieren.
\end{ex}

\begin{kor} \label{18.3-5}
	Sei $K \subset E$ endliche Körpererweiterung (d.h. $[E : K] < \infty$).
	Dann sind alle Elemente von $E$ algebraisch über $K$.
	\begin{proof}
		Sei $\alpha \in E$, also $K \subset K(\alpha) \subset E$.
		Dann ist mit \ref{18.3-3} $[K(\alpha) : K] \le [E : K] < \infty$ und $\alpha$ algebraisch über $K$.
	\end{proof}
\end{kor}

Sind alle Elemente einer Körpererweiterung $K \subset E$ algebraisch über $K$, so heißt $E$ \emphdef{algebraisch über} $K$ und man spricht von einer algebraischen Körpererweiterung.
\ref{18.3-5} sagt dann gerade, dass endliche Körpererweiterungen immer algebraisch sind.
Die Umkehrung gilt nicht: es gibt durchaus unendliche algebraische Körpererweiterungen.

\begin{lem} \label{18.3-6}
	Seien $K \subset L \subset E$ Körper.
	Ist $\alpha \in E$ algebraisch über $K$, so ist $\alpha$ auch algebraisch über $L$ und $\mu_{L, \alpha}$ ist Teiler von $\mu_{K, \alpha}$.
	\begin{proof}
		Sei $\eps_\alpha: L[x] \to L(\alpha) \subset E$ der Auswertungshomomorphismus.
		Nun ist $0 \neq \mu_{K, \alpha} \in K[x] \subset L[x]$ ein Polynom mit $\mu_{K, \alpha}(\alpha) = 0$ und daher ist $\eps_\alpha(\mu_{K, \alpha}) = 0$, d.h. $\mu_{K, \alpha} \in \ker \eps_\alpha$ und $\eps_\alpha$ nicht injektiv, d.h. $\alpha$ algebraisch über $L$ und $\mu_{L,\alpha}$ ist definiert.

		Da per definitionem $\ker \eps_\alpha$ von $\mu_{L, \alpha}$ erzeugt wird, d.h. $\mu_{K,\alpha} \in \ker \eps_\alpha = \mu_{L,\alpha} L[x]$ ist $\mu_{L,\alpha}$ Teiler von $\mu_{K, \alpha}$ in $L[x]$.
	\end{proof}
\end{lem}

\begin{lem} \label{18.3-7}
	Sei $E := K(S)$ die Körpererweiterung von $K$ mit erzeugender Menge $S \subset E$.
	Dann ist
	\[
		E = \Set{ \beta = \dfrac{f(\alpha_1, \dotsc, \alpha_k)}{g(\alpha_1, \dotsc, \alpha_k)} & \begin{aligned} k \in \N, f,g \in K[x_1, \dotsc, x_k], \\\alpha_1, \dotsc, \alpha_k \in S, g(\alpha_1, \dotsc, \alpha_k) \neq 0 \end{aligned} }.
	\]
	\begin{proof}
		„$\supset$“ ist klar, da $f,g \in K[x_1, \dotsc, x_k]$ und $\alpha_1, \dotsc, \alpha_k \in S \subset E$.
		Nun erkennt man leicht, dass die Menge der $\beta$ den Körper $K$ enthält (für beliebiges $k \in K$ mit der Wahl $g = 1, f = k$ ist $\beta = k$) und abgeschlossen ist bezüglich Summen-, Produkt- und Inversenbildung, also ein Unterkörper von $E$ ist.
		Außerdem ist $S$ in diesem Unterkörper enthalten, denn für $g = 1, f = x$ ist $\beta = f(s) = s \in S$ für alle $s \in S$.
		Per definitionem ist $E$ die kleinste Körpererweiterung von $K$, die $S$ enthält, also gilt Gleichheit.
	\end{proof}
\end{lem}

\begin{st} \label{18.3-8}
	Sei $E = K(S)$ Körpererweiterung, $S \subset E$.
	Ist jedes $\alpha \in S$ algebraisch über $K$, so ist $E$ algebraisch über $K$.
	\begin{proof}
		Sei $\alpha \in E = K(S)$.
		Nach \ref{18.3-7} finden wir eine Darstellung $\alpha = \f {f(s_1, \dotsc, s_k)}{g(s_1, \dotsc, s_k)}$ mit $k \in \N$, $f, g \in K[x_1, \dotsc, x_k]$, $s_1, \dotsc, s_k \in S$ und $g(s_1, \dotsc, s_k) \neq 0$.
		Damit liegt $\alpha \in K(s_1, \dotsc, s_k)$.
		$s_i$ ist algebraisch über $K$ und daher nach \ref{18.3-6} auch über $K(s_1, \dotsc, s_{i-1})$, d.h. nach \ref{18.3-3} $n_i := [K(s_1,\dotsc, s_i): K(s_1,\dotsc,s_{i-1})] < \infty$ und wegen $K(\alpha) \subset K(s_1,\dotsc,s_k)$ und \ref{18.1-2} ist
		\[
			[K(\alpha) : K]
			\le [K(s_1,\dotsc,s_k) : K]
			= n_1 n_2 \dotsb n_k
			< \infty.
		\]
		Nach \ref{18.3-3} ist also $\alpha$ algebraisch über $K$.
	\end{proof}
\end{st}

\begin{kor} \label{18.3-9}
	Seien $K \subset E$ Körper und seien $0 \neq \alpha, \beta \in E$ algebraisch über $K$.
	Dann sind $\alpha \pm \beta, \alpha \beta, \alpha^{-1}, \beta^{-1}$ algebraisch über $K$.
	\begin{proof}
		Folgt sofort aus \ref{18.3-8} wegen $\alpha \pm \beta, \alpha\beta, \alpha^{-1}, \beta^{-1} \in K(\alpha, \beta)$.
	\end{proof}
\end{kor}

\begin{kor} \label{18.3-10}
	Sei $E = K(\alpha_1, \dotsc, \alpha_k)$ endlich erzeugte Körpererweiterung von $K$.
	Dann sind äquivalent:
	\begin{enumerate}[i)]
		\item
			$\alpha_1, \dotsc, \alpha_k$ sind algebraisch über $K$,
		\item
			$[E : K] < \infty$,
		\item
			$E$ ist algebraisch über $K$,
		\item
			$E = \Set{ f(\alpha_1, \dotsc, \alpha_k) & f \in K[x_1, \dotsc, x_k] }$.
	\end{enumerate}
	\begin{proof}
		\begin{seg}{\ProofImplication)[1][2]}
			Im Körperturm
			\[
				K \subset K(\alpha_1)
				\subset K(\alpha_1, \alpha_2)
				\subset \dotsb
				\subset K(\alpha_1, \dotsc, \alpha_{n-1})
				\subset K(\alpha_1, \dotsc, \alpha_k)
				= E
			\]
			ist $\alpha_i$ algebraisch über $K(\alpha_1, \dotsc, \alpha_{i-1})$ nach \ref{18.3-6}, also $n_i := [K(\alpha_1, \dotsc, \alpha_i) : K(\alpha_1, \dotsc, \alpha_{i-1})] < \infty$ nach \ref{18.3-3} und daher $[E : K] = n_1 n_2 \dotsb n_k < \infty$ nach \ref{18.1-2}.
		\end{seg}
		\begin{seg}{\ProofImplication)[2][3]}
			Für $\alpha \in E$ ist $K \subset K(\alpha) \subset E$ und daher $[K(\alpha) : K] \le [E : K] < \infty$, also ist nach \ref{18.3-3} $\alpha$ algebraisch über $K$.
		\end{seg}
		\begin{seg}{\ProofImplication)[3][4]}
			Zeige mittels Induktion über $k$.
			Für $k = 1$ ist dies gerade \ref{18.3-2} ii).

			Nun ist $\alpha_k$ algebraisch über $K(\alpha_1, \dotsc, \alpha_k)$ nach \ref{18.3-6} und daher
			\[
				E = \Set{ g(\alpha_k) & g \in (K(\alpha_1, \dotsc, \alpha_{k-1}))[x_k] }.
			\]
			Sei $g \in K(\alpha_1, \dotsc, \alpha_{k-1})[x_k]$.
			Per Induktion finden wir für jeden Koeffizienten $\beta_j \in K(\alpha_1, \dotsc, \alpha_{k-1})$ von $g$ ein Polynom $f_j \in K[x_i, \dotsc, x_{k-1}]$ so, dass $\beta_j = f_j(\alpha_1, \dotsc, \alpha_{k-1})$ ist.
			Also ist
			\[
				g(\alpha_k)
				= \sum_{j=0}^n \beta_j \alpha_k^j
				= \sum_{j=0}^n f_j(\alpha_1, \dotsc, \alpha_{k-1}) \alpha_k^j
				= f(\alpha_1, \dotsc, \alpha_k),
			\]
			mit $f(x_1, \dotsc, x_k) = \sum_{j=0}^n f_j(x_1, \dotsc, x_{k-1}) x_k^j \in K[x_1, \dotsc, x_k]$.
		\end{seg}
		\begin{seg}{\ProofImplication)[4][1]}
			Zeige mittels Induktion über $k$.
			Für $k = 1$ ist $E = K(\alpha_1) = \Set{ f(\alpha_1) & f \in K[x_1] }$ und nach \ref{18.3-2} ii) ist $\alpha_1$ algebraisch über $K$.

			Sei $k > 1$ und $\alpha_1, \dotsc, \alpha_{k-1}$ bereits algebraisch über $K$.
			Sei $f(\alpha_1, \dotsc, \alpha_k) := \alpha_k^{-1}$ mit $f \in K[x_1, \dotsc, x_k] = K[x_1, \dotsc, x_{k-1}][x_k]$.
			Schreibe $f = \sum_{j=0}^n g_j x_k^j$ mit $g_j \in K[x_1, \dotsc, x_{k-1}]$ und setze $\beta_j := g_j(\alpha_1, \dotsc, \alpha_{k-1})$.
			Dann ist $\alpha_k^{-1} = \sum_{j=0}^n \beta_j \alpha_k^j$ und daher $1 = \alpha_k \alpha_k^{-1} = \sum_{j=0}^n \beta_j \alpha_k^{j+1}$.
			Wählt man
			\[
				h(x_k) := x_k \sum_{j=0}^n \beta_j x_k^j - 1 = \sum_{j=0}^n \beta_j x_k^{j+1} - 1
				\in K(\alpha_1, \dotsc, \alpha_{k-1})[x_k],
			\]
			so ist also $h(\alpha_k) = 0$, d.h. $\alpha_k$ ist algebraisch über $K(\alpha_1, \dotsc, \alpha_{k-1})$.
			Also ist $[K(\alpha_1, \dotsc, \alpha_k) : K(\alpha_1, \dotsc, \alpha_{k-1}) ] < \infty$ nach \ref{18.3-3} und nach Induktionsannahme und ii): $[K(\alpha_1, \dotsc, \alpha_{k-1} : K] < \infty$, es gilt somit
			\begin{align*}
				[K(\alpha_k) : K]
				&\le [K(\alpha_1, \dotsc, \alpha_k) : K] \\
				&= \underbrace{[K(\alpha_1, \dotsc, \alpha_k) : K(\alpha_1, \dotsc, \alpha_{k-1})]}_{<\infty} \cdot \underbrace{[ K(\alpha_1, \dotsc, \alpha_{k-1}) : K]}_{<\infty} \\
				&< \infty.
			\end{align*}
			Also ist $\alpha_k$ algebraisch über $K$ nach \ref{18.3-3}.
		\end{seg}
	\end{proof}
\end{kor}

\begin{kor} \label{18.3-11}
	Sei $K \subset E$ Körpererweiterung und sei \(L = \{ \alpha \in E : \text{$\alpha$ algebraisch über $K$}\}\).
	Dann ist $L$ ein Körper mit $K \subset L \subset E$ und wird \emphdef[algebraischer Abschluss]{algebraische Abschluss} von $K$ in $E$ genannt.
	\begin{proof}
		Nach \ref{18.3-8} ist die Erweiterung $K(L)$ algebraisch über $K$, also in $L$ enthalten und somit $L \subset K(L) \subset L$.
		Damit ist $L = K(L)$ ein Körper.
	\end{proof}
\end{kor}

\begin{st} \label{18.3-12}
	Seien $K \subset L \subset E$ Körpererweiterungen mit $E$ algebraisch über $L$ und $L$ algebraisch über $K$.
	Dann ist $E$ algebraisch über $K$, d.h. für Körper ist die Relation „algebraisch über“ transitiv.
	\begin{proof}
		Sei $\alpha \in E$ mit $\mu_{\alpha, L} = \lambda_0 + \lambda_1 x + \dotsb + \lambda_{n-1} x^{n-1} + x^n \in L[x]$.
		Dann sind $\lambda_0, \dotsc, \lambda_{n-1} \in L$ und daher algebraisch über $K$, also ist $F := K(\lambda_0, \dotsc, \lambda_{n-1})$ algebraisch über $K$ mit $[F : K] < \infty$ nach \ref{18.3-10}.

		Nun ist $\lambda_0, \dotsc, \lambda_{n-1} \in F$ und $\mu_{L, \alpha}$ irreduzibel in $F[x]$, da irreduzibel in $L[x] \supset F[x]$, also mit $\mu_{L, \alpha}(\alpha) = 0$ ist gerade $\mu_{F, \alpha} = \mu_{L, \alpha}$ das Minimalpolynom von $\alpha$ über $F$.
		Somit ist
		\[
			[K(\alpha) : K]
			\le [F(\alpha) : K]
			= \underbrace{[F(\alpha) : F]}_{=\deg(\mu_{L, \alpha}) < \infty} \cdot \underbrace{[F : K]}_{< \infty}
			< \infty
		\]
		und daher $\alpha$ algebraisch über $K$ nach \ref{18.3-3} oder \ref{18.3-10}.
		Somit ist $E$ algebraisch über $K$.
	\end{proof}
\end{st}


\section{Fortsetzungssatz und algebraischer Abschluss}


Sei im Folgenden $K$ ein Körper.

\begin{df} \label{18.4-1}
	Seien $E, L$ Körpererweiterungen von $K$.
	Ein $K$-linearer Körperhomomorphismus $0 \neq \sigma: E \to L$ (diese sind stets injektiv) heißt \emphdef[K-Homomorphismus]{$K$-Homomorphismus}.
	Er lässt sich durch $\sum_{i=0}^n \alpha_i x^i \mapsto \sum_{i=0}^n \sigma(\alpha_i)x^i \in L[x]$ zu einem $K$-linearen, injektiven $K$-Algebra Homomorphismus von $E[x]$ in $L[x]$ fortsetzen.
	$E$ und $L$ heißen $K$-isomorph oder \emphdef{isomorph über} $K$, falls es einen $K$-linearen Körperisomorphismus $\sigma: E \to L$ gibt.
	Dann ist $\sigma: E[x] \to L[x]$ ebenfalls $K$-Algebra Isomorphismus.
	Weiter ist $\sigma(\lambda) = \lambda$ für alle $\lambda \in K$, d.h. $\sigma|_K = \id_K$.
\end{df}

\begin{df} \label{18.4-2}
	Seien $K \subset E$, $L$ Körper und $0 \neq \sigma: K \to L$ ein Körperhomomorphismus (also $K \isomorphic \im \sigma = \sigma(K) \subset L$).
	Wir sagen, der Körperhomomorphismus $\tau: E \to L$ \emphdef{erweitert} $\sigma$, bzw. \emphdef[fortsetzen]{setzt $\sigma: K \to L$ auf $E$ fort}, falls $\tau|_K = \sigma$ ist, d.h. das folgende Diagram kommutiert:
	\[
		\begin{tikzcd}
			K \arrow[inj]{r}{\iota} \arrow{rd}{\sigma} & E \arrow{d}{\tau} \\
			& L
		\end{tikzcd}
	\]
	($\tau \circ \iota = \sigma$).
	Ist dazuhin $K \subset L$ und $\sigma: K \to L$ die natürliche Einbettung, so ist $\tau|_K = \id_K$, d.h. $\tau$ ist $K$-linear (und wir sind in der Situation von \ref{18.4-1}).
\end{df}

\begin{lem} \label{18.4-3}
	Seien $E, L$ Körper, $0 \neq \sigma: E \to L$ Körperhomomorphismus und sei $\eps \in E$ Nullstelle von $f \in E[x]$.
	Dann ist $\sigma(\eps) \in L$ Nullstelle von $\sigma(f) \in L[x]$.
\end{lem}

Sei nun $K \subset E$, $\alpha \in E$ algebraisch über $K$, so ist $K \subset K(\alpha) \subset E$.
Sei $0 \neq \sigma: K \to L$ Körperhomomorphismus, $L$ Körper.
Im nächsten Satz formulieren wir eine notwendige und hinreichende Bedingung, wann $\sigma$ zu einem Homomorphismus von $K(\alpha)$ nach $L$ fortgesetzt werden kann und untersuchen, wie viele verschiedene Fortsetzungen von $\sigma$ auf $K(\alpha)$ existieren.

\begin{st} \label{18.4-4}
	Seien $K \subset E$ und $L$ Körper, $\alpha \in E$ algebraisch über $K$ mit Minimalpolynom $\mu_{\alpha, K} = p \in K[x]$.
	Sei $0 \neq \sigma: K \to L$ Körperhomomorphismus und sei $\beta \in L$ Nullstelle von $\sigma(p) \in L[x]$.
	Dann gibt es genau eine Fortsetzung $\sigma_\beta: K(\alpha) \to L$ von $\sigma$ auf $K(\alpha)$ mit $\sigma_\beta(\alpha) = \beta$.
	Dies induziert eine Bijektion $\beta \leftrightarrow \sigma_\beta$ zwischen den verschiedenen Nullstellen von $\sigma(p) \in L[x]$ und den Fortsetzungen $\sigma_\beta$ von $\sigma$ auf $K(\alpha)$.
	Insbesondere besitzt daher $\sigma$ höchstens $\deg p$ viele Fortsetzungen auf $K(\alpha)$ (mit Gleichheit genau dann, wenn $\sigma(p)$ über $L[x]$ in $\deg p$ viele verschiedene lineare Polynome zerfällt).
\end{st}

\begin{df} \label{18.4-5}
	Ein Körper $E$ heißt \emphdef{algebraisch abgeschlossen}, falls jedes $f \in E[x]$ mit $\deg f \ge 1$ eine Nullstelle in $E$ besitzt.
	Ist $K \subset E$ algebraische Körpererweiterung mit algebraisch abgeschlossenen $E$, so heißt $E$ \emphdef{algebraischer Abschluss} von $K$.
\end{df}

\begin{note}
	Sei $f = x - \alpha \in E[x], \alpha \in E$.
	Dann ist $f$ irreduzibel in $E[x]$ und $E[x] / fE[x] \isomorphic E$.
	Nur irreduzible Polyonome aus $E[x]$ vom Grad größer gleich 2 „produzieren“ echte Körpererweiterungen von $E$.
	Auch die Gradgleichung $[E[x] / fE[x] : E] = \deg f$ liefert dies, siehe \ref{18.3-2} ii).
\end{note}

\begin{kor} \label{18.4-6}
	Sei $E$ ein Körper.
	Dann ist $E$ algebraisch abgeschlossen genau dann, wenn jedes nicht konstante Polynom in $E[x]$ in Linearfaktoren zerfällt oder äquivalent, wenn $\{x - \alpha : \alpha \in E\} \subset E[x]$ exakt die Menge der irreduziblen, normierten Polynome ist.
	$E$ besitzt dann keine algebraische echte Körpererweiterungen und insbesondere sind daher alle Körpererweiterungen von $E$ unendlich dimensional über $E$.
	$E(x) = Q(E(x))$ ist dann echte, notwendig transzendente Körpererweiterung von $E$.
	Diese ist nicht algebraisch abgeschlossen, z.B. ist $t^2 - x \in (E(x))[t]$ irreduzible mit Nullstelle $\sqrt x \not\in E$.
\end{kor}

\begin{st}[Fortsetzungsatz] \label{18.4-7}
	Sei $K \subset E$ algebraische Körpererweiterung und sei $L$ algebraisch abgeschlossen.
	Dann besitzt jeder Körperhomomorphismus $0 \neq \sigma: K \to L$ eine Fortsetzung $\sigma': E \to L$:
	\[
		\begin{tikzcd}
			E \arrow{dr}{\exists \sigma'} & \\
			K \arrow{u}{\iota} \arrow{r}{\sigma} & L
		\end{tikzcd}
	\]
	mit $\sigma'|_K = \sigma$.
\end{st}

\begin{nt} \label{18.4-8}
	\begin{enumerate}[1.)]
		\item
			Der Beweis von \ref{18.4-7} benötigt das Lemma von Zorn, ist also nicht konstruktiv.
		\item
			Die Fortsetzung $\sigma'$ von $\sigma: K \to L$ in \ref{18.4-7} ist im Allgemeinen nicht eindeutig.
	\end{enumerate}
\end{nt}

\begin{st} \label{18.4-9}
	Sei $K$ ein Körper.
	Dann besitzt $K$ einen algebraischen Abschluss und dieser ist bis auf $K$-Isomorphie eindeutig bestimmt.
\end{st}

\begin{conv*}
	Im folgenden wird der algebraische Abschluss eines Körpers $K$ mit $\_K$ bezeichnet.
	Ist $K \subset E$ algebraische Körpererweiterung, so ist $\_E$ algebraisch über $E$ und daher nach \ref{18.3-12} auch über $K$, d.h. ist ein algebraischer Abschluss von $K$ und für dürfen daher $\_K = \_E$ setzen.

	Alle über $K$ algebraischen Elemente könnene ohne Einschränkung in $\_K$ gewählt werden.
\end{conv*}

\begin{nt} \label{18.4-10}
	\begin{enumerate}[1.)]
		\item
			$\C$ ist algebraisch abgeschlossen (Hauptsatz der Algebra) und $\C = \R(i), i^2 = -1$ ist also der algebraische Abschluss von $\R$.
		\item
			Man kann zeigen:
			Der algebraische Abschluss endlicher Körper ist abzählbar unendlich (kann konkret konstruiert werden).
		\item
			Ist $|K| = \infty$, so ist $|\_K| = |K|$.
			Insbesondere ist der algebraische Abschluss $\_\Q$ von $\Q$ abzählbar und daher $\_\Q \subsetneq \C$.
			In der Tat besitzt $\C$ überabzählbar viele Elemente, die transzendent über $\Q$ sind: $\C \isomorphic \_{\Q(X)}$, wobei $\Q(X)$ der Körper der rationalen Funktionen in überabzählbar vielen Variablen $x \in X$ ist.
	\end{enumerate}
\end{nt}


\section{Konstruktion mit Zirkel und Lineal}


Um die Konstruktionen mit Zirkel und Lineal greifbar zu machen, wollen wir Geometrie in Algebra „übersetzen“.

\begin{df} \label{18.5-1}
	Sei $\scr P \subset \R^2$ eine Punktmenge.
	Konstruktionen mit Zirkel und Lineal erlauben folgende Operationen:
	\begin{enumerate}[1)]
		\item
			Für $A, B \in \scr P, A \neq B$ ziehe eine Gerade durch $A$ und $B$ (Lineal benötigt).
			Wir bezeichnen diese Gerade mit $AB$.
		\item
			Für $A, B \in \scr P, A \neq B$ schlage einen Kreis $K(A, B)$ um $A$ mit Radius $d(A, B)$, d.h. dem Abstand von $A$ zu $B$.
	\end{enumerate}
	Wir definieren die Menge aller \emphdef[konstruierbar!im ersten Schritt]{aus $\scr P$ im ersten Schritt konstruierbaren Elemente} als
	\[
		S(\scr P) :=
		\scr P \cup \l\{ \begin{aligned} AB \cap CD,\; AB \cap K(C, D), \\ K(A, B) \cap K(C, D) \end{aligned} \;\Bigg|\; \begin{aligned} A,B,C,D \in \scr P, \\ A \neq B, C \neq D \end{aligned} \r\}.
	\]
	Sei $\scr P_0 := \scr P$ und induktiv $\scr P_i := \scr P_{i-1}$ für alle $i \in \N$.
	Die Menge aller \emphdef[konstruierbar]{aus $\scr P$ konstruierbaren Elemente} ist definiert als
	\[
		\_{\scr P} := \bigcup_{i=0}^\infty \scr P_i.
	\]
\end{df}

Besteht $\scr P_0$ nur aus einem Punkt, lässt sich nichts konstruieren, d.h. $\_{\scr P} = \scr P$.
Sei im Folgenden deshalb stets $|\scr P_0| \ge 2$.


\begin{ex}[Geometrische Konstruktionen] \label{18.5-2}
	\begin{enumerate}[i)]
		\item
			Für $A, B \in \scr P$ ist $\Z B \subset \_{\scr P}$ konstruierbar.
		\item
			Die Winkelhalbierende ist konstruierbar.
		\item
			Die Parallele durch einen Punkt ist konstruierbar.
		\item
			Für drei gegeben Punkte ist das Parallelogramm konstruierbar.
		\item
			Unsere Definition der Konstruktionsoperationen erlaubt nicht das Übertragen von Streckenlängen von einer Geraden auf eine andere (mit einem Zirkel könnte man die Länge einstellen und auf die andere Stelle übertragen).
			Man spricht von einem \emphdef[kollabierender Zirkel]{kollabierendem Zirkel}, der „zuschnappt“, sobald wir ihn vom Papier abheben.

			Trotzdem ist das Übertragen von Streckenlängen auch mit einem kollabierendem Zirkel möglich.
			Für gegebenes $A, B, C, E \in \scr P$ lässt sich die Streckenlänge $d(A, B)$ auf die Gerade $CE$ an $C$ übertragen, d.h. ein Punkt $F$ in $CE$ ist konstruierbar, sodass $d(A, B) = d(C, F)$.
			Nutze dafür iv) und iii).
		\item
			Teilen einer Strecke in $n$ gleiche Teile.
		\item
			Verdopplung/Halbierung von Winkeln, Übertrag von Winkeln, Dreieckskonstruktionen, Addition von Winkeln, Beweis der Strahlensätze, \dots
	\end{enumerate}
\end{ex}

Nun wirds algebraisch/analytisch.
Wir identifizieren die Punktmenge der euklidischen Ebene $E$ mit dem $\R^2$ und den $\R^2$ mit $\C$, der komplexen Zahlenebene.
Wir wählen $A, B \in \scr P_0$ mit $A \neq B$ und koordinatisieren $E$, indem wir $A$ als Ursprung, $AB$ als $x$-Achse und $d(A,B)$ als Längeneinheit wählen: $A = 0 \in \C, B = 1 \in \C, d(A, B) = 1$.
Dann ist nach \ref{18.5-2} i) auch $-B = -1 \in \C$ in $\_{\scr P_0}$ enthalten.
Als Imaginärachse wählen wir die Mittelsenkrechte von $-B$ und $B$ und tragen dort (mit $K(A,B)$ den Punkt $i \in \C$ mit $d(A, i) = d(A, B) = 1$ als Längeneinheit ab).
Dies definiert eine Bijektion zwischen den Punkten der euklidischen Ebene $E$ und dem Körper der komplexen Zahlen.

\begin{df} \label{18.5-3}
	Sei $\scr P_0$ ein Punktmenge der euklidischen Ebene $E$ mit $|\scr P_0| \ge 2$, $A, B \in \scr P_0$ mit $A \neq B$.
	Sei $S \in E$, dann bezeichnen wir mit $z_S \in \C$ das Bild von $S$ in $\C$ unter der oben definierten Bijektion $E \to \C$.
	Sei $M = \{z_C : C \in \scr P_0\}$.
	Dann heißt $z \in \C$ \emphdef[geometrisch konstruierbar]{geometrisch aus $M$ konstruierbar}, wenn $z = z_D$ für ein $D \in \_{\scr P_0}$.
	Die Menge der aus $M$ geometrisch konstruierbaren komplexen Zahlen wird mit $G(M)$ bezeichnet.
	So ist $G(M) = \{ z_D : D \in \_{\scr P_0} \} \subset \C$.
\end{df}

\begin{note}
	Für $z = r + si \in \C$ ($r, s \in \R$) ist $\_z = r - si \in \C$ die konjugiert komplexe Zahl.
	In der komplexen Zahlenebene erhält man $\_z$ geometrisch durch Spiegelung von $z$ an der rellen Achse ($x$-Achse).
\end{note}

\begin{st} \label{18.5-4}
	Sei $0, 1 \in M \subset \C$.
	Dann ist $G(M)$ ein Unterkörper von $\C$, also Abgeschlossen bezüglich Addition und Multiplikation und Bildung additiver und multiplikativer Inversen.
	Außerdem gilt: $z \in G(M)$, so gilt auch $\_z$, $\sqrt z \in G(M)$.
	\begin{proof}
		\begin{enumerate}[i)]
			\item
				$z \in G(M) \implies -z \in G(M)$
			\item
				Addition (mit Parallelogramm)
			\item
				Multiplikation/Division von reellen Zahlen mit Strahlensätzen.
				Mit Polarkoordinaton und Addition von Winkeln folgt Multiplikation/Division im Komplexen.
			\item
				Komplexe Konjugation
			\item
				Quadratwurzel:
				genügt wegen Polarkoordinaten für reelle Zahlen: $\sqrt{r}$ mit Thaleskreis über $-r, 1$.
		\end{enumerate}
	\end{proof}
\end{st}

Ist $0, 1 \in M \subset \C$ und sei $S \subset \C$ aus $M$ geometrisch konstruierbar, dann ist $G(M) = G(M \cup S)$.
Insbesondere ist die imaginäre Einheit $i \in \C$ immer aus $1, 0 \in M$ konstruierbar.
Wir können daher immer $i \in M$ annehmen.

Für $z \in \C$ sei $P_z \in \R^2$ der Punkt mit den Koordinaten $\alpha, \beta$ für $z = \alpha + \beta i$, $\alpha, \beta \in \R$.

\begin{conv} \label{18.5-5}
	Für $x,y \in \C$, $x \neq y$ sei $x \vee y$ die Verbindungsgreade $P_x P_y \subset \R^2$ und $K(X, Y) \subset \R^2$ der Kreis um $x$ mit Radius $|x-y| = \sqrt{(x-y)\_{(x-y)}} \in \R_{\ge 0}$.
\end{conv}

\begin{st} \label{18.5-6}
	Sei $L = \_L \subset \C$ ein Körper und $z_1, w_1, z_2, w_2 \in L$, $(z_1, w_1) \neq (z_2, w_2)$ und sei $g_j = z_j \vee w_j$ oder $g_j = K(z_j, w_j)$ ($j \in \{1, 2\}$).
	Sei $z = g_1 \cap g_2 \in \C, L_1 = L(z) \subset \C$.
	Dann ist $L_1 = \_{L_1}$ und $[L_1 : L] \le 2$.
\end{st}

\begin{kor} \label{18.5-7}
	Sei $0, 1 \in M \subset \C, L_0 = \Q(M \cup \_{M}) \subset \C$.
	Sei $z \in G(M)$.
	Dann gibt es eine Folge von Körpern
	\[
		L_0 \subset L_1 \subset \dotsb \subset L_k \subset \C
	\]
	mit $z \in L_k$ und $[L_i : L_{i-1}] \le 2$ für $i \in \{1, \dotsc, k\}$, $L_k = L_{k-1}(z)$
	Insbesondere ist $z$ und daher $G(M)$ algebraisch über $L_0$ und $[L_0(z) : L_0]$ eine Potenz von $2$.
\end{kor}

Als Konsequenz lässt sich nun zeigen, dass einige (antike) Konstruktionsaufgaben der elmentaren Geometrie nicht lösbar sind.

\begin{ex}[Quadratur des Kreises] \label{18.5-8}
	Konstruiere mit Zirkel und Lineal ein Quadrat mit dem selben Flächeninhalt wir der Einheitskreis (d.h. mit Fläche $\pi^2$, Seitenlängen $\pi$).

	Die Lösung würde auf die geometrische Konstruktion von $\pi \in \C$ führen, d.h. $\pi$ müsste algebraisch über $\Q(i)$ für $\scr P_0 = \{0, 1\}$ sein nach \ref{18.5-7}, $\pi$ ist aber transzendent (siehe Analysis).
\end{ex}
