\documentclass[a4paper,10pt]{scrbook}
\usepackage{mathe-vorlesung}
\usepackage{legacy}

\title{Analysis 3}

\begin{document}

\maketitle

\tableofcontents
\newpage


% Kapitel 1
\chapter{Fundamentalgruppe}

\begin{df}
    Sei $X$ ein topologischer Raum, $x_0 \in X$ ein Punkt.
    \begin{math}
        P(X) &= \scr C([0,1], X), \\
        P(X, a, b) &= \Set{\gamma \in P(X) & \gamma(0) = a, \gamma(1) = b}.
    \end{math}
    Spezielle Wege, bzw Wegoperationen:
    \begin{itemize}
        \item
            $1_a \in P(X, a, a)$, $1_a(t) := a$ für $0 \le t \le 1$,
        \item
            $\_\argdot: P(X, a, b) \to P(X, b, a)$, $\gamma \mapsto \_\gamma$, $\_\gamma(t) := \gamma(1-t)$,
        \item
            $\ast: P(X, a,b) \times P(X, b,c) \to P(X,a,c)$,
            \begin{math}
                (\gamma_1 \ast \gamma_2)(t) := \begin{cases}
                    \gamma_1(2t) & \text{für $0 \le t \le \frac{1}{2}$} \\
                    \gamma_2(2t - 1) & \text{für $\frac{1}{2} \le t \le 1$}
                \end{cases}
            \end{math}
    \end{itemize}
    Zwei Wege $\alpha, \alpha' \in P(X,a,b)$ heißen \emphdef{äquivalent}, genauer \emphdef{homotop bei festen Endpunkten}, wenn eine stetige Abbildung $H: [0,1] \times [0,1] \to X$ existiert mit $H(0, t) = \alpha(t)$, $H(1, t) = \alpha'(t)$, sowie $H(s, 0) = a$, $H(s, 1) = b$ für alle $s,t \in [0,1]$.
    Wir schreiben dann $H: \alpha \sim \alpha'$ oder kurz $\alpha \sim \alpha'$.
\end{df}

\begin{prop}
    \begin{itemize}
        \item
            Die Äquivalenz $\sim$ ist eine Äquivalenzrelation.
        \item
            Aus $\alpha \sim \beta$ folgt $\_\alpha \sim \_\beta$.
        \item
            Aus $\alpha \sim \alpha'$ und $\beta \sim \beta'$ folgt $\alpha \ast \beta \sim \alpha' \sim \beta'$.
    \end{itemize}
\end{prop}

Wir erhalten wohldefinierte Abbildungen auf $\Pi(X, a, b) := P(X, a, b) / \sim$ durch
\begin{itemize}
    \item
        $\_\argdot: \Pi(X,a,b) \to \Pi(X,a,b)$, $\_{[\gamma]} := [\_\gamma]$.
    \item
        $\ast: \Pi(X,a,b) \times \Pi(X,b,c) \to \Pi(X,a,c)$, $[\alpha] \ast [\beta] := [\alpha \ast \beta]$.
\end{itemize}

\begin{st}
    Jeder topologische Raum $X$ definiert so seine \emphdef{Wegekategorie} $\Pi(X)$ (auch \emphdef{Fundamentalgruppoid} genannt).
    \begin{enumerate}[a)]
        \item
            Objekte sind die punkte $a, b, c, \dotsc \in X$,
        \item
            Morphismen $[\alpha]: a \to b$ sind Wegeklassen von $a$ nach $b$.
        \item
            Verknüpfung $\ast$ ist die Konkatenation wie oben.
    \end{enumerate}
    Die Verknüpfung erfüllt
    \begin{enumerate}[1)]
        \item
            Identität: Für $\alpha: a \to b$ gilt
            \begin{math}
                1_a \ast \alpha \sim \alpha \sim \alpha \ast 1_b,
            \end{math}
            also
            \begin{math}
                [1_a] \sim [\alpha] = [\alpha] = [\alpha] \ast [1_b].
            \end{math}
        \item
            Inversion: Für $\alpha: a \to b$ und $\_\alpha: b \to a$ gilt
            \begin{math}
                \alpha \ast \_\alpha &\sim 1_a, &
                \_\alpha \ast \alpha &\sim 1_b
            \end{math}
            also
            \begin{math}
                [\alpha] \ast [\_\alpha] &= [1_a], &
                [\_\alpha] \ast [\alpha] &= [1_b],
            \end{math}
        \item
            Assoziativität: Für $a \xto{\alpha} b \xto{\beta} c \xto{\gamma} d$ gilt $(\alpha \ast \beta) \ast \gamma \sim \alpha \ast (\beta \ast \gamma)$, also
            \begin{math}
                ([\alpha] \ast [\beta]) \ast [\gamma] = [\alpha] \ast ([\beta] \ast [\gamma]).
            \end{math}
    \end{enumerate}
    \begin{proof}
        Skizzenbeweis.
    \end{proof}
\end{st}

\begin{st}
    Jede stetige Abbildung $f: X \to Y$ induziert einen Funktor
    \begin{math}
        f_\#: \Pi(X) &\to \Pi(Y) \\
        a &\mapsto f(a) \\
        [\alpha: a \to b] &\mapsto [f \circ \alpha: f(a) \to f(b)]
    \end{math}
\end{st}

\begin{df}
    Vom Fundamentalgruppoid zur Fundamentalgruppe, definiere
    \begin{math}
        \pi_1(X, x_0) := \Pi(X, x_0, x_0)
        = \frac{\Set{\text{Schleifen $\alpha: ([0,1], \Set{0,1}) \to (X,x_0)$}}}{\text{Homotopie relativ $\Set{0,1}$}}.
    \end{math}
    Dies ist eine Gruppe (Übung) mit der Verknüpfung $[\alpha] \ast [\beta] = [\alpha \ast \beta]$.

    Jede stetige Abbildung $f: (X, x_0) \to (Y, y_0)$ induziert einen Gruppenhomomorphismus $f_\# = \pi_1(f): \pi_1(X, x_0) \to \pi_1(Y, y_0)$ mit $[\alpha] \mapsto f_\#([\alpha]) = [f \circ \alpha]$.
    Wir erhalten einen Funktor
    \begin{math}
        \pi_1: \Cat{Top}_* &\to \Cat{Grp} \\
        (X, x_0) &\mapsto \pi_1(X, x_0) \\
        (f: (X,x_0) \to (Y, y_0)) &\mapsto (\pi_1(f): \pi_1(X, x_0) \to \pi_1(Y, y_0)).
    \end{math}
\end{df}

\begin{ex}
    Sei $X = \R^n$ oder $X \subset \R^n$ konvex oder sternförmig bezüglich $x_0$.
    Dann ist $X$ wegzusammenhängend, d.h. $\pi_0(X) = \Set{[x_0]}$, und sogar einfach zusammenhängend, d.h. zudem
    \begin{math}
        \pi_1(X, x_0) = \Set{[1_{x_0}]},
    \end{math}
    kurz $\pi_1(X, x_0) = \Set{1}$.
    \begin{proof}
        Zu $\alpha: [0,1] \to X$ betrachte $H(s, t) = (1-s)\alpha(t) + s x_0$.
        $H: [0,1] \times [0,1] \to X$ ist eine Abbildung, da $X$ sternförmig ist.
        Sie ist stetig und erfüllt $H: \alpha \sim 1_{x_0}$.
    \end{proof}
\end{ex}

\paragraph{Offene Mengen $X \subset \R^n$ und polygonale Fundamentalgruppe}

Sei $X \subset \R^n$ und $x_0 \in X$. Wir definieren die polygonale Fundamentalgruppe
\begin{math}
    \pi_1^{\text{pl}}(X,x_0) := \Pi^{\text{pl}}(X, x_0, x_0)
    = \frac{\Set{\text{geschlossene Polygonzüge in $(X, x_0)$}}}{\text{polygonale Homotopie in $X$}}.
\end{math}

\begin{st}
    Sei $X \subset \R^n$ offen und $x_0 \in X$.
    Wir haben einen Gruppenisomorphismus
    \begin{math}
        \phi: \pi_1^{\text{pl}}(X, x_0) \to \pi_1(X, x_0)
    \end{math}
    \begin{proof}
        (Skizze: stetiger/polygonaler Weg)
        Surjektivität: Jede stetige Abbildung lässt sich beliebig genau durch Polygone approximieren.
        Injektivität: Stetige Homotopie lässt sich durch polygonale Homotopie approximieren.
    \end{proof}
\end{st}

\begin{ex}
    Sei $X := \R^2 \setminus \Set{0}$, $x_0 := (1, 0)$ und $\gamma$ ein geschlossener polygonaler Weg in $X$ von $x_0$ (Skizze).
    Durch zählen der Übergänge über die negative reelle Achse erhalten wir $\deg: \pi_1^{\text{pl}}(X, x_0) \to \Z$.
    Dies ist ein Gruppenisomorphismus
    \begin{proof}
        Wohldefiniertheit: Übergang von Wegen zu Wegeklassen.        
        Homomorphismus.
        Surjektivität: Konstruktion.
        Injektivität: Umlaufzahl $0$ betrachten: neg/pos Übergänge eliminieren (Punkte sternförmig um $x_0$).
    \end{proof}
    Kurz:
    \begin{math}
        \deg: \pi_1(X, x_0) \isomorphic \pi_1^{\text{pl}}(X, x_0) \isomorphic \Z.
    \end{math}
\end{ex}

\begin{ex}
    Sei $X := \C \setminus \Set{0, -1, \dotsc, 1 - n}$, $x_0 = 1$ (Skizze mit Weg).
    Kodiere Übergänge: $s_1, \dotsc, s_n$.

    Wir erhalten $\phi: \pi_1^{\text{pl}}(X, x_0) \to \Gen{ s_1, \dotsc, s_n & - }$.
    Dies ist ein Gruppenisomorphismus.
    \begin{proof}
        Wohldefiniertheit: Übergang von Wegen zu Wegeklassen: Kürzung.
        Homomorphismus: klar.
        Surjektivität: Konstruktion.
        Injektivität: Betrachte Wege, die auf $1$ abgebildet werden, Induktion über Wortlänge durch Kürzen.
    \end{proof}
    \begin{note}
        Die Gruppe ist für $n \ge 2$ nicht kommutativ!
    \end{note}
\end{ex}

\paragraph{Präsentation von Gruppen durch Erzeuger und Relationen}

Kurzfassung: Sei $A$ eine Menge. $A^* := \bigcup_{n \in \N} A^n$ ist die Menge aller Wörter über dem Alphabet $A$.
Für $n = 0$ ist $e = ()$ das leere Wort.
Für $n = 1$ identifizieren wir $(a) \in A^*$ mit $a \in A$.

Die Verkettung $\circ: A^* \times A^*$ ist gegeben durch die Konkatenation der Wörter
\begin{math}
    (a_1, \dotsc, a_n)(a_1', \dotsc, a_m') := (a_1, \dotsc, a_m, a_1', \dotsc, a_n').
\end{math}
Damit ist $(A^*, \circ, e)$ ein Monoid, genannt das \emphdef[freies Monoid]{freie Monoid} über $A$.

Wir wollen Relationen der Form $w_1 = w_2$ einführen.
Hierzu sei $K \subset A^* \times A^*$.
Auf $A^*$ sei $\equiv$ die Äquivalenzrelation, die erzeugt wird durch die elementaren Umformungen
\begin{math}
    u \circ w_1 \circ v \equiv u \circ w_2 \circ v, && \text{für $(w_1, w_2) \in K$},
\end{math}
Diese Kongruenz ist verträglich mit $\circ$, d.h. $u \equiv u'$ und $v \equiv v'$, dann ist $u \circ v \equiv u' \circ v'$.

Auf $Q := A^* / K := A^* / \equiv$ erhalten wir $\argdot: Q \times Q \to Q$, $[u] \cdot [v] := [u \circ v]$.
Damit ist auch $(Q, \cdot, [e])$ ein Monoid.


\begin{df}
    Das durch $(A, K)$ \emphdef{präsentierte Monoid} ist
    \begin{math}
        \GenMonoid{A & K} := A^* / K.
    \end{math}
\end{df}

\begin{ex}
    \begin{itemize}
        \item
            \begin{math}
                (N = \GenMonoid{a & -}, \cdot) &\xto[homeomorphic] (\N, +) \\
                a^n &\mapsto n \\
                a^n &\mapsfrom n
            \end{math}
        \item
            \begin{math}
                \GenMonoid{a,b & -}
                = \Set{e, a, b, aa, ab, ba, bb, \dotsc}
            \end{math}
        \item
            \begin{math}
                C = \GenMonoid{s^+, s^- & s^+s^- = e, s^-s^+ = e},
            \end{math}
            d.h. $A = \Set{s^+, s^-}$, $K = \Set{(s^+s^-, e), (s^-s^+, e)}$.
            Dies ist eine Gruppe.
            Definiere
            \begin{math}
                \phi: (\Z, +) &\to (C, \cdot) \\
                k &\mapsto \begin{cases}
                    (s^+)^k & \text{für $k > 0$}, \\
                    e & \text{für $k = 0$}, \\
                    (s^-)^{-k} & \text{für $k < 0$}.
                \end{cases}
            \end{math}
            Dies ist ein Gruppenhomomorphismus, surjektiv (auf Wortklasssen).
            Inverse:
            \begin{math}
                \psi: (C, \cdot) &\to (\Z, +), \\
                s^{\eps_1} s^{\eps_2} \dotsb s^{\eps_l} &\mapsto \eps_1 + \dotsb + \eps_l.
            \end{math}
            Dies ist wohldefiniert, Gruppenhomomorphismus, surjektiv.
            Es gilt $\psi \circ \phi: \id_\Z$, aber auch $\phi \circ \psi = \id_C$.
        \item
            $C_n := C_{n, 0} := \GenMonoid{a & a^n = 1}$ (Skizze: Kreis).
            Es gilt $(C_n, \cdot) \isomorphic (\Z / n, +)$.
        \item
            $C_{n,m} := \GenMonoid{a & a^n = a^m}$ für $0 \le m < n$ (Skizze: Anfang + Schleife). 
    \end{itemize}
\end{ex}

Speziell für Gruppen:
Zur Menge $S$ wählen wir das Alphabet
\begin{math}
    A = S \times \Set{\pm} = \Set{s^+, s^- & s \in S}
\end{math}
Zu $R \subset A^*$ setzen wir $K = \Set{ r = 1 & r \in R} \cup \Set{s^+s^- = 1, s^-s^+ = 1 & s \in S}$.
Formal:
\begin{math}
    K = \Set{(r, e) & r \in R} \cup \Set{(s^+s^-, e), (s^-s^+, e) & s \in S}.
\end{math}
Die durch $(S, R)$ \emphdef{präsentierte Gruppe} ist $\Gen{S & R} := \GenMonoid{A & K} = A^* / K$.

\begin{nt}
    In jeder Gruppe lässt sich $a = b$ umformen als $ab^{-1} = 1$.
\end{nt}

\begin{ex}
    \begin{itemize}
        \item
            $\Gen{s & -} := \GenMonoid{s^+ s^- & s^+s^- = 1, s^-s^+ = 1} \isomorphic (\Z, +)$,
        \item
            $\Gen{s & s^n} = \Gen{s & s^n = 1} = \GenMonoid{s^+, s^- & (s^+)^n, s^+s^- = 1, s^-s^+ = 1} \isomorphic (\Z / n, +)$,
        \item
            $\Gen{a,b & ab = ba} = \Gen{a,b & aba^{-1}b^{-1}} \isomorphic (\Z^2, +)$
        \item
            $\Gen{a,b & -} = \Set{e, a, a^{-1}, b, b^{-1}, a^2, a^{-2}, ab, ab^{-1}, a^{-1}b^{-1}, b^2, b^{-2}, ba, ba^{-1}, b^{-1}a, b^{-1}a^{-1}}$

            Skizze: Baum in der Ebene, $a$ nach rechts, $b$ nach oben.

            Im Kontrast dazu $\Z^2 = \Gen{a,b & ab = ba}$: Cayley-Graph.
    \end{itemize}
\end{ex}


\Timestamp{2015-10-23}


\section{Simplizialkomplexe}


Kombinatorische Kodierung eines Simplizialkomplexes:
\begin{math}
    \Set{\emptyset, \Set{a}, \dotsc, \Set{f}, \Set{a,b}, \dotsc, \Set{g,f}, \Set{c,e,f}, \dotsc, \Set{c,g,f}, \Set{c,e,f,g}}.
\end{math}

\begin{df}
    Ein (abstrakter) \emphdef{Simplizialkomplex} $K$ ist ein System endlicher Mengen mit
    \begin{enumerate}[i)]
        \item
            $\emptyset \in K$,
        \item
            $T \subset S \in K \implies T \in K$,
    \end{enumerate}
    Wir setzen
    \begin{math}
        \dim S &:= \card(S) - 1,\\
        \dim K &:= \sup\Set{\dim S & S \in K}, \\
        \Omega(K) &:= \bigcup K = \bigcup_{S \in K} S.
    \end{math}
    $a \in \Omega(K)$ heißt \emphdef{Ecke}, $S \in K$ heißt \emphdef{Simplex} von $K$.
    \begin{math}
        K_{\le n} := \Set{S \in K & \dim S \le n}
    \end{math}
\end{df}

\begin{df}
    Eine \emphdef{Darstellung} $f: K \to V$ in einen $\R$-Vektorraum ist eine Abbildung $f: \Omega(K) \to V$, sodass
    \begin{enumerate}[i)]
        \item
            Für $S \in K$ ist $f(S)$ affin unabhängig.
        \item
            Für $S, T \in K$ gilt $[f(S)] \cap [f(T)] = [f(S\cap T)]$.
    \end{enumerate}
    Die \emphdef{kanonische Darstellung} von $K$ ist $f: K \to \R^{(\Omega)}$, $s \mapsto e_s$.
    \begin{note}
        Hierbei ist
        \begin{math}
            \R^{(\Omega)} = \Set{x: \Omega \to \R & \text{$\supp x$ endlich}}.
        \end{math}
        Dieser hat als kanonische Basis $(e_s)_{s\in\Omega}$ mit $e_s: \Omega \to \R$, $e_s(s') = \delta_{s,s'}$.

        Wir identifizieren $s$ mit $e_s$.
        Dann schreibt sich jedes Element $x \in \R^{(\Omega)}$ als formale Linearkombination
        \begin{math}
            x = \sum_{s \in \Omega} x(s) e_s
            = \sum_{s \in \Omega} x(s) s.
        \end{math}
        Man nennt $\R^{(\Omega)}$ den Vektorraum „frei über $\Omega$“.
    \end{note}
\end{df}

\begin{df}
    Sei $f: K \to V$ eine Darstellung.
    $[f(S)]$ ist ein affiner Simplex in $V$ mit $\dim [f(S)] = \dim S$.
    $\Set{[f(S)] & S \in K}$ ist ein \emphdef{affiner Simplizialkomplex} in $V$, d.h.
    ein System affiner Simplizies, sodass sich je zwei höchstens in einer gemeinsamen Seite schneiden.

    Das Polyeder
    \begin{math}
        |K|_f| := \bigcup_{S \in K} [f(S)] \subset V
    \end{math}
    versehen wir mit der \emphdef{simplizialen Topologie}.
    Eine Teilmenge $U \subset |K|$ ist offen genau dann, wenn $U \cap [f(S)]$ offen ist in $[f(S)]$ für alle $S \in K$.
    \begin{note}
        Für $K$ endlich genügt $f: K \to \R^n$ und die Teilraumtopologie von $|K|_f \subset \R^n$ ist die simpliziale Topologie.

        Für $\Omega$ unendlich ist die simpliziale Topologie wesentlich.
        Betrachte (Skizze: diskrete Variante der Sinuskurve des Topologen)
        \begin{math}
            \Omega &:= \Set{a,b} \cup \N,
            K &:= \Set{\emptyset} \cup \binom{\Omega}{1} \Set{\Set{k, k+1} & k \in \N}
        \end{math}
        mit Darstellung $f: K \to \R^2$, $a \mapsto (0,1)$, $b \mapsto (0,-1)$,
        \begin{math}
            f(k) = \frac{\frac{1}{k}}{(-1)^k}.
        \end{math}
        Wir erhalten $|K|_f \subset \R^2$.
        $[f(a), f(b)]$ ist offen in der simplizialen Topologie, aber nicht offen in der Teilraumtopologie.
    \end{note}
\end{df}


\section{Simpliziale Fundamentalgruppen}

\begin{df}
    Sei $K$ ein Simplizialkomplex mit $\Omega = \Omega(K)$.
    \begin{itemize}
        \item
            Ein Kantenzug $v_0v_1 \dotsc v_n$ ist eine endlich Folge von Eckpunkten mit $\Set{v_0, v_1}, \dotsc, \Set{v_{n-1}, v_n} \in K$.
        \item
            Zwei Kantenzüge $w = v_0 \dotsc v_n$ und $w' = v_0' \dotsc v_m'$ heißen \emphdef{verknüpfbar}, wenn $v_n = v_0'$.
            In diesem Fall ist $w \ast w' := v_0 \dotsc v_n v_1' \dotsc v_m'$ die \emphdef{Verknüpfung} beider Kantenzüge.
        \item
            Zwei Kantenzüge $w = v_0 \dotsc v_{k-1} v_k v_{k+1} \dotsc v_n$ und $w' = v_0 \dotsc v_{k-1} v_{k+1} \dotsc v_n$ heißen äquivelent, geschrieben $w \approx w'$, falls $\Set{v_{k-1}, v_k, v_{k+1}} \in K$.
        \item
            Zu $w = v_0 v_1 \dotsc v_n$ setze $\_w := v_n \dotsc v_1 v_0$.
            Es gilt $w \ast \_w = v_0 v_1 \dotsc v_{n-1} v_n v_{n-1} \dotsc v_1 v_0 \approx v_0$.
        \item
            Simpliziales Fundamentalgruppoid:
            \begin{math}
                \Pi(K) = \frac{\Set{\text{Kantenzüge in $K$}}}{\approx}.
            \end{math}
        \item
            Simpliziale Fundamentalgruppe:
            \begin{math}
                \pi_1(K, x_0) := \Pi(K, x_0, x_0) = \frac{\Set{\text{Kantenzüge in $K$}}}{\approx}
            \end{math}
            Wir erhalten einen Gruppenisomorphismus
            \begin{math}
                \phi: \pi_1(K, x_0) &\xto[isomorphic] \pi_1(|K|, x_0)
                [w] &\mapsto [|w|].
            \end{math}
            \begin{proof}[Beweisidee]
                Prüfe: Wohldefiniertheit (Verträglichkeit der Äquivalenzen),
                Surjektivität: simpliziale Approximation von $\gamma:[0,1] \to (K, x_0)$,
                Injektivität: simpliziale Approximation von $H: [0,1]^2 \to (K, x_0)$.
            \end{proof}
    \end{itemize}
\end{df}

\begin{ex}
    \begin{itemize}
        \item
            Ein (simplizialer) Graph ist ein Simplizialkomplex $K$ mit $\dim K \le 1$.

            Für Kantenzüge gibt es dann nur die Äquivalenzen (Kürzungen/Erweiterungen) der Art
            \begin{math}
                uu &\approx u, &
                uvu &\approx u.
            \end{math}
            Sind keine solchen Kürzungen möglich, so nennen wir den Kantenzug \emphdef{gekürzt} (eigentlich \emph{lokal} gekürzt, greedy).
        \item
            Ein \emphdef{Baum} ist ein Graph, der zusammenhängend und zykelfrei ist, d.h.
            \begin{enumerate}[i)]
                \item
                    Zu je zwei Ecken $a,b \in \Omega(K)$ existiert ein Kantenzug von $a$ nach $b$.
                \item
                    Für jede Kante $\Set{a,b} \in K$, $a \neq b$ sind die Ecken $a,b$ in $K \setminus \Set{\Set{a,b}}$ nicht mehr verbindbar.
            \end{enumerate}
        \item
            Für jeden nicht-leeren endlichen Graphen $K$ sind äquivalent:
            \begin{enumerate}[i)]
                \item
                    $K$ ist ein Baum,
                \item
                    $|K|$ ist zusammenziehbar,
                \item
                    $K$ ist zusammenhängend und $\chi(K) = 1$,
                \item
                    $K$ ist zykelfrei und $\chi(K) = 1$.
                \item
                    Zu je zwei Ecken $a, b \in \Omega(K)$ existiert genau ein gekürzter Kantenzug von $a$ nach $b$.
            \end{enumerate}
            Für unendliche Graphen gilt die Äquivalenz noch zwischen i), ii) und v).
        \item
            Sei $K$ ein zusammenhängender Graph und $T \subset K$ ein Teilgraph, der alle Ecken von $K$ enthält, kurz: $\Omega(T) = \Omega(K)$.
            Dann sind äquivalent:
            \begin{enumerate}[i)]
                \item
                    $T$ ist ein Baum,
                \item
                    $T$ ist zykelfrei und maximal,
                \item
                    $T$ ist zusammenhängend und minimal,
            \end{enumerate}
            In diesem Fall nennen wir $T$ \emphdef{Spannbaum}.
    \end{itemize}
\end{ex}

\begin{st}
    Für jeden Baum $T$ gilt
    \begin{math}
        \pi_1(T, x_0) = \pi_0(|T|, x_0) = \Set{1}.
    \end{math}
    \begin{proof}
        $\pi_1$ durch Kürzung.
        $\pi_0$ durch Zusammenziehen.
    \end{proof}
\end{st}

\begin{st}
    Sei $K$ ein zusammenhängender Graph, $x_0 \in \Omega(K)$, $T \subset K$ ein Spannbaum.
    Dann ist $\pi_1(K, x_0)$ frei über $|K \setminus T|$ Erzeugern.

    Genauer: $\psi: \pi_1(K, x_0) \to F = \GenMonoid{S & R} = \Gen{S & R}$ mit Erzeugern $S = \Set{s_{ab} & \Set{a,b} \in K \setminus T}$ und Relationen $R = \Set{s_{ab} s_{ba} & \Set{a,b} \in K \setminus T}$.

    Ist $K$ zudem endlich, so hat $\pi_1(K, x_0)$ den Rang $|K \setminus T| = 1 - \chi(K)$.
    \begin{proof}
        Siehe Verallgemeinerung unten
    \end{proof}
\end{st}

\begin{st}
    Sei $K$ ein zusammenhängender Simplizialkomplex, $x_0 \in \Omega(K)$, $T \subset K$ ein Spannbaum (im 1-Skelett).
    Dann gilt $\pi_1(K, x_0) \isomorphic \Gen{S & R} = G$ mit
    \begin{math}
        S &= \Set{s_{ab} & \Set{a,b} \in K}, \\
        R &= \Set{s_{ab} & \Set{a,b} \in T} \cup \Set{s_{ab} s_{ba} & \Set{a,b} \in K}
        \cup \Set{s_{ab} s_{bc} s_{ca} & \Set{a,b,c} \in K }
    \end{math}
    Genauer: existieren zueinander inverse Gruppenisomorphismen
    \begin{math}
        \psi&:& \pi_1(K, x_0) &\to G, &
        [v_0 \dotsc v_n] &\mapsto s_{v_0v_1} \dotsb s_{v_{n-1} v_n}, \\
        \phi&:& G &\to \pi_1(K,x_0), &
        s_{ab} &\mapsto [x_0 \dotsc a \ast ab \ast b \dotsc x_0].
    \end{math}
    \begin{proof}
        Durch Nachrechnen: $\psi$ wohldefiniert, $\phi$ wohldefiniert.
        Es gilt $\psi \circ \phi = \id_G$, denn $\psi(\phi(s_{ab})) = s_{ab}$.
        Ebenso $\phi \circ \psi = \id_{\pi_1}$ (nach Kürzen der antisymmetrischen Wege in $T$).
    \end{proof}
\end{st}

\begin{ex}
    \begin{itemize}
        \item
            Skizze: Triangulierter Torus $T$ mit 9 Ecken, Spannbaum $U$.
            Plausibilität: $\chi(T) = 9 - 27 + 18 = 0$.
            Nicht-triviale Elemente $s_{xy} \in T \setminus U$ (wende Relationen an) bilden Erzeuger: $s := s_{ac}$, $t := s_{ea}$.
            Wir erhalten
            \begin{math}
                \pi_1(T, a) = \Gen{S & R}
                \xto* \Gen{s, t & st = ts} \isomorphic \Z^2.
            \end{math}
            Surjektiv nach Bild: Alle Erzeuger werden getroffen.
            Injektiv nach Bild: Alle Relationen wurden verwendet.

            Wie erwartet
            \begin{math}
                \pi_1(|T|, x_0)
                \isomorphic \pi_1(\S^1 \times \S^1, x_0)
                \isomorphic \pi_1(\S^1, x_0) \times \pi_1(\S^1, x_0)
                \isomorphic \Z \times \Z
                \isomorphic \Z^2.
            \end{math}
    \end{itemize}
\end{ex}


\Timestamp{2015-10-30}

\section{Der Satz von Seifert-van-Kampen}

Sei $X = \bigcup_{i \in I} U_i$ eine offene Überdeckung.
Ziel: Wie berechnet man $\pi_1(X, x_0)$ aus den Teilen $(U_i)_{i \in I}$?

Einfaches Beispiel (Skizze: drei Mengen mit paarweisen Schnitten)
Fordere
\begin{itemize}
    \item
        $U_i$ einfach zusammenhängend
    \item
        $U_i \cap U_j$ einfach zusammenhängend
    \item
        $U_i \cap U_j \cap U_k$ einfach zusammenhängend
\end{itemize}
Man denke an $U_i \subset \R^n$ konvex, dann sind alle weiteren Schnitte konvex (ebenso $U_i \subset M$ in einer Riemannschen Mannigfaltigkeit).

Wir bilden folgenden Simplizialkomplex, genannt der \emphdef{Nerv} von $\scr U = (U_i)_{i \in I}$.
Für $S = \Set{s_0, \dotsc, s_n} \subset I$ setze $U_S := U_{s_0} \cap \dotsb \cap U_{s_n}$, sowie $U_{\emptyset} := X$.
Der \emphdef{Nerv} ist
\begin{math}
    N(\scr U) = \Set{\text{$S \subset I$ endlich} & U_S \neq \emptyset}.
\end{math}

Im Beispiel $I = \Set{1, 2, 3}$, $\scr U$ wie skizziert,
\begin{math}
    N(\scr U) = \Set{\emptyset, \Set 1, \Set 2, \Set 3, \Set{1, 2}, \Set{1, 3}, \Set{2, 3}}
\end{math}

\begin{prop}
    $N(\scr U)$ ist ein (abstrakter) Simplizialkomplex.
\end{prop}

\begin{st}
    Sei $X$ ein topologischer Raum, $\scr U = (U_i)_{i \in I}$ eine offene Überdeckung sodass $U_i$ einfach zusammenhängend und $U_i \cap U_j$ wegzusammenhängend ist.
    Wähle $i_0 \in I$ und $x_0 \in U_{i_0}$.

    Dann existiert ein Gruppenisomorphismus
    \begin{math}
        \Phi: \pi_1(N(\scr U), i_0) &\xto \pi_1(X, x_0), \\
        [(i_0, i_1, \dotsc, i_n)] &\mapsto [\gamma_{i_0, i_1} \ast \dotsb \ast \gamma_{i_{n-1}, i_n}].
    \end{math}
    Genauer:
    Hierzu wählen wir $x_i \in U_i$ für $i \in I$ sowie $x_{ij} \in U_{ij} := U_i \cap U_j$ für $i \neq j$ mit $U_{ij} \neq \emptyset$, $x_{ij} = x_{ji}$.
    Sei $\gamma_{ij}$ ein Weg von $x_i$ nach $x_{ij}$ in $U_i$ und dann von $x_{ij}$ nach $x_j$ in $U_j$.
    Damit ist $[\gamma_{ij}]$ eindeutig festgelegt (da $U_i$, $U_j$ einfach zusammenhängend, $U_{ij}$ wegzusammenhängend).
    % \gamma_{ij} \sim \gamma_{ij} \iff \gamma_{ij}\_{\gamma_{ij}} \sim *
    \begin{proof}[Skizze]
        \begin{enumerate}[1)]
            \item
                $\Phi$ surjektiv (Skizze: Weg $\omega$ über endlich viele $U_i$ von $x_{i_0}$ nach $x_{i_n} = x_{i_0}$, homotop zu zweitem Weg):
                Sei $\omega:[0,1] \to X$ ein Weg von $x_0$ nach $x_0$.
                Wir haben eine offene Überdeckung $[0,1] = \bigcup_{i \in I} \omega^{-1}(U_i)$.
                Es existiert eine Lebesgue-Zahl $\frac{1}{n}$, $n \in \nu$, sodass $\omega([\frac{k-1}{n}, \frac{k}{n}]) \subset U_{i_k}$ für $k = 1, \dotsc, n$.
                Für $i_k \neq i_{k+1}$ liegt $\omega(\frac{k}{n}$ in $U_{i_k} \cap U_{i_{k+1}}$.
                Wähle $\beta_k$ von $\omega(\frac{k}{n})$ nach $x_{i_k}{i_{k+1}}$ in $U_{i_k} \cap U_{i_{k+1}}$.
                Dann gilt $\gamma \ast \_\omega \sim 1_{x_0}$.
                %Nutze Kompaktheit des Weges (endliche Überdeckung, Intervallteilung mit Lebesgue-Zahl), konstruiere so die Folge von Mengen $U_i$.
            \item
                $\Phi$ injektiv: später
        \end{enumerate}
    \end{proof}
\end{st}

Im Allgemeinen sind unsere Überdeckungen jedoch nicht so schön.

Sei $X$ ein topologischer Raum und $\scr U = (U_s)_{s \in \Omega}$ eine offene Überdeckung ($U_s$ muss nicht wegzusammenhängend sein, ebensowenig $U_s \cap U_t$).

Der \emphdef{Wegnerv} von $\scr U$ ist definiert durch
\begin{math}
    N^\circ(\scr U) = \Set{(S, C) & \text{$S \subset \Omega$ endlich, $C \in \pi_0(U_s)$} }
\end{math}
($\pi_0(U_s)$ Menge der Wegzusammenhangskomponenten).
Setze $\dim(S, C) = |S| - 1$.
Definiere $(S, C) \to (T, D)$ durch $T \subset S$ und $D \supset C$.

Skizze: $U_1, U_2$, $U_1$ mit zwei Komponenten, mit jeweils $3$ Schnitten $D_1, D_2, D_3$ und einem Schnitt $D_4$ mit $U_2$, $U_2$ einfach zusammenhängend.
\begin{math}
    \begin{tikzcd}
        & (\Set{1,2}, D_1) \ar[ld] \ar[rd]& \\
        (\Set 1, C) & (\Set{1,2}, D_2) \ar[l] \ar[r] & (\Set{2}, U_2) \\
        & (\Set{1,2}, D_3) \ar[lu] \ar[ru] & \\
        (\Set{1}, C') & (\Set{1,2}, D_4) \ar[l] \ar[ruu]
    \end{tikzcd}
\end{math}
Damit ist $I = N^\circ(\scr U)$ mit $\to$ ein Poset, d.h. reflexiv ($i \to i$) und transitiv ($i \to j \to k \implies i \to k$).
Jedem Index $i = (S, C)$ ordnen wir den Teilraum $X_i = C$ zu.
Wir wählen $x_i \in X_i$.
Für $i \to j$ gilt $X_i \subset X_j$.
Da $X_j$ wegzusammenhängend ist, wähle einen Weg $\gamma_{ij}: [0,1] \to X_j$ von $\gamma_{ij}(0) = x_i$ nach $\gamma_{ij}(1) = x_j$.
Für $i = j$ setze $\gamma_{ii} = 1_{x_i}$.
Für $i \to j$, $j \to i$ gilt $X_i = X_j$ und evtl. $x_i \neq x_j$, wir wollen dann $\gamma_{ji} = \_{\gamma_{ij}}$.

Aus $\scr U = (U_s)_{s \in \Omega}$ erhalten wir $(I, \to, (X_i)_{i \in I}, (x_i)_{i \in I}, (\gamma_{ij})_{i \to j})$.

Zu $i \in I$ setzen wir $G_i := \pi_1(X_i, x_i)$.
Für $i \to j$ induziert $\iota_{ij}: X_i \injto X_j$ einen Gruppenhomomorphismus $h_{ij} : G_i \to G_j$ durch
\begin{math}
    h_{ij}([\alpha]) := \_{\gamma_{ij}} \ast (\iota_{ij} \circ \alpha) \ast \gamma_{ij}
\end{math}
(Skizze: !!)
Für $i \to j \to k$ (Skizze: Venn-Diagramm mit Drei Mengen, $i \to j \to k$ und $i \to k$)
setze $g_{ijk} := [\_{\gamma_{jk}} \ast \_{\gamma_{ij}} \ast \gamma_{ik}] \in G_k$.

Es gilt
\begin{math}
    h_{jk} \circ h_{ij} = g_{ijk} h_{ik} g_{ijk}^{-1}.
\end{math}
D.h. $g_{ijk}$ misst die Abweichung von $h_{jk} \circ h_{ij}$ zu $h_{ik}$.

Wir erhalten hieraus den Gruppenkomplex
\begin{math}
    \Gamma &= (I, \to, G_\argdot, h_{\argdot, \argdot}, g_{\argdot, \argdot, \argdot}) \\
    &= (I, \to, (G_i)_{i\in I}, (h_{ij})_{i \to j}, (g_{ijk})_{i \to j \to k}).
\end{math}
Hierin betrachten wir \emphdef{Kantenzüge}
\begin{math}
    w = (i_0 \xto[lr]{g_1} i_1 \xto[lr] \dotsb \xto[lr] i_n)
\end{math}
Hierbei seien $i_0, i_1, \dotsc, i_n \in I$ und $i \xto[lr]{g} j$ steht entweder für $i \to j$ oder $i \xto* j$ in Graphen $(I, \to)$ oder aber $i \xto{g} i$ oder $i \xto*{g} i$ mit $g \in G_i$.
Die Verknüpfung $w \ast w'$ ist die Aneinanderhängung.
Wir nutzen folgende Relationen
\begin{math}
    (i \to i) \approx (i \xto{1} i) &\approx (i), \\
    (i \to j \xto* i) &\approx (i), \\
    (j \xto* i \xto{g} i \to j) &\approx (j \xto{h_{ij}(g)} j), \\
    (i \xto{g} i) &\approx (i \xto*{g^{-1}} i), \\
    (i \xto{g} i \xto{h} i) &\approx (i \xto{gh} i), \\
    (i \to j \to k) &\approx (i \to k \xto{g_{ijk}^{-1}} k).
\end{math}
Die Kantengruppe des Gruppenkomplexes $\Gamma$ ist
\begin{math}
    \pi_1(\Gamma, i_0) = \frac{\text{geschl. Kantenzüge in $\Gamma$ von $i_0$ nach $i_0$}}{\text{Äquivalenz $\approx$}}.
\end{math}

\begin{st}[Seifert-van-Kampen]
    Jede offene Überdeckung $X = \bigcup{s \in \Omega} U_s$ definiert einen Gruppenkomplex $\Gamma$ (nach Wahl von Fußpunkten und Verbindungswegen).
    Seine Kantengruppe $\pi_1(\Gamma, i_0)$ ist isomorph zu $\pi_1(X, x_0)$.

    Genauer: Für das $1$-Skelett $\Gamma_{\le 1}$ liefert die topologische Realisierung eine Surjektion
    \begin{math}
        \Phi_1: \pi_1(\Gamma_{\le 1}, i_0) \xto[surjective] \pi_1(X, x_0).
    \end{math}
    Für das $2$-Skelett $\Gamma_{\le 2}$ erhalten wir einen Isomorphismus $\Phi_2 : \pi_1(\Gamma_{\le 2}, i_0) \xto[isomorphic] \pi_1(X, x_0)$.
\Timestamp{2015-11-06}
    \begin{proof}[Skizze]
        Betrachte $\Phi_1: \pi_1(\Gamma_{\le 1}, i_0) \xto[surjective] \pi_1(X, x_0)$, zeige Surjektivität.
        Sei dazu $[\omega] \in \pi_1(X, x_0)$, d.h. $\omega:[0,1] \to X$ eine Schleife in $x_0$.
        Zeige: $\omega$ homotop zu einem Weg $\Phi_1(w)$ mit $w \in \pi_1(\Gamma_{\le 1}, i_0)$.
        \begin{math}
            w = (i_0 \xto{g_0} i_0 \xto* i_{0,1} \to i_1 \xto{g_1} i_1 \xto* i_{12} \to i_2 \xto{g_2} i_2 \dotsb \xto* j_{n-1} \to i_n \xto{g_n} i_n).
        \end{math}
        Die topologische Realisierung von $w$ ist homotop zu $\omega$ nach Konstruktion.

        Betrachte nun $\Phi_2: \pi_2(\Gamma_{\le 2}, i_0) \to \pi_1(X, x_0))$, zeige Bijektivität.
        Wohldefiniert: Man vergewissere sich, dass alle definierten Äquivalenzen entsprechende Homotopien erlauben.
        Surjektivität wie zuvor.
        Zeige nun Injektivität: Werden zwei Wörter $w_1, w_2$ durch homotope Wege $w_1, w_2$ realisiert, dann sind $w_1$ und $w_2$ äquivalent.
        Sei $H: [0,1]^2 \to X$ eine Homotopie von $w_1$ nach $w_2$.
        Idee: Modifziere $H$ derart, dass sich eine Triangulierung ergibt und jedes Dreieck einer Relation entspricht.
        Wir nutzen $X = \bigcup_{s \in \Omega} U_s$ für $[0,1]^2 = \bigcup_{s \in \Omega} H^{-1}(U_S)$.
        Dank Kompaktheit existiert eine Lebesgue-Zahl dieser Überdeckung.
        Wir unterteilen $[0,1]^2$ wie folgt (Skizze: Ziegel-Mauerwerk):
        Nach hinreichend feiner Unterteilung gilt für jeden Ziegel $\Z_\alpha$ die Bedingung $H(Z_\alpha) \subset U_{s(\alpha)}$ für eine geeignete Abbildung $s$.
        Wir dicken die Fugen auf (z.B. erst horizontale Fugen, dann vertikale durch konstante Homotopien, es entstehen kleine Quadrate, in denen die Homotopie konstant ist).
        Lokal in einem kleinen Quadrat ist $H$ konstant $c$ und es gilt $H(Q) \subset U_1 \cap U_2 \cap U_3$.
        Wähle $i \in I$ sodass $H(Q) \subset X_i$, füge $x_i$ im Quadrat ein mit „radialer Homotopie“.
        Zusätzlich $x_{01}, x_{12}, x_{12}$ an den Kantenmittelpunkten.
        Hilfspunkte für $g_{012}$.

        Von Fugen-Quadraten zu den Fugen-Rechtecken.
        Damit können wir jede Fuge auffüllen durch
        Schließlich Ziegel.

        (Skizzen sind hilfreich)

        Zusammenfassung:
        Wir beginnen mit der gegebenen Homotopie $H$.
        \begin{enumerate}[1.]
            \item
                Aufdicken von horizontalen und vertikalen Fugen.
            \item
                Korrektur um Eckpunkten.
            \item
                Korrektur auf Fugenstücken.
            \item
                Korrektur auf Ziegeln.
        \end{enumerate}
        Ablesen dieser einfachen Teile liefert die kombinatorische Äquivalenz von $w_1$ nach $w_2$.
    \end{proof}
\end{st}

\begin{ex}
    \begin{itemize}
        \item
            Kreisring aus zwei Mengen gebildet.
            $\pi_1(\Gamma, x_1) \isomorphic \Z$.
            Wir benötigen nur $U_1, U_2$ einfach zusammenhängend, $U_1 \cap U_2$ hat zwei Wegkomponenten.

            Betrachte $U_1, U_2$ mit $U_1 \cap U_2 = C_1 \dunion C_2$ und $C_1 \isomorphic C_2 \isomorphic \S^1 \times \B^2$.
        \item
            $U_1$ Kreisring, $U_2$ Kreisscheibe, $U_1 \cap U_2$ Kreisring.
        \item
            Venn-Diagramm, ohne Mittelteil, $\pi_1(\Gamma, x_0) \isomorphic \Z$.
        \item
            Wie voriges, mit mittlerer Kreisscheibe, $\pi_1(\Gamma, x_0) \isomorphic \Set e$.
        \item
            $\pi_0(U_1) = \pi_0(U_2) = \pi_0(U_1 \cap U_2) = \Set *$.
            $\pi_1(U_1) = G_1$, $\pi_1(U_2) = G_2$ beliebig, $\pi_1(U_1 \cap U_2) = \Set e$.
            \begin{math}
                \pi(\Gamma, x_0) = G_1 \ast G_2
            \end{math}
            (freies Produkt).
        \item
            $\pi_1(U_1) = G_1$, $\pi_1(U_2) = G_2$ beliebig, $U_1 \cap U_2$ wegzusammenhängend, $\pi_1(U_1 \cap U_2) = K$.
            Sei $U_{12} := U_1 \cap U_2$,
            \begin{math}
                i: U_{12} &\injto U_1, &i_\#: \pi_1(U_{12}) &\injto \pi_1(U_1), \\
                j: U_{12} &\injto U_2, &j_\#: \pi_1(U_{12}) &\injto \pi_1(U_2). \\
            \end{math}
            Das führt zum amalgamierten Produkt $G_1 \ast_K G_2$.

            Ausführlich und etwas allgemeiner:
            Sei $X = U_1 \cup U_2$, $U_1, U_2$ offen und wegzusammenhängend,
            $U_{12} = U_1 \cap U_2$ (offen und) wegzusammenhängend.
            Wir wählen $x_0 \in U_{12} \subset U_1, U_2$.
            Sei $\pi_1(U_i, x_0) = G_i = \Gen{S_i & R_i}$ für $i \in \Set{1, 2, (1,2)}$.
            (Skizze: $\Gamma$)
            Dann gilt
            \begin{math}
                \pi_1(X, x_0) = \pi_1(\Gamma, x_0)
                = \Gen{S_1, S_2 & R_1 \dunion R_2 \dunion T}
            \end{math}
            mit
            \begin{math}
                T = \Set{h_1(s) h_2(s)^{-1} & s \in S_{12}}
            \end{math}
        \item
            $g_{ijk}$ nichttrivial:
            $U_1 \isomorphic U_2 \isomorphic U_3 \isomorphic \S^1 \times (0,1)$ mit geeigneter Wahl der Fußpunkte.
    \end{itemize}
    \begin{note}
        Gruppenkomplexe dienen zur Analyse von Gruppen, siehe Serre, Trees.
    \end{note}
\end{ex}





% Kapitel B
\chapter{Knotengruppen}

\Timestamp{2015-05-06}

Ziel: Zu Knoten $K \subset \R^3$ wollen wir $\pi_1(\R^3 \setminus K, *)$ berechnen und nutzen.


% §B1
\section{Erinnerung: Präsentation von Gruppen}

\begin{ex}
    \begin{itemize}
        \item
            Zyklische Gruppe:
            \begin{math}
                G = \Set{g, g^2, g^3, \dotsc, g^n = 1},
            \end{math}
            wobei $g^i \neq g^j$ für $0 \le i < j \le n$.
            Wir nutzen dafür die Schreibweise
            \begin{math}
                G = \<g | g^n = 1\>
                = \<g | g^n\>.
            \end{math}
            Dann erhalten wir den Gruppenisomorphismus $\Z / n \to G$, $k + n\Z \mapsto g^k$.
        \item
            Unendliche zyklische Gruppe:
            \begin{math}
                G = \Set{g^k & k \in \Z},
            \end{math}
            mit $g^i \neq g^j$ für $i \neq j$ in $\Z$.
            Wir nutzen die Schreibweise
            \begin{math}
                G = \< g | - \>.
            \end{math}
            Dann haben wir den Gruppenisomorphismus $\Z \to G, k \mapsto g^k$.
        \item
            Wir wollen folgende Notationen nutzen können:
            \begin{math}
                G = \Gen{a,b & ab = ba}
            \end{math}
            Wir haben einen Gruppenisomorphismus $Z^2 \to G$, $(k,l) \mapsto a^kb^l$.
    \end{itemize}
\end{ex}

\subsection{Freie Gruppen}

\begin{df}
    Sei $∈(G, \cdot)$ eine Gruppe, $S \subset G$.
    Die von $S$ erzeugte Untergruppe ist
    \begin{math}
        \<S\> = \Set{s_1^{e_1} \dotsc s_n^{e_n} & n \in \N, s_i \in S, e_i \in \Z}.
    \end{math}
    Wir nennen $s_1^{e_1} \dotsc s_n^{e_n}$, genauer $(s_1,e_1; \dotsc; s_n, e_n) \in (S\times \Z)^n$ ein \emphdef{Wort} über $S$.
    Ein Wort heißt \emphdef{reduziert}, wenn $s_i \neq s_{i+1}$ und $e_i \neq 0$.
\end{df}

\begin{df}
    Eine Gruppe $G$ heißt \emphdef{frei}, über $S \subset G$, wenn sich jedes $g \in G$ eindeutig schreiben lässt als reduziertes Wort über $S$.
\end{df}

\begin{ex}
    \begin{itemize}
        \item
            $G \isomorphic \Z/5$ ist nicht frei über $S = \Set{g}$, weil $1 = g^0 = g^5 = g^{10} = \dotsc$.
        \item
            $G \isomorphic \Z$ ist frei über $S = \Set{g}$.
        \item
            $G \isomorphic \Z^2$ ist nicht frei über $S = \Set{a,b}$, denn $ab = ba$, also
            \begin{math}
                (a,1;b,1) \neq (b,1;a,1).
            \end{math}
    \end{itemize}
\end{ex}

\begin{st}
    Zu jeder Menge $S$ existiert eine freie Gruppe $F(S) = \<S| - \>$ über $S$.
    \begin{proof}
        Übung.
    \end{proof}
\end{st}

\begin{st}[universelle Abbildungseigenschaft]
    Eine Gruppe $F$ ist genau dann frei über $S \subset F$, wenn gilt:
    zu jeder Abbildung $f: S \to G$ in eine Gruppe $G$ existiert genau ein Gruppenhomomorphismus $h: F \to G$ mit $h|_S = f$.
    %\begin{note}
    %    \Hom(F,G) \stack\isomorphic\to \App(S,G),
    %    h \mapsto h|_S.
    %\end{note}
    \begin{proof}
        Sei $F$ frei über $S$, dann besteht $F$ aus reduzierten Wörtern der Form $s_1^{e_1} \dotsc s_n^{e_n}$.
        Setze $h: F \to G$ durch $h(s_1^{e_1} \dotsc, s_n^{e_n}) = f(s_1)^{e_1} \dotsc f(s_n)^{e_n}$, dies ist die einzige Möglichkeit, eine solche Abbildung zu definieren.
        Sie ist wohldefiniert und multiplikativ.

        Die Umkehrung ist abstract general nonsense:
        Angenommen $f$ besitzt die universelle Abbildungseigenschaft.
        Dann existiert genau ein Gruppenhomomorphismus $h: F \to F(S)$ mit $h|_S = \id_S$ und genau ein $k: F(s) \to F$ mit $k|_S = \id_S$.
        Für diese gilt $k \circ h = \id_F$ und $h \circ k = \id_{F(S)}$.
    \end{proof}
\end{st}

\begin{df}
    Sei $S$ eine Menge, $F = F(S)$ eine freie Gruppe über $S$.
    Sei $R \subset F$ eine Menge von reduzierten Gruppenwörtern über $S$.
    Wir nennen $(S, R)$ eine \emphdef{Präsentation} mit Erzeugern $S$ und Relationen $R$.
    Die hierdurch \emphdef{prästentierte Gruppe} ist
    \begin{math}
        \<S |R\> := F / \<R^F\>.
    \end{math}
    Hierbei ist $\<R^F\>$ die von $R$ normal erzeugte Untergruppe in $F$, d.h.
    \begin{math}
        \<R^F\> = \Gen{ r^f & r \in R, f \in F }
    \end{math}
    wird erzeugt von allen Konjugierten von $r \in R$ in $F$.
    Dies ist die kleinste normale Untergruppe, die $R$ enthält.
\end{df}

\begin{ex}
    \begin{enumerate}[1)]
        \item
            Für $R = \emptyset$ ist $\<S | \emptyset\> = \Gen{S & -} = F(S)$.
        \item
            $\Gen{a & -} \leftarrow \Z, a^k \mapsfrom k$,
        \item
            $\Gen{a & a^n} \leftarrow \Z /n, a^k \mapsfrom k$.
        \item
            $\Gen{a, b & aba^{-1}b^{-1}} \leftarrow \Z^2, a^kb^l \mapsfrom (k,l)$
        \item
            Zopfgruppen, symmetrische Gruppen
    \end{enumerate}
\end{ex}

\begin{st}[Universelle Abbildungseigenschaft]
    Sei $(S,R)$ wie oben, $f : S \to G$.
    Dann sind äquivalent:
    \begin{enumerate}[1)]
        \item
            Der Gruppenhomomorphismus $h: F(S) \to G$ mit $h|_S = f$ erfüllt $h(R) = \Set{1}$ (und faktorisiert somit).
        \item
            Es existiert ein Gruppenhomomorphismus $\_h: \<S|R\> \to G$ mit $\_h \circ q = f$ ($q$ sei hierbar Quotientenhomomorphismus).
    \end{enumerate}
    \begin{proof}
        Leichte Übung.
    \end{proof}
\end{st}

\begin{prop}
    Jede Gruppe $G$ erlaubt eine Präsentation, d.h. ein Tripel $(S,R,h)$ mit $h: \<S | R\> \stack\isomorphic\to G$.
\end{prop}

\begin{ex}
    Sei $G = \Z / R = \Set{0,1,2,3,4}$.
    $h: \< a | a^5\> \to \Z^5$, $a \mapsto 1$ (wohldefiniert, surjektiv, injektiv).
    Aber auch $k: \<a|a^5\> \to \Z^5$, $a \mapsto 2$ ist ein Gruppenisomorphismus.
\end{ex}

Zu einer gegebenen Gruppe $G$ gibt es stets unendlich viele Präsentationen!
Die folgenden Tietze-Operationen ändern die Präsentation, nicht aber die präsentierte Gruppe.
\begin{enumerate}[(T1)]
    \item
        Hinzufügen/Entfernen einer redundanten Relation: $(S,R) \leadsto (S,R')$ mit $R' = R \cup \Set{r}$, $r \in \<R^F\> \setminus R$.
    \item
        Hinzufügen/Entfernen eines redundanten Erzeugers: $(S,R) \leadsto (S',R')$ mit $S' = S \dotcup \Set{s}$, $R' = R \cup \Set{s^{-1} w}$, $w \in \<S\>$.
\end{enumerate}

\begin{st}[Tietze, 1908]
    Zwei (endliche) Präsentationen $(S,R)$ und $(S',R')$ präsentieren genau dann isomorphe Gruppen, wenn sie sich durch (T1), (T2) ineinander überführen lassen.
\end{st}

\Timestamp{2015-05-11}

Ziel: Zu jedem Knoten $K \subset \R^3$ wollen wir die Knotengruppe $\pi_K := \pi_1(\R^3 \setminus K, *)$ „berechnen“, d.h. präsentieren und auswerten.


% B2
\section{Wirtinger-Präsentation}


Stelle $K$ durch ein ebenes Diagramm $D$ dar, $K$ und $D$ seien orientiert.
Erzeuger sind die Bögen $x_1, \dotsc, x_n$ von $D$.
Relationen sind Kreuzungen: $x_ix_j = x_jx_{i+1}$, $x_i^{-1}x_i x_j = x_{i+1}$, $x_jx_i = x_{i+1}x_j$, $x_jx_ix_j^{-1} = x_{i+1}$, oder zusammengefasst
\begin{math}
    x^{-\eps(i)}_{j(i)} x_i x_{j(i)}^{\eps(i)} = x_{i+1},
\end{math}
die Daten $\eps: \Set{1,\dotsc, n} \to \Set{\pm 1}$ und $j: \Set{1, \dotsc, n} \to \Set{1, \dotsc, n}$ liest man leicht am Diagramm ab.

\begin{df}
    Wir setzen
    \begin{math}
        \pi_D := \Gen{x_1, \dotsc, x_n & x_{j(i)}^{-\eps(i)} x_i x_{j(i)}^{\eps(i)} = x_{i+1}, i = 1,\dotsc, n}
    \end{math}
\end{df}

\begin{ex}
    \begin{itemize}
        \item
            $\pi_{\KnotTriv} = \Gen{x_1 & -} \isomorphic \Z$.
        \item
            \begin{math}
                \pi_{\KnotKlee} &= \Gen{a,b,c & ac = cb, cb = ba, ba = ac} \\
                &= \Gen{a,b,c & a^c = b, b^a = c, c^b = a}
            \end{math}
    \end{itemize}
\end{ex}

\begin{note}
    $\pi_D$ ist unendlich, denn wir haben die Abelschmachung $(\pi_D,\cdot) \to (\Z,+), x_i \mapsto 1$.

    Für Verschlingungen, bzw. Schlingel mit $n$ Komponenten entsprechend $\pi_D \to \Z^n$.
\end{note}

\begin{ex}
    $\pi_{\KnotKlee}$ ist nicht abelsch, also $\pi_{\KnotKlee} \not\isomorphic \pi_{\KnotTriv}$.
    \begin{proof}
        Betrachte $h: \pi_D \to S_3$ mit $a \mapsto (12)$, $b \mapsto (23)$, $c \mapsto (13)$.
        Man rechnet:
        \begin{math}
            a^c &= (23) = b, &
            b^a &= (13) = c, &
            c^b &= (12) = a,
        \end{math}
        Da $h$ surjektiv und $S_3$ nicht abelsch, ist auch $\pi_D$ nicht abelsch.
    \end{proof}
\end{ex}

Noch zu zeigen: $\pi_D$ ist eine Invariante des Knotentyps.
Es gibt folgende Möglichkeiten:
\begin{enumerate}[1)]
    \item
        Reidemeister-Züge verändern die Gruppe nicht: die Präsentationen unterscheiden sich um Tietze-Transformationen.
    \item
        Es existiert ein Isomorphismus $\pi_D \isomorphic \pi_1(\R^3 \setminus K, *)$.
\end{enumerate}


% todo: Appendix:

%\subsection*{Erinnerung: Permutationen, Zykelschreibweise, Konjugation}
%
%In $S_n$ nutzen wir folgende Schreibweise:
%Seien $i_1, \dotsc, i_l \in \Set{1, \dotsc, n}$ verschieden.
%Definiere die Zykel: $c := (i_1, \dotsc, i_l)$ durch $i_1 \mapsto i_2 \mapsto \dotsb \mapsto i_l \mapsto i_1$.
%
%\begin{prop}
%    Jedes $\sigma \in S_i$ ist Produkt disjunkter Zykel.
%    Dieses ist eindeutig bis auf Umordnung der Faktoren.
%\end{prop}
%
%\begin{ex}
%    $\sigma = (1352)(476) = (476)(1352)$.
%\end{ex}
%
%Für die Konjugation gilt
%\begin{math}
%    (i_1, \dotsc, i_l)^\sigma &=
%    \sigma^{-1} (i_1, \dotsc, i_l) \sigma \\
%    &= (\sigma(i_1), \dotsb, \sigma(i_l))
%\end{math}
%

\begin{st}[Wirtinger, <1900]
    Es existiert ein Gruppenisomorphismus $\pi_D \to \pi_1(\R^3 \setminus K, *)$.
    \begin{proof}
        \begin{enumerate}[1),start=0]
            \item
                Konstruktion von $h$:
                Wie in der Skizze, ordnen wir jedem Bogen $b_i$ von $D$ einen (polygonalen) Weg $\gamma_i: [0,1] \to \R^3 \setminus K$ zu.
                Dieser definiert ein Gruppenelement $w_i = [\gamma_i] \in \pi_1(\R^3 \setminus K)$.
                An jeder Kreuzung gilt die Wirtinger-Relation:
                \begin{math}
                    x_i x_j = x_j x_{i+1}
                \end{math}
                und analog die anderen.
                Wir haben nun einen Gruppenhomomorphismus $h: \pi_D \to \pi_1(\R^3 \setminus K, *)$ mit $x_i \mapsto w_i = [\gamma_i]$.
            \item
                $h$ ist surjektiv, d.h. $\pi_1(\R^3 \setminus K, *)$ wird erzeugt von $w_1, \dotsc, w_n$:

                Wir nutzen die polygonale Fundamentalgruppe (mittels polygonaler Approximation)
                \begin{math}
                    \pi_1(\R^3 \setminus K, *)
                    = \frac{\Set{\text{Schleifen}}}{\text{Homotopie}}
                    = \frac{\Set{\text{polygonale Schleifen}}}{\text{polygonale Homotopie}}.
                \end{math}
                Ohne Einschränkung betrachten wir also polygonale Schleifen $\gamma$ in $\R^3 \setminus K$.
                Zu zeigen ist $\gamma \homotopic \gamma_{i_1}^{e_1} \dotsb \gamma_{i_l}^{e_l}$.
                Trick: Betrachte den „Schatten“ des Knotens $K \subset \R^3$ unter senkrecht von oben einfallendem Licht.
                Genauer: Zu $K \subset \R^3$ ist der Schatten $\hat K = \Set{(x,y,z) \in \R^3 & \exists z' \ge z: (x,y,z') \in K}$.
                Dies ist die Vereinigung über alle Schatten $\hat A$ der Kanten $A$ von $K$.
                \begin{prop}
                    Es gilt $\R^3 \setminus \hat K \homequiv *$
                    \begin{proof}
                        Übung: explizite Formel, vgl. Sternförmig bei Zentralprojektion.
                    \end{proof}
                \end{prop}
                Ablesen an $w = [\gamma]$ eines Wortes in $w_i = [\gamma_i]$.
                Wir nehmen an, dass $\gamma$ die Wände $\hat A$ transversal im Inneren trifft.
                Jeder Durchgang liefert einen Erzeuger $w_i^{\pm 1}$.
                Damit gilt $\gamma = \gamma_{i_1}^{e_1} \dotsb \gamma_{i_l}^{e_l}$.
            \item
                $h$ ist injektiv, d.h. die Wirtinger-Relationen erzeugen alle Relationen.
                Polygonale Homotopie:
                \begin{enumerate}[1)]
                    \item
                        Keine Wand wird getroffen: Keine Änderung des Wortes.
                    \item
                        Eine Wand wird geschnitten:
                        Zwei Unterfälle: Jeweils keine Änderung des Wortes.
                    \item
                        Schatten einer Kreuzung wird geschnitten.
                        Dank Wirtinger-Relation keine Änderung des Wortes.
                \end{enumerate}
        \end{enumerate}
    \end{proof}
\end{st}





%\subsection{Grundlagen}
%
%\begin{df}
%	\label{df:1.1}	
%	Die komplexen Zahlen bestehen aus
%	\[
%		\C := \{(x,y) : x,y\in \R\}
%	\]
%	und den Verknüpfungen
%	\begin{align*}
%		(x_1,y_1) + (x_2,y_2) &:= (x_1 + x_2, y_1 + y_2) \in \C \\
%		(x_1,y_1) \cdot (x_2,y_2) &:= (x_1x_2 - y_1y_2, x_1y_2 + x_2y_1) \in \C
%	\end{align*}
%\end{df}
%
%\begin{nt}
%	\label{nt:1.1}
%	\begin{enumerate}
%		\item $(\C,+,\cdot)$ ist ein Körper mit $(0,0)$ und Einselement $(1,0)$.
%		\item 
%			$\phi: \R\to \C : x\mapsto (x,0)$ ist eine injektiver Körperhomomorphismus.
%			Insbesondere gilt
%			\begin{align*}
%				\phi(x_1+x_2) &= \phi(x_1) + \phi(x_2)\\
%				\phi(x_1+x_2) &= \phi(x_1) \cdot \phi(x_2)
%			\end{align*}
%			Identifiziere $\R$ mit $\phi(\R) = \{(x,0) : x\in \R\}$.
%			Schreibe dazu: $(x,0) =: x \in \R$.
%		\item
%			Imaginäre Einheit $i:= (0,1)$.
%			\[
%				\implies \begin{cases}
%				i^2 = (0,1)\cdot (0,1) = (0\cdot 0 - 1\cdot 1, 0\cdot 1 + 0\cdot 1) = (-1,0) = -1 \\
%				(x,y) = (x,0) + (0,y) = (x,0) + y\cdot i = x + yi
%				\end{cases}
%			\]
%			Rechnen in $\C$:
%			\begin{align*}
%				(x_1,y_1)\cdot (x_2,y_2) &= (x_1 +iy_1)\cdot (x_2 + iy_2)\\
%				&= x_1x_2 + ix_1y_2 + iy_1x_2 + (i)^2y_1y_2\\
%				&= x_1x_2 - y_1y_2 + i(x_1y_2 + x_2y_1)
%			\end{align*}
%			oder
%			\begin{align*}
%				\f 1{x+iy} = \f 1{x+iy}\cdot \f {x-iy}{x-iy} = \f{x-y}{x^2-(iy)^2} = \f x{x^2+y^2} + i \f{-y}{x^2+y^2}
%			\end{align*}
%			Realteil: $\Re(x+iy) = x\in \R$ \\
%			Imaginärteil: $\Im(x+iy) = y\in\R$
%		\item
%			Gaußsche Zahlenebene:
%	\end{enumerate}
%\end{nt}
%
%\begin{df}
%	\label{df:1.3}
%	\begin{enumerate}
%		\item 
%			Für $z=x+iy$ heißt
%			\[
%				\_z = x-iy
%			\]
%			\emph{konjugiert komplexe Zahl}
%		\item
%			\[
%				|z| = \sqrt{x^2+y^2} = \sqrt{z\cdot \_z}
%			\]
%		\item
%			Polardarstellung: $z=x+iy = |z|(\cos \phi + i\sin\phi)$ wobei $\phi = \arg(z)$ (Argument von $z$) eindeutig durch
%			\[
%			-\pi \le \phi \le \pi, \qquad \cos\phi = \f x{\sqrt{x^2+y^2}}, \qquad \sin\phi = \f y{\sqrt{x^2+y^2}}
%			\]
%			Rechnen mit Polardarstellung:
%			\begin{align*}
%				z_1 \cdot z_2 &= |z_1|\cdot |z_2|\cdot (\cos(\phi_1+\phi_2) + i\sin(\phi_1+\phi_2))\\
%				z^n &= |z|^n (\cos(n\phi) + i\sin(n\phi))
%			\end{align*}
%			Lösung von $z^n=r(\cos\psi + i\sin\psi)$ ist gegeben durch
%			\begin{align*}
%				|z| &= r^{\f 1n}\\
%				\phi &= \f \psi n + \f {2\pi k}n \qquad k\in 0,1,\dotsc, n-1
%			\end{align*}
%	\end{enumerate}
%\end{df}
%
%\begin{st}
%	\label{st:1.4}
%	$(\C, +, \cdot, |\cdot|)$ ist ein \emph{bewerteter Körper}, d.h. für $|\cdot|: \C \to \R$ gelten:
%	\begin{enumerate}
%		\item $|z| \ge 0 \land (|z| = 0 \iff z = 0)$
%		\item $|z_1\cdot z_2| = |z_1|\cdot |z_2|$
%		\item $|z_1+z_2| \le |z_1| + |z_2|$
%	\end{enumerate}
%	\begin{proof}
%		Beweis durch Nachrechnen
%	\end{proof}
%	\begin{note}
%		Außerdem gilt die Dreiecksungleichung nach unten:
%		\[
%			|z_1 \pm z_2| \ge ||z_1| - |z_2||
%		\]
%		\begin{proof}
%			Beweis durch Nachrechnen.
%		\end{proof}
%	\end{note}
%\end{st}
%
%\begin{df}
%	\label{df:1.5}
%	Eine Folge $(z_n)$ in $\C$ \emph{konvergiert} gegen $z\in \C$, falls
%	\[
%		\forall \eps > 0 \exists \N_\eps\in \N \forall n\ge \N_\eps : |z_n -z| < \eps
%	\]
%	Man schreibt dann $z = \lim_{n\to \infty} z_n$ oder $z_n \to z$ $(n\to \infty)$.
%\end{df}
%
%\begin{st}
%	\label{st:1.6}
%	Es gelte $z_n\to z$ und $w_n \to w$ in $\C$.
%	Dann gilt
%	\begin{enumerate}
%		\item $z_n \pm w_n \to z \pm w$ 
%		\item $z_n\cdot w_n \to z\cdot w$
%		\item
%			Falls $w\neq 0$ und $w_n' = \begin{cases} 1 & w_n=0 \\ w_n & \text{sonst}\end{cases}$, dann gilt
%			\[
%				\f {z_n}{w_n} \to \f zw
%			\]
%		\item $z_n\to z \quad\iff \Re z_n \to \Re z \land  \Im z_n \to \Im z$
%
%	\end{enumerate}
%\end{st}
%
%\begin{df}
%	\label{df:1.7}
%	\begin{enumerate}
%		\item 
%			Sei $r> 0, z_0\in \C$.
%			\[
%				K_r(z_0) := \{z\in \C : |z-z_0| < r\}
%			\]
%			heißt \emph{offene Kreisscheibe} um $z_0$ mit Radius $r$.
%		\item
%			Eine Teilmenge $O \subset \C$ heißt \emph{offen}, falls
%			\[
%				\forall z\in 0 \exists r_z > 0 : K_{r_z}(z) \subset O
%			\]
%			Eine Teilmenge $A \subset \C$ heißt \emph{abgeschlossen}, falls $\C\setminus A$ offen ist.
%
%			Beliebige Vereinigungen und endliche Schnitte offener Mengen sind offen.
%			Beliebige Schnitte und endliche Vereinigungen abgeschlossener Mengen sind abgeschlossen.
%		\item
%			Für eine beliebige Teilmenge $M\subset \C$ ist
%			\[
%				\mathring M := \bigcup_{O\in \{O\subset \C: O \text{ offen} \land O \subset M\}} O
%			\]
%			das \emph{Innere} von $M$ (die größte offene Menge $O\subset M$).
%			\[
%				\_M := \bigcap_{A\in \{A\subset \C: A \text{ abgeschlossen} \land M \subset A\}} A
%			\]
%			der \emph{Abschluss} von $M$ (die kleinste abgeschlossene Menge $A$ mit $M\subset A$).
%	\end{enumerate}
%\end{df}
%
%\begin{ex}
%	\label{ex:1.8}
%	\begin{enumerate}
%		\item 
%			$\emptyset, \C$ sind offen und abgeschlossen. 
%			Alle anderen Teilmengen von $\C$ sind entweder offen oder abgeschlossen oder keins von beidem.
%		\item
%			$K_r(z_0)$ ist offen. $\_{K_r(z_0)}=\{z\in \C : |z -z_0| \le r\}$
%		\item
%			$\R\subset \C$, $\R$ ist nicht offen, betrachte $\C\setminus \R$.
%			$\R\subset \C$ ist abgeschlossen, betrachte $\C\setminus \R$.			
%	\end{enumerate}
%\end{ex}
%
%\begin{df}
%	\label{df:1.9}
%	Sei $O\subset \C$ offen, $f: O \to \C$.
%	Dann heißt $f$ \emph{stetig} in $z_0\in O$, falls
%	\[
%		\forall \eps> 0 \exists \delta_\eps > 0 \forall z\in O : |z-z_0| < \delta \implies |f(z) -f(z_0)| < \eps
%	\]
%	oder äquivalent
%	\[
%		\forall (z_n) \text{ Folge in } O: z_n \to z \implies f(z_n) \to f(z_0)
%	\]
%	$f$ heißt \emph{stetig}, falls $f$ in jedem $z_0\in O$ stetig ist.
%\end{df}
%
%\begin{st}
%	\label{st:1.10}
%	\begin{enumerate}
%		\item 
%			Seien $f,g: O\to \C$, $z_0\in O$, $f,g$ stetig in $z_0$.
%			Dann sind $f\pm g$, $f\cdot g$ und (falls $g(z_0)\neq 0$) $\f fg$ stetig in $z_0$.
%		\item
%			Sei $f:O\to \C$ stetig in $z_0\in O$ und $g:\tilde O \to \C$ stetig in $f(z_0)$, $f(O) \subset \tilde O$.
%			Dann ist $g\circ f$ stetitg in $z_0$
%	\end{enumerate}
%\end{st}
%\begin{proof}
%Beweis über Folgendefinition der Stetigkeit.
%\end{proof}
%
%\begin{nt}
%	\label{nt:1.10}
%	Stetigkeit genauso für $f:M\to \C$ und beliebiger Menge $M\subset \C$.
%\end{nt}
%
%\begin{df}[Funktionenfolgen]
%	Sei $M\subset \C$, $f_n,f: M \to \C$.
%	\begin{enumerate}[1)]
%		\item 
%			$(f_n)$ heißt \emph{punktweise konvergent} gegen $f$ auf $M$, falls
%			\[
%				\forall z\in M \forall \eps > 0 \exists N_{\eps,z} \in \N \forall n>N_{\eps,z}: |f_n(z) - f(z)| < \eps
%			\]
%		\item
%			$(f_n)$ heißt \emph{gleichmäßig konvergent} gegen $f$ auf $M$, falls
%			\[
%				\forall \eps>0 \exists N_{\eps} \in \N \forall n>N_{\eps} \forall z\in M : |f_n(z)-f(z)| < \eps
%			\]
%	\end{enumerate}
%\end{df}
%
%\begin{st}
%	\label{st:1.12}
%	Seien $f_n:M\to \C$ stetig, $(f_n)$ gleichmäßig konvergent auf $M$ gegen $f$. 
%	Dann ist $f$ auch stetig auf $M$.
%	\begin{proof}
%		Seien $z_0\in M, \eps > 0$ fest.
%		\begin{align*}
%			|f(z)-f(z_0)| &\le |f(z)-f_n(z)| + |f_n(z)-f_n(z_0) + |f_n(z)-f(z_0)|
%		\end{align*}
%		1. Wähle $N_{\eps}$, so, dass $|f(z)-f_n(z)| < \frac \eps 3$ für $n>N_{\eps}$. \\
%		2. Setze jetzt aber konkret $n:= N_{\eps}+1$. Mit der Dreiecksgleichung, Stetigkeit und geschicktes Addieren mit der Null ergibt sich.
%		
%		 \[
%		  |f(z)-f(z_0)|\le \underbrace{|f(z)-f_m(z)|}_{\frac\eps 3} + \underbrace{|f_m(z)-f_m(z_0)|}_{<\frac \eps 3, |z-z_0|<\delta_{\frac{\eps}{3}}}+\underbrace{|f_m(z_0)-f(z_0)|}_{<\frac{\eps}{3}}<\eps \text{für } |z-z_0|<\delta
%		 \]
%			
%	\end{proof}
%\end{st}
%
%\begin{df}
%	\label{df:1.13}
%	Eine Reihe $\sum_{n=0}^\infty a_n$ heißt \emph{absolut konvergent}, falls $\sum_{n=0}^\infty |a_n|$ konvergiert.
%\end{df}
%
%\begin{thm}[Weierstraß-Kriterium]
%	\label{thm:1.14}
%	Ist $\sum_{n=0}^\infty a_n$ mit $a_n \ge 0$ konvergent und gilt $f_n:M\to \C$ und $f_n(z)| \le a_n$, so ist die Reihe
%	\[
%		\sum_{n=0}^\infty f_n(z)
%	\]
%	gleichmäßig konvergent auf $M$ und absolute konvergent für $z\in M$.
%\end{thm}
%
%\begin{ex}
%	$M:= \_{K_2(0)} \subset \C$, $f_n(z) := \f {z^n}{(n+1)^22^n}$.
%	Wähle $a_n:= \f 1{(n+1)^2}$, dann folgt
%	\[
%		\begin{cases}
%			\sum_{n=0}^\infty \f 1{(n+1)^2} < \infty \\
%			|f_n(z)| \le a_n
%		\end{cases}
%	\]
%	Also ist $g(z) = \sum_{n=0}^\infty f_n(z)$ nach \ref{thm:1.14} und \ref{st:1.12} stetig auf $\_{K_2(0)}$.
%\end{ex}
%
%\begin{df}[Potenzreihen]
%	Sei $(a_n)$ eine Folge in $\C$
%	\[
%		R := \f 1{\limsup_{n\to \infty}\sqrt[n]{|a_n|}} \qquad\left (\f 10 := \infty, \f 1\infty := 0 \right )
%	\]
%	dann konvergiert die Potenzreihe
%	\[
%		f(z) = \sum_{n=0}^\infty a_n(z-z_0)^n
%	\]
%	für $|z-z_0|<R$ und divergiert für $|z-z_0|>R$.
%	Sei konvergiert gleichmäßig auf jedem Kreis $\_{K_r(z_0)}$ mit $0<r<R$.
%	Insbesondere ist $f$ stetig auf $K_R(z_0)$.
%
%	Falls $(|\f {a_{n+1}}{a_n}|)$ konvergiert, gilt
%	\[
%		R = \f 1{\lim |\f{a_{n+1}}{a_n}|}
%	\]
%\end{df}
%
%\begin{df}
%	Wir definieren
%	\begin{align*}
%		e^z &:= \sum_{n=0}^\infty \f {z^n}{n!}\\
%		\cos z &:= \sum_{n=0}^\infty (-1)^n \f {z^2n}{(2n)!} \\
%		\sin z &:= \sum_{n=0}^\infty (-1)^n \f{z^{2n+1}}{(2n+1)!}
%	\end{align*}
%	\begin{proof}
%		Für die Konvergenz:
%		\begin{align*}
%			\f {a_{n+1}}{a_n} = \f {\f1{(n+1)!}}{\f 1{n!}} = \f {1}{n+1} \to 0
%		\end{align*}
%		Also $R=\infty$, die Potenzreihe konvergiert auf ganz $\C$.		
%	\end{proof}
%\end{df}
%
%\begin{thm}[Cauchy-Produkt von Reihen]
%	\label{thm:1.18}
%	Ist $\sum_{n=0}^\infty a_n$ absolut konvergent, $\sum_{n=0}^\infty b_n$ konvergent in $\C$, so gilt
%	\[
%		\left( \sum_{n=0}^\infty a_n\right) \left(\sum_{n=0}^\infty f_n\right) = \sum_{n=0}^\infty \sum_{k=0}^n a_k b_{n-k}
%	\]
%	\begin{proof}
%		Setze $A:= \sum_{n=0}^\infty$, $B:= \sum_{n=0}^\infty b_n, b_l := \sum_{n=0}^l b_n$.
%		\begin{align*}
%			\sum_{n=0}^N \sum_{k=0}^n a_k b_{n-k}
%				&= \sum_{n=0}^N a_n \sum_{m=0}^{N-n} b_m\\
%				&= \sum_{n=0}^N a_n (B_{N-m}-B + B)\\
%				&= \sum_{l=0}^N a_{N-l} (B_l -B) + \underbrace{\sum_{n=0}^N a_nB}_{\to AB}\\
%		\end{align*}
%		Weiterhin ist für entsprechendes $N_\eps$
%		\begin{align*}
%			\left|\sum_{l=0}^N a_{N-l} (B_l-B) \right| \\
%				&= \left|\sum_{l=0}^{N_\eps} a_{N-l}(B_l-B) + \sum_{l=N_\eps +1}^N a_{N-l}(B_l-B)\right|\\
%				&\le \left|\max_{0\le l \le N_\eps} |B_l-B|\sum_{n=N-N_\eps}^\infty |a_n| \right| \\ &+     \left|\sup_{l\ge N_{\eps}+M} |B_l-B| \sum_{l=N_\eps+1}^N |a_{N-l}| |\right| \qquad n= N-l, \text{wähle hier $N_\eps$} \\
%				&\le \f \eps 2  +  \left|\sup_{l\ge N_{\eps}+M} |B_l-B| \sum_{n=0}^\infty |a_{n}| \right| \\
%				&<     \eps
%		\end{align*}
%		für $N-N_\eps > \tilde N_\eps$ bzw. für  $N>\tilde N_\eps + N_\eps$.
%	\end{proof}
%\end{thm}
%
%\begin{prop}
%	\label{prop:1.19}
%	Es gilt
%	\[
%		e^{z+w}  = e^z\cdot e^w
%	\]
%	\begin{proof}
%		\begin{align*}
%			e^ze^w &= \left(\sum_{n=0}^\infty \f{z^n}{n!}\right) \left( \sum_{n=0}^\infty \f {w^n}{n!}\right) \\
%				&= \sum_{n=0}^\infty \sum_{k=0}^n \f {z^k}{k!} \f {w^{n-k}}{(n-k)!} \\
%				&= \sum_{n=0}^\infty \f 1{n!} \sum_{k=0}^n \binom{n}{k}z^k w^{n-k} \\
%				&= \sum_{n=0}^\infty \f {(z+k)^n}{n!} = e^{z+w}
%		\end{align*}
%	\end{proof}
%\end{prop}
%
%\begin{nt}
%	\label{1.20}
%	Aus der Taylorreihe folgt außerdem, dass $e^z, \sin z, \cos z$ für $z\in \R$ die selben Funktionen sind, wie aus der Schule bekannt.
%\end{nt}
%
%\begin{prop}
%	\label{1.21}
%	\begin{enumerate}[1)]
%		\item 
%			$e^{iz} = \cos z + i \sin z$ für $z\in \C$ (Eulersche Formel)
%			\begin{proof}
%				\[
%					e^{iz} = \sum_{n=0}^\infty \f {(iz)^n}{n!} = \sum_{k=0}^\infty \f {(iz)^{2k}}{(2k)!} + \sum_{k=0}^\infty \f {(iz)^{2k+1}}{(2k+1)!} =
%\sum_{k=0}^\infty (-1)^k\f {z^{2k}}{(2k)!} + \sum_{k=0}^\infty \f (-1)^k {z^{2k+1}}{(2k+1)!}
%\]
%			\end{proof}
%		\item
%			Mit der Polardarstellung $z = r(\cos \phi + i \sin \phi) = r e^{i\phi}$ gilt
%			\begin{align*}
%				z^n &= r^n e^{in\phi}\\
%				z_1 z_2 &= r_1 r_2 e^{\phi_1 + \phi_2}\\
%				\f {z_1}{z_2} &= \f {r_1}{r_2} e^{i(\phi_1-\phi_2)}
%			\end{align*}
%	\end{enumerate}
%\end{prop}
%
%\begin{df}
%	\label{df:1.22}
%	\begin{enumerate}[1)]
%		\item 
%			$f: O \to \C$ heißt \emph{differenzierbar} in $z_0\in O$, falls
%			\[
%				f'(z_0 := \lim_{\substack{z\to z_0 \\ z\neq z_0}}\f {f(z) -f(z_0)}{z-z_0}
%			\]
%			existiert.
%			$f'(z_0)$ heißt \emph{Ableitung} von $f$ in $z_0$.
%		\item
%			$f$ heißt \emph{differenzierbar}, falls $f$ in jedem $z_0\in O$ differenzierbar ist.
%			$f':O\to \C$ heißt \emph{Ableitung(sfunktion)} von $f$.
%	\end{enumerate}
%\end{df}
%
%\begin{ex*}
%	\begin{enumerate}[1)]
%		\item 
%			$f: \C \to \C : z\mapsto c$, dann ist $f'=0$.
%		\item
%			$f: \C \to \C : z\mapsto z$, dann ist $f'=1$.
%		\item
%			ohne Beweis: $f:\C \to \C :z\mapsto z^m$, dann ist $f'(z) = nz^{n-1}$.
%		\item
%			$f: \C\to \C : z\mapsto |z|^2$ ist in $z_0\neq 0$ nicht differenzierbar.
%			\begin{proof}
%				Sei $z_0 = x_0+iy_0 \neq 0$ und $z_h:= x_0+h +iy_0$, dann ist
%				\[
%					\f {|z_h|^2 -|z_0|^2}{z_h-z_0} = \f {2hx_0+h^2}h \to 2x_0
%				\]
%				Sei jetzt $z_h := x_0 + i(y_0+h)$, dann ist
%				\[
%					\f {|z_h|^2 -|z_0|^2}{z_h-z_0} := \f {2hx_0+h^2}{ih} \to -2ix_0
%				\]
%			\end{proof}
%	\end{enumerate}
%\end{ex*}
%
%\begin{st}
%	\label{st:1.24}
%	Sei $O\subset \C$ offen, $f:O\to \C$, $z_0\in O$.
%	Dann sind äquivalent
%	\begin{enumerate}[(i)]
%		\item
%			$f$ ist differenzierbar in $z_0$ mit Ableitung $f'(z_0)$
%		\item
%			$\exists f'(z_0) \in \C : f(z) =f(z_0)+f'(z_0)(z-z_0)+o(|z-z_0|)$
%	\end{enumerate}
%	\begin{proof}
%		\begin{enumerate}
%			\item 
%				\[
%					\iff \f{f(z)-f(z_0)}{z-z_0}-f'(z_0) \to 0
%				\]
%				\fixme
%		\end{enumerate}
%	\end{proof}
%\end{st}
%
%\begin{st}
%	\label{1.25}
%	\begin{enumerate}[1)]
%		\item 
%			$f$ differenzierbar in $z_0$ $\implies$ $f$ stetig in $z_0$.
%		\item
%		\item
%		\item
%		\item
%	\end{enumerate}
%\end{st}
%
%\begin{ex}
%	\begin{enumerate}[1)]
%		\item 
%			Polynomfunktionen sind differenzierbar auf $\C$
%		\item
%			Gebrochenrationale Funktionen sind auf ihrem Definitionsbereich differenzierbar
%	\end{enumerate}
%\end{ex}
%
%\begin{df}
%	\begin{enumerate}[1)]
%		\item 
%			Sei $\gamma\in C^1([a,b]\to \C)$.
%			Dann heißt $\gamma$ \emph{Weg} von $z_1=\gamma(a)$ nach $z_2=\gamma(b)$.
%			Falls $z_1=z_2$, heißt $\gamma$ \emph{geschlossen}.
%		\item
%			Sei $\gamma$ eine Weg, $f\in C^1(\Re(\gamma)\to \C)$.
%			Dann heißt
%			\begin{align*}
%				\int_\gamma f(z) dz := \int_a^b f(\gamma(t)) \gamma'(t) dt\\
%				=: \int_a^b(\Re f(\gamma(t)) \Re \gamma'(z) - \Im f(\gamma(t)) \Im \gamma'(t))dt
%				+ i\int_a^b(\Im f(\gamma(t)) \Re \gamma'(t) + \Re f(\gamma(t)) \Im \gamma'(t))dt
%			\end{align*}
%			das \emph{Integral} längs $\gamma$.
%	\end{enumerate}
%\end{df}
%
%\begin{nt}
%	\label{st:1.28}
%	\begin{enumerate}[1)]
%		\item 
%			Ist $\tilde \gamma$ eine andere Parametrisierung des Weges $\gamma$, sodass $\phi'\in C^1([a',b']\to [a,b])$
%			\begin{align*}
%				\tilde \gamma &= (\gamma \circ \phi)(s) \qquad a'\le s\le b'\\
%				\phi(a') &= a\\
%				\phi(b') &= b
%			\end{align*}
%			Dann folgt aus der Substitutionsregel für reelle Integration
%			\begin{align*}
%				\int_\gamma f(z) dz &= \int_a^b f(\gamma(t)) \gamma'(t) dt \\
%				&= \int_{a'}^{b'} \underbrace{f(\gamma(\phi(s)))}_{f(\tilde \gamma(s)} \underbrace{\gamma'(\phi(s))\phi'(s)}_{\f d{dx}\tilde \gamma(s)=\f d{dx}(\gamma\circ \phi)(s)} ds\\
%				&= \int_{\tilde \gamma} f(z) dz
%			\end{align*}
%		\item
%			$-\gamma := $  derselbe Weg, aber entgegengesetzt orientiert:
%			\[
%				-\gamma(t) := \gamma(a+b-t) \qquad a\le t\le b
%			\]
%			Dann ist auch (selber nachrechnen)
%			\[
%				\int_{-\gamma}f(z) dz = - \int_\gamma f(z) dz
%			\]
%		\item
%			Jede Kurve kann so umparametrisiert werden, dass $a=0, b=1$ gilt.
%		\item
%			Verbindungsstrecke $\gamma=[z_1,z_2]$ ist definiert als
%			\[
%				\gamma(t) = z_1 + t(z_2-z_1) \qquad 0\le t\le 1
%			\]
%		\item
%			Verallgemeinerung: Ein Weg kannn auch nur stückweise $C^1$ sein:
%			\[
%				\gamma\in C([a,b]\to \C)
%			\]
%			(stetig).
%			Es existieren $t_0=a<t_1<t_2<\dotsb<t_n=b$ so dass
%			\[
%				\gamma_j = \gamma\big|_[t_{j-1},t_j] \in C^1([t_{j-1},t_j]\to \C)
%			\]
%			d.h. die Ableitung, darf endlich viele Sprünge haben.
%			Dann ist
%			\[
%				\int_\gamma f(z) dz := \sum_{j-1}^n \int_{\gamma_j}f(z) dz
%			\]
%	\end{enumerate}
%\end{nt}
%
%\begin{ex}
%	\begin{enumerate}[1)]
%		\item 
%			$f(z)=z^3$, Verbindungsstrecke: $\gamma=[0,1+i]$
%			Dann ist
%			\[
%				\int_{[0,1+i]}z^3 dz = \int_0^1(t(1+i))^3(1+i) dt = (1+i)^4 \int_0^1 t^3 dt = \f {(1+i)^4}4
%			\]
%		\item
%			$f(z)=\f 1z$, $\gamma(t)=e^{it}$, $0\le t\le 2\pi$.
%			Das ist wegen $\gamma(0)=\gamma(2\pi)$ ein geschlossener Weg.
%			\[
%				\int_\gamma \f 1z dz = \int_0^{2\pi}\f 1{e^{it}} ie^{it} dt = 2\pi i \neq 0 
%			\]
%			Obwohl die Kurve geschlossen ist, kommt nicht $0$ heraus!
%	\end{enumerate}
%\end{ex}
%
%\begin{st}
%	\label{st:1.30}
%	Sei $O\subset \C$ offen, $F:O\to \C$ differenzierbar, $f=F'$ in $O$ (d.h. $F$ ist eine \emph{Stammfunktion} von $f$).
%	Ist $\gamma$ ein Weg in $O$, so gilt
%	\[
%		\int_\gamma f(z) dz f F(\gamma(b)) - F(\gamma(a))
%	\]
%	\begin{proof}
%		\begin{align*}
%			\int_\gamma f(z) dz &= \int_a^b f(\gamma(t)) \gamma'(t) dt \\
%			\intertext{
%				Wegen
%				\begin{align*}
%					\f d{dt} F(\gamma(t)) &= \lim_{\substack{s \to t \\ s\neq t}} \f {F(\gamma(s) -F(\gamma(t))}{\gamma(s) - \gamma(t)} \cdot \f {\gamma(s)-\gamma(t)}{s-t} \\
%				&= F'(\gamma(t)) \cdot \gamma'(t) \\
%				&= f(\gamma(t)) \gamma'(t)
%				\end{align*}
%				gilt weiter
%			}
%			&= \int_a^b (\f d{dt} \Re F( \gamma(t)) + i \f d{dt} \Im F(\gamma(t))) dt \\
%			&= \int_a^b \Re F(\gamma(b)) + i \Im F(\gamma(b))\\
%			&= F(\gamma(b)r - F(\gamma(a))
%		\end{align*}
%	\end{proof}
%\end{st}
%
%\begin{prop}
%	\label{prop:1.31}
%	\begin{enumerate}[1)]
%		\item 
%			Besitzt $f$ eine Stammfunktion und ist $\gamma$ geschlossen, folgt $\int_\gamma f(z)dz = 0$.
%		\item
%			$f:\C\setminus \{0\} \to \C : z\mapsto \f 1z$ besitzt kein Stammfunktion (siehe \ref{ex:1.29} 2.)
%	\end{enumerate}
%\end{prop}
%
%\begin{st}
%	\label{st:1.32}
%	Sei $\gamma$ Weg, $f\in C(\im(\gamma) \to \C)$.
%	Dann ist
%	\[
%		\left| \int_\gamma f(z) dz \right| \le \int_a^b |f(\gamma(t))| |\gamma'(t)| dt \le
%\le \max_{a\le t \le b} |f(\gamma(t))| \int_a^b |\gamma'(t)|| dt
%	\]
%	\begin{proof}
%		\begin{align*}
%			\left| \int_\gamma f(z) dz \right| &= e^{i\phi} \int_\gamma f(z) dz \qquad \phi= -\arg\left(\int_\gamma f(z)dz\right)\\
%			&= \int_a^b e^{i\phi} f(\gamma(t) \gamma'(t) dt \\
%			&= \int_a^b \Re\left(e^{i\phi} f(\gamma(t)) \gamma'(t)\right)dt + i \underbrace{\int_a^b \Im\left( \dotso\right)dt}_{=0} \\
%			&\le \int_a^b |\Re e^{i\phi} f(\gamma(t)) \gamma'(t)| dt\\
%			&\le \int_a^b |e^{i\phi} f(\gamma(t)) \gamma'(t)| dt\\
%			&\le \int_a^b |e^{i\phi}| |f(\gamma(t))| |\gamma'(t)| dt\\
%			&\le \int_a^b |f(\gamma(t))| |\gamma'(t)| dt\\
%		\end{align*}
%		Der Beweis funktioniert auch, falls $\gamma$ nur stückweise $C^1$ ist.
%	\end{proof}
%\end{st}
%
%\begin{df}
%	\label{df:1.33}
%	Man nennt
%	\[
%		L(\gamma) := \int_a^b |\gamma'(t)| dt
%	\]
%	\emph{Länge} von $\gamma$.
%	Damit wird \ref{st:1.32} zu
%	\[
%		\left| \int_\gamma f(z) dz \right| \le L(\gamma) \cdot \max_{a\le t\le b}|f(\gamma(t))|
%	\]
%\end{df}
%
%\begin{ex}
%	\begin{enumerate}[1)]
%		\item 
%			$\gamma=[z_1,z_2]$, $\gamma(t) = z_1+t(z_2+z_1)$, dann ist
%			\[
%				L(\gamma) = \int_0^1 |z_2-z_1| dt = |z_2-z_1|
%			\]
%		\item
%			\begin{enumerate}[a)]
%				\item 
%					$\gamma(t) = e^{it}$, $0\le t\le 2\pi$.
%					Es gilt nach 1.29 und 1.32
%					\[
%						|2\pi i| = \left| \int_\gamma \f 1z dz \right| \le \max_{\substack{z=e^{it}\\0\le t \le 2\pi}} \cdot \int_0^{2\pi} |ie^{it}| dt = 2\pi
%					\]
%				\item
%					Es gilt
%					\[
%						|0| = \left|\int_\gamma z dz\right| \le \max_{\substack{z=e^{it}}} \cdot 2\pi = 2\pi
%					\]
%			\end{enumerate}
%	\end{enumerate}
%\end{ex}
%
%\begin{st}
%	Sei $(a_n)$ eine Folge in $\C$ mit Konvergenzradius $R:= \f {1}{\limsup_{n\to \infty}\sqrt[n]{|a_n|}}>0$ und 
%	\[
%		f(z) := \sum_{n=0}^\infty a_n z^n
%	\]
%	ür $|z|<R$.
%	Dann ist $f$ in $K_R(0)$ beliebig oft differenzierbar mit
%	\[
%		f'(z) = \sum_{n=1}^\infty a_n nz^{n-1}, \qquad f''(z) = \sum_{n=2}^\infty a_nn(n-1)z^{n-2}
%	\]
%	\begin{proof}
%		Sei $g(z) := \sum_{n=1}^{\infty}a_n nz^{n-1}, R':= \f 1{\limsup_{n\to \infty}\sqrt[n]{a_n n}}$ (Konvergenzradius für $z$ ist der selbe wie für $g(z)$).
%		\begin{enumerate}[1)]
%			\item 
%				$R'=R$ wegen $\lim_{n\to \infty} \sqrt[n]{a_n} = 1$.
%				D.h. $g$ ist auf demselbe $K_R(0)$ definiert, wi $f$.
%			\item
%				$g=f'$: Seien $w\in K_R(0), z\in K_R(0)\setminus \{w\}$.
%				\begin{align*}
%					\f {f(z) -f(w)}{z-w} - g(w) &= \sum_{n=1}^\infty a_n \left( \f {z^n-w^n}{z-w} - nw^{n-1}\right) \\
%					&= \sum_{n=1}^\infty a_n \left( \f 1{z-w}\int_{[w,z]}nu^{n-1}-nw^{n-1}dt\right) \\
%					|\cdot| &\le \sum_{n=1}^\infty |a_n| \f 1{|z-w|} \cdot L([w,z]) \cdot n|z^{n-1}-w^{n-1}| \\
%					&= \begin{cases} \to 0  & z\to w\\
%						\le n|z|^{n-1}+ n|w|^{n-1} & \text{sonst}
%					\end{cases}
%				\end{align*}
%				Die Reihe konvergiert also gleichmäßig in $K_r(0)$ für jedes $r<R$.
%				Also ist der Grenzwert $z\to w$ und die Reihe vertauschbar.
%				Damit gilt
%				\[
%					\f {f(z) -f(w)}{z-w}- g(w) \to 0
%				\]
%		\end{enumerate}
%	\end{proof}
%\end{st}
%



\section{Vektoranalysis}



\begin{proof}
	\begin{enumerate}[1)]
		\item
			Es gelte o.B.d.A
			\[
				\omega = a(x) dx_J \qquad J \in \scr G^{(k)}
			\]
		\item
			Lokalisierung:
			Sei $A(S)$ orientierter Atlas, sortiert wie in \ref{8.6}.
			\begin{align*}
				\int_S d\omega &= \sum_{j=1}^N \int_{U_j} \psi_j d\omega \\
				&= \sum_{i=1}^n \sum_{j=1}^N \int_{U_j}\d_{x_i}(\psi_j a) dx_i \wedge dx_J \\
				&= \sum_{i=1}^n \underbrace{\sum_{j=1}^N \int_{U_j}(\d_{x_i}\psi_j) a xx_i \wedge dx_J}_{\int_S (\underbrace{\sum_{j=1}^N \d_{x_i}\psi_j)}_{\d x_i \sum \psi_j = \d x_{i}1 = 0}a dx_i \wedge dx_J} \\
				&= \sum_{j=1}^N \int_{U_j}d(\psi_j \omega)
			\end{align*}
			Wir zeigen jetzt
			\[
				\int_{U_j} d(\psi_j \omega) = \begin{cases}
					\int_{\tilde U_j = U_j \cap \d S} \psi_j \omega & 1 \le j \le L \\
					0& L+1 \le j \le N
				\end{cases}
			\]
			Dann gilt, da $\{\psi_1,\dotsc, \psi_L\}$ eine Zerlegung der Eins ist für $\d S$ mit Altas $\{(\tilde \phi_j, \tilde U_j), 1 \le j \le L \}$
			\[
				\int_S d\omega = \sum_{j=1}^L \int_{\tilde U_j} \psi_j \omega  = \int_{\d S} \omega
			\]
		\item
			Sei $j= \{1,\dotsc, N\}$ fest, $\phi_j = (g_1,\dotsc, g_n)$, $D=K_1^{(k+1)}(0)$ falls $1\le j \le L$, $D=K_1^{(k+1)}(0)$ sonst.
			Also
			\begin{align*}
				\int_{U_j} d(\psi_j \omega) &= \sum_{i=1}^n \int_{y\in D} \d_{x_i} (\psi_j a)(\phi_j(y))) \det(\tf{\d(g_i,g_J)}{\d y}) dy 
				\intertext{
					Entwickeln der Determinante nach der ersten Zeile ergibt
				}
				&= \sum_{i=1}^n \sum_{l=1}^{k+1} (-1)^{k+1} \int_{y\in D} \d_{x_i}(\psi_j a)(\phi_j(y)) \tf{\d_{g_i}}{\d_{y_l}}(y) \cdot \det \bigg( \f{\d(g_{i_1},\dotsc,g_{i_k})}{\underbrace{\d(y_1,\dotsc, y_{l-1},y_{l+1},\dotsc, y_{k+1})}_{y^{(l)}}} dy \\
				&= \sum_{l=1}^{k+1} (-1)^{l+1} \int_{y \in D} \d_{y_l}((\psi_j a)\circ \phi_j)(y) \det(\dotsc) dy \\
				&= \sum_{l=1}^{k+1} (-1)^{l+1} \int_{y^{(l)}\in K_1^{(k)}(0)} \int_{y_l=- \sqrt{1 - |y^{(l)}|^2}}^{y_l = + \sqrt{1-|y^{(l)}|^2} \text{ oder $y_l=0$ falls $1\le j \le L$}} \dotsc d y_l dy^{(l)}
				\intertext{
					Integriere jetzt partiell und beachte: $(\psi_j a)\circ \psi = 0$ für $y_l = \pm \sqrt{1-|y^{(l)}|^2}$ da $\supp \psi_j \le U_j$.
					Außerdem $\sum_{l=1}^{k+1} (-1)^{l+1} \d y_l (\det (\dotsc)) = 0$ (Nachrechnen).
				}
				&= \begin{cases}
					0 & L+1 \le j \le N \\
					\int_{\tilde U_j} \psi \omega \stackrel{\tilde U_j = \tilde \phi_j(K_1^{(k)}(0))}= (-1)^{i+1} \int_{y^{(i)}\in K_1^{(k)}(0)} (\psi_j a) \circ \phi_j (0,y^{(1)}) \det(\tf{\d g_J}{\d y^{(1)}}) dy^{(1)} & 
				\end{cases}
			\end{align*}
	\end{enumerate}
\end{proof}

\begin{nt*} \label{8.8}
	\begin{enumerate}[1)]
		\item
			Im Fall $k=0, \omega = a(x), S = \phi([\alpha,\beta]), \d S = \{\phi(\alpha), \phi(\beta)\}$.
			Nach \ref{8.4}:
		\item
			Satz von Stokes gilt auch für $S= \_{O}$, d.h. $S \subset O$ nicht notwendig.
		\item
			Folgerungen
			\begin{enumerate}[a)]
				\item
					Der Satz von Gauß-Ostrogradski:
					Sei $f\in C^1(O\to \R^n)$.
					\[
						\int_{S} \nabla \cdot f dV^{(n)} = \int_{\d S} \<f,n_0\> dV{(n-1)}
					\]
					Setze dazu $\omega := \sum_{j=1}^n (f_j dx_1 \wedge \dotsc \wedge dx_{j-1} \wedge dx_{j+1} \wedge \dotsc \wedge dx_n)$
				\item
					Klassischer Satz von Stokes:
					Sei $S \subset \R^3$ eine Mannigfaltigkeit der Dimension $2$ und $f \in C^1(S \to \R^3)$, dann gilt
					\[
						\int_S (\nabla \times f) dV^{(2)} = \int_{\d S}\<f,t_0\> dV^{(1)}
					\]
					Setze dazu $\omega := f_1 dx_1 + f_2 dx_2 + f_3 dx_3$.
			\end{enumerate}
	\end{enumerate}
\end{nt*}


\section{Gewöhnliche Differentialgleichungen}


\subsection{Funktionalanalysis}

\begin{def} \label{1.1}
	Sei $(B,+,\cdot)$ ein linearer Raum, d.h. Vektoraum und $\|\cdot\|: B \to \R$ eine Norm (d.h. $\|u\| \ge 0$, $\|u\|=0 \iff u = 0$, $\|u+v\| \le \|u\| + \|v\|$) (Dreiecksungleichung), $\|\alpha u \| = |\alpha| \|u\|$)
	Ist $B$ vollständig, d.h. jede Cauchy-Folge 
\end{def}

\begin{ex} \label{1.2}
	\begin{enumerate}[1)]
		\item
			\[
				B = \C^n, \|u\| = (|u_1|^p + \dotsb + |u_n|^p)^{\f 1p} \qquad 1 \le p < \infty
			\]
		\item
			Sei $B = C([a,b] \to \C)$ mit
			\[
				\|f\|_\infty = \max_{a \le x \le b} |f(x)|
			\]
			z.B. $\|f\|_2^2 = \int_a^b |f(x)|^2 dx$, dann ist $B$ kein Banach-Raum.
		\item
			Definiere
			\[
				L^p(I) := \bigg\{ f : I \to \C : f \text{ messbar} \land \int_I |f|^p d\my < \infty \bigg\}
			\]
			für ein Intervall $I \subset \R$.
			$L^p(I)$ mit
			\[
				\|f\|_p := \bigg( \int_I |f|^p d\my \bigg)^{\f 1p}
			\]
			(Identifiziere $f,\tilde f$, falls $\int_I |f-\tilde f| d\my = 0$)
	\end{enumerate}
\end{ex}


\begin{st}[Banachscher Fixpunktsatz] \label{1.3}
	Sei $B$ ein Banaachraum, $\emptyset \neq D \subset B$ mit $D$ abgeschlossen und $F: D \to B$ eine \emph{Kontraktion}, d.h.
	\[
		\exists q \in [0,1[ \forall x,y \in D : \|F(x) - F(y)\| \le q\|x-y\|
	\]
	mit $F(D) \subset D$. Dann gilt
	\begin{enumerate}[1)]
		\item
			Es existiert $x\in D$ mit $F(x)=x$, d.h. die Abbildung $F$ hat genau einen \emph{Fixpunkt} $x\in D$.
		\item
			Ist $x_0 \in D$ und $x_n := F(x_{n-1})$ für $n\in \N$, so gilt
			\[
				x_n \to x \qquad n \to \infty
			\]
			mit der \emph{Fehlerabschätzung}
			\[
				\|x_n - x\| \le \f{q^n}{1-q} \|x_1 - x_0\|
			\]
	\end{enumerate}
	\begin{proof}
		\begin{enumerate}[1)]
			\item
				Wegen $F(D) \subset D$ ist $x_n$ für $n\in \N_0$ definiert
			\item
				Zeige 
				\[
					\|x_{n+1} - x_n\| \le q^n \|x_1 - x_0\|
				\]
				induktiv
			\item
				Siehe Numerik (wurde dort schöner bewiesen).
			\item
				Eindeutigkeit:
				\[
					\|x-y\| = \|F(x) - F(y)\| \le q \|x-y\|
				\]
				und damit $\|x-y\| = 0$, also $x=y$.
		\end{enumerate}
	\end{proof}
\end{st}

\begin{nt} \label{1.4}
	Man kann genauso beweisen
	\[
		\|x - x_n\| \le \f 1{1-q} \|x_{n+1} -x_n \|
	\]
\end{nt}

\end{document}

