\documentclass[a4paper,10pt]{scrartcl}
\usepackage{mathe-vorlesung}

\title{Analysis 3}

\begin{document}

\maketitle

\tableofcontents
\newpage

\section{Funktionentheorie}

\subsection{Grundlagen}

\begin{df}
	\label{df:1.1}	
	Die komplexen Zahlen bestehen aus
	\[
		\C := \{(x,y) : x,y\in \R\}
	\]
	und den Verknüpfungen
	\begin{align*}
		(x_1,y_1) + (x_2,y_2) &:= (x_1 + x_2, y_1 + y_2) \in \C \\
		(x_1,y_1) \cdot (x_2,y_2) &:= (x_1x_2 - y_1y_2, x_1y_2 + x_2y_1) \in \C
	\end{align*}
\end{df}

\begin{nt}
	\label{nt:1.1}
	\begin{enumerate}
		\item $(\C,+,\cdot)$ ist ein Körper mit $(0,0)$ und Einselement $(1,0)$.
		\item 
			$\phi: \R\to \C : x\mapsto (x,0)$ ist eine injektiver Körperhomomorphismus.
			Insbesondere gilt
			\begin{align*}
				\phi(x_1+x_2) &= \phi(x_1) + \phi(x_2)\\
				\phi(x_1+x_2) &= \phi(x_1) \cdot \phi(x_2)
			\end{align*}
			Identifiziere $\R$ mit $\phi(\R) = \{(x,0) : x\in \R\}$.
			Schreibe dazu: $(x,0) =: x \in \R$.
		\item
			Imaginäre Einheit $i:= (0,1)$.
			\[
				\implies \begin{cases}
				i^2 = (0,1)\cdot (0,1) = (0\cdot 0 - 1\cdot 1, 0\cdot 1 + 0\cdot 1) = (-1,0) = -1 \\
				(x,y) = (x,0) + (0,y) = (x,0) + y\cdot i = x + yi
				\end{cases}
			\]
			Rechnen in $\C$:
			\begin{align*}
				(x_1,y_1)\cdot (x_2,y_2) &= (x_1 +iy_1)\cdot (x_2 + iy_2)\\
				&= x_1x_2 + ix_1y_2 + iy_1x_2 + (i)^2y_1y_2\\
				&= x_1x_2 - y_1y_2 + i(x_1y_2 + x_2y_1)
			\end{align*}
			oder
			\begin{align*}
				\f 1{x+iy} = \f 1{x+iy}\cdot \f {x-iy}{x-iy} = \f{x-y}{x^2-(iy)^2} = \f x{x^2+y^2} + i \f{-y}{x^2+y^2}
			\end{align*}
			Realteil: $\Re(x+iy) = x\in \R$<br />
			Imaginärteil: $\Im(x+iy) = y\in\R$
		\item
			Gaußsche Zahlenebene:
	\end{enumerate}
\end{nt}

\begin{df}
	\label{df:1.3}
	\begin{enumerate}
		\item 
			Für $z=x+iy$ heißt
			\[
				\_z = x-iy
			\]
			\emph{konjugiert komplexe Zahl}
		\item
			\[
				|z| = \sqrt{x^2+y^2} = \sqrt{z\cdot \_z}
			\]
		\item
			Polardarstellung: $z=x+iy = |z|(\cos \phi + i\sin\phi)$ wobei $\phi = \arg(z)$ (Argument von $z$) eindeutig durch
			\[
			-\pi \le \phi \le \pi, \qquad \cos\phi = \f x{\sqrt{x^2+y^2}}, \qquad \sin\phi = \f y{\sqrt{x^2+y^2}}
			\]
			Rechnen mit Polardarstellung:
			\begin{align*}
				z_1 \cdot z_2 &= |z_1|\cdot |z_2|\cdot (\cos(\phi_1+\phi_2) + i\sin(\phi_1+\phi_2))\\
				z^n &= |z|^n (\cos(n\phi) + i\sin(n\phi))
			\end{align*}
			Lösung von $z^n=r(\cos\psi + i\sin\psi)$ ist gegeben durch
			\begin{align*}
				|z| &= r^{\f 1n}\\
				\phi &= \f \psi n + \f {2\pi k}n \qquad k\in 0,1,\dotsc, n-1
			\end{align*}
	\end{enumerate}
\end{df}

\begin{st}
	\label{st:1.4}
	$(\C, +, \cdot, |\cdot|)$ ist ein \emph{bewerteter Körper}, d.h. für $|\cdot|: \C \to \R$ gelten:
	\begin{enumerate}
		\item $|z| \ge 0 \land (|z| = 0 \iff z = 0)$
		\item $|z_1\cdot z_2| = |z_1|\cdot |z_2|$
		\item $|z_1+z_2| \le |z_1| + |z_2|$
	\end{enumerate}
	\begin{proof}
		Beweis durch Nachrechnen
	\end{proof}
	\begin{note}
		Außerdem gilt die Dreiecksungleichung nach unten:
		\[
			|z_1 \pm z_2| \ge ||z_1| - |z_2||
		\]
		\begin{proof}
			Beweis durch Nachrechnen.
		\end{proof}
	\end{note}
\end{st}

\begin{df}
	\label{df:1.5}
	Eine Folge $(z_n)$ in $\C$ \emph{konvergiert} gegen $z\in \C$, falls
	\[
		\forall \eps \gt 0 \exists \N_\eps\in \N \forall n\ge \N_\eps : |z_n -z| \lt \eps
	\]
	Man schreibt dann $z = \lim_{n\to \infty} z_n$ oder $z_n \to z$ $(n\to \infty)$.
\end{df}

\begin{st}
	\label{st:1.6}
	Es gelte $z_n\to z$ und $w_n \to w$ in $\C$.
	Dann gilt
	\begin{enumerate}
		\item $z_n \pm w_n \to z \pm w$ 
		\item $z_n\cdot w_n \to z\cdot w$
		\item
			Falls $w\neq 0$ und $w_n' = \begin{cases} 1 & w_n=0 \\ w_n & \text{sonst}\end{cases}$, dann gilt
			\[
				\f {z_n}{w_n} \to \f zw
			\]
		\item $z_n\to z \quad\iff \Re z_n \to \Re z \land  \Im z_n \to \Im z$

	\end{enumerate}
\end{st}

\begin{df}
	\label{df:1.7}
	\begin{enumerate}
		\item 
			Sei $r\gt 0, z_0\in \C$.
			\[
				K_r(z_0) := \{z\in \C : |z-z_0| \lt r\}
			\]
			heißt \emph{offene Kreisscheibe} um $z_0$ mit Radius $r$.
		\item
			Eine Teilmenge $O \subset \C$ heißt \emph{offen}, falls
			\[
				\forall z\in 0 \exists r_z \gt 0 : K_{r_z}(z) \subset O
			\]
			Eine Teilmenge $A \subset \C$ heißt \emph{abgeschlossen}, falls $\C\setminus A$ offen ist.

			Beliebige Vereinigungen und endliche Schnitte offener Mengen sind offen.
			Beliebige Schnitte und endliche Vereinigungen abgeschlossener Mengen sind abgeschlossen.
		\item
			Für eine beliebige Teilmenge $M\subset \C$ ist
			\[
				\mathring M := \bigcup_{O\in \{O\subset \C: O \text{ offen} \land O \subset M\}} O
			\]
			das \emph{Innere} von $M$ (die größte offene Menge $O\subset M$).
			\[
				\_M := \bigcap_{A\in \{A\subset \C: A \text{ abgeschlossen} \land M \subset A\}} A
			\]
			der \emph{Abschluss} von $M$ (die kleinstel abgeschlossene Menge $A$ mit $M\subset A$).
	\end{enumerate}
\end{df}

\begin{ex}
	\label{ex:1.8}
	\begin{enumerate}
		\item 
			$\emptyset, \C$ sind offen und abgeschlossen. 
			Alle anderen Teilmengen von $\C$ sind entweder offen oder abgeschlossen oder keins von beidem.
		\item
			$K_r(z_0)$ ist offen. $\_{K_r(z_0)}=\{z\in \C : |z -z_0| \le r\}$
		\item
			$\R\subset \C$, $\R$ ist nicht offen, betrachte $\C\setminus \R$.
			$\R\subset \C$ ist abgeschlossen, betrachte $\C\setminus \R$.			
	\end{enumerate}
\end{ex}

\begin{df}
	\label{df:1.9}
	Sei $O\subset \C$ offen, $f: O \to \C$.
	Dann heißt $f$ \emph{stetig} in $z_0\in O$, falls
	\[
		\forall \eps\gt 0 \exists \delta_\eps \gt 0 \forall z\in O : |z-z_0| \lt \delta \implies |f(z) -f(z_0)| \lt \eps
	\]
	oder äquivalent
	\[
		\forall (z_n) \text{Folge in} O: z_n \to z \implies f(z_n) \to f(z_0)
	\]
	$f$ heißt \emph{stetig}, falls $f$ in jedem $z_0\in O$ stetig ist.
\end{df}

\begin{st}
	\label{st:1.10}
	\begin{enumerate}
		\item 
			Seien $f,g: O\to \C$, $z_0\in O$, $f,g$ stetig in $z_0$.
			Dann sind $f\pm g$, $f\cdot g$ und (falls $g(z_0)\neq 0$) $\f fg$ stetig in $z_0$.
		\item
			Sei $f:O\to \C$ stetig in $z_0\in O$ und $g:\tilde O \to \C$ stetig in $f(z_0)$, $f(O) \subset \tilde O$.
			Dann ist
	\end{enumerate}
\end{st}

\begin{nt}
	\label{nt:1.10}
	Stetigkeit genauso für $f:M\to \C$ und beliebiger Menge $M\subset \C$.
\end{nt}

\begin{df}[Funktionenfolgen]
	Sei $M\subset \C$, $f_n,f: M \to \C$.
	\begin{enumerate}[1)]
		\item 
			$(f_n)$ heißt \emph{punktweise konvergent} gegen $f$ auf $M$, falls
			\[
				\forall z\in M \forall \eps > 0 \exists N_{\eps,z} \in \N \forall n>N_{\eps,z}: |f_n(z) - f(z)| < \eps
			\]
		\item
			$(f_n)$ heißt \emph{gleichmäßig konvergent} gegen $f$ auf $M$, falls
			\[
				\forall \eps>0 \exists N_{\eps} \in \N \forall n>N_{\eps} \forall z\in M : |f_n(z)-f(z)| < \eps
			\]
	\end{enumerate}
\end{df}

\begin{st}
	\label{st:1.12}
	Seien $f_n:M\to \C$ stetig, $(f_n)$ gleichmäßig konvergent auf $M$ gegen $f$. 
	Dann ist $f$ auch stetig auf $M$.
	\begin{proof}
		Seien $z_0\in M, \eps > 0$ fest.
		\begin{align*}
			|f(z)-f(z_0)| &\le |f(z)-f_n(z)| + |f_n(z)-f_n(z_0) + |f_n(z)-f(z_0)|
		\end{align*}
		Wähle $N_{\eps}$, so, dass $|f_(z)-f_n(z)| < \f \eps 3$ für $n>N_{\eps}$.
		Setze jetzt $n:= N_{\eps}+1$.
		\fixme[übersichtlicher machen]		
	\end{proof}
\end{st}

\begin{df}
	\label{df:1.13}
	Eine Reihe $\sum_{n=0}^\infty a_n$ heißt \emph{absolut konvergent}, falls $\sum_{n=0}^\infty |a_n|$ konvergiert.
\end{df}

\begin{thm}[Weierstraß-Kriterium]
	\label{thm:1.14}
	Ist $\sum_{n=0}^\infty a_n$ mit $a_n \ge 0$ konvergent und gilt $f_n:M\to \C$ und $f_n(z)| \le a_n$, so ist die Reihe
	\[
		\sum_{n=0}^\infty f_n(z)
	\]
	gleichmäßig konvergent auf $M$ und absolute konvergent für $z\in M$.
\end{thm}

\begin{ex}
	$M:= \_{K_2(0)} \subset \C$, $f_n(z) := \f {z^n}{(n+1)^22^n}$.
	Wähle $a_n:= \f 1{(n+1)^2}$, dann folgt
	\[
		\begin{cases}
			\sum_{n=0}^\infty \f 1{(n+1)^2} < \infty \\
			|f_n(z)| \le a_n
		\end{cases}
	\]
	Also ist $g(z) = \sum_{n=0}^\infty f_n(z)$ nach \ref{thm:1.14} und \ref{st:1.12} stetig auf $\_{K_2(0)}$.
\end{ex}

\begin{df}[Potenzreihen]
	Sei $(a_n)$ eine Folge in $\C$
	\[
		R := \f 1{\limsup_{n\to \infty}\sqrt[n]{|a_n|}} \qquad (\f 10 := \infty, \f 1\infty := 0)
	\]
	dann konvergiert die Potenzreihe
	\[
		f(z) = \sum_{n=0}^\infty a_n(z-z_0)^n
	\]
	für $|z-z_0|<R$ und divergiert für $|z-z_0|>R$.
	Sei konvergiert gleichmäßig auf jedem Kreis $\_{K_r(z_0)}$ mit $0<r<R$.
	Insbesondere ist $f$ stetig auf $K_R(z_0)$.

	Falls $(|\f {a_{n+1}}{a_n}|)$ konvergiert, gilt
	\[
		R = \f 1{\lim |\f{a_{n+1}}{a_n}|}
	\]
\end{df}

\begin{df}
	Wir definieren
	\begin{align*}
		e^z &:= \sum_{n=0}^\infty \f {z^n}{n!}\\
		\cos z &:= \sum_{n=0}^\infty (-1)^n \f {z^2n}{(2n)!} \\
		\sin z &:= \sum_{n=0}^\infty (-1)^n \f{z^{2n+1}}{(2n+1)!}
	\end{align*}
	\begin{proof}
		Für die Konvergenz:
		\begin{align*}
			\f {a_{n+1}}{a_n} = \f {\f1{(n+1)!}}{\f 1{n!}} = \f {1}{n+1} \to 0
		\end{align*}
		Also $R=\infty$, die Potenzreihe konvergiert auf ganz $\C$.		
	\end{proof}
\end{df}

\begin{thm}[Cauchy-Produkt von Reihen]
	\label{thm:1.18}
	Ist $\sum_{n=0}^\infty a_n$ absolut konvergent, $\sum_{n=0}^\infty b_n$ konvergent in $\C$, so gilt
	\[
		\left( \sum_{n=0}^\infty a_n\right) \left(\sum_{n=0}^\infty f_n\right) = \sum_{n=0}^\infty \sum_{k=0}^n a_k b_{n-k}
	\]
	\begin{proof}
		Setze $A:= \sum_{n=0}^\infty$, $B:= \sum_{n=0}^\infty b_n, b_l := \sum_{n=0}^l b_n$.
		\begin{align*}
			\sum_{n=0}^N \sum_{k=0}^n a_k b_{n-k}
				&= \sum_{n=0}^N a_n \sum_{m=0}^{N-n} b_m\\
				&= \sum_{n=0}^N a_n (B_{N-m}-B + B)\\
				&= \sum_{l=0}^N a_{N-l} (B_l -B) + \underbrace{\sum_{n=0}^N a_nB}_{\to AB}\\
		\end{align*}
		Weiterhin ist für entsprechendes $N_\eps$
		\begin{align*}
			\left|\sum_{l=0}^N a_{N-l} (B_l-B) \right|
				&= \left|\sum_{l=0}^{N_\eps} a_{N-l}(B_l-B) + \sum_{l=N_\eps +1}^N a_{N-l}(B_l-B)\right|\\
				&\le \left|\max_{0\le l \le N_\eps} |B_l-B|\sum_{n=N-N_\eps}^\infty |a_n| \right| +     \left|\sup_{l\ge N_{\eps}+M} |B_l-B| \sum_{l=N_\eps+1}^N |a_{N-l}| |\right| \qquad n= N-l, \text{wähle hier $N_\eps$}
				&\le \f \eps 2  +  \left|\sup_{l\ge N_{\eps}+M} |B_l-B| \sum_{n=0}^\infty |a_{n}| \right|
				&<     \eps
		\end{align*}
		für $N-N_\eps > \tilde N_\eps$ bzw. für  $N>\tilde N_\eps + N_\eps$.
	\end{proof}
\end{thm}

\begin{prop}
	\label{prop:1.19}
	Es gilt
	\[
		e^{z+w}  = e^z\cdot e^w
	\]
	\begin{proof}
		\begin{align*}
			e^ze^w &= \left(\sum_{n=0}^\infty \f{z^n}{n!}\right) \left( \sum_{n=0}^\infty \f {w^n}{n!}\right) \\
				&= \sum_{n=0}^\infty \sum_{k=0}^n \f {z^k}{k!} \f {w^{n-k}}{(n-k)!} \\
				&= \sum_{n=0}^\infty \f 1{n!} \sum_{k=0}^n \binom{n}{k}z^k w^{n-k} \\
				&= \sum_{n=0}^\infty \f {(z+k)^n}{n!} = e^{z+w}
		\end{align*}
	\end{proof}
\end{prop}

\begin{nt}
	\label{1.20}
	Aus der Taylorreihe folgt außerdem, dass $e^z, \sin z, \cos z$ für $z\in \R$ die selben Funktionen sind, wie aus der Schule bekannt.
\end{nt}

\begin{prop}
	\label{1.21}
	\begin{enumerate}[1)]
		\item 
			$e^{iz} = \cos z + i \sin z$ für $z\in \C$ (Eulersche Formel)
			\begin{proof}
				\[
					e^{iz} = \sum_{n=0}^\infty \f {(iz)^n}{n!} = \sum_{k=0}^\infty \f {(iz)^{2k}}{(2k)!} + \sum_{k=0}^\infty \f {(iz)^{2k+1}}{(2k+1)!} =
\sum_{k=0}^\infty (-1)^k\f {z^{2k}}{(2k)!} + \sum_{k=0}^\infty \f (-1)^k {z^{2k+1}}{(2k+1)!}
\]
			\end{proof}
		\item
			Mit der Polardarstellung $z = r(\cos \phi + i \sin \phi) = r e^{i\phi}$ gilt
			\begin{align*}
				z^n &= r^n e^{in\phi}\\
				z_1 z_2 &= r_1 r_2 e^{\phi_1 + \phi_2}\\
				\f {z_1}{z_2} &= \f {r_1}{r_2} e^{i(\phi_1-\phi_2)}
			\end{align*}
	\end{enumerate}
\end{prop}

\begin{df}
	\label{df:1.22}
	\begin{enumerate}[1)]
		\item 
			$f: O \to \C$ heißt \emph{differenzierbar} in $z_0\in O$, falls
			\[
				f'(z_0 := \lim_{\substack{z\to z_0 \\ z\neq z_0}}\f {f(z) -f(z_0)}{z-z_0}
			\]
			existiert.
			$f'(z_0)$ heißt \emph{Ableitung} von $f$ in $z_0$.
		\item
			$f$ heißt \emph{differenzierbar}, falls $f$ in jedem $z_0\in O$ differenzierbar ist.
			$f':O\to \C$ heißt \emph{Ableitung(sfunktion)} von $f$.
	\end{enumerate}
\end{df}

\begin{ex*}
	\begin{enumerate}[1)]
		\item 
			$f: \C \to \C : z\mapsto c$, dann ist $f'=0$.
		\item
			$f: \C \to \C : z\mapsto z$, dann ist $f'=1$.
		\item
			ohne Beweis: $f:\C to \C :z\mapsto z^m$, dann ist $f'(z) = nz^{n-1}$.
		\item
			$f: \C\to \C : z\mapsto |z|^2$ ist in $z_0\neq 0$ nicht differenzierbar.
			\begin{proof}
				Sei $z_0 = x_0+iy_0 \neq 0$ und $z_h:= x_0+h +iy_0$, dann ist
				\[
					\f {|z_h|^2 -|z_0|^2}{z_h-z_0} = \f {2hx_0+h^2}h \to 2x_0
				\]
				Sei jetzt $z_h := x_0 + i(y_0+h)$, dann ist
				\[
					\f {|z_h|^2 -|z_0|^2}{z_h-z_0} := \f {2hx_0+h^2}{ih} \to -2ix_0
				\]
			\end{proof}
	\end{enumerate}
\end{ex*}
\section{Vektoranalysis}


\end{document}
