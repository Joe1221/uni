\documentclass[a4paper,10pt]{scrbook}
\usepackage{mathe-vorlesung}
\usepackage{legacy}
\newcommand{\coloneq}{:=}

\title{Analysis 3}

\begin{document}

\maketitle

\tableofcontents
\newpage


\chapter{Grundlagen}

\section{Definitionen und Notationen}

\begin{df}[Multiindex und partielle Ableitung] \label{1.1}
	Sei $u: \R^d \to \R$ hinreichend oft differenzierbar.
	Wir nennen $\beta = (\beta_1, \dotsc, \beta_d)^T \in \N_0^d$ mit $k := |\beta| := \sum_{i=1}^d \beta_i$ einen \emphdef[Multiindex]{Multiindex der Ordnung $k$}.

	Wir definieren
	\[
		\partial^\beta u := (\pddx[x_1])^{\beta_1} \dotsb (\pddx[x_d])^{\beta_d} u,
	\]
	die \emphdef[partielle Ableitung]{partielle Ableitung von $u$ zum Index $\beta$}.

	Sei $\mathbb{B}_k := \Set{\beta \in \N_0^d & |\beta| = k}$ die Menge aller Multiindizes der Ordnung $k$ und
	\[
		\D^k u := (\partial^\beta u)_{\beta \in \mathbb{B}_k}
	\]
	der Vektor aller partieller Ableitungen der Ordnung $k$ (mit beliebiger Reihenfolge).
\end{df}

\begin{df}[Ableitungsoperatoren] \label{1.2}
	Für $u: \R^d \to \R$ hinreichend oft differenzierbar definieren wir den \emphdef[Gradient]{Gradienten}
	\[
		\grad u(x) := \nabla u(x) := \Vector{ \partial_{x_1} u(x) & \dots & \partial_{x_d} u(x) },
	\]
	wobei $x = (x_1, \dotsc, x_d)$ und $\partial_{x_i} := \pddx[x_i]$ für $i = 1, \dotsc, d$.

	Für ein hinreichend oft differenzierbares Vektorfeld $v: \R^d \to \R^d$ definieren wir die \emphdef{Divergenz} durch
	\[
		\div v(x) := \nabla \cdot v(x) = \sum_{i=1}^d \partial_{x_i} v_i (x)
	\]
	und im Fall $d = 3$ zusätzlich die \emphdef{Rotation} durch
	\[
		\rot v(x) := \nabla \times v(x) = \Vector{ \partial_{x_2} v_3 - \partial_{x_3} v_2 & \partial_{x_3} v_1 - \partial_{x_1} v_3 & \partial_{x_1} v_2 - \partial_{x_2} v_1 }.
	\]
	Wir nutzen die Abkürzung $\partial_{x_i}^2 := (\partial_{x_i})^2$ und definieren den \emphdef{Laplace-Operator} durch
	\[
		\Laplace u(x) := \nabla \cdot (\nabla u) = \div( \grad  u(x) ) = \sum_{i=1}^d \partial_{x_i}^2 u(x).
	\]
	Skalare Operatoren werden für vektorielle Funktionen komponentenweise definiert, z.B.
	\[
		\Laplace v(x) := \Vector{\Laplace v_1(x) & \dots & \Laplace v_d(x)},
	\]
	und für $b \in \R^d$
	\[
		(b \cdot \nabla) v := \Big(\sum_{i=1}^d b_i \partial_{x_i}\Big) v
		= \Vector { \sum_{i=1}^d b_i \partial_{x_i} v_1 & \dots & \sum_{i=1}^d b_i \partial_{x_i} v_d }.
	\]
\end{df}

\begin{df}[Räume stetig differenzierbarer Funktionen] \label{1.3}
	Sei $\Omega \subset \R^d$ offen und beschränkt.
	Wir bezeichnen mit $C^m(\_\Omega, \R^n)$ den Raum der $m$-mal stetig differenzierbaren Funktionen (differenzierbar auf $\Omega$, sodass die $m$-ten Ableitungen stetig auf $\_\Omega$ fortsetzbar sind) von $\_\Omega$ nach $\R^n$.

	Für $n = 1$ schreiben wir auch kurz $C^m(\_\Omega) = C^m(\_\Omega, \R^1)$ und definieren hier für $u \in C^0(\_\Omega)$ die \emphdef{Supremumsnorm}
	\[
		\|u\|_\infty := \sup_{x\in \_\Omega} = |u(x)|.
	\]
	Auf $C^m(\_\Omega)$ definieren wir damit eine Norm:
	\[
		\|u\|_{C^m(\_\Omega)} := \sum_{|\beta| \le m} \|\partial^\beta u\|_\infty,
	\]
	wobei $u \in C^m(\_\Omega)$.
	\begin{note}
		\begin{itemize}
			\item
				$C^m(\_\Omega)$ ist ein Banachraum, d.h. ein vollständiger, normierter Raum (Alt, Lemma 1.8)
			\item
				Man kann auch $C^m(\Omega)$ für offenes oder potentiell unbeschränktes $\Omega$ und auch $m = \infty$ definieren.
				Statt einer Norm wird dann eine Metrik (Frechét-Metrik) eingeführt, bzgl. der $C^m(\Omega)$ immernoch vollständig ist (Alt, Abschnitt 1.6).
		\end{itemize}
	\end{note}
\end{df}

\begin{df}[$L^p$-Räume] \label{1.4}
	Für $p \in [1, \infty)$ definieren wir
	\[
		\tilde L^p(\Omega) := \Set{ u: \Omega \to \R \text{ Lebesgue-messbar} & \big(\mathsmaller{\int}_\Omega |u|^p \big)^{\f 1p} < \infty}
	\]
	mit Seminorm
	\[
		\|u\|_p := \|u\|_{L^p(\Omega)}  := \Big(\int_\Omega |u|^p \Big)^{\f 1p}.
	\]
	Für $p = \infty$ definieren wir
	\[
		\tilde L^\infty(\Omega) := \Set{ u: \Omega \to \R \text{ Lebesgue-messbar} & \esssup_{x\in \Omega} |u(x)| < \infty }
	\]
	mit $\|u\|_\infty := \|u\|_{L^\infty(\Omega)} := \esssup_{x\in \Omega} |u(x)|$.

	Sei $\sim$ die Äquivalenzrelation auf $\tilde L^p(\Omega)$ via
	\[
		u \sim v \defiff \exists N \subset \Omega \text{ Nullmenge} : \forall x \in \Omega \setminus \N : u(x) = v(x).
	\]
	Dann definieren wir $L^p(\Omega)$ als die Menge der Äquivalenzklassen
	\[
		L^p(\Omega) := \tilde L^p(\Omega) / \sim.
	\]
	$\|u\|_p$ ist auf jeder einzelnen Äquivalenzklasse konstant und ist daher ohne weiteres auf $L^p(\Omega)$ erweiterbar.
	Wir nennen $(L^p(\Omega), \|\argdot\|_p)$ \emphdef[$L^p$-Raum]{normierter $L^p$-Raum}.
	\begin{note}
		\begin{itemize}
			\item
				$L^p(\Omega)$ ist vollständig bezüglich $\|\argdot\|_p$, also ein Banachraum (Alt, Lemma 1.1, Satz 1.14),
			\item
				Elemente von $L^p(\Omega)$ sind also Äquivalenzklassen von Funktionen, die sich nur auf einer Nullmenge unterscheiden.
				Konsequente Unterscheidung zwischen Funktionen und Äquivalenzklassen wäre mühsam.
				Daher folgende praktische Konvention: $u \in L^p(\Omega)$ soll heißen $u \in U \in L^p(\Omega)$ für eine geeignete Äquivalenzklasse $U$ mit $u: \Omega \to \R$ als Repräsentant von $U$.
			\item
				Wir nennen $L^p(\Omega)$ trotz der Äquivalenzklassen einen \emph{Funktionenraum}.
				Beim Arbeiten mit $L^p$-Räumen muss jedoch immer bedacht/hinterfragt werden, ob die betrachteten Operationen sinnvoll definiert sind, d.h. unabhängig vom Repräsentanten sind.
				Beispielsweise ist die Punktauswertung $u(x)$ nicht wohldefiniert, eine Mittelung über alle Funktionswerte jedoch schon.
\Timestamp{2014-10-17}
			\item
				$(u,v) := \<u,v\>_{L^(\Omega)} := \int_\Omega uv$ ist ein Skalarprodukt auf $L^2(\Omega)$ und $\|u\|_2 = \sqrt{(u,v)}$, also $L^2(\Omega)$ ist vollständig bezüglich einer aus einem Skalarprodukt induzierten Norm, also ein sogenannter \emphdef{Hilbertraum}.
			\item
				Zu einem Banachraum $V$ ist der \emph{Dualraum} $V'$
				\[
					V' := \Set{ \phi: V \to \R & \phi \text{ linear und stetig} }
				\]
				mit der induzierten Norm
				\[
					\|\phi\|_{V'} := \sup_{u\in V\setminus \Set 0} \frac{\phi((u)}{\|u\|_V}
				\]
				wieder ein Banachraum.
			\item
				Für $1 < p,q < \infty$ mit $\f 1p + pq = 1$ ist $L^p(\Omega)$ ist isomorph zu $(L^q(\Omega))'$.
			\item
				$L^2(\Omega)$ ist alsowegen $\f 12 + \f 12 = 1$ isomorph zu $L^2(\Omega)'$.
		\end{itemize}
	\end{note}
\end{df}

\begin{df}[lokal integrierbare Funktionen] \label{1.5}
	Wir definieren
	\[
		L^1_{\text{loc}}(\Omega) :=
		\Set{ u : \Omega \to \R \text{ lebesgue-messbar} & \forall  K \subset \Omega \text{ kompakt } : \int_k |u(x)| \di < \infty }
	\]
\end{df}

\begin{ex*}
	\begin{itemize}
		\item
			$L^1(\Omega) \subset L^1_{\text{loc}}(\Omega)$.
			Aus $u(x) = 1, x \in \Omega := \R$ folgt $u \not\in L^1$, aber $u \in L^1_{\text{loc}}(\Omega)$.
	\end{itemize}
\end{ex*}

\begin{df}[Funktionen mit kompaktem Träger] \label{1.6}
	Wir definieren für $\Omega \subset \R^d$ offen (möglicherweise nubeschränkt), $m \in \N_0 \cup \Set \infty$.
	\[
		C_0^m(\Omega)
		:= \Set{ u \in C^m(\Omega) & \supp(u) \subset \Omega \text{ ist beschränkt} },
	\]
	wobei $\supp(u) := \_{\Set{x \in \Omega & u(x) \neq 0}}$ (also abgeschlossen) den \emphdef{Träger} (engl. “support”) von $u$ bezeichnet.
\end{df}

\begin{st}[Fundamentalsatz der Variatonsrechnung] \label{1.7}
	Sei $u \in L^1_{\text{loc}}, \Omega \in \R^d$ offen. Dann sind äquivalent
	\begin{enumerate}[i)]
		\item
			$\forall v \in C_0^\infty(\Omega) : \int_\Omega uv = 0$
		\item
			$u = 0$ fast überall in $\Omega$.			
	\end{enumerate}
	\begin{proof}
		2.11 in Alt.
	\end{proof}
\end{st}

\begin{df}[Skalare PDE] \label{1.8}
	Sei $F: \R^{|\mathbb{B}_k} \times \R^{|B_{k-1}} \times \dotsb \times \R^d \times \R \times \Omega \to \R$ gegeben.
	Dann ist
	\[ \label{eq:1.1}
		F(\D^k u(x), D^{k-1} u(x), \dotsc, D^1 u(x), u(x), x) = 0,
		\qquad x \in \Omega
	\]
	eine \emphdef[partielle Differentialgleichung!skalare]{skalare partielle Differentialgleichung der Ordnung $k$} für eine unbekannte Lösung $u: \Omega \to \R$.
\end{df}

\begin{df}[Lineare/Nichtlineare PDE] \label{1.9}
	Die PDE \eqref{1.1} ist
	\begin{enumerate}[i)]
		\item
			\emphdef{linear}, falls sie die Form $\sum_{|\beta| \le k} a_\beta(x) \partial^\beta u(x) = f(x)$ für Multiindex $\beta \in \N_0^d$ und gegebenen Funktionen $a_\beta, f$ besitzt.
			Die PDE heißt \emphdef{homogen}, falls $f(x) = 0$, sonst \emphdef{inhomogen}.
		\item
			\emphdef{semilinear}, falls sie die Form
			\[
				\sum_{|\beta| = k} a_\beta(x) \partial^\beta u(x) + a (D^{k-1} u(x), \dotsc, D^1 u(x), u, x) = 0
			\]
			besitzt.
		\item
			\emphdef{quasilinear}, falls sie die Form
			\[
				\sum_{|\beta| = k} a_\beta(D^{k-1}u, \dotsc, D^1 u, u, x) \partial^\beta uu(x) + a(D^{k-1} u(x), \dotsc, u, x) = 0
			\]
			besitzt.
		\item
			\emphdef{voll nichtlinear} falls sie die nichtlinear von $D^k$ abhängt.
	\end{enumerate}
	\begin{note}[Systeme]
		Ein System von PDEs ist eine Sammliung mehrerer skalarer PDEs für mehrere unbekannte Funktionen $u = (u_1, \dotsc, u_n)^T$.
		Typischerweise sind die einzelnen PDEs miteinander gekoppelt und die Anzahl der Gleichungen und der Unbekannten stimmen überein. 
	\end{note}
	\begin{note}[zeitäbhängige Probleme]
		Alle Notationen und Definitionen für $\Omega \subset \R^d$ mit $x = (x_1, \dotsc, x_d)^T \in \Omega$ erweitern wir auf Orts-Zeit-Zylinder $\Omega_T := \Omega \times (0, T) \subset \R^d \times \R$ mit $(x,t) \in \Omega_T$, $T \in \R^+ \cup \Set \infty$.
		Ortsvariable $x$, Zeitvariable $t$.
		Insbesondere $\partial_t := \pddx[t], \partial_t^2 := \pddx[t^2]$.
		Dann bezeichnet für $u \in C^1(\Omega_t)$
		\begin{align*}
			\nabla_x u(x) &:= \Vector{ \partial_{x_1} u(x) & \dots & \partial_{x_d} u(x) }, \\
			\Laplace_x u(x) &:= \sum_{i=1}^d \partial_{x_i}^2 u(x)
		\end{align*}
		Falls keine Verwechslungsgefahr besteht, lässt man $x$ bei $\nabla_x, \Laplace_x$ auch weg.
	\end{note}
\end{df}

\begin{ex}[Lineare PDEs] \label{1.10}
	Einige häufig begegneten PDEs (Koeffizienten der Einfachheit halber $1$).
	\begin{itemize}
		\item
			\emphdef{Laplace-Gleichung}: $-\Laplace u = 0$,
		\item
			\emphdef{Poisson-Gleichung}: $-\Laplace u = f$,
		\item
			\emphdef{Helmholtz-Gleichung}: $-\Laplace u - \lambda u = 0$ für $\lambda > 0$,
		\item
			\emphdef{Advektions-Gleichung}: $\partial_t u + b \cdot \nabla u = 0$ für $b \in \R^d$
		\item
			\emphdef{Wärmeleitungs-Gleichung} oder \emphdef{Diffusionsgleichung}: $\partial_t u - \Laplace u = 0$.
		\item
			\emphdef{Schrödinger-Gleichung}: $i \partial_t u + \Laplace u = 0$, wobei $i = \sqrt{-1} \in \C$.
		\item
			\emphdef{Wellengleichung}: $\partial_t^2 u - \Laplace u = 0$,
		\item
			\emphdef{Airy's-Gleichung}: $\partial_t u + \partial_x^3 u = 0$
		\item
			\emphdef{Balken-Gleichung}: $\partial_t u + \partial_x^4 u = 0$.
		\item
			\emphdef{Allgemeine Diffusions-Advektions-Reaktions-Gleichung}:
			\[
				- \div \cdot (A \nabla u)0+ b \cdot \nabla u + c u = f,
			\]
			wobei $A \in \R^{d\times d}, b \in \R^d, c \in \R$.
			% fixme: bezeichnung : Diffusion, Advektion, Reaktions, Quellterm
	\end{itemize}
\end{ex}

\begin{ex}[Nichtlineare PDEs] \label{1.11}
	\begin{itemize}
		\item
			\emphdef{Nichtlineare Poission-Gleichung}: $-\Laplace u = f(u)$,
		\item
			\emphdef{$p$-Laplace-Gleichung}: $\div (|\nabla u|^{p-2} \nabla u) = 0$,
		\item
			\emphdef{Minimalflächen-Gleichung}: $\div ( \f{\nabla u}{\sqrt{1 + |\nabla u|^2}} ) = 0$,
		\item
			\emphdef{Hamilton-Jacobi-Gleichung}: $\partial_t u + H(\nabla u, u) = 0$,
		\item
			\emphdef{Burgers-Gleichung}: $\partial_t u + \partial_x(\f 12 u^2) = 0$,
		\item
			\emphdef{skalare Erhaltungsgleichung}: $\partial_t u + \nabla \cdot(f(u, \nabla u)) = 0$,
		\item
			\emphdef{Korteweg de Vries Gleichung (KdV)}: $\partial_t u + u \partial_x u + \partial_x^3 u = 0$,
		\item
			\emphdef{allgemeine Transport-Reaktions-Gleichung}: $\partial_t u + \div(f(u, \nabla u)) = g(u)$.
	\end{itemize}
\end{ex}

\begin{ex}[Lineare Systeme] \label{1.12}
	\begin{itemize}
		\item
			\emphdef{Maxwell-Gleichungen}:
			\begin{align*}
				\partial_t E &= \rot B, \\
				\partial_t B &= - \rot E, \\
				0 &= \div B = \div E.
			\end{align*}
		\item
			\emphdef{Oseen-Gleichungen}:
			\begin{align*}
				(b \cdot \nabla) u - \mu \Laplace u + \nabla p &= 0, \\
				\div u &= 0.
			\end{align*}
			Für $b = 0$ sind dies die \emphdef{Stokes-Gleichungen}.
		\item
			\emphdef[Poission-Gleichung!gemischte Formulierung]{gemischte Formulierung der Poission-Gleichung}:
			\begin{align*}
				\div v &= f, \\
				v + \nabla u &= 0
			\end{align*}
	\end{itemize}
\end{ex}

\begin{ex}[Nichtlineare Systeme] \label{1.13}
	\begin{itemize}
		\item
			System von Erhaltungsgleichungen
			\[
				\partial_t u + \div(F(u)) = 0
			\]
			für $F: \R^d \to \R^d$
		\item
			\emphdef{Navier-Stokes-Gleichungen}
			\begin{align*}
				\partial_t u + (u\cdot \nabla) u - \mu \Laplace u + \nabla p &= 0
				\div u &= 0
			\end{align*}
			Für $\mu = 0$ ergeben sich die \emphdef{Euler-Gleichungen} für ein nichtviskoses, inkompressibles Fluid.
	\end{itemize}
\end{ex}


\section{Klassifikation linearer PDEs zweiter Ordnung}


\begin{df}[linearer Differentialoperator zweiter Ordnung] \label{1.14}
	Sei $\Omega \subset \R^d$ offen, $A = (a_{ij})_{i,j = 1}^d \in C^0 (\Omega, \R^{d\times d}), b = (b_i)_{i=1}^d \in C^0(\Omega)^d, c \in C^0(\Omega)$.
	Dann nennen wir $\scr L: C^2(\Omega) \to C^0(\Omega)$ mit
	\[ \label{eq:1.2}
		(\scr L u)(x) := - \sum_{i,j=1}^d a_{ij}(x) \partial_{x_i} \partial_{x_j} u(x) + \sum_{i=1}^d b_i \partial_{x_i} u(x) + c(x) u(x)
	\]
	\emphdef[Differentialoperator!allgemein, linear, zweiter Ordnung]{allgemeiner linearer Differentialoperator zweiter Ordnung}.
	\begin{note}
		\begin{itemize}
			\item
				$\scr L$ erfasst die Differential-Operatoren in Laplace-, Poisson-, Helmholtz, Wärmeleitungs-, Diffusions  und allgeimener Diffusions-Advektions-Reaktionsgleichung aus \ref{1.10}.
			\item
				Zu $f \in C^0(\Omega)$ ergibt sich eine entsprechende PDE
				\[ \label{eq:1.3}
					\scr L u(x) = f(x), x \in \Omega.
				\]
			\item
				Wir nennen $-\sum_{i,j=1}^d a_{ij}(x) \partial_{x_i} \partial_{x_j} u$ \emphdef{Hauptteil} von $\scr L$.
			\item
				Ohne Einschränkung kann $A$ als symmetrisch vorausgesetzt werden, denn $\partial_{x_i} \partial_{x_j} u = \partial_{x_j} \partial_{x_i} u$.
				Falls $A$ nicht symmetrisch ist, so ergibt $A_s := \f 12 (A + A^T)$ identisches $\scr L$. \Exercise
				$A$ hat somit ohne Einschränkung nur reelle Eigenwerte.
			\item
				Auch geläufig ist die sogennante \emphdef{Divergenzform}, welche man bei differenzierbaren $a_{ij}, b_i$ leicht in obige Form bringen kann.
				Sei
				\begin{align*}
					(\_{\scr L} u)(x) &:= - \nabla \cdot (A(x) \cdot \nabla u) + \nabla \cdot (b \cdot u) + c u \\
					&= - \sum_{i,j=1}^d \partial_{x_i} (a_{ij} \partial_{x_j} u) + \sum_{i=1}^d \partial_{x_i} (b_i u) + cu \\
					&= -\sum_{i,j=1}^d a_{ij} \partial_{x_i} \partial_{x_j} u - \sum_{j=1}^d \sum_{i=1}^d (\partial_{x_i} a_{ij}) \partial_{x_j} + \sum_{i=1}^d b_i \partial_{x_i} u + \sum_{i=1}^d (\partial_{x_i} b_i) \cdot u + cu \\
					&= %fixme
				\end{align*}
				mit der Wahl
				\[
					\tilde b_i := b_i - \sum_{j=1}^d (\partial_{x_j} a_{ji})
					%fixme: \tilde c_i
				\]
		\end{itemize}
	\end{note}
\end{df}

\begin{df}[Klassifikation] \label{1.15}
	Der Operator $\scr L$ aus \eqref{eq:1.2} ist
	\begin{itemize}
		\item
			\emphdef{elliptisch} in $x$, falls alle Eigenwerte von $A(x)$ positiv,
		\item
			\emphdef{parabolisch} in $x$, falls $d-1$ Eigenwerte von $A(x)$ positiv, ein Eigenwert Null ist, aber $\rg([A(x),b(x)]) = d$.
		\item
			\emphdef{hyperbolisch} in $x$, falls $d-1$ Eigenwerte von $A(x)$ positiv und ein Eigenwert negativ ist.
	\end{itemize}
	$\scr L$ \emphdef{elliptisch}, \emphdef{parabolisch}, bzw. \emphdef{hyperbolisch}, wenn er es in jedem $x \in \Omega$ ist.
	Die PDE \eqref{eq:1.3} ist \emphdef{elliptisch}, \emphdef{parabolisch}, bzw. \emphdef{hyperbolisch}, wenn $\scr L$ dies ist.
	\begin{note}
		\begin{itemize}
			\item
				Die Begriffe sind motiviert aus Kegelschnitten/Quadriken:
				\[
					z^T A(x) z = 1
				\]
				beschreibt unter den genannten Voraussetzungen ein Ellipsoid/Paraboloid/Hyperboloid.
		\end{itemize}
	\end{note}
\end{df}




\chapter{Finite Differenzen Verfahren für elliptische Probleme} \label{chap:2}



\section{Finite Differenzen für Poisson-Gleichung}


\begin{df}[Finite Differenz] \label{2.1}
	Sei $h \in \R$, $e_j \in \R^d$ Einheitsvektor für $j = 1, \dotsc, d$ und $u: \Set{x, x \pm e_j h & j = 1, \dotsc, d} \to \R$, dann definieren wir \emphdef[Vorwärtsdifferenz]{Vorwärts-} oder \emphdef{rechtsseitige Differenz} durch
	\[
		(\partial_{x_j}^{+h}u)(x) := \frac{u(x+he_j) -u(x)}{h}
	\]
	analog \emphdef[Rückwärtsdifferenz]{Rückwärts-} oder \emphdef{linksseitige Differenz} durch
	\[
		(\partial_{x_j}^{-h}u)(x) := \frac{u(x) - u(x-he_j)}{h}
	\]
	und \emphdef[symmetrische Differenz]{symmetrische} oder \emphdef{zentrale Differenz} durch
	\[
		(\partial_{x_j}^{ch} u)(x)
		:= \f 12 (\partial_{x_j}^{+h} u(x) + \partial_{x_j}^{-h} u(x))
		= \frac{u(x+he_j) - u(x-he_j)}{2h}
	\]
\end{df}


\begin{st}[Approximationsgüte] \label{2.2}
	Sei $u: \Omega \to \R, x \in \Omega \subset \R^d, r \in \N^+$ mit $B_r(x) \subset \Omega$.
	Für $h < r$ gilt dann
	\begin{enumerate}[i)]
		\item
			$|\partial_{x_j} u(x) - \partial_{x_j}^{\pm h} u(x)| \le \f h2 \|\partial_{x_j}^2 u\|_\infty$ für $u \in C^2(\_\Omega)$
		\item
			$|\partial_{x_j} u(x) - \partial_{x_j}^{c, h} u(x)| \le \f {h^2}6 \|\partial_{x_j}^3 u\|_\infty$ für $u \in C^3(\_\Omega)$
		\item
			$|\partial_{x_j}^2 u(x) - \partial_{x_j}^{-h} \partial_{x_j}^{+h} u(x)| \le \f {h^2}{12} \|\partial_{x_j}^4 u\|_\infty$ für $u \in C^4(\_\Omega)$
	\end{enumerate}
	\begin{proof}
		Es genügt dies für $d = 1$ zu zeigen, denn  sei $u \in C^k(\_\Omega), v(t) = u(x + th e_j)$.
		Dann ist $\|\ddx[t^k] v(t)\|_\infty \le \|\partial_{x_j}^k u\|_{C^0(\_\Omega)}$.
		\begin{enumerate}[i)]
			\item
				Taylor liefert für ein $\xi \in (x, x + h)$
				\[
					u(x+h) = u(x) + hu'(x) + \f {h^2}2 u''(\xi)
				\]
				Es folgt
				\[
					\partial_x u(x) - \partial_x^{+h} u(x)
					= u'(x) - \f{u(x+h) - u(x)}{h}
					= - \f h2 u''(\xi).
				\]
				Analog für die linksseitige Differenz.
			\item
				Subtraktion von
				\begin{align*}
					u(x+h) &= u(x) + hu'(x) + \f {h^2}2 u''(x) + \f {h^3}6 u'''(\xi) \\
					u(x-h) &= u(x) - hu'(x) + \f {h^2}2 u''(x) - \f {h^3}6 u'''(\_\xi) \\
				\end{align*}
				mit $\xi \in (x,x+h), \_\xi \in (x-h,x)$ liefert
				\[
					u(x+h) - u(x-h) = 2h u'(x) + \f{h^3}6 \big(u'''(\xi) + u'''(\_\xi)\big),
				\]
				also
				\[
					u'(x) - \f{u(x+h) - u(x-h)}{2h}
					= - \f{h^2}{12} \big( u'''(\xi) + u'''(\_\xi) \big)
					\le \f{h^2}6 \|u'''\|_\infty.
				\]
			\item
				Addition von
				\begin{align*}
					u(x+h) &= u(x) + hu'(x) + \f{h^2}2 u''(x) + \f {h^3}6 u'''(x) + \f{h^4}{24} u''''(\xi) \\
					-2u(x) &= -2u(x) \\
					u(x+h) &= u(x) - hu'(x) + \f{h^2}2 u''(x) - \f {h^3}6 u'''(x) + \f{h^4}{24} u''''(\_\xi) \\
				\end{align*}
				mit $\xi \in (x,x+h), \_\xi \in (x-h, x)$ liefert
				\begin{align*}
					\partial_x^{-h} \partial_x^{+h} u(x)
					&= \dfrac{\frac{u(x+h)-u(x)}{h}-\frac{u(x)-u(x-h)}{h}}{h} \\
					&= \frac{u(x+h) - 2u(x) +  u(x+h)}{h^2} \\
					&= \f 1{h^2} \big( \f {h^2}2 + \f {h^2}2 \big) u''(x) + \f 1{h^2} \f {h^4}{24} \big( u''''(\xi) + u''''(\_\xi) \big)
				\end{align*}
		\end{enumerate}
	\end{proof}
	\begin{note}
		\begin{itemize}
			\item
				Die Approximation in iii) ist also eine zweite zentrale Differenz
				\[
					\partial_{x_j}^{-h} \partial_{x_j}^{+h} u(x)
					= \partial_{x_j}^{+h} \partial_{x_j}^{-h} u(x)
					= \partial_{x_j}^{c, \f h2} \partial_{x_j}^{c, \f h2} u(x)
					= \f{u(x+h) - 2u(x) + u(x-h)}{h^2}.
				\]
			\item
				Aus dem Beweis folgt, dass $\partial_x^{-h} \partial_{x}^{+h} u(x) = u''(x)$ falls $u^{(4)} = 0$, z.B. für $u \in \P_3$.
			\item
				Man kann zentrale Differenzen für höhere Ableitungen verallgemeinern:
				\[
					\partial_{x_j}^{h,m} u(x)
					:= (\partial_{x_{j}}^{c, \f h2})^m u(x)
				\]
				falls $u: \Set{u+(k-\f m2) h e_j & k = 0, \dotsc, m } \to \R$.
				Dann ist
				\[
					\partial_{x_j}^{h,m} u(x)
					= \f 1{h^m} \sum_{k=0}^m \binom{m}{k} (-1)^{k+m} u\big( x + (k-\f m2) h e_j \big).
				\]
		\end{itemize}
	\end{note}
\end{st}

\begin{kor}[FD-Approximation für Laplace] \label{2.3}
	Sei $u : \Set{x, x \pm h e_j} \to \R$.
	Dann definieren
	\begin{equation} \label{eq:2.1}
		\Laplace_h u(x) :=
		\Big(\sum_{i=1}^d \partial_{x_j}^{-h} \partial_{x_j}^{+h} u \Big) (x)
	\end{equation}
	und es gilt unter Voraussetzungen von \ref{2.2}
	\[
		|\Laplace_h u(x) - \Laplace u(x)| \le C h^2
	\]
	für $u \in C^4(\_\Omega)$.
	\begin{proof}
		Dreiecksungleichung und \ref{2.2} iii) liefert
		\begin{align*}
			|\Laplace u(x) - \Laplace_h u(x)|
			&= \Big| \sum_{i=1}^d \partial_{x_j}^2 u(x) - \partial_{x_j}^{-h} \partial_{x_j}^{+h} u(x) \Big| \\
			&\le \sum_{i=1}^d | \partial_{x_j}^2 u(x) - \partial_{x_j}^{-h} \partial_{x_j}^{+h} u(x) \Big| \\
			&\le \sum_{j=1}^d \f {h^2}{12} \|\partial_{x_j}^{(4)} u\|_\infty \\
			&\le d \f {h^2}{12} \|u\|_{C^4(\_\Omega)}.
		\end{align*}
	\end{proof}
	\begin{note}
		\begin{itemize}
			\item
				Für $p(x) := \prod_{i=1}^d p_i(x_i)$ mit $p_i \in \P_3$ ist $\Laplace_h$ exakt, d.h. $\Laplace_h p(x) = \Laplace p(x)$.
		\end{itemize}
	\end{note}
\end{kor}


\begin{df}[Würfelgebiet] \label{2.4}
	Sei $\Omega \subset \R^d$ offen, beschränkt.
	$\Omega$ heißt \emphdef{Würfelgebiet} zu $h \in \R^+$, falls $Z \subset \Z^d$ sodass $\Omega = W \setminus \Boundary W =: \mathring W$ mit $W := \bigcup_{z \in Z} W(z)$ und $W(z) := [z_1h , (z_1+1)h] \times \dotsc \times [z_d h, (z_d+1)h] \subset \R^d$.
	\begin{note}
		\begin{itemize}
			\item
				Ist $\Omega$ ein Würfelgebiet zu $h$, dann ist $\Omega$ auch ein Würfelgebiet zu $\f hn$ für alle $n \in \N$.
		\end{itemize}
	\end{note}
\end{df}

Wir wollen uns im Folgenden nur mit Würfelgebieten beschäftigen

\begin{df}[FD-Gitter] \label{2.5}
	Sei $\Omega \subset \R^d$ ein Würfelgebiet zu $h \in \R^+$, $\Gamma := \Boundary \Omega$, also $\_\Omega = \Omega \cup \Gamma$.
	Wir definieren das \emphdef{Gitter} $\_\Omega_h$ durch \emphdef{innere Punkte} $\Omega_h$ und \emphdef{Randpunkt} $\Gamma_h$, wobei
	\begin{align*}
		\Omega_h &:= \Set{ x \in \Omega & \exists z \in \Z^d : x = hz } \\
		\Gamma_h &:= \Set{ x \in \Gamma & \exists z \in \Z^d : x = hz } \\
		\_\Omega_h &:= \Omega_h \cup \Gamma_h.
	\end{align*}
	\begin{note}
		\begin{itemize}
			\item
				Jeder innere Punkt hat genau $2d$ Nachbarn im Abstand von $h$ in $\_\Omega_h$
			\item
				Erweiterung für allgemeine Gebiete später.
		\end{itemize}
	\end{note}
\end{df}

\begin{df}[Gitterfunktionen] \label{2.6}
	Zu einem Gitter $\_\Omega_h$ definieren wir den Raum der \emphdef{Gitterfunktionen} $X_h := \Set{ v : \_\Omega_h \to \R }$
	und den Teilraum der Funktionen mit Nullrandwerten $X_h^0 := \Set{ v \in X_h & \forall x \in \Gamma_h : v(x) = 0 } \subset X_h$
	und den Raum der Funktionen auf inneren Punkten $Y_h := \Set{v: \Omega_h \to \R}$ mit Maximumsnorm $\|v\|_{\_\Omega_h} := \max_{x\in \_\Omega_h} |v(x)|$ und $\|v\|_{\Omega_h} := \max_{x\in\Omega_h} |v(x)|$.
\end{df}

\begin{nt*}[Nebenbemerkungen]
	\begin{itemize}
		\item
			Also sind $(X_h, \|\argdot\|_{\_\Omega_h}), (X_h^0, \|\argdot\|_{\Omega_h}), (X_h^ , \|\argdot\|_{\_\Omega_h}), (Y_h, \|\argdot\|_{\Omega_h})$ Banachräume, weil endlichdimensional und damit vollständig.
		\item
			Man kann auch $X_h$ mit einer Hilbertraumstruktur versehen, indem man das \emphdef[diskretes $l_2$-Skalarprodukt]{diskrete $L_2$-Skalarprodukt} definiert:
			\[
				\<u,v\>_{l_2} := h^d \sum_{x\in\_\Omega_h} u(x) v(x),
			\]
			welches die Norm $\|u\|_{l_,} := \sqrt{h^d \sum_{x\in\_\Omega} u(x)^2}$ induziert.
			Dann ist $X_h$ auch vollständig bezüglich $\|\argdot\|_{l_2}$, weiter gilt: $\lim_{h\to 0} \|u\|_{l_2} = \|u\|_{L^2(\Omega)}$ für $u \in C^0(\Omega)$.
		\item
			Man kann $X_h$ auch mit einer Seminorm versehen, welche auch die Ableitungen miteinbezieht
			\[
				|u|_{h_1}
				:= \Big( h^d \sum_{x\in\Omega} \sum_{j=1}^d  \big(\partial_{x_j}^{+h} u(x)\big)^2 \Big)^{\f 12},
			\]
			die \emphdef{diskrete $h_1$ Seminorm}.
			Dies ist eine Norm auf $X_h^0$, aber nicht auf $X_h$.
			Damit erhält man durch Kombination mit der diskreten $l_2$-Norm eine Norm auf $X_h$ („diskrete $h_1$ Norm“):
			\[
				\|u\|_{h_1}
				:= \sqrt{\|u\|_{l_2}^2 + |u|_{h_1}^2},
			\]
			bezüglich welcher $X_h$ ein Hilbertraum ist.
	\end{itemize}
\end{nt*}

Für $v \in X$ ist mit \eqref{eq:2.1} der Operator $-\Laplace_h v(x)$ für $x \in \Omega_h$ wohldefiniert, d.h. wir können $\Laplace_h : X_h \to Y_h$ als linearen Operator sehen.

\begin{df}[FD-Approximation für Poisson-RWP] \label{2.7}
	Sei ein Gitter $\_\Omega_h$ gegeben.
	Dann nennen wir $u_h \in X_h$ \emphdef{Finite Differenzen Lösung} des Poisson-RWPs aus \ref{1.23}, falls
	\begin{align} \label{eq:2.2}
		\Laplace_h u_h(x) &= f(x) && \text{$x \in \Omega_h$} \\
		u_h(x) &= g(x) && \text{$x\in \Gamma_h$}. \notag
	\end{align}
\end{df}

\begin{nt*}[Berechnung via LGS]
	\begin{itemize}
		\item
			Lege eine Aufzählung $\Set{x_1, \dotsc, x_n} = \Omega_h$ fest.
			Dann ist \eqref{eq:2.2} äquivalent zu einem LGS für Unbekannte $\underbar{u}_h = (u_i)_{i=1}^n$ mit $u_i = u_h(x_i)$ für $i=1,\dotsc, n$, denn $u_h(x)$ für $x \in \Gamma$ ist schon festgelet durch $g$.
		\item
			Sei FD-Operator in $x_i \in \Omega_h$ gegeben durch
			\[
				\Laplace_h u(x_i) = \sum_{j=1}^n \alpha_{ij} u(x_j) + \sum_{x\in \Gamma_h} \beta_{ix} u(x).
			\]
			Dann ist das LGS gegeben durch $A_h \underbar{u}_h = b_h$ mit $(A_h)_{ij} = \alpha_{ij}$ und $(b_h)_i = f(x_i) - \sum_{x\in \Gamma_h} \beta_{ix} g(x)$.
		\item
			$A_h$ ist dünn besetzt (sparse), da sie nur sehr wenige nichtnull-Einträge pro Zeile enthält.
	\end{itemize}
\end{nt*}

\Timestamp{2014-10-31}

\begin{ex} \label{2.8}
	Sei $d=1, \Omega = (0,1), n \in \N, h := \f 1{n+1}, x_i := i h, i \in \Set{0, \dotsc, n+1}$.
	Dann ist $\Omega_h := \Set{x_1, \dotsc, x_n}, \Gamma_h := \Set{x_0, x_{n+1}}$.
	Betrachte die Poisson-Gleichung
	\begin{align*}
		-u''(x) &= f(x) && \text{in $\Omega$} \\
		u(0) &= \alpha \\
		u(1) &= \beta
	\end{align*}
	Sei $u_i \approx u(x_i)$ für $i=1, \dotsc, n$, $u_0 = \alpha, u_{n+1} = \beta$.
	Diskretisierung ergibt
	\[
		- \f{u_{i+1} - 2u_i + 2u_{i-1}}{h^2} = f(x_i)
	\]
	für $i = 1, \dotsc, n$.
	Das LGS ergibt sich als
	\[
		\overbrace{\f 1{h^2} \underbrace{\Matrix{2 & -1 & & \\ -1 & \ddots & \ddots & \\ & \ddots & \ddots & -1 \\ & & -1 & 2}}_{\tilde A_n}}^{A_h}
		\Vector{u_1 & \dots & u_n}
		= \Vector{f(x_1) + \f{\alpha}{h^2} & f(x_2) & \dots & f(x_{n-1}) & f(x_n) + \f{\beta}{h^2}}.
	\]
	\begin{note}
		\begin{itemize}
			\item
				$A_h$ ist tridiagonal, symmetrisch, $A_h = \f 1{h^2} \tilde A_n$
			\item
				$A_h$ ist regulär, denn per Induktion folgt $\det \tilde A_{n} = n + 1$:
				\begin{proof}
					Für $n = 1$ ist $\det(\tilde A_n) = \det(2) = n+1$.
					Die Aussage gelte für $n-1$, es gilt
					\begin{align*}
						\det \tilde A_n
						&= \det \Matrix{\tilde A_{n-1} & & \\ & & -1 \\ & -1 & 2} \\
						&= 2 \det \tilde A_{n-1} - (-1) \det \Matrix{\tilde A_{n-2} & & \\ & & & \\ & -1 & -1} \\
						&= 2 n + (-1) \det A_{n-2}
						= 2n - (n-1)
						= n+1
					\end{align*}
				\end{proof}
			\item
				$A_h$ ist positiv definit, denn Gerschgorin liefert $\lambda_i(\tilde A_n) \in [0,4]$ und wegen Regularität $\lambda_i(\tilde A_n) \neq 0$.
				Es folgt also $\lambda_i > 0$ für $i = 1, \dotsc, n$.
			\item
				Also existiert eine eindeutige FD-Lösung für das Poisson-RWP in einer Dimension.
			\item
				Wegen $A_h$ symmetrisch, positiv definit kann das CG oder das PCG Verfahren zum iterativen Lösen des LGS verwendet werden.
			\item
				Man kann hier auch direkt stetige Abhängigkeit von den Daten beweisen:
				Seien $u, u_h$ Lösungen zu $f, \alpha, \beta$, bzw. $\_f, \_\alpha, \_\beta$.
				Dann existiert $C > 0$ unabhängig von $f, \_f, \alpha, \_\alpha, \beta, \_\beta$ sodass
				\[
					\|u_h - \_u_h\| \le C \Big( \|f-\_f\| + |\alpha - \_\alpha| + |\beta - \_\beta| \Big).
				\]
		\end{itemize}
	\end{note}
\end{ex}

\begin{ex} \label{2.9}
	Sei $d = 2, \Omega = (0,1)^2$ und betrachte das Poisson-RWP mit $g(x) = 0$, $m \in \N, h:= \f 1{m+1}, n := m^2$.
	Statt Einzelindex ist ein Doppelindex übersichtlicher, setze dazu
	\begin{align*}
		x_{ij} &= (ih, jh),&
		u_{ij} &\approx u(x_{ij}) \quad \text{für $0 \le i,j \le m+1$}.
	\end{align*}
	Die Diskretisierung des RWP liefert für jeden inneren Punkt eine Gleichung
	\[
		\f 1{h^2} \Big( 4 u_{ij} - u_{i-1,j} - u_{i+1,j} - u_{i,j-1} - u_{i,j+1}\Big)
		= f(ih, jh).
	\]
	Am Rand gilt
	\[
		u_{0,j} = u_{m+1,j} = u_{j,0} = u_{j,m+1}
	\]
	für $0 < i,j < m$.
	Für jede beliebige Wahl einer Aufzählung der $\Set{u_{ij}}$ erhält man ein System $A_h u_h = b_h$ mit $A_h$ symmetrisch, $\f 4{h^2}$ auf der Diagonalen und $-\f 1{h^2}$ an bis zu $4$ Einträgen pro Zeile/Spalte.

	Falls eine lexikographische Aufzahlung gewählt wird:
	\[
		\_u_h = \Big(u_{11}, u_{12}, \dotsc, u_{1m}, u_{21}, \dotsc, u_{2m}, \dotsc, u_{mm} \Big)
	\]
	so erhält $A_h$ eine Bandstruktur, jedoch nicht mehr tridiagonal wie in \ref{2.8}, sondern „Block-tridiagonal“:
	\begin{align*}
		A_h &= \f 1{h^2} \Matrix{B & C & & \\C & \ddots & \ddots & \\ & \ddots & \ddots & C \\ & & C & B}, &
		B &= \Matrix{4 & -1 & & \\ -1 & \ddots & \ddots & \\ & \ddots & \ddots & -1 \\ & & -1 & 4}, &
		C &= \Matrix{-1 & & \\ & \ddots & \\ & & -1},
	\end{align*}
	\begin{note}[Beliebige Gebiete]
		\begin{itemize}
			\item
				Falls $\Omega$ beschränkt ist, aber kein Würfelgebiet, muss das Gitter modifiziert werden, indem Schnittpunkte von $\Gamma$ mit den $h \Z^d$ Würfelkanten hinzugenommen werden:
				\[
					\Gamma_h := \Set{ x \in \Gamma & \exists z\in \Z^d, j =1,\dotsc, d : x \in z + \R e_j }
				\]
				Dann wird wie gehabt $\_\Omega_h := \Omega_h \cup \Gamma_h$ mit $\Omega_h$ aus \ref{2.5} definiert.
			\item
				% Rand in Süd/West richtung
				Koeffizienten der FD-Diskretisierung werden angepasst.
				Die Taylor-Entwicklung liefert
				\begin{align*}
					u''(x) &= \f{2}{h_W(h_O + h_W)} u(x-h_W) - \f 2{h_Oh_W} u(x) + \f 2{h_O(h_O+h_W)} u(x+h_O) \\
					&\quad + \LandauO(h) \\
					\Laplace(u) &= \f{2}{h_W(h_O+h_W)} u(x-e_1h_W) - \f 2{h_Oh_N} u(x) + \f 2{h_O(h_O+h_W)} u(x+h_O e_1) \\
					&\quad + \f 2{h_S(h_N+h_S)} u(x-e_2 h_S) - \f 2{h_S h_N} u(x) + \f 2{h_N(h_S+h_N)} u(x+ e_2 h_N) \\
					&\quad + \LandauO(h)
				\end{align*}
				„Shortley-Weller Approximation“.
		\end{itemize}
	\end{note}
	\begin{note}[Andere Randbedingungen]
		Neben Dirichlet- können auch andere Randbedingungen realisiert werden, z.B. Neumann-Randbedingungen:
		\begin{align*}
			(\Nabla u) \cdot n &= g_N
			&& \text{auf $\Gamma_N \subset \Gamma$}.
		\end{align*}
		Wir nennen $\Gamma_N$ \emphdef{Neumann-Randteil}.
		Wir nehmen an, dass $x$ auf einer Kante liegt und nicht auf einer Ecke von $\Gamma$ (sonst ist keine Normale $n$ definiert).
		Sei $n = \pm e_j$ (da Würfelgebiet) äußerer Normalenvektor für ein $j = 1, \dotsc, d$.
		Wir Approximieren $\Nabla u$ durch
		\[
			(\Nabla_h u)_i :=
			\begin{cases}
				\partial_{x_i}^{c,h} u & i \neq j \\
				\partial_{x_j}^{-h} u & i = j
			\end{cases}.
		\]
		Nun sind $u_h(x), x \in \Gamma_h \cap \Gamma_N$ Unbekannte.
		Es fällt eine weitere Gleichung für das LGS an:
		\[
			(\Nabla_h u_h(x)) \cdot n = g_N(x)
		\]
		für $x \in \Gamma_h \cap \Gamma_N$.
	\end{note}
\end{ex}


\section{Allgemeine Elliptische PDEs zweiter Ordnung}

\begin{df}[Allgemeine Elliptische RWP] \label{2.10}
	Zu $\Omega \subset \R^d$ beschränkt, $f \in C^0(\Omega), g \in C^0(\Boundary \Omega)$ sei
	\begin{equation} \label{eq:2.3}
		(\scr L u)(x)
		= - \sum_{i,j=1}^d a_{ij}(x) \partial_{x_j} \partial_{x_i} u(x)
		+ \sum_{i=1}^d b_i(x) \partial_{x_i} u(x) + c(x) u(x)
	\end{equation}
	gleichmäßig elliptisch und $a_{ij}, b_i, c \in C^0(\_\Omega)$.
	Gesucht ist $u \in C^2(\Omega) \cap C^0(\_\Omega)$.
	\begin{align*}
		\scr L u(x) &= f(x) && \text{$x \in \Omega$}, \\
		u(x) &= g(x) && \text{$x\in\Gamma$}.
	\end{align*}
\end{df}

\begin{df}[FD-Approximation] \label{2.11}
	Für $\scr L$ aus \eqref{eq:2.3}, $x \in \Omega$ mit $\_B_h(x)  \subset \_\Omega$ definiere
	\begin{equation} \setcounter{equation}{5} \label{eq:2.5}
		\begin{aligned}
			\scr L_h u(x)
			&:= - \sum_{i=1}^d a_{ii} (x) \partial_{x_i}^{-h} \partial_{x_i}^{+h} u(x)
			- \sum_{\substack{i,j=1 \\ i\neq j}}^d a_{ij} (x) \partial_{x_i}^{c,h} \partial_{x_j}^{c,h} u(x) \\
			&\qquad + \sum_{i=1}^d b_i(x) \partial_{x_i}^{c,h} u(x)	+ c(x) u(x).
		\end{aligned}
	\end{equation}
	\begin{note}
		Wir werden sehen, dass
		\begin{itemize}
			\item
				\ref{2.11} nur unter weiteren Annahmen an $a_{ij}, b_i, c, h$ eine „stabile Diskretisierung“ ergibt,
			\item
				Eiene etwas sorgfältigere Diskretisierung des Hauptteils eine erweiterte Klasse von Problemen stabil diskretisiert.
		\end{itemize}
	\end{note}
\end{df}

\begin{st}[FD-Approximationsfehler für $\scr L_h$] \label{2.12}
	Sei $u \in C^4(\_\Omega), x \in \Omega$ sodass $x + \sum_{i=1}^d \sigma_i e_i h \in \_\Omega$ für $\sigma_i = \Set{0,+1, -1}$, $i = 1, \dotsc, d$.
	Dann existiert ein $C$ (unabhängig von $x,h$) sodass
	\[
		\big|\scr L u(x) - \scr L_h u(x)\big| \le C h^2.
	\]
	\begin{proof}
		Taylor analog zu \ref{2.2} und \ref{2.3}.
	\end{proof}
\end{st}

\begin{note}[FD-Stern]
	\begin{itemize}
		\item
			Die Diskretisierung $\scr L_h$ kann man anschaulicher notieren: z.B. für $d = 2$.
			Falls $\scr L_h u(x_1, x_2) = \f 1{h^2} \sum_{i,j=-m}^m \alpha_{ij} u(x_1 + ih, x_2 + jh)$ dann ist
			\begin{equation} \setcounter{equation}{4} \label{eq:2.4}
				\Matrix[{
					\alpha_{-m,m} & \hdots & \alpha_{0,m} & \hdots & \alpha_{m,m} \\
					\vdots && \vdots && \vdots \\
					\alpha_{-m, 0} & \hdots & \alpha_{0,0} & \hdots & \alpha_{m,0} \\
					\vdots && \vdots && \vdots \\
					\alpha_{-m,-m} & \hdots & \alpha_{0,-m} & \hdots & \alpha_{m,-m}
				}_*
			\end{equation}
			Für $\scr L_h = -\Laplace_h$ aus \ref{2.3} ergibt sich für $m = 1$ der \emphdef{5-Punkt-FD-Stern}
			\[
				\Matrix[{ 0 & -1 & 0 \\ -1 & 4 & -1 \\ 0 & -1 & 0 }_*
			\]
			Für $\scr L_h$ aus \ref{2.11} (ohne Einschräkung $a_{12} = a_{21}$):
			\[
				\f 12 \Matrix[{ a_{12}(x) & - 2 a_{22}(x) & - a_{12}(x) \\ -2 a_{11}(x) & 4(a_{11}(x) + a_{22}(x)) & -2a_{11}(x) \\ -a_{12}(x) & -2 a_{22}(x) & a_{12}(x)}_*
				+ \f h2 \Matrix[{ 0 & b_2(x) & 0 \\ -b_1(x) & 0 & b_1(x) \\ 0 & -b_2(x) & 0 }_*
				+ h^2 \Matrix[{0 & 0 & 0 \\ 0 & c(x) & 0 \\ 0 & 0 & 0}_*.
			\]
		\item
			Für $m = 1$ (also $3\times 3$ Sterne) ist höchstens Approximationsordnung 2 erreichbar, wie in \ref{2.12} und \ref{2.3} für spezielle $\scr L_h$ gesehen
		\item
			Für $m > 1$ sind bessere Approximationsordnungen erreichbar
	\end{itemize}
\end{note}

\begin{df}[FD-Approximation für elliptisches RWP] \label{2.13}
	Sei $\Omega$ Würfelgebiet zu $h \in \R^+$, $\_\Omega_h$ zugehöriges Gitter.
	Dann ist $u_h \in X_h$ FD-Lösung falls
	\begin{align*}
		\scr L_h u_h(x) &= f(x) && \text{$x \in \Omega_h$}, \\
		u_h(x) &= g(x) && \text{$x\in \Gamma_h$}.
	\end{align*}
\end{df}

\begin{df}[Diskreter Zusammenhang] \label{2.14}
	Ein Gitter $\_\Omega_h$ heißt \emphdef{diskret zusammenhängend} falls es für alle $x,y \in \Omega_h$ (innere Punkte) eine Punktfolge $z_i \in \Omega_h$, $i = 0, \dotsc, k$ gibt mit $z_0 = x, z_k = y$ und $|z_i - z_{i-1}| = h$ für $i = 1, \dotsc, k$.
	\begin{note}
		Wenn zu grob gesampled wird, kann es vorkommen, dass $\Omega$ zwar zusammenhängend ist, $\Omega_h$ jedoch nicht.
		Falls $\Omega$ zusammenhängend, ist jedoch für hinreichend kleines $h$ auch $\_\Omega_h$ diskret zusammenhängend.
	\end{note}
\end{df}

\begin{lem}[Sternlemma] \label{2.15}
	Sei $k \ge 1$, $\Set{\alpha_i}_{i=0}^k \subset \R$ und $\Set{p_i}_{i=0}^n \subset \R$ gegeben und es gelte
	\begin{enumerate}[i)]
		\item
			$\alpha_0 > 0$ und $\alpha_i < 0$ für $i = 1, \dotsc, k$2
		\item
			$\sum_{i=0}^k \alpha_i = 0$,
		\item
			$\sum_{i=0}^k \alpha_i p_i \le 0$.
	\end{enumerate}
	Dann folgt aus $p_0 \ge \max_{i=1,\dotsc, k} p$ die Gleichheit $p_0 = p_1 = \dotsb = p_k$.
	\begin{proof}
		Es gilt
		\[
			\sum_{i=1}^k \alpha_i(p_i-p_0)
			= \sum_{i=0}^k \alpha_i (p_i-p_0)
			= \underbrace{\sum_{i=0}^k \alpha_i p_i}_{\le 0} - \underbrace{p_0 \sum_{i=0}^k \alpha_i}_{=0}
			\le 0.
		\]
		Wegen $p_0 \ge p_i, \alpha_i < 0$ für $i = 1, \dotsc, k$ sind die Summanden links nicht-negativ, also null und es folgt $p_i = p_0$.
	\end{proof}
\end{lem}

\begin{st}[Diskretes Maximumsprinzip] \label{2.16}
	Sei $u_h \in X_h$ eine FD-Lösung zu dem RWP \ref{2.13} mit $f(x) \le 0$ für $x \in \Omega_h$.
	Der Differenzenstern \eqref{eq:2.4} zu $m = 1$ (also ein $3\times 3$ Stern) erfülle in allen Punkten
	\begin{enumerate}[i)]
		\item
			$\sum_{i,j=-1}^1 \alpha_{i,j} = 0$,
		\item
			$\alpha_{0,0} > 0$,
		\item
			$\forall (i,j) \neq (0,0) : \alpha_{i,j} \le 0$,
		\item
			$\alpha_{1,0}, \alpha_{0,1}, \alpha_{0,-1}, \alpha_{-1,0} < 0$, „Koeffizienten in Hauprichtungen negativ“.
	\end{enumerate}
	Dann gilt
	\[
		\max_{x\in\_\Omega_h} u_h(x) = \max_{x\in \Gamma_h} u_h(x).
	\]
\Timestamp{2014-11-04}
	\begin{proof}
		Sei $x \in \_\Omega_h$ mit $u_h(x) = \max_{\_x\in \_\Omega_h} u_h (\_x)$.
		Falls $x \in \Gamma_h$, so sind wir fertig.
		Falls $x \in \Omega_h$: setze $p_0 := u_h(x)$, $(p_i)_{i=1}^k$ als „Nachbarwerte“ von $u_h(x)$, $\alpha_0 := \alpha_{0,0}$, $(\alpha_i)_{i=1}^k$ als Nicht-Null-Koeffizienten des FD-Sterns.
		Es gilt also $\f 1{h^d} \sum_{k=0}^k \alpha_i p_i = \scr L_h u(x) = f(x) \le 0$.
		Das Sternlemma \ref{2.15} hier angewandt ergibt $p_0 = p_1 = \dotsc = p_k$, also $u_h$ konstant in $x$ und seinen Nachbarn, welche im Differenzenstern auftreeten.
		Wiederholung dieses Argumentes in alle $2d$ Hauptrichtungen führt zum Rand (Beschränktheit von $\Omega$)
	\end{proof}
\end{st}

Falls $\Omega_h$ diskret zusammenhängend ist, führt das Argument aus dem Beweis von \ref{2.16} zu allen Punkten in $\Omega_h$, also folgt

\begin{kor}[Lösung $u_h$ konstant] \label{2.17}
	Falls $\_\Omega_h$ diskret zusammenhängend und die FD-Lösung für $f \le 0$ nimmt ihr Maximum im Inneren an, so ist $u_h$ konstant.
\end{kor}

\begin{note}
	\begin{itemize}
		\item
			Obiges diskretes Maximum-Prinzip gilt für $\scr L_h = - \Laplace_h$, weil Voraussetzungen erfüllt sind.
		\item
			Für $\scr L_h$ aus \eqref{eq:2.5} mit $a_{1,2} = 0$, $c = 0$, $b \neq 0$ sind die Voraussetzungen von \ref{2.16} erfüllt, falls $h$ hinreichend klein gewählt wird.
			Aus gleichmäßiger Elliptizität folgt für $z = e_i$
			\[
				0 < \tilde \alpha \underbrace{\|z\|^2}_{= 1} \le z^T A(x) z = a_{ii}(x),
			\]
			also $\alpha_{0,0} = 2(a_{11} + a_{22}) > 0$.
			Falls $|b_i| \le B$ und $h < \f 2B \tilde \alpha$, dann gilt für $\alpha_{1,0}$
			\[
				\alpha_{1,0}
				= -a_{11} + \f {h2} b_1
				< -\tilde \alpha + \f 2B \tilde \alpha \f 12 B
				= 0.
			\]
			Analog für $\alpha_{-1,0} = \alpha_{0,-1} = \alpha_{0,1} = 0$.
		\item
			Falls $c(x) > 0$ in \eqref{eq:2.5}, so ist $\sum_{i,j = -1}^1 \alpha_{ij} > 0$.
			Für diesen Fall lassen sich Abschwächungen des Sternlemmas und des Maximumsprinzips formulieren und beweisen.
		\item
			Falls $a_{1,2} \neq 0$, so ist eine Modifikation der Diskretisierung notwendig.
	\end{itemize}
\end{note}

\begin{kor}[Diskretes Vergleichsprinzip] \label{2.18}
	Seien $u_h, v_h \in X_h$ und es gelte $\scr L_h u_h \le \scr L_h v_h$ in $\Omega_h$ und $u_h \le v_h$ auf $\Gamma_h$ und es gelte das diskrete Maximumsprinzip.
	Dann gilt $u_h \le v_h$ in $\_\Omega_h$.
	\begin{proof}
		Für $w_h := u_h - v_h$ gilt $\scr L_h w_h = \scr L_h u_h - \scr L_h v_h \le 0$ in $\Omega_h$ und $w_h \le 0$ auf $\Gamma_h$.
		Mit dem diskreten Maximumsprinzip wird das Maximum auf dem Rand $\Gamma_h$ angenommen und daher $\max_{x\in \_\Omega_h} w_h(x) \le \max_{x\in\Gamma_h}w_h(x) \le 0$, also $u_h \le v_h$ auf $\_\Omega_h$.
	\end{proof}
\end{kor}

\begin{kor}[Existenz und Eindeutigkeit] \label{2.19}
	Sei ein diskretes RWP gemäß \ref{2.13} gegeben und es gelte das Maximumsprinzip.
	Dann existiert eine eindeutige FD-Lösung $u_h \in X_h$.
	\begin{proof}
		\begin{seg}{Eindeutigkeit}
			Seien $u_h, \_u_h \in X_h$ zwei Lösungen, setze $v := u_h - \_u_h$.
			Es gilt
			\begin{align*}
				\scr L_h v &= \scr L_h u_h - \scr L_h \_u_h = f - f = 0 && \text{in $\Omega_h$} \\
				v &= u_h - \_u_h = g - g = 0 && \text{auf $\Gamma_h$}.
			\end{align*}
			Das Maximumsprinzip liefert $v(x) \le \sup_{x\in\Gamma_h} v(x) = 0$ für alle $x \in \_\Omega_h$.
			Analoge Argumentation für $-v$ liefert $-v(x) \le 0$ in $\_\Omega_h$, folglich $v(x) = 0$ in ganz $\_\Omega_h$
		\end{seg}
		\begin{seg}{Existenz}
			Die FD-Diskretisierung führt auf ein $n \times n$ System $A_h \underbar{u}_h = b_h$ mit $\ker A_h = \Set 0$ wegen der Eindeutigkeit.
			Folglich hat $A_h$ vollen Rang und ist regulär, also $\underbar{u}_h := A_k^{-1} b_k$ ist eindeutiger DOF-Vektor von $u_h \in X_h$.
		\end{seg}
	\end{proof}
\end{kor}

\begin{kor}[Stetige Abhängigkeit von Randdaten] \label{2.20}
	Seien $u_h, \_u_h \in X_h$ FD-Lösungen zum RWP \ref{2.13} mit identischem $f$ aber unterschiedlichen Randdaten $g, \_g$ und es gelte das diskrete Maximumsprinzip.
	Dann gilt
	\[
		\|u_h - \_u_h\|_{\Omega_h} = \|g-\_g\|_{\Gamma_h} := \sup_{x\in\Gamma_h} |g(x) - \_g(x)|.
	\]
	\begin{proof}
		Setze $v :=: u_h - \_u_h$, dann ist $\scr L_h v = 0$ in $\Omega_h$ mit diskretem Maximumsprinzip, also
		\[
			v(x) \le \max_{\_x \in \Gamma_h} v(x)
			\le \max_{x \in \Gamma_h} |v(x)|
			= \|g - \_g\|_{\Gamma_h},
		\]
		analog für $-v(x)$, es folgt die Behauptung.
	\end{proof}
	\begin{note}
		\begin{itemize}
			\item
				Anschauliche Bedeutung: Leichte Änderung in Daten ergibt nur leichte Änderung in Lösung.
			\item
				Ähnlich: Stetige Abhängige bezüglich der rechten Seite $f$.
		\end{itemize}
	\end{note}
\end{kor}

\begin{df}[Stabilität, Konsistenz, Konvergenz] \label{2.21}
	Sei $u \in X := C^2(\Omega) \cap C^0(\_\Omega)$ Lösung des ellptischen RWP \ref{2.10} und $u_h \in X_h$ FD-Approximation aus \ref{2.13}.
	Die FD-Diskretisierung ist
	\begin{enumerate}[i)]
		\item
			\emphdef{konsistenz mit Ordnung $p$}, wenn für ein $C_c = C_c(x)$ unabhängig von $h$ gilt
			\[
				\|\scr L_h u - \scr L u\|_{\Omega_h} \le C_c h^p.
			\]
		\item
			\emphdef{stabil} (genauer: $(X_h^0, Y_h)$-stabil), wenn $C_s$ unabhängig von $h$ existiert, sodass
			\[
				\|v_h\|_{\_\Omega_h} \le C_s \|\scr L_h v_h\|_{\Omega_h}
			\]
			für alle $v_h \in X_h^0$.
		\item
			\emphdef{konvergent mit Ordnung $p$}, wenn für ein $C = C(u)$ unabhängig von $h$ gilt:
			\[
				\|u - u_h\|_{\_\Omega_h} \le C h^p.
			\]
	\end{enumerate}
	\begin{note}[Konsistenz der FD-Approximation]
		\begin{itemize}
			\item
				Punktweise Fehlerschranke aus \ref{2.3}, oder \ref{2.12} besagten $|\scr L_h u(x) - \scr L u(x)| \le C h^2$ mit $C$ unabhängig von $x$.
				Also liegt eine \emphdef{uniforme Schranke in $x$} vor, falls $u \in C^4(\_\Omega)$.
				Mit $C_c := C$ gilt Konsistenz der FD-Approximation mit Ordnung 2 für $\Omega$ Würfelgebiet und Ordnung $1$ für Nicht-Würfelgebiete, z.B. durch die Shortley-Weller-Approximation.
		\end{itemize}
	\end{note}
	\begin{note}[Stabilität]
		\begin{itemize}
			\item
				Stabilität bedeutet anschaulisch, dass die Lösung der PDE durch die rechte Seite beschränkt bleibt, unabhängig von $h$.
				Sei $w_h: \Omega_h \to \R$, also $w_h \in Y_h$ und $v_h \in X_h^0$ Lösung von $\scr L_h v_h = w_h$ in $\Omega_h$, $v_h = 0$ auf $\Gamma_h$.
				Dann ist also
				\[
					\|v_h\|_{\_\Omega_h} \le C_s \|\scr L_h v_h\| = C_s \|w_h\|_{\Omega_h}.
				\]
		\end{itemize}
	\end{note}
\end{df}

\begin{st}[Hinreichende Bedingung für Stabilität] \label{2.22}
	Sei $A_h \in \R^{n\times n}$ die FD-System-Matrix.
	Falls $C_s$ unabhängig von $h$ existiert, sodass $\|A_h^{-1}\|_\infty \le C_s$.
	Dann ist das FD-Verfahren stabil.
	\begin{proof}
		Seien $\underbar{v}_h, \underbar{w}_h \in \R^n$ Vektor der inneren Knotenwerten für $v_h \in X_h^0$, $w_h \in Y_h$, also $A_h \underbar{v}_h = \underbar{w}_h$.
		Dann gilt ($v_h$ hat Nullrandwerte)
		\[
			\|v_h\|_{\_\Omega_h}
			= \|\underbar{v}_h\|_\infty
			= \|A_h^{-1} \underbar{w}_h\|_\infty
			\le \|A_h^{-1}\|_\infty \|\underbar{w}_h\|_\infty
			\le C_s \|w_h\|_{\Omega_h}
			= C_s \|\scr L_h v_h\|
		\]
	\end{proof}
\end{st}

\begin{st}[Stabiltität für Poisson-RWP, FD-Diskretisierung] \label{2.23}
	Sei $\Omega \subset \R^d$ beschränktes Gebiet mit $\Omega \subset B_R(0)$ für ein $R > 0$.
	Dann gilt für alle $v_h \in X_h^0$, dass
	\[
		\|v_h\|_{\_\Omega_h} \le \f {R^2}{2d} \|\Laplace_h v_h\|_{\Omega_h}
	\]
	also das FD-Verfahren stabil mit $C_s := \f {R^2}{2d}$.
	\begin{proof}
		Sei $v_h \in X_h^0$ und $w_h$ aus \ref{2.24}.
		Dann ist für $x \in \Omega_h$
		\[
			- \frac{\Laplace_h v_h(x)}{\|\Laplace_h v_h\|_{\Omega_h}}
			\le \frac{\Laplace_h v_h(x)}{\|\Laplace_h v_h\|_{\Omega_h}}
			\le 1
			= - \Laplace_h w_h(x)
		\]
		Für $x \in \Gamma_h$ gilt $- \frac{v_h(x)}{\|\Laplace_h v_h\|} = 0 = w_h(x)$.
		Also mit diskretem Vergleichsprinzip \ref{2.18} für alle $x \in \Omega_h$
		\[
			\frac{v_h(x)}{\|\Laplace_h v_h\|_{\Omega_h}}
			\le w_h(x)
			\stack{\ref{2.14}}{\le} \f 1{2d} (R^2 - \|x\|^2)
			\le \f {R^2}{2d}.
		\]
		Eine analoge Argumentation für $-v_h$ liefert
		\[
			- \frac{v_h(x)}{\|\Laplace_h v_h\|_{\Omega_h}}
			\le \frac{R^2}{2d},
		\]
		also $\|v_h\|_{\_\Omega_h} \le \f {R^2}{2d} \|\Laplace_h v_h\|_{\Omega_h}$.
	\end{proof}
\end{st}

\begin{lem} \label{2.24}
	Sei $w_h \in X_h$ Lösung von $-\Laplace_h w_h = 1$ in $\Omega_h$, $w_h = 0$ auf $\Gamma_h$.
	Dann gilt
	\begin{equation} \setcounter{equation}{6} \label{eq:2.6}
		0 \le w_h(x) \le \f 1{2d} (R^2 - \|x\|_2^2)
	\end{equation}
	für $x \in \_\Omega_h$.
	\begin{proof}
		Sei $w(x) := \f 1{2d} (R^2 - \|x\|_2^2)$.
		Dann ist $w$ Polynom zweiten Grades, also $\Laplace_h$ exakt für $w$ gemäß Bemerkung nach \ref{2.3}.
		Für $x \in \Omega_h$ gilt somit
		\begin{align*}
			-\Laplace_h w(x)
			= - \Laplace w(x)
			&= - \sum_{i=1}^d \partial_{x_i}^2 \Big(R^2 - \sum_{j=1}^d x_j^2 \Big) \f 1{2d} \\
			&= - \sum_{i=1}^d (-2) \f 1{2d}
			= 1
			= - \Laplace_h w_h(x).
		\end{align*}
		Weiter ist $w \ge 0 = w_h$ auf $\Gamma_h$ nach Wahl von $R$.
		Aus dem diskreten Vergleichsprinzip \ref{2.18} folgt $w \ge w_h$ auf $\_\Omega_h$, also die zweite Gleichung in der Behauptung.
		Die erste Ungleichung folgt aus dem diskreten Maximumsprinzip für $-w_h$:
		\[
			- \Laplace_h (-w_h) =  -1 \le 0
		\]
		für $x \in \Omega_h$, also $\max_{x\in\_\Omega_h} (-w_h(x)) \le \max_{x\in \Gamma_h} (-w_h(x)) = 0$ und damit $w_h \ge 0$ auf $\Omega_h$.
	\end{proof}
\end{lem}

\Timestamp{2014-11-07}

\begin{st}[Konvergenz] \label{2.25}
	Sei ein FD-Verfahren für ein elliptisches RWP gemäß \ref{2.10} gegeben.
	Falls das Verfahren staabil und konsistenz mit Ordnung $p$ ist, so auch konvergent mit Ordnung $p$.
	\begin{proof}
		Seien $u \in X$ die exakte und $u_h \in X_h$ die FD-Lösung.
		Dann hat $u - u_h$ Nullrandwerte auf $\Gamma_h$, also folgt mit Stabilität:
		\[
			\|u-u_h\|_{\_\Omega_h}
			\le C_S \| \scr L_h (u-u_h) \|_{\Omega_h}
			= C_s \|\scr L_h u - \scr L_h u_h\|_{\Omega_h}
		\]
		Wegen $(\scr L_h u_h)(x) = f(x) = (\scr L u)(x)$ für alle $x \in \Omega_h$ folgt mit Konsistenz:
		\[
			\|u - u_h\|_{\_\Omega_h}
			\le C_s \|\scr L_h u - \scr L u\|_{\Omega_h}
			\le \underbrace{C_s C_c}_{=:C} h^p.
		\]
	\end{proof}
\end{st}

\ref{2.25} ist auf die FD-Diskretisierung des Poisson-RWPs anwendbar, denn wir haben Konsistenz (siehe Bemerkung nach \ref{2.21}) und Stabilität in \ref{2.23} nachgewiesen.

\begin{kor}[Konvergenz für FD-Diskretisierung, Poisson-RWP] \label{2.26}
	Sei $\Omega \subset \R^d$ beschränktes Gebiet und die Lösung $u$ des Poisson-RWP erfülle $u \in C^4(\_\Omega)$.
	Dann konvergiert das FD-Verfahren, d.h.
	\[
		\|u - u_h\|_{\_\Omega_h}
		\le C h^p
	\]
	mit $p = 2$ für Würfelgebiete und $p = 1$ für allgemeine Gebiete.
\end{kor}

Ein Weg, Stabilität zu zeigen, führt über das diskrete Maximumsprinzip.
Ein alternativer Weg bietet \ref{2.22}:
es genügt $\|A_h^{-1}\|_\infty \le C_s$ für geeignetes $C_s$ unabhängig von $h$ zu zeigen.
Dies ist mit der sogenannten „M-Matrix-Theorie“ möglich.

\begin{df} \label{2.27}
	\begin{enumerate}[i)]
		\item
			Eine Matrix $A = (a_{ij})_{i,j=1}^n \in \R^{n\times n}$ heißt \emphdef{$L_0$-Matrix}, falls $a_{ij} \le 0$ für alle $i \neq j$.
		\item
			Eine Matrix $A = (a_{ij})_{i,j=1}^n \in \R^{n\times n}$ heißt \emphdef{$L$-Matrix}, falls $A$ ein $L_0$-Matrix ist und $a_{ii} > 0$ für $i= 1, \dotsc, n$.
		\item
			Eine $L_0$-Matrix $A$, für die $A^{-1}$ existiert und $A^{-1} \ge 0$ (komponentenweise größergleich Null) heißt \emphdef{M-Matrix}.
	\end{enumerate}
	\begin{note}
		\begin{itemize}
			\item
				Wenn $A^{-1}$ existiert und $A^{-1} \ge 0$, so nennt man $A$ auch \emphdef{inversmonoton}.
			\item
				Eine M-Matrix ist also eine inversmonotone $L_0$-Matrix.
			\item
				Ziel: für gegebenen $L_0$- oder $L$-Matrix, finde Zusatzbedingungen, welche $M$-Matrix-Eigenschaft implizieren und $\|A^{-1}\|_\infty$ abzuschätzen erlauben.
		\end{itemize}
	\end{note}
\end{df}

\begin{st}[$M$-Kriterium] \label{2.28}
	Sei $A \in \R^{n\times m}$ ein $L_0$-Matrix.
	\begin{enumerate}[i)]
		\item
			$A$ ist inversmonoton (also $M$-Matrix) genau dann, wenn $e \in \R^n$ existiert mit $e > 0$ und $Ae > 0$.
		\item
			Falls $A$ eine $M$-Matrix und $e$ wie in i), so gilt
			\[
				\|A^{-1}\|_{\infty} \le \frac{\|e\|_\infty}{\min_{k}(Ae)_k}.
			\]
	\end{enumerate}
	\begin{proof}
		\begin{enumerate}[i)]
			\item
				\begin{segnb}{\ProofImplication}
					Setze $e := A^{-1} \Vector{1 & \dots & 1}$, dann ist $e > 0$ und $Ae = \Vector{1 & \dots & 1} > 0$.
				\end{segnb}
				\begin{segnb}{\ProofImplication*}
					Sei $e > 0$ und $Ae > 0$, d.h. $\sum_{j} a_{ij} e_j > 0$ für alle $i = 1, \dotsc, n$.
					Weil $a_{ij} e_j \le 0$ für $i \neq j$, da $A$ $L_0$-Matrix, muss $a_{ii} e_i > 0$, also $a_{ii} > 0$ und $A$ ist eine $L$-Matrix.
					Setze $D := \diag(a_{11}, \dotsc, a_{nn})$, diese ist offenbar invertierbar.
					Setze $P d= D^{-1}(D-A) = I - D^{-1} A$, also $A = D(I-P)$ und $P \ge 0$ (wegen $D - A \ge 0$ und $D^{-1} \ge 0$).
					Weiter ist $(I-P)e = D^{-1}A e > 0$, also folgt
					\begin{equation} \label{eq:2.7}
						e = Ie > Pe
					\end{equation}
					Führe eine spezielle Norm ein: $\|x\|_e := \max_i \frac{|x_i|}{e_i}$ mit induzierter Matrixnorm $\|P\|_e := \sup_{\|x\|_e = 1} \|Px\|_e$.
					Es gilt $\|e\|_e = \max \frac{|e_i|}{e_i} = 1$, also $\|P\|_e \ge \|Pe\|_e$, andererseits ist für $y \in \R^n$ mit $\|y\|_e = 1$, d.h. $\max_i \frac{|y_i|}{e_i} = 1$ auch $y \le e$ und es folgt wegen $P \ge 0$, dass $Py \le Pe$, also $\|Py\|_e \le \|Pe\|_e$.
					Somit ist
					\[
						\|P\|_e = \sup_{\|x\|_e = 1} \|Px\|_e = \|Pe\|_e = \max_i \frac{(Pe)_i}{e_i}.
					\]
					Wegen \eqref{eq:2.7} ist $Pe < e$ und somit $\|P\|_e < 1$.
					Damit existiert $(I - P)^{-1}$ und es gilt die Darstellung als Neumannsche Reihe:
					\[
						(I - P)^{-1} = \sum_{j=0}^\infty P^j.
					\]
					Wegen $A = D(I-P)$ existiert auch $A^{-1} = (I-P)^{-1}D^{-1}$ und wegen $P \ge 0$ auch $P^j \ge 0$, also mittel Neumannscher Reihe $(I-P)^{-1} \ge 0$, und somit $A^{-1} \ge 0$ inversmonoton.
				\end{segnb}
			\item
				Sei $Aw = f$, d.h. $w = A^{-1} f$ und
				\[
					w_i = (A^{-1}f)_i = \sum_{j} (A^{-1})_{ij} f_j \le \|f\|_\infty \sum_{j=1}^n (A^{-1})_{ij}.
				\]
				Also gilt
				\begin{equation} \label{eq:2.8}
					w \le \|f\|_\infty A^{-1} \Vector{1 & \dots & 1}
				\end{equation}
				und analog $-w \le \|f\|_\infty A^{-1} \Vector{1 & \dots & 1}$.
				Es gilt $Ae \ge (\min_k (Ae_k)) \Vector{1 & \dots & 1}$, mit $Ae > 0$ folgt
				\[
					\frac{Ae}{\min_k (Ae)_k} \ge \Vector{1 & \dots & 1}.
				\]
				Also ist mit \eqref{eq:2.8} $\pm w \le \|f\|_\infty A^{-1} \frac{Ae}{\|\min_k (Ae)_k\|} = \|f\|_\infty \frac{c}{\min_k (Ae)_k}$, es folgt $\|w\|_\infty \le \|f\|_\infty \frac{\|e\|_\infty}{\min_k(Ae)_k}$, also
				\[
					\|A^{-1}\|_\infty
					= \sup_{f\neq 0} \frac{\|A^{-1} f\|_\infty}{\|f\|_\infty}
					= \sup_{f \neq 0} \frac{\|w\|_\infty}{\|f\|_\infty}
					\le \frac{\|e\|_\infty}{\min_k (Ae)_k}.
				\]
		\end{enumerate}
	\end{proof}
\end{st}

\begin{ex*}
	Betrachte die FD für das Poisson-RWP, $d = 1$ wie in \ref{2.8}, $h := \f 1{n+1}$, $A_h = \Matrix{2 & -1 &  & \\ -1 & \ddots & \ddots & \\ &\ddots & \ddots & -1 \\ & & -1 & 2} \in \R^{n\times n}$.
	$A_h$ ist eine $L_0$-Matrix.
	Finde $e \in \R^n, e > 0$ und z.B. $A_h e = \Vector{1 & \dots & 1} > 0$.
	Die Lösung ist $e = (e_k)_{k=1}^n$ mit $e_k = \f 12 k h  (1-kh) = \f 12 kh - \f 12 k^2h^2$.
	Es gilt
	\[
		(A_h e_h)_1 = \frac{1}{h^2} (2e_1 - e_2)
	\]
	mit $e_1 = \f 12 h - \f 12 h^2, e_2 = h - 2h^2$ folgt
	\[
		(A_h e_h)_1 = \f 1{h^2}(h - h^2 - h + 2h^2) = 1
	\]
	und analog $(Ae)_n = 1$ wegen Symmetrie in $A_h$ und $e$.
	Für $1 < k < n$ haben wir
	\begin{align*}
		h^2 (Ae)_k
		&= (-e_{k-1} + 2e_k - e_{k+1}) \\
		&= -\f 12 (k-1)h + \f 12(k-1)^2 h^2 + kh -k^2 h^2 - \f 12 (k+1)h + \f 12 (k+1)^2 h^2 \\
		&= -\f 12 kh + \f 12 h + \f 12(k^2 - 2k + 1)h^2 + kh - k^2h^2 \\
		&\qquad - \f 12 kh - \f 12 h + \f 12 (k^2 + 2k+1)h^2 \\
		&= h^2( \f 12 k^2 - k^2 + \f 12 k^2 - k + \f 12 + k + \f 12) \\
		&= h^2
	\end{align*}
	und damit ist $A_h$ eine $M$-Matrix.

	Es gilt $\|e\|_\infty = \max_k \f 12 kh(1-kh) \le \f 12 \f 12 ( 1 - \f 12) = \f 18$.
	Wegen $Ae = \Vector{1 & \dots & 1}$ ist $\min_k (Ae)_k = 1$ und aus \ref{2.28} ii) folgt
	\[
		\|A_k^{-1}\| \le \frac{\|e\|_\infty}{\min_k (Ae)_k} = \f 18.
	\]
	Vergleich zu $C_S = \frac{R^2}{2d}$ aus \ref{2.23} mit $R = 1, d=1$ liefert $C_s = \f 12$.
	Die Konstante $\f 18$ ist also besser als die in \ref{2.23}.
\end{ex*}

\begin{df}[Diagonaldominanz, Irreduzibilität]
	\begin{enumerate}[i)]
		\item
			$A \in \R^{n\times n}$ ist \emphdef[diagonaldominant!stark]{stark (zeilen-)diagonaldominant}, falls $|a_{ii}| > \sum_{i \neq j}^n |a_{ij}|$ für alle $i = 1, \dotsc, n$.
		\item
			$A \in \R^{n\times n}$ ist \emphdef[diagonaldominant!schwach]{schwach (zeilen-)diagonoldominant}, falls $|a_{ii}| \ge \sum_{i \neq j}^n |a_{ij}|$ für alle $i = 1, \dotsc, n$ mit mindestens einem $k \in \Set{1,\dotsc, n}$ mit $|a_{kk}| > \sum_{j \neq k} |a_{kj}|$.
		\item
			$A$ ist \emphdef{irreduzibel}, falls keine Permutationsmatrix $P$ existiert, sodass
			\[
				P A P^T = \Matrix{ A_{11} & 0 \\ A_{21} & A_{22} }
			\]
			mit $A_{11} \in \R^{p\times p}$ mit $p \in {1, \dotsc, n-1}$.
	\end{enumerate}
\end{df}

\begin{st} \label{2.30}
	Sei $A_h$ eine $L$-Matrix.
	Falls $A_h$ stark diagonaldominant, oder schwach diagonaldominant und irreduzibel, so ist $A_h$ ein $M$-Matrix.
	\begin{proof}
		Siehe Großmann/Roos, Satz 2.8.
	\end{proof}
\end{st}

\begin{note}
	Falls $A_h$ strikt diagonaldominant, so ist $e = \Vector{1 & \dots & 1}$ ein geeigneter Vektor für \ref{2.28}, denn $e > 0$ und $A_h e > 0$.
	Also $\|A_h^{-1}\|_\infty \le \f 1{\min_k(Ae)_k}$ (aber potentiell von $h$ abhängig).
\end{note}

\begin{note}[gemischte Ableitung]
	\begin{itemize}
		\item
			In \ref{2.11} wurde $-\sum_{i\neq j} a_{ij}(x) \partial_{x_i}^{c,h} \partial_{x_j}^{c,h} u(x)$ gewählt, was zu $\f 12 \Matrix[{a_{12} & 0 & -a_{12} \\ 0 & 0 & 0 \\ -a_{12} & 0 & a_{12}}_*$ führte.
			Weder das Maximumsprinzip, noch $M$-Matrix-Eigenschaft können mit unseren Techniken gezeigt werden, wegen unterschiedlichen Vorzeichen im Fall $a_{12} \neq 0$.
		\item
			Man kann FD-Sterne 2. Ordnung mit nichtpositiven Eck-Koeffizienten für $-2a_{12} \partial_{x_1} \partial_{x_2} u$ konstruieren:
			\begin{enumerate}[i)]
				\item
					Falls $- a_{12} > 0$:
					\[
						\Matrix[{ a_{12} & -a_{12} & 0 \\ a_{12} & 2 a_{12} & - a_{12} \\ 0 & - a_{12} & a_{12} }_*
					\]
				\item
					Falls $- a_{12} < 0$:
					\[
						\Matrix[{ 0 & a_{12} & -a_{12} \\ a_{12} & - 2a_{12} & a_{12} \\ - a_{12} & a_{12} & 0}_*
					\]
			\end{enumerate}
			Mit der Konvention $a_{12}^+ := \max\{0, a_{12}\}, a_{12}^- := \min\{a_{12}, 0\}$ folgt
			\begin{equation} \label{eq:2.9}
				\Matrix{
					a_{12}^- & -(a_{22}-|a_{12}|) & -a_{12}^+ \\
					-(a_{11}-|a_{12}|) & 2(a_{11} + a_{22}-|a_{12}|) & - (a_{11} -|a_{12}|) \\
					-a_{12}^+ & -(a_{22} - |a_{12}|) & a_{12}^-
				}_*
				+ \f 12 \Matrix{
					0 & b_2 & \\
					-b_1 & 2hc & b_1 \\
					0 & -b_2 & 0
				}_*
			\end{equation}
	\end{itemize}
\end{note}

\Timestamp{2014-11-11}

\begin{kor} \label{2.31}
	Falls $a_{ii} > |a_{12}| + \f h2 |b_i|$ und $c > 0$ für $i= 1,2$, so ist die FD-Systemmatrix $A_h$ zum FD-Stern \eqref{eq:2.9} eine $M$-Matrix.
	\begin{proof}
		Die Nichtdiagonalelemente von $A_h$ sind nicht-positiv.
		Die Diagonale von $A_h$ ist echt positiv, also ist $A_h$ eine $L$-Matrix.
		$A_h$ ist streng diagonaldominant (Summe aller FD-Stern-Einträge ist größer 0, FD-Stern wird in jeweils eine Zeile von $A_h$ geschrieben).
	\end{proof}
	\begin{note}
		\begin{itemize}
			\item
				Man kann zeigen, dass $\|A_h^{-1}\|_\infty$ unabhängig von $h$ beschränkt ist.
			\item
				Die Bedingung $a_{ii} > |a_{12}| + \f h2 |b_i|$ liefert eine Bedingung für $h$, d.h. eine hinreichend kleine Gitterweite ist erforderlich für Stabilität im Fall $b \neq 0$.
			\item
				Falls $b_i$ sehr „groß“ (sogenannter \emphdef{konvektionsdominanter Fall}, kann dies zu impraktikablen Gitterweiten führen).
			\item
				Falls $A(x) = 0, c = 0$ (also reine Advektion), $b \neq 0$ kann man leicht analytisch sehen, dass der FD-Vektor mit zentralen Differenzen nicht stabil ist: „\emphdef{hyperbolische Gleichung erster Ordnung}“.

				Betrachte $\Omega = (0,1)^2, b = \Vector{1 & 1}, g(x_1, x_2) = (x_1 - x_2)^2$ mit $\partial_{x_1} u + \partial_{x_2} u = 0$ in $\Omega$ und $u = g$ auf $\Gamma$.
				Dies hat die exakte Lösung $u(x_1, x_2) = (x_1 - x_2)^2 = x_1^2 - 2x_1x_2 + x_2^2$.

				Mit $h = \f 12$ folgt $|\Omega_h| = 1$ wegen $\Omega_h = \Set{(\f 12, \f 12)}$, $|\Gamma_h| = 8$.
				Es ergibt sich ein $1\times 1$-System für $u_h(\f 12, \f 12)$.
				Mit zentralen Differenzen (wie in \eqref{eq:2.5}, \eqref{eq:2.9}) für $\partial_{x_1} u, \partial_{x_2} u$ liefert folgende Gleichung für das LGS:
				\[
					\dfrac{u(\f 12 + \f 12, \f 12) - u_h(\f 12 - \f 12, \f 12)}{2 \cdot \f 12}
					+ \dfrac{u(\f 12, \f 12 + \f 12) - u_h(\f 12, \f 12 - \f 12)}{2 \cdot \f 12}
					= 0.
				\]
				Alle Punktauswertungen liegen in $\Gamma_h$, also setzen wir die Werte für $g$ ein:
				\[
					g(1, \f 12) - g(0, \f 12) + g(\f 12, 1) - g(\f 12, 0) = 0,
				\]
				was zu $0 = 0$ führt, die Systemmatrix $A_h = (0)$ ist singulär, insbesondere ist nicht $\|A_h^{-1}\| \le C_s$ mit $C_s$ unabhängig von $h$, das FD-Verfahren also nicht stabil.
			\item
				Für konvektionsdominante Probleme oder hyperbolische Probleme erster Ordnung sind sorgfältige Diskretisierungen erforderlich (siehe \ref{chap:5}, FV-Verfahren).
		\end{itemize}
	\end{note}
\end{kor}

\begin{ex*}
	Betrachte das Poisson-RWP auf $\Omega = (0,1)^2$.
	\begin{enumerate}[i)]
		\item
			Gebe exakte Lösung vor: $u(x_1, x_2) := x_1(1-x_1)x_2(1-x_2)$.
			Für die Daten wählen wir dann entsprechend: $g(x) := 0 = u(x)$ auf $\Gamma$ und $f(x) := 2x_2(1-x_2) + 2x_1(1-x_1) = -\Laplace u$
			\begin{table}
				\centering
				\begin{tabular}{l|c|l}
					$h$ & $n$ & $\|u - u_h\|_{\_\Omega_h}$ \\ \hline
					0.5 & 1 & 0 \\
					0.25 & 9 & $6 \cdot 10^{-18}$ \\
					\vdots & \vdots & \vdots \\
					0.03125 & 961 & $4.8 \cdot 10^{16}$
				\end{tabular}
				\caption{\texttt{elliptic\textunderscore fd\textunderscore demos(4)}}
			\end{table}
			Wir sehen, dass $u_h$ für jedes $h$ exakt ist (bis auf numerische Rundungseffekte).
			Dies deckt sich mit der Beobachtung, dass $-\Laplace_h u = - \Laplace u$ für Polynome.
		\item
			Setze als exakte Lösung: $u(x_1, x_2) := \sin(2\pi x_1) \sin(2\pi x_2) \in C^\infty$ für Daten $f(x) := 8\pi^2 \sin(2\pi x_1) \sin(2\pi x_2), g(x) := 0$.
			\begin{table}
				\centering
				\begin{tabular}{l|cl}
					h & h & $\|u - u_h\|_{\_\Omega_h}$ \\ \hline
					0.25 & 9 & $0.2337$ \\
					0.125 & 49 & $0.053024$ \\
					\vdots & \vdots & \vdots \\
					0.00097656 & 1046529 & $3.1375 \cdot 10^{-6}$
				\end{tabular}
				\caption{\texttt{elliptic\textunderscore fd\textunderscore demos(5)}}
			\end{table}
			Das LGS mit $A \in 10^{10^6 \times 10^6}$ ist sehr schnell lösbar dank sparse-Matrizen.
			Konvergenzordnung $2$ ist erkennbar in Übereinstimmung mit der Theorie.
	\end{enumerate}
\end{ex*}

\begin{note}[Relevanz der FD-Verfahren]
	\begin{itemize}
		\item
			Bis Mitte des 20. Jahrhunderts wurden FD-Diskretisierungen als Allzweckwerkzeug gesehen, weil sie eine Vielzahl von Problemen sehr leicht mit ausreichender Genauigkeit diskretisieren.
			Allmählich kamen FEM-Methoden (Finite-Elemente-Methoden), welche eine wesentlich aufwändigere Assemblierung von $A_h$ und $b_h$ erfordern, aber bei gleicher Gitterfeinheit bessere Ergebnisse liefern.
		\item
			Die Konvergenzanalysis von FD-Verfahren macht häufig starke (unrealistische) Glattheits-Annahmen an die Lösung.
			Das wird bei FEM- und FV-Verfahren etwas abgeschwächt.
	\end{itemize}
\end{note}


%\subsection{Grundlagen}
%
%\begin{df}
%	\label{df:1.1}	
%	Die komplexen Zahlen bestehen aus
%	\[
%		\C := \{(x,y) : x,y\in \R\}
%	\]
%	und den Verknüpfungen
%	\begin{align*}
%		(x_1,y_1) + (x_2,y_2) &:= (x_1 + x_2, y_1 + y_2) \in \C \\
%		(x_1,y_1) \cdot (x_2,y_2) &:= (x_1x_2 - y_1y_2, x_1y_2 + x_2y_1) \in \C
%	\end{align*}
%\end{df}
%
%\begin{nt}
%	\label{nt:1.1}
%	\begin{enumerate}
%		\item $(\C,+,\cdot)$ ist ein Körper mit $(0,0)$ und Einselement $(1,0)$.
%		\item 
%			$\phi: \R\to \C : x\mapsto (x,0)$ ist eine injektiver Körperhomomorphismus.
%			Insbesondere gilt
%			\begin{align*}
%				\phi(x_1+x_2) &= \phi(x_1) + \phi(x_2)\\
%				\phi(x_1+x_2) &= \phi(x_1) \cdot \phi(x_2)
%			\end{align*}
%			Identifiziere $\R$ mit $\phi(\R) = \{(x,0) : x\in \R\}$.
%			Schreibe dazu: $(x,0) =: x \in \R$.
%		\item
%			Imaginäre Einheit $i:= (0,1)$.
%			\[
%				\implies \begin{cases}
%				i^2 = (0,1)\cdot (0,1) = (0\cdot 0 - 1\cdot 1, 0\cdot 1 + 0\cdot 1) = (-1,0) = -1 \\
%				(x,y) = (x,0) + (0,y) = (x,0) + y\cdot i = x + yi
%				\end{cases}
%			\]
%			Rechnen in $\C$:
%			\begin{align*}
%				(x_1,y_1)\cdot (x_2,y_2) &= (x_1 +iy_1)\cdot (x_2 + iy_2)\\
%				&= x_1x_2 + ix_1y_2 + iy_1x_2 + (i)^2y_1y_2\\
%				&= x_1x_2 - y_1y_2 + i(x_1y_2 + x_2y_1)
%			\end{align*}
%			oder
%			\begin{align*}
%				\f 1{x+iy} = \f 1{x+iy}\cdot \f {x-iy}{x-iy} = \f{x-y}{x^2-(iy)^2} = \f x{x^2+y^2} + i \f{-y}{x^2+y^2}
%			\end{align*}
%			Realteil: $\Re(x+iy) = x\in \R$ \\
%			Imaginärteil: $\Im(x+iy) = y\in\R$
%		\item
%			Gaußsche Zahlenebene:
%	\end{enumerate}
%\end{nt}
%
%\begin{df}
%	\label{df:1.3}
%	\begin{enumerate}
%		\item 
%			Für $z=x+iy$ heißt
%			\[
%				\_z = x-iy
%			\]
%			\emph{konjugiert komplexe Zahl}
%		\item
%			\[
%				|z| = \sqrt{x^2+y^2} = \sqrt{z\cdot \_z}
%			\]
%		\item
%			Polardarstellung: $z=x+iy = |z|(\cos \phi + i\sin\phi)$ wobei $\phi = \arg(z)$ (Argument von $z$) eindeutig durch
%			\[
%			-\pi \le \phi \le \pi, \qquad \cos\phi = \f x{\sqrt{x^2+y^2}}, \qquad \sin\phi = \f y{\sqrt{x^2+y^2}}
%			\]
%			Rechnen mit Polardarstellung:
%			\begin{align*}
%				z_1 \cdot z_2 &= |z_1|\cdot |z_2|\cdot (\cos(\phi_1+\phi_2) + i\sin(\phi_1+\phi_2))\\
%				z^n &= |z|^n (\cos(n\phi) + i\sin(n\phi))
%			\end{align*}
%			Lösung von $z^n=r(\cos\psi + i\sin\psi)$ ist gegeben durch
%			\begin{align*}
%				|z| &= r^{\f 1n}\\
%				\phi &= \f \psi n + \f {2\pi k}n \qquad k\in 0,1,\dotsc, n-1
%			\end{align*}
%	\end{enumerate}
%\end{df}
%
%\begin{st}
%	\label{st:1.4}
%	$(\C, +, \cdot, |\cdot|)$ ist ein \emph{bewerteter Körper}, d.h. für $|\cdot|: \C \to \R$ gelten:
%	\begin{enumerate}
%		\item $|z| \ge 0 \land (|z| = 0 \iff z = 0)$
%		\item $|z_1\cdot z_2| = |z_1|\cdot |z_2|$
%		\item $|z_1+z_2| \le |z_1| + |z_2|$
%	\end{enumerate}
%	\begin{proof}
%		Beweis durch Nachrechnen
%	\end{proof}
%	\begin{note}
%		Außerdem gilt die Dreiecksungleichung nach unten:
%		\[
%			|z_1 \pm z_2| \ge ||z_1| - |z_2||
%		\]
%		\begin{proof}
%			Beweis durch Nachrechnen.
%		\end{proof}
%	\end{note}
%\end{st}
%
%\begin{df}
%	\label{df:1.5}
%	Eine Folge $(z_n)$ in $\C$ \emph{konvergiert} gegen $z\in \C$, falls
%	\[
%		\forall \eps > 0 \exists \N_\eps\in \N \forall n\ge \N_\eps : |z_n -z| < \eps
%	\]
%	Man schreibt dann $z = \lim_{n\to \infty} z_n$ oder $z_n \to z$ $(n\to \infty)$.
%\end{df}
%
%\begin{st}
%	\label{st:1.6}
%	Es gelte $z_n\to z$ und $w_n \to w$ in $\C$.
%	Dann gilt
%	\begin{enumerate}
%		\item $z_n \pm w_n \to z \pm w$ 
%		\item $z_n\cdot w_n \to z\cdot w$
%		\item
%			Falls $w\neq 0$ und $w_n' = \begin{cases} 1 & w_n=0 \\ w_n & \text{sonst}\end{cases}$, dann gilt
%			\[
%				\f {z_n}{w_n} \to \f zw
%			\]
%		\item $z_n\to z \quad\iff \Re z_n \to \Re z \land  \Im z_n \to \Im z$
%
%	\end{enumerate}
%\end{st}
%
%\begin{df}
%	\label{df:1.7}
%	\begin{enumerate}
%		\item 
%			Sei $r> 0, z_0\in \C$.
%			\[
%				K_r(z_0) := \{z\in \C : |z-z_0| < r\}
%			\]
%			heißt \emph{offene Kreisscheibe} um $z_0$ mit Radius $r$.
%		\item
%			Eine Teilmenge $O \subset \C$ heißt \emph{offen}, falls
%			\[
%				\forall z\in 0 \exists r_z > 0 : K_{r_z}(z) \subset O
%			\]
%			Eine Teilmenge $A \subset \C$ heißt \emph{abgeschlossen}, falls $\C\setminus A$ offen ist.
%
%			Beliebige Vereinigungen und endliche Schnitte offener Mengen sind offen.
%			Beliebige Schnitte und endliche Vereinigungen abgeschlossener Mengen sind abgeschlossen.
%		\item
%			Für eine beliebige Teilmenge $M\subset \C$ ist
%			\[
%				\mathring M := \bigcup_{O\in \{O\subset \C: O \text{ offen} \land O \subset M\}} O
%			\]
%			das \emph{Innere} von $M$ (die größte offene Menge $O\subset M$).
%			\[
%				\_M := \bigcap_{A\in \{A\subset \C: A \text{ abgeschlossen} \land M \subset A\}} A
%			\]
%			der \emph{Abschluss} von $M$ (die kleinste abgeschlossene Menge $A$ mit $M\subset A$).
%	\end{enumerate}
%\end{df}
%
%\begin{ex}
%	\label{ex:1.8}
%	\begin{enumerate}
%		\item 
%			$\emptyset, \C$ sind offen und abgeschlossen. 
%			Alle anderen Teilmengen von $\C$ sind entweder offen oder abgeschlossen oder keins von beidem.
%		\item
%			$K_r(z_0)$ ist offen. $\_{K_r(z_0)}=\{z\in \C : |z -z_0| \le r\}$
%		\item
%			$\R\subset \C$, $\R$ ist nicht offen, betrachte $\C\setminus \R$.
%			$\R\subset \C$ ist abgeschlossen, betrachte $\C\setminus \R$.			
%	\end{enumerate}
%\end{ex}
%
%\begin{df}
%	\label{df:1.9}
%	Sei $O\subset \C$ offen, $f: O \to \C$.
%	Dann heißt $f$ \emph{stetig} in $z_0\in O$, falls
%	\[
%		\forall \eps> 0 \exists \delta_\eps > 0 \forall z\in O : |z-z_0| < \delta \implies |f(z) -f(z_0)| < \eps
%	\]
%	oder äquivalent
%	\[
%		\forall (z_n) \text{ Folge in } O: z_n \to z \implies f(z_n) \to f(z_0)
%	\]
%	$f$ heißt \emph{stetig}, falls $f$ in jedem $z_0\in O$ stetig ist.
%\end{df}
%
%\begin{st}
%	\label{st:1.10}
%	\begin{enumerate}
%		\item 
%			Seien $f,g: O\to \C$, $z_0\in O$, $f,g$ stetig in $z_0$.
%			Dann sind $f\pm g$, $f\cdot g$ und (falls $g(z_0)\neq 0$) $\f fg$ stetig in $z_0$.
%		\item
%			Sei $f:O\to \C$ stetig in $z_0\in O$ und $g:\tilde O \to \C$ stetig in $f(z_0)$, $f(O) \subset \tilde O$.
%			Dann ist $g\circ f$ stetitg in $z_0$
%	\end{enumerate}
%\end{st}
%\begin{proof}
%Beweis über Folgendefinition der Stetigkeit.
%\end{proof}
%
%\begin{nt}
%	\label{nt:1.10}
%	Stetigkeit genauso für $f:M\to \C$ und beliebiger Menge $M\subset \C$.
%\end{nt}
%
%\begin{df}[Funktionenfolgen]
%	Sei $M\subset \C$, $f_n,f: M \to \C$.
%	\begin{enumerate}[1)]
%		\item 
%			$(f_n)$ heißt \emph{punktweise konvergent} gegen $f$ auf $M$, falls
%			\[
%				\forall z\in M \forall \eps > 0 \exists N_{\eps,z} \in \N \forall n>N_{\eps,z}: |f_n(z) - f(z)| < \eps
%			\]
%		\item
%			$(f_n)$ heißt \emph{gleichmäßig konvergent} gegen $f$ auf $M$, falls
%			\[
%				\forall \eps>0 \exists N_{\eps} \in \N \forall n>N_{\eps} \forall z\in M : |f_n(z)-f(z)| < \eps
%			\]
%	\end{enumerate}
%\end{df}
%
%\begin{st}
%	\label{st:1.12}
%	Seien $f_n:M\to \C$ stetig, $(f_n)$ gleichmäßig konvergent auf $M$ gegen $f$. 
%	Dann ist $f$ auch stetig auf $M$.
%	\begin{proof}
%		Seien $z_0\in M, \eps > 0$ fest.
%		\begin{align*}
%			|f(z)-f(z_0)| &\le |f(z)-f_n(z)| + |f_n(z)-f_n(z_0) + |f_n(z)-f(z_0)|
%		\end{align*}
%		1. Wähle $N_{\eps}$, so, dass $|f(z)-f_n(z)| < \frac \eps 3$ für $n>N_{\eps}$. \\
%		2. Setze jetzt aber konkret $n:= N_{\eps}+1$. Mit der Dreiecksgleichung, Stetigkeit und geschicktes Addieren mit der Null ergibt sich.
%		
%		 \[
%		  |f(z)-f(z_0)|\le \underbrace{|f(z)-f_m(z)|}_{\frac\eps 3} + \underbrace{|f_m(z)-f_m(z_0)|}_{<\frac \eps 3, |z-z_0|<\delta_{\frac{\eps}{3}}}+\underbrace{|f_m(z_0)-f(z_0)|}_{<\frac{\eps}{3}}<\eps \text{für } |z-z_0|<\delta
%		 \]
%			
%	\end{proof}
%\end{st}
%
%\begin{df}
%	\label{df:1.13}
%	Eine Reihe $\sum_{n=0}^\infty a_n$ heißt \emph{absolut konvergent}, falls $\sum_{n=0}^\infty |a_n|$ konvergiert.
%\end{df}
%
%\begin{thm}[Weierstraß-Kriterium]
%	\label{thm:1.14}
%	Ist $\sum_{n=0}^\infty a_n$ mit $a_n \ge 0$ konvergent und gilt $f_n:M\to \C$ und $f_n(z)| \le a_n$, so ist die Reihe
%	\[
%		\sum_{n=0}^\infty f_n(z)
%	\]
%	gleichmäßig konvergent auf $M$ und absolute konvergent für $z\in M$.
%\end{thm}
%
%\begin{ex}
%	$M:= \_{K_2(0)} \subset \C$, $f_n(z) := \f {z^n}{(n+1)^22^n}$.
%	Wähle $a_n:= \f 1{(n+1)^2}$, dann folgt
%	\[
%		\begin{cases}
%			\sum_{n=0}^\infty \f 1{(n+1)^2} < \infty \\
%			|f_n(z)| \le a_n
%		\end{cases}
%	\]
%	Also ist $g(z) = \sum_{n=0}^\infty f_n(z)$ nach \ref{thm:1.14} und \ref{st:1.12} stetig auf $\_{K_2(0)}$.
%\end{ex}
%
%\begin{df}[Potenzreihen]
%	Sei $(a_n)$ eine Folge in $\C$
%	\[
%		R := \f 1{\limsup_{n\to \infty}\sqrt[n]{|a_n|}} \qquad\left (\f 10 := \infty, \f 1\infty := 0 \right )
%	\]
%	dann konvergiert die Potenzreihe
%	\[
%		f(z) = \sum_{n=0}^\infty a_n(z-z_0)^n
%	\]
%	für $|z-z_0|<R$ und divergiert für $|z-z_0|>R$.
%	Sei konvergiert gleichmäßig auf jedem Kreis $\_{K_r(z_0)}$ mit $0<r<R$.
%	Insbesondere ist $f$ stetig auf $K_R(z_0)$.
%
%	Falls $(|\f {a_{n+1}}{a_n}|)$ konvergiert, gilt
%	\[
%		R = \f 1{\lim |\f{a_{n+1}}{a_n}|}
%	\]
%\end{df}
%
%\begin{df}
%	Wir definieren
%	\begin{align*}
%		e^z &:= \sum_{n=0}^\infty \f {z^n}{n!}\\
%		\cos z &:= \sum_{n=0}^\infty (-1)^n \f {z^2n}{(2n)!} \\
%		\sin z &:= \sum_{n=0}^\infty (-1)^n \f{z^{2n+1}}{(2n+1)!}
%	\end{align*}
%	\begin{proof}
%		Für die Konvergenz:
%		\begin{align*}
%			\f {a_{n+1}}{a_n} = \f {\f1{(n+1)!}}{\f 1{n!}} = \f {1}{n+1} \to 0
%		\end{align*}
%		Also $R=\infty$, die Potenzreihe konvergiert auf ganz $\C$.		
%	\end{proof}
%\end{df}
%
%\begin{thm}[Cauchy-Produkt von Reihen]
%	\label{thm:1.18}
%	Ist $\sum_{n=0}^\infty a_n$ absolut konvergent, $\sum_{n=0}^\infty b_n$ konvergent in $\C$, so gilt
%	\[
%		\left( \sum_{n=0}^\infty a_n\right) \left(\sum_{n=0}^\infty f_n\right) = \sum_{n=0}^\infty \sum_{k=0}^n a_k b_{n-k}
%	\]
%	\begin{proof}
%		Setze $A:= \sum_{n=0}^\infty$, $B:= \sum_{n=0}^\infty b_n, b_l := \sum_{n=0}^l b_n$.
%		\begin{align*}
%			\sum_{n=0}^N \sum_{k=0}^n a_k b_{n-k}
%				&= \sum_{n=0}^N a_n \sum_{m=0}^{N-n} b_m\\
%				&= \sum_{n=0}^N a_n (B_{N-m}-B + B)\\
%				&= \sum_{l=0}^N a_{N-l} (B_l -B) + \underbrace{\sum_{n=0}^N a_nB}_{\to AB}\\
%		\end{align*}
%		Weiterhin ist für entsprechendes $N_\eps$
%		\begin{align*}
%			\left|\sum_{l=0}^N a_{N-l} (B_l-B) \right| \\
%				&= \left|\sum_{l=0}^{N_\eps} a_{N-l}(B_l-B) + \sum_{l=N_\eps +1}^N a_{N-l}(B_l-B)\right|\\
%				&\le \left|\max_{0\le l \le N_\eps} |B_l-B|\sum_{n=N-N_\eps}^\infty |a_n| \right| \\ &+     \left|\sup_{l\ge N_{\eps}+M} |B_l-B| \sum_{l=N_\eps+1}^N |a_{N-l}| |\right| \qquad n= N-l, \text{wähle hier $N_\eps$} \\
%				&\le \f \eps 2  +  \left|\sup_{l\ge N_{\eps}+M} |B_l-B| \sum_{n=0}^\infty |a_{n}| \right| \\
%				&<     \eps
%		\end{align*}
%		für $N-N_\eps > \tilde N_\eps$ bzw. für  $N>\tilde N_\eps + N_\eps$.
%	\end{proof}
%\end{thm}
%
%\begin{prop}
%	\label{prop:1.19}
%	Es gilt
%	\[
%		e^{z+w}  = e^z\cdot e^w
%	\]
%	\begin{proof}
%		\begin{align*}
%			e^ze^w &= \left(\sum_{n=0}^\infty \f{z^n}{n!}\right) \left( \sum_{n=0}^\infty \f {w^n}{n!}\right) \\
%				&= \sum_{n=0}^\infty \sum_{k=0}^n \f {z^k}{k!} \f {w^{n-k}}{(n-k)!} \\
%				&= \sum_{n=0}^\infty \f 1{n!} \sum_{k=0}^n \binom{n}{k}z^k w^{n-k} \\
%				&= \sum_{n=0}^\infty \f {(z+k)^n}{n!} = e^{z+w}
%		\end{align*}
%	\end{proof}
%\end{prop}
%
%\begin{nt}
%	\label{1.20}
%	Aus der Taylorreihe folgt außerdem, dass $e^z, \sin z, \cos z$ für $z\in \R$ die selben Funktionen sind, wie aus der Schule bekannt.
%\end{nt}
%
%\begin{prop}
%	\label{1.21}
%	\begin{enumerate}[1)]
%		\item 
%			$e^{iz} = \cos z + i \sin z$ für $z\in \C$ (Eulersche Formel)
%			\begin{proof}
%				\[
%					e^{iz} = \sum_{n=0}^\infty \f {(iz)^n}{n!} = \sum_{k=0}^\infty \f {(iz)^{2k}}{(2k)!} + \sum_{k=0}^\infty \f {(iz)^{2k+1}}{(2k+1)!} =
%\sum_{k=0}^\infty (-1)^k\f {z^{2k}}{(2k)!} + \sum_{k=0}^\infty \f (-1)^k {z^{2k+1}}{(2k+1)!}
%\]
%			\end{proof}
%		\item
%			Mit der Polardarstellung $z = r(\cos \phi + i \sin \phi) = r e^{i\phi}$ gilt
%			\begin{align*}
%				z^n &= r^n e^{in\phi}\\
%				z_1 z_2 &= r_1 r_2 e^{\phi_1 + \phi_2}\\
%				\f {z_1}{z_2} &= \f {r_1}{r_2} e^{i(\phi_1-\phi_2)}
%			\end{align*}
%	\end{enumerate}
%\end{prop}
%
%\begin{df}
%	\label{df:1.22}
%	\begin{enumerate}[1)]
%		\item 
%			$f: O \to \C$ heißt \emph{differenzierbar} in $z_0\in O$, falls
%			\[
%				f'(z_0 := \lim_{\substack{z\to z_0 \\ z\neq z_0}}\f {f(z) -f(z_0)}{z-z_0}
%			\]
%			existiert.
%			$f'(z_0)$ heißt \emph{Ableitung} von $f$ in $z_0$.
%		\item
%			$f$ heißt \emph{differenzierbar}, falls $f$ in jedem $z_0\in O$ differenzierbar ist.
%			$f':O\to \C$ heißt \emph{Ableitung(sfunktion)} von $f$.
%	\end{enumerate}
%\end{df}
%
%\begin{ex*}
%	\begin{enumerate}[1)]
%		\item 
%			$f: \C \to \C : z\mapsto c$, dann ist $f'=0$.
%		\item
%			$f: \C \to \C : z\mapsto z$, dann ist $f'=1$.
%		\item
%			ohne Beweis: $f:\C \to \C :z\mapsto z^m$, dann ist $f'(z) = nz^{n-1}$.
%		\item
%			$f: \C\to \C : z\mapsto |z|^2$ ist in $z_0\neq 0$ nicht differenzierbar.
%			\begin{proof}
%				Sei $z_0 = x_0+iy_0 \neq 0$ und $z_h:= x_0+h +iy_0$, dann ist
%				\[
%					\f {|z_h|^2 -|z_0|^2}{z_h-z_0} = \f {2hx_0+h^2}h \to 2x_0
%				\]
%				Sei jetzt $z_h := x_0 + i(y_0+h)$, dann ist
%				\[
%					\f {|z_h|^2 -|z_0|^2}{z_h-z_0} := \f {2hx_0+h^2}{ih} \to -2ix_0
%				\]
%			\end{proof}
%	\end{enumerate}
%\end{ex*}
%
%\begin{st}
%	\label{st:1.24}
%	Sei $O\subset \C$ offen, $f:O\to \C$, $z_0\in O$.
%	Dann sind äquivalent
%	\begin{enumerate}[(i)]
%		\item
%			$f$ ist differenzierbar in $z_0$ mit Ableitung $f'(z_0)$
%		\item
%			$\exists f'(z_0) \in \C : f(z) =f(z_0)+f'(z_0)(z-z_0)+o(|z-z_0|)$
%	\end{enumerate}
%	\begin{proof}
%		\begin{enumerate}
%			\item 
%				\[
%					\iff \f{f(z)-f(z_0)}{z-z_0}-f'(z_0) \to 0
%				\]
%				\fixme
%		\end{enumerate}
%	\end{proof}
%\end{st}
%
%\begin{st}
%	\label{1.25}
%	\begin{enumerate}[1)]
%		\item 
%			$f$ differenzierbar in $z_0$ $\implies$ $f$ stetig in $z_0$.
%		\item
%		\item
%		\item
%		\item
%	\end{enumerate}
%\end{st}
%
%\begin{ex}
%	\begin{enumerate}[1)]
%		\item 
%			Polynomfunktionen sind differenzierbar auf $\C$
%		\item
%			Gebrochenrationale Funktionen sind auf ihrem Definitionsbereich differenzierbar
%	\end{enumerate}
%\end{ex}
%
%\begin{df}
%	\begin{enumerate}[1)]
%		\item 
%			Sei $\gamma\in C^1([a,b]\to \C)$.
%			Dann heißt $\gamma$ \emph{Weg} von $z_1=\gamma(a)$ nach $z_2=\gamma(b)$.
%			Falls $z_1=z_2$, heißt $\gamma$ \emph{geschlossen}.
%		\item
%			Sei $\gamma$ eine Weg, $f\in C^1(\Re(\gamma)\to \C)$.
%			Dann heißt
%			\begin{align*}
%				\int_\gamma f(z) dz := \int_a^b f(\gamma(t)) \gamma'(t) dt\\
%				=: \int_a^b(\Re f(\gamma(t)) \Re \gamma'(z) - \Im f(\gamma(t)) \Im \gamma'(t))dt
%				+ i\int_a^b(\Im f(\gamma(t)) \Re \gamma'(t) + \Re f(\gamma(t)) \Im \gamma'(t))dt
%			\end{align*}
%			das \emph{Integral} längs $\gamma$.
%	\end{enumerate}
%\end{df}
%
%\begin{nt}
%	\label{st:1.28}
%	\begin{enumerate}[1)]
%		\item 
%			Ist $\tilde \gamma$ eine andere Parametrisierung des Weges $\gamma$, sodass $\phi'\in C^1([a',b']\to [a,b])$
%			\begin{align*}
%				\tilde \gamma &= (\gamma \circ \phi)(s) \qquad a'\le s\le b'\\
%				\phi(a') &= a\\
%				\phi(b') &= b
%			\end{align*}
%			Dann folgt aus der Substitutionsregel für reelle Integration
%			\begin{align*}
%				\int_\gamma f(z) dz &= \int_a^b f(\gamma(t)) \gamma'(t) dt \\
%				&= \int_{a'}^{b'} \underbrace{f(\gamma(\phi(s)))}_{f(\tilde \gamma(s)} \underbrace{\gamma'(\phi(s))\phi'(s)}_{\f d{dx}\tilde \gamma(s)=\f d{dx}(\gamma\circ \phi)(s)} ds\\
%				&= \int_{\tilde \gamma} f(z) dz
%			\end{align*}
%		\item
%			$-\gamma := $  derselbe Weg, aber entgegengesetzt orientiert:
%			\[
%				-\gamma(t) := \gamma(a+b-t) \qquad a\le t\le b
%			\]
%			Dann ist auch (selber nachrechnen)
%			\[
%				\int_{-\gamma}f(z) dz = - \int_\gamma f(z) dz
%			\]
%		\item
%			Jede Kurve kann so umparametrisiert werden, dass $a=0, b=1$ gilt.
%		\item
%			Verbindungsstrecke $\gamma=[z_1,z_2]$ ist definiert als
%			\[
%				\gamma(t) = z_1 + t(z_2-z_1) \qquad 0\le t\le 1
%			\]
%		\item
%			Verallgemeinerung: Ein Weg kannn auch nur stückweise $C^1$ sein:
%			\[
%				\gamma\in C([a,b]\to \C)
%			\]
%			(stetig).
%			Es existieren $t_0=a<t_1<t_2<\dotsb<t_n=b$ so dass
%			\[
%				\gamma_j = \gamma\big|_[t_{j-1},t_j] \in C^1([t_{j-1},t_j]\to \C)
%			\]
%			d.h. die Ableitung, darf endlich viele Sprünge haben.
%			Dann ist
%			\[
%				\int_\gamma f(z) dz := \sum_{j-1}^n \int_{\gamma_j}f(z) dz
%			\]
%	\end{enumerate}
%\end{nt}
%
%\begin{ex}
%	\begin{enumerate}[1)]
%		\item 
%			$f(z)=z^3$, Verbindungsstrecke: $\gamma=[0,1+i]$
%			Dann ist
%			\[
%				\int_{[0,1+i]}z^3 dz = \int_0^1(t(1+i))^3(1+i) dt = (1+i)^4 \int_0^1 t^3 dt = \f {(1+i)^4}4
%			\]
%		\item
%			$f(z)=\f 1z$, $\gamma(t)=e^{it}$, $0\le t\le 2\pi$.
%			Das ist wegen $\gamma(0)=\gamma(2\pi)$ ein geschlossener Weg.
%			\[
%				\int_\gamma \f 1z dz = \int_0^{2\pi}\f 1{e^{it}} ie^{it} dt = 2\pi i \neq 0 
%			\]
%			Obwohl die Kurve geschlossen ist, kommt nicht $0$ heraus!
%	\end{enumerate}
%\end{ex}
%
%\begin{st}
%	\label{st:1.30}
%	Sei $O\subset \C$ offen, $F:O\to \C$ differenzierbar, $f=F'$ in $O$ (d.h. $F$ ist eine \emph{Stammfunktion} von $f$).
%	Ist $\gamma$ ein Weg in $O$, so gilt
%	\[
%		\int_\gamma f(z) dz f F(\gamma(b)) - F(\gamma(a))
%	\]
%	\begin{proof}
%		\begin{align*}
%			\int_\gamma f(z) dz &= \int_a^b f(\gamma(t)) \gamma'(t) dt \\
%			\intertext{
%				Wegen
%				\begin{align*}
%					\f d{dt} F(\gamma(t)) &= \lim_{\substack{s \to t \\ s\neq t}} \f {F(\gamma(s) -F(\gamma(t))}{\gamma(s) - \gamma(t)} \cdot \f {\gamma(s)-\gamma(t)}{s-t} \\
%				&= F'(\gamma(t)) \cdot \gamma'(t) \\
%				&= f(\gamma(t)) \gamma'(t)
%				\end{align*}
%				gilt weiter
%			}
%			&= \int_a^b (\f d{dt} \Re F( \gamma(t)) + i \f d{dt} \Im F(\gamma(t))) dt \\
%			&= \int_a^b \Re F(\gamma(b)) + i \Im F(\gamma(b))\\
%			&= F(\gamma(b)r - F(\gamma(a))
%		\end{align*}
%	\end{proof}
%\end{st}
%
%\begin{prop}
%	\label{prop:1.31}
%	\begin{enumerate}[1)]
%		\item 
%			Besitzt $f$ eine Stammfunktion und ist $\gamma$ geschlossen, folgt $\int_\gamma f(z)dz = 0$.
%		\item
%			$f:\C\setminus \{0\} \to \C : z\mapsto \f 1z$ besitzt kein Stammfunktion (siehe \ref{ex:1.29} 2.)
%	\end{enumerate}
%\end{prop}
%
%\begin{st}
%	\label{st:1.32}
%	Sei $\gamma$ Weg, $f\in C(\im(\gamma) \to \C)$.
%	Dann ist
%	\[
%		\left| \int_\gamma f(z) dz \right| \le \int_a^b |f(\gamma(t))| |\gamma'(t)| dt \le
%\le \max_{a\le t \le b} |f(\gamma(t))| \int_a^b |\gamma'(t)|| dt
%	\]
%	\begin{proof}
%		\begin{align*}
%			\left| \int_\gamma f(z) dz \right| &= e^{i\phi} \int_\gamma f(z) dz \qquad \phi= -\arg\left(\int_\gamma f(z)dz\right)\\
%			&= \int_a^b e^{i\phi} f(\gamma(t) \gamma'(t) dt \\
%			&= \int_a^b \Re\left(e^{i\phi} f(\gamma(t)) \gamma'(t)\right)dt + i \underbrace{\int_a^b \Im\left( \dotso\right)dt}_{=0} \\
%			&\le \int_a^b |\Re e^{i\phi} f(\gamma(t)) \gamma'(t)| dt\\
%			&\le \int_a^b |e^{i\phi} f(\gamma(t)) \gamma'(t)| dt\\
%			&\le \int_a^b |e^{i\phi}| |f(\gamma(t))| |\gamma'(t)| dt\\
%			&\le \int_a^b |f(\gamma(t))| |\gamma'(t)| dt\\
%		\end{align*}
%		Der Beweis funktioniert auch, falls $\gamma$ nur stückweise $C^1$ ist.
%	\end{proof}
%\end{st}
%
%\begin{df}
%	\label{df:1.33}
%	Man nennt
%	\[
%		L(\gamma) := \int_a^b |\gamma'(t)| dt
%	\]
%	\emph{Länge} von $\gamma$.
%	Damit wird \ref{st:1.32} zu
%	\[
%		\left| \int_\gamma f(z) dz \right| \le L(\gamma) \cdot \max_{a\le t\le b}|f(\gamma(t))|
%	\]
%\end{df}
%
%\begin{ex}
%	\begin{enumerate}[1)]
%		\item 
%			$\gamma=[z_1,z_2]$, $\gamma(t) = z_1+t(z_2+z_1)$, dann ist
%			\[
%				L(\gamma) = \int_0^1 |z_2-z_1| dt = |z_2-z_1|
%			\]
%		\item
%			\begin{enumerate}[a)]
%				\item 
%					$\gamma(t) = e^{it}$, $0\le t\le 2\pi$.
%					Es gilt nach 1.29 und 1.32
%					\[
%						|2\pi i| = \left| \int_\gamma \f 1z dz \right| \le \max_{\substack{z=e^{it}\\0\le t \le 2\pi}} \cdot \int_0^{2\pi} |ie^{it}| dt = 2\pi
%					\]
%				\item
%					Es gilt
%					\[
%						|0| = \left|\int_\gamma z dz\right| \le \max_{\substack{z=e^{it}}} \cdot 2\pi = 2\pi
%					\]
%			\end{enumerate}
%	\end{enumerate}
%\end{ex}
%
%\begin{st}
%	Sei $(a_n)$ eine Folge in $\C$ mit Konvergenzradius $R:= \f {1}{\limsup_{n\to \infty}\sqrt[n]{|a_n|}}>0$ und 
%	\[
%		f(z) := \sum_{n=0}^\infty a_n z^n
%	\]
%	ür $|z|<R$.
%	Dann ist $f$ in $K_R(0)$ beliebig oft differenzierbar mit
%	\[
%		f'(z) = \sum_{n=1}^\infty a_n nz^{n-1}, \qquad f''(z) = \sum_{n=2}^\infty a_nn(n-1)z^{n-2}
%	\]
%	\begin{proof}
%		Sei $g(z) := \sum_{n=1}^{\infty}a_n nz^{n-1}, R':= \f 1{\limsup_{n\to \infty}\sqrt[n]{a_n n}}$ (Konvergenzradius für $z$ ist der selbe wie für $g(z)$).
%		\begin{enumerate}[1)]
%			\item 
%				$R'=R$ wegen $\lim_{n\to \infty} \sqrt[n]{a_n} = 1$.
%				D.h. $g$ ist auf demselbe $K_R(0)$ definiert, wi $f$.
%			\item
%				$g=f'$: Seien $w\in K_R(0), z\in K_R(0)\setminus \{w\}$.
%				\begin{align*}
%					\f {f(z) -f(w)}{z-w} - g(w) &= \sum_{n=1}^\infty a_n \left( \f {z^n-w^n}{z-w} - nw^{n-1}\right) \\
%					&= \sum_{n=1}^\infty a_n \left( \f 1{z-w}\int_{[w,z]}nu^{n-1}-nw^{n-1}dt\right) \\
%					|\cdot| &\le \sum_{n=1}^\infty |a_n| \f 1{|z-w|} \cdot L([w,z]) \cdot n|z^{n-1}-w^{n-1}| \\
%					&= \begin{cases} \to 0  & z\to w\\
%						\le n|z|^{n-1}+ n|w|^{n-1} & \text{sonst}
%					\end{cases}
%				\end{align*}
%				Die Reihe konvergiert also gleichmäßig in $K_r(0)$ für jedes $r<R$.
%				Also ist der Grenzwert $z\to w$ und die Reihe vertauschbar.
%				Damit gilt
%				\[
%					\f {f(z) -f(w)}{z-w}- g(w) \to 0
%				\]
%		\end{enumerate}
%	\end{proof}
%\end{st}
%



\section{Vektoranalysis}



\begin{proof}
	\begin{enumerate}[1)]
		\item
			Es gelte o.B.d.A
			\[
				\omega = a(x) dx_J \qquad J \in \scr G^{(k)}
			\]
		\item
			Lokalisierung:
			Sei $A(S)$ orientierter Atlas, sortiert wie in \ref{8.6}.
			\begin{align*}
				\int_S d\omega &= \sum_{j=1}^N \int_{U_j} \psi_j d\omega \\
				&= \sum_{i=1}^n \sum_{j=1}^N \int_{U_j}\d_{x_i}(\psi_j a) dx_i \wedge dx_J \\
				&= \sum_{i=1}^n \underbrace{\sum_{j=1}^N \int_{U_j}(\d_{x_i}\psi_j) a xx_i \wedge dx_J}_{\int_S (\underbrace{\sum_{j=1}^N \d_{x_i}\psi_j)}_{\d x_i \sum \psi_j = \d x_{i}1 = 0}a dx_i \wedge dx_J} \\
				&= \sum_{j=1}^N \int_{U_j}d(\psi_j \omega)
			\end{align*}
			Wir zeigen jetzt
			\[
				\int_{U_j} d(\psi_j \omega) = \begin{cases}
					\int_{\tilde U_j = U_j \cap \d S} \psi_j \omega & 1 \le j \le L \\
					0& L+1 \le j \le N
				\end{cases}
			\]
			Dann gilt, da $\{\psi_1,\dotsc, \psi_L\}$ eine Zerlegung der Eins ist für $\d S$ mit Altas $\{(\tilde \phi_j, \tilde U_j), 1 \le j \le L \}$
			\[
				\int_S d\omega = \sum_{j=1}^L \int_{\tilde U_j} \psi_j \omega  = \int_{\d S} \omega
			\]
		\item
			Sei $j= \{1,\dotsc, N\}$ fest, $\phi_j = (g_1,\dotsc, g_n)$, $D=K_1^{(k+1)}(0)$ falls $1\le j \le L$, $D=K_1^{(k+1)}(0)$ sonst.
			Also
			\begin{align*}
				\int_{U_j} d(\psi_j \omega) &= \sum_{i=1}^n \int_{y\in D} \d_{x_i} (\psi_j a)(\phi_j(y))) \det(\tf{\d(g_i,g_J)}{\d y}) dy 
				\intertext{
					Entwickeln der Determinante nach der ersten Zeile ergibt
				}
				&= \sum_{i=1}^n \sum_{l=1}^{k+1} (-1)^{k+1} \int_{y\in D} \d_{x_i}(\psi_j a)(\phi_j(y)) \tf{\d_{g_i}}{\d_{y_l}}(y) \cdot \det \bigg( \f{\d(g_{i_1},\dotsc,g_{i_k})}{\underbrace{\d(y_1,\dotsc, y_{l-1},y_{l+1},\dotsc, y_{k+1})}_{y^{(l)}}} dy \\
				&= \sum_{l=1}^{k+1} (-1)^{l+1} \int_{y \in D} \d_{y_l}((\psi_j a)\circ \phi_j)(y) \det(\dotsc) dy \\
				&= \sum_{l=1}^{k+1} (-1)^{l+1} \int_{y^{(l)}\in K_1^{(k)}(0)} \int_{y_l=- \sqrt{1 - |y^{(l)}|^2}}^{y_l = + \sqrt{1-|y^{(l)}|^2} \text{ oder $y_l=0$ falls $1\le j \le L$}} \dotsc d y_l dy^{(l)}
				\intertext{
					Integriere jetzt partiell und beachte: $(\psi_j a)\circ \psi = 0$ für $y_l = \pm \sqrt{1-|y^{(l)}|^2}$ da $\supp \psi_j \le U_j$.
					Außerdem $\sum_{l=1}^{k+1} (-1)^{l+1} \d y_l (\det (\dotsc)) = 0$ (Nachrechnen).
				}
				&= \begin{cases}
					0 & L+1 \le j \le N \\
					\int_{\tilde U_j} \psi \omega \stackrel{\tilde U_j = \tilde \phi_j(K_1^{(k)}(0))}= (-1)^{i+1} \int_{y^{(i)}\in K_1^{(k)}(0)} (\psi_j a) \circ \phi_j (0,y^{(1)}) \det(\tf{\d g_J}{\d y^{(1)}}) dy^{(1)} & 
				\end{cases}
			\end{align*}
	\end{enumerate}
\end{proof}

\begin{nt*} \label{8.8}
	\begin{enumerate}[1)]
		\item
			Im Fall $k=0, \omega = a(x), S = \phi([\alpha,\beta]), \d S = \{\phi(\alpha), \phi(\beta)\}$.
			Nach \ref{8.4}:
		\item
			Satz von Stokes gilt auch für $S= \_{O}$, d.h. $S \subset O$ nicht notwendig.
		\item
			Folgerungen
			\begin{enumerate}[a)]
				\item
					Der Satz von Gauß-Ostrogradski:
					Sei $f\in C^1(O\to \R^n)$.
					\[
						\int_{S} \nabla \cdot f dV^{(n)} = \int_{\d S} \<f,n_0\> dV^{(n-1)}
					\]
					Setze dazu $\omega := \sum_{j=1}^n (f_j dx_1 \wedge \dotsc \wedge dx_{j-1} \wedge dx_{j+1} \wedge \dotsc \wedge dx_n)$
				\item
					Klassischer Satz von Stokes:
					Sei $S \subset \R^3$ eine Mannigfaltigkeit der Dimension $2$ und $f \in C^1(S \to \R^3)$, dann gilt
					\[
						\int_S (\nabla \times f) dV^{(2)} = \int_{\d S}\<f,t_0\> dV^{(1)}
					\]
					Setze dazu $\omega := f_1 dx_1 + f_2 dx_2 + f_3 dx_3$.
			\end{enumerate}
	\end{enumerate}
\end{nt*}


\chapter{Gewöhnliche Differentialgleichungen}


\section{Funktionalanalysis}

\begin{def} \label{1.1}
	Sei $(B,+,\cdot)$ ein linearer Raum, d.h. Vektoraum und $\|\cdot\|: B \to \R$ eine Norm (d.h. $\|u\| \ge 0$, $\|u\|=0 \iff u = 0$, $\|u+v\| \le \|u\| + \|v\|$) (Dreiecksungleichung), $\|\alpha u \| = |\alpha| \|u\|$)
	Ist $B$ vollständig, d.h. jede Cauchy-Folge 
\end{def}

\begin{ex} \label{1.2}
	\begin{enumerate}[1)]
		\item
			\[
				B = \C^n, \|u\| = (|u_1|^p + \dotsb + |u_n|^p)^{\f 1p} \qquad 1 \le p < \infty
			\]
		\item
			Sei $B = C([a,b] \to \C)$ mit
			\[
				\|f\|_\infty = \max_{a \le x \le b} |f(x)|
			\]
			z.B. $\|f\|_2^2 = \int_a^b |f(x)|^2 dx$, dann ist $B$ kein Banach-Raum.
		\item
			Definiere
			\[
				L^p(I) := \bigg\{ f : I \to \C : f \text{ messbar} \land \int_I |f|^p d\my < \infty \bigg\}
			\]
			für ein Intervall $I \subset \R$.
			$L^p(I)$ mit
			\[
				\|f\|_p := \bigg( \int_I |f|^p d\my \bigg)^{\f 1p}
			\]
			(Identifiziere $f,\tilde f$, falls $\int_I |f-\tilde f| d\my = 0$)
	\end{enumerate}
\end{ex}


\begin{st}[Banachscher Fixpunktsatz] \label{1.3}
	Sei $B$ ein Banaachraum, $\emptyset \neq D \subset B$ mit $D$ abgeschlossen und $F: D \to B$ eine \emph{Kontraktion}, d.h.
	\[
		\exists q \in [0,1[ \forall x,y \in D : \|F(x) - F(y)\| \le q\|x-y\|
	\]
	mit $F(D) \subset D$. Dann gilt
	\begin{enumerate}[1)]
		\item
			Es existiert $x\in D$ mit $F(x)=x$, d.h. die Abbildung $F$ hat genau einen \emph{Fixpunkt} $x\in D$.
		\item
			Ist $x_0 \in D$ und $x_n := F(x_{n-1})$ für $n\in \N$, so gilt
			\[
				x_n \to x \qquad n \to \infty
			\]
			mit der \emph{Fehlerabschätzung}
			\[
				\|x_n - x\| \le \f{q^n}{1-q} \|x_1 - x_0\|
			\]
\fixme[Es scheint was zu fehlen?]
	\end{enumerate}
	\begin{proof}
		\begin{enumerate}[1)]
			\item
				Wegen $F(D) \subset D$ ist $x_n$ für $n\in \N_0$ definiert
			\item
				Zeige 
				\[
					\|x_{n+1} - x_n\| \le q^n \|x_1 - x_0\|
				\]
				induktiv
			\item
				Siehe Numerik (wurde dort schöner bewiesen).
			\item
				Eindeutigkeit:
				\[
					\|x-y\| = \|F(x) - F(y)\| \le q \|x-y\|
				\]
				und damit $\|x-y\| = 0$, also $x=y$.
		\end{enumerate}
	\end{proof}
\end{st}

\begin{nt} \label{1.4}
	Man kann genauso beweisen
	\[
		\|x - x_n\| \le \f 1{1-q} \|x_{n+1} -x_n \|
	\]
\end{nt}


\subsection{Beispiele} % 2.1

\begin{ex}[Tee] \label{2.1}
	Beschreibe $y(t)$ die Temperatur des Tee's und $y_A = \const$ die Außentemperatur.
	Es gilt folgende Differentialgleichung
	\[
		y'(t) = -K( y(t) - y_A)
	\]
	Die Lösung lautet
	\[
		y(t) = y_A + c e^{-Kt} \qquad t\in \R, c\in \R
	\]
	Probe:
	\[
		y' = c(-K)e^{-Kt} = -K(ce^{-Kt}) = -K(y(t) - y_A)
	\]
	Die Lösung ist erst eindeutig, wenn z.B. $y(t_0) = y_0$ vorgegeben wird (\emph{Anfangsbedingung}).
\end{ex}

\begin{df}[nicht-formale Beschreibung] \label{2.1}
	Eine \emph{Differentialgleichung} ist eine Glichung für eine gesuchte Funktion $y$, in der auch die Ableitung(en) von $y$ auftreten.
	Sie heißt \emph{gewöhnlich}, falls keine partiellen Ableitungen auftreten, sonst \emph{partiell}
\end{df}

\subsection{Separierbare Differentialgleichungen} % 2.2

Seien $I_f, I_g \subset \R$ Intervalle und $f \in C(I_f \to \R), g \in C(I_g \to \R)$.
Gesucht ist ein Intervall $I \subset \R$ und $y \in C^1(I \to \R)$, sodass
\[
	y' = f(x) g(y)
\]
Die Variablen sind also separierbar.
\begin{enumerate}[a)]
	\item
		Falls $y_0 \in I_g$ sodass $g(y_0) = 0$, dann existiert eine konstante Lösung:
		\[
			y(x) = y_0
		\]
	\item
		Sei $y \in C^1(I \to \R)$ eine Lösung mit $g(y(x)) \neq 0$ für $x \in I$.
		\begin{align*}
			y'(x) &= f(x)g(y(x)) \\
			\iff \f {y'(x)}{g(y(x))} &= f(x) \\
			\iff G(y(x)) + c_1 := \int \f 1{g(y)} dy = \int \f {y'(x)}{g(y(x))} dx &= \int f(x) dx = F(x) + c \qquad F' = f, G' = \f 1g \\
			\iff G(y(x)) &= F(x) + c
			\intertext{da $G' = \f 1g \neq 0$ ist $G$ injektiv und lokal umkehrbar.}
			\iff y(x) &= G^{-1}(F(x) + c)
		\end{align*}
\end{enumerate}

\begin{nt*}[Merkregel]
	\begin{align*}
		\f {dy}{dx} &= f(x)g(y)
		\intertext{Alle $y$ nach links, $x$ nach rechts und Integral davor}
		\iff \int \f{dy}{g(y)} &= \int f(x) dx
	\end{align*}
\end{nt*}

\begin{ex} \label{2.2}
	Sei folgende Differentialgleichung gegeben:
	\[
		y' = \cos^2 y \cos x
	\]
	\begin{enumerate}[a)]
		\item
			Die konstante Lösung ergibt sich für $\cos^2 y = 0$, also sind alle der Form
			\[
				y(x) = (n + \f 12) \pi \qquad \qquad n \in \Z
			\]
			konstante Lösungen.
		\item
			Für $y \neq \f {n + \f 12} \pi$ ergibt sich nach der Merkregel
			\begin{align*}
				\int \f {dy}{\cos^2 y} &= \int \cos x dx
				\iff \tan y &= \sin x + c \\
				\iff y &= \arctan(\sin x + c) + n \pi
			\end{align*}
			Die Lösungen sind demnach alle Funktionen der Form
			\[
				y(x) = \arctan(\sin x + c) + n \pi \qquad c \in \R, n \in \Z
			\]
	\end{enumerate}
	\begin{note}[Beobachtungen]
		\begin{itemize}
			\item
				Durch jeden Punkt $(x_0, y_0)$ geht genau eine Lösung.
				\[
					y(x) = \arctan(\sin x + \underbrace{\tan y_0 - \sin x_0}_{=c})
				\]
				falls $- \f \pi 2 < y_0 < \f \pi 2$.
			\item
				Jede Lösung ist auf ganz $\R$ definiert: \emph{globale} Lösung.
			\item
				Für festes $x_1 \in \R$ hängt $y(x_1)$ stetig von $(x_0, y_0)$ ab.
			\item
				Die Lösungsvielfalt ist durch die Parameter $c$ und $n$ beschrieben.
			\item
				Die Lösung ist eindeutig durch die \emph{Anfangsbedingung}
				\[
					y(x_0) = y_0 
				\]
				vorgegeben (für gegebenes $x_0, y_0$).
		\end{itemize}
	\end{note}
\end{ex}

\begin{ex} \label{2.3}
	Sei folgende DGL gegeben
	\[
		y' = (y^2)^{\f 13}
	\]
	\begin{enumerate}[a)]
		\item
			Die konstante Lösung ergibt sich als
			\[
				y(x) = 0
			\]
		\item
			Für $y > 0$ oder $y < 0$ ergibt sich nach der Merkregel
			\[
				3 y^{\f 13} = \int \f {dy}{y^{\f 23}} = \int  dx = x + c 
			\]
			für $x > -c$ im Fall $y > 0$, und für $x < -c$ im Fall $y < 0$.
			Es ergibt sich dann
			\[
				y(x) = \f 1{27}(x + c)^3
			\]
	\end{enumerate}
	\begin{note}[Beobachtungen]
		Für $y_0 \neq 0$ geht durch jeden Punkt genau eine Lösung:
		(exemplarisch für $y_0 > 0$):
		\[
			y(x) = \f 1{27}(x+c)^3  
			\qquad x > -c, c = 3y_0^{\f 13} - x_0
		\]
		Dies ist keine globale Lösung (\emph{lokale Lösung}).

		Setzt man diese Lösung fest zu einer globalen Lösung, geht die Eindeutigkeit verloren.
		\[
			y_r(x) = \begin{cases}
				\f 1{27} x^3 \qquad x > 0 \\
				0 & r \le x \le 0 \\
				\f 1{27} (x+r)^3 & x < -r
			\end{cases}
			\qquad \forall r \le 0
		\]
		Außerdem: durch $(x_0, 0)$ gehen beliebig viele Lösungen.
	\end{note}
\end{ex}


\subsection{Systeme von Differentialgleichungen}


Sei $A \in \R^{n\times n}$ gegeben.
Gesucht ist $y \in C^{\R \to \R^n}$ mit
\[
	y' = Ay
\]
ausgeschrieben ergeben sich
\begin{align*}
	y_1' &= a_{11} y_1 + a_{12} y_2 + \dotsb + a_{1n} y_n \\
	\vdots \; &= \qquad\qquad\qquad \vdots \\
	y_n' &= a_{n1} y_1 + a_{n2} y_2 + \dotsb + a_{nn} y_n \\
\end{align*}
also $n$ \emph{gekoppelte} Differentialgleichungen.

\begin{seg}[Fall 1: ${v_1,\dotsc, v_n}$ bildet eine Basis aus Eigenvektoren: $Av_j = \lambda_j v_j$]
	Dann ergibt sich die Lösung als
	\[
		y(t) := \sum_{j=1}^n c_j e^{\lambda_j t} v_j
		\qquad c_1,\dotsc, c_m \in \R
	\]
	denn
	\begin{align*}
		y' = \sum_{j=1}^n c_j \lambda_j e^{\lambda_j t} v_j = \sum_{j=1}^n c_j e^{\lambda_j t} Av_j = A y(t)
	\end{align*}
	Die Eindeutigkeit ist durch Anfangsbedingungen gegeben:
	\[
		y(t_0) = y_0
		\qquad t_0 \in \R, y_0 \in \R^n
	\]
	also
	\[
		\sum_{j=1}^n \underbrace{c_j e^{\lambda_j t_0}}_{d_j} v_j = y_0
	\]
	Die $d_j$ existieren und sind eindeutig, da $\{v_1,\dotsc,v_n\}$ Basis ist.
	Also existieren auch
	\[
		c_j  = e^{-\lambda_j t_0} d_j
	\]
	und sind eindeutig.

	\begin{note}[Beobachtungen]
		\begin{itemize}
			\item
				Es existiert immer eine globale Lösung.
			\item
				Die Lösungsgesamtkeit ist durch $n$ skalierbare Gleichungen gegeben mit Parametern $c_j$.
			\item
				Die Eindeutigkeit ist stets durch die Anfangsbedingung gewährleistet.
			\item
				Für festes $T \in \R$ hängt $y(T)$ stetig von $(t_0, y_0)$ ab.
			\item
				Der Lösungsraum ist ein reeller linearer Raum mit Basis
				\[
					\{ t \to e^{\lambda_1 t v_1, \dotsc, t\to e^{\lambda_n t}v_n}
				\]
		\end{itemize}
	\end{note}
\end{seg}

\begin{seg}[Fall 2: sonst]
	Bilde die Jordan-Normalform:
	\[
		J = T^{-1} A T
	\]
	(Einsen auf der ersten Nebendiagonalen \fixme[Beispielmatrix])

	Sei $u(t) := T^{-1} y(t)$ für eine Lösung $y(t)$.
	Dann ist
	\[
		u'(t) = T^{-1}y'(t) = T^{-1}Ay(t) = T^{-1}AT u(t)
	\]
	Also
	\[
		y' = Ay
		\qquad \iff \qquad
		u' = Ju
	\]
	beispielsweise	\fixme[Jordansystem]
	\begin{align*}
		u_1' &= \lambda_1 u_1& &+ u_2
	\end{align*}
	Die Lösungen sind gegeben durch (gehe Blockweise von unten nach oben vor):
	\begin{align*}
		u_3 &= c_3 e^{\lambda_1 t} \\
		u_2 &= c_2 e^{\lambda_1 t} + c_3 t e^{\lambda_1 t} \\
		u_1 &= c_1 e^{\lambda_1 t} + c_2 t e^{\lambda_1 t} + c_3 \f {t^2}2 e^{\lambda_1 t} \\
	\end{align*}
	Rücktransformation ergibt dann
	\[
		y(t) := T u(t)
	\]
\end{seg}


\subsection{Differentialgleichungen höherer Ordnung} % 2.4


Seien $a_0, \dotsc, a_{n-1}$ gegeben, gesucht ist $y \in C^n (\R \to \R)$ mit
\[
	y^{n} = a_{n-1} y^{n-1} + \dotsb + a_1 y' + a_0 y
\]
Setze dazu
\[
	u_1 := y \qquad u_2 := y' \qquad \dotsc \qquad u_n := y^{(n-1)}
\]
Es ergibt sich im Beispiel für 
\[
	y''' = y'
\]
\begin{align*}
	u_1' = u_2 \\
	u_2' = u_3 \\
	u_3' = y''' = y' = u_2
\end{align*}
Löse dieses System und setze dann
\[
	y(t) := u_1(t)
\]
Die Eindeutigkeit ist durch die Anfangsbedingungen
\[
	y(t_0) = y_0, \qquad y'(t_0) = y_1, \qquad y''(t_0) = y_2
\]
\end{document}
