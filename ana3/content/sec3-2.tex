% Stephan Hilb, 2012 Universität Stuttgart.
%
% Dieses Werk ist unter einer Creative Commons Lizenz vom Typ
% Namensnennung - Nicht-kommerziell - Weitergabe unter gleichen Bedingungen 3.0 Deutschland
% zugänglich. Um eine Kopie dieser Lizenz einzusehen, konsultieren Sie
% http://creativecommons.org/licenses/by-nc-sa/3.0/de/ oder wenden Sie sich
% brieflich an Creative Commons, 444 Castro Street, Suite 900, Mountain View,
% California, 94041, USA.

\section{Beispiele} % 2
\addtocounter{thmn}{1}
\setcounter{theorem}{0}


\subsection{Tee} % 2.1


Beschreibe $y(t)$ die Temperatur des Tee's und $y_A = \mathrm{const.}$ die Außentemperatur.
Es gilt folgende Differentialgleichung
%
\begin{align*}
  y'(t) = -K( y(t) - y_A)
\end{align*}
%
Die Lösung lautet
%
\begin{align*}
  y(t) = y_A + c e^{-Kt} \qquad t\in \mathbb{R}, c\in \mathbb{R}
\end{align*}
%
Probe:
%
\begin{align*}
  y' = c(-K)e^{-Kt} = -K(ce^{-Kt}) = -K(y(t) - y_A)
\end{align*}
%
Die Lösung ist erst eindeutig, wenn z.B. $y(t_0) = y_0$ vorgegeben wird (\emph{Anfangsbedingung}).

\begin{theorem}[Definition: nicht-formale Beschreibung] \label{thm:2.1}
  Eine \emph{Differentialgleichung} ist eine Glichung für eine gesuchte Funktion $y$, in der auch die Ableitung(en) von $y$ auftreten.
  Sie heißt \emph{gewöhnlich}, falls keine partiellen Ableitungen auftreten, sonst \emph{partiell}
\end{theorem}


\subsection{Separierbare Differentialgleichungen} % 2.2


Seien $I_f, I_g \subset \mathbb{R}$ Intervalle und $f \in C(I_f \to \mathbb{R}), g \in C(I_g \to \mathbb{R})$.
Gesucht ist ein Intervall $I \subset \mathbb{R}$ und $y \in C^1(I \to \mathbb{R})$, sodass
%
\begin{align*}
  y' = f(x) g(y)
\end{align*}
%
Die Variablen sind also separierbar.
\begin{enum-alph}
\item
  Falls $g(y)$ in $y_0$ eine Nullstelle hat, dann existiert die konstante Lösung:
  %
  \begin{align*}
    y(x) = y_0
  \end{align*}
  %
  Alle Nullstellen von $g(y)$ repräsentieren also konstante Lösungen.
\item
  Sei $y \in C^1(I \to \mathbb{R})$ eine Lösung mit $g(y(x)) \neq 0$ für $x \in I$.
  Forme $y'(x) = f(x)g(y(x))$ um und betrachte die Stammfunktionen:
  %
  \begin{align*}
    & &\frac {y'(x)}{g(y(x))} &= f(x) \\
    &\iff& G(y(x)) + c_1 := \int \frac 1{g(y)} dy 
    = \int \frac {y'(x)}{g(y(x))} dx &= \int f(x) dx 
    =: F(x) + c_2
    \qquad F' = f, G' = \frac 1g \\
    &\iff& G(y(x)) &= F(x) + c
    \intertext{da $G' = \frac 1g \neq 0$ ist $G$ injektiv und lokal umkehrbar.}
    &\iff& y(x) &= G^{-1}(F(x) + c)
  \end{align*}
  %
\end{enum-alph}

\begin{notice*}[Merkregel]
  %
  \begin{align*}
    \frac {dy}{dx} &= f(x)g(y)
    \intertext{Alle $y$ nach links, $x$ nach rechts und Integral davor:}
    \int \frac{dy}{g(y)} &= \int f(x) dx
  \end{align*}
  %
\end{notice*}

\begin{example} \label{thm:2.2}
  Sei folgende Differentialgleichung gegeben:
  %
  \begin{align*}
    y' = \cos^2 y \cos x
  \end{align*}
  %
  \begin{enum-alph}
  \item
    Die konstante Lösung ergibt sich für $\cos^2 y = 0$, also sind alle Funktionen der Form
    %
    \begin{align*}
      y(x) = (n + \tfrac 12) \pi \qquad \qquad n \in \mathbb{Z}
    \end{align*}
    %
    konstante Lösungen.
  \item
    Für $y \neq (n + \frac 12) \pi$ ergibt sich nach der Merkregel
    %
    \begin{align*}
      \int \frac {dy}{\cos^2 y} &= \int \cos x dx \\
      \iff \tan y &= \sin x + c \\
      \iff y &= \arctan(\sin x + c) + n \pi
    \end{align*}
    %
    Die Lösungen sind demnach alle Funktionen der Form
    %
    \begin{align*}
      y(x) = \arctan(\sin x + c) + n \pi \qquad c \in \mathbb{R}, n \in \mathbb{Z}
    \end{align*}
    %
  \end{enum-alph}
  \begin{notice*}[Beobachtungen]
    \begin{itemize}
      \item
        Durch jeden Punkt $(x_0, y_0)$ geht genau eine Lösung:
        %
        \begin{align*}
          y(x) = \arctan(\sin x + \underbrace{\tan y_0 - \sin x_0}_{=c})
        \end{align*}
        %
        falls $- \frac \pi 2 < y_0 < \frac \pi 2$.
      \item
        Jede Lösung ist auf ganz $\mathbb{R}$ definiert: \emph{globale} Lösung.
      \item
        Für festes $x_1 \in \mathbb{R}$ hängt $y(x_1)$ stetig von $(x_0, y_0)$ ab.
      \item
        Die Lösungsvielfalt ist durch die Parameter $c$ und $n$ beschrieben.
      \item
        Die Lösung ist eindeutig durch die \emph{Anfangsbedingung}
        %
        \begin{align*}
          y(x_0) = y_0 
        \end{align*}
        %
        vorgegeben (für gegebenes $x_0, y_0$).
    \end{itemize}
  \end{notice*}
\end{example}

\begin{example} \label{thm:2.3}
  Sei folgende DGL gegeben
  %
  \begin{align*}
    y' = (y^2)^{\frac 13}
  \end{align*}
  %
  \begin{enum-alph}
  \item
    Die konstante Lösung ergibt sich als
    %
    \begin{align*}
      y(x) = 0
    \end{align*}
    %
  \item
    Für $y > 0$ oder $y < 0$ ergibt sich nach der Merkregel
    %
    \begin{align*}
      3 y^{\frac 13} = \int \frac {dy}{y^{\frac 23}} = \int  dx = x + c 
    \end{align*}
    %
    für $x > -c$ im Fall $y > 0$, und für $x < -c$ im Fall $y < 0$.
    Es ergibt sich dann
    %
    \begin{align*}
      y(x) = \frac 1{27}(x + c)^3
    \end{align*}
    %
  \end{enum-alph}
  \begin{notice*}[Beobachtungen]
    Für $y_0 \neq 0$ geht durch jeden Punkt genau eine Lösung:
    (jetzt exemplarisch für $y_0 > 0$):
    %
    \begin{align*}
      y(x) = \frac 1{27}(x+c)^3  
      \qquad c = 3y_0^{\frac 13} - x_0
    \end{align*}
    %
    Dies ist wegen der Bedingung $x > -c$ keine globale Lösung (\emph{lokale Lösung}).

    Setzt man diese Lösung fest zu einer globalen Lösung, geht die Eindeutigkeit verloren.
    Man kann beispielsweise den unteren Ast an einer beliebigen Stelle $r \le 0$ ansetzen:
    %
    \begin{align*}
      y_r(x) = \begin{cases}
        \frac 1{27} x^3  & x > 0 \\
        0 & r \le x \le 0 \\
        \frac 1{27} (x-r)^3 & x < r
      \end{cases}
      \qquad \forall r \le 0
    \end{align*}
    %
    Insbesondere gehen durch $(x_0, 0)$ beliebig viele Lösungen.
  \end{notice*}
\end{example}


\subsection{Systeme von Differentialgleichungen} % 2.3


Sei $A \in \mathbb{R}^{n\times n}$ gegeben.
Gesucht ist $y \in C^1(\mathbb{R} \to \mathbb{R}^n)$ mit
%
\begin{align*}
  y' = Ay
\end{align*}
%
ausgeschrieben ergeben sich
%
\begin{align*}
  y_1' &= a_{11} y_1 + a_{12} y_2 + \dotsb + a_{1n} y_n \\
  \vdots \; &= \qquad\qquad\qquad \vdots \\
  y_n' &= a_{n1} y_1 + a_{n2} y_2 + \dotsb + a_{nn} y_n \\
\end{align*}
%
also $n$ \emph{gekoppelte} Differentialgleichungen.

\paragraph{Fall 1: ${v_1,\dotsc, v_n}$ bildet eine Basis aus Eigenvektoren: $Av_j = \lambda_j v_j$}
Dann ergibt sich die Lösung als
%
\begin{align*}
  y(t) := \sum_{j=1}^n c_j e^{\lambda_j t} v_j
  \qquad c_1,\dotsc, c_n \in \mathbb{R}
\end{align*}
%
denn
%
\begin{align*}
  y' = \sum_{j=1}^n c_j \lambda_j e^{\lambda_j t} v_j = \sum_{j=1}^n c_j e^{\lambda_j t} Av_j = A y(t)
\end{align*}
%
Die Eindeutigkeit ist durch Anfangsbedingungen gegeben:
%
\begin{align*}
  y(t_0) = y_0
  \qquad t_0 \in \mathbb{R}, y_0 \in \mathbb{R}^n
\end{align*}
%
damit gilt	
%
\begin{align*}
  \sum_{j=1}^n \underbrace{c_j e^{\lambda_j t_0}}_{=:d_j} v_j = y_0
\end{align*}
%
Die $d_j$ existieren und sind eindeutig, da $\{v_1,\dotsc,v_n\}$ Basis ist.
Also existieren auch die
%
\begin{align*}
  c_j  = e^{-\lambda_j t_0} d_j
\end{align*}
%
und sind eindeutig.

\begin{notice*}[Beobachtungen]
  \begin{itemize}
    \item
      Es existiert immer eine globale Lösung.
    \item
      Die Lösungsgesamtheit ist durch $n$ skalierbare Gleichungen gegeben mit Parametern $c_j$.
    \item
      Die Eindeutigkeit ist stets durch die Anfangsbedingung gewährleistet.
    \item
      Für festes $t \in \mathbb{R}$ hängt $y(t)$ stetig von $(t_0, y_0)$ ab.
    \item
      Der Lösungsraum ist ein reeller linearer Raum mit Basis
      %
      \begin{align*}
        \{ t \to e^{\lambda_1 t v_1}, \dotsc, t\to e^{\lambda_n t}v_n\}
      \end{align*}
      %
  \end{itemize}
\end{notice*}


\paragraph{Fall 2: sonst}
Bilde die Jordan-Normalform:
%
\begin{align*}
  J = T^{-1} A T
\end{align*}
%
beispielsweise
%
\begin{align*}
  J = \begin{pmatrix}
    \lambda_1 & 1 & 0\\
    0 & \lambda_1 & 1 \\
    0 & 0 & \lambda_1
  \end{pmatrix}
\end{align*}
%
Setze $u(t) := T^{-1} y(t)$ für eine Lösung $y(t)$.
Dann ist
%
\begin{align*}
  u'(t) = T^{-1}y'(t) = T^{-1}Ay(t) = T^{-1}AT u(t)
\end{align*}
%
Also
%
\begin{align*}
  y' = Ay
  \qquad \iff \qquad
  u' = Ju
\end{align*}
%
Wir können also statt $y' = Ay$ das System $u' = Ju$ berechnen und anschließend zurücktransformieren.
In unserem Beispiel:
\begin{alignat*}{3}
  u_1' &= \lambda_1 u_1& &+ u_2 \\
  u_2' &= &&+ \lambda_1 u_2 && + u_3 \\
  u_3' &= && &&+\lambda_1 u_3
\end{alignat*}
Die Lösungen sind gegeben durch (gehe Blockweise von unten nach oben vor und nutze die umgekehrte Produktregel):
%
\begin{align*}
  u_3 &= c_3 e^{\lambda_1 t} \\
  u_2 &= c_2 e^{\lambda_1 t} + c_3 t e^{\lambda_1 t} \\
  u_1 &= c_1 e^{\lambda_1 t} + c_2 t e^{\lambda_1 t} + c_3 \frac {t^2}2 e^{\lambda_1 t} \\
\end{align*}
%
Rücktransformation ergibt dann
%
\begin{align*}
  y(t) := T u(t)
\end{align*}
%



\subsection{Differentialgleichungen höherer Ordnung} % 2.4


Seien $a_0, \dotsc, a_{n-1}$ gegeben, gesucht ist $y \in C^n (\mathbb{R} \to \mathbb{R})$ mit
%
\begin{align*}
  y^{n} = a_{n-1} y^{n-1} + \dotsb + a_1 y' + a_0 y
\end{align*}
%
Setze dazu
%
\begin{align*}
  u_1 := y \qquad u_2 := y' \qquad \dotsc \qquad u_n := y^{(n-1)}
\end{align*}
%
Es ergibt sich im Beispiel für die DGL
%
\begin{align*}
  y''' = y'
\end{align*}
%
\begin{alignat*}{2}
  u_1' &= &&u_2 \\
  u_2' &= &&u_3 \\
  u_3' &= y''' = y' = &&u_2
\end{alignat*}
Löse dieses System und setze dann
%
\begin{align*}
  y(t) := u_1(t)
\end{align*}
%
Die Eindeutigkeit ist durch die Anfangsbedingungen
%
\begin{align*}
  y(t_0) = y_0, \qquad y'(t_0) = y_1, \qquad y''(t_0) = y_2
\end{align*}
%
gegeben.
