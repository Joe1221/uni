% Stephan Hilb, 2012 Universität Stuttgart.
%
% Dieses Werk ist unter einer Creative Commons Lizenz vom Typ
% Namensnennung - Nicht-kommerziell - Weitergabe unter gleichen Bedingungen 3.0 Deutschland
% zugänglich. Um eine Kopie dieser Lizenz einzusehen, konsultieren Sie
% http://creativecommons.org/licenses/by-nc-sa/3.0/de/ oder wenden Sie sich
% brieflich an Creative Commons, 444 Castro Street, Suite 900, Mountain View,
% California, 94041, USA.

\section{Separierbare Differentialgleichungen} % 2.2

Seien $I_f, I_g \subset \R$ Intervalle und $f \in C(I_f \to \R), g \in C(I_g \to \R)$.
Gesucht ist ein Intervall $I \subset \R$ und $y \in C^1(I \to \R)$, sodass
\[
	y' = f(x) g(y)
\]
Die Variablen sind also separierbar.
\begin{enumerate}[a)]
	\item
		Falls $g(y)$ in $y_0$ eine Nullstelle hat, dann existiert die konstante Lösung:
		\[
			y(x) = y_0
		\]
		Alle Nullstellen von $g(y)$ repräsentieren also konstante Lösungen.
	\item
		Sei $y \in C^1(I \to \R)$ eine Lösung mit $g(y(x)) \neq 0$ für $x \in I$.
		Forme $y'(x) = f(x)g(y(x))$ um und betrachte die Stammfunktionen:
		\begin{align*}
			& &\f {y'(x)}{g(y(x))} &= f(x) \\
			&\iff& G(y(x)) + c_1 := \int \f 1{g(y)} dy 
			= \int \f {y'(x)}{g(y(x))} dx &= \int f(x) dx 
			=: F(x) + c_2
			\qquad F' = f, G' = \f 1g \\
			&\iff& G(y(x)) &= F(x) + c
			\intertext{da $G' = \f 1g \neq 0$ ist $G$ injektiv und lokal umkehrbar.}
			&\iff& y(x) &= G^{-1}(F(x) + c)
		\end{align*}
\end{enumerate}

\begin{nt*}[Merkregel]
	\begin{align*}
		\f {dy}{dx} &= f(x)g(y)
		\intertext{Alle $y$ nach links, $x$ nach rechts und Integral davor:}
		\int \f{dy}{g(y)} &= \int f(x) dx
	\end{align*}
\end{nt*}

\begin{ex} \label{2.2}
	Sei folgende Differentialgleichung gegeben:
	\[
		y' = \cos^2 y \cos x
	\]
	\begin{enumerate}[a)]
		\item
			Die konstante Lösung ergibt sich für $\cos^2 y = 0$, also sind alle Funktionen der Form
			\[
				y(x) = (n + \tf 12) \pi \qquad \qquad n \in \Z
			\]
			konstante Lösungen.
		\item
			Für $y \neq (n + \f 12) \pi$ ergibt sich nach der Merkregel
			\begin{align*}
				\int \f {dy}{\cos^2 y} &= \int \cos x dx \\
				\iff \tan y &= \sin x + c \\
				\iff y &= \arctan(\sin x + c) + n \pi
			\end{align*}
			Die Lösungen sind demnach alle Funktionen der Form
			\[
				y(x) = \arctan(\sin x + c) + n \pi \qquad c \in \R, n \in \Z
			\]
	\end{enumerate}
	\begin{note}[Beobachtungen]
		\begin{itemize}
			\item
				Durch jeden Punkt $(x_0, y_0)$ geht genau eine Lösung:
				\[
					y(x) = \arctan(\sin x + \underbrace{\tan y_0 - \sin x_0}_{=c})
				\]
				falls $- \f \pi 2 < y_0 < \f \pi 2$.
			\item
				Jede Lösung ist auf ganz $\R$ definiert: \emph{globale} Lösung.
			\item
				Für festes $x_1 \in \R$ hängt $y(x_1)$ stetig von $(x_0, y_0)$ ab.
			\item
				Die Lösungsvielfalt ist durch die Parameter $c$ und $n$ beschrieben.
			\item
				Die Lösung ist eindeutig durch die \emph{Anfangsbedingung}
				\[
					y(x_0) = y_0 
				\]
				vorgegeben (für gegebenes $x_0, y_0$).
		\end{itemize}
	\end{note}
\end{ex}

\begin{ex} \label{2.3}
	Sei folgende DGL gegeben
	\[
		y' = (y^2)^{\f 13}
	\]
	\begin{enumerate}[a)]
		\item
			Die konstante Lösung ergibt sich als
			\[
				y(x) = 0
			\]
		\item
			Für $y > 0$ oder $y < 0$ ergibt sich nach der Merkregel
			\[
				3 y^{\f 13} = \int \f {dy}{y^{\f 23}} = \int  dx = x + c 
			\]
			für $x > -c$ im Fall $y > 0$, und für $x < -c$ im Fall $y < 0$.
			Es ergibt sich dann
			\[
				y(x) = \f 1{27}(x + c)^3
			\]
	\end{enumerate}
	\begin{note}[Beobachtungen]
		Für $y_0 \neq 0$ geht durch jeden Punkt genau eine Lösung:
		(jetzt exemplarisch für $y_0 > 0$):
		\[
			y(x) = \f 1{27}(x+c)^3  
			\qquad c = 3y_0^{\f 13} - x_0
		\]
		Dies ist wegen der Bedingung $x > -c$ keine globale Lösung (\emph{lokale Lösung}).

		Setzt man diese Lösung fest zu einer globalen Lösung, geht die Eindeutigkeit verloren.
		Man kann beispielsweise den unteren Ast an einer beliebigen Stelle $r \le 0$ ansetzen:
		\[
			y_r(x) = \begin{cases}
				\f 1{27} x^3  & x > 0 \\
				0 & r \le x \le 0 \\
				\f 1{27} (x-r)^3 & x < r
			\end{cases}
			\qquad \forall r \le 0
		\]
		Insbesondere gehen durch $(x_0, 0)$ beliebig viele Lösungen.
	\end{note}
\end{ex}
