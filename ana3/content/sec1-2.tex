% Henri Menke, 2012 Universität Stuttgart.
%
% Dieses Werk ist unter einer Creative Commons Lizenz vom Typ
% Namensnennung - Nicht-kommerziell - Weitergabe unter gleichen Bedingungen 3.0 Deutschland
% zugänglich. Um eine Kopie dieser Lizenz einzusehen, konsultieren Sie
% http://creativecommons.org/licenses/by-nc-sa/3.0/de/ oder wenden Sie sich
% brieflich an Creative Commons, 444 Castro Street, Suite 900, Mountain View,
% California, 94041, USA.

\section{Holomorphie und Analytizität}
\addtocounter{thmn}{1}
\setcounter{theorem}{0}

% % % Vorlesung vom 25.10.2012

\begin{theorem}[Vereinbarung]
  Zu $f : O \to \mathbb{C}$ setzen wir
  %
  \begin{gather*}
    \widetilde{O} \coloneq \{ (x,y) \in \mathbb{R}^2 : x + \mathrm{i} y \in O \} , \\
    \begin{pmatrix} u \\ v \end{pmatrix} : \widetilde{O} \to \mathbb{R}^2 : (x,y) \mapsto \begin{pmatrix} u(x,y) \\ v(x,y) \end{pmatrix} \coloneq \begin{pmatrix} \Re f(x + \mathrm{i} y) \\ \Im f(x + \mathrm{i} y) \end{pmatrix}.
  \end{gather*}
  %
  $f$ wird interpretiert als Abbildung von $\mathbb{R}^2$ nach $\mathbb{R}^2$.
\end{theorem}

\begin{theorem}[Satz] \label{thm:2.2}
  $O \subseteq \mathbb{C}$ offen, $f : O \to \mathbb{C}$. Dann sind äquivalent:
  %
  \begin{enum-roman}
    \item \label{itm:2.2 i} $f$ ist differenzierbar (in $O$).
    
    \item \label{itm:2.2 ii} $f$ ist stetig (in $O$) und für alle Wege $\gamma_1, \gamma_2 \in C^1([0,1] \to O)$ gilt
    %
    \begin{align*}
      \gamma_1 \sim \gamma_2 \implies \int_{\gamma_1} f(z) \, \mathrm{d}z = \int_{\gamma_2} f(z) \, \mathrm{d}z.
    \end{align*}
    
    \item \label{itm:2.2 iii} $f$ ist stetig und für alle $\gamma \in C^1([0,1] \to O)$ gilt
    %
    \begin{align*}
      \gamma \text{ nullhomotop } \implies \int_\gamma f(z) \, \mathrm{d}z = 0.
    \end{align*}
    
    \item \label{itm:2.2 iv} Für jedes $z_0 \in O$ existiert $R > 0$ und eine Folge $(a_n)$ in $\mathbb{C}$, sodass
    %
    \begin{align*}
      f(z) = \sum\limits_{n=0}^{\infty} a_n (z-z_0)^n \; , \quad \text{für } |z-z_0| < R.
    \end{align*}
    
    \item \label{itm:2.2 v} $f$ ist in $O$ beliebig oft differenzierbar.
    
    \item \label{itm:2.2 vi} $u, v \in C^1(\widetilde{O} \to \mathbb{R})$ und $u, v$ erfüllen die \acct{Cauchy-Riemannschen Differentialgleichungen}:
    %
    % u_x = \frac{\partial u}{\partial x}
    %
    \begin{align*}
      u_x = v_y \quad \text{und} \quad u_y = -v_x \; , \quad \text{in } \widetilde{O}.
    \end{align*}
  \end{enum-roman}
\end{theorem}

\begin{theorem}[Definition]
  Sei $O \subseteq \mathbb{C}$ offen, $f : O \to \mathbb{C}$. Erfüllt $f$ eine (und damit alle) Bedingungen aus \ref{thm:2.2}, so heißt $f$ \acct{holomorph} (dies betont die beliebige Differenzierbarkeit) oder \acct{analytisch} (dies betont \ref{itm:2.2 iv}).
\end{theorem}

\begin{theorem}[Cauchy'scher Integralsatz für die Bilder von Rechtecken] \label{thm:2.4}
  Sei $O \subseteq \mathbb{C}$ offen, $f : O \to \mathbb{C}$ differenzierbar, $R \coloneq [a,b] \times [c,d]$, $\varphi \in C^1(R \to O)$, $\gamma$ die geschlossene Randkurve von $R$ (stückweise $C^1$). Dann gilt
  %
  \begin{align*}
    \int_{\varphi \circ \gamma} f(z) \, \mathrm{d}z = 0.
  \end{align*}
  %
  \begin{figure}[H]
    \centering
    \begin{pspicture}(0,-1)(5.5,1.0)
      \psline[linecolor=DarkOrange3]{->}(0,0)(1,0)(1,0.5)(0.5,0.5)
      \psline[linecolor=DarkOrange3](0.5,0.5)(0,0.5)(0,0)
      \rput(0.5,0.25){\color{DimGray} $R$}
      \uput[90](0.5,0.5){\color{DarkOrange3} $\gamma$}
      \pnode(1.5,0.25){A}
      \pnode(2.5,0.25){B}
      \ncarc[arrows=->,linecolor=MidnightBlue]{A}{B}
      \naput{\color{MidnightBlue} $\varphi$}
      \psccurve(3,0)(4,-1)(5,1)
      \psline[linecolor=DarkOrange3]{->}(3.7,0.3)(3.5,-0.3)(4,-0.3)(4.5,0.5)(3.7,0.3)
      \uput*[-45](4.5,0.5){\color{DarkOrange3} $\varphi \circ \gamma$}
      \uput[-160](5,1){\color{DimGray} $O$}
    \end{pspicture}
  \end{figure}
  %
  \begin{proof}
    \begin{enum-arab}
      \item $\varphi \in C^1(R \to O) \implies \varphi \circ \gamma$ ist stückweise $C^1$, also ein Weg in $O$.
      
      \item Da $R$ kompakt und $\partial_1 \varphi$ und $\partial_2 \varphi$ stetig auf $R$:
      %
      \begin{align*}
        |\nabla \varphi| = \left| \binom{\partial_1 \varphi}{\partial_2 \varphi} \right| \leq c < \infty \quad  \text{auf } R.
      \end{align*}
      
      \item \label{itm:2.4 3.} Konstruiere eine Folge $(R_n)$ von Rechtecken mit Randkurven $\gamma_n$:
      %
      \begin{align*}
        R_0 \coloneq R \; , \quad \gamma_0 = \gamma.
      \end{align*}
      %
      Teile $R_n$ durch Seitenhalbierung in vier Rechtecke. Wähle als $R_{n+1}$ dasjenige der vier, für das
      %
      \begin{align*}
        \left| \int_{\varphi \circ \gamma_{n+1}} f(z) \, \mathrm{d}z \right|
      \end{align*}
      %
      am größten ist.
      %
      \begin{figure}[H]
        \centering
        \begin{pspicture}(0,0)(4,3)
          \psline[linecolor=MidnightBlue](0,0)(4,0)(4,3)(0,3)(0,0)
          \psline[linecolor=MidnightBlue,linestyle=none]{->}(4,0)(4,1.5)
          \psline[linestyle=dotted](0,1.5)(4,1.5)
          \psline[linestyle=dotted](2,3)(2,0)
          % links oben
          \psline[linecolor=DarkOrange3,linestyle=none]{>->}(1,1.6)(1.9,1.6)(1.9,2.25)
          \psline[linecolor=DarkOrange3](0.1,1.6)(1.9,1.6)(1.9,2.9)(0.1,2.9)(0.1,1.6)
          \rput(1,2.25){\color{DarkOrange3} $\beta_2$}
          % links unten
          \psline[linecolor=DarkOrange3,linestyle=none]{>->}(1.9,0.75)(1.9,1.4)(1,1.4)
          \psline[linecolor=DarkOrange3](0.1,0.1)(1.9,0.1)(1.9,1.4)(0.1,1.4)(0.1,0.1)
          \rput(1,0.75){\color{DarkOrange3} $\beta_3$}
          % rechts oben
          \psline[linecolor=DarkOrange3,linestyle=none]{>->}(2.1,2.25)(2.1,1.6)(3,1.6)
          \psline[linecolor=DarkOrange3](2.1,1.6)(3.9,1.6)(3.9,2.9)(2.1,2.9)(2.1,1.6)
          \rput(3,2.25){\color{DarkOrange3} $\beta_1$}
          % rechts unten
          \psline[linecolor=DarkOrange3,linestyle=none]{>->}(3,1.4)(2.1,1.4)(2.1,0.75)
          \psline[linecolor=DarkOrange3](2.1,0.1)(3.9,0.1)(3.9,1.4)(2.1,1.4)(2.1,0.1)
          \rput(3,0.75){\color{DarkOrange3} $\beta_4$}
          
          \psellipse[linecolor=DarkRed](2,2.2)(0.3,0.2)
          \psellipse[linecolor=DarkRed](2,0.8)(0.3,0.2)
          \psellipse[linecolor=DarkRed](1.1,1.5)(0.2,0.3)
          \psellipse[linecolor=DarkRed](2.9,1.5)(0.2,0.3)
          
          \uput[0](4,3){\color{MidnightBlue} $R_n$}
        \end{pspicture}
      \end{figure}
      %
      Die Integrale über die {\color{DarkRed} rot} markierten Teile heben sich gegenseitig weg, denn
      %
      \begin{align*}
        \int_{-\gamma} f(z) \, \mathrm{d}z = - \int_{\gamma} f(z) \, \mathrm{d}z.
      \end{align*}
      %
      Daraus folgt
      %
      \begin{align*}
        \int_{\varphi \circ \gamma_n} f(z) \, \mathrm{d}z = \sum\limits_{j = 1}^{4} \int_{\varphi \circ \beta_j} f(z) \, \mathrm{d}z.
      \end{align*}
      %
      Weiterhin folgt
      %
      \begin{align*}
        \left| \int_{\varphi \circ \gamma_n} f(z) \, \mathrm{d}z \right|
        \leq \sum\limits_{j = 1}^{4} \left| \int_{\varphi \circ \gamma_{n+1}} f(z) \, \mathrm{d}z \right|
        \leq 4 \, \left| \int_{\varphi \circ \gamma_{n+1}} f(z) \, \mathrm{d}z \right|.
      \end{align*}
      %
      Durch vollständige Induktion zeigt sich
      %
      \begin{align*}
        \left| \int_{\varphi \circ \gamma_0} f(z) \, \mathrm{d}z \right|
        \leq 4^n \left| \int_{\varphi \circ \gamma_n} f(z) \, \mathrm{d}z \right|.
      \end{align*}
      
      \item \label{itm:2.4 4.} Sei $x_n$ der Mittelpunkt von $R_n$, dann gilt
      %
      \begin{align*}
        \forall \, x \in R_n : |x - x_n| < L(\gamma_n) \leq \frac{1}{2^n} L(\gamma_0).
      \end{align*}
      %
      Für $m \geq n$ folgt dann
      %
      \begin{align*}
        |x_m - x_n| < \frac{1}{2^n} L(\gamma_0) \; , \quad \text{da } R_m \subseteq R_n.
      \end{align*}
      %
      Also ist $(x_n)$ Cauchyfolge mit $x_m \to y \in \bigcap_{n \in \mathbb{N}_0}R_n$, da $R_n$ abgeschlossen und $R_{n+1} \subseteq R_n$.
      
      Für $m \geq n$ gilt $x_m \in R_n$, daraus folgt
      %
      \begin{align*}
        |x_n - y| \leq \frac{1}{2^n} L(\gamma_0).
      \end{align*}
      
      \item \label{itm:2.4 5.} Aus $x_n \to y \in R_0$ folgt mit der Stetigkeit von $\varphi$
      %
      \begin{align*}
        \varphi(x_n) \to \varphi(y) \eqcolon z_0 \in O
      \end{align*}
      %
      Sei nun $z \coloneq \varphi(x)$ mit $x \in R_n$, dann folgt mit dem Mittelwertsatz
      %
      \begin{align*}
        |\Re(z-z_0)| &= |\Re \varphi(x) - \Re \varphi(y)| \\
        &= | \nabla (\Re \varphi)(\xi)(x-y)| \\
        &\leq c |x-y| \leq c \, \frac{1}{2^n} L(\gamma_0).
      \end{align*}
      %
      Analog zeigt sich $|\Im (z - z_0)| \le c \frac 1{2^n} L(\gamma_0)$.
      Folglich gilt die Abschätzung
      %
      \begin{align*}
        |z-z_0| \leq \sqrt{2} \, c \, \frac{1}{2^n} L(\gamma_0).
      \end{align*}
      
% % % Vorlesung vom 29.10.2012
      
      \item \label{itm:2.4 6.} Sei $\varepsilon > 0$ vorgegeben.
      %
      \begin{align*}
        f(z) = f(z_0) + f'(z_0) (z-z_0) + r(z,z_0)
      \end{align*}
      %
      mit 
      \begin{align*}
        \frac{|r(z,z_0)|}{|z-z_0|} < \varepsilon \quad \text{für } |z-z_0| < \delta
      \end{align*}
      %
      Für $n \geq N_\delta$ gilt
      %
      \begin{align*}
        |z-z_0| < \delta \quad \forall \, z \in \varphi(R_n)
      \end{align*}
      %      
      und somit
      %
      \begin{align*}
        |r(z,z_0)| < \varepsilon |z-z_0| \overset{\text{\ref{itm:2.4 5.}}}{\leq} \sqrt{2} c \frac{\varepsilon}{2^n} L(\gamma_0) \; , \quad z,z_0 \in \varphi(R_n).
      \end{align*}
      
      \item 
        Es gilt
      %
      \begin{gather*}
        \begin{aligned}
          \left| \int_{\varphi \circ \gamma_n} f(z) \, \mathrm{d}z \right|
          &\leq \underbrace{\left| \int_{\varphi \circ \gamma_n} \underbrace{\Big(f(z_0) + f'(z_0) (z-z_0)\Big)}_{\text{Besitzt Stammfunktion}} \, \mathrm{d}z \right|}_{=0 \text{ da $\varphi \circ \gamma_n$ geschlossen}}
          + \left| \int_{\varphi \circ \gamma_n} r(z,z_0) \, \mathrm{d}z \right| \\
          &\leq L(\varphi \circ \gamma_n)
          \, \max\limits_{z \in \mathrm{Bild}(\varphi \circ \gamma_n)} |r(z,z_0)| \\
          &\leq \int |(\varphi \circ \gamma_n)'(t)| \mathrm{d}t
          \, \max\limits_{z \in \mathrm{Bild}(\varphi \circ \gamma_n)} |r(z,z_0)| \\
          &\leq \int |\varphi'(\gamma_n(t))| \, |\gamma_n'(t)| \mathrm{d}t
          \, \max\limits_{z \in \mathrm{Bild}(\varphi \circ \gamma_n)} |r(z,z_0)| \\
          &\leq c \cdot L(\gamma_n)
          \, \max\limits_{z \in \mathrm{Bild}(\varphi \circ \gamma_n)} |r(z,z_0)| \\
          &\leq \frac{c}{2^n} L(\gamma_0)
          \, \max\limits_{z \in \mathrm{Bild}(\varphi \circ \gamma_n)} |r(z,z_0)| \\
          &\overset{\text{\ref{itm:2.4 6.}}}{\leq} \frac{c}{2^n} L(\gamma_0) \, \frac{\varepsilon}{2^n} L(\gamma_0) \\
        \end{aligned} \\
        \begin{aligned}
          \overset{\text{\ref{itm:2.4 3.}}}{\implies}& \left| \int_{\varphi \circ \gamma_0} f(z) \, \mathrm{d}z \right|
          \leq 4^n \left| \int_{\varphi \circ \gamma_n} f(z) \, \mathrm{d}z \right| \leq c \, \varepsilon \, L(\gamma_0)^2 \quad \text{für jedes } \varepsilon > 0 \\
          \implies& \int_{\varphi \circ \gamma_n} f(z) \, \mathrm{d}z = 0.
        \end{aligned}
      \end{gather*}
    \end{enum-arab}
  \end{proof}
\end{theorem}

\begin{notice}[Folgerung] \label{thm:2.5}
  Sei $O \subseteq \mathbb{C}$ offen, $f : O \to \mathbb{C}$ differenzierbar, $\gamma_1 \sim \gamma_2$ in $O$. Dann gilt
  %
  \begin{align*}
    \int_{\gamma_1} f(z) \, \mathrm{d}z = \int_{\gamma_2} f(z) \, \mathrm{d}z.
  \end{align*}
  
  \begin{proof}
    Sei $\Phi$ die Homotopie, insbesondere
    %
    \begin{gather*}
      \Phi(t,0) = \gamma_1(t) \; , \quad \Phi(t,1) = \gamma_2(t) \\
      \Phi \in C^1([0,1] \times [0,1] \to O)
    \end{gather*}
    %
    Setze zur Anwendung von \ref{thm:2.4} $R \coloneq [0,1] \times [0,1]$, $\varphi \coloneq \Phi$.
    
    \textbf{Fall 1:} $\Phi(0,1) = \Phi(0,0)$ für $0 \leq s \leq 1$ und $\Phi(1,s) = \Phi(1,0)$
    
    \begin{figure}[H]
      \centering
      \begin{pspicture}(0,0)(7,2)
        \psaxes[ticks=none,labels=none]{->}(0,0)(-0.5,-0.5)(2,2)[\color{DimGray} $t$,0][\color{DimGray} $s$,180]
        \psline[linecolor=DarkOrange3]{->}(0,0)(0.75,0)\psline[linecolor=DarkOrange3](0.75,0)(1.5,0)
        \psline[linecolor=DarkRed]{->}(1.5,0)(1.5,0.75)\psline[linecolor=DarkRed](1.5,0.75)(1.5,1.5)
        \psline[linecolor=DarkGreen]{->}(1.5,1.5)(0.75,1.5)\psline[linecolor=DarkGreen](0.75,1.5)(0,1.5)
        \psline[linecolor=DarkBlue]{->}(0,1.5)(0,0.75)\psline[linecolor=DarkBlue](0,0.75)(0,0)
        
        \uput[-90](0.75,0){\color{DarkOrange3} $\beta_1$}
        \uput[0](1.5,0.75){\color{DarkRed} $\beta_2$}
        \uput[90](0.75,1.5){\color{DarkGreen} $\beta_3$}
        \uput[180](0,0.75){\color{DarkBlue} $\beta_4$}
        
        \pnode(2,1){A}
        \pnode(4,1){B}
        \ncarc[arcangle=30,arrows=->]{A}{B}
        \naput{\color{DimGray} $\varphi = \Phi$}
        \psccurve(4.2,0.5)(5,-0.5)(7,-0.5)(8,-1)(9,2)(7,1.8)(6,2)(5,2)(4.2,1)
        \uput[-90](8,1.9){\color{DimGray} $O$}
        
        \cnode[linecolor=DarkBlue](6,0){3pt}{C}
        \uput[-90](C){\color{DarkBlue} $\varphi \circ \beta_4$}
        \cnode[linecolor=DarkRed](7.5,1){3pt}{D}
        \uput[0](D){\color{DarkRed} $\varphi \circ \beta_2$}
        \psdots*[dotscale=0.7](C)(D)
        \ncarc[linecolor=DarkOrange3,arcangle=-40,arrows=->]{C}{D}
        \nbput{\color{DarkOrange3} $\gamma_1 = \varphi \circ \beta_1$}
        \ncarc[linecolor=DarkGreen,arcangle=-50,arrows=->]{D}{C}
        \nbput{\color{DarkGreen} $-\gamma_2 = \varphi \circ \beta_3$}
      \end{pspicture}
    \end{figure}
    
    %
    \begin{gather*}
      \overset{\text{\ref{thm:2.4}}}{\implies}
      \int_{\varphi \circ \beta_1 = \gamma_1} f(z) \, \mathrm{d}z
      + \underbrace{\int_{\varphi \circ \beta_2} f(z) \, \mathrm{d}z}_{=0 \text{, da } \varphi \circ \beta_2 = \mathrm{const.}}
      + \underbrace{\int_{\varphi \circ \beta_3 = -\gamma_2} f(z) \, \mathrm{d}z}_{= - \int_{\gamma_2} f(z) \, \mathrm{d}z}
      + \underbrace{\int_{\varphi \circ \beta_4} f(z) \, \mathrm{d}z}_{=0 \text{, da } \varphi \circ \beta_4 = \mathrm{const.}} = 0 \\
      \implies \int_{\gamma_1} f(z) \, \mathrm{d}z = \int_{\gamma_2} f(z) \, \mathrm{d}z.
    \end{gather*}
    
    \textbf{Fall 2:} $\Phi(0,s) = \Phi(1,s)$ für $0 \leq s \leq t$
    
    \begin{figure}[H]
      \centering
      \begin{pspicture}(0,0)(7,2.5)
        \psaxes[ticks=none,labels=none]{->}(0,0)(-0.5,-0.5)(2,2)[\color{DimGray} $t$,0][\color{DimGray} $s$,180]
        \psline[linecolor=DarkOrange3]{->}(0,0)(0.75,0)\psline[linecolor=DarkOrange3](0.75,0)(1.5,0)
        \psline[linecolor=DarkRed]{->}(1.5,0)(1.5,0.75)\psline[linecolor=DarkRed](1.5,0.75)(1.5,1.5)
        \psline[linecolor=DarkGreen]{->}(1.5,1.5)(0.75,1.5)\psline[linecolor=DarkGreen](0.75,1.5)(0,1.5)
        \psline[linecolor=DarkBlue]{->}(0,1.5)(0,0.75)\psline[linecolor=DarkBlue](0,0.75)(0,0)
        
        \uput[-90](0.75,0){\color{DarkOrange3} $\beta_1$}
        \uput[0](1.5,0.75){\color{DarkRed} $\beta_2$}
        \uput[90](0.75,1.5){\color{DarkGreen} $\beta_3$}
        \uput[180](0,0.75){\color{DarkBlue} $\beta_4$}
        
        \pnode(2,1){A}
        \pnode(4,1){B}
        \ncarc[arcangle=30,arrows=->]{A}{B}
        \naput{\color{DimGray} $\varphi = \Phi$}
        \psccurve(4.2,0.5)(5,-0.5)(7,-0.5)(8,-1)(9,2)(7,1.8)(6,2)(5,2)(4.2,1)
        \uput[-90](8.5,1.9){\color{DimGray} $O$}
        
        \cnode*[linecolor=DimGray](5.5,0.8){2pt}{C}
        \pnode(5.5,0.7){C1}
        \uput[-135](C){\color{DimGray} $t=0$}
        \cnode*[linecolor=DimGray](7.5,0.8){2pt}{D}
        \pnode(7.5,0.7){D1}
        \psarc[linecolor=DarkOrange3]{->}(5.5,1.3){0.5}{90}{450}
        \psarc[linecolor=DarkGreen]{->}(7.5,1.3){0.5}{90}{450}
        \ncarc[linecolor=DarkRed,arcangle=-40,arrows=->]{C}{D}
        \naput{\color{DarkRed} $\varphi \circ \beta_1$}
        \ncarc[linecolor=DarkBlue,arcangle=40,arrows=->]{D1}{C1}
        \naput{\color{DarkBlue} $\varphi \circ \beta_4$}
        
        \uput[-45](D){\color{DarkGreen} $-\gamma_2 = \varphi \circ \beta_3$}
        \uput[90](5.5,2){\color{DarkOrange3} $\gamma_1 = \varphi \circ \beta_1$}
      \end{pspicture}
    \end{figure}
    
    \begin{gather*}
      \overset{\text{\ref{thm:2.4}}}{\implies}
      \int_{\varphi \circ \beta_1 = \gamma_1} f(z) \, \mathrm{d}z
      + \int_{\varphi \circ \beta_2} f(z) \, \mathrm{d}z
      + \underbrace{\int_{\varphi \circ \beta_3 = - \gamma_2} f(z) \, \mathrm{d}z}_{= - \int_{\gamma_2} f(z) \, \mathrm{d}z}
      + \int_{\varphi \circ \beta_4 = - \varphi \circ \beta_2} f(z) \, \mathrm{d}z = 0 \\
      \implies \int_{\gamma_1} f(z) \, \mathrm{d}z = \int_{\gamma_2} f(z) \, \mathrm{d}z.
    \end{gather*}
  \end{proof}
\end{notice}

\begin{proof}[Teilbeweis \ref{thm:2.2}]
  \ref{itm:2.2 i} $\implies$ \ref{itm:2.2 ii}: $f$ differenzierbar $\implies$ $f$ stetig, benutze \ref{thm:2.5}
  
  \ref{itm:2.2 ii} $\implies$ \ref{itm:2.2 iii}: $\gamma$ nullhomotop $\implies$ $\gamma \sim \widetilde{\gamma}$, da $\widetilde{\gamma} = \text{const.}$, und aus \ref{thm:2.5} folgt
  %
  \begin{align*}
    \int_{\gamma} f(z) \, \mathrm{d}z = \int_{\widetilde{\gamma}} f(z) \, \mathrm{d}z = 0 \; , \quad \text{da } \widetilde{\gamma} = \text{const.} \text{ (Integral über Punkt ist $0$)}
  \end{align*}
\end{proof}
  
\begin{theorem}[Cauchy-Integralformel für die Kreisscheibe] \label{thm:2.6}
  \text{Kreisscheibe:} Sei $O \subseteq \mathbb{C}$ offen, $f : O \to \mathbb{C}$ differenzierbar, $z_0 \in O$, $r > 0$ mit $\overline{K_r(z_0)} \subseteq O$. Dann gilt
  %
  \begin{align*}
    f(w) = \frac{1}{2 \pi \mathrm{i}} \int_{|z-z_0| = r} \frac{f(z)}{z - w} \, \mathrm{d}z \; , \quad \text{für } |w-z_0| < r
  \end{align*}
  %
  Vereinbarung: Das Integral ist zu verstehen längs der Kurve $\gamma : [0,1] \to O : t \mapsto z_0 + r \, \mathrm{e}^{\mathrm{i} 2 \pi t}$.
  
  Insbesondere: Ist $f(z)$ auf dem Kreisrand bekannt, dann auch im Kreisinneren.
  
  \begin{proof} $\gamma_\varepsilon(t) \coloneq w + \varepsilon \, \mathrm{e}^{2 \pi \mathrm{i} t}$, $0 \leq t \leq 1$
    
    \begin{figure}[H]
      \centering
      \begin{pspicture}(-1.5,-1.5)(1.5,1.5)
        \psarc[linecolor=MidnightBlue]{->}(0,0){1.5}{0}{360}
        \psdot*[linecolor=MidnightBlue](0,0)
        \uput[-45](0,0){\color{MidnightBlue} $z_0$}
        \psarc[linecolor=DarkOrange3]{->}(-0.5,0.5){0.5}{-90}{270}
        \psdot*[linecolor=DarkOrange3](-0.5,0.5)
        \uput[0](-0.5,0.5){\color{DarkOrange3} $w$}
        \uput[-90](-0.5,0){\color{DarkOrange3} $\gamma_\varepsilon$}
        \uput[-90](0,-1.5){\color{MidnightBlue} $\gamma$}
        \uput{1.7}[-30](0,0){\color{MidnightBlue} $|z-z_0| = r$}
      \end{pspicture}
    \end{figure}
    
    Offensichtlich: $\gamma \sim \gamma_\varepsilon$ in $O \setminus \{ w \}$ und $f(z)/(z-w)$ ist differenzierbar in $O \setminus \{ w \}$.
    %
    \begin{align*}
      \overset{\text{\ref{thm:2.4}}}{\implies} \int_{|z-z_0| = r} \frac{f(z)}{z-w} \, \mathrm{d}z 
      &= \int_{\gamma_\varepsilon} \frac{f(z)}{z-w} \, \mathrm{d}z \\
      &= \underbrace{\int_{\gamma_\varepsilon} \underbrace{\frac{f(z) - f(w)}{z-w}}_{|\cdot| \leq c \text{, weil diffbar}} \, \mathrm{d}z}_{|\cdot| \leq c \, L(\gamma_\varepsilon) = c \, 2 \pi \varepsilon \to 0 \text{ für } \varepsilon \downarrow 0}
      + f(w) \underbrace{\int_{\gamma_\varepsilon} \frac{1}{z-w} \, \mathrm{d}z}_{= 2 \pi \mathrm{i}} \\
      %
      \overset{\varepsilon \downarrow 0}{\implies} \int_{|z-z_0| = r} \frac{f(z)}{z-w} \, \mathrm{d}z &= 2 \pi \mathrm{i} f(w)
    \end{align*}
    
  \end{proof}
\end{theorem}

\begin{theorem}[Potenzreihenentwicklungssatz] \label{thm:2.7}
  Sei $O \subseteq \mathbb{C}$ offen, $f : O \to \mathbb{C}$ differenzierbar, $z_0 \in O$, $\overline{K_r(z_0)} \subseteq O$. Dann gibt es eine Folge $(a_n) \in \mathbb{C}$ mit
  %
  \begin{align*}
    f(z) = \sum\limits_{n=0}^{\infty} a_n (z-z_0)^n \; , \quad \text{für } |z-z_0| < r.
  \end{align*}
  
  \begin{proof}
    Für $w \in K_r(z_0)$ gilt
    %
    \begin{align*}
      f(w) &\overset{\text{\ref{thm:2.6}}}{=} \frac{1}{2 \pi \mathrm{i}} \int_{|z-z_0| = r} \frac{f(z)}{z-w} \, \mathrm{d}z \\
      &= \frac{1}{2 \pi \mathrm{i}} \int_{|z-z_0| = r} f(z) \frac{1}{z-z_0-(w-z_0)} \, \mathrm{d}z \\
      &= \frac{1}{2 \pi \mathrm{i}} \int_{|z-z_0| = r} f(z) \frac{1}{z-z_0} \frac{1}{1-\frac{w-z_0}{z-z_0}} \, \mathrm{d}z \\
      &= \frac{1}{2 \pi \mathrm{i}} \int_{|z-z_0| = r} f(z) \frac{1}{z-z_0} \sum\limits_{n=0}^{\infty} \left( \frac{w-z_0}{z-z_0} \right)^n \, \mathrm{d}z
    \intertext{
      Die geometrische Reihe konvergiert, wegen $\left| \frac{w-z_0}{z-z_0} \right| = \frac{|w-z_0|}{r} < 1$.
      Außerdem konvergiert $\sum\limits_{n=0}^{\infty} \left(\frac{|w-z_0|}{r}\right)^n$ und somit ist die Reihe nach Weierstraß (\ref{thm:1.14}) gleichmäßig konvergent bezüglich $z$.
      Wir dürfen also Reihe und Integral vertauschen: 
    }
      &= \sum\limits_{n=0}^{\infty} \underbrace{\frac{1}{2 \pi \mathrm{i}} \int_{|z-z_0| = r} \frac{f(z)}{(z-z_0)^{n+1}} \, \mathrm{d}z}_{a_n} \left( w-z_0 \right)^n
    \end{align*}
  \end{proof}
\end{theorem}

\begin{notice}[Folgerung]
  Unter diesen Voraussetzungen gilt für den Konvergenzradius $R$ der Potenzreihe
  %
  \begin{align*}
    R \geq \sup \{ r > 0 : \overline{K_r(z_0)} \subseteq O \}.
  \end{align*}
\end{notice}

\begin{example}
  $O = \mathbb{C} \setminus \{ \pm \mathrm{i} \}$, $f(z) = \dfrac{1}{1+z^2}$.
  
  \begin{figure}[H]
    \centering
    \begin{pspicture}(-0.5,-1.2)(2,1.5)
      \psaxes[ticks=none,labels=none]{->}(0,0)(-0.5,-1.2)(2,1.5)[\color{DimGray} Re,0][\color{DimGray} Im,180]
      \uput[180](0,1){\color{DimGray} $1$}
      \uput[180](0,-1){\color{DimGray} $-1$}
      \pnode(0,1){A}
      \pnode(0,-1){B}
      \pnode(1.5,-0.3){C}
      \ncline[linecolor=DarkOrange3]{A}{C}\naput{\color{DarkOrange3} $|z_0 - \mathrm{i}|$}
      \ncline[linecolor=DarkOrange3]{B}{C}\nbput{\color{DarkOrange3} $|z_0 + \mathrm{i}|$}
      \uput[-45](C){\color{DimGray} $z_0$}
      \psdots*(A)(B)(C)
    \end{pspicture}
  \end{figure}
  
  Entwickle $f$ um $z_0$ in eine Potenzreihe mit Konvergenzradius $R$. Aus 2.8 folgt
  %
  \begin{align*}
    R \geq \min \{ |z_0+\mathrm{i}|,|z_0-\mathrm{i}| \}
  \end{align*}
  %
  Da $|f(z)| \to \infty$ für $z \to \pm \mathrm{i}$ kann die Potenzreihe in $z = \pm \mathrm{i}$ nicht konvergieren, also ist $R = \min \{ |z_0+\mathrm{i}|,|z_0-\mathrm{i}|$.
\end{example}

% % % Vorlesung vom 5.11.2012

\begin{notice}[Folgerung] \label{thm:2.10}
  \begin{enum-arab}
    \item 
    \begin{align*}
      a_n = \frac{1}{2 \pi \mathrm{i}} \int_{|z-z_0| = r} \frac{f(z)}{(z-z_0)^{n+1}} \, \mathrm{d}z \text{ für die $a_n$ in \ref{thm:2.7}}
    \end{align*}
    
    \item
    \begin{align*}
      f(z) &= \sum\limits_{n=0}^{\infty} a_n (z-z_0)^n \overset{\text{Übung}}{\implies} a_n = \frac{f^{(n)}(z_0)}{n!} \\
      \implies f^{(n)}(z_0) &= \frac{n!}{2 \pi \mathrm{i}} \int_{|z-z_0| = r} \frac{f(z)}{(z-z_0)^{n+1}} \, \mathrm{d}z
    \end{align*}
    %
    Auf einer Kreisscheibe um $z_0$ gilt analog zur gewöhnlichen Cauchy-Integralformel (\ref{thm:2.6}) sogar allgemeiner:
    %
    \begin{align*}
      f^{(n)}(w) &= \frac{n!}{2 \pi \mathrm{i}} \int_{|z-z_0| = r} \frac{f(z)}{(z-w)^{n+1}} \, \mathrm{d}z \qquad |w-z_0|<r
    \end{align*}
    %
    (Verwende dazu $z_0 := w$ und $r := |w-z_0|$, sowie die Weghomotopie beider Kurven auf der Kreisscheibe)
    Man nennt diese Form auch \acct{Erweiterte Cauchy'sche Integralformel}.
  \end{enum-arab}
\end{notice}

\begin{theorem}[Definition] \label{thm:2.11}
  Ist $f:\mathbb{C} \to \mathbb{C}$ differenzierbar, so heißt $f$ \acct{ganze Funktion}. 
\end{theorem}
  
\begin{notice*}
  Für eine ganze Funktion $f: \mathbb{C} \to \mathbb{C}$ gilt mit beliebigem $z_0 \in \mathbb{C}$
  %
  \begin{align*}
    f(z) = \sum\limits_{n=0}^{\infty} a_n (z-z_0)^n
  \end{align*}
  %
  mit Konvergenzradius $R = \infty$ und $a_n$ gegeben durch \ref{thm:2.10}
\end{notice*}

\begin{example}
  $\mathrm{e}^{(\cdot)}$, Polynomfunktionen, $\sin$, $\cos$ sind ganze Funktionen.
\end{example}

\begin{proof}[Teilbeweis \ref{thm:2.2}]
  \ref{itm:2.2 i} $\implies$ \ref{itm:2.2 iv}
  %
  \begin{align*}
    & z_0 \in O \; , \quad O \text{ offen} \\
    \implies& \exists \, r > 0 : K_r(z_0) \subseteq O \\
    \implies& \overline{K_{r/2}(z_0)} \subseteq K_r(z_0) \subseteq O \\
    \overset{\text{\ref{thm:2.7}}}{\implies}& \text{\ref{itm:2.2 iv}} \text{ mit } R \geq \frac{r}{2}
  \end{align*}
  
  \ref{itm:2.2 iv} $\implies$ \ref{itm:2.2 v}: \ref{thm:1.35}
  
  \ref{itm:2.2 v} $\implies$ \ref{itm:2.2 i}: offensichtlich
\end{proof}

\begin{theorem}[Cauchy-Abschätzung für Taylor-Koeffizienten] \label{thm:2.13}
  Sei $O \subseteq \mathbb{C}$ offen, $f:O \to \mathbb{C}$ differenzierbar, $\overline{K_r(z_0)} \subseteq O$, $|f(z)| \leq M$ auf $|z-z_0| = r$ und
  %
  \begin{align*}
    f(z) = \sum\limits_{n=0}^{\infty} a_n (z-z_0)^n \; , \quad \text{für } |z-z_0| < r.
  \end{align*}
  %
  Dann gilt
  %
  \begin{align*}
    \left| \frac{f^{(n)}(z_0)}{n!} \right| = |a_n| \leq \frac{M}{r^n} \; , \quad \text{für } n \in \mathbb{N}_0.
  \end{align*}
  
  \begin{proof}
    \begin{align*}
      |a_n| &\overset{\text{\ref{thm:2.10}}}{=} \Bigg| \frac{1}{2 \pi \mathrm{i}} \int_{|z-z_0| = r} \underbrace{\frac{f(z)}{(z-z_0)^{n+1}}}_{|\cdot| \leq \frac{M}{r^{n+1}}} \, \mathrm{d}z \Bigg| \\
      &\overset{\text{\ref{thm:1.32}}}{\leq} \frac{1}{2 \pi} \frac{M}{r^{n+1}} \underbrace{L(|z-z_0| = r)}_{2 \pi r} \\
      &= \frac{M}{r^n}.
    \end{align*}
  \end{proof}
\end{theorem}

\begin{theorem}[Satz von Liouville]
  Jede beschränkte, ganze Funktion ist konstant.
  
  \begin{proof}
    Aus \ref{thm:2.11} erhalten wir die Potenzreihenentwicklung um $0$:
    %
    \begin{align*}
      f(z) = \sum\limits_{n=0}^{\infty} a_n z^n \; , \quad \text{für } z \in \mathbb{C}.
    \end{align*}
    %
    Sei $M$ obere Schranke von $|f(z)|$. 
    Wegen $\overline{K_r(0)} \subseteq \mathbb{C}$ für beliebige $r>0$, liefert \ref{thm:2.13}
    %
    \begin{align*}
      |a_n| \leq \frac{M}{r^n} 
    \end{align*}
    %
    Dann folgt mit $r \to \infty$
    %
    \begin{gather*}
      \implies a_n = 0 \; , \quad \text{für } n = 1, 2, \ldots \\
      \implies f(z) = a_0 + 0  \; , \quad \text{für } z \in \mathbb{C}
    \end{gather*}
  \end{proof}
\end{theorem}

\begin{theorem}[Riemannscher Hebbarkeitssatz]
  Sei $O \subseteq \mathbb{C}$ offen, $z_0 \in O$, $f:O \setminus \{ z_0 \} \to \mathbb{C}$ differenzierbar und 
  %
  \begin{align*}
    \exists \, r > 0 \, \exists \, M > 0 : |f(z)| \leq M \; , \quad \text{für } 0 < |z-z_0| < r
  \end{align*}
  %
  Dann kann $f$ in $z=z_0$ holomorph ergänzt werden, d.h. es existiert $a \in \mathbb{C}$, so dass
  %
  \begin{align*}
    \widetilde{f} : O \to \mathbb{C} : z \mapsto
    \begin{cases}
      a & z=z_0 \\
      f(z) & \text{sonst}
    \end{cases}
  \end{align*}
  %
  differenzierbar ist.
  
  \begin{proof}
    Setze
    %
    \begin{align*}
      g(z) \coloneq
      \begin{cases}
        0 & z=z_0 \\
        (z-z_0)^2 \, f(z) & \text{sonst}
      \end{cases}
    \end{align*}
    %
    Dann ist $g$ in $O \setminus \{ z_0 \}$ differenzierbar und auch in $z_0$
    %
    \begin{align*}
      & \lim\limits_{\substack{z \to z_0}{z \neq z_0}} \frac{g(z) - g(z_0)}{z - z_0} \\
      =& \lim\limits_{\substack{z \to z_0}{z \neq z_0}} \underbrace{(z-z_0)}_{\to 0} \, \overbrace{f(z)}^{\mathclap{\text{beschränkt}}} = 0
    \end{align*}
    %
    Also ist $g$ differenzierbar in $O$ mit $g(z_0) = g'(z_0) = 0$.
    %
    \begin{align*}
      \overset{\text{\ref{thm:2.2}, \ref{itm:2.2 i} $\implies$ \ref{itm:2.2 iv}}}{\implies} &
        g(z) = \sum\limits_{n=0}^{\infty} a_n (z-z_0)^n \; , \quad |z-z_0| \leq r,
    \intertext{Für die ersten beiden Folgenglieder gilt $a_0 = g(z_0) = 0$ und $a_1 = g'(z_0) = 0$:}
     \implies & g(z) = \sum\limits_{n=2}^{\infty} a_n (z-z_0)^n \; , \quad |z-z_0| < r \\
     \implies & f(z) = \frac{g(z)}{(z-z_0)^2} = \sum\limits_{n=2}^{\infty} a_n (z-z_0)^{n-2} \; , \quad \text{für } 0 < |z-z_0| < r
    \end{align*}
    %
    Setze $a \coloneq a_2$ für Definition von $\widetilde{f}$
    %
    \begin{align*}
      \implies & \widetilde{f}(z) = \sum\limits_{n=2}^{\infty} a_n (z-z_0)^{n-2} \; , \quad |z-z_0| < r
    \end{align*}
    %
    also differenzierbar in $O$.
  \end{proof}
\end{theorem}

\begin{example}
  $f : O \to \mathbb{C}$ differenzierbar, $z_0 \in O$ und
  %
  \begin{align*}
    g(z) \coloneq 
    \begin{dcases}
      \frac{f(z) - f(z_0)}{z - z_0} & \text{sonst} \\
      f'(z_0) & z = z_0
    \end{dcases}
  \end{align*}
  %
  Dann ist $g$ holomorph auf $O$.
\end{example}

\begin{theorem}[Hilfssatz]\label{thm:2.17}
  Sei $O \subseteq \mathbb{C}$ offen, $f:O \to \mathbb{C}$ erfülle \ref{itm:2.2 iii} aus \ref{thm:2.2}.
  Ist $D \subseteq O$ eine abgeschlossene Dreiecksfläche mit geschlossener Randkurve $\partial D$ (stückweise Intervalle), so gilt
  %
  \begin{align*}
    \int_{\partial D} f(z) \, \mathrm{d}z = 0.
  \end{align*}
  
  \begin{proof}
    Hauptidee: $\partial D$ umparametrisieren zu einer $C^1$-Kurve
    %
    \begin{align*}
      p(t) \coloneq 3 t^2 - 2 t^3.
    \end{align*}
    
    \begin{figure}[H]
      \centering
      \begin{pspicture}(-0.6,-0.5)(2,1.5)
        \psaxes[labels=none,ticks=none]{->}(0,0)(-0.5,-0.5)(2,2)
        \psplot[linecolor=DarkOrange3]{-0.6}{1.6}{3*x^2 - 2*x^3}
        \uput[45](1,1){\color{DarkOrange3} $y = p(t)$}
      \end{pspicture}
    \end{figure}
    
    \begin{align*}
      \implies
      \begin{dcases}
        p(0) = 0 \; , \quad p(1) = 1 \\
        p'(t) > 0 \; , \quad \text{für } 0 < t < 1 \\
        p'(0) = 0 \; , \quad p'(1) = 0
      \end{dcases}
    \end{align*}
    %
    Sei $\partial D : [0,3] \to O$ gegeben durch
    %
    \begin{align*}
      \gamma(t) =
      \begin{dcases}
        \alpha_1(t) & 0 \leq t \leq 1 \\
        \alpha_2(t) & 1 \leq t \leq 2 \\
        \alpha_3(t) & 2 \leq t \leq 3
      \end{dcases}
    \end{align*}
    
    \begin{figure}[H]
      \centering
      \begin{pspicture}(0,0)(2,2)
        \pnode(0,0){A}
        \pnode(2;10){B}
        \pnode(2;70){C}
        \ncline[linecolor=MidnightBlue,arrows=->]{A}{B}
        \nbput{\color{MidnightBlue} $\alpha_1$}
        \ncline[linecolor=DarkOrange3,arrows=->]{B}{C}
        \nbput{\color{DarkOrange3} $\alpha_2$}
        \ncline[linecolor=DarkGreen,arrows=->]{C}{A}
        \nbput{\color{DarkGreen} $\alpha_3$}
      \end{pspicture}
    \end{figure}
    
    Sei 
    %
    \begin{align*}
      \beta_1(t) \coloneq \alpha_1(p(3t)) \qquad 0 \leq t \leq \frac 13
    \end{align*}
    %
    Dann ist
    %
    \begin{align*}
      \int_{\beta_1} f(z) \, \mathrm{d}z \overset{\text{1.28}}{=} \int_{\alpha_1} f(z) \, \mathrm{d}z \; , \quad \beta_1'(0) = 0, \; \beta_1' (\frac{1}{3}) = 0
    \end{align*}
    %
    Analog ergibt sich für $\beta_2$ und $\beta_3$:
    %
    \begin{align*}
      \beta_2(t) &\coloneq \alpha_2\left( 1 + p \left( 3 \left( t - \frac{1}{3} \right) \right) \right) \; , \quad \frac{1}{3} \leq t \leq \frac{2}{3}  \\
      \beta_3(t) &\coloneq \alpha_3\left( 2 + p \left( 3 \left( t - \frac{2}{3} \right) \right) \right) \; , \quad \frac{2}{3} \leq t \leq 1
    \end{align*}
    \begin{align*}
      \int_{\beta_2} f(z) \, \mathrm{d}z &\overset{\text{1.28}}{=} \int_{\alpha_2} f(z) \, \mathrm{d}z \; , \quad \beta_2'\left(\frac 13\right) = 0, \; \beta_2' \left(\frac{2}{3}\right) = 0, \\
      \int_{\beta_3} f(z) \, \mathrm{d}z &\overset{\text{1.28}}{=} \int_{\alpha_3} f(z) \, \mathrm{d}z \; , \quad \beta_3'\left(\frac 23\right) = 0, \; \beta_3' (1) = 0.
    \end{align*}
    %
    Setze
    %
    \begin{gather*}
      \widetilde{\gamma}(t) \coloneq
      \begin{dcases}
        \beta_1(t) & 0 \leq t \leq \frac{1}{3} \\
        \beta_2(t) & \frac{1}{3} \leq t \leq \frac{2}{3} \\
        \beta_3(t) & \frac{2}{3} \leq t \leq 1
      \end{dcases} \quad, \\
      \implies \widetilde{\gamma} \in C^1([0,1] \to \mathbb{C}) \text{ ist Randkurve von $D$}
    \end{gather*}
    %
    Dann ist $\widetilde{\gamma}$ nullhomotop:
    %
    \begin{gather*}
      \Phi(t,s) \coloneq \underbrace{(1-s) \widetilde{\gamma}(t) + s \widetilde{\gamma}(0)}_{\in D \subseteq O \text{ da $D$ konvex}} \; , \quad \Phi \in C^1 \\
      \implies \widetilde{\gamma} \sim \widetilde{\gamma}(0) \text{, also $C^1$-nullhomotop} \\
      \overset{\text{\ref{itm:2.2 iii} von \ref{thm:2.2}}}{\implies} \int_{\widetilde{\gamma}} f(z) \, \mathrm{d}z = 0.
    \end{gather*}
  \end{proof}
\end{theorem}

\begin{theorem}[Satz von Morera]
  Sei $O \subseteq \mathbb{C}$ offen, $f:O \to \mathbb{C}$ stetig und für jede abgeschlossene Dreiecksfläche $D \subseteq O$ gelte
  %
  \begin{align*}
    \int_{\partial D} f(z) \, \mathrm{d}z = 0
  \end{align*}
  %
  Dann ist $f$ holomorph auf $O$.
  
  \begin{proof}
    Zeige, dass $f$ in einem Kreis $K_r(z_0)$ mit $\overline{K_r(z_0)} \subseteq O$ eine Stammfunktion $F$ besitzt. 

    Sei dazu $z_0 \in O$ beliebig und $\overline{K_r(z_0)} \subset O$ gewählt.
    Setze
    %
    \begin{align*}
      F(z) \coloneq \int_{[z_0,z]} f(\zeta) \, \mathrm{d}\zeta \; , \quad \text{für } z \in K_r(z_0)
    \end{align*}
    %
    Behauptung: $F' = f$ in $K_r(z_0)$. Sei $z \in K_r(z_0)$.
    Dann gilt
    %
    \begin{align*}
       \left| \frac{F(w) - F(z)}{w - z} - f(z) \right| 
      &= \left| \frac{1}{w - z} \left( \int_{[z_0,w]} f(\zeta) \, \mathrm{d}\zeta - \int_{[z_0,z]} f(\zeta) \, \mathrm{d}\zeta - \int_{[z,w]} f(z) \, \mathrm{d}\zeta \right. \right. \\
      &\qquad \left. \left. \phantom{\frac{1}{w - z} \qquad} + \int_{[w,z]} f(\zeta) \, \mathrm{d}\zeta + \int_{[z,w]} f(\zeta) \, \mathrm{d}\zeta \right) \right|
    \end{align*}
    
    \begin{figure}[H]
      \centering
      \caption{Es gilt $\int_{[z_0,w]} f(\zeta) \, \mathrm{d}\zeta - \int_{[z_0,z]} f(\zeta) \, \mathrm{d}\zeta + \int_{[w,z]} f(\zeta) \, \mathrm{d}\zeta = 0$}
      \begin{pspicture}(0,0)(2,1.5)
        \cnode(0,1){2pt}{A}
        \cnode(1,-0.5){2pt}{B}
        \cnode(2,1){2pt}{C}
        \uput[180](A){\color{DimGray} $z$}
        \uput[-90](B){\color{DimGray} $z_0$}
        \uput[0](C){\color{DimGray} $w$}
        \ncline[linecolor=DarkOrange3,arrows=->]{A}{B}
        \nbput{\color{DarkOrange3} $-[z_0,z]$}
        \ncline[linecolor=DarkOrange3,arrows=->]{B}{C}
        \nbput{\color{DarkOrange3} $[z_0,w]$}
        \ncline[linecolor=MidnightBlue,arrows=->]{C}{A}
        \nbput{\color{MidnightBlue} $[w,z]$}
      \end{pspicture}
    \end{figure}
    
    \begin{align*}
      &= \frac{1}{|w - z|} \left| \int_{[z,w]} \Big( f(\zeta) - f(z) \Big) \, \mathrm{d}\zeta \right| \\
      &\leq \frac{1}{|w - z|} \underbrace{\max\limits_{\zeta \in [z,w]} \Big|f(\zeta) - f(z)\Big|}_{< \varepsilon \text{ für } |w-z| < \delta \text{, da $f$ stetig}} \, \underbrace{L([z,w])}_{=|w-z|} \\
      &< \varepsilon \; , \quad \text{für } |w-z| < \delta
    \end{align*}
    %
    Damit gilt $F' = f$. 
    $F$ ist demnach differenzierbar auf $K_r(z_0)$, also nach \ref{thm:2.2} beliebig oft differenzierbar.
    Also ist auch $f$ (beliebig oft) differenzierbar auf $K_r(z_0)$.
    Da $z_0\in O$ beliebig gewählt war, ist $f$ auf ganz $O$ differenzierbar und damit holomorph.
  \end{proof}
\end{theorem}

% % % Vorlesung vom 8.11.2012

\begin{proof}[Teilbeweis \ref{thm:2.2}]
  \ref{itm:2.2 iii} $\overset{\text{\ref{thm:2.17}}}{\implies}$ Vorraussetzung von Morera erfüllt.
  
  $\overset{\text{Morera}}{\implies}$ \ref{itm:2.2 i} $f$ differenzierbar in $O$.
\end{proof}

\begin{notice}
  \begin{enum-arab}
    \item Beweis von Morera zeigt
    %
    \begin{align*}
      \text{$f$ differenzierbar } &\iff \text{ $f$ besitzt eine lokale Stammfunktion, d.h.} \\
      &\phantom{\iff} \forall \, z_0 \in O \, \exists \, r > 0 \, \exists \, F : F' = f \text{ in } K_r(z_0)
    \end{align*}
    
    \item Wenn $f$ differenzierbar in $O$ ist, muss $f$ nicht unbedingt eine globale Stammfunktion (Stammfunktion auf $O$) besitzen: Z.B. $O = \mathbb{C} \setminus \{0\}$, $f(z) = 1/z$.
    
    Später: Stammfunktionen können holomorph fortgesetzt werden, z.B. $\ln z$ als Stammfunktion von $1/z$.
  \end{enum-arab}
\end{notice}

\begin{theorem}[Satz] \label{thm:2.20}
  Sei $O \subseteq \mathbb{C}$ offen, $f : O \to \mathbb{C}$, $z_0 = x_0 + \mathrm{i} y_0 \in O$
  %
  \begin{align*}
    \widetilde{O} &\coloneq \{(x,y) \in \mathbb{R}^2 : x + \mathrm{i} y \in O\} \\
    u(x,y) &\coloneq \Re f(x + \mathrm{i} y) \\
    v(x,y) &\coloneq \Im f(x + \mathrm{i} y)
  \end{align*}
  %
  Es sind äquivalent:
  %
  \begin{enum-roman}
    \item \label{itm:2.20 i} $f$ ist differenzierbar in $z_0$.
    
    \item \label{itm:2.20 ii} $\begin{pmatrix} u \\ v \end{pmatrix} : \widetilde{O} \subseteq \mathbb{R}^2 \to \mathbb{R}^2$ ist differenzierbar in $(x_0,y_0)$ und es gelten die Cauchy-Riemannschen Differentialgleichungen:
    %
    \begin{align*}
      u_x(x_0,y_0) &= v_y(x_0,y_0) \\
      u_y(x_0,y_0) &= -v_x(x_0,y_0) .
    \end{align*}
  \end{enum-roman}
  %
  Ist (i) oder (ii) erfüllt, so gilt
  %
  \begin{align*}
    u_x(x_0,y_0) &= \Re f'(z_0) \\
    u_y(x_0,y_0) &= - \Im f'(z_0).
  \end{align*}
  
  \begin{proof}
    %
    \begin{align*}
      \text{$f$ differenzierbar } &\iff f(z) = f(z_0) + f'(z_0)(z-z_0) + o(|z-z_0|) \\
      &\iff
      \begin{dcases}
        \Re f(z) = \Re f(z_0) + \Re f'(z_0) \Re (z-z_0) - \Im f'(z_0) \Im (z-z_0) \\ \mspace{100mu} + o(|z-z_0|) \\
        \Im f(z) = \Im f(z_0) + \Im f'(z_0) \Re (z-z_0) - \Re f'(z_0) \Im (z-z_0) \\ \mspace{100mu} + o(|z-z_0|)
      \end{dcases}
    \end{align*}
    %
    Sei $z = x + \mathrm{i} y$, dann zerfallen
    %
    \begin{align*}
      \Re (z-z_0) &= x - x_0 \\
      \Im (z-z_0) &= y - y_0
    \end{align*}
    %
    \begin{align*}
      \phantom{\text{\ref{itm:2.20 i}}} &\iff
      \begin{pmatrix}
        u(x,y) \\ v(x,y)
      \end{pmatrix}
      =
      \begin{pmatrix}
        u \\ v
      \end{pmatrix}
      (x_0,y_0) +
      \begin{pmatrix}
        \Re f'(z_0) & -\Im f'(z_0) \\
        \Im f'(z_0) & \Re f'(z_0)
      \end{pmatrix}
      \begin{pmatrix}
        x-x_0 \\
        y-y_0
      \end{pmatrix}\\
      &\phantom{\iff\quad}
      +
      o
      \left(\left|
      \begin{pmatrix}
        x \\ y
      \end{pmatrix}
      -
      \begin{pmatrix}
        x_0 \\ y_0
      \end{pmatrix}
      \right|\right) \\
      %
      &\iff
      \begin{pmatrix}
        u \\ v
      \end{pmatrix}
      \text{ ist differenzierbar in } (x_0,y_0) \text{ mit der Jacobi-Matrix} \\
      &\phantom{\iff\quad}
      \begin{pmatrix}
        \Re f'(z_0) & -\Im f'(z_0) \\
        \Im f'(z_0) & \Re f'(z_0)
      \end{pmatrix}
      =
      \begin{pmatrix}
        u_x & u_y \\
        v_x & v_y
      \end{pmatrix}
      (x_0,y_0)
    \end{align*}
  \end{proof}
\end{theorem}

\begin{proof}[Restbeweis \ref{thm:2.2}]
  \begin{align*}
    \ref{itm:2.2 vi}
    &\implies \begin{pmatrix} u \\ v \end{pmatrix} : \widetilde{O} \to \mathbb{R}^2 \text{ differenzierbar und es gelten die Cauchy-Riemann DGLs} \\
    &\overset{\text{\ref{thm:2.20}}}{\implies} \ref{itm:2.2 i}
  \end{align*}
  
  \begin{align*}
    \ref{itm:2.2 i}
    &\overset{\text{\ref{thm:2.20}}}{\implies} \begin{pmatrix} u \\ v \end{pmatrix} \text{ differenzierbar in $O$ und es gelten die Cauchy-Riemann DGLs} \\
    &\implies \ref{itm:2.2 v} \\
    &\implies \ref{itm:2.2 v} \implies f' \text{ stetig } \overset{\text{\ref{thm:2.20}}}{\implies} u_x, u_y, v_x, v_y \text{ stetig} \\
    &\implies \ref{itm:2.2 vi}
  \end{align*}
\end{proof}

\begin{example}
  Sei $O = \mathbb{C}$, $f(z) = \sin z = \frac{1}{2 \mathrm{i}} \left( e^{\mathrm{i} z} - e^{-\mathrm{i} z} \right)$. Zum selbst nachrechnen
  %
  \begin{align*}
    = \underbrace{\cosh y \sin x}_{u(x,y)} + \mathrm{i} \underbrace{\sinh y \cos x}_{v(x,y)}
  \end{align*}
  %
  \begin{align*}
    \left.
    \begin{matrix}
      u_x = \cosh y \cos x \\
      v_y = \cosh y \cos x \\
    \end{matrix}
    \right\} \text{>> $=$ <<}
    \\
    \left.
    \begin{matrix}
      u_y = \sinh y \sin x \\
      v_x = - \sinh y \sin x \\
    \end{matrix}
    \right\} \text{>> $=$ <<}
  \end{align*}
\end{example}

\begin{theorem}[Definition]
  \begin{enum-arab}
    \item $M \subseteq \mathbb{C}$ heißt \acct{wegzusammenhängend}, falls gilt
    %
    \begin{align*}
      \forall \, z_0,z_1 \in M \, \exists \, \gamma:[a,b] \to M \text{ Weg } \text{ stückweise } C^1 : \gamma(a) = z_0 \lor \gamma(b) = z_1.
    \end{align*}
    %
    \item $G \subseteq \mathbb{C}$ heißt \acct{Gebiet}, falls $G$ offen und wegzusammenhängend.
  \end{enum-arab}
\end{theorem}

\begin{theorem}[Satz]
  Sei $G \subseteq \mathbb{C}$ Gebiet, $f$ holomorph in $G$ mit $f' = 0$ in $G$. Dann ist $f$ konstant in $G$.
  
  \begin{proof}
    Sei $z_0 \in G$ fest, $z \in G$, $\gamma$ Weg von $z_0$ nach $z$.
    Dann gilt
    %
    \begin{gather*}
      0 = \int_{\gamma} f'(\zeta) \, \mathrm{d}\zeta = f(z) - f(z_0) \\
      \implies \forall \, z \in G : f(z) = f(z_0)
    \end{gather*}
  \end{proof}
\end{theorem}

\begin{theorem}[Satz] \label{thm:2.24}
  Sei $G \in \mathbb{C}$ Gebiet, $f$ holomorph in $G$ und
  %
  \begin{align*}
    \exists \, z_0 \in G \, \forall \, n \in \mathbb{N} : f^{(n)}(z_0) = 0.
  \end{align*}
  %
  Damit ist $f$ konstant in G.
  
  \begin{proof}
    Sei $z \in G$, $\gamma$ ein Weg von $z_0$ nach $z$: $\gamma(a) = z_0$, $\gamma(b) = z$. Setze
    %
    \begin{align*}
      T \coloneq \sup \left\{ t \in [a,b] : f \circ \gamma \Big|_{[a,t]} = \text{const.} \right\} \neq \emptyset \text{, da $t=a$ enthalten.}
    \end{align*}
    
    \begin{figure}[H]
      \centering
      \begin{pspicture}(-0.5,-0.5)(4,0.5)
        \pscircle[linecolor=DarkOrange3,fillstyle=hlines,hatchcolor=DarkOrange3,hatchsep=2pt](A){0.5}
        \pscircle[linecolor=DarkOrange3,fillstyle=hlines,hatchcolor=DarkOrange3,hatchsep=2pt](1,0.2){0.3}
        \cnode*(0,0){2pt}{A}
        \cnode*(4,0){2pt}{B}
        \uput[-90](A){\color{DimGray} $z_0$}
        \uput[-90](B){\color{DimGray} $z$}
        \pscurve{->}(A)(1,0.2)(3,-0.2)(B)
        \psline[linecolor=DarkRed]{->}(1,0.2)(2,0.15)
        \uput[0](2,0.15){\color{DarkRed} $\gamma$}
        \psdot*[linecolor=MidnightBlue](1,0.2)
        \uput{0.35}[-75](1,0.2){\color{MidnightBlue} $\gamma(T)$}
      \end{pspicture}
    \end{figure}
    
    \begin{enum-arab}
      \item \label{itm:2.24 1} Zeige $T > a$.
      
      \item \label{itm:2.24 2} Zeige $T = b$ (dann $f(z) = f(z_0)$ für alle $z \in G$).
    \end{enum-arab}
    
    Zu \ref{itm:2.24 1}
    %
    \begin{align*}
      f(z) &\overset{\text{\ref{thm:2.7}}}{\underset{\text{\ref{thm:2.10}}}{=}} \sum\limits_{n=0}^{\infty} \frac{f^{(n)}(z_0)}{n!} (z-z_0)^n \quad \text{für } |z-z_0| < R \\
      &= f(z_0)
    \end{align*}
    %
    $\gamma$ stetig $\implies$ $\exists \, \delta > 0 \, \forall \, t \in [a,a+\delta[ : |\gamma(t) - \underbrace{\gamma(a)}_{=z_0}| < R$
    
    $\implies$ $T \geq a + \delta > a$
    
    Zu \ref{itm:2.24 2}: Annahme: $T < b$. Behauptung
    %
    \begin{align*}
      f^{(n)} (\gamma(t)) = 0 \quad \text{für } n \in \mathbb{N}, a \leq t \leq T
    \end{align*}
    %
    Induktionsanfang:
    %
    \begin{align*}
      f'(\gamma(t)) &= \lim\limits_{z \to \gamma(t)} \frac{f(z) - f(\gamma(t))}{z - \gamma(t)} \\
      &\overset{\text{Teilfolge}}{=} \lim\limits_{\substack{s \to t \\ s \in [a,T]}} \frac{f(\gamma(s)) - f(\gamma(t))}{\gamma(s) - \gamma(t)} \overset{f|_{[a,T]}}{=} 0
    \end{align*}
    %
    Induktionsschritt genauso.
    
    $\implies$ $\forall \, n \in \mathbb{N} : f^{(n)}(\gamma(T)) = 0$
    
    $\overset{\text{wie \ref{itm:2.24 1}}}{\implies}$ $f(z) = f(\gamma(T)) = f(z_0)$ für $|z-\gamma(T)| < R$
  \end{proof}
\end{theorem}
