% Henri Menke, 2012 Universität Stuttgart.
%
% Dieses Werk ist unter einer Creative Commons Lizenz vom Typ
% Namensnennung - Nicht-kommerziell - Weitergabe unter gleichen Bedingungen 3.0 Deutschland
% zugänglich. Um eine Kopie dieser Lizenz einzusehen, konsultieren Sie
% http://creativecommons.org/licenses/by-nc-sa/3.0/de/ oder wenden Sie sich
% brieflich an Creative Commons, 444 Castro Street, Suite 900, Mountain View,
% California, 94041, USA.

\section{Mannigfaltigkeiten} \index{Mannigfaltigkeit}
\addtocounter{thmn}{1}
\setcounter{theorem}{0}

% % % Vorlesung vom 07.01.2013

\begin{figure}[H]
  Fläche im $\mathbb{R}^3$
  
  \centering
  \begin{pspicture}(-0.3,-0.3)(7,3)
    \psaxes[labels=none,ticks=none]{->}(0,0)(-0.3,-0.3)(2,2)[\color{DimGray} $x_1$,0][\color{DimGray} $x_3$,0]
    \psline{->}(0,0)(1,1)\uput[-45](1,1){\color{DimGray} $x_2$}
    \pnode(0.5,1.1){A}
    \pnode(2.5,1.1){B}
    \pnode(1.5,2.5){C}
    \pnode(3.5,2.5){D}
    \ncarc{A}{B}
    \ncarc{A}{C}
    \ncarc{B}{D}
    \ncarc{C}{D}
    \uput{0.3}[30](A){\color{DimGray} $F$}
    \pnode(2.2,2.2){E}
    \psellipse[fillstyle=hlines,hatchcolor=DarkOrange3](E)(0.5,0.3)
    \pnode(2,2.1){X}
    \psdot*[linecolor=MidnightBlue](X)
    \uput[-135](X){\color{MidnightBlue} $x$}
    \uput{0.5}[110](2.2,2.2){\color{DarkOrange3} $U(x) = \varphi(K_1^{(2)}(0))$}
    
    \rput(5,1){
      \psaxes[labels=none,ticks=none]{->}(0,0)(-1.3,-1.3)(1.3,1.3)[\color{DimGray} $y_1$,0][\color{DimGray} $y_2$,0]
      \pscircle[fillstyle=hlines,hatchcolor=DarkOrange3](0,0){1}
      \uput{1.2}[45](0,0){\color{DarkOrange3} $K_1^{(2)}(0)$}
      \pnode(-0.3,0.5){S}
    }
    \ncarc[arrows=->,arcangle=-20]{S}{E}\naput{\color{DimGray} $\varphi$}
  \end{pspicture}
\end{figure}

\begin{theorem}[Definition] \label{thm:9.1}
  \begin{enum-arab}
    \item $S \subseteq \mathbb{R}^n$ heißt \acct{$k$-dimensionale differenzierbare Mannigfaltigkeit} in $\mathbb{R}^n$ ($1 \leq k \leq n$), falls zu jedem $x \in S$ eine offene Umgebung $U(x)$ auf $S$ (d.h. $U(x) = O \cap S$, $O \subseteq \mathbb{R}^n$ offen) existiert und eine Abbildung \[ \varphi_x : K_1^{(k)}(0) \to U(x) \subset O \qquad \left( K_1^{(k)}(0) \coloneq \{ y \in \mathbb{R}^k : |y| < 1 \} \right) \] mit
    %
    \begin{enum-alph}
      \item $\varphi_x$ bijektiv und $\varphi_x^{-1}$ stetig
      \item $\varphi_x$ stetig differenzierbar
      \item $\mathrm{Rang}(\tfrac{\partial \varphi_x}{\partial y}) = k$
    \end{enum-alph}
    %
    Das Tupel $(\varphi_x,U(x))$ nennt man \acct{Karte}.
    
    \item Eine Menge
    %
    \begin{align*}
      A(S) \coloneq \big\{ (\varphi_j,U_j) : 1 \leq j \leq N \big\} 
      \coloneq \Big\{ \big(\varphi_{x_j},U(x_j)\big) : x_j \in S, 1 \leq j \leq N \Big\}
    \end{align*}
    %
    mit
    %
    \begin{align*}
      S = \bigcup\limits_{j=1}^{N} U_j
    \end{align*}
    %
    heißt \acct{Atlas} von S.
    
    \item $S$ ist von der \acct{Klasse $m$} $\in N$ (wir schreiben dann: $S \in C^m$) falls ein Atlas $A(S)$ existiert, sodass $\varphi_j \in C^m$.
    
      Falls $S \in C^m$, betrachten wir nur solche Atlanten.

    \item $S \subseteq \mathbb{R}^n$ mit $S = \{ x_0,\ldots,x_N \}$ heißt \acct{$ 0 $-dimensionale Mannigfaltigkeit} \index{Mannigfaltigkeit!$0$-dimensional}.
  \end{enum-arab}
\end{theorem}

\begin{theorem}[Vereinbarung] \label{thm:9.2}
  Ab jetzt betrachten wir nur $C^m$ Mannigfaltigkeiten, die einen Atlas besitzen.
\end{theorem}

\begin{example} \label{thm:9.3}
  Sei $S = \{ x \in \mathbb{R}^3 : |x| = R \}$ eine Kugel mit Radius $r$ und seien folgende Karten $\varphi_k : K_1^{(2)}(0) \to S$ für $k \in \{1,2,\dotsc,6\}$ gegeben:
  %
  \begin{align*}
    \varphi_{1,2}(y_1,y_2) &\coloneq \left( y_1,y_2,\pm \sqrt{1-y_1^2-y_2^2} \right) \\
    U_{1,2} &= S \cap \{ x \in \mathbb{R}^3 : x_3 \gtrless 0 \} \\
    %
    \varphi_{3,4}(y_1,y_2) &\coloneq \left( y_1,\pm \sqrt{1-y_1^2-y_2^2},y_2 \right) \\
    U_{3,4} &= S \cap \{ x \in \mathbb{R}^3 : x_2 \gtrless 0 \} \\
    %
    \varphi_{5,6}(y_1,y_2) &\coloneq \left( \pm \sqrt{1-y_1^2-y_2^2},y_1,y_2 \right) \\
    U_{5,6} &= S \cap \{ x \in \mathbb{R}^3 : x_1 \gtrless 0 \}
  \end{align*}
  %
  Dann $A(S) = \{ (\varphi_j,U_j) , 1 \leq j \leq 6 \}$ ein Atlas. Also ist $S$ eine $2$-dimensionale Mannigfaltigkeit mit Atlas $A$.
\end{example}
  
\begin{figure}[H]
  Fläche im $\mathbb{R}^3$ mit Rand:
  
  \centering
  \begin{pspicture}(-0.3,0.2)(7,3.5)
    \pnode(0.5,1.1){A}
    \pnode(2.5,1.1){B}
    \pnode(1.5,2.5){C}
    \pnode(3.5,2.5){D}
    \ncarc{A}{B}
    \ncarc{A}{C}
    \ncarc{B}{D}
    \ncarc{C}{D}
    \uput{0.3}[30](A){\color{DimGray} $F$}
    \pnode(2.2,2.2){E}
    \psellipse[fillstyle=hlines,hatchcolor=DarkOrange3](E)(0.5,0.3)
    \uput{0.5}[110](E){\color{DarkOrange3} $U(x_1)$}
    \pnode(2,2.1){X}
    \psdot*[linecolor=MidnightBlue](X)
    \uput[-135](X){\color{MidnightBlue} $x_1$}
    
    \pnode(2.9,1.6){F}
    \pscustom[fillstyle=hlines,hatchcolor=DarkOrange3]{
      \psellipticarc(F)(0.5,0.3){78}{220}
    }
    \uput{0.2}[-60](F){\color{DarkOrange3} $U(x_2)$}
    \pnode(2.6,1.6){Y}
    \psdot*[linecolor=MidnightBlue](Y)
    \uput[-135](Y){\color{MidnightBlue} $x_2$}
    
    \rput(5,2.5){
      \psaxes[labels=none,ticks=none]{->}(0,0)(-0.7,-0.7)(0.7,0.7)
      \pscircle[fillstyle=hlines,hatchcolor=DarkOrange3](0,0){0.5}
      \pnode(-0.3,0.3){S}
    }
    \rput(5,0.5){
      \psaxes[labels=none,ticks=none]{->}(0,0)(-0.7,-0.2)(0.7,0.7)
      \pscustom[fillstyle=hlines,hatchcolor=DarkOrange3]{
        \psarc(0,0){0.5}{0}{180}
        \psline[linecolor=DarkOrange3](-0.5,0)(0.5,0)
      }
      \pnode(-0.3,0.3){T}
    }
    \pnode(2.6,1.7){Z}
    \ncarc[arrows=->,arcangle=-20]{S}{E}\nbput{\color{DimGray} $\varphi_{x_1}$}
    \ncarc[arrows=->,arcangle=-20]{T}{Z}\nbput{\color{DimGray} $\varphi_{x_2}$}
  \end{pspicture}
\end{figure}

\begin{theorem}[Definition] \label{thm:9.4}
  \begin{enum-arab}
    \item $S \subseteq \mathbb{R}^n$ heißt \acct[0]{$k$-dimensionale Mannigfaltigkeit mit Rand}\index{$k$-dimensionale Mannigfaltigkeit!mit Rand} ($1 \leq k \leq n$), wenn es zu jedem $x \in S$ eine offene Umgebung $U(x) = O \cap S$, $O \subseteq \mathbb{R}^n$ offen gibt und eine Abbildung  
    %
    \begin{multline*}
      \varphi_x : K_1^{(k)}(0) \to U(x) \text{ oder } \varphi_x : K_1^{(k)+}(0) \to U(x) \; , \\
      K_1^{(k)+}(0) \coloneq \{ y \in K_1^{(k)}(0) : y_k \geq 0 \}, k \geq 2
    \end{multline*}
    %
    bzw. im Fall $k=1$
    %
    \begin{align*}
      \varphi_x &: ]-1,1[ \to U(x) \text{ oder } \varphi_x : [0,1[ \to U(x) \text{ oder } \varphi_x : ]-1,0] \to U(x)
    \end{align*}
    %
    mit den Eigenschaften wie in \ref{thm:9.1} existiert. Die Begriffe >>Atlas<< und >>$C^m$<< werden analog definiert.
    
    \item $x \in S$ heißt \acct{Randpunkt} von $S$\footnote{Vorraussetzung: $\partial S \neq \emptyset$}, falls eine Karte $(\phi_x, U(x))$ existiert, sodass
    %
    \begin{align*}
      \varphi_x &:  K_1^{(k)+}(0) \to U(x)
    \intertext{ bzw.}
      \varphi_x &: \begin{matrix} [0,1[ \\ ]-1,0] \end{matrix} \to U(x)
    \intertext{und}
      \varphi_x^{-1}(x) &\in \{ x \in \mathbb{R}^k : x_k = 0 \} \quad (k \geq 2)
    \intertext{bzw.}
      \varphi_x^{-1}(x) &= 0 \quad (k = 1) .
    \end{align*}
    %
    Die Menge aller Randpunkte ist $\partial S$.
  \end{enum-arab}
\end{theorem}

\begin{notice} \label{thm:9.5}
  Die Festlegung $x \in \partial S$ hängt nicht von der Wahl der Karte ab. Sind $(\varphi_1,U_1)$, $(\varphi_2,U_2)$ zwei Karten mit $x \in U_1 \cap U_2$, so bildet
  %
  \begin{align*}
    \varphi_1^{-1} \circ \varphi_2 : K_1^{(k)+}(0) &\to K_1^{(k)+}(0) \\
    {[0,1[} &\to {]-1,0]} \\
    &\phantom{\to}\!\vdots
  \end{align*}
  %
  innere Punkte auf innere Punkte ab (Ohne Beweis). Also auch Randpunkte auf Randpunkte.
\end{notice}

\begin{theorem}[Satz] \label{thm:9.6}
  Ist $S \in C^m$ eine $k$-dimesionale Mannigfaltigkeit mit Rand im $\mathbb{R}^n$ ($k \ge 1$), so ist $\partial S$ eine $(k-1)$-dimensionale Mannigfaltigkeit der Klasse $C^m$ ohne Rand. Insbesondere ist $\partial(\partial S) = \emptyset$.
  %
  \begin{proof}
    \begin{enumerate}
      \item[$k=1$:] $x \in \partial S \iff x = \varphi_x(0)$. Da es nur endlich viele Karten gibt, bekommt man endlich viele Punkte in $\partial S$, also ergibt sich eine $0$-dimensionale Mannigfaltigkeit.
      
      \item[$k=2$:] Zu $x \in \partial S$ existiert eine Karte $(\varphi_x,U(x))$, $\varphi_x \in C^m$, $\varphi_x^{-1} \in C^m$. Sei
      %
      \begin{align*}
        \widetilde{\varphi}_x(y_1,\ldots,y_{k-1}) &\coloneq \varphi_x(y_1,\ldots,y_{k-1},0) \\
        \widetilde{U}(x) &\coloneq U(x) \cap \partial S = O \cap S \cap \partial S = O \cap \partial S \; , \quad O \subseteq \mathbb{R}^n \text{ offen.} \\
        \implies (\widetilde{\varphi}_x,\widetilde{U}(x)) \text{ ist Karte.}
      \end{align*}
      %
      endlich viele überdecken $\partial S$
      %
      \begin{align*}
        \widetilde{\varphi}_x \in C^m \; , \quad \widetilde{\varphi}_x^{-1} = \underbrace{\left(\varphi_x^{-1}\Big|_{\partial S}\right)}_{\mathclap{\text{erste $k-1$ Koordinaten}}} \in C^m
      \end{align*}
      %
      Alle $\widetilde{\varphi}_x$ sind auf $K^{(k-1)}(0)$ definiert, also kein Rand.
    \end{enumerate}
  \end{proof}
\end{theorem}

\begin{theorem}[Satz] \label{thm:9.7}
  Sei $k \geq 1$ und $S \in C^1$ eine $k$-dimensionale Mannigfaltigkeit in $\mathbb{R}^n$. Sei außerdem $x_0 \in S$ mit einer Karte $(\varphi_1,U_1)$, $x_0 \in U_1$. Dann heißt
  %
  \begin{align*}
    T_{x_0} \coloneq \mathrm{LH} \left\{ \frac{\partial \varphi_1}{\partial y_1} \left(\varphi_1^{-1}(x_0)\right) , \ldots , \frac{\partial \varphi_1}{\partial y_k} \left(\varphi_1^{-1}(x_0)\right) \right\}
  \end{align*}
  %
  \acct{Tangentialraum} in $x_0$. Es gilt
  %
  \begin{enum-alph}
    \item $\dim T_{x_0} = k$,
    
    \item $T_{x_0}$ ist unabhängig von der Karte.
  \end{enum-alph}
  %
  \begin{proof}
    \begin{enum-alph}
      \item
      \begin{align*}
        \mathrm{Id} &= \varphi_1^{-1} \circ \varphi_1 : K_1^{(k)}(0) \to K_1^{(k)}(0) : y \to y \\
        \implies E_k &=
        \underbrace{
        \begin{pmatrix}
          1 & 0 & \cdots & 0 \\
          0 & \ddots & & \vdots \\
          \vdots & & \ddots & 0 \\
          0 & \cdots & 0 & 1 \\
        \end{pmatrix}
        }_{\mathrm{Rang}=k}
        = \left( \frac{\partial \mathrm{Id}}{\partial y_1} \frac{\partial \mathrm{Id}}{\partial y_2} \ldots \frac{\partial \mathrm{Id}}{\partial y_k} \right)(y_0) \; , \quad y_0 = \varphi^{-1}(x_0) \\
        &\overset{\text{Kettenregel}}{=} \left( \frac{\partial \varphi_1^{-1}}{\partial x_1} \frac{\partial \varphi_1^{-1}}{\partial x_2} \ldots \frac{\partial \varphi_1^{-1}}{\partial x_n} \right)(x_0)
        \cdot
        \underbrace{\left( \frac{\partial \varphi_1}{\partial y_1} \frac{\partial \varphi_1}{\partial y_2} \ldots \frac{\partial \varphi_1}{\partial y_k} \right)(y_0)}_{\implies \mathrm{Rang} \leq k} \\
        \implies \dim T_{x_0} &= k
      \end{align*}
      
      \item ~
      \begin{figure}[H]
        \centering
        \begin{pspicture}(-1,-2)(6,2.5)
          \pnode(0.5,1.1){A}
          \pnode(4.5,1.1){B}
          \pnode(1.5,2.5){C}
          \pnode(5.5,2.5){D}
          \ncarc{A}{B}
          \ncarc{A}{C}
          \ncarc{B}{D}
          \ncarc{C}{D}
          \uput{0.3}[30](A){\color{DimGray} $F$}
          \pnode(2.2,2.0){E}
          \psellipse[linecolor=DarkGreen](E)(0.7,0.4)
          
          \pnode(3.2,2.0){F}
          \psellipse[linecolor=DarkRed](F)(0.7,0.4)
          
          \pnode(2.7,2.0){X}
          \psdot*[linecolor=MidnightBlue](X)
          \uput[0](X){\color{MidnightBlue} $x_0$}
          
          \rput(0,-1){
            \psaxes[labels=none,ticks=none]{->}(0,0)(-1.2,-1.2)(1.2,1.2)[\color{DimGray} $y_1$,0][\color{DimGray} $y_2$,180]
            \pscircle[fillstyle=hlines,hatchcolor=DarkOrange3](0,0){1}
            
            \pnode(0.4,0.2){S}
          }
          
          \rput(5,-1){
            \psaxes[labels=none,ticks=none]{->}(0,0)(-1.2,-1.2)(1.2,1.2)[\color{DimGray} $z_1$,0][\color{DimGray} $z_2$,0]
            \pscircle[fillstyle=hlines,hatchcolor=DarkOrange3](0,0){1}
            \pnode(-0.4,0.2){T}
          }
          \psdot*[linecolor=MidnightBlue](S)
          \psdot*[linecolor=MidnightBlue](T)
          \pscircle[linecolor=DarkGreen](S){0.5}
          \pscircle[linecolor=DarkRed](T){0.5}
          \uput{0.6}[30](S){\color{DarkGreen} $D_1$}
          \uput{0.6}[150](T){\color{DarkRed} $D_2$}
          \uput{0.1}[-90](S){\color{MidnightBlue} $y_0$}
          \uput{0.1}[-90](T){\color{MidnightBlue} $z_0$}
          \ncarc[arrows=->,arcangle=20]{S}{E}\nbput{\color{DimGray} $\varphi_1$}
          \ncarc[arrows=->,arcangle=-20]{T}{F}\naput{\color{DimGray} $\varphi_2$}
        \end{pspicture}
      \end{figure}
      %
      Seien $(\varphi_j,U_j)$, $j=1,2$ zwei Karten mit $x_0 \in U_1 \cap U_2$. Setze $D_j \coloneq \varphi_j^{-1}(U_1 \cap U_2)$. Dann ist $\varphi_1^{-1} \circ \varphi_2 : D_2 \to D_1$ bijektiv.
      %
      \begin{align*}
        \frac{\partial \varphi_2}{\partial z_i}(z_0) &= \frac{\partial}{\partial z_i} \left( \varphi_1 \circ \left( \varphi_1^{-1} \circ \varphi_2 \right) \right)(z_0) \\
        &= \sum\limits_{j=1}^{k} \frac{\partial \varphi_1}{\partial y_j} \underbrace{\left( \frac{\partial \left( \varphi_1^{-1} \circ \varphi_2 \right)}{\partial z}(z_0) \right)_{\text{$j$-te Koord.}}}_{:= \lambda_j} \\
        &= \sum_{j=1}^k \lambda_j \frac{\partial \varphi_1}{\partial y_j} \in T_{x_0} \\
        \implies& \mathrm{LH} \left\{ \frac{\partial \varphi_2}{\partial z_1} (z_0) , \ldots , \frac{\partial \varphi_2}{\partial z_k} (z_0) \right\} \subset T_{x_0}
      \end{align*}
      Betrachtet man $\varphi_2^{-1} \circ \varphi_1$, ergibt sich die andere Richtung und damit $\mathrm{LH} \{ \ldots \} = T_{x_0}$
    \end{enum-alph}
  \end{proof}
\end{theorem}

\begin{theorem}[Orientierung von Karten] \label{thm:9.8}
  \begin{enum-arab}
    \item Zwei Basen $\{b_1,\ldots,b_k\}$, $\{c_1,\ldots,c_k\}$ eines $k$-dimensionalen Raumes heißen \acct[0]{gleich orientiert}\index{Karte!gleich orientiert}, falls
    %
    \begin{align*}
      b_j = \sum\limits_{i=1}^{k} \alpha_{ji} c_i \quad\land\quad \det(\alpha_{ji}) > 0
    \end{align*}
    
    \item Sei $S \in C^1$. Zwei Karten $(\varphi_1,U_1)$ heißen gleich orientiert, falls
    %
    \begin{align*}
      \det \left( \frac{\partial \left( \varphi_1^{-1} \circ \varphi_2 \right)_j}{\partial z}(z_0) \right) > 0 \text{ für } z_0 \in D_2 \text{(vgl. vorher)}
    \end{align*}
    %
    (Vorzeichen der Determinante ist auf $D_2$ konstant, da stetig und immer $\neq 0$)
  \end{enum-arab}
\end{theorem}

% % % Vorlesung vom 10.01.2013

\paragraph{Korrektur} zu \ref{thm:9.1}, Punkt 3: $S \in C^m$, $m \geq 1$, falls $\varphi_j \in C^m$ $(j = 1,\ldots,N)$ und 

Damit wird in \ref{thm:9.7} $\dim T_{x_0} = k$ offensichtlich, da \[ T_{x_0} = \mathrm{LH}\left\{ \frac{\partial \varphi}{\partial y_1} \left( \varphi^{-1}(x_0) \right),\ldots,\frac{\partial \varphi}{\partial y_k} \left( \varphi^{-1}(x_0) \right) \right\} \]

\begin{theorem}[Orientierung von Mannigfaltigkeiten] \label{thm:9.9}
  \begin{enum-arab}
    \item Zwei Karten $(\varphi_1,U_1)$, $(\varphi_2,U_2)$ heißen \acct[0]{kompatibel} \index{Karte!Kompatibilität}, falls $U_1 \cap U_2 = \emptyset$ oder falls sie gleich orientiert sind.
    
    \item Ein Atlas heißt \acct[0]{orientiert} \index{Atlas!orientiert}, wenn alle seine Karten paarweise kompatibel sind.
    
    \item Eine Mannigfaltigkeit $S \in C^1$ heißt \acct[0]{orientierbar} \index{Mannigfaltigkeit!orientierbar}, falls sie mindestens einen orientierten Atlas besitzt.
    
    \item Zwei orientierte Atlanten $A(S)$, $\widetilde{A}(S)$ heißen \acct[0]{kompatibel} \index{Atlas!kompatibel}, falls $A(S) \cup \widetilde{A}(S)$ orientiert ist. Dies definiert eine Äquivalenzrelation auf der Menge aller orientierten Atlanten von $S$. Jede Äquivalenzklasse heißt \acct[0]{Orientierung} \index{Mannigfaltigkeit!Orientierung} von $S$.
  \end{enum-arab}
\end{theorem}

\begin{example} \label{thm:9.10}
  \begin{enum-arab}
    \item ~
    %
    \begin{align*}
      S &= \left\{ t \left(\begin{smallmatrix} 1 \\ 1 \\ 1 \end{smallmatrix}\right) : 0 \leq t \leq 3 \right\} \\
      \varphi_1(y) &\coloneq \frac{3}{2} (y+1) \begin{pmatrix} 1 \\ 1 \\ 1 \end{pmatrix} \; , \quad y \in K_1^{(1)}(0) = ]-1,1[ \; , \; U_1 = \left\{ t \left(\begin{smallmatrix} 1 \\ 1 \\ 1 \end{smallmatrix}\right) : 0 < t < 3 \right\} \\
      \varphi_2(y) &\coloneq z \begin{pmatrix} 1 \\ 1 \\ 1 \end{pmatrix} \; , \quad z \in [0,1[ \; , \; U_2 = \left\{ t \left(\begin{smallmatrix} 1 \\ 1 \\ 1 \end{smallmatrix}\right) : 0 \leq t < 1 \right\}
    \end{align*}
    %
    Kompatibilität? $U_1 \cap U_2 \neq \emptyset$
    %
    \begin{align*}
      \varphi_1^{-1} \circ \varphi_2(z) &= \frac{2}{3} z - 1 \\
      \frac{\mathrm{d}}{\mathrm{d}z} \left( \varphi_1^{-1} \circ \varphi_2 \right) &= \frac{2}{3} > 0
    \end{align*}
    %
    $\implies$ gleich orientiert.
    %
    \begin{align*}
      \varphi_3(z) &\coloneq (3-z) \begin{pmatrix} 1 \\ 1 \\ 1 \end{pmatrix} \; , \quad z \in [0,1[ , U_3 = \left\{ t \left(\begin{smallmatrix} 1 \\ 1 \\ 1 \end{smallmatrix}\right) : 2 < t \leq 3 \right\}
    \end{align*}
    %
    Kompatibilität? $U_3 \cap U_2 = \emptyset \quad \checkmark$
    %
    \begin{align*}
      U_3 \cap U_1 &\neq \emptyset \\
      \varphi_1^{-1} \circ \varphi_3(z) &= \frac{2}{3} (3-z) - 1 \\
      \frac{\mathrm{d}}{\mathrm{d}z} \left( \varphi_1^{-1} \circ \varphi_3 \right) &= -\frac{2}{3} < 0
    \end{align*}
    %
    $\implies$ nicht gleich orientiert. Stattdessen:
    %
    \begin{align*}
      \varphi_3(z) &\coloneq (3+z) \begin{pmatrix} 1 \\ 1 \\ 1 \end{pmatrix} \; , \quad z \in ]-1,0] , U_3 = \left\{ t \left(\begin{smallmatrix} 1 \\ 1 \\ 1 \end{smallmatrix}\right) : 2 < t \leq 3 \right\}
    \end{align*}
    %
    \begin{itemize}
      \item[$\implies$] $(\varphi_1,U_1)$, $(\varphi_3,U_3)$ gleich orientiert.
      
      \item[$\implies$] $A(S) = \{ (\varphi_j,U_j), j=1,2,3 \}$ ist orientierbarer Atlas von $S$.
    \end{itemize}
    
    \item ~
    %
    \begin{align*}
      S &= \left\{ \left(\begin{smallmatrix} x_1 \\ x_2 \\ x_3 \end{smallmatrix}\right) : x_1^2 + x_2^2 = 1 \land 0 \leq x_3 \leq 1 \right\}
    \end{align*}
    %
    \begin{figure}[H]
      \centering
      \psset{Alpha=60}
      \begin{pspicture}(-2,-1.5)(3,2.5)
        \pstThreeDCoor[linecolor=DimGray,xMax=2.5,yMax=3,zMax=2.5,xMin=-0.5,yMin=-0.5,zMin=-0.5,nameX=\color{DimGray}$x_1$,nameY=\color{DimGray}$x_2$,nameZ=\color{DimGray}$x_3$]
        \pstThreeDEllipse[linecolor=MidnightBlue,beginAngle=30,endAngle=390,arrows=->](0,0,0)(1,0,0)(0,1,0)
        \parametricplotThreeD[linecolor=MidnightBlue,plotstyle=curve,yPlotpoints=20](0,1)(0,\psPiTwo){cos(u) | sin(u) | t}
        \pstThreeDEllipse[linecolor=MidnightBlue,beginAngle=30,endAngle=390,arrows=<-](0,0,1)(1,0,0)(0,1,0)
        \psellipse[linecolor=Purple,linestyle=dotted,dotsep=1pt](0,0)(0.3,0.7)
        \uput{1.6}[45](0,0){\color{MidnightBlue} $S$}
        \uput{0.8}[-60](0,0){\color{Purple} induzierte Orientierung}
      \end{pspicture}
    \end{figure}
    %
    $\partial S$ besteht aus zwei Kreisen. $S$ ist orientierbar (ohne Beweis). Durch die Orientierung von $S$ wird eine Orientierung von $\partial S$ induziert.
    
    \item Möbius-Band: Nicht orientierbare Mannigfaltigkeit.
  \end{enum-arab}
\end{example}

\begin{theorem}[Definition] \label{thm:9.11}
  Sei $S \in C^1$ orientierbare Mannigfaltigkeit mit Rand der Dimension $k \geq 2$, $A(S)$ orientierter Atlas. Durch die Konstruktion von \ref{thm:9.6} wird ein orientierter Atlas $A(\partial S)$ gegeben. Die dadurch definierte Orientierung von $\partial S$ heißt \acct[0]{verträglich} \index{Atlas!verträglich orientiert} mit der Orientierung von $S$.
  %
  \begin{figure}[H]
    \centering
    \begin{pspicture}(-1.5,-2)(6,2.3)
      \uput[200](0,2){\color{DimGray} $S$}
      \pscustom[fillstyle=vlines,hatchcolor=DimGray,hatchsep=10pt,linestyle=none]{
        \pscurve(0,0)(1.5,1)(2.5,1.5)(4,2)
        \psline(4,2)(0,2)(0,0)
      }
      \pscustom[fillstyle=hlines,hatchcolor=DarkOrange3]{
        \psarc(1.5,1){0.7}{28}{212}
      }
      \pnode(1.3,1.2){A1}
      \uput{0.8}[135](1.5,1){\color{DarkOrange3} $U_2$}
      \pscustom[fillstyle=vlines,hatchcolor=DarkOrange3]{
        \psarc(2.5,1.5){0.7}{20}{205}
      }
      \pnode(2.8,1.8){B1}
      \uput{0.8}[45](2.5,1.5){\color{DarkOrange3} $U_1$}
      \pscustom[fillstyle=solid,linestyle=none]{
        \pscurve(0,0)(1.5,1)(2.5,1.5)(4,2)
        \psline(4,2)(4,0)(0,0)
      }
      \pscurve[linecolor=MidnightBlue](0,0)(1.5,1)(2.5,1.5)(4,2)
      \uput[0](4,2){\color{MidnightBlue} $\partial S$}
      \psdot(2,1.25)
      \uput[-45](2,1.25){\color{DimGray} $x_0$}
      
      \rput(0.5,-1.5){
        \psaxes[labels=none,ticks=none]{->}(0,0)(-1,-0.3)(1,1)[\color{DimGray},0][\color{DimGray} $z_k$,180]
        \pscustom[fillstyle=hlines,hatchcolor=DarkOrange3]{
          \psarc(0,0){0.7}{0}{180}
        }
        \psline[linecolor=DarkOrange3](0.7;0)(0.7;180)
        \pnode(0.4;45){A2}
        \psdot(0.35;0)
        \uput[-60](0.35;0){\color{DimGray} $\varphi_2^{-1}(x_0)$}
        \uput{0.8}[150](0,0){\color{DimGray} $K_1^{(k)+}(0)$}
      }
      
      \rput(4.5,-1.5){
        \psaxes[labels=none,ticks=none]{->}(0,0)(-1,-0.3)(1,1)[\color{DimGray},0][\color{DimGray} $y_k$,0]
        \pscustom[fillstyle=hlines,hatchcolor=DarkOrange3]{
          \psarc(0,0){0.7}{0}{180}
        }
        \psline[linecolor=DarkOrange3](0.7;0)(0.7;180)
        \pnode(0.4;135){B2}
        \psdot(0.35;180)
        \uput[-120](0.35;180){\color{DimGray} $\varphi_1^{-1}(x_0)$}
        \uput{0.8}[30](0,0){\color{DimGray} $K_1^{(k)+}(0)$}
      }
      
      \ncarc[arrows=->,arcangle=20]{A2}{A1}\nbput{\color{DimGray} $\varphi_2$}
      \ncarc[arrows=->,arcangle=-20]{B2}{B1}\nbput{\color{DimGray} $\varphi_1$}
    \end{pspicture}
  \end{figure}
  %
  Wir wissen: $x_0 \in \partial S$, $x_0 \in U_1 \cap U_2$, $(\varphi_1,U_1),(\varphi_2,U_2) \in A(S)$ orientiert
  \[ \implies \det \left( \frac{\partial (\varphi_1^{-1} \circ \varphi_2)}{\partial z} \right) > 0 \]
  Sei $\psi \coloneq \varphi_1^{-1} \circ \varphi_2 . \varphi_2^{-1}(U_1 \cap U_2) \to \varphi_1^{-1}(U_1 \cap U_2)$. $\psi$ bildet Randpunkte auf Randpunkte ab.
  \[ \psi_k(z_1,\ldots,z_{k-1},0) = 0 \quad \implies \quad \frac{\partial}{\partial z_j} \psi_k(\varphi_2^{-1}(x_0)) = 0 \; , \quad j=1,\ldots,k-1 \]
  $\psi$ bildet innere Punkte auf innere Punkte ab.
  \[ \psi_k(z_1,\ldots,z_{k-1},z_k>0) > 0 \quad \implies \quad \frac{\partial}{\partial z_k} \psi_k(\varphi_2^{-1}(x_0)) > 0 \]
  %
  \begin{align*}
    \implies 0 < \det\left( \frac{\partial \psi}{\partial z} \right)= \det
    \begin{pmatrix}
    \ldots & \ldots & \ldots & \ldots \\
    \ldots & \ldots & \ldots & \ldots \\
    \frac{\partial \psi_k}{\partial z_1} & \ldots & \frac{\partial \psi_k}{\partial z_{k-1}} & \frac{\partial \psi_k}{\partial z_k}
    \end{pmatrix}
  \end{align*}
  %
  Nach dem Entwicklungssatz nach der $k$-ten Zeile:
  %
  \begin{align*}
    = (-1)^{2k} \frac{\partial \psi_k}{\partial z_k} \left(\varphi_2^{-1}(x_0)\right) \det\left(\frac{\partial (\psi_1,\ldots,\psi_k)}{\partial (z_1,\ldots,z_k)}\right) \left(\varphi^{-1}(x_0)\right)
  \end{align*}
  %
  Zugehörige Karten $(\widetilde{\varphi}_j,\widetilde{U}_j)$ des Randes $\partial S$:
  %
  \begin{align*}
    \widetilde{\varphi}_j(y_1,\ldots,y_{k-1}) \coloneq \varphi_j(y_1,\ldots,y_{k-1},0) \; , \quad \widetilde{U}_j = U_j \cap \partial S
  \end{align*}
  %
  Aus \ref{thm:9.6}: $\{ (\widetilde{\varphi}_1,\widetilde{U}_1),\ldots,(\widetilde{\varphi}_n,\widetilde{U}_n) \}$ ist Atlas von $S$. Es gilt \[(\widetilde{\varphi}_1^{-1} \circ \widetilde{\varphi}_2)_i(z_1,\ldots,z_{k-1}) = \underbrace{(\varphi_1^{-1} \circ \varphi_2)_i}_{\psi_i}(z_1,\ldots,z_{k-1},0) \; , \quad i = 1,\ldots,k-1.\]
  %
  \begin{align*}
    \implies \det \left( \frac{\partial (\widetilde{\varphi}_1^{-1} \circ \widetilde{\varphi}_2)}{\partial (z_1,\ldots,z_{k-1})} \right) = \det \left( \frac{\partial (\psi_1,\ldots,\psi_{k-1})}{\partial (z_1,\ldots,z_{k-1})} \right) > 0
  \end{align*}
  %
  $\implies$ $\{ (\widetilde{\varphi}_1,\widetilde{U}_1),\ldots,(\widetilde{\varphi}_N,\widetilde{U}_N) \}$ ist orientierter Atlas von $\partial S$.
\end{theorem}

\begin{example}
  $S = \overline{K_1^{(2)}(0)} \subseteq \mathbb{R}^2$, $\partial S = \left\{ \left(\begin{smallmatrix} x_1 \\ x_2 \end{smallmatrix}\right) : x_1^2 + x_2^2 = 1 \right\}$
  %
  \begin{enum-arab}
    \item Sei $(\varphi,U) : \varphi\left(\begin{smallmatrix} x_1 \\ x_2 \end{smallmatrix}\right) = \left(\begin{smallmatrix} x_1 \\ x_2 \end{smallmatrix}\right)$, $U = K_1^{(2)}(0)$. Annahme: $A(S)$ ist orientierter Atlas der $(\varphi,U)$ enthält.
    %
    \begin{figure}[H]
      \centering
      \begin{pspicture}(-1,-2)(4,1.5)
        \pscircle[fillstyle=hlines,hatchcolor=DimGray](0,0){1}
        \pscustom[fillstyle=vlines,hatchcolor=DarkOrange3]{
          \psarc(1;-45){0.5}{60}{210}
          \psarc(0,0){1}{-60}{-15}
        }
        \uput*{0.5}[135](0,0){\color{DimGray} $S$}
        \psdot(1;-50)
        \uput[-90](1;-50){\color{DimGray} $x_0$}
        \uput[0](1;-15){\color{DarkOrange3} $U_1$}
        \pnode(0.8;-30){A1}
        \pnode(0.3;50){B1}
        
        \rput(2.5,-1.5){
          \psaxes[labels=none,ticks=none]{->}(0,0)(-1,-0.3)(1,1)[\color{DimGray},0][\color{DimGray} $z_2$,0]
          \pscustom[fillstyle=hlines,hatchcolor=DarkOrange3]{
            \psarc(0,0){0.7}{0}{180}
          }
          \psline[linecolor=DarkOrange3](0.7;0)(0.7;180)
          \pnode(0.4;135){A2}
          \psdot(0.35;180)
          \uput[-120](0.35;180){\color{DimGray} $\varphi_1^{-1}(x_0)$}
        }
        
        \pscircle[fillstyle=hlines,hatchcolor=DimGray](3,1){0.5}
        \pnode(2.8,1.2){B2}
        
        \ncarc[arrows=->,arcangle=-20]{A2}{A1}\naput{\color{DimGray} $\varphi_1$}
        \ncarc[arrows=->,arcangle=-20]{B2}{B1}\nbput{\color{DimGray} $\varphi$}
      \end{pspicture}
    \end{figure}
    %
    $\implies$ verträgliche Orientierung von $\partial S$ im Gegenuhrzeigersinn.
    
    Falls $A(S)$ die Karte $(\widetilde{\varphi},\widetilde{U})$ mit $\widetilde{\varphi}\left(\begin{smallmatrix} x_1 \\ x_2 \end{smallmatrix}\right) = \left(\begin{smallmatrix} -x_1 \\ x_2 \end{smallmatrix}\right)$ enthält, ist die verträgliche Orientierung von $\partial S$ im Uhrzeigersinn.
  \end{enum-arab}
\end{example}
