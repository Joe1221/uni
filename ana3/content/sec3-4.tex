% Stephan Hilb, 2012 Universität Stuttgart.
%
% Dieses Werk ist unter einer Creative Commons Lizenz vom Typ
% Namensnennung - Nicht-kommerziell - Weitergabe unter gleichen Bedingungen 3.0 Deutschland
% zugänglich. Um eine Kopie dieser Lizenz einzusehen, konsultieren Sie
% http://creativecommons.org/licenses/by-nc-sa/3.0/de/ oder wenden Sie sich
% brieflich an Creative Commons, 444 Castro Street, Suite 900, Mountain View,
% California, 94041, USA.

\section{Lineare Differentialgleichungen}
\addtocounter{thmn}{1}
\setcounter{theorem}{0}


\begin{theorem}[Definition] \label{4.1}
  Sei $I \subset \mathbb{R}$ ein Intervall, $A \in C(I \to \mathbb{R}^{n^2})$, $g \in C(I \to \mathbb{R}^n)$.
  Die Differentialgleichung
  %
  \begin{align*}
    y' = \underbrace{A(x)y + g(x)}_{=R(y)}
  \end{align*}
  %
  für die Unbekannte $y \in C^1(I \to \mathbb{R})$ heißt \emph{lineares System 1. Ordnung}.
  Für $g = 0$ nennen wir es \emph{homogen}, sonst \emph{inhomogen}.
\end{theorem}

\begin{notice} \label{4.2}
  Anwendung von Picard-Lindelöf auf obige Differentialgleichung mit $y(x_0) = y_0$ (für vorgegebenes $(x_0,y_0) \in I \times \mathbb{R}^n$), $B := \mathbb{R}^n$, $D = \mathbb{R}^n$ und Lipschitzbedingung
  %
  \begin{align*}
    \|f(x,y) - f(x,\tilde y) \| = \| A(x) (y-\tilde y) \| \le c \|y - \tilde y\|
  \end{align*}
  %
  für $x \in K \subset I$ mit $K$ kompakt.

  Der Satz liefert also eine lokale Lösung $\phi := y \in C^1([x_0 - r, x_0 + r] \to \mathbb{R}^n)$.
  Erneute Anwendung mit Anfangsbedingung $y(x_0 + r) = \phi(x_0 + r)$ liefert die Lösung
  %
  \begin{align*}
    \psi := y \in C^1([x_0 + r - \tilde r, x_0 + r + \tilde r] \to \mathbb{R}^n)
  \end{align*}
  %
  Zusammenkleben der Lösungen liefert
  %
  \begin{align*}
    y(x) = \begin{cases}
      \phi(x) & x_0-r \le x \le x_0 + r \\
      \psi(x) & x_0+r \le x \le x_0 + r + \tilde r
    \end{cases}
  \end{align*}
  %
  Dieses $y$ ist stetig auf $[x_0-r, x_0 + r + \tilde r]$.
  Weiter ist $y'$ stetig auf dem selben Intervall, da
  %
  \begin{align*}
    \phi'(x_0 + r) &= A(x_0 + r) \phi(x_0 + r) + g(x_0 + r) \\
    &= A(x_0+r) \psi(x_0+r) + g(x_0 +r) \\
    &= \psi'(x_0 +r)
  \end{align*}
  %
  Als ist dieses $y \in C^1([x_0-r, x_0+r+r'] \to \mathbb{R}^n)$ eine Lösung von
  %
  \begin{align*}
    y' = A(x)y + g(x)
    \qquad \land \qquad
    y(x_0) = y_0
  \end{align*}
  %
  Die Fortsetzung ist so lange möglich, bis $y \in C^1(I \to \mathbb{R}^n)$ (ohne Beweis).
\end{notice}

\begin{theorem}[Satz] \label{4.3}
  \begin{enum-arab}
  \item
    Homogener Fall ($g = 0$):
    \begin{enum-alph}
    \item
      Die Menge der Lösungen
      %
      \begin{align*}
        \mathcal L_{\text{hom}} = \{y \in C^1(I \to \mathbb{R}^n : y' = A(x)y)
      \end{align*}
      %
      bildet einen Untervektorraum von $C^1(I \to \mathbb{R}^n)$ der Dimension $n$.
      Für festes $x_0 \in I$ ist
      %
      \begin{align*}
        P_{x_0} : \mathcal L_{\text{hom}} \to \mathbb{R}^n : y \mapsto P_{x_0} y := y(x_0)
      \end{align*}
      %
      ein Vektorraum-Isomorphismus.
      Eine Basis von $\mathcal L_{\text{hom}}$ heißt \emph{Fundamentalsystem} (auch im Fall $g \neq 0$).
    \item
      Für $n$ Lösungen $y_1, \dotsc, y_n \in \mathcal L_{\text{hom}}$ sind äquivalent:
      \begin{enum-roman}
      \item
        $\{y_1,\dotsc,y_n\}$ ist ein Fundamentalsystem
      \item
        $\forall x \in I : W(x) := \det(y_1(x) \dotso y_n(x)) \neq 0$ (Wronski-Determinante)
      \item
        $\exists x_0 \in I : W(x_0) \neq 0$.
      \end{enum-roman}
    \end{enum-alph}
  \item
    Inhomogener Fall ($g \neq 0$):
    \begin{enum-alph}
    \item
      Die Menge der Lösungen $\mathcal L_{\text{inhom}}$ bildet einen affinen Unterraum von $C^1(I \to \mathbb{R}^n)$
      %
      \begin{align*}
        \mathcal L_{\text{inhom}} = \{y \in C^1(I \to \mathbb{R}^n: y' = A(x)y + g(x))
      \end{align*}
      %

      Ist $y_{\text{part}} \in \mathcal L_{\text{inhom}}$ eine beliebige, aber fest gewählt (\emph{partikuläre Lösung}), so gilt
      %
      \begin{align*}
        \mathcal L_{\text{inhom}} = y_{\text{part}} + \mathcal L_{\text{hom}}
      \end{align*}
      %
    \item
      Nutze \emph{Variaton der Konstanten}:
      Ist $\{y_1,\dotsc, y_n\}$ ein Fundamentalsystem, so ist
      %
      \begin{align*}
        y(x) = c_1(x) y_1(x) + \dotsb + c_n(x) y_n(x)
      \end{align*}
      %
      genau dann Lösung von $y' = A(x)y + g(x)$, wenn
      %
      \begin{align*}
        c_1'y_1 + \dotsb + c_n'y_n = g
        \qquad &\iff \quad
        \underbrace{\begin{pmatrix}
            y_1 & \cdots & y_n
          \end{pmatrix}}_{\det(\dotsc) = W(x) \neq 0} \begin{pmatrix}
          c_1' \\ \vdots \\ c_n'
        \end{pmatrix} = g \\
        & \iff \qquad \begin{pmatrix}
          c_1' \\ \vdots \\ c_n'
        \end{pmatrix} = \begin{pmatrix}
          y_1 & \cdots & y_n
        \end{pmatrix}^{-1} g
      \end{align*}
      %
    \end{enum-alph}
  \end{enum-arab}
  \begin{proof}
    \begin{enum-arab}
    \item
      \begin{enum-alph}
      \item
        Nach \ref{4.1} ist $\mathcal L_{\text{hom}} \neq \emptyset$.
        Seien $y, \tilde y$ Lösungen von $y' = A(x)y$ und $\alpha, \beta \in \mathbb{R}$, dann ist für $u = \alpha y + \beta \tilde y$
        %
        \begin{align*}
          u' = \alpha y' + \beta \tilde y' = \alpha Ay' + \beta A \tilde y' = Au
        \end{align*}
        %
        und damit $u \in \mathcal L_{\text{hom}}$.
        Damit ist $\mathcal L_{\text{hom}}$ Untervektorraum.
        Die Abbildung $P_{x_0}$ ist linear und wohldefiniert, da die Lösung der DGL eindeutig.
      \item
        $P_{x_0}$ ist ein Vektorraum-Isomorphismus, da
        %
        \begin{align*}
          \{y_1, \dotsc, y_n\} \text{ lin. unabhängig}
          & \iff \{P_{x_0} y_1, \dotsc, P_{x_0} y_n \} \text{ lin. unabhängig} \\
          & \iff \det(y_1(x_0) \dotso y_n(x_0)) = W(x_0) \neq 0
        \end{align*}
        %
      \end{enum-alph}
      % FIXME: zu zeigende Äquivalenzen verdeutlichen
    \item
      \begin{enum-alph}

      \item
        Zeige $y_{\text{part}} + \mathcal L_{\text{hom}} \subset \mathcal L_{\text{inhom}}$.
        Sei $y \in \mathcal L_{\text{hom}}, u := y + y_{\text{part}}$, dann ist
        %
        \begin{align*}
          u' = y' + y_{\text{part}}' = Ay + Ay_{\text{part}} + g = Au + g
        \end{align*}
        %
        also $u \in \mathcal L_{\text{inhom}}$.

        Zeige $\mathcal L_{\text{inhom}} \subset y_{\text{part}} + \mathcal L_{\text{hom}}$.
        Sei $u \in \mathcal L_{\text{inhom}}, y := u - y_{\text{part}}$, dann ist
        %
        \begin{align*}
          y' = Au + g - (A y_{\text{part}} + g) = Ay
        \end{align*}
        %
        also $y \in \mathcal L_{\text{hom}}$ und damit $u = y_{\text{part}} + y \in y_{\text{part}} + \mathcal L_{\text{hom}}$.
      \item
        Setze $y(x) = c_1(x) y_1(x) + \dotsb + c_n(x) y_n(x)$ in die DGL ein:
        %
        \begin{align*}
          y' = \sum_{i=1}^n (c_j' y_j + c_j y_j') &\stackrel != A(x) \bigg( \sum_{j=1}^n c_j y_j \bigg) + g \\
          \iff \qquad \sum_{j=1}^n c_j'y_j + \sum_{j=1}^n c_j y_j' &= \sum_{j=1}^n c_j \underbrace{A(x) y_j}_{y_j'} + g \\
          \iff \sum_{j=1}^n c_j' y_j' &= g
        \end{align*}
        %
      \end{enum-alph}
    \end{enum-arab}
  \end{proof}
\end{theorem}

\begin{theorem}[Definition] \label{4.4}
  Sei $I \subset \mathbb{R}$ ein Intervall und $a_0, \dotsc, a_{n-1}, g \in C(I \to \mathbb{R})$.
  Dann heißt
  %
  \begin{align*}
    y^{(n)} + a_{n-1} y^{(n-1)} + \dotsb + a_1 y' + a_0 y = g
  \end{align*}
  %
  für die Unbekannte $y \in C^n(I \to \mathbb{R})$ \emph{lineare Differentialgleichung $n$-ter Ordnung}.
  Für $g = 0$ nennen wir sie \emph{homogen}, sonst \emph{inhomogen}.
\end{theorem}

\begin{notice} \label{4.5}
  \begin{enum-arab}
  \item
    Sei $y \in C^n(I \to \mathbb{R})$ eine Lösung von obiger DGL, $u_1 := y, u_2 := y', \dotsc, u_n := y^{(n-1)}$, dann ist
    %
    \begin{align*}
      u' =
      \begin{pmatrix}
        y' \\ y'' \\ \vdots \\ y^{(n-1)} \\ y^{(n)}
      \end{pmatrix}
      =
      \begin{pmatrix}
        u_2 \\ u_3 \\ \vdots \\ u_n \\ -a_{n-1} u_n - a_{n-2} u_{n-1} - \dotsb - a_0 u_1 + g
      \end{pmatrix}
      = \begin{pmatrix}
        0 & 1 & \cdots & 0 \\
        \vdots  & \ddots & \ddots & 0 \\
        0 & 0 & 0 & 1 \\
        -a_0 & -a_1 & \cdots & -a_{n-1}
      \end{pmatrix} + \begin{pmatrix}
        0 \\ \vdots \\ 0 \\ g
      \end{pmatrix}
    \end{align*}
    %
  \item
    Ist umgekehrt $u \in C^1(I \to \mathbb{R}^n)$ Lösung von obigem System und $y := u_1$, dann löst $y \in C^n(I \to \mathbb{R})$ die ursprünglich lineare DGL $n$-ter Ordnung.
  \end{enum-arab}
\end{notice}

\begin{theorem}[Korollar: Folgerung] \label{4.6}
  \begin{enum-arab}
  \item
    Im Fall $g = 0$:
    \begin{enum-alph}
    \item
      Die Menge der Lösungen
      %
      \begin{align*}
        \mathcal L_{\text{hom}} = \{y \in C^n(I \to \mathbb{R}) : \dotso \}
      \end{align*}
      %
      bildet einen linearen Unterraumvektor von $C^n(I \to \mathbb{R})$ der Dimension $n$.
      Die Abbildung
      %
      \begin{align*}
        P_{x_0} : \mathcal L_{\text{hom}} \to \mathbb{R}^n : y \mapsto P_{x_0}(y) := \begin{pmatrix}
          y(x_0) \\ y'(x_1) \\ \vdots \\ y^{(n-1)}(x_0)
        \end{pmatrix}
      \end{align*}
      %
      ist ein Vektorraum-Isomorphismus.
      Eine Basis von $\mathcal L_{\text{hom}}$ heißt \emph{Fundamentalsystem}.
    \item
      Für $y_1, \dotsc, y_n \in \mathcal L_{\text{hom}}$ sind äquivalent
      \begin{enum-roman}
      \item
        $\{y_1,\dotsc,y_n\}$ ist Fundamentalsystem
      \item
        %
        \begin{align*}
          \forall x \in I : W(x) = \det \begin{pmatrix}
            y_1(x) & \cdots & y_n(x) \\
            \vdots & \ddots & \vdots \\
            y_1^{(n-1)}(x) & \cdots & y_n^{(n-1)}(x)
          \end{pmatrix} \neq 0
        \end{align*}
        %
        (Wronski-Determinante)
      \item
        $\exists x_0 \in I : W(x) \neq 0$
      \end{enum-roman}
    \end{enum-alph}
  \item
    Im inhomogenen Fall ($g \neq 0$):

    \begin{enum-alph}
    \item
      %
      \begin{align*}
        \mathcal L_{\text{inhom}} = y_{\text{part}} + \mathcal L_{\text{hom}}
      \end{align*}
      %
      wobei $y_{\text{part}} \in \mathcal L_{\text{inhom}}$ beliebig aber fest gewählt.
    \item
      Variation der Konstanten: Ist $\{y_1, \dotsc, y_n\}$ ein Fundamentalsystem, so ist
      %
      \begin{align*}
        y(x) := c_1(x) y_1(x) + \dotsb + c_n(x)y_n(x)
      \end{align*}
      %
      genau dann Lösung der DGL, falls
      %
      \begin{align*}
        c_1' \begin{pmatrix}
          y_1 \\ y_1' \\ \vdots \\ y_1^{(n-1)}
        \end{pmatrix} +
        c_2' \begin{pmatrix}
          y_1 \\ y_1' \\ \vdots \\ y_1^{(n-1)}
        \end{pmatrix} + \dotsb +
        c_n' \begin{pmatrix}
          y_1 \\ y_1' \\ \vdots \\ y_1^{(n-1)}
        \end{pmatrix} &= \begin{pmatrix}
          0 \\ 0 \\ \vdots \\ g
        \end{pmatrix} \\
        \iff \qquad
        \begin{pmatrix}
          c_1' \\ \vdots \\ c_n'
        \end{pmatrix} &=
        \underbrace{\begin{pmatrix}
            y_1(x) & \cdots & y_n(x) \\
            \vdots & \ddots & \vdots \\
            y_1^{(n-1)}(x) & \cdots & y_n^{(n-1)}
          \end{pmatrix}^{-1}}_{\det(\dotso) = W(x) \neq 0} \begin{pmatrix}
          0 \\ \vdots \\ 0 \\ g
        \end{pmatrix}
      \end{align*}
      %
    \end{enum-alph}
  \end{enum-arab}
\end{theorem}

\begin{example} \label{4.7}
  Gegeben sei
  %
  \begin{align*}
    xy'' + 2(x+1)y' + 2y = 1 - e^{-2x}
  \end{align*}
  %
  Forme in passende Form um:
  %
  \begin{align*}
    y'' + 2(1+\tfrac 1x) y' + \frac 2x y = \frac {1-e^{-2x}}x
  \end{align*}
  %
  Die Lösung existiert auf $I = (0,\infty)$ oder auf $I' = (-\infty,0)$
  \begin{enum-arab}
  \item
    Löse zunächst das homogene System.
    Der Ansatz $y = x^{\alpha}$ liefert ein Lösung
    %
    \begin{align*}
      y_1(x) = \frac 1 x
    \end{align*}
    %
    für $x \in I$.
    Der Ansatz von d'Alembert:
    %
    \begin{align*}
      y(x) = c(x) \frac 1x
    \end{align*}
    %
    führt zu
    %
    \begin{align*}
      y_2(x) = \frac {e^{-2x}}x
    \end{align*}
    %
    auf $I$.
    Es ergibt sich
    %
    \begin{align*}
      W(x) = \det \underbrace{\begin{pmatrix}
          \frac 1x & \frac {e^{-2x}}{x} \\
          - \frac 1{x^2} & \frac {-2xe^{-2x}-e^{-2x}}{x^2}
        \end{pmatrix}}_{=:B} = - \frac {2e^{-2x}}{x^2} \neq 0 \qquad \text{(auf $I$)}
    \end{align*}
    %
    und
    %
    \begin{align*}
      B^{-1} = \begin{pmatrix}
        x+\frac 12 & \frac x2 \\
        -\frac {e^{2x}}{2} & \frac {xe^{2x}}2
      \end{pmatrix}
    \end{align*}
    %
    Variation der Konstanten:
    %
    \begin{align*}
      \begin{pmatrix}
        c_1' \\ c_2'
      \end{pmatrix} = B^{-1} \begin{pmatrix}
        0 \\ \frac {1-e^{-2x}}x
      \end{pmatrix} = \frac 12 \begin{pmatrix}
        1 - e^{-2x} \\ 1 - e^{2x}
      \end{pmatrix}
    \end{align*}
    %
    berechne $c_1, c_2$:
    %
    \begin{align*}
      c_1 = \frac 12 (x + \tfrac {e^{-2x}}2)
      \qquad
      c_2 = \frac 12 (x - \tfrac {e^{2x}}2)
    \end{align*}
    %
    für $y_{\text{part}} (x)$ muss gelten
    %
    \begin{align*}
      y_{\text{part}} (x) &= \frac 12 (x + \tfrac{e^{-2x}}2)\frac 1x + \frac 12 (x - \tfrac {e^{2x}}2) \frac {e^{-2x}}x
      &= \frac 12 + \frac {e^{-2x}}{4x} + \frac {e^{-2x}}2 - \frac 1{4x}
    \end{align*}
    %
    Wähle $y_{\text{part}} = \frac 12 + \frac {e^{-2x}}2$ und damit
    %
    \begin{align*}
      y_{\text{inhom}} = \frac 12 + \frac {e^{-2x}}2 + c_1 \frac 1x + c_2 \frac {e^{-2x}}x
      \qquad
      c_1, c_2 \in \mathbb{R}
    \end{align*}
    %
  \end{enum-arab}
\end{example}


