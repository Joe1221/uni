% Stephan Hilb, 2012 Universität Stuttgart.
%
% Dieses Werk ist unter einer Creative Commons Lizenz vom Typ
% Namensnennung - Nicht-kommerziell - Weitergabe unter gleichen Bedingungen 3.0 Deutschland
% zugänglich. Um eine Kopie dieser Lizenz einzusehen, konsultieren Sie
% http://creativecommons.org/licenses/by-nc-sa/3.0/de/ oder wenden Sie sich
% brieflich an Creative Commons, 444 Castro Street, Suite 900, Mountain View,
% California, 94041, USA.

\section{Differentialgleichungen höherer Ordnung} % 2.4


Seien $a_0, \dotsc, a_{n-1}$ gegeben, gesucht ist $y \in C^n (\R \to \R)$ mit
\[
	y^{n} = a_{n-1} y^{n-1} + \dotsb + a_1 y' + a_0 y
\]
Setze dazu
\[
	u_1 := y \qquad u_2 := y' \qquad \dotsc \qquad u_n := y^{(n-1)}
\]
Es ergibt sich im Beispiel für die DGL
\[
	y''' = y'
\]
\begin{alignat*}{2}
	u_1' &= &&u_2 \\
	u_2' &= &&u_3 \\
	u_3' &= y''' = y' = &&u_2
\end{alignat*}
Löse dieses System und setze dann
\[
	y(t) := u_1(t)
\]
Die Eindeutigkeit ist durch die Anfangsbedingungen
\[
	y(t_0) = y_0, \qquad y'(t_0) = y_1, \qquad y''(t_0) = y_2
\]
gegeben.
