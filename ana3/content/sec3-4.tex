% Stephan Hilb, 2012 Universität Stuttgart.
%
% Dieses Werk ist unter einer Creative Commons Lizenz vom Typ
% Namensnennung - Nicht-kommerziell - Weitergabe unter gleichen Bedingungen 3.0 Deutschland
% zugänglich. Um eine Kopie dieser Lizenz einzusehen, konsultieren Sie
% http://creativecommons.org/licenses/by-nc-sa/3.0/de/ oder wenden Sie sich
% brieflich an Creative Commons, 444 Castro Street, Suite 900, Mountain View,
% California, 94041, USA.

\section{Lineare Differentialgleichungen}

\begin{df} \label{4.1}
	Sei $I \subset \R$ ein Intervall, $A \in C(I \to \R^{n^2})$, $g \in C(I \to \R^n)$.
	Die Differentialgleichung
	\[
		y' = \underbrace{A(x)y + g(x)}_{=R(y)}
	\]
	für die Unbekannte $y \in C^1(I \to \R)$ heißt \emph{lineares System 1. Ordnung}.
	Für $g = 0$ nennen wir es \emph{homogen}, sonst \emph{inhomogen}.
\end{df}

\begin{nt} \label{4.2}
	Anwendung von Picard-Lindelöf auf obige Differentialgleichung mit $y(x_0) = y_0$ (für vorgegebenes $(x_0,y_0) \in I \times \R^n$), $B := \R^n$, $D = \R^n$ und Lipschitzbedingung
	\[
		\|f(x,y) - f(x,\tilde y) \| = \| A(x) (y-\tilde y) \| \le c \|y - \tilde y\|
	\]
	für $x \in K \subset I$ mit $K$ kompakt.

	Der Satz liefert also eine lokale Lösung $\phi := y \in C^1([x_0 - r, x_0 + r] \to \R^n)$.
	Erneute Anwendung mit Anfangsbedingung $y(x_0 + r) = \phi(x_0 + r)$ liefert die Lösung
	\[
		\psi := y \in C^1([x_0 + r - \tilde r, x_0 + r + \tilde r] \to \R^n)
	\]
	Zusammenkleben der Lösungen liefert
	\[
		y(x) = \begin{cases}
			\phi(x) & x_0-r \le x \le x_0 + r \\
			\psi(x) & x_0+r \le x \le x_0 + r + \tilde r
		\end{cases}
	\]
	Dieses $y$ ist stetig auf $[x_0-r, x_0 + r + \tilde r]$.
	Weiter ist $y'$ stetig auf dem selben Intervall, da
	\begin{align*}
		\phi'(x_0 + r) &= A(x_0 + r) \phi(x_0 + r) + g(x_0 + r) \\
		&= A(x_0+r) \psi(x_0+r) + g(x_0 +r) \\
		&= \psi'(x_0 +r)
	\end{align*}
	Als ist dieses $y \in C^1([x_0-r, x_0+r+r'] \to \R^n)$ eine Lösung von
	\[
		y' = A(x)y + g(x)
		\qquad \land \qquad
		y(x_0) = y_0
	\]
	Die Fortsetzung ist so lange möglich, bis $y \in C^1(I \to \R^n)$ (ohne Beweis).
\end{nt}

\begin{st} \label{4.2}
	\begin{enumerate}[1)]
		\item
			Homogener Fall ($g = 0$):
			\begin{enumerate}[a)]
				\item
					Die Menge der Lösungen
					\[
						\scr L_{\text{hom}} = \{y \in C^1(I \to \R^n : y' = A(x)y)
					\]
					bildet einen Untervektorraum von $C^1(I \to \R^n)$ der Dimension $n$.
					Für festes $x_0 \in I$ ist
					\[
						P_{x_0} : \scr L_{\text{hom}} \to \R^n : y \mapsto P_{x_0} y := y(x_0)
					\]
					ein Vektorraum-Isomorphismus.
					Eine Basis von $\scr L_{\text{hom}}$ heißt \emph{Fundamentalsystem} (auch im Fall $g \neq 0$).
				\item
					Für $n$ Lösungen $y_1, \dotsc, y_n \in \scr L_{\text{hom}}$ sind äquivalent:
					\begin{enumerate}[(i)]
						\item
							$\{y_1,\dotsc,y_n\}$ ist ein Fundamentalsystem
						\item
							$\forall x \in I : W(x) := \det(y_1(x) \dotso y_n(x)) \neq 0$ (Wronski-Determinante)
						\item
							$\exists x_0 \in I : W(x_0) \neq 0$.
					\end{enumerate}
			\end{enumerate}
		\item
			Inhomogener Fall ($g \neq 0$):
			\begin{enumerate}[a)]
				\item
					Die Menge der Lösungen $\scr L_{\text{inhom}}$ bildet einen affinen Unterraum von $C^1(I \to \R^n)$
					\[
						\scr L_{\text{inhom}} = \{y \in C^1(I \to \R^n: y' = A(x)y + g(x))
					\]

					Ist $y_{\text{part}} \in \scr L_{\text{inhom}}$ eine beliebige, aber fest gewählt (\emph{partikuläre Lösung}), so gilt
					\[
						\scr L_{\text{inhom}} = y_{\text{part}} + \scr L_{\text{hom}}
					\]
				\item
					Nutze \emph{Variaton der Konstanten}:
					Ist $\{y_1,\dotsc, y_n\}$ ein Fundamentalsystem, so ist
					\[
						y(x) = c_1(x) y_1(x) + \dotsb + c_n(x) y_n(x)
					\]
					genau dann Lösung von $y' = A(x)y + g(x)$, wenn
					\begin{align*}
						c_1'y_1 + \dotsb + c_n'y_n = g
						\qquad &\iff \quad
						\underbrace{\begin{pmatrix}
							y_1 & \cdots & y_n
						\end{pmatrix}}_{\det(\dotsc) = W(x) \neq 0} \begin{pmatrix}
							c_1' \\ \vdots \\ c_n'
						\end{pmatrix} = g \\
						& \iff \qquad \begin{pmatrix}
							c_1' \\ \vdots \\ c_n'
						\end{pmatrix} = \begin{pmatrix}
							y_1 & \cdots & y_n
						\end{pmatrix}^{-1} g
					\end{align*}
			\end{enumerate}
	\end{enumerate}
	\begin{proof}
		\begin{enumerate}[{1}a)]
			\item
				Nach \ref{4.1} ist $\scr L_{\text{hom}} \neq \emptyset$.
				Seien $y, \tilde y$ Lösungen von $y' = A(x)y$ und $\alpha, \beta \in \R$, dann ist für $u = \alpha y + \beta \tilde y$
				\[
					u' = \alpha y' + \beta \tilde y' = \alpha Ay' + \beta A \tilde y' = Au
				\]
				und damit $u \in \scr L_{\text{hom}}$.
				Damit ist $\scr L_{\text{hom}}$ Untervektorraum.
				Die Abbildung $P_{x_0}$ ist linear und wohldefiniert, da die Lösung der DGL eindeutig.
			\item
				$P_{x_0}$ ist ein Vektorraum-Isomorphismus, da
				\begin{align*}
					\{y_1, \dotsc, y_n\} \text{ lin. unabhängig}
					& \iff \{P_{x_0} y_1, \dotsc, P_{x_0} y_n \} \text{ lin. unabhängig} \\
					& \iff \det(y_1(x_0) \dotso y_n(x_0)) = W(x_0) \neq 0
				\end{align*}
		\end{enumerate}
		\fixme[zu zeigende Äquivalenzen verdeutlichen]
		\begin{enumerate}[{2}a)]
			\item
				Zeige $y_{\text{part}} + \scr L_{\text{hom}} \subset \scr L_{\text{inhom}}$.
				Sei $y \in \scr L_{\text{hom}}, u := y + y_{\text{part}}$, dann ist
				\[
					u' = y' + y_{\text{part}}' = Ay + Ay_{\text{part}} + g = Au + g
				\]
				also $u \in \scr L_{\text{inhom}}$.

				Zeige $\scr L_{\text{inhom}} \subset y_{\text{part}} + \scr L_{\text{hom}}$.
				Sei $u \in \scr L_{\text{inhom}}, y := u - y_{\text{part}}$, dann ist
				\[
					y' = Au + g - (A y_{\text{part}} + g) = Ay
				\]
				also $y \in \scr L_{\text{hom}}$ und damit $u = y_{\text{part}} + y \in y_{\text{part}} + \scr L_{\text{hom}}$.
			\item
				Setze $y(x) = c_1(x) y_1(x) + \dotsb + c_n(x) y_n(x)$ in die DGL ein:
				\begin{align*}
					y' = \sum_{i=1}^n (c_j' y_j + c_j y_j') &\stackrel != A(x) \bigg( \sum_{j=1}^n c_j y_j \bigg) + g \\
					\iff \qquad \sum_{j=1}^n c_j'y_j + \sum_{j=1}^n c_j y_j' &= \sum_{j=1}^n c_j \underbrace{A(x) y_j}_{y_j'} + g \\
					\iff \sum_{j=1}^n c_j' y_j' &= g
				\end{align*}
		\end{enumerate}
	\end{proof}
\end{st}

\begin{df} \label{4.4}
	Sei $I \subset \R$ ein Intervall und $a_0, \dotsc, a_{n-1}, g \in C(I \to \R)$.
	Dann heißt
	\[
		y^{(n)} + a_{n-1} y^{(n-1)} + \dotsb + a_1 y' + a_0 y = g
	\]
	für die Unbekannte $y \in C^n(I \to \R)$ \emph{lineare Differentialgleichung $n$-ter Ordnung}.
	Für $g = 0$ nennen wir sie \emph{homogen}, sonst \emph{inhomogen}.
\end{df}

\begin{nt} \label{4.5}
	\begin{enumerate}[1)]
		\item
			Sei $y \in C^n(I \to \R)$ eine Lösung von obiger DGL, $u_1 := y, u_2 := y', \dotsc, u_n := y^{(n-1)}$, dann ist
			\[
				u' =
				\begin{pmatrix}
					y' \\ y'' \\ \vdots \\ y^{(n-1)} \\ y^{(n)}
				\end{pmatrix}
				=
				\begin{pmatrix}
					u_2 \\ u_3 \\ \vdots \\ u_n \\ -a_{n-1} u_n - a_{n-2} u_{n-1} - \dotsb - a_0 u_1 + g
				\end{pmatrix}
				= \begin{pmatrix}
					0 & 1 & \cdots & 0 \\
					\vdots  & \ddots & \ddots & 0 \\
					0 & 0 & 0 & 1 \\
					-a_0 & -a_1 & \cdots & -a_{n-1}
				\end{pmatrix} + \begin{pmatrix}
					0 \\ \vdots \\ 0 \\ g
				\end{pmatrix}
			\]
		\item
			Ist umgekehrt $u \in C^1(I \to \R^n)$ Lösung von obigem System und $y := u_1$, dann löst $y \in C^n(I \to \R)$ die ursprünglich lineare DGL $n$-ter Ordnung.
	\end{enumerate}
\end{nt}

\begin{kor}[Folgerung] \label{4.6}
	\begin{enumerate}[1)]
		\item
			Im Fall $g = 0$:
			\begin{enumerate}[a)]
				\item
					Die Menge der Lösungen
					\[
						\scr L_{\text{hom}} = \{y \in C^n(I \to \R) : \dotso \}
					\]
					bildet einen linearen Unterraumvektor von $C^n(I \to \R)$ der Dimension $n$.
					Die Abbildung
					\[
						P_{x_0} : \scr L_{\text{hom}} \to \R^n : y \mapsto P_{x_0}(y) := \begin{pmatrix}
							y(x_0) \\ y'(x_1) \\ \vdots \\ y^{(n-1)}(x_0)
						\end{pmatrix}
					\]
					ist ein Vektorraum-Isomorphismus.
					Eine Basis von $\scr L_{\text{hom}}$ heißt \emph{Fundamentalsystem}.
				\item
					Für $y_1, \dotsc, y_n \in \scr L_{\text{hom}}$ sind äquivalent
					\begin{enumerate}[(i)]
						\item
							$\{y_1,\dotsc,y_n\}$ ist Fundamentalsystem
						\item
							\[
									\forall x \in I : W(x) = \det \begin{pmatrix}
									y_1(x) & \cdots & y_n(x) \\
									\vdots & \ddots & \vdots \\
									y_1^{(n-1)}(x) & \cdots & y_n^{(n-1)}(x)
								\end{pmatrix} \neq 0
							\]
							(Wronski-Determinante)
						\item
							$\exists x_0 \in I : W(x) \neq 0$
					\end{enumerate}
			\end{enumerate}
		\item
			Im inhomogenen Fall ($g \neq 0$):

			\begin{enumerate}[a)]
				\item
					\[
						\scr L_{\text{inhom}} = y_{\text{part}} + \scr L_{\text{hom}}
					\]
					wobei $y_{\text{part}} \in \scr L_{\text{inhom}}$ beliebig aber fest gewählt.
				\item
					Variation der Konstanten: Ist $\{y_1, \dotsc, y_n\}$ ein Fundamentalsystem, so ist
					\[
						y(x) := c_1(x) y_1(x) + \dotsb + c_n(x)y_n(x)
					\]
					genau dann Lösung der DGL, falls
					\begin{align*}
						c_1' \begin{pmatrix}
							y_1 \\ y_1' \\ \vdots \\ y_1^{(n-1)}
						\end{pmatrix} +
						c_2' \begin{pmatrix}
							y_1 \\ y_1' \\ \vdots \\ y_1^{(n-1)}
						\end{pmatrix} + \dotsb +
						c_n' \begin{pmatrix}
							y_1 \\ y_1' \\ \vdots \\ y_1^{(n-1)}
						\end{pmatrix} &= \begin{pmatrix}
							0 \\ 0 \\ \vdots \\ g
						\end{pmatrix} \\
						\iff \qquad
						\begin{pmatrix}
							c_1' \\ \vdots \\ c_n'
						\end{pmatrix} &=
						\underbrace{\begin{pmatrix}
							y_1(x) & \cdots & y_n(x) \\
							\vdots & \ddots & \vdots \\
							y_1^{(n-1)}(x) & \cdots & y_n^{(n-1)}
						\end{pmatrix}^{-1}}_{\det(\dotso) = W(x) \neq 0} \begin{pmatrix}
							0 \\ \vdots \\ 0 \\ g
						\end{pmatrix}
					\end{align*}
			\end{enumerate}
	\end{enumerate}
\end{kor}

\begin{ex} \label{4.7}
	Gegeben sei
	\[
		xy'' + 2(x+1)y' + 2y = 1 - e^{-2x}
	\]
	Forme in passende Form um:
	\[
		y'' + 2(1+\tf 1x) y' + \f 2x y = \f {1-e^{-2x}}x
	\]
	Die Lösung existiert auf $I = (0,\infty)$ oder auf $I' = (-\infty,0)$
	\begin{enumerate}[1)]
		\item
			Löse zunächst das homogene System.
			Der Ansatz $y = x^{\alpha}$ liefert ein Lösung
			\[
				y_1(x) = \f 1 x
			\]
			für $x \in I$.
			Der Ansatz von d'Alembert:
			\[
				y(x) = c(x) \f 1x
			\]
			führt zu
			\[
				y_2(x) = \f {e^{-2x}}x
			\]
			auf $I$.
			Es ergibt sich
			\[
				W(x) = \det \underbrace{\begin{pmatrix}
					\f 1x & \f {e^{-2x}}{x} \\
					- \f 1{x^2} & \f {-2xe^{-2x}-e^{-2x}}{x^2}
				\end{pmatrix}}_{=:B} = - \f {2e^{-2x}}{x^2} \neq 0 \qquad \text{(auf $I$)}
			\]
			und
			\[
				B^{-1} = \begin{pmatrix}
					x+\f 12 & \f x2 \\
					-\f {e^{2x}}{2} & \f {xe^{2x}}2
				\end{pmatrix}
			\]
			Variation der Konstanten:
			\[
				\begin{pmatrix}
					c_1' \\ c_2'
				\end{pmatrix} = B^{-1} \begin{pmatrix}
					0 \\ \f {1-e^{-2x}}x
				\end{pmatrix} = \f 12 \begin{pmatrix}
					1 - e^{-2x} \\ 1 - e^{2x}
				\end{pmatrix}
			\]
			berechne $c_1, c_2$:
			\[
				c_1 = \f 12 (x + \tf {e^{-2x}}2)
				\qquad
				c_2 = \f 12 (x - \tf {e^{2x}}2)
			\]
			für $y_{\text{part}} (x)$ muss gelten
			\begin{align*}
				y_{\text{part}} (x) &= \f 12 (x + \tf{e^{-2x}}2)\f 1x + \f 12 (x - \tf {e^{2x}}2) \f {e^{-2x}}x
				&= \f 12 + \f {e^{-2x}}{4x} + \f {e^{-2x}}2 - \f 1{4x}
			\end{align*}
			Wähle $y_{\text{part}} = \f 12 + \f {e^{-2x}}2$ und damit
			\[
				y_{\text{inhom}} = \f 12 + \f {e^{-2x}}2 + c_1 \f 1x + c_2 \f {e^{-2x}}x
				\qquad
				c_1, c_2 \in \R
			\]
	\end{enumerate}
\end{ex}


