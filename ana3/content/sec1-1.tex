% Henri Menke, 2012 Universität Stuttgart.
%
% Dieses Werk ist unter einer Creative Commons Lizenz vom Typ
% Namensnennung - Nicht-kommerziell - Weitergabe unter gleichen Bedingungen 3.0 Deutschland
% zugänglich. Um eine Kopie dieser Lizenz einzusehen, konsultieren Sie
% http://creativecommons.org/licenses/by-nc-sa/3.0/de/ oder wenden Sie sich
% brieflich an Creative Commons, 444 Castro Street, Suite 900, Mountain View,
% California, 94041, USA.

\section{Grundlagen}
\addtocounter{thmn}{1}
\setcounter{theorem}{0}

\begin{theorem}[Definition]
  Die \acct{komplexen Zahlen} werden definiert durch
  %
  \begin{align*}
    \mathbb{C} \coloneq \{ (x,y) : x,y \in \mathbb{R} \}
  \end{align*}
  %
  mit den Verknüpfungen
  %
  \begin{align*}
    (x_1,y_1) + (x_2,y_2) &\coloneq (x_1 + x_2,y_1 + y_2), \\
    (x_1,y_1) \cdot (x_2,y_2) &\coloneq (x_1 x_2 - y_1 y_2, x_1 y_2 + y_1 x_2).
  \end{align*}
\end{theorem}

\begin{notice}
  \begin{enum-arab}
    \item $(\mathbb{C},+,\cdot)$ ist ein Körper mit Nullelement $(0,0)$ und Einselement $(1,0)$.
    
    \item $\varphi \, : \, \mathbb{R} \to \mathbb{C} \, : \, x \mapsto (x,0)$ ist ein injektiver Körperhomomorphismus, insbesondere gilt
    %
    \begin{align*}
      \varphi(x_1 + x_2) &= \varphi(x_1) + \varphi(x_2), \\
      \varphi(x_1 \cdot x_2) &= \varphi(x_1) \cdot \varphi(x_2).
    \end{align*}
    %
    Identifiziere $\mathbb{R}$ mit $\varphi(\mathbb{R})=\{ (x,0) : x \in \mathbb{R} \}$. Schreibe $(x,0) \eqcolon x \in \mathbb{R}$.
    
    \item Imaginäre Einheit $\mathrm{i} \coloneq (0,1)$.
      Es gilt:
    %
    \begin{align*}
      \mathrm{i}^2 &= (0,1) \cdot (0,1) = (-1,0) = -1 \\
      (x,y) &= (x,0) + (0,y) = (x,0) + y \cdot \mathrm{i} = x + \mathrm{i} y
    \end{align*}
    %
    Rechnen in $\mathbb{C}$:
    %
    \begin{align*}
      (x_1,y_1) \cdot (x_2,y_2) &= (x_1 + \mathrm{i} y_1) \cdot (x_2 + \mathrm{i} y_2) \\
      &= x_1 x_2 + \mathrm{i} x_1 y_2 + \mathrm{i} x_2 y_1 + (\mathrm{i})^2 y_1 y_2 \\
      &= x_1 x_2 - y_1 y_2 + \mathrm{i} (x_1 y_2 + x_2 y_1)
      \quad , \\
      \frac{1}{x + \mathrm{i} y} &= \frac{1}{x + \mathrm{i} y} \, \frac{x - \mathrm{i} y}{x - \mathrm{i} y} = \frac{x - \mathrm{i} y}{x^2 - (\mathrm{i} y)^2} = \frac{1}{x^2 + y^2} (x - \mathrm{i} y) \\
      &= \frac{x}{x^2 + y^2} + \mathrm{i} \frac{-y}{x^2 + y^2} \quad {\color{DimGray} = \left( \frac{x}{x^2 + y^2}, \frac{-y}{x^2 + y^2} \right)}.
    \end{align*}
  \end{enum-arab}
  
  Der Realteil einer komplexen Zahl ist definiert als
  %
  \begin{align*}
    \Re(x,y) = \Re(x + \mathrm{i} y) = x,
  \end{align*}
  %
  der Imaginärteil ist definiert als
  %
  \begin{align*}
    \Im(x,y) = \Im(x + \mathrm{i} y) = y.
  \end{align*}
  
  \item Gaußsche Zahlenebene:
  
  \begin{figure}[H]
    \centering
    \begin{pspicture}(-0.5,-0.5)(3,2)
      \psaxes[ticks=none,labels=none]{->}(0,0)(-0.5,-0.5)(3,2)[$\color{DimGray} \Re$,0][$\color{DimGray} \Im$,180]
      \psline[linecolor=MidnightBlue]{-o}(0,0)(2.5;30)
      \psarc[linecolor=DarkOrange3]{->}(0,0){1}{0}{30}
      \uput{0.6}[15](0,0){$\color{DarkOrange3} \varphi$}
      \uput[120]{30}(1.25;30){$\color{MidnightBlue} |z|$}
      \psline(2.165,-0.1)(2.165,0.1)
      \psline(-0.1,1.25)(0.1,1.25)
      \uput[-90](2.165,0){$\color{DimGray} x = \Re(z)$}
      \uput[180](0,1.25){$\color{DimGray} y = \Im(z)$}
      \uput[30](2.5;30){$\color{DimGray} z = x + \mathrm{i} y$}
    \end{pspicture}
  \end{figure}
\end{notice}

\begin{theorem}[Definition]
  \begin{enum-arab}
    \item $z = x + \mathrm{i} y \, \implies \, \overline{z} = x - \mathrm{i} y$ heißt \acct{konjugiert komplexe Zahl} zu $z$.
    
    \item $|z| = \sqrt{x^2 + y^2} = \sqrt{z \cdot \overline{z}}$ heißt \acct{Betrag} von $z$.
    
    \item Polardarstellung: Sei $z = x + \mathrm{i} y = |z| (\cos \varphi + \mathrm{i} \sin \varphi)$, wobei $\varphi = \arg(z)$ (\acct{Argument} von $z$) eindeutig gegeben ist durch
    %
    \begin{align*}
      - \pi \leq \varphi < \pi \; , \quad \cos \varphi = \frac{x}{\sqrt{x^2 + y^2}} \; , \quad \sin \varphi = \frac{y}{\sqrt{x^2 + y^2}}.
    \end{align*}
    %
    Rechnen mit Polardarstellung:
    %
    \begin{align*}
      z_1 \cdot z_2 &= |z_1| \cdot |z_2| ( \cos(\varphi_1 + \varphi_2) + \mathrm{i} \sin(\varphi_1 + \varphi_2) ), \\
      z^n &= |z|^n (\cos n \varphi + \mathrm{i} \sin n \varphi).
    \end{align*}
    %
    Die Lösung von $z^n = r (\cos \varphi + \mathrm{i} \sin \varphi)$ ist gegeben durch
    %
    \begin{align*}
      |z| &= r^{1/n}, \\
      \varphi &= \frac{\psi}{n} + \frac{2 \pi k}{n} \; , \quad k = 0,1,\ldots,n-1.
    \end{align*}
  \end{enum-arab}
\end{theorem}

\begin{theorem}[Satz]
  $(\mathbb{C},+,\cdot,|\cdot|)$ ist ein \acct{bewerteter Körper}, das heißt für $|\cdot| : \mathbb{C} \to \mathbb{R}$ gelten:
  %
  \begin{enum-arab}
    \item $|z| \geq 0 \; \land \; (|z| = 0 \iff z = 0)$,
    
    \item $|z_1 \cdot z_2| = |z_1| \cdot |z_2|$,
    
    \item $|z_1 + z_2| \leq |z_1| + |z_2|$ ($\triangle$-Ungleichung).
    
    Außerdem gilt die umgekehrte $\triangle$-Ungleichung:
    %
    \begin{align*}
      |z_1 \pm z_2| \geq \Big| |z_1| - |z_2| \Big|.
    \end{align*}
  \end{enum-arab}
\end{theorem}

\begin{theorem}[Definition]
  Eine Folge $(z_n)$ in $\mathbb{C}$ \acct{konvergiert} gegen $z \in \mathbb{C}$, falls
  %
  \begin{align*}
    \forall \, \varepsilon > 0 \, \exists \, N_\varepsilon \in \mathbb{N} \, \forall \, n > N_\varepsilon : |z_n - z| < \varepsilon.
  \end{align*}
  %
  Man schreibt $z = \lim\limits_{n \to \infty} z_n$ oder $z_n \to z \; (n \to \infty)$.
\end{theorem}

\begin{theorem}[Satz] \label{thm:1.6}
  Es gelte $z_n \to z$ und $w_n \to w$ in $\mathbb{C}$. Dann gelten
  \begin{enum-arab}
    \item $z_n \pm w_n \to z \pm w$,
    
    \item $z_n \cdot w_n \to z \cdot w$,
    
    \item Falls $w \neq 0$ und
    %
    \begin{align*}
      w_n' \coloneq
      \begin{cases}
        1 & \text{falls } w_n = 0 \\
        w_n & \text{sonst}
      \end{cases},
    \end{align*}
    %
    dann
    %
    \begin{align*}
      \frac{z_n}{w_n} \to \frac{z}{w}.
    \end{align*}
    
    \item und außerdem für jede Folge $(v_n)$ in $\mathbb{C}$:
      \[
        v_n \to v \, \iff \, \Re\, v_n \to \Re\, v \, \land \, \Im\, v_n \to \Im\, v.
      \]
  \end{enum-arab}
\end{theorem}

\begin{theorem}[Definition]
  \begin{enum-arab}
    \item Seien $r > 0$, $z_0 \in \mathbb{C}$. Dann heißt
    %
    \begin{align*}
      K_r(z_0) \coloneq \{ z \in \mathbb{C} : |z - z_0| < r \}
    \end{align*}
    %
    offene Kreisscheibe um $z_0$.
    
    \begin{figure}[H]
      \centering
      \begin{pspicture}(-0.5,-0.5)(3,2)
        \psaxes[ticks=none,labels=none]{->}(0,0)(-0.5,-0.5)(3,2)[$\color{DimGray} \Re$,0][$\color{DimGray} \Im$,180]
        \pscircle[fillstyle=hlines,hatchcolor=DarkOrange3](1.5,1){0.8}
        \psdot*[linecolor=MidnightBlue](1.5,1)
        \rput(1.5,1){
          \psline[linecolor=MidnightBlue](0,0)(0.8;-160)
          \uput[135](0.4;-160){$\color{MidnightBlue} r$}
          \uput{1}[30](0,0){$\color{DarkOrange3} K_r(z_0)$}
        }
        \uput[0](1.5,1){$\color{MidnightBlue} z_0$}
      \end{pspicture}
    \end{figure}
    
    \item Eine Teilmenge $O \subseteq \mathbb{C}$ heißt \acct{offen}, falls
    %
    \begin{align*}
      \forall \, z \in O \, \exists \, r_z > 0 : K_{r_z}(z) \subseteq O.
    \end{align*}
    %
    $A \subseteq \mathbb{C}$ heißt \acct{abgeschlossen}, falls $\mathbb{C} \setminus A$ offen.
    
    Beliebige Vereinigungen und endliche Schnitte offener Mengen sind offen. Beliebige Schnitte und endliche Vereinigungen abgeschlossener Mengen sind abgeschlossen.
    
    Für eine beliebige Teilmenge $M \subseteq \mathbb{C}$ ist
    %
    \begin{align*}
      \mathring{M} \coloneq \underset{O \in \{O \subseteq \mathbb{C} : O \, \mathrm{offen} \, \land \, O \subseteq M \}}{\bigcup} O \quad \text{(ist offen)}
    \end{align*}
    %
    das \acct{Innere} von $M$ (die größte offene Menge $O \subseteq M$) und
    %
    \begin{align*}
      \overline{M} \coloneq \underset{A \in \{A \subseteq \mathbb{C} : A \, \mathrm{abgeschlossen} \, \land \, M \subseteq A \}}{\bigcap} A \quad \text{(ist abgeschlossen)}
    \end{align*}
    %
    der \acct{Abschluss} von $M$ (die kleinste abgeschlossene Menge $A$ mit $M \subseteq A$).
  \end{enum-arab}
\end{theorem}

\begin{example}
  \begin{enum-arab}
    \item $\emptyset$, $\mathbb{C}$ sind offen und abgeschlossen. Alle anderen Teilmengen von $\mathbb{C}$ sind entweder offen, abgeschlossen oder keins von beidem (beispielsweise halboffene Intervalle).
    
    \item $K_r(z_0)$ ist offen,
    %
    \begin{align*}
      \overline{K_r(z_0)} = \{ z \in \mathbb{C} : |z-z_0| \leq r \}.
    \end{align*}
    
    \item $\mathbb{R}$ ist nicht offen in $\mathbb{C}$.

	Für $z_0\in \mathbb{R}$ ist für beliebig kleines $r>0$ stets
    %
    \begin{align*}
      K_r(z_0) \;\cap\; \mathbb{C} \setminus \mathbb{R} \neq \emptyset.
    \end{align*}
    
    \begin{figure}[H]
      \centering
      \begin{pspicture}(-0.5,-1)(3,1)
        \pscircle[fillstyle=hlines,hatchcolor=MidnightBlue](1,0){0.6}
        \psaxes[ticks=none,labels=none]{->}(0,0)(-0.5,-1)(3,1)[$\color{DimGray} \Re$,0][$\color{DimGray} \Im$,180]
        \psline[linecolor=DarkOrange3,linewidth=1.5pt](-0.3,0)(2.7,0)
        \uput[-90](2,0){$\color{DarkOrange3} \mathbb{R}$}
        \psline(1,-0.1)(1,0.1)
        \uput[-90](1,0){$\color{DimGray} 1$}
        \uput[45](1.6,0.4){$\color{MidnightBlue} K_r(1)$}
      \end{pspicture}
    \end{figure}
    
    Ist $\mathbb{R} \subseteq \mathbb{C}$ abgeschlossen? Ja. $\iff$ Ist $\mathbb{C} \setminus \mathbb{R}$ offen? Ja.
  \end{enum-arab}
\end{example}

\begin{theorem}[Definition]
  Sei $O \subseteq \mathbb{C}$ offen, $f : O \to \mathbb{C}$. Dann heißt $f$ \acct{stetig} in $z_0 \in O$, falls
  %
  \begin{align*}
    \forall \, \varepsilon > 0 \, \exists \, \delta_\varepsilon \, \forall \, z \in O : |z - z_0| < \delta \implies |f(z) - f(z_0)| < \varepsilon
  \end{align*}
  %
  oder
  %
  \begin{align*}
    \forall \, (z_n) \text{ Folge in $O$} : z_n \to z_0 \implies f(z_n) \to f(z_0).
  \end{align*}
  %
  $f$ heißt stetig, falls $f$ in jedem $z_0 \in O$ stetig ist.
\end{theorem}

\begin{theorem}[Satz]
  \begin{enum-arab}
    \item Seien $f,g : O \to \mathbb{C}$, $z_0 \in O$, $f,g$ stetig in $z_0$. Dann sind
    %
    \begin{align*}
      f \pm g \; , \quad f \cdot g \; , \quad \frac{f}{g} \; \text{ falls $g(z_0) \neq 0$}
    \end{align*}
    %
    stetig in $z_0$.
    
    \item Sei $f : O \to \widetilde{O} \subseteq \mathbb{C}$ stetig in $z_0 \in O$ und $g : \widetilde{O} \to \mathbb{C}$ stetig in $f(z_0)$.

    Dann ist $g \circ f$ stetig in $z_0$. 
    \begin{proof}
      Lässt sich über Folgen zeigen.
    \end{proof}
  \end{enum-arab}
\end{theorem}

\begin{notice}
  Stetigkeit genauso für $f: M \to \mathbb{C}$ mit beliebiger Menge $M \subseteq \mathbb{C}$.
\end{notice}

\begin{theorem}[Funktionenfolgen]
  Sei $M \subseteq \mathbb{C}$, $f_n$, $f : M \to \mathbb{C}$
  %
  \begin{enum-arab}
    \item $(f_n)$ heißt \acct{punkteweise konvergent} gegen $f$ auf $M$, falls
    %
    \begin{align*}
      \forall \, z \in M \, \forall \, \varepsilon > 0 \, \exists \, N_{\varepsilon,z} \in \mathbb{N} \, \forall \, n > N_{\varepsilon,z} \, : \, |f_n(z) - f(z)| < \varepsilon.
    \end{align*}
    %
    \item $(f_n)$ heißt \acct{gleichmäßig konvergent} gegen $f$ auf $M$, falls
    %
    \begin{align*}
      \forall \, \varepsilon > 0 \, \exists \, N_{\varepsilon} \in \mathbb{N} \, \forall \, n > N_{\varepsilon} \, \forall \, z \in M \, : \, |f_n(z) - f(z)| < \varepsilon.
    \end{align*}
  \end{enum-arab}
\end{theorem}

\begin{theorem}[Satz]
  Seien $f_n : M \to \mathbb{C}$ stetig, $(f_n)$ gleichmäßig konvergent auf $M$ gegen $f$. Dann ist $f$ auch stetig auf $M$.
  %
  \begin{proof}
    Seien $z_0 \in M$, $\varepsilon > 0$ fest.
    Dann gilt
    %
    \begin{align*}
      |f(z) - f(z_0)| \leq |f(z) - f_n(z)| + |f_n(z) - f_n(z_0)| + |f_n(z_0)-f(z_0)|.
    \end{align*}
    %
    \textbf{1. Schritt:} Wähle ein $N_{\varepsilon}$ so, dass $|f(z) - f_n(z)| < \varepsilon/3$ für $n > N_{\varepsilon}$ und beliebige $z$ (gleichmäßige Konvergenz von $f_n$), also
    %
    \begin{align*}
      |f(z) - f(z_0)| \leq \underbrace{|f(z) - f_n(z)|}_{< \varepsilon/3} + |f_n(z) - f_n(z_0)| + \underbrace{|f_n(z_0)-f(z_0)|}_{< \varepsilon/3}.
    \end{align*}
    %
    \textbf{2. Schritt:} Setze $n \coloneq N_{\varepsilon} + 1$ und nutze die Stetigkeit von $f_n$. Für $|z - z_0| < \delta$ gilt, dann
    %
    \begin{align*}
      |f(z) - f(z_0)| &\leq \underbrace{|f(z) - f_n(z)|}_{< \varepsilon/3} + \overbrace{|f_n(z) - f_n(z_0)|}^{\mathclap{{|f_{N_{\varepsilon} + 1}(z) - f_{N_{\varepsilon} + 1}(z_0)| < \varepsilon/3}}} + \underbrace{|f_n(z_0)-f(z_0)|}_{< \varepsilon/3} \\
      &\leq \frac{\varepsilon}{3} + \frac{\varepsilon}{3} + \frac{\varepsilon}{3} = \varepsilon.
    \end{align*}
    %
  \end{proof}
\end{theorem}

\begin{theorem}[Definition]
  Eine Reihe $\sum\limits_{n=0}^{\infty} a_n$ heißt \acct{absolut konvergent}, falls $\sum\limits_{n=0}^{\infty} |a_n|$ konvergent ist.
\end{theorem}

\begin{theorem}[Weierstraß-Kriterium] \label{thm:1.14}
  Ist $\sum\limits_{n=0}^{\infty} a_n$ mit $a_n \geq 0$ konvergent und gilt $f_n : M \to \mathbb{C}$, $|f_n(z)| \leq a_n$ auf $M \subseteq \mathbb{C}$, so ist die Reihe $\sum\limits_{n=0}^{\infty} f_n(z)$ gleichmäßig konvergent auf $M$ und absolut konvergent für $z \in M$.
\end{theorem}

\begin{example}
  Seien $M \coloneq \overline{K_2(0)} \subseteq \mathbb{C}$, $f_n(z) \coloneq \frac{z^n}{(n+1)^2 2^n}$. Wähle
  %
  \begin{align*}
    a_n \coloneq \frac{1}{(n+1)^2} \; \implies \;
    \begin{dcases}
      \sum\limits_{n=0}^{\infty} \frac{1}{(n+1)^2} < \infty \; , \quad \text{d.h. konvergent} \\
      |f_n(z)| \leq a_n
    \end{dcases}
  \end{align*}
  %
  $\implies$ $g(z) \coloneq \sum\limits_{n=0}^{\infty} f_n(z)$ ist stetig auf $\overline{K_2(0)}$.
\end{example}

\begin{theorem}[Potenzreihen]
  Sei $(a_n)$ eine Folge in $\mathbb{C}$,
  %
  \begin{align*}
    R \coloneq \frac{1}{\limsup\limits_{n \to \infty} \sqrt[n]{|a_n|}} \; , \quad \text{mit } \frac{1}{0} \coloneq \infty, \frac{1}{\infty} \coloneq 0.
  \end{align*}
  %
  Dann konvergiert die Potenzreihe
  %
  \begin{align*}
    f(z) \coloneq \sum\limits_{n=0}^{\infty} a_n (z-z_0)^n
  \end{align*}
  %
  für $|z - z_0| < R$ und divergiert für $|z - z_0| > R$. Sie konvergiert gleichmäßig auf jedem Kreis $\overline{K_r(z_0)}$ mit $0 < r < R$. Insbesondere ist $f$ stetig auf $K_R(z_0)$.
  
  \begin{figure}[H]
    \centering
    \begin{pspicture}(-1,-1)(1,1)
      \SpecialCoor
      \pscircle(0,0){1}
      \pscircle[linecolor=MidnightBlue](0,0){0.8}
      \uput[0](0,0){\color{DimGray} $z_0$}
      \uput[0](0.7;60){\color{DarkOrange3} $z$}
      \psline{->}(0,0)(1;-135)
      \uput[90](0.5;-135){\color{DimGray} $R$}
      \psline[linecolor=MidnightBlue]{->}(0,0)(0.8;-90)
      \uput[0](0.5;-90){\color{MidnightBlue} $r$}
      \psdot(0,0)
      \psdot[linecolor=DarkOrange3](0.7;60)
    \end{pspicture}
    \vspace*{-2em}
  \end{figure}
  %
  Falls die Folge $(\frac{a_{n+1}}{a_n})$ konvergiert, gilt
  %
  \begin{align*}
    R = \dfrac{1}{\lim_{n\to \infty} \left| \frac{a_{n+1}}{a_n} \right|}.
  \end{align*}
\end{theorem}

\begin{theorem}[Definition]
  \begin{align*}
    \mathrm{e}^z &\coloneq \sum\limits_{n=0}^{\infty} \frac{z^n}{n!} \; , \quad \text{für } z \in \mathbb{C}, \\
    \cos z &\coloneq \sum\limits_{n=0}^{\infty} (-1)^n \frac{z^{2n}}{(2n)!} \; , \quad \text{für } z \in \mathbb{C}, \\
    \sin z &\coloneq \sum\limits_{n=0}^{\infty} (-1)^n \frac{z^{2n + 1}}{(2n + 1)!} \; , \quad \text{für } z \in \mathbb{C}.
  \end{align*}
\end{theorem}

\begin{example*}
  Berechne den Konvergenzradius von $\mathrm{e}^z = \sum_{n=0}^\infty \frac 1{n!} z^n$:
  %
  \begin{align*}
    \left| \frac{a_{n+1}}{a_n} \right| = \frac{\frac{1}{(n+1)!}}{\frac{1}{n!}} = \frac{1}{n+1} \to 0
  \end{align*}
  %
  $\implies R = \infty$, die Potenzreihe ist konvergent auf ganz $\mathbb{C}$.
\end{example*}

\begin{theorem}[Cauchy-Produkt von Reihen]
  Ist $\sum\limits_{n=0}^{\infty} a_n$ absolut konvergent und $\sum\limits_{n=0}^{\infty} b_n$ konvergent in $\mathbb{C}$, so gilt
  %
  \begin{align*}
    \left( \sum\limits_{n=0}^{\infty} a_n \right) \left( \sum\limits_{n=0}^{\infty} b_n \right) =
    \sum\limits_{n=0}^{\infty} \left( \sum\limits_{k=0}^{n} a_k b_{n-k} \right).
  \end{align*}
  %
  \begin{proof}
    Seien
    %
    \begin{align*}
      A \coloneq \sum\limits_{n=0}^{\infty} a_n \; , \quad
      B \coloneq \sum\limits_{n=0}^{\infty} b_n \; , \quad
      B_{\ell} \coloneq \sum\limits_{n=0}^{\ell} b_n
    \end{align*}
    %
    Dann gilt
    %
    \begin{align*}
      \sum\limits_{n=0}^{N} \sum\limits_{k=0}^{n} a_k b_{n-k} &\overset{m = n - k}{=} \sum\limits_{n=0}^{N} \sum\limits_{m=0}^{N-m} a_n b_m = \sum\limits_{n=0}^{N} a_n \underbrace{\sum\limits_{m=0}^{N-m} b_m}_{B_{N-m} - B + B} \\
      &\overset{l \coloneq N - n}{\underset{\substack{n = N-\ell \\ \ell = 0,1,\ldots,N}}{=}}
        \underbrace{\sum\limits_{\ell = 0}^{N} a_{N - \ell} (B_{\ell} - B)}_{\hypertarget{eq:stern1}{{\color{DarkRed} (\ast)}} \text{ Z.Z. } \to 0} + \underbrace{\sum\limits_{n=0}^{N} a_n B}_{\to A \cdot B} \\
      \hyperlink{eq:stern1}{{\color{DarkRed} (\ast)}} &= \sum\limits_{\ell = 0}^{N_{\varepsilon}} a_{N-\ell} (B_{\ell} - B) + \sum\limits_{\ell=N_{\varepsilon}+1}^{N} a_{N-\ell} (B_{\ell} - B).
    \end{align*}
    %
    \begin{figure}[H]
      \caption{Veränderung der Indizierung in der Reihe}
      \centering
      \begin{pspicture}(-0.5,-0.5)(3.5,3.5)
        \psaxes[ticks=none,labels=none]{->}(0,0)(-0.5,-0.5)(3.5,3.5)[{\color{DimGray} $k$ bzw. $n$},0][{\color{DimGray} $m$},0]
        \uput[-135](0,0){\color{DimGray} $0$}
        
        \uput[180](0,0.5){\color{DimGray} $1$}
        \uput[180](0,1){\color{DimGray} $2$}
        \uput[180](0,1.5){\color{DimGray} $3$}
        \uput[180](0,2.25){\color{DimGray} $\vdots$}
        \uput[180](0,3){\color{DimGray} $N$}
        
        \uput[-90](0.5,0){\color{DimGray} $1$}
        \uput[-90](1,0){\color{DimGray} $2$}
        \uput[-90](1.5,0){\color{DimGray} $3$}
        \uput[-90](2.25,0){\color{DimGray} $\cdots$}
        \uput[-90](3,0){\color{DimGray} $N$}
        
        \psdot[linecolor=MidnightBlue](0,0)
        \psline[showpoints=true,linecolor=MidnightBlue](0.5,0)(0,0.5)
        \psline[showpoints=true,linecolor=MidnightBlue](1,0)(0.5,0.5)(0,1)
        \psline[showpoints=true,linecolor=MidnightBlue](1.5,0)(1,0.5)(0.5,1)(0,1.5)
        \psline[showpoints=true,linecolor=MidnightBlue](3,0)(2.5,0.5)
        \psline[showpoints=true,linecolor=MidnightBlue](0,3)(0.5,2.5)
        \psline[linestyle=dotted,linecolor=MidnightBlue](2.5,0.5)(0.5,2.5)
        \uput[45](0.5,2.5){\color{MidnightBlue} $m=N-n$}
      \end{pspicture}
    \end{figure}
    %
    \textbf{1. Schritt:} Wähle ein passendes $N_{\varepsilon}$:
    %
    \begin{align*}
      \left| \sum\limits_{\ell=N_{\varepsilon}+1}^{N} a_{N-\ell} (B_{\ell} - B) \right|
      &\leq \sup\limits_{\ell \geq N_{\varepsilon}+1} |B_{\ell} - B| \sum\limits_{\ell=N_{\varepsilon}+1}^{N} |a_{N-\ell}| \\
      &\leq \sup\limits_{\ell \geq N_{\varepsilon}+1} |B_{\ell} - B| \sum\limits_{n=0}^{\infty} |a_n| \\
      &\leq \frac{\varepsilon}{2} \; , \quad \text{da } B_{\ell} \to B.
    \end{align*}
    %
    \textbf{2. Schritt:} Abschätzen liefert
    %
    \begin{align*}
      \left| \sum\limits_{\ell = 0}^{N_{\varepsilon}} a_{N-\ell} (B_{\ell} - B) \right|
      &\leq \max\limits_{0 \leq \ell \leq N_{\varepsilon}} |B_{\ell} - B| \sum\limits_{n=N-N_{\varepsilon}}^{\infty} |a_n| \\
      &\leq \frac{\varepsilon}{2} \; , \quad \text{für } N - N_{\varepsilon} > \widetilde{N}_{\varepsilon} \quad \text{, da } \sum\limits_{n=0}^{\infty} |a_n| \text{ konvergent}.
    \end{align*}
    %
    $\implies |\hyperlink{eq:stern1}{{\color{DarkRed} (\ast)}}| < \varepsilon$ für $N - N_{\varepsilon} > \widetilde{N}_{\varepsilon}$ bzw. für $N > \widetilde{N}_{\varepsilon} + N_{\varepsilon}$.
  \end{proof}
\end{theorem}

\begin{notice}[Folgerung] \label{thm:1.20}
  \begin{align*}
    \mathrm{e}^{z+w} &= \mathrm{e}^{z} \, \mathrm{e}^{w} \; \, \quad \text{für } z,w \in \mathbb{C}.
  \end{align*}
  %
  \begin{proof} Wir verwenden die Reihendarstellung der Exponentialfunktion
    \begin{align*}
      \mathrm{e}^{z} \, \mathrm{e}^{w}
      &= \bigg( \sum\limits_{n=0}^{\infty} \underbrace{\frac{z^n}{n!}}_{a_n} \bigg) \bigg( \sum\limits_{n=0}^{\infty} \underbrace{\frac{w^n}{n!}}_{b_n} \bigg).
    \end{align*}
    %
    Beide Reihen sind absolut konvergent, also gilt
    %
    \begin{align*}
      e^z e^w &= \sum\limits_{n=0}^{\infty} \sum\limits_{k=0}^{n} \frac{z^k}{k!} \frac{w^{n-k}}{(n-k)!} \\
      &= \sum\limits_{n=0}^{\infty} \frac{1}{n!} \sum\limits_{k=0}^{n} \frac{z^k}{k!} \frac{w^{n-k}}{(n-k)!} n! \\
      &= \sum\limits_{n=0}^{\infty} \frac{1}{n!} \sum\limits_{k=0}^{n} \binom{n}{k} z^k w^{n-k} \quad\qquad \text{\color{DimGray} (Binomischer Lehrsatz)} \\
      &= \sum\limits_{n=0}^{\infty} \frac{1}{n!} (z+w)^n = \mathrm{e}^{z+w}.
    \end{align*}
  \end{proof}
\end{notice}

\begin{notice}
  An der Taylorreihe erkennt man jeweils, dass $\mathrm{e}^z\Big|_{z = x \in \mathbb{R}}$, $\cos z\Big|_{z = x \in \mathbb{R}}$ und $\sin z\Big|_{z = x \in \mathbb{R}}$ mit den bereits bekannten Definition auf $\mathbb{R}$ übereinstimmen.
\end{notice}

\begin{notice}[Folgerung] \label{thm:1.22}
  \begin{enum-arab}
    \item Es gilt für $z \in \mathbb{C}$ die \acct{Eulersche Formel}
      %
      \begin{align*}
        \mathrm{e}^{\mathrm{i} z} = \cos z + \mathrm{i} \sin z.
      \end{align*}
      \begin{proof}
        \begin{align*}
          \mathrm{e}^{\mathrm{i} z}
          &= \sum\limits_{n=0}^{\infty} \frac{(\mathrm{i} z)^n}{n!}
          = \underbrace{\sum\limits_{n=0}^{\infty} (-1)^{n} \frac{z^{2n}}{(2n)!}}_{=\cos z} + \mathrm{i} \underbrace{\sum\limits_{n=0}^{\infty} (-1)^{n} \frac{z^{2n+1}}{(2n+1)!}}_{=\sin z}.
        \end{align*}
      \end{proof}
    
    \item Mit der Polardarstellung: $z = r (\cos \varphi + \mathrm{i} \sin \varphi) = r e^{\mathrm{i}\phi}$ ergibt sich
    %
    \begin{align*}
      z^n &= r^n \mathrm{e}^{\mathrm{i} n \varphi}, \\
      z_1 z_2 &= r_1 r_2 \mathrm{e}^{\mathrm{i} (\varphi_1 + \varphi_2)}, \\
      \frac{z_1}{z_2} &= \frac{r_1}{r_2} \mathrm{e}^{\mathrm{i} (\varphi_1 - \varphi_2)}.
    \end{align*}
    \begin{proof}
      Man vergewissere sich zunächst die Existenz der Polardarstellung.
      Die Gleichungen sind dann mit \ref{thm:1.20} leicht zu zeigen.
    \end{proof}

  \end{enum-arab}
\end{notice}

\begin{theorem}[Definition]
  \begin{enum-arab}
    \item $f : O \to \mathbb{C}$ heißt \acct{differenzierbar} in $z_0 \in O$, falls
    %
    \begin{align*}
      f'(z_0) \coloneq \lim\limits_{\substack{z \to z_0}{z \neq z_0}} \frac{f(z) - f(z_0)}{z - z_0}
    \end{align*}
    %
    existiert; $f'(z_0)$ heißt \acct{Ableitung} von $f$ in $z_0$.
    
    \item $f$ heißt \acct{differenzierbar}, falls $f$ in jedem $z_0 \in O$ differenzierbar ist. $f' : O \to \mathbb{C}$ heißt \acct{Ableitungsfunktion} von $f$.
  \end{enum-arab}
\end{theorem}

\begin{example} \label{thm:1.23}
  \begin{enum-arab}
    \item Für $f : \mathbb{C} \to \mathbb{C} : z \mapsto c$ ist $f'(z)=0$.
    
    \item Für $f : \mathbb{C} \to \mathbb{C} : z \mapsto z$ ist $f'(z)=1$.
    
    \item Für $f : \mathbb{C} \to \mathbb{C} : z \mapsto z^n$ ist $f'(z)=n \, z^{n-1}$.
    
    \item $f : \mathbb{C} \to \mathbb{C} : z \mapsto |z|^2$ ist in $z_0 \neq 0$ nicht differenzierbar.
    
    Sei $z_0 = x_0 + \mathrm{i} y_0 \neq 0$.
    Dann gilt
    %
    \begin{align*}
      z_h &= x_0 + h + \mathrm{i} y_0 \; \implies \; \frac{|z_h|^2 - |z_0|^2}{z_h - z_0} = \frac{2 h x_0 + h^2}{h} \to 2 x_0 \qquad (h \to 0),\\
      z_h &= x_0 + \mathrm{i} (y_0 + h) \; \implies \; \frac{|z_h|^2 - |z_0|^2}{z_h - z_0} = \frac{2 h x_0 + h^2}{\mathrm{i} h} \to - \mathrm{i} 2 x_0 \qquad (h \to 0).
    \end{align*}
  \end{enum-arab}
\end{example}

% % % Vorlesung vom 22.10.2012

\begin{theorem}[Satz]
  Sei $O \subseteq \mathbb{C}$ offen und $f : O \to \mathbb{C}$, $z_0 \in \mathbb{C}$. Dann sind äquivalent:
  %
  \begin{enum-roman}
    \item $f$ ist differenzierbar in $z_0$ mit Ableitung $f'(z_0)$.
    
    \item $\exists f'(z_0) \in \mathbb{C} : f(z) = f(z_0) + f'(z_0) (z-z_0) + o(|z-z_0|)$.
  \end{enum-roman}
  
  \begin{proof}
    \begin{align*}
      &\qquad \text{$f$ differenzierbar in $z_0$} \\
      &\iff \quad \exists f'(z_0) \in \mathbb{C} : \lim_{z\to z_0} \frac{f(z)-f(z_0)}{z-z_0} = f'(z_0) \\
      &\iff \quad \exists f'(z_0) \in \mathbb{C} : \lim_{z\to z_0} \frac{f(z)-f(z_0)-f'(z_0)(z-z_0)}{z-z_0} = 0 \\
      &\iff \quad \exists f'(z_0) \in \mathbb{C} : f(z) = f(z_0) + f'(z_0)(z-z_0) + o(|z-z_0|)
    \end{align*}
  \end{proof}
\end{theorem}

\begin{theorem}[Satz] \label{thm:1.25}
  Seien $f,g : O \to \mathbb{C}$ differenzierbar in $z_0$.
  Dann gilt
  \begin{enum-arab}
    \item $f,g$ stetig in $z_0$.
    
    \item $(f+g)'(z_0) = f'(z_0) + g'(z_0)$.
    
    \item $(f \, g)'(z_0) = f'(z_0) \, g(z_0) + f(z_0) \, g'(z_0)$.
    
    \item Falls $g(z_0) \neq 0$: $\displaystyle \left( \frac{f}{g} \right)'(z_0) = \frac{f'(z_0) \, g(z_0) - f(z_0) \, g'(z_0)}{g^2(z_0)}$
    
    \item $(f \circ g)'(z_0) = f'(g(z_0)) \cdot g'(z_0)$.
  \end{enum-arab}
\end{theorem}

\begin{example}
  \begin{enum-arab}
    \item Polynomfunktionen sind differenzierbar auf $\mathbb{C}$ (folgt aus \ref{thm:1.23}, 3.) und \ref{thm:1.25}, 2.)).
    
    \item Gebrochen rationale Funktionen sind auf ihrem Definitionsbereich differenzierbar.
  \end{enum-arab}
\end{example}

\begin{theorem}[Definition]
  \begin{enum-arab}
    \item Sei $\gamma \in C^1([a,b] \to \mathbb{C})$ (also auch $\Re \gamma, \Im \gamma \in C^1$). Dann heißt $\gamma$ \acct{Weg} von $z_1 \coloneq \gamma(a)$ nach $z_2 \coloneq \gamma(b)$. Falls $z_1 = z_2$, heißt $\gamma$ \acct{geschlossen}.
    
    \item Sei $\gamma$ ein Weg und $f \in C(\mathrm{Bild}(\gamma) \to \mathbb{C})$. Dann heißt
    %
    \begin{align*}
      \int_\gamma f(z) \, \mathrm{d}z &\coloneq \int\limits_{a}^{b} f(\gamma(t)) \, \gamma'(t) \, \mathrm{d}t 
    \end{align*}
    %
    das \acct{Integral von $f$} längs $\gamma$.
  \end{enum-arab}
\end{theorem}

\begin{notice*}
  Das obige, komplexe Integral definiert man durch getrennte Betrachtung von Real- und Imaginärteil:
    \begin{align*}
      \int\limits_{a}^{b} f(\gamma(t)) \, \gamma'(t) \, \mathrm{d}t 
      &\coloneq \int\limits_{a}^{b} \Big[ \Re(f(\gamma(t))) \, \Re(\gamma'(t)) - \Im(f(\gamma(t))) \, \Im(\gamma'(t)) \Big] \mathrm{d}t \\
      &\quad + \mathrm{i} \int\limits_{a}^{b} \Big[ \Im(f(\gamma(t))) \, \Re(\gamma'(t)) + \Re(f(\gamma(t))) \, \Im(\gamma'(t)) \Big] \mathrm{d}t
    \end{align*}
\end{notice*}

\begin{notice}
  \begin{enum-arab}
    \item Sei $\widetilde{\gamma}$ eine andere Parametrisierung des Weges $\gamma$.
      Existiert
      \begin{align*}
        \varphi \in C^1([a',b'] \to [a,b])  \quad\text{mit}\quad \varphi(a') = a \; , \; \varphi(b') = b
      \end{align*}
      und
      \begin{align*}
        \widetilde{\gamma}(s) = (\gamma \circ \varphi)(s) \quad \text{für } a' \leq s \leq b'.
      \end{align*}
    %
    Dann folgt aus der Substitutionsregel für reelle Integration
    %
    \begin{align*}
      \int_\gamma f(z) \, \mathrm{d}z &= \int\limits_{a}^{b} f(\gamma(t)) \, \gamma'(t) \, \mathrm{d}t \; , \quad \left(t=\varphi(s), \frac {\mathrm{d}t}{\mathrm{d}s} = \varphi'(s)\right) \\
      &= \int\limits_{a'}^{b'} \underbrace{f(\gamma(\varphi(s)))}_{f(\widetilde{\gamma}(s))} \, \underbrace{\gamma'(\varphi(s)) \, \varphi'(s)}_{\frac{\mathrm{d}}{\mathrm{d}s} \widetilde{\gamma}(s) = \frac{\mathrm{d}}{\mathrm{d}s} (\gamma \circ \varphi)(s)} \, \mathrm{d}s \\
      &= \int_{\widetilde{\gamma}} f(z) \, \mathrm{d}z
    \end{align*}
    
    \item Es sei $-\gamma$ der zu $\gamma$ entgegengesetzt orientierte Weg,
    %
    \begin{align*}
      -\gamma(t) \coloneq \gamma(a+b-t) \; , \quad a \leq t \leq b.
    \end{align*}
    %
    Es gilt 
    %
    \begin{align*}
      \int_{-\gamma} f(z) \, \mathrm{d}z = - \int_{\gamma} f(z) \, \mathrm{d}z.
    \end{align*}
    
    \item Jede Kurve kann so umparametrisiert werden, dass $a=0$ und $b=1$ gilt.
    
    \item Wir schreiben für die Verbindungsstrecke von $z_1$ zu $z_2$ mit der Parametrisierung
    %
    \begin{align*}
      \gamma(t) = z_1 + t(z_2 - z_1) \; , \quad 0 \leq t \leq 1
    \end{align*}
    %
    auch $\gamma = [z_1,z_2]$.
    
    \begin{figure}[H]
      \centering
      \begin{pspicture}(0,0)(1,1)
        \cnode(0,0){2pt}{A}
        \cnode(1,1){2pt}{B}
        \uput[180](A){\color{DimGray} $z_1$}
        \uput[0](B){\color{DimGray} $z_2$}
        \ncline[arrows=->,linecolor=DarkOrange3]{A}{B}
        \naput*{\color{DarkOrange3} $[z_1,z_2]$}
      \end{pspicture}
    \end{figure}
    
    \item Verallgemeinerung: Ein Weg kann auch nur stückweise $C^1$ sein, d.h. es gilt
    %
    \begin{align*}
      \gamma \in C([a,b] \to \mathbb{C})
    \end{align*}
    %
    und es existieren $a < t_0 < t_1 < t_2 < \ldots < t_n = b$, sodass
    %
    \begin{align*}
      \gamma_j \coloneq \gamma \Big|_{[t_{j-1},t_j]} \in C^1([t_{j-1},t_j] \to \mathbb{C}).
    \end{align*}
    %
    (d.h., die Ableitung darf endlich viele Sprünge haben). Dann setzt man
    %
    \begin{align*}
      \int_\gamma f(z) \, \mathrm{d}z \coloneq \sum\limits_{j=1}^{n} \int_{\gamma_j} f(z) \, \mathrm{d}z.
    \end{align*}
  \end{enum-arab}
\end{notice}

\begin{example} \label{thm:1.29}
  \begin{enum-arab}
    \item Es sei $f(z) \coloneq z^3$ und $\gamma = [0,1+\mathrm{i}]$.
      Dann gilt
    %
    \begin{align*}
      \int_{\mathrlap{[0,1+\mathrm{i}]}} z^3 \, \mathrm{d}z
      &= \int\limits_{0}^{1} (t (1+\mathrm{i}))^3 \, (1+\mathrm{i}) \, \mathrm{d}t \\
      &= (1+\mathrm{i})^4 \int\limits_{0}^{1} t^3 \, \mathrm{d}t \\
      &= \frac{(1+\mathrm{i})^4}{4}
    \end{align*}
    
    \item $f(z) \coloneq \dfrac{1}{z}$, $\gamma(t) = \mathrm{e}^{\mathrm{i} t}$, $0 \leq t \leq 2 \pi$. Da $\gamma(0) = \gamma(2 \pi)$ ist der Weg geschlossen.
    %
    \begin{figure}[H]
      \centering
      \begin{pspicture}(-1,-1)(1,1)
        \psaxes[ticks=none,labels=none]{->}(0,0)(-1,-1)(1,1)[\color{DimGray} $\!\Re$,-45][\color{DimGray} $\!\Im$,0]
        \psarc[linecolor=DarkOrange3]{->}(0,0){0.7}{30}{390}
        \uput[30](0.7;30){\color{DarkOrange3} $\gamma$}
        \psline(0.7,-0.1)(0.7,0.1)
        \uput[-75](0.7,0){\color{DimGray} $1$}
      \end{pspicture}
      \vspace*{-4em}
    \end{figure}
    %
    \begin{align*}
      \int_{\gamma} \frac{1}{z} \mathrm{d}z = \int\limits_{0}^{2 \pi} \frac{1}{\mathrm{e}^{\mathrm{i} t}} \mathrm{i} \mathrm{e}^{\mathrm{i} t} \mathrm{d}t = 2 \pi \mathrm{i} \neq 0
    \end{align*}
  \end{enum-arab}
\end{example}

\begin{theorem}[Satz]
  Sei $O \subseteq \mathbb{C}$ offen, $F : O \to \mathbb{C}$ differenzierbar, $f \coloneq F'$ (d.h. $F$ ist eine \acct{Stammfunktion} von $f$). Ist $\gamma$ ein Weg in $O$, so gilt
  %
  \begin{align*}
    \int_\gamma f(z) \, \mathrm{d}z = F(\gamma(b)) - F(\gamma(a)).
  \end{align*}
  
  \begin{proof}
    Wegen
    \begin{align*}
      \frac{\mathrm{d}}{\mathrm{d}t} F(\gamma(t)) &= \lim\limits_{\substack{s \to t}{s \neq t}} \frac{F(\gamma(s)) - F(\gamma(t))}{\gamma(s) - \gamma(t)} \, \frac{\gamma(s) - \gamma(t)}{s - t} \\
      &= F'(\gamma(t)) \, \gamma'(t) = f(\gamma(t))  \, \gamma'(t) 
    \end{align*}
    %
    können wir die Kettenregel verwenden und es ergibt sich
    %
    \begin{align*}
      \int_\gamma f(z) \, \mathrm{d}z &= \int\limits_{a}^{b} f(\gamma(t)) \, \gamma'(t) \, \mathrm{d}t \\
      &= \int\limits_{a}^{b} \frac{\mathrm{d}}{\mathrm{d}t} F(\gamma(t))  \, \mathrm{d}t \\
      &= \int\limits_{a}^{b} \left[ \frac{\mathrm{d}}{\mathrm{d}t} \Re F(\gamma(t)) + \mathrm{i} \frac{\mathrm{d}}{\mathrm{d}t} \Im F(\gamma(t)) \right] \, \mathrm{d}t.
    \intertext{Nach dem Hauptsatz der Integral- und Differentialrechung im Reellen gilt}
      &= \Big( \Re F(\gamma(b)) + \mathrm{i} \Im F(\gamma(b)) \Big) - \Big( \Re F(\gamma(a)) + \mathrm{i} \Im F(\gamma(a)) \Big) \\
      &= F(\gamma(b)) - F(\gamma(a)).
    \end{align*}
  \end{proof}
\end{theorem}

\begin{notice}[Folgerung]
  \begin{enum-arab}
    \item Besitzt $f$ eine Stammfunktion und ist $\gamma$ geschlossen, so folgt
    %
    \begin{align*}
      \int_\gamma f(z) \, \mathrm{d}z = 0.
    \end{align*}
    
    \item $f : \mathbb{C} \setminus \{0\} \to \mathbb{C} : z \mapsto \dfrac{1}{z}$ besitzt keine Stammfunktion (siehe \ref{thm:1.29}, 2.)).
  \end{enum-arab}
\end{notice}

\begin{theorem}[Satz] \label{thm:1.32}
  Sei $\gamma$ ein Weg, $f \in C(\mathrm{Bild}(\gamma) \to \mathbb{C})$. Dann
  %
  \begin{align*}
    \left| \int_\gamma f(z) \, \mathrm{d}z \right| 
    &\leq \int\limits_{a}^{b} \left| f(\gamma(t)) \right| \, \left| \gamma'(t) \right| \, \mathrm{d}t \\
    &\leq \max\limits_{a \leq t \leq b} \left| f(\gamma(t)) \right| \int\limits_{a}^{b} \left| \gamma'(t) \right| \, \mathrm{d}t.
  \end{align*}
  
  \begin{proof} Sei $\varphi = - \arg \left( \int_\gamma f(z) \, \mathrm{d}z \right)$.
    Dann gilt
    %
    \begin{align*}
      \left| \int_\gamma f(z) \, \mathrm{d}z \right| &= \mathrm{e}^{\mathrm{i} \varphi} \int_\gamma f(z) \, \mathrm{d}z \\
      &= \int\limits_{a}^{b} \mathrm{e}^{\mathrm{i} \varphi} \, f(\gamma(t)) \, \gamma'(t) \, \mathrm{d}t \\
      &= \underbrace{\int\limits_{a}^{b} \Re \left( \mathrm{e}^{\mathrm{i} \varphi} \, f(\gamma(t)) \, \gamma'(t) \right) \, \mathrm{d}t}_{\ge 0}
        + \underbrace{\mathrm{i} \int\limits_{a}^{b} \Im \left( \mathrm{e}^{\mathrm{i} \varphi} \, f(\gamma(t)) \, \gamma'(t) \right) \, \mathrm{d}t}_{=0} \\
      &= \bigg| \int\limits_{a}^{b} \Re \left( \mathrm{e}^{\mathrm{i} \varphi} \, f(\gamma(t)) \, \gamma'(t) \right) \, \mathrm{d}t \bigg| \\
      &\leq \int\limits_{a}^{b} \left| \mathrm{e}^{\mathrm{i} \varphi} \, f(\gamma(t)) \, \gamma'(t) \right| \, \mathrm{d}t \\
      &\leq \int\limits_{a}^{b} \left| f(\gamma(t)) \right| \, \left| \gamma'(t) \right| \, \mathrm{d}t.
    \end{align*}
    %
    Genau so, falls $\gamma$ nur stückweise $C^1$.
  \end{proof}
\end{theorem}

\begin{theorem}[Definition]
  %
  \begin{align*}
    L(\gamma) \coloneq \int\limits_{a}^{b} \left| \gamma'(t) \right| \, \mathrm{d}t
  \end{align*}
  %
  heißt \acct{Länge} von $\gamma$. Damit wird \ref{thm:1.32} zu
  %
  \begin{align*}
    \left| \text{Integral von $f$ über $\gamma$} \right| \leq \left( \text{Länge von $\gamma$} \right) \, \left( \text{$\max |f|$ auf $\gamma$} \right).
  \end{align*}
\end{theorem}

\begin{example}
  \begin{enum-arab}
    \item $\gamma \coloneq [z_1,z_2]$, $\gamma(t) = z_1 + t(z_2 - z_1)$,
    %
    \begin{align*}
      L(\gamma) = \int\limits_{0}^{1} |z_2 - z_1| \, \mathrm{d}t = |z_2 - z_1|.
    \end{align*}
    
    \item $\gamma \coloneq \mathrm{e}^{\mathrm{i} t}$, $0 \leq t \leq 2 \pi$
    %
    \begin{enum-alph}
      \item
      %
      \begin{align*}
        |2 \pi \mathrm{i}| = \left| \int_\gamma \frac{1}{z} \, \mathrm{d}z \right| \leq \max\limits_{z = \mathrm{e}^{\mathrm{i} t},\, 0 \leq t \leq 2 \pi} \left| \frac{1}{z} \right| \, \int\limits_{0}^{2 \pi} \left| \mathrm{i} \,  \mathrm{e}^{\mathrm{i} t} \right| \, \mathrm{d}t = 2 \pi.
      \end{align*}
      
      \item
      %
      \begin{align*}
        |0| = \left| \int_\gamma z \, \mathrm{d}z \right| \leq \max\limits_{z = \mathrm{e}^{\mathrm{i} t}} |z| \, 2 \pi = 2 \pi.
      \end{align*}
    \end{enum-alph}
  \end{enum-arab}
\end{example}

\begin{theorem}[Satz] \label{thm:1.35}
  Sei $(a_n)$ eine Folge in $\mathbb{C}$, $R = \dfrac{1}{\limsup\limits_{n \to \infty} \sqrt[n]{|a_n|}} > 0$.
  Dann ist
  %
  \begin{align*}
    f(z) = \sum\limits_{n=0}^{\infty} a_n \, z^n \; , \quad \text{für } |z| < R
  \end{align*}
  %
  auf dem offenen Konvergenzkreis $\mathring{K}_R(0)$ beliebig oft differenzierbar mit
  %
  \begin{align*}
    f'(z) = \sum\limits_{n=1}^{\infty} a_n \, n \, z^{n-1} \; , \quad
    f''(z) = \sum\limits_{n=2}^{\infty} a_n \, n \, (n-1) \, z^{n-2} \; , \quad \ldots
  \end{align*}
  
  \begin{proof} Seien
    %
    \begin{align*}
      g(z) \coloneq \sum\limits_{n=1}^{\infty} a_n \, n \, z^{n-1} \; , \quad R' \coloneq \frac{1}{\limsup\limits_{n \to \infty} \sqrt[n]{|a_n \, n|}}
    \end{align*}
    %
    dann gilt
    %
    \begin{enum-arab}
      \item $R' = R$ wegen $\lim\limits_{n \to \infty} \sqrt[n]{n} = 1$. 
        $g$ ist also auf $K_R(0)$ definiert.
      
      \item $g = f'$. Seien $w \in K_R(0)$, $z \in K_R(0) \setminus \{w\}$. Dann gilt
      %
      \begin{align*}
        \frac{f(z) - f(w)}{z - w} - g(w) = \sum\limits_{n = 1}^{\infty} a_n \underbrace{\left( \frac{z^n - w^n}{z - w} - n \, w^{n-1} \right)}_{{\hypertarget{eq:stern2}{{\color{DarkRed} (\ast)}}}}
      \end{align*}
      %
      sowie
      %
      \begin{align*}
          {\hyperlink{eq:stern2}{{\color{DarkRed} (\ast)}}} &= \left| \frac{1}{z - w} \int_{[w,z]} \left( n \, u^{n-1} - n \, w^{n-1} \right) \mathrm{d}u \right| \\
          &\leq \frac{1}{|z-w|} \, L([w,z]) \, n \, |z^{n-1} - w^{n-1}|
      \end{align*}
      %
      Insgesamt folgt
      \begin{align*}
        \begin{dcases}
          {\hyperlink{eq:stern2}{{\color{DarkRed} (\ast)}}} \to 0 \; , \quad \text{für } z \to w \\
          |a_n(\ldots)| \leq n \, |z|^{n-1} + n \, |w|^{n-1}
        \end{dcases}
      \end{align*}
      %
      Die Reihe konvergiert also gleichmäßig in $K_r(0)$ für jedes $r < R$. Der Grenzwert von $z \to w$ und die Reihe sind vertauschbar, also gilt
      %
      \begin{align*}
        \frac{f(z) - f(w)}{z - w} - g(w) \to 0.
      \end{align*}
    \end{enum-arab}
  \end{proof}
\end{theorem}

% % % Vorlesung vom 25.10.2012

\begin{example}
  \begin{enum-arab}
    \item
    %
    \hfill$\begin{aligned}[t]
      \mathrm{e}^z &= \sum\limits_{n=0}^{\infty} \frac{z^n}{n!} \; , \quad z \in \mathbb{C} \\
      (\mathrm{e}^z)' &= \sum\limits_{n=1}^{\infty} \frac{n z^{n-1}}{n!} \overset{k = n-1}{=}
        \sum\limits_{k=0}^{\infty} \frac{z^k}{k!}.
    \end{aligned}$\hfill\null
    
    \item
    %
    \hfill$\begin{aligned}[t]
      (\cos z)'
      &= \left( \sum\limits_{n=0}^{\infty} (-1)^n \frac{z^{2n}}{(2n)!} \right)' \\
      &= - \sum\limits_{n=1}^{\infty} (-1)^{n-1} \frac{z^{2n-1}}{(2n-1)!} \\
      &\overset{k=n-1}{\underset{\substack{n=k+1\\2n = 2k+2}}{=}}
      - \sum\limits_{k=0}^{\infty} (-1)^{k} \frac{z^{2k+1}}{(2k+1)!} \\
      &= - \sin z \; , \quad \text{für } z \in \mathbb{C}.
    \end{aligned}$\hfill\null
    
    \item Analog $(\sin z)' = \cos z$.
  \end{enum-arab}
\end{example}

\begin{theorem}[Definition]
  Sei $O \subseteq \mathbb{C}$ offen, $\gamma_1, \gamma_2 \in C^1([0,1] \to O)$. Dann heißen $\gamma_1$, $\gamma_2$ \acct{$C^1$-homotop} in $O$, falls es eine Abbildung $\Phi \in C^1([0,1] \times [0,1] \to O)$ gibt, sodass
  %
  \begin{align*}
    \Phi(\cdot,0) = \gamma_1 \; , \quad \Phi(\cdot,1) = \gamma_2
  \end{align*}
  %
  gilt und eine der folgenden Bedingungen erfüllt ist:
  %
  \begin{enum-roman}
    \item $\Phi(0,s) = \gamma_1(0)$, $\Phi(1,s) = \gamma_1(1)$ für $0 \leq s \leq 1$. Insbesondere folgt daraus, dass $\gamma_1(0) = \gamma_2(0)$ und $\gamma_1(1) = \gamma_2(1)$. Die Kurven stimmen also in den Anfangs- und Endpunkten überein.
    
    \item $\Phi(0,s) = \Phi(1,s)$ für $0 \leq s \leq 1$. $\gamma_1$ und $\gamma_2$ sind also geschlossen, ebenso die Wege $\beta_s \coloneq \Phi(\cdot,s)$.
  \end{enum-roman}
  %
  Wir schreiben $\gamma_1 \sim \gamma_2$ ($\sim$ ist eine Äquivalenzrelation). $\Phi$ heißt Homotopie zwischen $\gamma_1$ und $\gamma_2$. Ein geschlossener Weg $\gamma$ heißt \acct{nullhomotop}, falls $\gamma$ homotop zu einem konstanten Weg ist.
\end{theorem}

\begin{example}
  \begin{enum-arab}
    \item
    \begin{enum-alph}
      \item $O = \mathbb{C}$. Sind $\gamma_1,\gamma_2 \in C^1([0,1] \to \mathbb{C})$ mit $\gamma_1(0) = \gamma_2(0)$ und $\gamma_1(1) = \gamma_2(1)$, dann gilt $\gamma_1 \sim \gamma_2$.
        Betrachte dazu
      %
      \begin{align*}
        \Phi(t,s) \coloneq \gamma_1(t) + s \underbrace{(\gamma_2(t) - \gamma_1(t))}_{=0 \text{ für } t=0, t=1}.
      \end{align*}
      %
      \begin{figure}[H]
        \centering
        \begin{pspicture}(-1,-1)(2,2)
          \psaxes[labels=none,ticks=none]{->}(0,0)(-1,-1)(2,2)[\color{DimGray} Re,-90][\color{DimGray} Im,180]
          \cnode*(-0.5,-0.5){2pt}{A}
          \cnode*(1.5,1.5){2pt}{B}
          \ncarc[linecolor=DarkOrange3,arcangle=70,arrows=->]{A}{B}
          \naput{\color{DarkOrange3} $\gamma_1$}
          \ncarc[arcangle=40,arrows=->]{A}{B}
          \ncarc[arcangle=25,arrows=->]{A}{B}
          \ncarc[arcangle=10,arrows=->]{A}{B}
          \ncarc[linecolor=DarkOrange3,arcangle=-30,arrows=->]{A}{B}
          \ncput*{\color{DarkOrange3} $\gamma_2$}
          \rput{0}(0.4,0.6){
            \pscurve(0.1,0)(0.7,0.1)(0.7,-0.1)(1.4,0)
            \uput[0]{0}(1.5,0){\color{DimGray} Weg $t \mapsto \Phi(t,s)$ für $0 < s < 1$}
          }
        \end{pspicture}
      \end{figure}
      
      \item Jeder geschlossene Weg $\gamma \in C^1([0,1] \to \mathbb{C})$ ist in $O = \mathbb{C}$ nullhomotop: Sei dazu $\gamma_2(t) \coloneq z_0 \in \mathbb{C}$ ein konstanter Weg und setze
      %
      \begin{align*}
        \Phi(t,s) \coloneq \gamma(t) + s (z_0 - \gamma(t)).
      \end{align*}
      %
      \begin{figure}[H]
        \centering
        \begin{pspicture}(0,0)(4,1.5)
          \rput{-30}(0,0){
            \psdot*(0,0)
            \psccurve[linecolor=DarkOrange3](0,0)(0.5,0.5)(0.3,1)(0.5,1.5)(0,2)(-0.5,1.5)(-0.3,1)(-0.5,0.5)(0,0)
          }
          \rput{-30}(1.5,0.5){
            \psdot*(0,0)
            \psccurve[linecolor=DarkOrange3,unit=0.5cm](0,0)(0.5,0.5)(0.3,1)(0.5,1.5)(0,2)(-0.5,1.5)(-0.3,1)(-0.5,0.5)(0,0)
          }
          \rput{-30}(2.5,0.8333){
            \psdot*(0,0)
            \psccurve[linecolor=DarkOrange3,unit=0.3cm](0,0)(0.5,0.5)(0.3,1)(0.5,1.5)(0,2)(-0.5,1.5)(-0.3,1)(-0.5,0.5)(0,0)
          }
          \psline{o-o}(0,0)(3,1)
          \uput[-45](1.5,0.5){\color{DarkOrange3} $\Phi(\cdot,1/2)$}
          \uput[45](2.5,1.3){\color{DarkOrange3} $\Phi(\cdot,0.9)$}
          \uput[0](3,1){\color{DimGray} $z_0$}
        \end{pspicture}
      \end{figure}
    \end{enum-alph}
    
    \item Sei $O = \mathbb{C} \setminus \{ 0 \}$
    \begin{enum-alph}
      \item ~
      %
      \begin{figure}[H]
        \centering
        \begin{pspicture}(-1,-1)(2,2)
          \psaxes[labels=none,ticks=none]{->}(0,0)(-1,-1)(2,2)[\color{DimGray} Re,-90][\color{DimGray} Im,180]
          \cnode*(-0.5,-0.5){2pt}{A}
          \cnode*(1.5,1.5){2pt}{B}
          \ncarc[linecolor=DarkOrange3,arcangle=50,arrows=->]{A}{B}
          \naput{\color{DarkOrange3} $\gamma_1$}
          \ncarc[arcangle=25,arrows=->]{A}{B}
          \nbput{\color{DimGray} $\gamma_2$}
          \ncarc[linecolor=DarkOrange3,arcangle=-30,arrows=->]{A}{B}
          \nbput*{\color{DarkOrange3} $\gamma_3$}
          \psdot[linewidth=1pt,dotstyle=o](0,0)
        \end{pspicture}
      \end{figure}
      %
      Offensichtlich gilt $\gamma_1 \sim \gamma_2$, aber nicht $\gamma_1 \sim \gamma_3$ und auch nicht $\gamma_2 \sim \gamma_3$, weil „wir über die $0$ hinüber müssten“.
      
      Anschaulich: Zwei Wege sind homotop, wenn >>dazwischen<< nur Elemente aus $O$ liegen.
      
      \item ~
      %
      \begin{figure}[H]
        \centering
        \begin{pspicture}(-1.5,-1.5)(1.5,1.5)
          \psaxes[labels=none,ticks=none]{->}(0,0)(-1.5,-1.5)(1.5,1.5)[\color{DimGray} Re,-90][\color{DimGray} Im,180]
          \psccurve[linecolor=DarkOrange3](-1,0)(-0.3,-0.3)(0,-1)(0.4,-0.5)(1,0)(0.3,0.4)(0,1)(-0.3,0.5)
          \psdot[linewidth=1pt,dotstyle=o](0,0)
          \uput[0](0.3,0.6){\color{DarkOrange3} $\gamma$}
        \end{pspicture}
      \end{figure}
      %
      $\gamma$ ist nicht nullhomotop, aber $\gamma \sim \widetilde{\gamma} : t \mapsto \mathrm{e}^{\mathrm{i} \, 2\pi \, t}$, $0 \leq t \leq 1$.
      
      \item ~
      %
      \begin{figure}[H]
        \centering
        \begin{pspicture}(-1.5,-1.5)(1.5,1.5)
          \psaxes[labels=none,ticks=none]{->}(0,0)(-1.5,-1.5)(1.5,1.5)[\color{DimGray} Re,-90][\color{DimGray} Im,180]
          \psccurve[linecolor=DarkOrange3](-0.5,0)(0,-0.4)(0.5,0)(0,0.8)(-1,0)(0,-0.8)(0.5,0)(0,0.4)
          \psdot[linewidth=1pt,dotstyle=o](0,0)
          \uput[0](0.3,0.6){\color{DarkOrange3} $\gamma$}
        \end{pspicture}
      \end{figure}
      
      $\gamma \sim \hat{\gamma} : t \mapsto \mathrm{e}^{\mathrm{i} \, 4\pi \, t}$, $0 \leq t \leq 1$. Zweimal durchlaufener Kreis.
    \end{enum-alph}
  \end{enum-arab}
\end{example}
