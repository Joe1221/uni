% Henri Menke, 2012 Universität Stuttgart.
%
% Dieses Werk ist unter einer Creative Commons Lizenz vom Typ
% Namensnennung - Nicht-kommerziell - Weitergabe unter gleichen Bedingungen 3.0 Deutschland
% zugänglich. Um eine Kopie dieser Lizenz einzusehen, konsultieren Sie
% http://creativecommons.org/licenses/by-nc-sa/3.0/de/ oder wenden Sie sich
% brieflich an Creative Commons, 444 Castro Street, Suite 900, Mountain View,
% California, 94041, USA.

\section{Differentialformen}
\addtocounter{thmn}{1}
\setcounter{theorem}{0}

% % % Vorlesung vom 21.01.2013

\begin{theorem}[Vereinbarung] \label{thm:12.1}
  Im Folgenden immer: $S$ ist eine kompakte orientierbare $k$-dimensionale $C^1$-Mannigfaltigkeit mit orientiertem Atlas $A(S) = \{ (\varphi_j,U_j) , 1 \leq j \leq N \}$ und passender Zerlegung der Eins $\{ \psi_1,\ldots,\psi_N \}$.
\end{theorem}

\begin{theorem}[Arbeit] \label{thm:12.2}
  Sei $n=3$, $k=1$, $F \in C(S \to \mathbb{R}^3)$. Die \acct{Arbeit} längs $S$ im Kraftfeld $F$ ist gegeben durch
  %
  \begin{align*}
    A &\coloneq \int_S \braket{F,t_0} \mathrm{d}V^{(1)}
  \intertext{wobei $t_0$ der Tangenteneinheitsvektor ist.}
    &= \sum\limits_{j=1}^{N} \int_{U_j} \psi_j \braket{F,t_0} \mathrm{d}V^{(1)} \\
    &= \sum\limits_{j=1}^{N}
    \int\limits_{\substack{y \in ]-1,1[ \\ \lor y \in ]-1,0] \\ \lor y \in [0,1[}}
    \psi_j(\varphi_j(y))
    %\Braket{F(\varphi_j(y)), \frac{\varphi_j'(y)}{\| \varphi_j'(y) \|}}
    \left\langle F(\varphi_j(y)), \frac{\varphi_j'(y)}{\| \varphi_j'(y) \|} \right\rangle
    \left[ \det \left( \left( \varphi_j'(y) \right)^* \left(  \varphi_j'(y) \right) \right) \right]^{1/2} \, \mathrm{d}y \\
    &= \sum\limits_{j=1}^{N} \int\limits_{\ldots} \psi_j(\varphi_j(y)) \left( \sum\limits_{k=1}^{N} F_k(\varphi_j(y)) \varphi_{jk}'(y) \right) \, \mathrm{d}y
  \end{align*}
\end{theorem}

\begin{theorem}[Fluss] \label{thm:12.3}
  Sei $n=3$, $k=2$, $V \in C(S \to \mathbb{R}^3)$. Der \acct{Fluss} von $V$ durch $S$ ist definiert durch
  
  \begin{align*}
    F &\coloneq \int_S \braket{V,n_0} \, \mathrm{d}V^{(k)}
  \end{align*}
  %
  wobei $n_0$ der Normaleneinheitsvektor ist.
  %
  \begin{align*}
    = \sum\limits_{j=1}^{N}
    \int\limits_{\substack{y \in K_1^{(2)}(0) \\ \lor y \in K_1^{(2)+}(0)}}
    \psi_j(\varphi_j(y)) \Bigg(
      & V_1(\varphi_j(y)) \cdot \det\left( \frac{\partial(\varphi_{j_2},\varphi_{j_3})}{\partial y} \right) \\
      &+ V_2(\varphi_j(y)) \cdot \det\left( \frac{\partial(\varphi_{j_3},\varphi_{j_1})}{\partial y} \right) \\
      &+ V_3(\varphi_j(y)) \cdot \det\left( \frac{\partial(\varphi_{j_1},\varphi_{j_2})}{\partial y} \right)
    \Bigg)
    \, \mathrm{d}y
  \end{align*}
\end{theorem}

\begin{theorem}[Definition] \label{thm:12.4}
  Sie $O \subseteq \mathbb{R}^n$ offen. Eine \acct{Differentialform} in $O$ oder $k$-Form in $O$ ist eine Abbildung
  %
  \begin{align*}
    \omega : \left\{
      \begin{gathered}
        \text{$k$-dim. kompakten orientierbaren} \\
        \text{$C^1$-Mannigfaltigkeiten in $O$}
      \end{gathered}
    \right\}
    \to \mathbb{R}
  \end{align*}
  %
  symbolisch gegeben durch
  %
  \begin{align*}
    \omega = \sum\limits_{i_1 = 1}^{n} \ldots \sum\limits_{i_k = 1}^{n} a_{i_1 ,\ldots, i_k} (x) \, \mathrm{d}x_{i_1} \wedge \mathrm{d}x_{i_2} \wedge \ldots \wedge \mathrm{d}x_{i_k}
  \end{align*}
  %
  $a_{i_1 ,\ldots, i_k} \in C(S \to \mathbb{R})$ mit
  %
  \begin{align*}
    \omega(S) \coloneq \int_S \omega \coloneq \sum\limits_{j=1}^{N} \int_{D_j} \left( \sum\limits_{i_1 = 1}^{n} \ldots \sum\limits_{i_k = 1}^{n} (\psi_j \cdot a_{i_1 ,\ldots, i_k})(\varphi_j(y)) \right) \cdot \det\left( \frac{\partial (\varphi_{ji_1},\ldots,\varphi_{ji_k})}{\partial y} \right) \, \mathrm{d}y
  \end{align*}
  %
  $D_j = K_1^{(k)}(0)$ oder $\varphi_j = \left(\begin{smallmatrix} \varphi_{j_1} \\ \vdots \\ \varphi_{j_k} \end{smallmatrix}\right)$. Eine $0$-Form in $O$ ist gegeben durch $a \in C(O \to \mathbb{R})$ und
  %
  \begin{align*}
    \omega(S) \coloneq \sum\limits_{j=1}^{N} a(x_j) \eqcolon \int_S \omega
  \end{align*}
  %
  $\omega$ ist von der Klasse $C^m$, falls $a_{i_1 ,\ldots, i_k} \in C^m(O \to \mathbb{R})$. Das bedeutet, $\omega$ ist durch Vorgabe der $a_{i_1 ,\ldots, i_k}$ definiert.
\end{theorem}

\begin{notice} \label{thm:12.5}
  Die Definition von $\int_S \omega$ ist unabhängig vom Atlas und der Zerlegung der Eins, falls der neue Atlas gleich orientiert wie der ursprüngliche ist.
\end{notice}

\begin{example} \label{thm:12.6}
  Seien$n=3$, $k=2$, $O \subseteq R^3$, $V \in C(O \to R^3)$.
  %
  \begin{align*}
    \omega \coloneq V_1 \mathrm{d}x_2 \wedge \mathrm{d}x_3 + V_2 \mathrm{d}x_3 \wedge \mathrm{d}x_1 + V_3 \mathrm{d}x_1 \wedge \mathrm{d}x_2 \implies \omega(S)=\text{Fluss von V durch S}
  \end{align*}
  %
\end{example}

\begin{notice}[Eigenschaften:] \label{thm:12.7}
  Sei $k \geq 2$.
  %
  \begin{enum-arab}
    \item Falls $\omega = a(x) \, \mathrm{d}x_{i_1} \wedge \ldots \wedge \mathrm{d}x_{i_k}$ und $i_j = i_\ell$ für ein Paar $(j,\ell)$ mit $j \neq \ell$. Dann ist $\omega = 0$. Insbesondere $\mathrm{d}x_i \wedge \mathrm{d}x_i = 0$.
    
    \item
    \begin{gather*}
      \mathrm{d}x_{i_1} \wedge \ldots \wedge \mathrm{d}x_{i_j} \wedge \ldots \wedge \mathrm{d}x_{i_\ell} \wedge \ldots \wedge \mathrm{d}x_{i_k} \\
      \mathrm{d}x_{i_1} \wedge \ldots \wedge \mathrm{d}x_{i_\ell} \wedge \ldots \wedge \mathrm{d}x_{i_j} \wedge \ldots \wedge \mathrm{d}x_{i_k} \\
    \end{gather*}
  \end{enum-arab}
  
  \begin{proof}
    \begin{enum-arab}
      \item Sieht man direkt aus
      %
      \begin{align*}
        \det \left( \frac{\partial (\varphi_{i_1} ,\ldots, \varphi_{i_k})}{\partial y} \right) = 0
      \end{align*}
      %
      wobei die Spalten $i_j$ und $i_\ell$ gleich sind.
      
      \item Folgt aus: Vertauscht man in der Matrix $A$ zwei Spalten, so wird $\det A$ mit $-1$ multipliziert.
    \end{enum-arab}
  \end{proof}
\end{notice}

\begin{notice}[Lineare Struktur:] \label{thm:12.8}
  Für $k$-Formen $\omega_1,\omega_2$ und $c_1, c_2 \in \mathbb{R}$ ist
  %
  \begin{align*}
    (c_1 \omega_1 + c_2 \omega_2)(S)
    &= \int_S (c_1 \omega_1 + c_2 \omega_2) \\
    &\coloneq c_1 \int_S \omega_1 + c_2 \int_S \omega_2 = c_1 \omega_1(S) + c_2 \omega_2(S)
  \end{align*}
\end{notice}

\begin{example} \label{thm:12.9}
  $\omega = 1 \cdot \mathrm{d}x_1 \wedge \mathrm{d}x_2 + 1 \cdot \mathrm{d}x_2 \wedge \mathrm{d}x_1$ $\implies$ $\omega = 0$.
\end{example}

\begin{theorem}[Definition] \label{thm:12.10}
  Es sei $I = (i_1\ldots,i_k)$ mit $1 \leq i_1 < i_2 < \ldots < i_k \leq n$. Dann heißt $I$ \acct{wachsender Index}. Wir setzen 
  %
  \begin{align*}
    \mathcal{G} &\coloneq \{ I : I \text{ wachender Index der Länge } k \} \\
    \mathrm{d}x_I &\coloneq \mathrm{d}x_{i_1} \wedge \ldots \wedge \mathrm{d}x_{i_k} \; , \quad \text{mit } I = (i_1,\ldots,i_k)
  \end{align*}
  %
  $\mathrm{d}x_I$ mit $I \in \mathcal{G}^{(k)}$ heißt \acct{$k$-Grundform} im $\mathbb{R}^n$.
\end{theorem}

\begin{theorem}[Satz] \label{thm:12.11}
  \begin{enum-arab}
    \item Ist \[ \omega = \sum\limits_{I \in \mathcal{G}^{(k)}} a_I(x) \, \mathrm{d}x_I = 0 \] so folgt $a_I = 0$ für $I \in \mathcal{G}^{(k)}$.
    
    \item Jede $k$-Form $\omega$ in $O$ besitzt eine eindeutige Darstellung
    %
    \begin{align*}
      \omega = \sum\limits_{I \in \mathcal{G}^{(k)}} a_I \, \mathrm{d}x_I
    \end{align*}
    %
    d.h. $(a_I : I \in \mathcal{G}^{(k)})$ sind die $\binom{n}{k}$ Koordinaten von $\omega$. Die Abbildung
    %
    \begin{align*}
      \omega \mapsto (a_I : I \in \mathcal{G}^{(k)}) \text{ mit } \omega = \sum a_I \, \mathrm{d}x_I
    \end{align*}
    %
    ist bijektiv und linear.
  \end{enum-arab}
  
  \begin{proof}
    \begin{enum-arab}
      \item Ann: $\omega = 0$, aber $a_{I_0} \neq 0$ für mindestens ein $I_0 \in \mathcal{G}^{(k)}$. Sei $x_0 \in O$ mit $a_{I_0} (x_0) \neq 0$, o.B.d.A. $a_{I_0}(x_0) > 0$. Wähle $\delta > 0$ mit
      %
      \begin{align*}
        a_{I_0} (x) \geq \frac{1}{2} a_{I_0}(x_0) > 0 \text{ für } x \in K_\delta(x_0), \overline{K_\delta(x_0)} \subseteq O
      \end{align*}
      %
      Wähle $D \coloneq \overline{K_1^{(k)}(0)}$.
      %
      \begin{align*}
        \varphi(y) \coloneq x_0 + \delta \sum\limits_{\ell=1}^{k} y_\ell e_{j_\ell}
      \end{align*}
      %
      wobei $I_0 = \{ j_1,\ldots,j_k \}$ und $e_{j_k} = (0,\ldots,1,\ldots,0)^\top$. Dann ist $\varphi(D)$ eine kompakte orientierbare $C^\infty$-Mannigfaltigkeit, $\dim k$
      %
      \begin{align*}
        \implies \omega(\varphi(D)) \overset{\text{\ref{thm:11.7}}}{=} \int_D \sum\limits_{I \in \mathcal{G}^{(k)}} a_I(\varphi(y)) \det\left( \frac{\partial \varphi_I}{\partial y} \right) \, \mathrm{d}y
      \end{align*}
      %
      Sei $I \in \mathcal{G}^{(k)}$, $I \neq I_0$, $I = (i_1,\ldots,i_k)$
      %
      \begin{gather*}
        \implies \det\left( \frac{\partial \varphi_I}{\partial y} \right) = \det\left( \frac{\partial (\varphi_{i_1} ,\ldots,\varphi_{i_k})}{\partial y} \right) \\
        \varphi(y) = x_0 + \delta(y_1 e_{j_1} + y_2 e_{j_2} + \ldots + y_k e_{j_k}) \\
        \implies \frac{\partial \varphi}{\partial y_\ell} = \delta e_{j_\ell} \; , \quad \text{insbesondere }
        \frac{\partial \varphi}{\partial y_\ell} =
        \begin{cases}
          \delta & i = j_\ell \\
          0 & \text{sonst}
        \end{cases} \\
        \implies \det
        \begin{pmatrix}
          \partial_{y_1} \varphi_{i_1} & \ldots & \partial_{y_k} \varphi_{i_1} \\
          \vdots & & \vdots \\
          \partial_{y_1} \varphi_{i_k} & \ldots & \partial_{y_k} \varphi_{i_k} \\
        \end{pmatrix} = 0
      \end{gather*}
      %
      Außerdem
      %
      \begin{gather*}
        \det \left( \frac{\partial \varphi_{I_0}}{\partial y} \right) =
        \det
        \begin{pmatrix}
          \delta & 0 & \ldots & 0 \\
          0 & \delta & \ldots & 0 \\
          \vdots & \ddots & & \vdots \\
          0 & \ldots & 0 & \delta \\
        \end{pmatrix}
        = \delta^n \\
        \implies \omega(\varphi(D)) = 0 + \ldots + 0 + \int_D a_{I_0}(\varphi(y)) \cdot \delta^n \, \mathrm{d}y \geq \frac{1}{2} a_{I_0}(x_0) \delta^n \int_D 1 \, \mathrm{d}y > 0
      \end{gather*}
      %
      $\lightning$ $\omega(S) = 0$ für alle $S$.
      
      \item Die Eindeutigkeit folgt aus 1.). Existenz: Forme alle Summanden $\mathrm{d}x_{i_1} ,\ldots, \mathrm{d}x_{i_k} = (-1)^{} \mathrm{d}x_I$, $I \in \mathcal{G}^{(k)}$ um und fasse gleiche $\mathrm{d}x_I$ zusammen.
    \end{enum-arab}
  \end{proof}
\end{theorem}

%================================
%=== Beginn - Mitschrieb Stephan
%================================

\begin{proof}
	\begin{enumerate}[1)]
		\item
			Es gelte o.B.d.A
			\[
				\omega = a(x) dx_J \qquad J \in \scr G^{(k)}
			\]
		\item
			Lokalisierung:
			Sei $A(S)$ orientierter Atlas, sortiert wie in \ref{8.6}.
			\begin{align*}
				\int_S d\omega &= \sum_{j=1}^N \int_{U_j} \psi_j d\omega \\
				&= \sum_{i=1}^n \sum_{j=1}^N \int_{U_j}\d_{x_i}(\psi_j a) dx_i \wedge dx_J \\
				&= \sum_{i=1}^n \underbrace{\sum_{j=1}^N \int_{U_j}(\d_{x_i}\psi_j) a xx_i \wedge dx_J}_{\int_S (\underbrace{\sum_{j=1}^N \d_{x_i}\psi_j)}_{\d x_i \sum \psi_j = \d x_{i}1 = 0}a dx_i \wedge dx_J} \\
				&= \sum_{j=1}^N \int_{U_j}d(\psi_j \omega)
			\end{align*}
			Wir zeigen jetzt
			\[
				\int_{U_j} d(\psi_j \omega) = \begin{cases}
					\int_{\tilde U_j = U_j \cap \d S} \psi_j \omega & 1 \le j \le L \\
					0& L+1 \le j \le N
				\end{cases}
			\]
			Dann gilt, da $\{\psi_1,\dotsc, \psi_L\}$ eine Zerlegung der Eins ist für $\d S$ mit Altas $\{(\tilde \phi_j, \tilde U_j), 1 \le j \le L \}$
			\[
				\int_S d\omega = \sum_{j=1}^L \int_{\tilde U_j} \psi_j \omega  = \int_{\d S} \omega
			\]
		\item
			Sei $j= \{1,\dotsc, N\}$ fest, $\phi_j = (g_1,\dotsc, g_n)$, $D=K_1^{(k+1)}(0)$ falls $1\le j \le L$, $D=K_1^{(k+1)}(0)$ sonst.
			Also
			\begin{align*}
				\int_{U_j} d(\psi_j \omega) &= \sum_{i=1}^n \int_{y\in D} \d_{x_i} (\psi_j a)(\phi_j(y))) \det(\tf{\d(g_i,g_J)}{\d y}) dy 
				\intertext{
					Entwickeln der Determinante nach der ersten Zeile ergibt
				}
				&= \sum_{i=1}^n \sum_{l=1}^{k+1} (-1)^{k+1} \int_{y\in D} \d_{x_i}(\psi_j a)(\phi_j(y)) \tf{\d_{g_i}}{\d_{y_l}}(y) \cdot \det \bigg( \f{\d(g_{i_1},\dotsc,g_{i_k})}{\underbrace{\d(y_1,\dotsc, y_{l-1},y_{l+1},\dotsc, y_{k+1})}_{y^{(l)}}} dy \\
				&= \sum_{l=1}^{k+1} (-1)^{l+1} \int_{y \in D} \d_{y_l}((\psi_j a)\circ \phi_j)(y) \det(\dotsc) dy \\
				&= \sum_{l=1}^{k+1} (-1)^{l+1} \int_{y^{(l)}\in K_1^{(k)}(0)} \int_{y_l=- \sqrt{1 - |y^{(l)}|^2}}^{y_l = + \sqrt{1-|y^{(l)}|^2} \text{ oder $y_l=0$ falls $1\le j \le L$}} \dotsc d y_l dy^{(l)}
				\intertext{
					Integriere jetzt partiell und beachte: $(\psi_j a)\circ \psi = 0$ für $y_l = \pm \sqrt{1-|y^{(l)}|^2}$ da $\supp \psi_j \le U_j$.
					Außerdem $\sum_{l=1}^{k+1} (-1)^{l+1} \d y_l (\det (\dotsc)) = 0$ (Nachrechnen).
				}
				&= \begin{cases}
					0 & L+1 \le j \le N \\
					\int_{\tilde U_j} \psi \omega \stackrel{\tilde U_j = \tilde \phi_j(K_1^{(k)}(0))}= (-1)^{i+1} \int_{y^{(i)}\in K_1^{(k)}(0)} (\psi_j a) \circ \phi_j (0,y^{(1)}) \det(\tf{\d g_J}{\d y^{(1)}}) dy^{(1)} & 
				\end{cases}
			\end{align*}
	\end{enumerate}
\end{proof}

\begin{nt*} \label{8.8}
	\begin{enumerate}[1)]
		\item
			Im Fall $k=0, \omega = a(x), S = \phi([\alpha,\beta]), \d S = \{\phi(\alpha), \phi(\beta)\}$.
			Nach \ref{8.4}:
		\item
			Satz von Stokes gilt auch für $S= \_{O}$, d.h. $S \subset O$ nicht notwendig.
		\item
			Folgerungen
			\begin{enumerate}[a)]
				\item
					Der Satz von Gauß-Ostrogradski:
					Sei $f\in C^1(O\to \R^n)$.
					\[
						\int_{S} \nabla \cdot f dV^{(n)} = \int_{\d S} \<f,n_0\> dV^{(n-1)}
					\]
					Setze dazu $\omega := \sum_{j=1}^n (f_j dx_1 \wedge \dotsc \wedge dx_{j-1} \wedge dx_{j+1} \wedge \dotsc \wedge dx_n)$
				\item
					Klassischer Satz von Stokes:
					Sei $S \subset \R^3$ eine Mannigfaltigkeit der Dimension $2$ und $f \in C^1(S \to \R^3)$, dann gilt
					\[
						\int_S (\nabla \times f) dV^{(2)} = \int_{\d S}\<f,t_0\> dV^{(1)}
					\]
					Setze dazu $\omega := f_1 dx_1 + f_2 dx_2 + f_3 dx_3$.
			\end{enumerate}
	\end{enumerate}
\end{nt*}
