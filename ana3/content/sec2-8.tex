% Stephan Hilb, 2012 Universität Stuttgart.
%
% Dieses Werk ist unter einer Creative Commons Lizenz vom Typ
% Namensnennung - Nicht-kommerziell - Weitergabe unter gleichen Bedingungen 3.0 Deutschland
% zugänglich. Um eine Kopie dieser Lizenz einzusehen, konsultieren Sie
% http://creativecommons.org/licenses/by-nc-sa/3.0/de/ oder wenden Sie sich
% brieflich an Creative Commons, 444 Castro Street, Suite 900, Mountain View,
% California, 94041, USA.

\section{Rechnen mit Differentialformen}
\addtocounter{thmn}{1}
\setcounter{theorem}{0}

\begin{enum-arab}
\item
\item
% Hier fehlt die Vorlesung vom 24.1.

Daraus folgt:
\begin{align*}
d(\omega_1 \land \omega_2)&=d(a(x) b(x))\wedge dx_J \wedge dx_J\\
&= \sum_{i=1}^n(\delta_{x_i}(a\cdot b) dx_i \wedge dx_J \wedge d_J\\
&= b\sum_{i=1}^n(\delta_{x_i} a) dx_i \wedge dx_J \wedge dx_J + a \sum_{i=1}^n (\delta_{x_i} b)\, \underbrace{dx_i \wedge dx_J}_{=(-1)^k dx_J \wedge dx_i}  \wedge dx_J \\
&\stackrel{\text{Def./1-Produkt}}=(\sum_{i=1}^n(\delta_{x_i}b) dx_i \wedge dx_J)\wedge bdx_J + (-1)^k a dx_J \wedge(\sum_{i=1}^n(\delta_{x_i} b) dx_i \wedge dx_J \\
&= d\omega_1 \wedge \omega_2+(-1)^k \omega_1 \wedge d\omega_2
\end{align*}
\item 
\begin{enum-alph}
\item Fall: $\omega=f\in C^2(O\to \R)$ ist $O$-Form. Dann folgt $d\omega=\sum_{i=1}^n (\delta_{x_i} f)dx_i$
\begin{align*}
d(d\omega)&=\sum_{i=1}^n (\delta_{x_i} f)\wedge dx_i\\
&= \sum_{i=1}^n \sum_{j=1}^n \underbrace{\delta_{x_j}(\delta_{x_j}f)}_{=\delta_{x_i}(\delta_{x_j} f)} dx_j  \wedge dx_i = 0
\end{align*}
Hierbei kommt jeder Summand ein Mal mit plus und ein Mal mit minus vor. Beachte, dass
\[
dx_j\wedge dx_i=\begin{cases} 0 &, \text{ falls } i=j\\ -dx_i\wedge dx_j &,\text{ falls } i\neq j\end{cases}
\]
\item Fall: O.B.d.A. $\omega=a_J dx_J, J \in G^{(k)} \implies \omega=a_j\wedge dx_J$
dann folgt mit der Produktregel
\[
d\omega=da_J \wedge dx_J+(-1)^0 a_J \underbrace{d(dx_J)}_{=0}
\]
dies gilt offensichtlich nach Definition, denn
\[
d(dx_J)=d(1\cdot dx_J)\stackrel{\text{Def.}}=\sum_{i=1}^n(\delta_{x_i} 1) dx_i \wedge dx_J
\]
Nochmaliges Anwenden der Produktregel unter Betrachtung von Fall a) und Vorherigem führt zu:
\[
d^2\omega=\underbrace{d^2 a_J}_{=0 \text{Fall a)}} \wedge dx_J+(-1)^1 da_J \wedge \underbrace{d^2x_J}{=0 \text{vorh.}}=0
\]
\end{enum-alph}
\end{enum-arab}
\begin{theorem}[Definition] \label{thm:8.6}
Sei $S$ $k$-dimensionale $C^1$-Mannigfaltigkeit ($k\ge 2$) mit Rand und orientiertem Atlas $A(S)=\{(\phi_1, U_1), ..., (\phi_N, U_N)\}$. Im Unterschied zu \ref{thm:4.4} sollen als Definitionsbereiche für die $\phi_j$. zugelassen sein: $K_1^(k)(0)$ oder $ K_1^{(k)-}(0):=\{y\in K_1^{(k)}(0):y_1\le 0 \}$. Sei $A(S)$ so sortiert, dass $\phi_1,..., \phi_L$ auf $K_1^{(k)-}(0)$ definiert sind und $\phi_{L+1},..., \phi_N$ auf $K_1^{(k)}(0)$.
Dann ist 
\[
A(\delta S):=\{(\tilde \phi_1, \tilde U_1),...,(\tilde \phi_L, \tilde U_L)\} 
\]
mit $\tilde \phi_j(y_1,..., y_{n-1}):=\phi_j(0, \phi_1,..., \phi_{k-1})$ und $\tilde U_j:= U_j \cap \delta S=\tilde \phi_j(K_1^{(k-1)}(0))$ ein orientierter Atlas von $\delta S$. Due si definierte Orientierung heißt  \acct{verträglich} zur Orientierung von $S$.
\end{theorem}
\subsection{Vektoranalysis}


\begin{proof}
  \begin{enum-arab}
    \item
      Es gelte o.B.d.A
      \[
        \omega = a(x) dx_J \qquad J \in \mathcal{G}^{(k)}
      \]
    \item
      Lokalisierung:
      Sei $A(S)$ orientierter Atlas, sortiert wie in \ref{8.6}.
      \begin{align*}
        \int_S d\omega &= \sum_{j=1}^N \int_{U_j} \psi_j d\omega \\
        &= \sum_{i=1}^n \sum_{j=1}^N \int_{U_j}\partial_{x_i}(\psi_j a) dx_i \wedge dx_J \\
        &= \sum_{i=1}^n \underbrace{\sum_{j=1}^N \int_{U_j}(\partial_{x_i}\psi_j) a xx_i \wedge dx_J}_{\int_S (\underbrace{\sum_{j=1}^N \partial_{x_i}\psi_j)}_{\partial x_i \sum \psi_j = \partial x_{i}1 = 0}a dx_i \wedge dx_J} \\
        &= \sum_{j=1}^N \int_{U_j}d(\psi_j \omega)
      \end{align*}
      Wir zeigen jetzt
      \[
        \int_{U_j} d(\psi_j \omega) = \begin{cases}
          \int_{\tilde U_j = U_j \cap \partial S} \psi_j \omega & 1 \le j \le L \\
          0& L+1 \le j \le N
        \end{cases}
      \]
      Dann gilt, da $\{\psi_1,\dotsc, \psi_L\}$ eine Zerlegung der Eins ist für $\partial S$ mit Altas $\{(\tilde \phi_j, \tilde U_j), 1 \le j \le L \}$
      \[
        \int_S d\omega = \sum_{j=1}^L \int_{\tilde U_j} \psi_j \omega  = \int_{\partial S} \omega
      \]
    \item
      Sei $j= \{1,\dotsc, N\}$ fest, $\phi_j = (g_1,\dotsc, g_n)$, $D=K_1^{(k+1)}(0)$ falls $1\le j \le L$, $D=K_1^{(k+1)}(0)$ sonst.
      Also
      \begin{align*}
        \int_{U_j} d(\psi_j \omega) &= \sum_{i=1}^n \int_{y\in D} \partial_{x_i} (\psi_j a)(\phi_j(y))) \det(\tfrac{\partial(g_i,g_J)}{\partial y}) dy 
        \intertext{
          Entwickeln der Determinante nach der ersten Zeile ergibt
        }
        &= \sum_{i=1}^n \sum_{l=1}^{k+1} (-1)^{k+1} \int_{y\in D} \partial_{x_i}(\psi_j a)(\phi_j(y)) \tfrac{\partial_{g_i}}{\partial_{y_l}}(y) \cdot \det \bigg( \frac{\partial(g_{i_1},\dotsc,g_{i_k})}{\underbrace{\partial(y_1,\dotsc, y_{l-1},y_{l+1},\dotsc, y_{k+1})}_{y^{(l)}}} dy \\
        &= \sum_{l=1}^{k+1} (-1)^{l+1} \int_{y \in D} \partial_{y_l}((\psi_j a)\circ \phi_j)(y) \det(\dotsc) dy \\
        &= \sum_{l=1}^{k+1} (-1)^{l+1} \int_{y^{(l)}\in K_1^{(k)}(0)} \int_{y_l=- \sqrt{1 - |y^{(l)}|^2}}^{y_l = + \sqrt{1-|y^{(l)}|^2} \text{ oder $y_l=0$ falls $1\le j \le L$}} \dotsc d y_l dy^{(l)}
        \intertext{
          Integriere jetzt partiell und beachte: $(\psi_j a)\circ \psi = 0$ für $y_l = \pm \sqrt{1-|y^{(l)}|^2}$ da $\mathrm{supp} \psi_j \le U_j$.
          Außerdem $\sum_{l=1}^{k+1} (-1)^{l+1} \partial y_l (\det (\dotsc)) = 0$ (Nachrechnen).
        }
        &= \begin{cases}
          0 & L+1 \le j \le N \\
          \int_{\tilde U_j} \psi \omega \stackrel{\tilde U_j = \tilde \phi_j(K_1^{(k)}(0))}= (-1)^{i+1} \int_{y^{(i)}\in K_1^{(k)}(0)} (\psi_j a) \circ \phi_j (0,y^{(1)}) \det(\tfrac{\partial g_J}{\partial y^{(1)}}) dy^{(1)} & 
        \end{cases}
      \end{align*}
  \end{enum-arab}
\end{proof}

\begin{notice} \label{8.8}
  \begin{enum-arab}
    \item
      Im Fall $k=0, \omega = a(x), S = \phi([\alpha,\beta]), \partial S = \{\phi(\alpha), \phi(\beta)\}$.
      Nach \ref{8.4}:
    \item
      Satz von Stokes gilt auch für $S= \_{O}$, d.h. $S \subset O$ nicht notwendig.
    \item
      Folgerungen
      \begin{enum-alph}
        \item
          Der Satz von Gauß-Ostrogradski:
          Sei $f\in C^1(O\to \mathbb{R}^n)$.
          \[
            \int_{S} \nabla \cdot f dV^{(n)} = \int_{\partial S} \braket{f,n_0} dV^{(n-1)}
          \]
          Setze dazu $\omega := \sum_{j=1}^n (f_j dx_1 \wedge \dotsc \wedge dx_{j-1} \wedge dx_{j+1} \wedge \dotsc \wedge dx_n)$
        \item
          Klassischer Satz von Stokes:
          Sei $S \subset \mathbb{R}^3$ eine Mannigfaltigkeit der Dimension $2$ und $f \in C^1(S \to \mathbb{R}^3)$, dann gilt
          \[
            \int_S (\nabla \times f) dV^{(2)} = \int_{\partial S}\braket{f,t_0} dV^{(1)}
          \]
          Setze dazu $\omega := f_1 dx_1 + f_2 dx_2 + f_3 dx_3$.
      \end{enum-alph}
  \end{enum-arab}
\end{notice}


