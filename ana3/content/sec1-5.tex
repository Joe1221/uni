% Henri Menke, 2012 Universität Stuttgart.
%
% Dieses Werk ist unter einer Creative Commons Lizenz vom Typ
% Namensnennung - Nicht-kommerziell - Weitergabe unter gleichen Bedingungen 3.0 Deutschland
% zugänglich. Um eine Kopie dieser Lizenz einzusehen, konsultieren Sie
% http://creativecommons.org/licenses/by-nc-sa/3.0/de/ oder wenden Sie sich
% brieflich an Creative Commons, 444 Castro Street, Suite 900, Mountain View,
% California, 94041, USA.

\section{Analytische Fortsetzung}
\addtocounter{thmn}{1}
\setcounter{theorem}{0}

% % % Vorlesung vom 29.11.2012

\begin{example}
  Betrachte
  %
  \begin{align*}
    f(z) = \sum\limits_{n=0}^{\infty} (-z^2)^n \quad \text{ für } |z| < 1
  \end{align*}
  %
  und $f(z) = \dfrac{1}{1 + z^2}$ (geometrische Reihe). Sei
  %
  \begin{align*}
    g(z) \coloneq \frac{1}{1 + z^2} \quad \text{ für } z \neq \pm \mathrm{i}
  \end{align*}
  %
  \begin{figure}[H]
    \centering
    \begin{pspicture}(-3.5,-2)(2,2)
      \psaxes[labels=none,ticks=none]{->}(0,0)(-3.5,-2)(2,2)[\color{DimGray} Re,0][\color{DimGray} Im,0]
      \pscircle[linecolor=DarkOrange3,fillstyle=hlines,hatchcolor=DarkOrange3](0,0){1}
      \pscircle[linecolor=MidnightBlue,fillstyle=hlines,hatchcolor=MidnightBlue](-0.5,0){1.11803}
      \pscircle[linecolor=DarkRed,fillstyle=hlines,hatchcolor=DarkRed](-1.5,0){1.8027756}
      \uput{1.2}[45](0,0){\color{DarkOrange3} $K_1(0)$}
      \uput{1.3}[-45](-0.5,0){\color{MidnightBlue} $K_{\sqrt{5}/2}\left( -\dfrac{1}{2} \right)$}
      \uput{2}[135](-1.5,0){\color{DarkRed} $K_{\sqrt{13}/2}\left( -\dfrac{3}{2} \right)$}
      \psdots*[linecolor=Purple](0,1)(0,-1)
    \end{pspicture}
  \end{figure}
  %
  Entwickle $f$ um $z = -1/2$ in eine Potenzreihe
  %
  \begin{align*}
    f_1(z) = \sum\limits_{n=0}^{\infty} a_n (z-z_1)^n
  \end{align*}
  %
  Wir wissen: $f = g$ in $K_1(0)$, also ist $f_1$ gleichzeitig Entwicklung von $g$.
  
  $\implies$ $f_1$ hat den Konvergenzradius $r_1 = \dfrac{\sqrt{5}}{2}$.
  
  Entwickle $f_1$ um $z_2 = -\dfrac{3}{2}$:
  %
  \begin{align*}
    f_2(z) = \sum\limits_{n=0}^{\infty} b_n (z-z_2)^n \quad \text{ in } K_{r_2}(z_2)
  \end{align*}
  %
  mit $r_2 = \dfrac{\sqrt{13}}{2}$ (da $f_2$ Entwicklung von $g$ ist). Wir haben $f$, das nur auf $K_1(0)$ definiert ist holomorph auf $K_1(0) \cup K_{\sqrt{5}/{2}}(-1/2) \cup K_{\sqrt{13}/2}(-3/2)$ fortgesetzt.
\end{example}

\begin{theorem}[Definition]
  Ein Tupel $\mathcal{K} = (K_0,\ldots,K_n)$ offener Kreisscheiben $K_j = K_{r_j}(z_j)$ heißt \acct{Kreiskette}, falls
  %
  \begin{align*}
    z_j \in K_{j-1} \lor z_{j-1} \in K_{j} \; , \quad j = 1,\ldots,n
  \end{align*}
  %
  Sind $f_j : K_j \to \mathbb{C}$ holomorph mit
  %
  \begin{align*}
    f_j \Big|_{K_j \cap K_{j-1}} = f_{j-1} \Big|_{K_j \cap K_{j-1}} \; , \quad j = 1,\ldots,n
  \end{align*}
  %
  so heißt $f_0$ \acct{analytisch fortsetzbar} längs $\mathcal{K}$, $f_n$ heißt \acct{analytische Fortsetzung} von $f_0$ längs $\mathcal{K}$.
\end{theorem}

\begin{notice}
  \begin{enum-arab}
    \item  Nach dem Identitätssatz ist $f_1$ und dann auch $f_2,\ldots,f_n$ eindeutig.
    
    \item Sind $(K_0,\ldots,K_n)$, $(\widetilde{K}_0,\ldots,\widetilde{K}_m)$ Kreisketten mit $\widetilde{K}_0 = K_0$ und $\widetilde{K}_m = K_n$, gilt dann $\widetilde{f}_m = f_n$? (Im Allgemeinen nein)
  \end{enum-arab}
\end{notice}

\begin{theorem}[Defintion]
  Sei $\gamma \in C([t_0,t_1] \to \mathbb{C})$. Eine Kreiskette $\mathcal{K} = (K_0,\ldots,K_n)$ verläuft \acct{längs} $\gamma$, falls es eine Unterteilung $t_0 = \tau_0 < \tau_1 < \ldots < \tau_n < t_1$ gibt, sodass
  %
  \begin{gather*}
    \gamma(\tau_j) \text{ Mittelpunkt von } K_j \; , \quad j = 0,\ldots,n \\
    \gamma([\tau_{j-1},\tau_{j}]) \subseteq K_{j-1} \cap K_j \; , \quad j = 1,\ldots,n
  \end{gather*}
  %
  Die zweite Bedingung verhindert, dass $\gamma$ wie im Bild aus der Kreiskette hinausläuft.
  %
  \begin{figure}[H]
    \centering
    \begin{pspicture}(-1,-0.5)(1,1)
      \pscircle[fillstyle=hlines,hatchcolor=DimGray](-1,0){0.5}
      \pscircle(-0.6,0){0.5}
      \pscircle(-0.2,0){0.5}
      \pscircle(0.2,0){0.5}
      \pscircle(0.6,0){0.5}
      \pscircle[fillstyle=hlines,hatchcolor=DimGray](1,0){0.5}
      
      \psline[linecolor=MidnightBlue]{o-o}(-1,0)(1,0)
      \psarc[linecolor=MidnightBlue]{->}(0,0.5){0.5}{45}{405}
      
      \uput[-90](-1,-0.5){\color{MidnightBlue} $\gamma(t_0)$}
      \uput[-90](1,-0.5){\color{MidnightBlue} $\gamma(t_1)$}
      \uput{0.7}[30](0,0.5){\color{MidnightBlue} $\gamma$}
      \uput{0.7}[160](-1,0){\color{DimGray} $K_0$}
      \uput{0.7}[20](1,0){\color{DimGray} $K_n$}
    \end{pspicture}
  \end{figure}
\end{theorem}

\begin{example}
  Sei $N \in \mathbb{N}$, $N \geq 2$, $\gamma(t) = \mathrm{e}^{2 \pi \mathrm{i} t}$, $0 \leq t \leq N$. Sei
  %
  \begin{align*}
    \tau_j &\coloneq \frac{j}{8} \; , \quad j = 0,\ldots,8N \\
    K_j &\coloneq K_1(\gamma(\tau_j)) \; , \quad j = 0,\ldots,8N
  \end{align*}
  %
  Dann verläuft $\mathcal{K} = (K_0,\ldots,K_{8N})$ längs $\gamma$.
\end{example}

% % % Vorlesung vom 03.12.2012

\begin{figure}[H]
  \centering
  \begin{pspicture}(0,-0.5)(1.5,2)
    \psline[ArrowInside=->,ArrowInsideOffset=0.2](0,0)(1.5,1.5)
    \psdots*(0,0)(1.5,1.5)
    \pscircle[linecolor=MidnightBlue](0,0){0.4}
    \pscircle(0.2,0.2){0.4}
    \pscircle(0.4,0.4){0.4}
    \pscircle(1.5,1.5){0.4}
    \uput{0.5}[-135](0,0){\color{MidnightBlue} $K$}
    \uput{0.5}[45](1.5,1.5){\color{DimGray} $K'$}
    \rput{-20}(0,-0.3){
      \psline{->}(0,0)(1,0)
      \uput{0.1}[-90]{20}(0.5,0){\color{DimGray} $f$}
      \uput[0]{20}(1,0){\color{DimGray} $\mathbb{C}$}
    }
    \rput{-20}(1.7,1.3){
      \psline{->}(0,0)(1,0)
      \uput{0.1}[90]{20}(0.5,0){\color{DimGray} $g,\widetilde{g}$}
      \uput[0]{20}(1,0){\color{DimGray} $\mathbb{C}$}
    }
  \end{pspicture}
\end{figure}

\begin{theorem}[Satz] \label{thm:5.6}
  Seien $\gamma \in C([t_0,t_1] \to \mathbb{C})$, $K,K'$ offene Kreisscheiben um $\gamma(t_0)$ bzw. $\gamma(t_1)$, $f$ holomorph in $K$ und $g,\widetilde{g}$ holomoprh in $K'$ und $g,\widetilde{g}$ seien aus $f$ durch analytische Fortsetzung längs Kreisketten entstanden, die längs $\gamma$ verlaufen. Dann gilt
  %
  \begin{align*}
    g = \widetilde{g}
  \end{align*}
  
  \begin{proof}
    \begin{enum-arab}
      \item Vorüberlegung
      
      \begin{figure}[H]
        \centering
        \begin{pspicture}(0,-0.5)(2,2.5)
          \pscircle[linecolor=MidnightBlue](0.5,0.7){0.6}
          \begin{psclip}{\pscircle[linecolor=DarkOrange3](0.75,1){0.6}}
            \pscircle[fillstyle=hlines,hatchcolor=lightgray,linestyle=none](0.5,0.7){0.6}
          \end{psclip}
          \pscircle[linecolor=Purple](0.6,0.85){0.4}
          \pscurve(0,0)(0.5,0.7)(0.75,1)(1,1.3)(2,2)
          \psdots*(0,0)(2,2)
          \psline[linecolor=DarkGreen](0.5,0.7)(0.75,1)
          \psline[linecolor=DarkRed](0.75,1)(1,1.3)
          \psdot*[linecolor=MidnightBlue](0.5,0.7)
          \psdot*[linecolor=DarkOrange3](0.75,1)
          \psdot*(1,1.3)
          
          \uput[-90](0,0){\color{DimGray} $\gamma(t_0)$}
          \uput[0](2,2){\color{DimGray} $\gamma(t_1)$}
          \rput{180}(0.2,0.85){
            \pscurve[linecolor=Purple](0.1,0)(0.7,0.1)(0.7,-0.1)(1.4,0)
            \uput[0]{180}(1.5,0){\color{Purple} $K_{r(t)}(\gamma(t))$}
          }
          \rput{-45}(0.5,0.7){
            \pscurve[linecolor=MidnightBlue](0.1,0)(0.7,0.1)(0.7,-0.1)(1.4,0)
            \uput[45]{45}(1.5,0){\color{MidnightBlue} $\gamma(\tau_{j-1})$}
          }
          \rput{-45}(0.75,1){
            \pscurve[linecolor=DarkOrange3](0.1,0)(0.7,0.1)(0.7,-0.1)(1.4,0)
            \uput[45]{45}(1.5,0){\color{DarkOrange3} $\gamma(\tau_{j})$}
          }
          \rput{-45}(1,1.3){
            \pscurve[linecolor=DimGray](0.1,0)(0.7,0.1)(0.7,-0.1)(1.4,0)
            \uput[45]{45}(1.5,0){\color{DimGray} $\gamma(\tau_{j+1})$}
          }
          \rput{135}(0.3,1.2){
            \pscurve[linecolor=lightgray](0,0)(0.7,0.1)(0.7,-0.1)(1.4,0)
            \uput[45]{-135}(1.5,0){\color{lightgray} $f_{j-1} = f_j$}
          }
          \rput{80}(0.8,1.4){
            \psline{->}(0,0)(1,0)
            \uput{0.1}[90]{-80}(0.5,0){\color{DimGray} $f_j$}
            \uput[0]{-80}(1,0){\color{DimGray} $\mathbb{C}$}
          }
          \rput{-140}(0.1,0.5){
            \psline{->}(0,0)(1,0)
            \uput{0.1}[-90]{140}(0.5,0){\color{DimGray} $f_{j-1}$}
            \uput[0]{140}(1,0){\color{DimGray} $\mathbb{C}$}
          }
        \end{pspicture}
      \end{figure}
      
      Sei $t_0 = \tau_0 < \tau_1 < \ldots < \tau_n = t_1$ die Unterteilung von $[t_0,t_1]$ mit Kreiskette $K = K_0,K_1,\ldots,K_n=K'$ und holomorphe Funktionen $f_j : K_j \to \mathbb{C}$, die $f$ zu $g$ fortsetzen. Für $t \in [\tau_{j-1},\tau_j]$ sei
      %
      \begin{align*}
        P_t(z) \coloneq \sum\limits_{n=0}^{\infty} a_n(t) (z-\gamma(t))^n \; , \quad z \in K_{r(t)}(\gamma(t))
      \end{align*}
      %
      die Potenzreihenentwicklung von $f_{j-1}$ um $\gamma(t)$. Beachte: $P_t$ ist auch die Potenzreihe von $f_j$ um $\gamma(t)$, da $f_j = f_{j-1}$ in $K_j \cap K_{j-1}$. Also in $P_t$ Potenzreihenentwicklung von $f_j$ um $\gamma(t)$ sogar für $t \in [\tau_{j-1},\tau_{j+1}]$. Da $\gamma$ stetig: Für festes $t \in [t_0,t_1]$
      %
      \begin{align*}
        \exists \, \delta > 0 : |\gamma(t') - \gamma(t)| < r(t) \text{ für } |t'-t| < \delta \text{, } t_0 \leq t' \leq t_1
      \end{align*}
      %
      Wähle $\delta$ so klein, dass $t,t'$ im selben Intervall $[\tau_{j-1},\tau_{j+1}]$ liegen.
      
      $\implies$ Für $|t'-t| < \delta$ erhält man $P_{t'}$ durch Entwicklung von $P_t$ um $\gamma(t')$, da um $\gamma(t') : P_t = f_j$
      
      Das nennt man die \acct[0]{lokale Verträglichkeit der Familie $(P_t)_{t_0 \leq t \leq t_1}$}.
      
      Dasselbe für $\widetilde{g}$: Unterteilung $t_0 = \sigma_0 < \sigma_1 < \ldots < \sigma_m = t_1$, $\widetilde{P}_t$ Entwicklung von $\widetilde{f}_j$ um $\gamma(t)$.
      
      \item Eigentlicher Beweis
      %
      \begin{align*}
        M \coloneq \{t \in [t_0,t_1] : P_t = \widetilde{P}_t \}
      \end{align*}
      %
      Zeige: $M = [t_0,t_1] \implies g = P_{t_1} = \widetilde{P}_{t_1} = \widetilde{g}$.
      %
      \begin{enum-arab}
        \item \label{itm:5.6 a)} $M \neq \emptyset$: Wegen $f_0 = f = \widetilde{f}_0$ gilt $[t_0,t_0 + \delta] \subseteq M$, $\delta$ so gewählt, dass $\gamma(t) \in K = K_0 = \widetilde{K}_0$ für $t_0 \leq t \leq t_0 + \delta$.
        
        \item \label{itm:5.6 b)} $M$ ist relativ offen\footnote{relativ offen: Ich kann $M$ bekommen durch den Schnitt einer offenen Menge mit $[t_0,t_1]$}: Sei $s \in M$, $\delta > 0$ so klein, dass \[|\gamma(t) - \gamma(s)| < \min\{r(s),\widetilde{r}(s)\} \text{ für } s - \delta < t < s + \delta\] (Stetigkeit von $\gamma$). Aus der lokalen Verträglichkeit folgt: $P_t$ entsteht aus Entwicklung von $P_s$ um $\gamma(t)$ genauso $\widetilde{P}_t$ aus $\widetilde{P}_s$.
        %
        \begin{align*}
          P_s \overset{s \in M}{=} \widetilde{P}_s \implies P_t = \widetilde{P}_t \text{ für } s - \delta < t < s + \delta
        \end{align*}
        
        \item \label{itm:5.6 c)} $M$ ist abgeschlossen: Sei $(s_n)$ in $M$, $s_n \to s$, $s_n \neq s$. Dann gilt $\gamma(s_n) \to \gamma(s)$ und entweder $\gamma(s_n) = \gamma(s)$ für ein $n$ $\implies$ $s \in M$, da $P_s = P_{s_n}$ oder $\gamma(s_n) \neq \gamma(s)$ für $n \in \mathbb{N}$: Für $n > N_\delta$: 
        \begin{align*}
          P_s(\gamma(s_n)) = P_{s_n}(\gamma(s_n)) = \widetilde{P}_{s_n}(\gamma(s_n)) = \widetilde{P}_s(\gamma(s_n))
          \overset{\text{Identitätssatz}}{\implies} \widetilde{P}_s = P_s .
        \end{align*}
      \end{enum-arab}
      %
      \ref{itm:5.6 a)}, \ref{itm:5.6 b)} und \ref{itm:5.6 c)} $\implies$ $M = [t_0,t_1]$.
    \end{enum-arab}
  \end{proof}
\end{theorem}

\begin{theorem}[Definition]
  Sei $\gamma \in C([t_0,t_1] \to \mathbb{C})$, $K,K'$ offene Kreisscheiben um $\gamma(t_0)$ bzw. $\gamma(t_1)$. $f$ holomorph in $K$, $g$ holomorph in $K'$. Dann heißt $g$ \acct{analytische Fortsetzung} von $f$ längs $\gamma$, falls es eine längs $\gamma$ verlaufende Kreiskette $\mathcal{K}$ gibt, sodass $g$ analytische Fortsetzung von $f$ längs $\mathcal{K}$ ist.
\end{theorem}

\begin{example} Seien
  $N \in \mathbb{N}$, $N \geq 2$, $f(z) = |z|^{1/N} \mathrm{e}^{\mathrm{i}\frac{1}{N}\arg z}$
  %
  \begin{align*}
    \gamma(t) = \mathrm{e}^{2 \pi \mathrm{i} t} \; , \quad 0 \leq t \leq N
  \end{align*}
  %
  $\tau_j = \dfrac{j}{8N}$, $K_j = K_1(\gamma(\tau_j))$, $j = 0,\ldots,8N$.
  %
  \begin{figure}[H]
    \centering
    \begin{pspicture}(-2.5,-2.5)(2.5,2.5)
      \psaxes[labelFontSize=\color{DimGray}\footnotesize]{->}(0,0)(-2.5,-2.5)(2.5,2.5)[\color{DimGray} Re,0][\color{DimGray} Im,0]
      
      \pscircle[linecolor=MidnightBlue](1;0){1}
      \pscircle[linecolor=MidnightBlue](1;90){1}
      \pscircle[linecolor=MidnightBlue](1;180){1}
      \pscircle[linecolor=MidnightBlue](1;270){1}
      \psdots*[linecolor=MidnightBlue](1;0)(1;90)(1;180)(1;270)
      \uput{1.2}[20](1;0){\color{MidnightBlue} $K_0$}
      \uput{1.2}[110](1;90){\color{MidnightBlue} $K_2$}
      \uput{1.2}[200](1;180){\color{MidnightBlue} $K_4$}
      \uput{1.2}[290](1;270){\color{MidnightBlue} $K_6$}
      
      \pscircle[linecolor=DarkOrange3](1;45){1}
      \pscircle[linecolor=DarkOrange3](1;135){1}
      \pscircle[linecolor=DarkOrange3](1;225){1}
      \pscircle[linecolor=DarkOrange3](1;315){1}
      \psdots*[linecolor=DarkOrange3](1;45)(1;135)(1;225)(1;315)
      \uput{1.2}[45](1;45){\color{DarkOrange3} $K_1$}
      \uput{1.2}[135](1;135){\color{DarkOrange3} $K_3$}
      \uput{1.2}[225](1;225){\color{DarkOrange3} $K_5$}
      \uput{1.2}[315](1;315){\color{DarkOrange3} $K_7$}
      
      \psarc[linecolor=Purple]{->}(0,0){1}{30}{390}
      \uput{1.2}[25](0,0){\color{Purple} $\gamma$}
    \end{pspicture}
  \end{figure}
  %
  Setze $f$ längs $K_0,K_1,\ldots,K_{8N}$ fort
  %
  \begin{align*}
    f_0 = f \Big|_{K_0} \text{ für } j = 1,2
  \end{align*}
  %
  Wie sieht $f_3$ aus?
  %
  \begin{align*}
    \arg_\pi (z) \coloneq
    \begin{dcases}
      \arg(z) & \Im z > 0 \\
      \pi & z \in ]-\infty,0[ \\
      \arg(z+2\pi) & \Im z < 0
    \end{dcases}
  \end{align*}
  %
  Setze $g_1(z) \coloneq |z|^{1/N} \mathrm{e}^{\mathrm{i}\frac{1}{N}\arg_\pi z}$
  %
  \begin{align*}
    \implies&
    \begin{cases}
      g_1 \text{ holomorph in } \mathbb{C} \setminus [0,\infty[ \\
      g_1 \Big|_{\{z: \Im z > 0\}} = f \Big|_{\{z: \Im z > 0\}}
    \end{cases} \\
    \implies&
    f_j = g_1 \Big|_{K_j} \; , \quad j = 3,4,5,6
  \end{align*}
  %
  Setze $g_2(z) \coloneq |z|^{1/N} \mathrm{e}^{\mathrm{i}\frac{1}{N}\arg_\pi (z+2\pi)}$
  %
  \begin{align*}
    \implies&
    \begin{cases}
      g_2 \text{ holomorph in } \mathbb{C} \setminus ]-\infty,0] \\
      g_2 \Big|_{\{z: \Im z < 0\}} = g_1 \Big|_{\{z: \Im z < 0\}}
    \end{cases} \\
    \implies&
    f_j = g_2 \Big|_{K_j} \; , \quad j = 7,8,9,10
  \end{align*}
  %
  Beachte: $K_8 = K_0$, aber $f_8 = \mathrm{e}^{\mathrm{i}\frac{2 \pi}{N}} f_0 \neq f_0$.
  %
  \begin{align*}
    f_{16} &= \mathrm{e}^{\mathrm{i} \, 2 \frac{2 \pi}{N}} f_0 \neq f_0 \\
    &\vdots \\
    f_{8N} &= \mathrm{e}^{\mathrm{i} \, N \frac{2 \pi}{N}} f_0 = f_0
  \end{align*}
  %
  Nach der $N$-ten Umkreisung der $0$ landen wir wieder bei der ursprünglichen Funktion.
\end{example}

\begin{theorem}[Umparametrisierung]
  Sei $\gamma \in C([t_0,t_1] \to \mathbb{C})$, $f_1$ analytische Fortsetzung von $f$ längs $\gamma$, $\varphi : [s_0,s_1] \to [t_0,t_1]$ stetig, streng monoton wachsend, $\varphi(s_0) = t_0$, $\varphi(s_1) = t_1$. Damit ist $f_1$ auch analytische Fortsetzung von $f$ längs $\gamma \circ \varphi$. Insbesondere kann immer $s_0 = 0$, $s_1 = 1$ gewählt werden.
  
  \begin{proof}
    Verwende dieselbe Kreiskette, als Unterteilung von $[s_0,s_1] : s_j = \varphi^{-1}(\tau_j)$.
  \end{proof}
\end{theorem}

\begin{theorem}[Monodromiesatz] \label{thm:5.10}
  Seien $\gamma, \widetilde{\gamma} \in C([0,1] \to \mathbb{C})$ mit $\gamma(0) = \widetilde{\gamma}(0)$, $\gamma(1) = \widetilde{\gamma}(1)$. Weiter sei $\Phi \in C([0,1] \times [0,1] \to \mathbb{C})$ eine Homotopie zwischen $\gamma,\widetilde{\gamma}$, das heißt
  %
  \begin{gather*}
    \Phi(\cdot,0) = \gamma \qquad \Phi(\cdot,1) = \widetilde{\gamma} \\
    \Phi(0,s) = \gamma(0) \qquad \Phi(1,s) = \gamma(1) \; , \quad 0 \leq s \leq 1
  \end{gather*}
  %
  Ist $f_0$ holomorph in einer Kreisscheibe $K_r(\gamma(0))$ und lässt sich $f_0$ längs jedes Weges $\gamma_s \coloneq \Phi(\cdot,s)$ analytisch fortsetzen, dann stimmen die analytischen Fortsetzungen von $f_0$ längs $\gamma$ und $\widetilde{\gamma}$ überein.
  
  \begin{figure}[H]
    \centering
    \begin{pspicture}(-0.5,-0.5)(1.5,1.5)
      \cnode*(-0.5,-0.5){2pt}{A}
      \cnode*(1.5,1.5){2pt}{B}
      \ncarc[linecolor=DarkOrange3,arcangle=70,arrows=->]{A}{B}
      \naput{\color{DarkOrange3} $\gamma_0 = \gamma$}
      \ncarc[arcangle=40,arrows=->]{A}{B}
      \ncarc[arcangle=10,arrows=->]{A}{B}
      \ncarc[linecolor=DarkOrange3,arcangle=-30,arrows=->]{A}{B}
      \nbput*{\color{DarkOrange3} $\widetilde{\gamma} = \gamma_1$}
      \rput{0}(0.4,0.6){
        \pscurve(0.1,0)(0.7,0.1)(0.7,-0.1)(1.4,0)
        \uput[0]{0}(1.5,0){\color{DimGray} $\gamma_{2/3}$}
      }
      \rput{180}(-0.35,0){
        \pscurve(0.1,0)(0.7,0.1)(0.7,-0.1)(1.4,0)
        \uput[0]{180}(1.5,0){\color{DimGray} $\gamma_{1/3}$}
      }
    \end{pspicture}
  \end{figure}
  %
  Der Beweis verläuft ähnlich wie in \ref{thm:5.6}.
\end{theorem}

\begin{theorem}[Definition]
  Ein Gebiet $G \subseteq \mathbb{C}$ heißt \acct{einfach zusammenhängend}, wenn jede geschlossene stetige Kurve $\gamma$ in $G$ nullhomotop ist, das heißt es existiert eine Homotopie $\Phi \in C([0,1] \times [0,1] \to G)$ zwischen $\gamma$ und einer konstanten Kurve.
  
  Oder äquivalent: Zu je zwei Kurven $\gamma_1,\gamma_2 \in C([0,1] \to G)$ mit $\gamma_1(0) = \gamma_2(0)$ und $\gamma_1(1) = \gamma_2(1)$ existiert eine Homotopie $\Phi \in C([0,1] \times [0,1] \to G)$.
\end{theorem}

\begin{example*}
  \begin{align*}
    \mathbb{C} \setminus [0,\infty[ \quad &\text{Gebiet, einfach zusammenhängend}\\
    \mathbb{C} \setminus \{0\} \quad  & \text{Gebiet, nicht einfach zusammenhängend} \\
    & \text{(Der Kreis um die $0$ ist nicht nullhomotop)}
  \end{align*}
\end{example*}

% % % Vorlesung vom 06.12.2012

\begin{notice}[Folgerung]
  $G \subseteq \mathbb{C}$ einfach zusammenhängendes Gebiet, $f_0$ holomorph in $K_r(z_0) \subseteq G$. Lässt sich $f_0$ längs jeder stetigen Kurve $\gamma$ mit Anfangspunkt $z_0$ analytisch fortsetzen, dann gibt es genau eine holomorphe Funktion
  %
  \begin{align*}
    f : G \to \mathbb{C} \quad \text{mit} \quad f \Big|_{K_r(z_0)} = f_0
  \end{align*}
  %
  \begin{proof}
    \begin{enum-arab}
      \item Definiere $f : G \to \mathbb{C}$. Sei $z_1 \in G$. Da $G$ ein Gebiet ist folgt: Es existiert eine stetige Kurve $\gamma$ von $z_0$ nach $z_1$. Setze $f_0$ längs $\gamma$ fort zu $f_1 : K_s(z_1) \to \mathbb{C}$ ($f_1$ ist holomorph). Setze $f(z_1) \coloneq f_1(z_1)$. $f$ ist sinnvoll definiert: Ist $\widetilde{\gamma}$ eine andere stetige Kurve von $z_0$ nach $z_1$. Da $G$ einfach zusammenhängend ist folgt $\gamma \sim \widetilde{\gamma}$.
      
      Monodromiesatz \ref{thm:5.10}: Fortsetzung längs $\widetilde{\gamma}$ liefert dieselbe Funktion $f_1$.
      
      \item Zeige: $f$ ist holomorph. Sei $z_1 \in G$ fest, $f_1$ wie oben. Zeige \[ f \Big|_{K_{s/3}}(z_1). \] Dann ist $f$ holomorph in $K_{s/3}(z_1)$. Da $z_1 \in G$ holomorph ist, ist $f$ holomorph in $G$.
      
      \begin{figure}[H]
        \centering
        \begin{pspicture}(0,0)(4,4)
          \pscircle(2,2){0.5}
          \pscircle[linecolor=DarkOrange3](2,2){1.5}
          \pscircle[linecolor=Purple](1.7,2.1){0.75}
          \rput(2,2){\psline(0,0)(0.5;45)}
          \rput(1.7,2.1){\psline[linecolor=Purple](0,0)(0.75;200)}
          \psbezier[linecolor=MidnightBlue](0,0)(0.5,2)(1.5,0)(2,2)
          \psline[linecolor=MidnightBlue]{->}(2,2)(1.7,2.1)
          \psbezier{->}(0.05,0)(0.55,2)(1.55,0)(2.05,2)
          \psdots*(2,2)(0,0)
          \psdot*[linecolor=DarkOrange3](1.7,2.1)
          \uput{1.7}[45](2,2){\color{DarkOrange3} $K_s(z_1)$}
          \uput[-45](0,0){\color{DimGray} $z_0$}
          \uput[-45](2,2){\color{DimGray} $z_1$}
          \uput[135](1.7,2.1){\color{DarkOrange3} $z_2$}
          \uput[-90](1,1){\color{DimGray} $\gamma$ \color{MidnightBlue} $\widetilde{\gamma}$}
        \end{pspicture}
      \end{figure}
      
      Sei $z_2 \in K_{s/3}(z_1)$. Bilde $\widetilde{\gamma}$ aus $\gamma$ durch Anhängen der Strecke $z_1 z_2$. Ergänze dei Kreiskette $\mathcal{K}$ längs $\gamma$, die zur Fortsetzung längs $\gamma$ verwendet wurde, durch $K_{s/2}(z_2)$ zur Kreiskette $\widetilde{\mathcal{K}}$ längs $\widetilde{\gamma}$. $f(z_2)$ wird definiert durch analytische Fortsetzung von $f_0$ längs $\widetilde{\gamma}$. Dies liefert $f_2 : K_{s/2}(z_2) \to \mathbb{C}$. Wegen \[ f_2 = f_1 \text{ in } K_{s/2}(z_2) \cap K_s(z_1) \] folgt $f(z_2) \coloneq f_2(z_2) = f_1(z_2)$.
      
      \item Zeige: $f$ ist eindeutig. Ist $\mathcal{K} = (K_1,\ldots,K_n)$ Kreiskette längs $\gamma$, so ist $f \Big|_{K_j}$ eine analytische Fortsetzung längs $\gamma$. Aus Satz \ref{thm:5.6} folgt: Jede andere Fortsetzung längs $\gamma$ liefert dieselbe analytische Fortsetzung.
    \end{enum-arab}
  \end{proof}
\end{notice}

\begin{example}
  \begin{align*}
    f_0(z) = |z|^{1/N} \mathrm{e}^{\mathrm{i}\frac{1}{N}\arg z} \quad \text{in } K_1(2)
  \end{align*}
  %
  \begin{figure}[H]
    \centering
    \begin{pspicture}(-0.2,-0.7)(2,0.7)
      \psaxes[labels=none,ticks=none]{->}(0,0)(-0.2,-0.7)(2,0.7)[\color{DimGray} Re,0][\color{DimGray} Im,0]
      \pscircle[linecolor=DarkOrange3,fillstyle=hlines,hatchcolor=DarkOrange3](1,0){0.5}
    \end{pspicture}
  \end{figure}
  %
  Falls $G = \mathbb{C} \setminus ]-\infty,0]$
  %
  \begin{align*}
    f(z) = |z|^{1/N} \mathrm{e}^{\mathrm{i}\frac{1}{N}\arg z}
  \end{align*}
  %
  falls $G = \mathbb{C} \setminus ]-\mathrm{i} \infty,\mathrm{i} 0]$
  %
  \begin{align*}
    f(z) = |z|^{1/N} \mathrm{e}^{\mathrm{i}\frac{1}{N}\arg_{\pi/2} z}
  \end{align*}
\end{example}

\begin{notice}[Ausblick]
  Erweiterung des Wegintegrals auf stetige Kurven. Ist $f$ analytische fortsetzbar längs $\gamma$ mit Unterteilung \[ t_0 = \tau_0 < \tau_1 < \ldots < \tau_n = t_1 , \] so definiere
  %
  \begin{align*}
    \int_\gamma f(z) \, \mathrm{d}z &\coloneq \sum\limits_{j=1}^{\infty} \int_{\gamma|_{[\tau_{j-1},\tau_j]}} f(z) \, \mathrm{d}z \\
    \int_{\gamma|_{[\tau_{j-1},\tau_j]}} f(z) \, \mathrm{d}z &\coloneq F_j(\gamma(\tau_j)) - F_j(\gamma(\tau_{j-1})).
  \end{align*}
  %
  $F$ ist die lokale Stammfunktion der Fortsetzung von $f$ in $K_j$.
\end{notice}
