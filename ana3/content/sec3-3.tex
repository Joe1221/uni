% Stephan Hilb, 2012 Universität Stuttgart.
%
% Dieses Werk ist unter einer Creative Commons Lizenz vom Typ
% Namensnennung - Nicht-kommerziell - Weitergabe unter gleichen Bedingungen 3.0 Deutschland
% zugänglich. Um eine Kopie dieser Lizenz einzusehen, konsultieren Sie
% http://creativecommons.org/licenses/by-nc-sa/3.0/de/ oder wenden Sie sich
% brieflich an Creative Commons, 444 Castro Street, Suite 900, Mountain View,
% California, 94041, USA.

\section{Existenz und Eindeutigkeit} % 3 ?


\begin{df} \label{3.1}
	Sei $B$ ein Banachraum.
	\begin{enumerate}[1)]
		\item
			Sei $f: \R \supset D \to B$ mit Häufungspunkt $x$ in $D$.
			$f$ heißt \emph{differenzierbar} in $x$, falls
			\[
				\exists f'(x) \in B : \bigg\| \f{f(x+h)-f(x)}{h} - f'(x) \bigg\| \to 0
				\qquad k \to 0, x + k \in D
			\]
		\item
			Für $-\infty < a < b < \infty$ heißt
			\[
				Z([a,b]) = \Big( (x_0,\dotsc,x_{N(Z)}, (\xi_1, \dotsc, \xi_{N(Z)}) \Big)
			\]
			mit $a = x_0 < x_1 < \dotsc < x_{N(Z)} = b$ und $\xi_j \in [x_{j-1}, x_j]$ eine \emph{Zerlegung} von $[a,b]$.
			Weiter heißt
			\[
				S_Z(f) := \sum_{j=1}^{N(Z)} f(\xi_j)(x_j - x_{j-1})
			\]
			\emph{Riemann-Summe}.
			Definiere die \emph{Feinheit} von $Z$ als
			\[
				\delta(Z) := \max_{1\le j \le N(Z)} (x_j-x_{j-1})
			\]
			$f: [a,b] \to B$ heißt \emph{Riemann-integrierbar}, falls
			\[
				\exists I \in B : \|S_z(f) - I\| \to 0
				\qquad \delta(Z) \to 0
			\]
			schreibe
			\[
				I := \int_a^b f(x) dx
			\]
	\end{enumerate}
\end{df}

\begin{kor}[Folgerung] \label{3.2}
	Für $a < c < b$ gilt
	\[
		\int_a^c f(x) dx + \int_c^b f(x) dx = \int_a^b dx
	\]
\end{kor}

\begin{st} \label{3.3}
	Wenn $f: [a,b] \to B$ stetig ist, dann ist $f$ Riemann-integrierbar.
	\begin{proof}
		Zeige: falls $\delta(Z), \delta(Z') < \delta$, dann ist $\|S_Z(f) - S_{Z'}(f)\| < \eps$.
		Für jede Folge $Z_n$ von Zerlegungen mit $\delta(Z_n) \to 0$ ($n\to\infty$) gilt dann.
		\begin{enumerate}[1)]
			\item
				$(S_{Z_n}(f))$ ist Cauchy-Folge in $B$, also konvergent.
			\item
				$I := \lim_{n\to \infty} S_{Z_n}(f)$ ist unabhängig von der Folge $(Z_n)$.
		\end{enumerate}
		Sei $\eps > 0$ vorgegeben und $Z, Z'$ zwei Zerlegungen.
		Definiere
		\[
			Z'' = \Big( (x_0'', \dotsc, x_{N(Z)}'') , (\xi_1'', \dotsc, \xi_{N(Z)}'') \Big)
		\]
		durch
		\[
			\{x_0'', \dotsc, x_{N(Z'')}''\} := \{x_0,\dotsc, x_{N(Z)}\} \cup \{ x_0', \dotsc, x_{N(Z)}' \}
			\qquad \xi_j'' \in [x_{j-1}'', x_j''] \text{ beliebig}
		\]
		Benutze
		\[
			\|S_Z(f) - S_{Z'}(f)\| \le \|S_Z(f) - S_{Z''}(f)\| + \|S_{Z''}(f) - S_{Z'}(f)\|
		\]
		\fixme[Zeichnung]
		\begin{align*}
			\|S_Z(f) - S_{Z''}(f)\| 
			&= \bigg\| \sum_{j=1}^{N(Z)} f(\xi_j) (x_j-x_{j-1}) - \sum_{j=1}^{N(Z'')} f(\xi_j'')(x_j'' - x_{j-1}'') \bigg\| \\
			&= \bigg\| \sum_{j=1}^{N(Z'')} f(\xi_{J(j)} (x_j''-x_{j-1}'') - \sum_{j=1}^{N(Z'')} f(\xi_j'')(x_j'' - x_{j-1}'') \bigg\| \\
			\intertext{
				Dabei ist $J(j)$ so definiert, dass $[x_{j-1}'',x_j''] \subset [x_{J(j)-1},x_{J(j)}]$, d.h. 
				\[
					J(j) := \min \Big\{ k\in \N : x_j'' \le x_k \Big\}
				\]
			}
			&\le \sum_{j=1}^{N(Z'')} \| f(\xi_{J(j)}) - f(\underbrace{\xi_j''}_{\in [x_{J(j)-1},x_{J(j)}]})\| (x_j''-x_{j-1}'')
			\intertext{Da $| \xi_{J(j)}-\xi_J''| \le \delta(Z)$ und $f$ gleichmäßig stetig ($f$ stetig und $[a,b]$ kompakt), ist
				\[
					\|f(\xi_{J(j)}) - f(\xi_j'')\| < \f{\eps}{2(b-a)}
					\qquad \text{für } \delta(Z)  < \delta
				\]
			}
			&\le \f {\eps}{2(b-a)} \underbrace{\sum_{j=1}^{N(Z'')} (x_j''-x_{j-1}'')}_{=b-a} = \f {\eps}2
		\end{align*}
		Verfahre analog für den anderen Teil, dann ergibt sich
		\[
			\|S_Z(f) - S_{Z'}(f)\| < \f \eps 2 + \f \eps 2 = \eps
			\qquad \delta(Z),\delta(Z') < \delta
		\]
	\end{proof}
\end{st}

\begin{st} \label{3.4}
	Wenn $f:[a,b] \to B$ stetig ist, dann gilt
	\[
		\bigg\| \int_a^b f(x) dx \bigg\| \le \int_a^b \|f(x)\| dx
	\]
	\begin{proof}
		Schreibe
		\begin{align*}
			\bigg\|\int_a^b f(x) dx \bigg\| 
			&= \lim_{\delta(Z)\to 0} \|S_Z(f)\| \\
			&= \lim_{\delta(Z)\to 0} \bigg\| \sum_{j=1}^{N(Z)} f(\xi_j) (x_{j} x_{j-1}) \bigg\| \\
			&= \limsup_{\delta(Z) \to 0} \underbrace{\sum_{j=1}^{N(Z)} \| f(\xi_j) \| (x_j - x_{j-1}}_{= S_Z(\|f\|) \to \int_a^b \|f(x)\| dx} \\
			&= \int_a^b \|f(x)\| dx
		\end{align*}
	\end{proof}
\end{st}

\begin{st}[Hauptsatz] \label{3.5}
	Sei $f: [a,b] \to B$ stetig und $F(x) := \int_a^b f(\xi) d\xi$ ($a \le x \le b)$.

	Dann ist $F$ differenzierbar und 
	\[
		F'(x) = f(x) 
		\qquad a \le x \le b
	\]
	\begin{proof}
		Für $h > 0$ betrachte
		\begin{align*}
			\Big\| \tf 1h \underbrace{(F(x+h) - F(x))}_{= \int_x^{x+h}f(\xi) d\xi \text{ (Additivität Integral)}} - \overbrace{f(x)}^{\f 1h \int_{x}^{x+h}f(x) d\xi} \Big\| \\
			&= \bigg\| \f 1h \int_x^{x+h} (f(\xi) - f(x)) d\xi \bigg\|  \\
			&\le \f 1h \int_x^{x+h} \underbrace{\|f(\xi) - f(x)\|}_{< \eps \text{ für } |\xi - x| \le h < \delta \text{ da $f$ stetig in $x$}} d\xi \\
			&\le \f 1h \int_x^{x+h} \eps d\xi = \eps
		\end{align*}
		für $h < \delta$.
	\end{proof}
\end{st}

\begin{kor}[Folgerung] \label{3.6}
	Falls $G \in C^1([a,b] \to B)$ und $G' = f$ ($G$ \emph{Stammfunktion} von $f$), dann ist
	\[
		(G - F)' = 0
	\]
	Also (ohne Beweis) $G = F + c$ mit $c \in B$.
	Damit ist das Integral
	\begin{align*}
		\int_a^b f(x) dx 
		&= F(b) - F(a) \\
		&= G(b) - c - (G(a) - c) \\
		&= G(b) - G(a)
	\end{align*}
	unabhängig von der Wahl der Stammfunktion.
\end{kor}

\begin{st}[Picard-Lindelöf] \label{3.7}
	Sei $(B, \|\cdot\|)$ ein Banachraum, $(x_0,y_0) \in \R \times B$, $I=[x_0-r, x_0+r] \subset \R$ ein Intervall und $D := \{y \in B : \|y-y_0\| \le R\}$ für ein $R \in \R$.
	Sei $f \in C(I \times D \to B)$ mit
	\[
		\exists L > 0 \; \forall(x,y), (x,\tilde y) \in I \times D : \|f(x,y) - f(x,\tilde y) \| \le L\|y-\tilde y\|
	\]
	d.h. $f$ erfüllt eine \emph{Lipschitz-Bedingung} bezüglich zwei Variablen.
	Weiter seien
	\[
		M := \sup_{I \times D} \|f(x,y)\| < \infty
		\qquad
		\delta := \min \bigg\{ \f 1{2L}, \f RM, r \bigg\}
	\]
	Dann existiert eine eindeutige Lösung $y$ von
	\begin{align} \label{eq:3.7.1}
		&\quad y \in C^1([x_0 - \delta, x_0 + \delta] \to B) \\
		\land&\quad y'(x) = f(x,y(x)) \qquad x_0 - \delta \le x \le x_0 + \delta \\
		\land&\quad y' \fixme[\text{hier fehlt was}]
	\end{align}
	\fixme[Veranschaulichung falls $B = \R$, Zeichnung].
	\begin{proof}
		\begin{enumerate}[1)]
			\item
				Formuliere eine äquivalente Integralgleichung.
				$y$ ist genau dann eine Lösung von \eqref{eq:3.7.1}, wenn
				\begin{align} \label{eq:3.7.2}
					y \in C ( [x_0 - \delta, x_0 + \delta] \to B)
					\qquad \land \qquad
					y(x) = y_0 + \int_{x_0}^x f(\xi, y(\xi)) d\xi
				\end{align}
				mit $\int_{x_0}^{x} \dotso = \int_x^{x_0} \dotso$ falls $x < x_0$.
				\begin{proof}
					\begin{seg}[„$\eqref{eq:3.7.1} \implies \eqref{eq:3.7.2}$“]
						Integriere die Differentialgleichung in \eqref{eq:3.7.1}:
						\begin{align*}
							\int_{x_0}^x y'(\xi) d\xi &= \int_{x_0}^x f(\xi, y(x)) d\xi
							y(x) - \underbrace{y(x_0)}_{=y_0} &= \int_{x_0}^x f(\xi, y(x)) d\xi
						\end{align*}
					\end{seg}
					\begin{seg}[„$\eqref{eq:3.7.2} \implies \eqref{eq:3.7.1}$“]
						$f(\xi, y(\xi))$ ist stetig in $\xi$.
						Nach dem Hauptsatz ist $y' = 0 + f(x, y(x))$ mit $y'$ stetig.
						Für $x = x_0$ in der Integralgleichung: $y(x_0) = y_0 + 0$.

					\end{seg}
				\end{proof}
			\item
				Zeige, dass $\eqref{eq:3.7.2}$ eine eindeutige Lösung besitzt.
				\begin{proof}
					Sei $(\tilde B, \|\cdot\|^{\sim}) := (C([x_0-\delta, x_0+\delta] \to B), \|\cdot\|_\infty)$ ein Banachraum und
					\[
						\tilde D := C([x_0 -\delta, x_0 + \delta] \to D) 
					\]
					(abgeschlossene Teilmenge in $\tilde B$)
					Definiere $T: \tilde D \to \tilde B : y \to F(y)$ durch
					\[
						F(y)(x) := y_0 + \int_{x_0}^x f(\xi, y(\xi)) d\xi
					\]
					\begin{enumerate}[a)]
						\item
							$F(y)$ ist stetig und
							\[
								\|F(y(x)) - y_0 \|
								\stackrel{\ref{3.4}}\le \bigg| \int_{x_0}^x \underbrace{\|f(\xi, y(\xi))\|}_{\le M} d\xi \bigg|
								\le M \cdot |x - x_0|
								\le M \cdot \delta
								\stackrel{\text{def } \delta}\le R
							\]
							Also $F(\tilde D ) \le \tilde D$.
						\item
							Außerdem ist $\tilde D \neq \emptyset$, denn $\phi \in \tilde D$ für $\phi(x) := y_0$.
						\item
							$F$ ist eine Kontraktion.
							\begin{align*}
								\|F(y) - F(\tilde y)\|_\infty
								&= \sup_{x_0-\delta \le x \le x_0+\delta} \bigg\| \int_{x_0}^x \Big( f(\xi, y(\xi)) - f(\xi, \tilde y(\xi)) \Big) d\xi \bigg\| \\
								&= \sup_{x_0-\delta \le x \le x_0+\delta} \bigg| \int_{x_0}^x \underbrace{\Big\| f(\xi, y(\xi)) - f(\xi, \tilde y(\xi)) \Big\|}_{\le L \|y(\xi)-\tilde y(\xi)\| \le L \|y-\tilde y\|_\infty} d\xi \bigg| \\
								&\le L \|y-\tilde y\|_\infty \cdot \underbrace{|x-x_0|}_{\le \delta} \\
								&\stackrel{L \delta \le \f 12} \le \underbrace{\f 12}_{=q} \|y-\tilde y\|_\infty
							\end{align*}
					\end{enumerate}
					Der Banachsche Fixpunktsatz besagt jetzt
					\[
						\exists! y \in \tilde B : F(y) = y
					\]
					also $y \in C([x_0-\delta, x_0+\delta] \to B)$ und aus $F(y) = y$ ergibt sich
					\[
						y_0  + \int_{x_0}^x f(\xi, y(\xi)) d \xi = y(x)
					\]
				\end{proof}
		\end{enumerate}
		Die Äquivalenz zur ursprünglichen DGL liefert somit den Beweis dafür, dass die Lösung eindeutig ist.
	\end{proof}
\end{st}


