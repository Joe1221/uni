% Stephan Hilb, 2012 Universität Stuttgart.
%
% Dieses Werk ist unter einer Creative Commons Lizenz vom Typ
% Namensnennung - Nicht-kommerziell - Weitergabe unter gleichen Bedingungen 3.0 Deutschland
% zugänglich. Um eine Kopie dieser Lizenz einzusehen, konsultieren Sie
% http://creativecommons.org/licenses/by-nc-sa/3.0/de/ oder wenden Sie sich
% brieflich an Creative Commons, 444 Castro Street, Suite 900, Mountain View,
% California, 94041, USA.

\section{Systeme von Differentialgleichungen} % 3.3


Sei $A \in \R^{n\times n}$ gegeben.
Gesucht ist $y \in C^1(\R \to \R^n)$ mit
\[
	y' = Ay
\]
ausgeschrieben ergeben sich
\begin{align*}
	y_1' &= a_{11} y_1 + a_{12} y_2 + \dotsb + a_{1n} y_n \\
	\vdots \; &= \qquad\qquad\qquad \vdots \\
	y_n' &= a_{n1} y_1 + a_{n2} y_2 + \dotsb + a_{nn} y_n \\
\end{align*}
also $n$ \emph{gekoppelte} Differentialgleichungen.

\begin{seg}[Fall 1: ${v_1,\dotsc, v_n}$ bildet eine Basis aus Eigenvektoren: $Av_j = \lambda_j v_j$]
	Dann ergibt sich die Lösung als
	\[
		y(t) := \sum_{j=1}^n c_j e^{\lambda_j t} v_j
		\qquad c_1,\dotsc, c_n \in \R
	\]
	denn
	\begin{align*}
		y' = \sum_{j=1}^n c_j \lambda_j e^{\lambda_j t} v_j = \sum_{j=1}^n c_j e^{\lambda_j t} Av_j = A y(t)
	\end{align*}
	Die Eindeutigkeit ist durch Anfangsbedingungen gegeben:
	\[
		y(t_0) = y_0
		\qquad t_0 \in \R, y_0 \in \R^n
	\]
	damit gilt	
	\[
		\sum_{j=1}^n \underbrace{c_j e^{\lambda_j t_0}}_{=:d_j} v_j = y_0
	\]
	Die $d_j$ existieren und sind eindeutig, da $\{v_1,\dotsc,v_n\}$ Basis ist.
	Also existieren auch die
	\[
		c_j  = e^{-\lambda_j t_0} d_j
	\]
	und sind eindeutig.

	\begin{note}[Beobachtungen]
		\begin{itemize}
			\item
				Es existiert immer eine globale Lösung.
			\item
				Die Lösungsgesamtheit ist durch $n$ skalierbare Gleichungen gegeben mit Parametern $c_j$.
			\item
				Die Eindeutigkeit ist stets durch die Anfangsbedingung gewährleistet.
			\item
				Für festes $t \in \R$ hängt $y(t)$ stetig von $(t_0, y_0)$ ab.
			\item
				Der Lösungsraum ist ein reeller linearer Raum mit Basis
				\[
					\{ t \to e^{\lambda_1 t v_1}, \dotsc, t\to e^{\lambda_n t}v_n\}
				\]
		\end{itemize}
	\end{note}
\end{seg}

\begin{seg}[Fall 2: sonst]
	Bilde die Jordan-Normalform:
	\[
		J = T^{-1} A T
	\]
	beispielsweise
	\[
		J = \begin{pmatrix}
			\lambda_1 & 1 & 0\\
			0 & \lambda_1 & 1 \\
			0 & 0 & \lambda_1
		\end{pmatrix}
	\]
	Setze $u(t) := T^{-1} y(t)$ für eine Lösung $y(t)$.
	Dann ist
	\[
		u'(t) = T^{-1}y'(t) = T^{-1}Ay(t) = T^{-1}AT u(t)
	\]
	Also
	\[
		y' = Ay
		\qquad \iff \qquad
		u' = Ju
	\]
	Wir können also statt $y' = Ay$ das System $u' = Ju$ berechnen und anschließend zurücktransformieren.
	In unserem Beispiel:
	\begin{alignat*}{3}
		u_1' &= \lambda_1 u_1& &+ u_2 \\
		u_2' &= &&+ \lambda_1 u_2 && + u_3 \\
		u_3' &= && &&+\lambda_1 u_3
	\end{alignat*}
	Die Lösungen sind gegeben durch (gehe Blockweise von unten nach oben vor und nutze die umgekehrte Produktregel):
	\begin{align*}
		u_3 &= c_3 e^{\lambda_1 t} \\
		u_2 &= c_2 e^{\lambda_1 t} + c_3 t e^{\lambda_1 t} \\
		u_1 &= c_1 e^{\lambda_1 t} + c_2 t e^{\lambda_1 t} + c_3 \f {t^2}2 e^{\lambda_1 t} \\
	\end{align*}
	Rücktransformation ergibt dann
	\[
		y(t) := T u(t)
	\]
\end{seg}
