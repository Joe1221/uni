% Stephan Hilb, 2012 Universität Stuttgart.
%
% Dieses Werk ist unter einer Creative Commons Lizenz vom Typ
% Namensnennung - Nicht-kommerziell - Weitergabe unter gleichen Bedingungen 3.0 Deutschland
% zugänglich. Um eine Kopie dieser Lizenz einzusehen, konsultieren Sie
% http://creativecommons.org/licenses/by-nc-sa/3.0/de/ oder wenden Sie sich
% brieflich an Creative Commons, 444 Castro Street, Suite 900, Mountain View,
% California, 94041, USA.

\section{Existenz und Eindeutigkeit} % 3 ?
\addtocounter{thmn}{1}
\setcounter{theorem}{0}


\begin{theorem}[Definition] \label{3.1}
  Sei $B$ ein Banachraum.
  \begin{enum-arab}
  \item
    Sei $f: \mathbb{R} \supset D \to B$ mit Häufungspunkt $x$ in $D$.
    $f$ heißt \emph{differenzierbar} in $x$, falls
    %
    \begin{align*}
      \exists f'(x) \in B : \bigg\| \frac{f(x+h)-f(x)}{h} - f'(x) \bigg\| \to 0
      \qquad k \to 0, x + k \in D
    \end{align*}
    %
  \item
    Für $-\infty < a < b < \infty$ heißt
    %
    \begin{align*}
      Z([a,b]) = \Big( (x_0,\dotsc,x_{N(Z)}, (\xi_1, \dotsc, \xi_{N(Z)}) \Big)
    \end{align*}
    %
    mit $a = x_0 < x_1 < \dotsc < x_{N(Z)} = b$ und $\xi_j \in [x_{j-1}, x_j]$ eine \emph{Zerlegung} von $[a,b]$.
    Weiter heißt
    %
    \begin{align*}
      S_Z(f) := \sum_{j=1}^{N(Z)} f(\xi_j)(x_j - x_{j-1})
    \end{align*}
    %
    \emph{Riemann-Summe}.
    Definiere die \emph{Feinheit} von $Z$ als
    %
    \begin{align*}
      \delta(Z) := \max_{1\le j \le N(Z)} (x_j-x_{j-1})
    \end{align*}
    %
    $f: [a,b] \to B$ heißt \emph{Riemann-integrierbar}, falls
    %
    \begin{align*}
      \exists I \in B : \|S_z(f) - I\| \to 0
      \qquad \delta(Z) \to 0
    \end{align*}
    %
    schreibe
    %
    \begin{align*}
      I := \int_a^b f(x) dx
    \end{align*}
    %
  \end{enum-arab}
\end{theorem}

\begin{theorem}[Korollar: Folgerung] \label{3.2}
  Für $a < c < b$ gilt
  %
  \begin{align*}
    \int_a^c f(x) dx + \int_c^b f(x) dx = \int_a^b dx
  \end{align*}
  %
\end{theorem}

\begin{theorem}[Satz] \label{3.3}
  Wenn $f: [a,b] \to B$ stetig ist, dann ist $f$ Riemann-integrierbar.
  \begin{proof}
    Zeige: falls $\delta(Z), \delta(Z') < \delta$, dann ist $\|S_Z(f) - S_{Z'}(f)\| < \varepsilon$.
    Für jede Folge $Z_n$ von Zerlegungen mit $\delta(Z_n) \to 0$ ($n\to\infty$) gilt dann.
    \begin{enum-arab}
    \item
      $(S_{Z_n}(f))$ ist Cauchy-Folge in $B$, also konvergent.
    \item
      $I := \lim_{n\to \infty} S_{Z_n}(f)$ ist unabhängig von der Folge $(Z_n)$.
    \end{enum-arab}
    Sei $\varepsilon > 0$ vorgegeben und $Z, Z'$ zwei Zerlegungen.
    Definiere
    %
    \begin{align*}
      Z'' = \Big( (x_0'', \dotsc, x_{N(Z)}'') , (\xi_1'', \dotsc, \xi_{N(Z)}'') \Big)
    \end{align*}
    %
    durch
    %
    \begin{align*}
      \{x_0'', \dotsc, x_{N(Z'')}''\} := \{x_0,\dotsc, x_{N(Z)}\} \cup \{ x_0', \dotsc, x_{N(Z)}' \}
      \qquad \xi_j'' \in [x_{j-1}'', x_j''] \text{ beliebig}
    \end{align*}
    %
    Benutze
    %
    \begin{align*}
      \|S_Z(f) - S_{Z'}(f)\| \le \|S_Z(f) - S_{Z''}(f)\| + \|S_{Z''}(f) - S_{Z'}(f)\|
    \end{align*}
    %
    % FIXME: Zeichnung

    %
    \begin{align*}
      \|S_Z(f) - S_{Z''}(f)\| 
      &= \bigg\| \sum_{j=1}^{N(Z)} f(\xi_j) (x_j-x_{j-1}) - \sum_{j=1}^{N(Z'')} f(\xi_j'')(x_j'' - x_{j-1}'') \bigg\| \\
      &= \bigg\| \sum_{j=1}^{N(Z'')} f(\xi_{J(j)} (x_j''-x_{j-1}'') - \sum_{j=1}^{N(Z'')} f(\xi_j'')(x_j'' - x_{j-1}'') \bigg\| \\
      \intertext{
        Dabei ist $J(j)$ so definiert, dass $[x_{j-1}'',x_j''] \subset [x_{J(j)-1},x_{J(j)}]$, d.h. 
        \begin{align*}
          J(j) := \min \Big\{ k\in \mathbb{N} : x_j'' \le x_k \Big\}
        \end{align*}
        %
      }
      &\le \sum_{j=1}^{N(Z'')} \| f(\xi_{J(j)}) - f(\underbrace{\xi_j''}_{\in [x_{J(j)-1},x_{J(j)}]})\| (x_j''-x_{j-1}'')
      \intertext{Da $| \xi_{J(j)}-\xi_J''| \le \delta(Z)$ und $f$ gleichmäßig stetig ($f$ stetig und $[a,b]$ kompakt), ist
        %
        \begin{align*}
          \|f(\xi_{J(j)}) - f(\xi_j'')\| < \frac{\varepsilon}{2(b-a)}
          \qquad \text{für } \delta(Z)  < \delta
        \end{align*}
        %
      }
      &\le \frac {\varepsilon}{2(b-a)} \underbrace{\sum_{j=1}^{N(Z'')} (x_j''-x_{j-1}'')}_{=b-a} = \frac {\varepsilon}2
    \end{align*}
    Verfahre analog für den anderen Teil, dann ergibt sich
    %
    \begin{align*}
      \|S_Z(f) - S_{Z'}(f)\| < \frac \varepsilon 2 + \frac \varepsilon 2 = \varepsilon
      \qquad \delta(Z),\delta(Z') < \delta
    \end{align*}
    %
  \end{proof}
\end{theorem}

\begin{theorem}[Satz] \label{3.4}
  Wenn $f:[a,b] \to B$ stetig ist, dann gilt
  %
  \begin{align*}
    \bigg\| \int_a^b f(x) dx \bigg\| \le \int_a^b \|f(x)\| dx
  \end{align*}
  %
  \begin{proof}
    Schreibe
    %
    \begin{align*}
      \bigg\|\int_a^b f(x) dx \bigg\| 
      &= \lim_{\delta(Z)\to 0} \|S_Z(f)\| \\
      &= \lim_{\delta(Z)\to 0} \bigg\| \sum_{j=1}^{N(Z)} f(\xi_j) (x_{j} x_{j-1}) \bigg\| \\
      &= \limsup_{\delta(Z) \to 0} \underbrace{\sum_{j=1}^{N(Z)} \| f(\xi_j) \| (x_j - x_{j-1}}_{= S_Z(\|f\|) \to \int_a^b \|f(x)\| dx} \\
      &= \int_a^b \|f(x)\| dx
    \end{align*}
    %
  \end{proof}
\end{theorem}

\begin{theorem}[Satz: Hauptsatz] \label{3.5}
  Sei $f: [a,b] \to B$ stetig und $F(x) := \int_a^b f(\xi) d\xi$ ($a \le x \le b)$.

  Dann ist $F$ differenzierbar und 
  %
  \begin{align*}
    F'(x) = f(x) 
    \qquad a \le x \le b
  \end{align*}
  %
  \begin{proof}
    Für $h > 0$ betrachte
    %
    \begin{align*}
      \Big\| \tfrac 1h \underbrace{(F(x+h) - F(x))}_{\mathclap{= \int_x^{x+h}f(\xi) d\xi \text{ (Additivität Integral)}}} - \overbrace{f(x)}^{\mathclap{\frac 1h \int_{x}^{x+h}f(x) d\xi}} \Big\|
      &= \bigg\| \frac 1h \int_x^{x+h} (f(\xi) - f(x)) d\xi \bigg\|  \\
      &\le \frac 1h \int_x^{x+h} \underbrace{\|f(\xi) - f(x)\|}_{\mathclap{< \varepsilon \text{ für } |\xi - x| \le h < \delta \text{ da $f$ stetig in $x$}}} d\xi \\
      &\le \frac 1h \int_x^{x+h} \varepsilon d\xi = \varepsilon
    \end{align*}
    %
    für $h < \delta$.
  \end{proof}
\end{theorem}

\begin{theorem}[Korollar: Folgerung] \label{3.6}
  Falls $G \in C^1([a,b] \to B)$ und $G' = f$ ($G$ \emph{Stammfunktion} von $f$), dann ist
  %
  \begin{align*}
    (G - F)' = 0
  \end{align*}
  %
  Also (ohne Beweis) $G = F + c$ mit $c \in B$.
  Damit ist das Integral
  %
  \begin{align*}
    \int_a^b f(x) dx 
    &= F(b) - F(a) \\
    &= G(b) - c - (G(a) - c) \\
    &= G(b) - G(a)
  \end{align*}
  %
  unabhängig von der Wahl der Stammfunktion.
\end{theorem}

\begin{theorem}[Satz: Picard-Lindelöf] \label{3.7}
  Sei $(B, \|\cdot\|)$ ein Banachraum, $(x_0,y_0) \in \mathbb{R} \times B$, $I=[x_0-r, x_0+r] \subset \mathbb{R}$ ein Intervall und $D := \{y \in B : \|y-y_0\| \le R\}$ für ein $R \in \mathbb{R}$.
  Sei $f \in C(I \times D \to B)$ mit
  %
  \begin{align*}
    \exists L > 0 \; \forall(x,y), (x,\tilde y) \in I \times D : \|f(x,y) - f(x,\tilde y) \| \le L\|y-\tilde y\|
  \end{align*}
  %
  d.h. $f$ erfüllt eine \emph{Lipschitz-Bedingung} bezüglich zwei Variablen.
  Weiter seien
  %
  \begin{align*}
    M := \sup_{I \times D} \|f(x,y)\| < \infty
    \qquad
    \delta := \min \bigg\{ \frac 1{2L}, \frac RM, r \bigg\}
  \end{align*}
  %
  Dann existiert eine eindeutige Lösung $y$ von
  \begin{align} \label{eq:3.7.1}
    &\quad y \in C^1([x_0 - \delta, x_0 + \delta] \to B) \\
    \land&\quad y'(x) = f(x,y(x)) \qquad x_0 - \delta \le x \le x_0 + \delta \\
    \land&\quad y' % FIXME: \text{hier fehlt was}
  \end{align}
  % FIXME: Veranschaulichung falls $B = \mathbb{R}$, Zeichnung
  .
  \begin{proof}
    \begin{enum-arab}
    \item
      Formuliere eine äquivalente Integralgleichung.
      $y$ ist genau dann eine Lösung von \eqref{eq:3.7.1}, wenn
      \begin{align} \label{eq:3.7.2}
        y \in C ( [x_0 - \delta, x_0 + \delta] \to B)
        \qquad \land \qquad
        y(x) = y_0 + \int_{x_0}^x f(\xi, y(\xi)) d\xi
      \end{align}
      mit $\int_{x_0}^{x} \dotso = \int_x^{x_0} \dotso$ falls $x < x_0$.
      \begin{proof}
        $\eqref{eq:3.7.1} \implies \eqref{eq:3.7.2}$

        Integriere die Differentialgleichung in \eqref{eq:3.7.1}:
        %
        \begin{align*}
          \int_{x_0}^x y'(\xi) d\xi &= \int_{x_0}^x f(\xi, y(x)) d\xi
          y(x) - \underbrace{y(x_0)}_{=y_0} &= \int_{x_0}^x f(\xi, y(x)) d\xi
        \end{align*}
        %

        $\eqref{eq:3.7.2} \implies \eqref{eq:3.7.1}$

        $f(\xi, y(\xi))$ ist stetig in $\xi$.
        Nach dem Hauptsatz ist $y' = 0 + f(x, y(x))$ mit $y'$ stetig.
        Für $x = x_0$ in der Integralgleichung: $y(x_0) = y_0 + 0$.
      \end{proof}
    \item
      Zeige, dass $\eqref{eq:3.7.2}$ eine eindeutige Lösung besitzt.
      \begin{proof}
        Sei $(\tilde B, \|\cdot\|^{\sim}) := (C([x_0-\delta, x_0+\delta] \to B), \|\cdot\|_\infty)$ ein Banachraum und
        %
        \begin{align*}
          \tilde D := C([x_0 -\delta, x_0 + \delta] \to D) 
        \end{align*}
        %
        (abgeschlossene Teilmenge in $\tilde B$)
        Definiere $T: \tilde D \to \tilde B : y \to F(y)$ durch
        %
        \begin{align*}
          F(y)(x) := y_0 + \int_{x_0}^x f(\xi, y(\xi)) d\xi
        \end{align*}
        %
        \begin{enum-alph}
        \item
          $F(y)$ ist stetig und
          %
          \begin{align*}
            \|F(y(x)) - y_0 \|
            \stackrel{\ref{3.4}}\le \bigg| \int_{x_0}^x \underbrace{\|f(\xi, y(\xi))\|}_{\le M} d\xi \bigg|
            \le M \cdot |x - x_0|
            \le M \cdot \delta
            \stackrel{\text{def } \delta}\le R
          \end{align*}
          %
          Also $F(\tilde D ) \le \tilde D$.
        \item
          Außerdem ist $\tilde D \neq \emptyset$, denn $\phi \in \tilde D$ für $\phi(x) := y_0$.
        \item
          $F$ ist eine Kontraktion.
          %
          \begin{align*}
            \|F(y) - F(\tilde y)\|_\infty
            &= \sup_{x_0-\delta \le x \le x_0+\delta} \bigg\| \int_{x_0}^x \Big( f(\xi, y(\xi)) - f(\xi, \tilde y(\xi)) \Big) d\xi \bigg\| \\
            &= \sup_{x_0-\delta \le x \le x_0+\delta} \bigg| \int_{x_0}^x \underbrace{\Big\| f(\xi, y(\xi)) - f(\xi, \tilde y(\xi)) \Big\|}_{\le L \|y(\xi)-\tilde y(\xi)\| \le L \|y-\tilde y\|_\infty} d\xi \bigg| \\
            &\le L \|y-\tilde y\|_\infty \cdot \underbrace{|x-x_0|}_{\le \delta} \\
            &\stackrel{L \delta \le \frac 12} \le \underbrace{\frac 12}_{=q} \|y-\tilde y\|_\infty
          \end{align*}
          %
        \end{enum-alph}
        Der Banachsche Fixpunktsatz besagt jetzt
        %
        \begin{align*}
          \exists! y \in \tilde B : F(y) = y
        \end{align*}
        %
        also $y \in C([x_0-\delta, x_0+\delta] \to B)$ und aus $F(y) = y$ ergibt sich
        %
        \begin{align*}
          y_0  + \int_{x_0}^x f(\xi, y(\xi)) d \xi = y(x)
        \end{align*}
        %
      \end{proof}
    \end{enum-arab}
    Die Äquivalenz zur ursprünglichen DGL liefert somit den Beweis dafür, dass die Lösung eindeutig ist.
  \end{proof}
\end{theorem}


