% Stephan Hilb, 2012 Universität Stuttgart.
%
% Dieses Werk ist unter einer Creative Commons Lizenz vom Typ
% Namensnennung - Nicht-kommerziell - Weitergabe unter gleichen Bedingungen 3.0 Deutschland
% zugänglich. Um eine Kopie dieser Lizenz einzusehen, konsultieren Sie
% http://creativecommons.org/licenses/by-nc-sa/3.0/de/ oder wenden Sie sich
% brieflich an Creative Commons, 444 Castro Street, Suite 900, Mountain View,
% California, 94041, USA.

\section{Funktionalanalysis} % Nummerierung?


\begin{theorem}[Definition] \label{1.1}
  Sei $(B,+,\cdot)$ ein linearer Raum (d.h. ein Vektorraum) und $\|\cdot\|: B \to \mathbb{R}$ eine Norm (d.h. $\|u\| \ge 0$, $\|u\|=0 \iff u = 0$, $\|u+v\| \le \|u\| + \|v\|$ und $\|\alpha u \| = |\alpha| \|u\|$).

  Ist $B$ vollständig, d.h. jede Cauchy-Folge $(a_k)_{k\in \mathbb{N}} \in B$ konvergiert gegen einen Grenzwert in $B$, so nennt man $B$ mit der Norm $\|\cdot\|$ \emph{Banachraum}.
\end{theorem}

\begin{example} \label{1.2}
  \begin{enum-arab}
  \item
    Die komplexen Zahlen mit der $p$-Norm
    %
    \begin{align*}
      B := \mathbb{C}^n,
      \qquad \|u\| := (|u_1|^p + \dotsb + |u_n|^p)^{\frac 1p} \qquad 1 \le p < \infty
    \end{align*}
    %
  \item
    Sei $B = C([a,b] \to \mathbb{C})$ mit der Norm
    %
    \begin{align*}
      \|f\|_\infty = \max_{a \le x \le b} |f(x)|
    \end{align*}
    %
    z.B. $\|f\|_2^2 = \int_a^b |f(x)|^2 dx$, dann ist $B$ kein Banach-Raum (% FIXME: warum?
    ).
  \item
    Definiere
    %
    \begin{align*}
      L^p(I) := \bigg\{ f : I \to \mathbb{C} : f \text{ messbar} \land \int_I |f|^p d\mu < \infty \bigg\}
    \end{align*}
    %
    für ein Intervall $I \subset \mathbb{R}$.
    Dann bildet $L^p(I)$ mit der Norm
    %
    \begin{align*}
      \|f\|_p := \bigg( \int_I |f|^p d\mu \bigg)^{\frac 1p}
    \end{align*}
    %
    (Identifiziere $f,\tilde f$, falls $\int_I |f-\tilde f| d\mu = 0$)
    einen Banachraum.
  \end{enum-arab}
\end{example}


\begin{theorem}[Satz: Banachscher Fixpunktsatz] \label{1.3}
  Sei $B$ ein Banachraum, $\emptyset \neq D \subset B$ mit $D$ abgeschlossen und $F: D \to B$ eine \emph{Kontraktion}, d.h.
  %
  \begin{align*}
    \exists q \in [0,1[ \; \forall x,y \in D : \|F(x) - F(y)\| \le q\|x-y\|
  \end{align*}
  %
  mit $F(D) \subset D$. Dann gilt
  \begin{enum-arab}
  \item
    Es existiert $x\in D$ mit $F(x)=x$, d.h. die Abbildung $F$ hat genau einen \emph{Fixpunkt} $x\in D$.
  \item
    Ist $x_0 \in D$ und $x_n := F(x_{n-1})$ für $n\in \mathbb{N}$, so gilt
    %
    \begin{align*}
      x_n \to x \qquad n \to \infty
    \end{align*}
    %
    mit der \emph{Fehlerabschätzung}
    %
    \begin{align*}
      \|x_n - x\| \le \frac{q^n}{1-q} \|x_1 - x_0\|
    \end{align*}
    %
  \end{enum-arab}
  \begin{proof}
    \begin{enum-arab}
    \item
      Eindeutigkeit: Seien $x,y$ zwei Fixpunkte, dann ist
      %
      \begin{align*}
        \|x-y\| = \|F(x) - F(y)\| \le q \|x-y\|
      \end{align*}
      %
      wegen $q < 1$ folgt $\|x-y\| = 0$, also $x=y$.
    \item
      Wegen $F(D) \subset D$ ist $x_n$ für $n\in \mathbb{N}_0$ wohldefiniert.

      Zeige zunächst
      %
      \begin{align*}
        \|x_{n+1} - x_n\| \le q^n \|x_1 - x_0\|
      \end{align*}
      %
      induktiv.

      Für den Rest (\ref{1.4}) siehe Numerik (wurde dort schöner bewiesen).
    \end{enum-arab}
  \end{proof}
\end{theorem}

\begin{notice} \label{1.4}
  Es lässt sich ebenfalls zeigen:
  %
  \begin{align*}
    \|x_n - x\| \le \frac 1{1-q} \|x_{n+1} -x_n \|
  \end{align*}
  %
\end{notice}
