% Stephan Hilb, 2012 Universität Stuttgart.
%
% Dieses Werk ist unter einer Creative Commons Lizenz vom Typ
% Namensnennung - Nicht-kommerziell - Weitergabe unter gleichen Bedingungen 3.0 Deutschland
% zugänglich. Um eine Kopie dieser Lizenz einzusehen, konsultieren Sie
% http://creativecommons.org/licenses/by-nc-sa/3.0/de/ oder wenden Sie sich
% brieflich an Creative Commons, 444 Castro Street, Suite 900, Mountain View,
% California, 94041, USA.

\section{Funktionalanalysis} % Nummerierung?


\begin{df} \label{1.1}
	Sei $(B,+,\cdot)$ ein linearer Raum (d.h. ein Vektorraum) und $\|\cdot\|: B \to \R$ eine Norm (d.h. $\|u\| \ge 0$, $\|u\|=0 \iff u = 0$, $\|u+v\| \le \|u\| + \|v\|$ und $\|\alpha u \| = |\alpha| \|u\|$).

	Ist $B$ vollständig, d.h. jede Cauchy-Folge $(a_k)_{k\in \N} \in B$ konvergiert gegen einen Grenzwert in $B$, so nennt man $B$ mit der Norm $\|\cdot\|$ \emph{Banachraum}.
\end{df}

\begin{ex} \label{1.2}
	\begin{enumerate}[1)]
		\item
			Die komplexen Zahlen mit der $p$-Norm
			\[
				B := \C^n,
				\qquad \|u\| := (|u_1|^p + \dotsb + |u_n|^p)^{\f 1p} \qquad 1 \le p < \infty
			\]
		\item
			Sei $B = C([a,b] \to \C)$ mit der Norm
			\[
				\|f\|_\infty = \max_{a \le x \le b} |f(x)|
			\]
			z.B. $\|f\|_2^2 = \int_a^b |f(x)|^2 dx$, dann ist $B$ kein Banach-Raum (\fixme[warum?]).
		\item
			Definiere
			\[
				L^p(I) := \bigg\{ f : I \to \C : f \text{ messbar} \land \int_I |f|^p d\my < \infty \bigg\}
			\]
			für ein Intervall $I \subset \R$.
			Dann bildet $L^p(I)$ mit der Norm
			\[
				\|f\|_p := \bigg( \int_I |f|^p d\my \bigg)^{\f 1p}
			\]
			(Identifiziere $f,\tilde f$, falls $\int_I |f-\tilde f| d\my = 0$)
			einen Banachraum.
	\end{enumerate}
\end{ex}


\begin{st}[Banachscher Fixpunktsatz] \label{1.3}
	Sei $B$ ein Banachraum, $\emptyset \neq D \subset B$ mit $D$ abgeschlossen und $F: D \to B$ eine \emph{Kontraktion}, d.h.
	\[
		\exists q \in [0,1[ \; \forall x,y \in D : \|F(x) - F(y)\| \le q\|x-y\|
	\]
	mit $F(D) \subset D$. Dann gilt
	\begin{enumerate}[1)]
		\item
			Es existiert $x\in D$ mit $F(x)=x$, d.h. die Abbildung $F$ hat genau einen \emph{Fixpunkt} $x\in D$.
		\item
			Ist $x_0 \in D$ und $x_n := F(x_{n-1})$ für $n\in \N$, so gilt
			\[
				x_n \to x \qquad n \to \infty
			\]
			mit der \emph{Fehlerabschätzung}
			\[
				\|x_n - x\| \le \f{q^n}{1-q} \|x_1 - x_0\|
			\]
	\end{enumerate}
	\begin{proof}
		\begin{enumerate}[1)]
			\item
				Eindeutigkeit: Seien $x,y$ zwei Fixpunkte, dann ist
				\[
					\|x-y\| = \|F(x) - F(y)\| \le q \|x-y\|
				\]
				wegen $q < 1$ folgt $\|x-y\| = 0$, also $x=y$.
			\item
				Wegen $F(D) \subset D$ ist $x_n$ für $n\in \N_0$ wohldefiniert.

				Zeige zunächst
				\[
					\|x_{n+1} - x_n\| \le q^n \|x_1 - x_0\|
				\]
				induktiv.

				Für den Rest (\ref{1.4}) siehe Numerik (wurde dort schöner bewiesen).
		\end{enumerate}
	\end{proof}
\end{st}

\begin{nt} \label{1.4}
	Es lässt sich ebenfalls zeigen:
	\[
		\|x_n - x\| \le \f 1{1-q} \|x_{n+1} -x_n \|
	\]
\end{nt}

\setcounter{section}{0}
\section{Beispiele} % 2.1

\begin{ex}[Tee] \label{2.1}
	Beschreibe $y(t)$ die Temperatur des Tee's und $y_A = \const$ die Außentemperatur.
	Es gilt folgende Differentialgleichung
	\[
		y'(t) = -K( y(t) - y_A)
	\]
	Die Lösung lautet
	\[
		y(t) = y_A + c e^{-Kt} \qquad t\in \R, c\in \R
	\]
	Probe:
	\[
		y' = c(-K)e^{-Kt} = -K(ce^{-Kt}) = -K(y(t) - y_A)
	\]
	Die Lösung ist erst eindeutig, wenn z.B. $y(t_0) = y_0$ vorgegeben wird (\emph{Anfangsbedingung}).
\end{ex}

\begin{df}[nicht-formale Beschreibung] \label{2.1}
	Eine \emph{Differentialgleichung} ist eine Glichung für eine gesuchte Funktion $y$, in der auch die Ableitung(en) von $y$ auftreten.
	Sie heißt \emph{gewöhnlich}, falls keine partiellen Ableitungen auftreten, sonst \emph{partiell}
\end{df}
