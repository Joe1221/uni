\chapter{Arithmetik modulo $n$}


\section{Struktur der Einheitengruppe $U(\Z/ n\Z)$}


% St 2.1
\begin{st} \label{2.1}
	Die multiplikative Gruppe $K^*$ eines endlichen Körpers $K$ ist zyklisch.
	Allgemein gilt: Jede endliche Untergruppe von $K^*$, wobei $K$ ein beliebiger Körper, ist zyklisch.
	\begin{note}
		\ref{2.1} klärt insbesondere die Struktur von $\Z / p\Z$ mit Primzahl $p$: $U(\Z / p\Z)$ ist zyklisch von Ordnung $p - 1$.

		Zum Beweis von \ref{2.1} werden gruppentheoretische Grundlagen benötig, welche wir im Folgenden teilweise wiederholen werden.
	\end{note}
\end{st}

\begin{df*}
	Die Ordnung einer Zahl $g \in G$ ist die kleinste Zahl $m \in \N$ mit $g^m = e$.
\end{df*}

% St 2.2
\begin{st} \label{2.2}
	Sei $G$ eine endliche Gruppe und $U \le G$ eine Untergruppe.
	\begin{enumerate}[a)]
		\item
			Lagrange: $|U|$ teilt $|G|$.
		\item
			Für $g \in G$ gilt $o(g) \divs |G|$.
	\end{enumerate}
\end{st}

\begin{df*}
	Wir setzen
	\[
		\pi(G) := \{ x \in \P : x \divs |G| \}.
	\]
\end{df*}

% St 2.3
\begin{st} \label{2.3}
	Sei $G$ eine endliche abelsche Gruppe und $p \in \pi(G)$, also $|G| = p^a n$ mit $(p, n) = 1$.
	Setze
	\[
		P = \{ g \in G : o(g) = p^m \text{ für ein $m \in \N_0$} \}.
	\]
	$P$ bildet eine Untergruppe der Ordnung $p^a$.

	Man nennt $P$ \emphdef{$p$-Sylowgruppe} von $G$.
	Zerlege $|G|$ in Primfaktoren
	\[
		|G| = p_1^{a_1} \dotsb p_k^{a_k}
	\]
	und $P_i$ sei die $p_i$-Sylowgruppe von $G$, dann ist $G \isomorphic P_1 \times \dotsb P_k$ ($G$ ist direktes Produkt ihrer Sylowgruppen).
	\begin{note}
		\ref{2.3} kann man leicht aus dem Hauptsatz für endlich erzeugte abelsche Gruppen herleiten.
		Trotzdem liefern wir einen direkten Beweis.
	\end{note}
	\begin{proof}
		Seien $x, y \in P$, also $o(x) = p^{m_1}, o(y) = p^{m_2}$.
		Damit ist $o(xy) = p^{\max\{m_1, m_2\}}$ (hier wird $G$ abelsch benötigt) und $P$ ist multiplikativ abgeschlossen.
		Da $P$ endlich, ist $P \le G$.
		Da $G$ abelsch, ist $P \normalsubgroup G$.
		Annahme $G / P$ hätte ein Element $xP$ der Ordnung $p$.
		Dann ist $x^p \in P$ ($x \not\in P$, da $o(x^p) = p \neq 1$).
		$x$ hat $p$-Potenzordnung, ein Widerspruch zu $x \not\in P$.
		In $G / P$ gibt es also keine Elemente der Ordnung $p$.

		Für $G = P$ ist \ref{2.3} sicherlich richtig.
		Also verwende Induktion nach $|\pi(G)|$ und Gruppenordnung.
		Wäre $p \in \pi(G / P)$, dann hätte $G / P$ nach Induktion eine nicht-triviale $p$-Sylowgruppe, $G / P$ hätte Elemente der Ordnung $p$.
		Nach obigem folgt $p \not\in \pi(G / P)$, aber $|G| = |P| \cdot |G / P|$, also $|P| = p^a$.

		Zum letzten Teil:
		Sei \oBdA $k > 1$.
		Nach den ersten Teil existiert zu jedem $p_i$ eine $p_i$-Sylowgruppe $P_i$.
		Setze $\tilde P_1 = \< P_i : i \ge 2 \> = P_2 \cdot P_3 \dotsb P_k$.
		Mit \ref{2.5} sieht man, dass $(|\tilde P_1|, |P_1|) = 1, \tilde P_1 \normalsubgroup G_1, P_1 \normalsubgroup G$.
		Damit ist nach \ref{2.4} $G \isomorphic P_1 \times \tilde P_1$ und somit $G \isomorphic P_1 \times P_2 \times \dotsb P_k$.
	\end{proof}
\end{st}

% Lem 2.4
\begin{lem} \label{2.4}
	Sind $M, N \normalsubgroup G$ und $M \cap N = 1$.
	Dann ist $M \cdot N \isomorphic M \times N$.
	\begin{proof}
		Definiere $\phi: M \times N \to M\cdot N$ durch $(m, n) \mapsto mn$.
		$\phi$ ist injektiv, denn aus $m_1 n_1 = m_2 n_2$ folgt $M \ni m_2^{-1} m_1 = n_2 n_1^{-1} \in N$, also $m_2^{-1}m_1 = 1 = n_2 n_1^{-1}$ und somit $m_1 = m_2 \land n_1 = n_2$.
		Zeige $\phi$ ist Gruppenhomomorphismus, indem wir zeigen, dass der \emphdef{Kommutator} $m n m^{-1} n^{-1} = 1$, also genau dann, wenn $m n = n m$.

		Es gilt $mn m^{-1}n^{-1} \in M, N$, also $mnm^{-1}n^{-1} = 1$.
		\begin{align*}
			\phi((m_1, n_1) \cdot (m_2, n_2))
			&= \phi(m_1m_2, n_1n_2) \\
			&= m_1 m_2 n_1 n_2 \\
			&= m_1 n_1 m_2 n_2
			&= \phi(m_1, n_1) \phi(m_2, n_2).
		\end{align*}
		$\phi$ surjektiv ist eine leichte Übung.
		Damit ist $\phi$ Gruppenisomorphismus.
	\end{proof}
\end{lem}

% Lem 2.5
\begin{lem} \label{2.5}
	Sind $M, N \normalsubgroup G$, dann ist $M\cdot N \normalsubgroup G$ und es gilt
	\[
		M / M\cap N
		\isomorphic
		MN / N.
	\]
	Insbesondere gilt $|MN| = |M|\cdot|N| \cdot \f 1{|M \cap N|}$.
\coursetimestamp{24}{04}{2014}
	\begin{proof}
		Zeige $M\cdot N \normalsubgroup G$.
		Zunächst ist $M \cdot N \le G$, denn
		\begin{align*}
			(mn)^{-1} &= n^{-1} m^{-1}
			= \underbrace{n^{-1}m^{-1}n}_{\in M} n^{-1}
		\end{align*}
		und für $m_1n_1, m_2n_2 \in \in M\cdot N$ ist
		\[
			m_1 n_1 m_2 n_2
			= \underbrace{m_1 m_1^{-1}}_{\in M} \underbrace{m_1 n_1 m_2}_{\in N} n_2
			\in M \cdot N.
		\]
		$M \cdot N$ ist Normalteiler:
		für $g \in G$ ist
		\[
			g^{-1} mn g
			= g^{-1} m g g^{-1} n g
			\in M \cdot N.
		\]
		Definiere jetzt $\phi: M / (M \cap N) \to (M \cdot N) / N$ durch
		\[
			m(M\cap N) \mapsto mN.
		\]
		$\phi$ ist wohldefiniert, denn wenn $m_1 (M \cap N) = m_2 (M \cap N)$, dann existiert $n \in M \cap N$ mit $m_1 n = m_2$, also $m_1 N = m_2 N$.
		$\phi$ ist injektiv, denn wenn $m_1 N = m_2 N$, existiert $\tilde n \in N: m_1 \tilde n = m_2$, also $\tilde n = m_1^{-1} m_2 \in M \cap N$ und somit $m_1 M\cap N = m_1 \tilde n M\cap N = m_2 M \cap N$.
		$\phi$ ist surjektiv, denn $m_1 n_1 N = m_1 N = \phi(m_1 M\cap N)$.
		$\phi$ ist Homomorphismus, denn
		\begin{align*}
			\phi(m_1 M\cap N \cdot m_2 M \cap N)
			&= \phi((m_1m_2)M\cap N) \\
			&= (m_1 m_2) N \\
			&= m_1N \cdot m_2 N \\
			&= \phi(m_1 M\cap N)
			= \phi(m_2 M\cap N).
		\end{align*}
		Falls $|M|, |N| < \infty$, dann ist $|M / (M\cap N) | = \f {|M|}{|M\cap N|} \stack{\phi}= \f {|MN|}{|N|}$.
	\end{proof}
\end{lem}

\begin{proof}[Letzter Schritt im Beweis von \ref{2.3}]
	Zu jedem $p_i \in \pi(G)$ existiert $P_i$.
	Setze $\tilde p_1 = \< P_2, \dotsc, P_k \> = P_2 \cdots \dotsb \cdots P_k$ (mittels Induktion aus \ref{2.5}) und $|\tilde p_1| = p_2^{\alpha_2} \cdot \dotsb \cdot p_k^{\alpha_k}$, also $\tilde P_1 \cap P_1 = 1$ und $P_1 \cdot \tilde P_1 = G$.
	Mit \ref{2.4} ist $G \isomorphic P_1 \times \tilde P_1$ und per Induktion somit
	\[
		\tilde P_1 = P_2 \times \dotsb \times P_k.
	\]
\end{proof}

\begin{nt*}
	Sind $M, N$ Normalteiler von $G$, dann ist $M \cdot N$ der kleinste Normalteiler von $G$, der $M$ und $N$ enthält.
	Es gilt $M \cdot N = \<M, N\>$.
	Sind $M, N$ keine Normalteiler, so gilt das nicht.
\end{nt*}

\begin{df}
	Eine Gruppe $G$ heißt \emphdef{$p$-Gruppe}, wenn $|G| = p^a$ ist für eine Primzahl $p$.
\end{df}

% Lem 2.6
\begin{lem} \label{2.6}
	Sei $\phi$ die Eulersche $\phi$-Funktion.
	Eine abelsche $p$-Gruppe $G$ ist genau dann zyklisch, wenn es zu jedem Teiler $t$ von $|G|$ höchstens $\phi(t)$ Elemente der Ordnung $t$ gibt (und es gibt dann auch genau $\phi(t)$ solche Elemente).
	\begin{proof}
		Zeige zunächst die Rückrichtung.
		Sei $w_1$ die Anzahl der Elemente der Ordnung $p^i$ von $G$, $|G| = p^m$.
		Sicherlich ist $|G| = \sum_{i=0}^m w_i$ (beachte $o(g) \divs |G|$, vgl. \ref{2.2}).
		Es gilt (Voraussetzung „höchstens“ benutzt)
		\[
			p^m = |G|
			= \sum_{i=0}^m \phi(p^i) \alpha_i
		\]
		mit $\alpha_i \in \{0, 1\}$, denn existiert ein Element der Ordnung $p^i$, dann gibt es $\phi(p^i)$ solche Elemente.
		Beachte: ist $o(g) = p^i$, dann ist $|\<g\>| = p^i$ und $\<g\> \isomorphic (\Z / p^i \Z, +)$.
		\begin{align*}
			\sum_{i=0}^m \phi(p^i) \alpha_i
			\le \sum_{i=0}^m \phi(p^i)
			&\stack{\ref{1.15}}= 1 + p-1 + \dotsb + (p-1) p^{m-1} \\
			&= 1 + (p-1) (1 + \dotsb + p^{m-1})
			= p^m
			= |G|
		\end{align*}
		Also gilt Gleichheit und somit insbesondere $\alpha_m = 1$ und es gibt Elemente der Ordnung $p^m$.
		Somit ist $G$ zyklisch.

		Die Hinrichtung ist eine leichte Übung.
	\end{proof}
\end{lem}

\begin{proof}[Beweis von \ref{2.1}]
	Sei $U \le K^*$ mit $|U| < \infty$.
	Nach \ref{2.3} ist $U \isomorphic P_1 \times \dotsb \times P_k$ und $P_i$ ist $p_i$-Sylowgruppe von $U$.
	Wir zeigen, dass $P_i$ zyklisch ist.

	Sei $d_i$ Teiler von $|P_i|$ und $g \in P_i$ mit $o(g) = d_i$.
	Also erfüllt $g$ die Gleichung $x^{d_i} - 1 = 0$.
	Diese Gleichung hat in $K$ höchstens $d_i$ verschiedene Lösungen.
	Ist $g$ eine Lösung, dann auch alle Potenzen von $g$.
	Die von $g$ erzeugte zyklische Gruppe $\<g\> \isomorphic \Z / d_i \Z$ stellt alle Lösungen der Gleichung.

	Es existieren also höchstens $\phi(d_i)$ Elemente der Ordnung $d_i$ in $P_i$.
	Mit \ref{2.6} ist also $P_i$ zyklisch, $P_i \isomorphic \Z / |P_i| \Z$ und
	\[
		U \isomorphic
		\Z / |P_1| \Z \times \dotsb \times \Z / |P_k| \Z
		\stack{\ref{1.14}}\isomorphic
		\Z / (|P_1| \cdot |P_k|) \Z,
	\]
	also ist $U$ zyklisch.
\end{proof}
\begin{note}
	In einem Schiefkörper sind im Allgemeinen nicht alle endlichen Untergruppen zyklisch, z.B. in $\H^*$ gibt es die Quaternionengruppe der Ordnung $8$.
\end{note}

Unser Ziel war es, die Struktur von $U(\Z / m \Z)$ zu untersuchen.
\ref{2.1} klärt dies für den Fall $\Z / p\Z$ mit Primzahl $p$:
\[
	(\Z / p \Z)^* \isomorphic C_{p-1},
\]
wobei $C_{p-1}$ die zyklische Gruppe der Ordnung $p-1$ ist (multiplikatives Analogon zu $\Z / p \Z$.

\begin{nt*}
	Für $K = \Z / p\Z$ würde man gerne Erzeuger (auch \emphdef{Primitivwurzel} genannt) von $(\Z / p\Z)^*$ angeben können.
	Insbesondere würde man gerne die kleinsten Erzeuger $k(p)$ kennen.
\end{nt*}

\begin{ex*}
	\begin{align*}
		k(3) &= 2, &
		k(5) &= 2, &
		k(7) &= 3, &
		k(11) &= 2, &
		k(5881) &= 31.
	\end{align*}
	Im Allgemeinen sind die Erzeuger unregelmäßig verteilt.
\end{ex*}

\begin{conj}[E. Artin]
	Sei $a \in \N$ und $a$ sei kein Quadrat, dann existiert eine ungerade Primzahl $p$ für die $a \pmod p$ eine Primitivwurzel ist.
	\begin{note}
		Warum darf $a$ kein Quadrat sein?
		siehe Übung.
	\end{note}
\end{conj}

Der nächste Schritt zur Struktur von $U(\Z / m \Z)$ ist die Betrachtung von Primzahlpotenzen $m = p^s$.

\begin{ex*}
	Es gilt
	\begin{align*}
		U(\Z / 4 \Z) &= \{ \_ 1, \_ 3 \} \isomorphic C_2 \\
		U(\Z / 8 \Z) &= \{ \_ 1, \_ 3, \_ 5, \_ 7 \} \isomorphic C_2 \times C_2
	\end{align*}
	$C_2 \times C_2$ ist nicht zyklisch.
\end{ex*}

Es wird sich zeigen, dann $U(\Z / p^s \Z)$ zyklisch ist, wenn $p$ ungerade Primzahl ist.
Zunächst ein Lemma zum Rechnen bezüglich Primzahlpotenzresten.

% Lem 2.7
\begin{lem} \label{2.7}
	Sei $p \in \P, s \in \N$ und $a, b \in \Z$.
	\begin{enumerate}[a)]
		\item
			$a \equiv b \pmod p^s \implies a^p \equiv b^p \pmod p^{s+1}$,
		\item
			$s \ge 2, p\neq 2$.
			Dann gilt
			\[
				(1 + ap)^{p^{s-2}}
				\equiv 1 + ap^{s-1} \mod p^s.
			\]
		\item
			$p \ndivs a, p \neq 2 \implies $ $(1+p) \pmod p^s$ hat Ordnung $p^{s-1}$ in $U(\Z / p^s \Z)$.
		\item
			$s > 2 \implies 5^{2^{s-3}} \equiv 1 + 2^{s-1} \pmod 2^s$, $5 \pmod 2^s$ hat die Ordnung $2^{s-2}$.
	\end{enumerate}
	\begin{proof}
		\begin{enumerate}[a)]
			\item
				Wenn $a \equiv b \pmod p^s$, dann ist $a - b = k p^s$ und
				\[
					a^p
					= (b + kp^s)^p
					= \sum_{j=0}^p \binom{p}{j} b^{p-j} (kp^s)^j
				\]
				Da $\binom{p}{r}$ durch $p$ teilbar ist, wenn $p$ Primzahl ist, gilt
				\[
					\sum_{j=1}^p \binom{p}{j} b^{p-j} (kp^s)^j \equiv 0 \mod p^{s+1}
				\]
				und damit $a^p \equiv b^p \pmod p^{s+1}$.
		\end{enumerate}
	\end{proof}
\end{lem}






