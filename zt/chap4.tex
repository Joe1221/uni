\chapter{Arithmetik in quadratischen Zahlenkörpern}

% Df 4.1
\begin{df} \label{4.1}
	Seien $K, L$ Körpre und $K \subset L$.
	Dann kann man $L$ als $K$-Vektorraum betrachten, wobei die Skalarmultiplikation
	\begin{align*}
		\cdot : K \times L &\to L \\
		(k,l) &\mapsto k \cdot l
	\end{align*}
	durch die Multiplikation in $L$ gegebene ist.
	Aus den Körperaxiomen folgen sofort alle Vektorraumaxiome.
\end{df}

% Df 4.2
\begin{df} \label{4.2}
	Unter einem \emphdef[quadratischer Zahlkörper]{quadratischen Zahlkörper} versteht man einen Körper $L$ mit $L \supset \Q$ und $\dim_{\Q} L = 2$.

	Allgemeiner sind \emphdef{Zahlkörper} Körper $L$ mit $L \supset \Q$ und $\dim_\Q L \le \infty$.
\end{df}

% Ex 4.3
\begin{ex} \label{4.3}
	\begin{enumerate}[a)]
		\item
			Statt $L \supset \Q$ schreibt man $L / \Q$.
			$\Q[i] / \Q$ ist ein quadratischer Zahlkörper, wobei $\Q[i] = \{ q_1 + q_2 i : q_j \in \Q \}$.

			Allgemeiner ist $\Q[\sqrt d] = \{ q_1 + q_2 \sqrt d : a_j \in \Q \}$ und $d \in \Z$ ist quadratischer Zahlkörper.
			Man kann annehmen, dass $d$ quadratfrei ist, welche Wurzel gemeint ist, ist egal, da $\Q[\sqrt d] = \Q[-\sqrt d]$.
			Auch ist $\Q[i] = \Q[\sqrt{-1}]$.
		\item
			Ist $L / \Q$ quadratischer Zahlkörpre, dann gibt für $\alpha \in L \setminus \Q$, dass $\alpha$ eine Nullstelle eines Polynoms vom Grad 2 in $\Q[x]$ ist.
			\begin{proof}
				$1, \alpha, \alpha^2$ ist linear abhängig über $\Q$.
				Es existiert also eine nichttriviale Linearkombination $q_1 \cdot 1 + q_2 \alpha + q_3 \alpha^2 = 0$.
				$\alpha$ ist Nullstelle von $q_1 + q_2 x + q_3 x^2$.
				Beachte $q_3 \neq 0$, denn sonst wäre $\alpha \in \Q$.
			\end{proof}
			Mit der Mitternachtsformel folgt, dass jeder quadratischer Zahlkörper von der Form $\Q[\sqrt d]$ ist.

			Da $\Q[\sqrt d]$ ein Körpre, gilt: $\Q[\sqrt d] = \Q(\sqrt d)$, wobei letztere Notation für den kleinsten Körper innerhalb $\C$ steht, der $\Q$ und $\sqrt{d}$ enthält.
		\item
			Setze
			\[
				\alpha := \begin{cases}
					\sqrt d & d \equiv 2, 3 \bmod 4 \\
					\f{1 + \sqrt d}2 & d \equiv 1 \bmod 4
				\end{cases}.
			\]
			Im Fall $d \equiv 1 \bmod 4$ ist $\Z[\sqrt a] = \{ z_1 + z_2 \sqrt{d} : z_j \in \Z \}$ ein echter Teilring von $\Z[\alpha]$ und $\Z[\alpha] = \{z_1 + z_2 \alpha : z_j \in \Z \}$ ist ein Teilring von $\Q[\sqrt d]$.

			Setzt man $\alpha = \f{1 + \sqrt d}2$ im Fall $d \equiv 2,3 \bmod 4$, dann ist $\Z[\alpha]$ kein Teilring.
	\end{enumerate}
\end{ex}

Ziel dieses Kapitels soll die Untersuchung von $\Z[\alpha]$ sein, insbesondere daraufhin, ob es ein Hauptidealbereich ist.
Klar ist, dass $\Z[\alpha]$ als Teilring von $\Q[\sqrt{d}]$ stets ein Integritätsbereich ist.
Was kann man über das Verhältnis von prim und irreduzibel aussagen?.
Wie sehen die Einheitengruppen aus?

Es gilt stets $\Z \subset \Z[\alpha]$ und in $\Z$ sind Primzahlen prim.
Was passiert mit diesen Primzahlen in $\Z[\alpha]$?

Alle diese Fragen führen in die algebraische Zahlentheorie.

% Lem 4.4
\begin{lem} \label{4.4}
	In $\Z[i]$ sind $1, -1, i, -i$ die einzigen Einheiten.
	Es gilt: $U(\Z[i]) = \< i \> \isomorphic \isomorphic C_4$.
	\begin{proof}
		Sei $a + bi \in U(\Z[i])$ mit
		$(a + bi)(c + di) = 1$, wobei $a, b, c, d \in \Z$.
		Dann ist
		\[
			(ac - bd)\cdot 1 + (ad + bc) i = 1
		\]
		Da $\{1, i\}$ eine Basis von $\Q[i]$ ist, gilt $ac - bd = 1$ und $ad +  bc = 0$.
		Falls $a = 0$, dann ist $bc = 0$, dann ist $b = 0 \lor c = 0$.
		Außerdem ist $-bd = 1$, also $b,d \in \{\pm 1\}$ und somit $b \neq 0$, also $c = 0$.
		Damit ist $a + bi = i$ und $c + di = -i$, oder umgekehrt.

		Falls $a \neq 0$, dann ist $d = - \f{bc}a$ und $ac + \f{b^2 c}a = 1$, also $(a^2 + b^2)c = a$ und insbesondere $(a^2 + b^2)|c| = |c|$.
		Es gilt $|z| = |z_1 + z_2 i| = \sqrt{z_1^2 + z_2^2} \ge 1$ nur, wenn wenn $z \neq 0$.
		Also ist $b = 0$ und $|c| = 1$ und $ac = 1$
		Somit sind $1, -1$ die einzigen Einheiten dieser Art.
	\end{proof}
\end{lem}

% Df 4.5
\begin{df} \label{4.5}
	Sei $R$ ein kommutativer Ring.
	\begin{enumerate}[a)]
		\item
			$R$ heißt \emphdef{euklidisch}, wenn es eine Abbildung $\phi: \R^* \to \N_0$ gibt, mit
			\[
				\forall a, b \in R, b \neq 0 \exists q,r \in R : (a = qb + r) \land (r = 0 \lor \phi(r) < \phi(b)).
			\]
			$\phi$ nennt man \emphdef{Gradfunktion}.
		\item
			Ein Integritätsbereich $R$ heißt \emphdef{faktoriell}, wenn sich jede Nichteinheit verschieden von Null als endliches Produkt von irreduziblen Elementen schreiben lässt.
		\item
			$R$ heißt \emphdef{noethersch}, wenn jedes Ideal endlich erzeugt ist.
	\end{enumerate}
\end{df}
