\chapter{Primzahlverteilung, Primzahlsatz} \label{chap:7}


% Df 7.1
\begin{df} \label{7.1}
	Für $x \in \R$ bezeichne $\pi(x)$ die Anzahl der Primzahlen $\le x$, also
	\[
		\pi(x)
		:= \l| \Set{ p \in \P & p \le x } \r|
		= \sum_{\substack{p \in \P}{p \le x}} 1.
	\]
	Mann nennt $\pi$ die \emphdef{Primzahlfunktion}.
\end{df}

\begin{nt*}
	\begin{enumerate}[a)]
		\item
			Es gilt $\lim_{x \to \infty} \pi(x) = \infty$, z.B. nach Euklid.
			$\pi$ ist monoton steigend.
		\item
			Man stellt sich die Frage: „Wie geht $\pi(x)$ gegen $\infty$?“.
			Sicherlich ist $\pi(x) \le x$ für $x \in \R^+$.
			Man sieht auch $\pi(x) \le \f x2$ für $x \ge 8$.

			Analog sieht man mit dem Sieb des Erastosthenes, dass für beliebiges $n \in \N$ die Aussage $\pi(x) \le \f xn$ für hinreichend großes $x$ erfüllt ist.
		\item
			$\lim_{x\to\infty} \f {\pi(x)}x = 0$.
	\end{enumerate}
\end{nt*}

% St 7.3
\setcounter{thm}{2}
\begin{st}[Primzahlsatz, 1. Version] \label{7.3}
	Es gilt
	\[
		\lim_{x\to\infty} \dfrac {\pi(x)}{\f x{\log x}} = 1,
	\]
	oder asymptotisch $\pi(x) \sim \f x{\log x}$.
\end{st}

% St 7.4
\begin{nt} \label{7.4}
	\begin{enumerate}[a)]
		\item
			\ref{7.3} wurde 1896 von Hadamard und de la Valée-Dusin gleichzeitig und unabhängig voneinendare bewiesen, aufbauend auf Vorarbeiten von Riemann.
			Insbesondere mit Hilfe von Eigenschaften der \emphdef{Riemann'schen Zetafunktion}:
			\[
				\zeta(s) := \sum_{n\in \N} \f 1{n^s},
			\]
			zunächst definiert für $s \in \C$ mit $\Re(s) > 1$.
			Mit analytischer Fortsetzung kann man $\zeta$ auf $\C \setminus \Set 1$ zu einer meromorphen Funktion ausdehnen.
		\item
			1949 wurde der Primzahlsatz von Erdös und Selberg elementar (ohne funktionentheoretischen, lediglich mit analytischen Methoden) bewiesen.
	\end{enumerate}
\end{nt}

