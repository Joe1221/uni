\chapter{Primzahlverteilung, Primzahlsatz} \label{chap:7}


% Df 7.1
\begin{df} \label{7.1}
	Für $x \in \R$ bezeichne $\pi(x)$ die Anzahl der Primzahlen $\le x$, also
	\[
		\pi(x)
		:= \l| \Set{ p \in \P & p \le x } \r|
		= \sum_{\substack{p \in \P}{p \le x}} 1.
	\]
	Mann nennt $\pi$ die \emphdef{Primzahlfunktion}.
\end{df}

\begin{nt*}
	\begin{enumerate}[a)]
		\item
			Es gilt $\lim_{x \to \infty} \pi(x) = \infty$, z.B. nach Euklid.
			$\pi$ ist monoton steigend.
		\item
			Man stellt sich die Frage: „Wie geht $\pi(x)$ gegen $\infty$?“.
			Sicherlich ist $\pi(x) \le x$ für $x \in \R^+$.
			Man sieht auch $\pi(x) \le \f x2$ für $x \ge 8$.

			Analog sieht man mit dem Sieb des Erastosthenes, dass für beliebiges $n \in \N$ die Aussage $\pi(x) \le \f xn$ für hinreichend großes $x$ erfüllt ist.
		\item
			$\lim_{x\to\infty} \f {\pi(x)}x = 0$.
	\end{enumerate}
\end{nt*}

% St 7.3
\setcounter{thm}{2}
\begin{st}[Primzahlsatz, 1. Version] \label{7.3}
	Es gilt
	\[
		\lim_{x\to\infty} \dfrac {\pi(x)}{\f x{\log x}} = 1,
	\]
	oder asymptotisch $\pi(x) \sim \f x{\log x}$.
\end{st}

% St 7.4
\begin{nt} \label{7.4}
	\begin{enumerate}[a)]
		\item
			\ref{7.3} wurde 1896 von Hadamard und de la Valée-Dusin gleichzeitig und unabhängig voneinander bewiesen, aufbauend auf Vorarbeiten von Riemann.
			Insbesondere mit Hilfe von Eigenschaften der \emphdef{Riemann'schen Zetafunktion}:
			\[
				\zeta(s) := \sum_{n\in \N} \f 1{n^s},
			\]
			zunächst definiert für $s \in \C$ mit $\Re(s) > 1$.
			Mit analytischer Fortsetzung kann man $\zeta$ auf $\C \setminus \Set 1$ zu einer meromorphen Funktion ausdehnen.
		\item
			1949 wurde der Primzahlsatz von Erdös und Selberg elementar (ohne funktionentheoretischen, lediglich mit analytischen Methoden) bewiesen.
\coursetimestamp{30}{06}{2014}
		\item
			Gauß hatte den Primzahlsatz in folgender Form vermutet:
			\[
				\pi(x) \sim \mathop{li}(x),
			\]
			wobei $\mathop{li} := \int_{2}^x \f 1{\log t} \di[t]$.
			Da nach L'Hospital $\f x{\log x} \sim \mathop{li}(x)$ ist diese Formulierung äquivalent zu \ref{7.3}.
			Die Formulierung von Gauß liefert eine bessere Abschätzung.
		\item
			\emph{Heuristische} Herleitung des Primzahlsatzes:
			
			Sei $\pi_2(x)$ die Anzahl der ungeraden Zahlen $\le x$.
			Dann liegt nahe $\pi_2'(x) = \f 12 = (1 - \f 12)$.
			Sei $\pi_{2,3}(x)$ die Anzahl der nicht durch $2$ und $3$ teilbaren Zahlen.
			Dann hat man $\pi_3'(x) = (1 - \f 12)(1 - \f 13)$.

			Angenommen $\pi(x)$ sei differenzierbar, dann liegt nahe, dass
			\[
				\pi'(x) = \prod_{\substack{p\in \P \\ p < x}} (1 - \f 1p).
			\]
			Logarithmieren ergibt
			\begin{align*}
				\log \pi'(x) &\approx \sum_{\substack{p \in \P \\ p < x}} \log(1 - \f 1p) \\
				&\approx \sum_{\substack{t \in \Z \\ 2 \le t < x}} \log(1 - \f 1t) \pi'(x) \\
				&\approx \int_2^x \log(1 - \f 1t) \pi'(t) \di[t] 
			\end{align*}
			Die Potenzreihe von $\log(1 - \f 1t) = - \f 1t - \f {2t^2} - \f 1{3t^3} - \dotsc$, also $\log \pi'(x) \approx -\int_2^x \f 1t \pi'(t) \di[t]$.
			Ableiten ergibt
			\[
				\f{\pi''(x)}{\pi'(x)} \approx - \f{\pi'(x)}{x},
			\]
			also $- \f{\pi''(x)}{\pi'(x)} = \f 1{x}$.
			Integrieren ergibt $\f 1{\pi'(x)} \approx \log x + c$, also $\pi'(x) \approx \f 1{\log x + c}$.
			Für große $x$ ergibt dies $\pi'(x) \approx \f 1{\log x}$ und Integrieren $\pi(x) \approx \mathrm{li} x$.
	\end{enumerate}
\end{nt}

% St 7.5
\begin{st}[Tschebyscheff, 1850] \label{7.5}
	Sei $n \in \N, n \ge 4$.
	Dann gilt
	\[
		\f 14 \f n{\log n}
		\le \pi(n) \le
		6 \f n{\log n}
	\]
	\begin{note}
		Dieser Satz gilt auch für $n \in \Set{2, 3}$.
	\end{note}
	\begin{proof}
		Idee: Man weiß gut Bescheid, wie die Primzahlzerlegung von $\binom{2n}{n}$ ist.
		\begin{enumerate}[1)]
			\item
				Es gilt $2^n < \binom{2n}{n} < 4^n$ für $n \ge 2$.

				Die rechte Seite folgt aus $4^n = (1+1)^{2n} = \sum_{k=0}^{2n} \binom{2n}{k} > \f{2n}n$.
				Die linke Seite ist richtig für $n = 2$, da $2^4 = 4 < 6 = \binom{4}{2}$.
				Per Induktion
				\begin{align*}
					\binom{2n}{n}
					&= \f {(2n)(2n-1)\dotsb (2-n+1)\cdot n}{n(n-1)\dotsb 2\cdot 1 \cdot n} \\
					&= \binom{2(n-1)}{n-1} \f {(2n)(2n-1)}{n^2}
					> 2^{n-1} \f {2(2n-1)}{n}
					= 2^n \f {2n - 1}{n}
					> 2^n.
				\end{align*}
			\item
				Sei $\nu_p(n!)$ die größte $p$-Potenz, die $n!$ teilt.
				Es gilt
				\[
					\nu_p(n!) = \sum_{m \ge 1} \ceil{\f{n}{p^m}} 
					= \sum_{m=1}^{\ceil{\f{\log n}{\log p}}} \ceil{\f {n}{p^m}}.
				\]

				Wie man 2) leicht einsieht, bzw. beweist, sieht man am besten an einem Beispiel:
				\begin{align*}
					\nu_2(10!)
					&= \underbrace{\ceil{\f {10}{8}}}_{\text{durch $8$ teilb. Zahlen}} + \underbrace{\ceil{\f {10}{4}}}_{\text{durch $4$ teilb. Zahlen}} + \underbrace{\ceil{\f {10}{2}}}_{\text{durch $2$ teilb. Zahlen}} \\
					&= 1 + 2 + 5
					= 8.
				\end{align*}
			\item
				Es gilt $n \log 2 < \log(2n!) - 2 \log(n!) < 2n \log n$ (Logarithmieren von 1)).
			\item
				Es gilt
				\[
					\log(2n)! - 2 \log(n!) = \sum_{p \le 2n} \sum_{m = 1}^{\ceil{\f{\log 2n}{\log p}}} \big( \ceil{\f {2n}{p^m}} -  2 \ceil{\f{n}{p^m}} \big) \log p.
				\]
				Denn $n! = \prod_{\substack{p \le n \\ p \in \P}} p^{\nu_p(n!)}$ und $(2n)! = \prod_{p \le 2n} p^{\nu_p((2n)!)}$.
				Nach Logarithmieren ergibt sich
				\[
					\log n! = \sum_{p \le n} \nu_p(n!) \log p
					\stack{2)}= \sum_{p \le n} \sum_{m \ge 1} \ceil{n}{p^m} \log p.
				\]
				Beachtet man $\ceil{\f{n}{p^m}} = 0$ für $p^m > n$, oder für $m > \ceil{\f {\log n}{\log p}}$, dann folgt 4) durch Einsetzen.

				Verwende $\ceil{2x} - 2 \ceil{x}$ aus \ref{7.6} und ersetze in der rechten Seite von 4) die innere Summe durch ihre Anzahl von Summanden.
				Dann ist mit 3)
				\[
					n \log 2 < \sum_{p \le 2n} \ceil{\f {\log 2n}{\log p}} \log p.
				\]
				Wegen $\ceil{\log 2n}{\log p} \log p \le \log 2n$ folgt
				\[
					\sum_{p \le 2n} \ceil{\f{2n}{\log p}} \log p
					\le \sum_{p \le 2n} \log 2n
					= \pi(2n) \log 2n.
				\]
				Insgesamt ergibt sich
				\[
					\pi(2n) > \f {n \log 2}{\log 2n}
					> \f{n \cdot 2}{(\log 2n) \cdot 4}
					= \f 14 \f {2n}{\log(2n)}
				\]
				und
				\[
					\pi(2n + 1) \ge \pi(2n)
					> \f{n\log 2}{\log 2n}
					= \f {n \log 2 (2n + 1)}{(2n + 1) \log 2n}
					> \f 14 \f{2n + 1}{\log(2n + 1)},
				\]
				da
				\[
					\f {n\log 2}{2n + 1}
					= \f{\log 2}{2 + \f 1n}
					> \dfrac {\f {6}{10}}{\f{9}{4}}
					= \f{24}{90}
					> \f 14.
				\]
		\end{enumerate}
		Für die rechte Ungleichung von \ref{7.5} betrachte die Funktion $\theta(x) := \sum_{\substack{p\in \P \\ p \le x}} \log p$, $\theta(1) := 0$.
		Dann ist $\theta(2n) - \theta(n) = \sum_{n < p < 2n} \log p$.
		Für $n < p < 2n$ ist $\ceil{\f {2n}p} - 2 \ceil{\f np} = 1$.
		Aus 4) folgt
		\[
			\log((2n)!) - 2 \log(n!)
			> \sum_{n<p<2n} \log p
		\]
		und aus 3) $2n \log 2 > \theta(2n) - \theta(n)$.
		Insbesondere mit $n = 2^r$ auch $\theta(2^{r+1}) - \theta(2^r) < 2^{r+1} \log 2$.
		Es gilt dann
		\[
			\theta(2^{r+1})
			= \sum_{i = 0}^r \theta(2^{i+1}) - \theta(2^i)
			< \log 2 (2 + \dotsb 2^{r+1})
			< 2^{r+2} \log 2.
		\]
		Sei $2^k \le n \le 2^{k+1}$ und $y < n$.
		Dann ist
		\[
			(\pi(n) - \pi(y))\log y
			\le \sum_{y \le p \le n} \log p
			\le \theta(n)
			\le \theta(2^{k+1})
			\stack{\text{s.o.}}\le 2^{k+2} \log 2
			\le 4 n \log 2.
		\]
		Mit $y = n^{\f 23}$, $\log y = \f 23 \log n$, $\pi(y) \le y = n^{\f 23}$ folgt dann $(\pi(n) - \pi(y)) \f 23 \log n	\le 4 n \log 2$.
		Also ist $\pi(n) - \pi(y) \le \f{6n \log 2}{\log n}$, oder
		\[
			\pi(n) \le n^{\f 23} + \f {6n \log 2}{\log n}
			= \f n{\log n}\big( \f {\log n}{n^{\f 13} + 6 \log 2} \big).
		\]
		Betrachte nun die Funktion $\f {\log x}{x^{\f 13}}$.
		Mit Kurvendiskussion hat diese ein Maximum bei $x = e^3$.
		Folglich hat man
		\[
			\pi(n) \le n {\log n} \big( \f {\log n}{n^{\f 13} + 6 \log 2} \big)
			< \f n{\log n} (\f 3e + 6 \log 2)
			< 6 \f {n}{\log n}.
		\]
	\end{proof}
\end{st}

% Bem 7.6
\begin{nt} \label{7.6}
	Für $\nu_p \binom{2n}{n}$ ergibt sich wie oben
	\[
		\nu_p \binom {2n}{n} \log p = \sum_{m=1}^{\ceil{\f{\log 2n}{\log p}}} \big( \ceil{\f{2n}{p^m}} - 2 \ceil{\f{n}{p^m}} \big) \log p.
	\]
	Verwende nun
	\[
		\ceil{2x} - {x} = \begin{cases}
			0 & x - \ceil{x} < \f 12 \\
			1 & x - \ceil{x} \ge \f 12
		\end{cases},
	\]
	dann ist
	\[
		\nu_p \binom{2n}{n} \log p
		\le \ceil{\f {\log 2n}{\log p}}	 \log p
		\le \log(2n)
	\]
	und somit
	\begin{equation} \label{eq:7.6}
		p^{\nu_p \binom{2n}{n}} \le 2n.
		\tag{7.6}
	\end{equation}
\end{nt}


\coursetimestamp{03}{07}{2014}


% Bem 7.7
\begin{nt} \label{7.7}
	Tschebyscheff bewies \ref{7.5} mit besseren Schranken, z.B.
	\[
		\f x{\log x} \cdot 0.92929
		\le \pi(x) \le
		\f x{\log x} \cdot 1.1056
	\]
	oder nach Scheid/Frommer % fixme ref
	\[
		a \f x{\log x}
		\le \pi(x) \le
		A \f x{\log x}
	\]
	mit $2 = a + A$, jeweils für ein hinreichend großes $x$.
\end{nt}

Unser nächstes Ziel ist das Bertrand'sche Resultat.
Bertrand stellte mit Hilfe von Primzahltabellen bis $\approx 5 \cdot 10^6$ fest, dass stets zwischen $n$ und $2n$ eine Primzahl liegt.

% Lem 7.8
\begin{lem} \label{7.8}
	Es gilt
	\begin{enumerate}[a)]
		\item
			Für $n \in \N$ ungerade, $n \ge 3$ und
			\begin{align*}
				k := \begin{cases}
					\f {n-1}2 & \f{n+1}2 \text{ ungerade} \\
					\f {n+1}2 & \f{n-1}2 \text{ gerade}
				\end{cases}
			\end{align*}
			gilt
			\[
				\prod_{\substack{k < p \le n \\ p \in \P}} p \le \binom{n}{k}.
			\]
			Und für $n \in \N$ gerade, $n \ge 4$ und
			\begin{align*}
				k := \begin{cases}
					\f {n-2}2 & \f{n-2}2 \text{ ungerade} \\
					\f{n}2 & \f{n-2}2 \text{ gerade}
				\end{cases}
			\end{align*}
			gilt
			\[
				\prod_{\substack{k < p \le n \\ p \in \P}} p \le \binom{n-1}{k}.
			\]
		\item
			Es gilt für $x \in \R$, $x \ge 2$:
			\[
				\prod_{\substack{p \le x \\ p \in \P}} p < 4^x.
			\]
	\end{enumerate}
	\begin{proof}
		\begin{enumerate}[a)]
			\item
				Sei $n$ ungerade und $k$ wie angegeben.
				Eine Primzahl $p$ mit $k < p \le n$ teilt $n!$, aber nicht $k!$ und $(n-k)!$.
				\[
					n-k = \begin{cases}
						\f{n+1}{2} & \f{n-1}2 = k \text{ ungerade} \\
						\f{n-1}{2} & \f{n-1}2 = k \text{ gerade}
					\end{cases}.
				\]
				Ist $k$ ungerade, dann $n-k$ gerade und $n - k \le k + 1$, also $p \ndivs n - k$.
				Nun ist
				\[
					\binom{n}{k} = \f {n!}{k!(n-k)!},
				\]
				also teilt jede Primzahl aus $\prod_{k < p \le n} p$ auch $\binom{n}{k}$.
				Insbesondere ist dann $\prod_{k < p \le n} p \le \binom{n}{k}$.

				Der zweite Teil von a) folgt aus dem ersten Teil.
				Ist $n$ gerade und $n \ge 4$, dann ist $n$ keine Primzahl, also
				\[
					\prod_{k < p \le n} p
					= \prod_{k < p \le n-1} p
				\]
				und $n-1$ ist dann ungerade.
			\item
				$\prod_{\substack{p \le x\\ p\in \P}} p \in \N$.
				Es genügt b) für $x = n \in \N$ und $n$ ungerade, $n \ge 3$ zu beweisen (für $2 \le x < 3$ sieht man die Behauptung direkt).
				Sei also $n$ ungerade und $k$ sei wie in a) gewählt.
				Es gilt
				\[
					2^n = (1+1)^n
					= \sum_{i=0}^n \binom{n}{i}
					> 2 \binom{n}{k},
				\]
				da $\binom{n}{k} = \binom{n}{\f{n-1}2} = \binom{n}{\f{n+1}2} = \binom{n}{n-k}$.
				Damit ist $\binom{n}{k} < 2^{n-1}$.

				Verwende nun Induktion.
				Der Induktionsanfang $n = 3$ ist erfüllt: $2 \cdot 3 < 4^3$.
				\[
					\prod_{p \le n} p
					= \Big(\prod_{p \le k} p\Big) \prod_{k < p \le n} p
					< 4^k \binom{n}{k}
					< 4^k 2^{n-1}
					= 2^{2k} 2^{n-1}
					\le 2^{2n}
					= 4^n.
				\]
		\end{enumerate}
	\end{proof}
\end{lem}


% St 7.9
\begin{st}[Bertrand'sches Postulat, bewiesen von Tschebyscheff] \label{7.9}
	Für jedes $n \in \N$ mit $n > 1$ gibt es eine Primzahl $p$ mit $n < p < 2n$.
	\begin{proof}
		Angenommen es existiere keine Primzahl $p$ mit $n < p < 2n$.
		Dann ist
		\[
			\binom {2n}{n}
			= \prod_{p \le n} p^{e_p},
		\]
		wobei $e_p = \nu_p(\binom{2n}{n})$.

		Ist $\f 23 n < p \le n$, dann ist $e_p = 0$, denn dann ist $n = 1p + a_0$ für ein $a_0 < \f p2$.
		Damit ist $2n = 2p + 2a_0$, also $n! = px$ mit $(x,p) = 1$ und
		\[
			(2n)(2n-1)\dotsb (n+1) = p \tilde x
		\]
		mit $(\tilde x, p) = 1$.

		Ist $\sqrt{2n} < p \le \f 23 n$, dann ist $e_p = 1$, denn dann ist $n = a_1 p + a_0$ für $1 \le a_1 < \f p2$ und $0 \le a_0 < p$ und $2n = 2a_1 p + 2a_0$ (beachte $p \f p2 > n$, wegen $p > \sqrt{2n}$).
		$2 a_0$ kann $\ge p$ sein, aber auf jeden Fall $< 2p$.
		Es gilt $n! = p^{a_1} x$ mit $(p,x) = 1$ und $\f {2n!}{n!} = p^{a_1 + \eps} \tilde x$ mit $(p, \tilde x) = 1$ und $\eps \in \Set{0, 1}$, also $e_p \le 1$.
		Im Beweis von Tschebyscheff \ref{7.6} war $p^{e_p} \le 2n$, also
		\[
			\binom{2n}{n}
			< \Big(\prod_{p \le \sqrt{2n}} p \Big) \prod_{\sqrt{2n} < p \le \f 23 n} p
		\]
		Die Anzahl der Faktoren im ersten Produkt ist $\pi(\sqrt{2n}) \le \floor{\f 12 \sqrt n}$.
		Ist $n \ge 128$, also $\sqrt{2n} \ge 16$, dann entfallen auch noch $9$ und $15$.
		Für $n \ge 128$ ist $\pi(\sqrt{2n}) \le \floor{\f 12 \sqrt{2n}} - 2 < \f 12 \sqrt{2n} - 1$.
		Nach Lemma \ref{7.8} b) gilt $\prod_{p \le \f 23 n} p < 4^{\f 23 n}$.
		Damit ist
		\[
			\binom{2n}{n} < 2 n^{\f 12 \sqrt 2n - 1} 4^{\f 23 n}
		\]
		Andererseits gilt
		\[
			(1 + 1)^{2n}
			< 2n \binom{2n}{n},
		\]
		d.h. $\f{2^n}{2n} < \binom{2n}{n}$ und insgesamt
		\begin{align*}
			\f{2^n}{2n} &< 2n^{\f 12 \sqrt{2n} - 1} 4^{\f{23}n} \\
			\iff 2^{2n} &< 2n^{\f 12 \sqrt{2n}} 2^{\f 43 n} \\
			\iff 2^{\f 23 n} &< 2n^{\f 12 \sqrt{2n}}
		\end{align*}
		Logarithmieren liefert
		\begin{align*}
			\f 23 n \log 2 &< \f 12 \sqrt{2n} \log(2n) \\
			\iff \f 4{\sqrt 2} \sqrt n \f 13 \log 2 &< \log(2n) \\
			\iff \sqrt{8n} \log 2 &< 3 \log(2n)
			\iff \sqrt{8n} \log 2 - 3 \log(2n) < 0.
		\end{align*}
		Betrachte nun $f(x) = \log 2 \sqrt{8} \sqrt{x} - 3 \log 2x$.
		Kurvendiskussion von $f(x)$ zeigt, dass $f(x) > 0$ für $x \ge 128$.
		Wir haben damit ein Widerspruch für $n \ge 128$.

		Zum Abschluss geben wir Primzahlen an:
		\[
			3, 5, 7, 13, 23, 43, 83, 163.
		\]
		Diese decken den Bereich $n \le 162$ ab.
	\end{proof}
\end{st}

% Bem 7.10
\begin{nt} \label{7.10}
	Weitere, bzw. andere Abschätzungen für $\pi(x)$ sind
	\begin{enumerate}[a)]
		\item
			Es gilt
			\[
				\pi(x) - \pi(\sqrt x) \approx (x-\sqrt x)\prod_{p \le \sqrt x} (1 - \f 1p).
			\]
			Für $x = 100$ ergibt sich
			\[
				\pi(100) - \pi(10)
				\approx 90 \prod_{p \le 10} (1 - \f 1p)
				= 90 \big( \f 12 \f 23 \f 45 \f 67 \big)
				= \f{90 \cdot 8}{35}
				\approx 20,571.
			\]
			Es gilt exakt $\pi(100) - \pi(10) = 25 - 4 = 21$.
		\item
			Legendre-Vermutung (bis heute unbewiesen):
			Zwischen $n^2$ und $(n+1)^2$ gibt es stets eine Primzahl.
			Vergleich zu Betrand:
			\[
				(n+1)^2 - n^2
				= n^2 + 2n + 1 - n^2
				= 2n + 1
				< n^2 = 2n^2 - n^2.
			\]
			Legendre ist also deutlich schärfer als Betrands Postulat.
		\item
			Bonsesche Ungleichung (1907):
			Seien $p_1, \dotsc, p_n$ die ersten $n$ Primzahlen, dann gilt
			\[
				p_n^2 < \prod_{k=1}^{n-1} p_k.
			\]
			Dies kann man leicht mit \ref{7.9} beweisen (siehe Übung).
		\item
			Die Summe
			\[
				\sum_{p\in\P} \f 1p
			\]
			divergiert.
			Anschaulich: „Es gibt mehr Primzahlen, als Quadratzahlen“ (da $\sum_{n \in \N} n^2 = \f{\pi^2}6$).
	\end{enumerate}
\end{nt}

% 7.11
\begin{nt}[Riemannsche Zetafunktion] \label{7.11}
	\begin{enumerate}[a)]
		\item
			\[
				\zeta(s) := \sum_{n=1}^\infty \f 1{n^s}
			\]
			ist konvergent für $s \in \Set{z\in \C & \Re(z) > 1}$.
			Für $s = 1$ erhält man die divergente harmonische Reihe.

			Reihen dieser Form nennt man \emphdef{Dirichlet-Reihe}.
		\item
			Die $\zeta$-Funktion lässt sich als \emphdef{Euler-Produkt} darstellen:
			\[
				\zeta(s)
				= \prod_{p\in\P} \sum_{n=0}^\infty \f 1{p^{ns}}
				= \prod_{p\in\P} \f 1{1 - \f 1{p^s}}.
			\]
			Dies beweist man durch Ausmultiplizieren und dem Hauptsatz der elementaren Zahlentheorie.
		\item
			$\zeta(s)$ lässt sich holomorph fortsetzen auf $\C \setminus \Set 1$.
			$\zeta$ hat in $s = 1$ einen Pol erster Ordnung.
		\item
			Nullstellen und Werte von $\zeta(s)$:
			Die auf $\C \setminus \Set 1$ definierte Zetafunktion hat Nullstellen für $s = -2, -4, -6, \dotsc$,
			diese sind die sogenannten \emphdef[Riemannsche Zetafunktion!triviale Nullstellen]{triviale Nullstellen}.
			Man weiß, dass alle andere Nullstellen im Streifen $0 < \Re s < 1$ liegen.
			Es gibt weiter Nullstellen.
			Man kennt ca. $10^{13}$ solche Nullstellen, welche alle auf der Geraden $\Re s = \f 12$.

			Die \emphdef{Riemannsche Vermutung} besagt nun:
			\begin{quote}
				Alle nicht-trivialen Nullstellen von $\zeta$ liegen auf der Geraden $\Re s = \f 12$.
			\end{quote}

			Die Riemannsche Vermutung liefert für den Primzahlsatz folgende Version 2:
			\[
				|\pi(x) - \li(x)|
				\le c \sqrt x \log x
			\]
			mit $c > 0$, oder mit Landausymbolen
			\[
				\pi(x) = \li(x) + \LandauO(\sqrt x \log x).
			\]
			Diese Aussage wurde nur mit der Riemannschen Vermutung bewiesen.

			Es gilt $\zeta(2) = \f{\pi^2}6$ (\emphdef{Basler Problem}).
			Man kennt gut $\zeta(2n)$ für $n \in \N$.

			Der Primzahlsatz Version 1 ist gleichwertig damit, dass $\zeta$ für $\Re s = 1$ keine Nullstellen hat.
	\end{enumerate}
\end{nt}
