\chapter{Primzahlverteilung, Primzahlsatz} \label{chap:7}


% Df 7.1
\begin{df} \label{7.1}
	Für $x \in \R$ bezeichne $\pi(x)$ die Anzahl der Primzahlen $\le x$, also
	\[
		\pi(x)
		:= \l| \Set{ p \in \P & p \le x } \r|
		= \sum_{\substack{p \in \P}{p \le x}} 1.
	\]
	Mann nennt $\pi$ die \emphdef{Primzahlfunktion}.
\end{df}

\begin{nt*}
	\begin{enumerate}[a)]
		\item
			Es gilt $\lim_{x \to \infty} \pi(x) = \infty$, z.B. nach Euklid.
			$\pi$ ist monoton steigend.
		\item
			Man stellt sich die Frage: „Wie geht $\pi(x)$ gegen $\infty$?“.
			Sicherlich ist $\pi(x) \le x$ für $x \in \R^+$.
			Man sieht auch $\pi(x) \le \f x2$ für $x \ge 8$.

			Analog sieht man mit dem Sieb des Erastosthenes, dass für beliebiges $n \in \N$ die Aussage $\pi(x) \le \f xn$ für hinreichend großes $x$ erfüllt ist.
		\item
			$\lim_{x\to\infty} \f {\pi(x)}x = 0$.
	\end{enumerate}
\end{nt*}

% St 7.3
\setcounter{thm}{2}
\begin{st}[Primzahlsatz, 1. Version] \label{7.3}
	Es gilt
	\[
		\lim_{x\to\infty} \dfrac {\pi(x)}{\f x{\log x}} = 1,
	\]
	oder asymptotisch $\pi(x) \sim \f x{\log x}$.
\end{st}

% St 7.4
\begin{nt} \label{7.4}
	\begin{enumerate}[a)]
		\item
			\ref{7.3} wurde 1896 von Hadamard und de la Valée-Dusin gleichzeitig und unabhängig voneinander bewiesen, aufbauend auf Vorarbeiten von Riemann.
			Insbesondere mit Hilfe von Eigenschaften der \emphdef{Riemann'schen Zetafunktion}:
			\[
				\zeta(s) := \sum_{n\in \N} \f 1{n^s},
			\]
			zunächst definiert für $s \in \C$ mit $\Re(s) > 1$.
			Mit analytischer Fortsetzung kann man $\zeta$ auf $\C \setminus \Set 1$ zu einer meromorphen Funktion ausdehnen.
		\item
			1949 wurde der Primzahlsatz von Erdös und Selberg elementar (ohne funktionentheoretischen, lediglich mit analytischen Methoden) bewiesen.
\coursetimestamp{30}{06}{2014}
		\item
			Gauß hatte den Primzahlsatz in folgender Form vermutet:
			\[
				\pi(x) \sim \mathop{li}(x),
			\]
			wobei $\mathop{li} := \int_{2}^x \f 1{\log t} \di[t]$.
			Da nach L'Hospital $\f x{\log x} \sim \mathop{li}(x)$ ist diese Formulierung äquivalent zu \ref{7.3}.
			Die Formulierung von Gauß liefert eine bessere Abschätzung.
		\item
			\emph{Heuristische} Herleitung des Primzahlsatzes:
			
			Sei $\pi_2(x)$ die Anzahl der ungeraden Zahlen $\le x$.
			Dann liegt nahe $\pi_2'(x) = \f 12 = (1 - \f 12)$.
			Sei $\pi_{2,3}(x)$ die Anzahl der nicht durch $2$ und $3$ teilbaren Zahlen.
			Dann hat man $\pi_3'(x) = (1 - \f 12)(1 - \f 13)$.

			Angenommen $\pi(x)$ sei differenzierbar, dann liegt nahe, dass
			\[
				\pi'(x) = \prod_{\substack{p\in \P \\ p < x}} (1 - \f 1p).
			\]
			Logarithmieren ergibt
			\begin{align*}
				\log \pi'(x) &\approx \sum_{\substack{p \in \P \\ p < x}} \log(1 - \f 1p) \\
				&\approx \sum_{\substack{t \in \Z \\ 2 \le t < x}} \log(1 - \f 1t) \pi'(x) \\
				&\approx \int_2^x \log(1 - \f 1t) \pi'(t) \di[t] 
			\end{align*}
			Die Potenzreihe von $\log(1 - \f 1t) = - \f 1t - \f {2t^2} - \f 1{3t^3} - \dotsc$, also $\log \pi'(x) \approx -\int_2^x \f 1t \pi'(t) \di[t]$.
			Ableiten ergibt
			\[
				\f{\pi''(x)}{\pi'(x)} \approx - \f{\pi'(x)}{x},
			\]
			also $- \f{\pi''(x)}{\pi'(x)} = \f 1{x}$.
			Integrieren ergibt $\f 1{\pi'(x)} \approx \log x + c$, also $\pi'(x) \approx \f 1{\log x + c}$.
			Für große $x$ ergibt dies $\pi'(x) \approx \f 1{\log x}$ und Integrieren $\pi(x) \approx \mathrm{li} x$.
	\end{enumerate}
\end{nt}

% St 7.5
\begin{st}[Tschebyscheff, 1850] \label{7.5}
	Sei $n \in \N, n \ge 4$.
	Dann gilt
	\[
		\f 14 \f n{\log n}
		\le \pi(n) \le
		6 \f n{\log n}
	\]
	\begin{proof}
		Idee: Man weiß gut Bescheid, wie die Primzahlzerlegung von $\binom{2n}{n}$ ist.
		\begin{enumerate}[1)]
			\item
				Es gilt $2^n < \binom{2n}{n} < 4^n$ für $n \ge 2$.

				Die rechte Seite folgt aus $4^n = (1+1)^{2n} = \sum_{k=0}^{2n} \binom{2n}{k} > \f{2n}n$.
				Die linke Seite ist richtig für $n = 2$, da $2^4 = 4 < 6 = \binom{4}{2}$.
				Per Induktion
				\begin{align*}
					\binom{2n}{n}
					&= \f {(2n)(2n-1)\dotsb (2-n+1)\cdot n}{n(n-1)\dotsb 2\cdot 1 \cdot n} \\
					&= \binom{2(n-1)}{n-1} \f {(2n)(2n-1)}{n^2}
					> 2^{n-1} \f {2(2n-1)}{n}
					= 2^n \f {2n - 1}{n}
					> 2^n.
				\end{align*}
			\item
				Sei $\nu_p(n!)$ die größte $p$-Potenz, die $n!$ teilt.
				Es gilt
				\[
					\nu_p(n!) = \sum_{m \ge 1} \ceil{\f{n}{p^m}} 
					= \sum_{m=1}^{\ceil{\f{\log n}{\log p}}} \ceil{\f {n}{p^m}}.
				\]

				Wie man 2) leicht einsieht, bzw. beweist, sieht man am besten an einem Beispiel:
				\begin{align*}
					\nu_2(10!)
					&= \underbrace{\ceil{\f {10}{8}}}_{\text{durch $8$ teilb. Zahlen}} + \underbrace{\ceil{\f {10}{4}}}_{\text{durch $4$ teilb. Zahlen}} + \underbrace{\ceil{\f {10}{2}}}_{\text{durch $2$ teilb. Zahlen}} \\
					&= 1 + 2 + 5
					= 8.
				\end{align*}
			\item
				Es gilt $n \log 2 < \log(2n!) - 2 \log(n!) < 2n \log n$ (Logarithmieren von 1)).
			\item
				Es gilt
				\[
					\log(2n)! - 2 \log(n!) = \sum_{p \le 2n} \sum_{m = 1}^{\ceil{\f{\log 2n}{\log p}}} \big( \ceil{\f {2n}{p^m}} -  2 \ceil{\f{n}{p^m}} \big) \log p.
				\]
				Denn $n! = \prod_{\substack{p \le n \\ p \in \P}} p^{\nu_p(n!)}$ und $(2n)! = \prod_{p \le 2n} p^{\nu_p((2n)!)}$.
				Nach Logarithmieren ergibt sich
				\[
					\log n! = \sum_{p \le n} \nu_p(n!) \log p
					\stack{2)}= \sum_{p \le n} \sum_{m \ge 1} \ceil{n}{p^m} \log p.
				\]
				Beachtet man $\ceil{\f{n}{p^m}} = 0$ für $p^m > n$, oder für $m > \ceil{\f {\log n}{\log p}}$, dann folgt 4) durch Einsetzen.

				Verwende $\ceil{2x} - 2 \ceil{x}$ aus \ref{7.6} und ersetze in der rechten Seite von 4) die innere Summe durch ihre Anzahl von Summanden.
				Dann ist mit 3)
				\[
					n \log 2 < \sum_{p \le 2n} \ceil{\f {\log 2n}{\log p}} \log p.
				\]
				Wegen $\ceil{\log 2n}{\log p} \log p \le \log 2n$ folgt
				\[
					\sum_{p \le 2n} \ceil{\f{2n}{\log p}} \log p
					\le \sum_{p \le 2n} \log 2n
					= \pi(2n) \log 2n.
				\]
				Insgesamt ergibt sich
				\[
					\pi(2n) > \f {n \log 2}{\log 2n}
					> \f{n \cdot 2}{(\log 2n) \cdot 4}
					= \f 14 \f {2n}{\log(2n)}
				\]
				und
				\[
					\pi(2n + 1) \ge \pi(2n)
					> \f{n\log 2}{\log 2n}
					= \f {n \log 2 (2n + 1)}{(2n + 1) \log 2n}
					> \f 14 \f{2n + 1}{\log(2n + 1)},
				\]
				da
				\[
					\f {n\log 2}{2n + 1}
					= \f{\log 2}{2 + \f 1n}
					> \dfrac {\f {6}{10}}{\f{9}{4}}
					= \f{24}{90}
					> \f 14.
				\]
		\end{enumerate}
		Für die rechte Ungleichung von \ref{7.5} betrachte die Funktion $\theta(x) := \sum_{\substack{p\in \P \\ p \le x}} \log p$, $\theta(1) := 0$.
		Dann ist $\theta(2n) - \theta(n) = \sum_{n < p < 2n} \log p$.
		Für $n < p < 2n$ ist $\ceil{\f {2n}p} - 2 \ceil{\f np} = 1$.
		Aus 4) folgt
		\[
			\log((2n)!) - 2 \log(n!)
			> \sum_{n<p<2n} \log p
		\]
		und aus 3) $2n \log 2 > \theta(2n) - \theta(n)$.
		Insbesondere mit $n = 2^r$ auch $\theta(2^{r+1}) - \theta(2^r) < 2^{r+1} \log 2$.
		Es gilt dann
		\[
			\theta(2^{r+1})
			= \sum_{i = 0}^r \theta(2^{i+1}) - \theta(2^i)
			< \log 2 (2 + \dotsb 2^{r+1})
			< 2^{r+2} \log 2.
		\]
		Sei $2^k \le n \le 2^{k+1}$ und $y < n$.
		Dann ist
		\[
			(\pi(n) - \pi(y))\log y
			\le \sum_{y \le p \le n} \log p
			\le \theta(n)
			\le \theta(2^{k+1})
			\stack{\text{s.o.}}\le 2^{k+2} \log 2
			\le 4 n \log 2.
		\]
		Mit $y = n^{\f 23}$, $\log y = \f 23 \log n$, $\pi(y) \le y = n^{\f 23}$ folgt dann $(\pi(n) - \pi(y)) \f 23 \log n	\le 4 n \log 2$.
		Also ist $\pi(n) - \pi(y) \le \f{6n \log 2}{\log n}$, oder
		\[
			\pi(n) \le n^{\f 23} + \f {6n \log 2}{\log n}
			= \f n{\log n}\big( \f {\log n}{n^{\f 13} + 6 \log 2} \big).
		\]
		Betrachte nun die Funktion $\f {\log x}{x^{\f 13}}$.
		Mit Kurvendiskussion hat diese ein Maximum bei $x = e^3$.
		Folglich hat man
		\[
			\pi(n) \le n {\log n} \big( \f {\log n}{n^{\f 13} + 6 \log 2} \big)
			< \f n{\log n} (\f 3e + 6 \log 2)
			< 6 \f {n}{\log n}.
		\]
	\end{proof}
\end{st}

% Bem 7.6
\begin{nt} \label{7.6}
	Für $\nu_p \binom{2n}{n}$ ergibt sich wie oben
	\[
		\nu_p \binom {2n}{n} \log p = \sum_{m=1}^{\ceil{\f{\log 2n}{\log p}}} \big( \ceil{\f{2n}{p^m}} - 2 \ceil{\f{n}{p^m}} \big) \log p.
	\]
	Verwende nun
	\[
		\ceil{2x} - {x} = \begin{cases}
			0 & x - \ceil{x} < \f 12 \\
			1 & x - \ceil{x} \ge \f 12
		\end{cases},
	\]
	dann ist
	\[
		\nu_p \binom{2n}{n} \log p
		\le \ceil{\f {\log 2n}{\log p}}	 \log p
		\le \log(2n)
	\]
	und somit
	\begin{equation} \label{eq:7.6}
		p^{\nu_p \binom{2n}{n}} \le 2n.
		\tag{7.6}
	\end{equation}
\end{nt}
