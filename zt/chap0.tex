\chapter{Einführung}


\coursetimestamp{07}{04}{2014}

\paragraph{Pythogoras}
(569 v. Chr. - 475 v. Chr.) „Alles ist Zahl.“

Pythagoraer konnten mit Brüchen umgehen und wussten, dass es auch irrationale Zahlen gab.
Auflösung ca 100 v. Chr.

\paragraph{Dedekind}
(1831 - 1916), Professor in Braunschweig, letzter Doktorand von Gauß.
Buch, 1888: „Was sind und was sollen die Zahlen?“

Zum Werk von R. Dedekind:
\begin{itemize}
	\item
		erste exakte Einführung der natürlichen Zahlen.
	\item
		erste exakte Einführung der reellen Zahlen („Dedekinschen Schnitte“).
	\item
		Aufbau der Idealtheorie, der algebraische Begriff des „Rings“ wurde von Dedekind eingeführt.
		„Ideal – ideale Zahlen“.
\end{itemize}


\paragraph{Was sind die Zahlen?}

\begin{align*}
	\N \subset \Z \subset \Q
	\begin{aligned}
		&\subset \R \subset \\
		&\subset \A \subset
	\end{aligned}
	\C
\end{align*}
wobei $\A$ die \emph{Algebraischen Zahlen} (Nullstellen von Polynomen aus $\Q[x]$) bezeichnet.
Von $\N$ nach $\Z$ der Schritt zur \emph{Gruppe}, von $\Z$ nach $\Q$ zum Körper, von $\Q$ nach $\R$ die Vervollständigung und von $\R$ nach $\C$ der Schritt, um die Lösbarkeit von Gleichungen zu erhalten.

\paragraph{Gibt es noch weitere Zahlen?}

1843 beschrieb Hamilton als erster ausführlich die Quaternionen $\H$.

\begin{df*}
	Der Raum der \emphdef{Quaternionen} $\H$ ist ein $4$-dimensionaler reeller Vektorraum mit einer ausgezeichneten Basis $\{1, i, j, k\}$.

	Eine Addition auf $\H$ ist durch die Vektorraumstruktur gegeben.
	Die Multiplikation ist zunächst auf den Basiselementen definiert.
	\begin{table}[ht]
		\centering
		\begin{tabular}{r|cccc}
			  & $1$ & $i$ & $j$ & $k$ \\ \hline
			$1$ & $1$ & $i$ & $j$ & $k$ \\
			$i$ & $i$ & $-1$ & $k$ & $-j$ \\
			$j$ & $j$ & $-k$ & $-1$ & $i$ \\
			$k$ & $k$ & $j$ & $-i$ & $-1$
		\end{tabular}
		\caption{Multiplikationstabelle der Quaternionen auf den Basiselementen}
	\end{table}

	Identifiziert man $\{ \lambda \cdot 1 : \lambda \in \R \}$ mit $\R$, so stimmt die Skalarmultiplikation auf $\H$ als $\R$-Vektorraum mit der Multiplikation von $\R \cdot 1$ mit den anderen Elementen überein.
	Die distributive Ausdehnung der Multiplikation auf den Basiselementen zusammen mit dieser Skalarmultiplikation als innere Multiplikation liefern für $x_i, y_j \in \R$ die sinnvolle Definition
	\begin{align*}
		&(x_0 \cdot 1  + x_1 i + x_2 j + x_3 k) \cdot (y_0 \cdot 1 + y_1 i + y_2 j + y_3 k) \\
		&\quad := ( x_0 y_0 - x_1 y_1 - x_2 y_2 - x_3 y_3) \cdot 1 \\
		&\qquad + ( x_0 y_1 + x_1 y_0 + x_2 y_3 - x_3 y_2) i \\
		&\qquad + ( x_0 y_2 + x_2 y_0 + x_1 y_3 - x_3 y_1) j \\
		&\qquad + ( x_0 y_3 + x_3 y_0 + x_1 y_2 - x_2 y_1) k. \\
	\end{align*}
	\begin{note}
		Mit dieser Multiplikation wird $\H$ zu einem Schiefkörper, d.h. zu einem Ring, in dem jedes Elemente ungleich Null bezüglich der Multiplikation invertierbar ist (im Gegensatz zum Körper ist die Multiplikation \emph{nicht} kommutativ).
	\end{note}
\end{df*}

Die Menge $\{ \lambda_1 1 + \lambda_2 i : \lambda_1, \lambda_2 \in \R$ bildet einen zu $\C$ isomorphen Teilkörper, aber auch $\{\my_1 1 + \my_2 j : \my_1, \my_2 \in \R\}$ und $\{\ny_1 1 + \ny_2 j : \ny_1, \ny_2 \in \R\}$ sind zu $\C$ isomorphe Teilkörper.

$\H$ ist als $\C$-Vektorraum $2$-dimensional.
$\C$ liegt nicht im Zentrum von $\H$, bezeichnet mit
\[
	Z(\H) := \{ w \in \H : \forall x \in \H : wx = xw \}.
\]
Es gilt $Z(\H) = \R$.

Man kann zeigen: $\H$ ist bis auf Isomorphie der einzige über $\R$ endlich dimensionaler Schiefkörper, der $\R$ als Zentrum besitzt.
Es gibt keinen Körper $K \supsetneq \C$ mit: $K$ ist endlich dimensional über $U$.

F. Klein: $\C$ ist wichtiger als $\H$ für die Mathematik.


\paragraph{Diophantische Gleichungen}

\begin{df*}
	Gegeben sei $f \in \Z[x_1, x_2, \dotsc, x_n]$ oder $f \in \Q[x_1, x_2, \dotsc, x_n]$.
	Gesucht sind \emph{ganzzahlige} Lösungen von $f = 0$.
	Solche Problemstellungen nennt man \emphdef{Diophantische Gleichung}.
\end{df*}
Die Lösbarkeit von Diophantischen Gleichungen ist ein extrem schwieriges Gebiet.

\begin{ex*}[Pythagoreische Tripel]
	Das Tripel $(3, 4, 5)$ erfüllt $3^2 + 4^2 = 5^2$ (Satz des Pythagoras).
	Mit einer 12-Knoten-Schnur lässt sich dies zur Erzeugung von rechten Winkeln verwenden.
	$(5, 12, 13)$ ist ebenfalls Lösung der Gleichung $x_1^2 + x_2^2 = x_3^2$.

	Es existieren unendlich viele nicht-triviale Lösungen dieser diophantischen Gleichung, die man sehr genau beschreiben kann.
\end{ex*}

\begin{ex*}[Großer Satz von Fermat]
	Für $n \in \N, n \ge 3$ wird die diophantische Gleichung
	\[
		x^n + y^n = z^n
	\]
	auch „großer Fermat'scher Satz“ genannt.

	A. Wiles hat erst 1995 bewiesen, dass es keine nicht-trivialen Lösungen dieser diophantischen Gleichung für $n \ge 3$ gibt.
	Wichtig für die algebraische Zahlentheorie ist nicht der Satz selbst, sondern sind die „Lösungsmethoden“ des Beweises.
\end{ex*}

\paragraph{Besondere natürliche Zahlen}

\paragraph{Vollkommene Zahlen}

\begin{df*}
	$n \in \N$ heißt \emphdef[vollkommene Zahl]{vollkommen}, wenn $n$ mit der Summe seiner echten Teiler, inklusive der 1, übereinstimmt.
\end{df*}

\begin{ex*}
	Kleinste Beispiele sind
	\begin{align*}
		2(2^2 - 1) &= 6  &&= 1 + 2 + 6, \\
		2(2^3 - 1) &= 28 &&= 1 + 2 + 4 + 7 + 14.
	\end{align*}
\end{ex*}

Es ist bis heute unbekannt, ob es unendlich viele vollkommene Zahlen gibt.

\coursetimestamp{10}{04}{2014}

Euklid hat bereits festgestellt:

\begin{st*}[Euklid]
	$m = (2^n - 1) 2^{n-1}$ ist vollkommen, wenn $2^n - 1$ eine Primzahl ist.
	\begin{proof}
		Die Teiler von $m$ sind
		\begin{alignat*}{5}
			1&,& 2 &,& \dotsc&,& 2^{n-2} &, 2^{n-1} \\
			p&,& 2p&,& \dotsc&,& p2^{n-2}
		\end{alignat*}
		Es gilt
		\begin{align*}
			\sum \text{echte Teiler}
			&= \underbrace{(p + 1)}_{=2^n} \underbrace{(1 + 2 + \dotsb + 2^{n-2})}_{=2^{n-1} - 1} + 2^{n-1} \\
			&= 2^{n-1} \underbrace{(2 - (2^n-1) + 1)}_{=2^n-1}
			= m
		\end{align*}
	\end{proof}
\end{st*}

Euler (1707-1783) konnte zeigen:

\begin{st*}
	Jede gerade vollkommene Zahl hat die Form $m = (2^n-1) 2^{n-1}$ mit $2^n - 1$ prim.
\end{st*}

\begin{df*}
	Primzahlen der Form $2^n - 1$ nennt man \emphdef{Mersenne-Primzahlen}.
\end{df*}

\begin{nt*}
	Es ist unbekannt, ob
	\begin{enumerate}[a)]
		\item
			unendlich viele, gerade vollkommene Zahlen existieren, äquivalent: ob unendlich viele Mersenne-Primzahlen existieren (es existieren mindestens 48 Mersenne Primzahlen, die letzt wurde 2013 gefunden),
		\item
			es eine ungerade vollkommene Zahl gibt (gibt es eine solche, so ist sie $> 1.9 \cdot 10^{2550}$ und hat mindesten $420$ verschiedene Primfaktoren.
	\end{enumerate}
\end{nt*}










