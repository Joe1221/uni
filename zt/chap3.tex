\chapter{Kryptographie, Primzahltests}



% 3.1
\section{Das RSA-Verfahren}

Das Verfahren wurde 1977/1978 am MIT von Rivest, Shamir und Adelman entwickelt (soll aber bereits vorher, zumindest in ähnlicher Form MI6 bekannt gewesen sein).

Das RSA-Verfahren ist ein sogenanntes asymmetrisches kryptographisches Verfahren.
Solche finden Anwendung bei Verschlüsselung von Nachrichten, bei digitalen Signaturen, Bezahlung mit Kreditkarten, Pay-TV.

Mathematisch gesehen war es der Retter der Königin der Mathematik (der elementaren Zahlentheorie).
Diese hatte bis dahin nämlich kaum praktischen Anwendungsgebiete gehabt.


\paragraph{Ein Szenario}
Ein Kunde möchte seiner Bank über einen unsicheren/öffentlichen Weg eine geheime Botschaft übermitteln.
Bezeichne den Kunden mit $K$, die Bank mit $B$.
$K$ teilt $B$ mit, dass er eine Nachricht $m$ übermitteln will.
$B$ wählet zwei (große) Primzahlen $p$ und $q$, $p \neq q$ und berechnet $n := pq$ und $\phi(n) = (p-1)(q-1)$.
$B$ wählt $e$ mit $(e, \phi(n)) = 1$.
Dann berechnet $B$ ein $d \in \N$ mit $ed \equiv 1 \bmod \phi(n)$.
Soein $d$ existiert, da $(e, \phi(n)) = 1$.
Wie berechnet man $d$?
Mit dem Lemma von Beszout ist $z_1 e + z_2 \phi(n) = 1$ und der euklidische Algorithmus liefert $z_1 = d$.
$B$ sendet das Paar $(n, e)$ (“public key”) öffentlich an $K$, $d$ bleibt geheim.
$K$ sendet $m$ (als Restklasse modulo $n$ betrachtet) mittels Chiffrierung
\[
	m \mapsto m^e \mod n
\]
an $B$.
$(n, e)$ und $m^e$ sind also öffentlich bekannt, während $m$ geheim bleibt.
$B$ dechiffriert mittels $d$
\[
	m \equiv m^{ed} \mod n
\]
Warum funktioniert dies?


Zunächst eine Art Verallgemeinerung des kleinen Satzes von Fermat.

\setcounter{thm}{1}
% Lem 3.2
\begin{lem} \label{3.2}
	In $\Z / n \Z$ gilt für $n = pq$ mit $p,q \in \P$, dass
	\[
		x^{k\phi(n) + 1} \equiv x \mod n
	\]
	für jedes $x$.
	\begin{proof}
		Mit \ref{1.14} ist $Z / n\Z = \Z /p\Z \times \Z q\Z$ und $x = (x_1, x_2)$, also $x^{\phi(n)} = (x_1^{\phi(n)}, x_2^{\phi(n)})$ und
		\[
			x_1^{\phi(n)}
			= x_1^{(p-1)(q-1)}
			= (x_1^{p-1})^{q-1}
			\equiv 1^{q-1}
			= 1,
		\]
		wenn $x_1 \neq 0$, oder $x_1^{\phi(n)} \equiv 0$, wenn $x_1 = 0$.
		Analog gilt
		\[
			x_2^{\phi(n)} \equiv \begin{cases}
				1 & x_2 \neq 0 \\
				0 & x_2 = 0
			\end{cases},
		\]
		also ist
		\[
			x^{k\phi(n)} = \begin{cases}
				(1, 0) & x_2=0, x_1 \neq 0 \\
				(0,1) & x_1=0, x_2 \neq 0 \\
				(1,1) & x_1\neq 0, x_2 \neq 0
			\end{cases}
		\]
		und
		\[
			x^{k\phi(n) + 1} = \begin{cases}
				(x_1, 0) & x_2=0, x_1 \neq 0 \\
				(0,x_2) & x_1=0, x_2 \neq 0 \\
				(x_1,x_2) & x_1\neq 0, x_2 \neq 0
			\end{cases},
		\]
		also in allen Fällen $x^{k\phi(n) + 1} = x$.
	\end{proof}
\end{lem}

Damit ein Außenstehender an die geheime Nachricht $m$ kommt, benötigt er $d$.

% Lem 3.3
\begin{lem} \label{3.3}
	Die Kenntnis von $n$ und $\phi(n)$ ist äquivalent zur Kenntnis von $p$ und $q$.
	\begin{proof}
		\begin{segnb}{$\impliedby$}
			Klar, denn für bekanntes $p, q$ ist $n = pq$ und $\phi(n) = (p-1)(q-1)$.
		\end{segnb}
		\begin{segnb}{$\implies$}
			Seien $n$ und $\phi(n)$ bekannt.
			\[
				\phi(n)
				= (p-1)(q-1)
				= pq - p - q + 1
				= n - p - q + 1,
			\]
			also
			\[
				n = pq = nq - \phi(n)q - q^2 + q,
			\]
			bzw.
			\[
				q^2 + (\phi(n) - n - 1)q + n = 0.
			\]
			Aus dieser quadratischen Gleichung ergeben sich $p, q$ als Lösungen.
		\end{segnb}
	\end{proof}
\end{lem}

Um $d$ zu berechnen benötigt man $\phi(n)$ und für $\phi(n)$ benötigt man $p$ und $q$.
Man benötigt also Verfahren, um große Zahlen in Primfaktoren zu zerlegen.
Dies ist schwer, oft praktisch unmöglich.


