\chapter{Kryptographie, Primzahltests}



% 3.1
\section{Das RSA-Verfahren}

Das Verfahren wurde 1977/1978 am MIT von Rivest, Shamir und Adelman entwickelt (soll aber bereits vorher, zumindest in ähnlicher Form MI6 bekannt gewesen sein).

Das RSA-Verfahren ist ein sogenanntes asymmetrisches kryptographisches Verfahren.
Solche finden Anwendung bei Verschlüsselung von Nachrichten, bei digitalen Signaturen, Bezahlung mit Kreditkarten, Pay-TV.

Mathematisch gesehen war es der Retter der Königin der Mathematik (der elementaren Zahlentheorie).
Diese hatte bis dahin nämlich kaum praktischen Anwendungsgebiete gehabt.


\paragraph{Ein Szenario}
Ein Kunde möchte seiner Bank über einen unsicheren/öffentlichen Weg eine geheime Botschaft übermitteln.
Bezeichne den Kunden mit $K$, die Bank mit $B$.
$K$ teilt $B$ mit, dass er eine Nachricht $m$ übermitteln will.
$B$ wählet zwei (große) Primzahlen $p$ und $q$, $p \neq q$ und berechnet $n := pq$ und $\phi(n) = (p-1)(q-1)$.
$B$ wählt $e$ mit $(e, \phi(n)) = 1$.
Dann berechnet $B$ ein $d \in \N$ mit $ed \equiv 1 \bmod \phi(n)$.
Soein $d$ existiert, da $(e, \phi(n)) = 1$.
Wie berechnet man $d$?
Mit dem Lemma von Beszout ist $z_1 e + z_2 \phi(n) = 1$ und der euklidische Algorithmus liefert $z_1 = d$.
$B$ sendet das Paar $(n, e)$ (“public key”) öffentlich an $K$, $d$ bleibt geheim.
$K$ sendet $m$ (als Restklasse modulo $n$ betrachtet) mittels Chiffrierung
\[
	m \mapsto m^e \mod n
\]
an $B$.
$(n, e)$ und $m^e$ sind also öffentlich bekannt, während $m$ geheim bleibt.
$B$ dechiffriert mittels $d$
\[
	m \equiv m^{ed} \mod n
\]
Warum funktioniert dies?


Zunächst eine Art Verallgemeinerung des kleinen Satzes von Fermat.

\setcounter{thm}{1}
% Lem 3.2
\begin{lem} \label{3.2}
	In $\Z / n \Z$ gilt für $n = pq$ mit $p,q \in \P$, dass
	\[
		x^{k\phi(n) + 1} \equiv x \mod n
	\]
	für jedes $x$.
	\begin{proof}
		Mit \ref{1.14} ist $Z / n\Z = \Z /p\Z \times \Z q\Z$ und $x = (x_1, x_2)$, also $x^{\phi(n)} = (x_1^{\phi(n)}, x_2^{\phi(n)})$ und
		\[
			x_1^{\phi(n)}
			= x_1^{(p-1)(q-1)}
			= (x_1^{p-1})^{q-1}
			\equiv 1^{q-1}
			= 1,
		\]
		wenn $x_1 \neq 0$, oder $x_1^{\phi(n)} \equiv 0$, wenn $x_1 = 0$.
		Analog gilt
		\[
			x_2^{\phi(n)} \equiv \begin{cases}
				1 & x_2 \neq 0 \\
				0 & x_2 = 0
			\end{cases},
		\]
		also ist
		\[
			x^{k\phi(n)} = \begin{cases}
				(1, 0) & x_2=0, x_1 \neq 0 \\
				(0,1) & x_1=0, x_2 \neq 0 \\
				(1,1) & x_1\neq 0, x_2 \neq 0
			\end{cases}
		\]
		und
		\[
			x^{k\phi(n) + 1} = \begin{cases}
				(x_1, 0) & x_2=0, x_1 \neq 0 \\
				(0,x_2) & x_1=0, x_2 \neq 0 \\
				(x_1,x_2) & x_1\neq 0, x_2 \neq 0
			\end{cases},
		\]
		also in allen Fällen $x^{k\phi(n) + 1} = x$.
	\end{proof}
\end{lem}

Damit ein Außenstehender an die geheime Nachricht $m$ kommt, benötigt er $d$.

% Lem 3.3
\begin{lem} \label{3.3}
	Die Kenntnis von $n$ und $\phi(n)$ ist äquivalent zur Kenntnis von $p$ und $q$.
	\begin{proof}
		\begin{segnb}{$\impliedby$}
			Klar, denn für bekanntes $p, q$ ist $n = pq$ und $\phi(n) = (p-1)(q-1)$.
		\end{segnb}
		\begin{segnb}{$\implies$}
			Seien $n$ und $\phi(n)$ bekannt.
			\[
				\phi(n)
				= (p-1)(q-1)
				= pq - p - q + 1
				= n - p - q + 1,
			\]
			also
			\[
				n = pq = nq - \phi(n)q - q^2 + q,
			\]
			bzw.
			\[
				q^2 + (\phi(n) - n - 1)q + n = 0.
			\]
			Aus dieser quadratischen Gleichung ergeben sich $p, q$ als Lösungen.
		\end{segnb}
	\end{proof}
\end{lem}

Um $d$ zu berechnen benötigt man $\phi(n)$ und für $\phi(n)$ benötigt man $p$ und $q$.
Man benötigt also Verfahren, um große Zahlen in Primfaktoren zu zerlegen.
Dies ist schwer, oft praktisch unmöglich.

\coursetimestamp{12}{05}{2014}


% Ex 3.4
\begin{ex} \label{3.4}
	\begin{enumerate}[a)]
		\item
			Sei $n = 11 \cdot 31 = 341$, dann ist $\phi(n) = 300$.
			Die Mitternachtsformel liefert für
			\[
				q^2 + q(300 - 342) + 341 = 0
			\]
			die Lösung $q_{1,2} = \f{42 \pm \sqrt{42^2 - 4\cdot 341}}2 = 21 \pm 10$.
		\item
			Sei $(n, e) = (667, 15)$.
			Gesendet wird $424$, wie lautet die Nachricht von K an B?.
	\end{enumerate}
\end{ex}

Das RSA-Verfahren liefert einen Grund gegegben Zahlen $n \in \N$ zu faktorisieren, oder zumindest zu prüfen, ob $n$ eine Primzahl ist.
Dies motiviert Primzahltests.

% Prop 3.5
\begin{prop}[Fermat'scher Primzahltest] \label{3.5}
	Sei $n \in \N$.
	Falls $a \in \N$ existiert mit $(a,n) = 1$ und $a^{n-1} \not\equiv 1 \bmod n$, dann ist $n$ keine Primzahl.
	\begin{proof}
		Umkehrung des kleinen Satzes von Fermat \ref{2.9} und Übung P4.4 auf Übungsblatt 4.
	\end{proof}
\end{prop}

% Def 3.6
\begin{df} \label{3.6}
	\begin{enumerate}[a)]
		\item
			$n \in \N$ nennt man \emphdef[Pseudoprimzahl]{Pseudoprimzahl zur Basis $a$}, wenn $n$ keine Primzahl ist, aber $a^{n-1} \equiv 1 \bmod n$ gilt.
		\item
			$n \in \N$ nennt man \emphdef{Carlmichaelzahl}, wenn $n$ keine Primzahl ist, aber für alle $a \in \N$ mit $(a,n) = 1$ gilt, dass $a^{n-1} \equiv 1 \bmod n$.
	\end{enumerate}
	\begin{note}
		Carlmichaelzahlen  kommen von der Umkehrung des kleinen Fermats: ist $p$ Primzahl, dann ist $a^{p-1} \equiv 1 \bmod p$ für alle $(a,p) = 1$.
		Man hat sich also die Frage gestellt, für welche Zahlen die Umkehrung nicht gilt, diese nennt man Carlmichaelzahlen.
	\end{note}
\end{df}

% Ex 3.7
\begin{ex} \label{3.7}
	\begin{enumerate}[a)]
		\item
			Sei $n=9, a= 2$.
			Dann ist $a^3 = 8 \equiv -1 \bmod 9$, also ist $a^6 \equiv 1 \bmod 9$ und $a^8 \not\equiv 1 \bmod 9$
			Also ist $9$ keine Primzahl.
		\item
			$341$ ist eine Pseudoprimzahl zur Basis $2$, denn $2^10 = 1024 = 3\cdot 341 + 1$, also ist $2^10 \equiv 1 \bmod 341$ und $2^{340} \equiv 1 \bmod 341$.
		\item
			$561 = 51 \cdot 11 = 3 \cdot 17 \cdot 11$ ist eine Carmichaelzahl (sogar die kleinste).

			Dies manuell nachzuprüfen ist mühsam.
			Glücklicherweise liefert der folgende Satz eine einfacherer Charakterisierung.
	\end{enumerate}
\end{ex}

% St 3.8
\begin{st}[Korselt, 1899] \label{3.8}
	$n$ ist genau dann eine Carmichaelzahl, wenn $n$ quadratfrei ist (d.h. kein Quadrat als Teiler hat) und für alle ihre Primteile $p$ gilt, dass $p-1$ ein Teiler von $n-1$ ist und $n = p_1 p_2 \cdot p_k$ mit $k \ge 3$ und $p_i$ ungerade Primzahl.
	\begin{proof}
		\begin{enumerate}[1.]
			\item
				Wenn $n = \prod_{i=1}^k p_i$ mit $p_i$ Primzahl, $p_i \neq p_j$ für $i \neq j$, dann ist mit \ref{1.14}
				\[
					U(\Z / n\Z) = U(\Z / p_1\Z) \times \dotsb \times U(\Z / p_k \Z).
				\]
				Aus $u \in U(\Z / n\Z)$ folgt $u^m = 1$ für $m = \kgV \{p_i-1 : 1 \le i \le k\}$.
				Nach Voraussetzung wird $n - 1$ von $m$ geteilt, also $u^{n-1} = 1$.
				$u$ repräsentiert jedes $a$ mit $(a,n) = 1$.
			\item
				Zeige: ist $n$ nicht quadratfrei, dann ist $n$ keine Carmichaelzahl.

				Sei $n = p^m a$ mit $(p,a) = 1$, Primzahl $p$ und $m \ge 2$.
				Aus $b^{n-1} \equiv 1 \bmod n$ folgt $b^{n-1} \equiv 1 \bmod p^m$, also $\phi(p^m)$ teilt $n-1$, denn $U(\Z / p^m \Z)$ ist zyklisch von Ordnung $\phi(p^m)$.
				Es folgt $p^{m-1} (p-1)$ teilt $n-1$.
				Da aber $n$ von $p$ geteilt wird und wegen $m \ge 2$ $p$ auch $n-1$ teilt, folgt dann $p$ teilt $n - (n-1) = 1$, ein Widerspruch.
			\item
				Ist $n$ eine Carmichaelzahl, also $p-1$ teilt $n-1$, wenn $p$ ein Teiler von $n$ ist.
				Zerlege $U(\Z / n\Z) = U(\Z / p_1 \Z) \times \dotsb \times U(\Z / p_k \Z)$.
				Also $1 = a^{n-1} = (a_1^{n-1}, \dotsc, a_k^{n-1})$.
				Insbesondere kann man die $a_i$ so wählen, dass sie $U(\Z/p_i \Z)$ erzeugen.
				$p_i - 1$ teilt $n-1$ für alle $1 \le i \le k$.
			\item
				$n$ ist ungerade, wenn $n$ Carmichaelzahl ist.
				Angenommen $2$ teile $n$, dann ist $-1 = (-1)^{n-1} \equiv 1 \bmod n$, da $n$ Carmichael, also $n \divs 2$ und somit $n = 2$, ein Widerspruch.
		\end{enumerate}
	\end{proof}
\end{st}

\begin{ex*}
	Sei $561 = 3 \cdot 17 \cdot 11$ die Zahl aus \ref{3.7} c).
	Offenbar ist $561$ quadratfrei und $2 \divs 560, 10 \divs 560, 16 \divs 560$, also ist $561$ eine Carmichaelzahl.
\end{ex*}

% Lem 2.9
\begin{lem} \label{2.9}
	Ist $n$ eine ungerade Pseudoprimzahl zur Basis $2$, dann ist auch $2^n - 1$ eine Pseudoprimzahl zur Basis $2$.
	\begin{proof}
		Es gilt $2^{n-1} \equiv 1 \bmod n$, also $2^{n-1} - 1 = kn$, also $2^{2^n-2} = 2^{2kn}$ und
		\[
			2^{2^n -2} - 1 = (2^n)^{2k} - 1.
		\]
		Also $2^n - 1 \divs 2^{2^n -2} - 1 = 2^{2^n - 1 - 1}$.
		Es bleibt zu zeigen, dass $2^{n-1}$ keine Primzahl ist.
		Sei $n$ keine Primzahl, dann folgt aus $d \divs n$, dass $2^d - 1 \divs 2^n - 1$ mit $d < n$.
		Dann ist $2^d - 1 < 2^n - 1$ und ein nicht trivialer Teiler.
	\end{proof}
	\begin{note}
		Im Beweis wurde mehrfach verwendet, dass
		\[
			2^{am} - 1
			= (2^q - 1)\Big(1 + 2^a + 2^{2a} + \dotsb + 2^{(m-1)a} \Big).
		\]
	\end{note}
\end{lem}

\begin{nt*}
	\begin{enumerate}[a)]
		\item
			Nach $3.9$ existieren unendlich viele Pseudoprimzahlen zur Basis 2.
			Lange zeit war es offen, ob es unendlich viele Carmichaelzahlen gibt.
			Seit 1992 (Granville) weiß man, dass es unendlich viele gibt.
		\item
			Den Fermat'schen Primzahltest kann man verbessern.
			Es gilt: $n$ ist eine Primzahl, wenn für alle $1 < m < n$ gilt $m^{n-1} \equiv 1 \bmod n$.
			Rückrichtung mit $| U(\Z/n\Z) | = n-1$, wenn $n$ eine Primzahl ist, Hinrichtung mit kleinem Fermat.

			Auch für Carmichaelzahlen $n$ gilt:
			Ist $p$ ein echter Teiler von $n$, dann ist $\_p = p \bmod n \not\in U(\Z / n\Z)$.
	\end{enumerate}
\end{nt*}

% St 3.10
\begin{st}[Lucas] \label{3.10}
	Sei $n > 1$, $n \in \N$.
	Wenn für jeden Primteiler $p$ von $n - 1$ eine ganze Zahl $a = a(p)$ existiert mit $a^{n-1} \equiv 1 \bmod n$ und $a^{\f {n-1}p} \not \equiv 1 \bmod n$, dann ist $n$ eine Primzahl.

	\begin{proof}
		Um zu zeigen, dass $n$ eine Primzahl ist, genügt es $\phi(n) = n - 1$ zu zeigen.
		Sei $p$ eine Primzahl und $p^r$ der maximale $p$-Potenzteiler von $n-1$.
		Sei $a = a(p)$ und $e$ die Ordnung von $a \bmod n$.
		Dann ist $e$ ein Teiler von $n-1$ und $e$ teilt nicht $\f {n-1}p$, also ist $p^r \divs e$.
		Sicherlich $e \divs \phi(n)$, dann $p^r \divs \phi(n)$.
		Dies gilt für jeden Primteiler von $n-1$, also folgt $n - 1 \divs \phi(n)$ und damit $\phi(n) = n-1$.
	\end{proof}
\end{st}

% Nt + Ex 3.11
\begin{nt} \label{3.11}
	Um \ref{3.10} anwenden zu können, sollte man die Primzahlen von $n-1$ kennen.
	In Spezialfallen ist dies der Fall, z.B. für $n = 2^{16} + 1 = 65537$.

	Nach \ref{3.10} genügt zum Primzahlnachweis eine Zahl $a$ zu finden mit $a^{2^{16}} \equiv 1 \bmod 2^{16} + 1$ und $a^{2^{15}} \not\equiv 1 \bmod 2^{16} + 1$.
	Man kann dann ausrechnen, dass dies für $a = 3$ der Fall ist.
\end{nt}

Nachstes Thema sind Primzahltests im Zusammenhang mit Fermatzahlen und Mersennezahlen.

% Lem 3.12
\begin{lem} \label{3.12}
	\begin{enumerate}[a)]
		\item
			Ist $2^k - 1$ eine Primzahl (Mersenneprimzahl), dann ist $k$ eine Primzahl.
		\item
			Ist $2^k + 1$ eine Primzahl, dann ist $k = 2^n$.
	\end{enumerate}
	\begin{proof}
		\begin{enumerate}[a)]
			\item
				Ist $k = uv$, dann ist $2^k - 1 = (2^n)^v - 1$.

				$x-1 \divs x^v - 1$, da $1$ Nullstelle, also $2^n -1 \divs 2^k - 1$.
			\item
				$k = 2^r \_ u$ mit $\_ u$ ungerade, dann ist $2^k + 1 = (2^{2r})^u + 1$.
				$x + 1 \divs x^{\_ u} + 1$, da $-1$ Nullstelle von $x^{\_ u} + 1$, also $2^{2r} + 1 \divs 2^k + 1$.
		\end{enumerate}
	\end{proof}
\end{lem}

% Df 3.13
\begin{df} \label{3.13}
	\begin{enumerate}[a)]
		\item
			Die Zahlen $M_p = 2^p - 1$ heißen \emphdef[Mersenne-Zahl]{Mersenne-Zahlen} (nach Marin Mersenne, 1588-1648).
		\item
			Die Zahlen $F_n = 2^{2^n} + 1$ heißen \emphdef[Fermat-Zahl]{Fermat-Zahlen} (nach P. de Fermat, 1601-1665).
	\end{enumerate}
\end{df}

% St 3.14
\begin{st} \label{3.14}
	$F_k$ für $k \ge 1$ ist genau dann eine Primzahl, wenn
	\[
		3^{\f{F_k - 1}2} \equiv -1 \mod F_k
	\]
	\begin{note}
		Dies zeigt, dass $3$ eine Primitivwurzel ist für Fermat'sche Primzahlen.
		\ref{3.14} verifiziert \ref{3.11}.
		\ref{3.10} steckt mit $a = 3$ im Beweis von \ref{3.14}.

		Angenommen $3$ wäre nur endlich oft Primitivwurzel von $\Z / p\Z$ mit Primzahl $p$, dann existieren nur endlich viele Fermatzahlen, die Primzahlen sind.
		Man kennt $F_0 = 3, F_1 = 5, F_2 = 17, F_3 = 257, F_4 = 65537$, aber \emph{andererseits} legt die Artin'sche Vermutung nache, dass so ein Beweis nicht zu führen ist.
	\end{note}
\end{st}
