\chapter{Einführung in die algebraische Zahlentheorie}


\begin{df} \label{8.1}
	Sei $L / K$ eine endliche Körpererweiterung, $x \in L$.

	Wir bezeichnen mit $T_x: L \to L$ den $K$-linearen Endomorphismus $a \mapsto x a$, die Linksmultiplikation mit $x$.
	\begin{enumerate}[a)]
		\item
			Die Spur von $T_x$ nennt man die \emphdef{Spur} $\Tr_{L/K}(x)$.
		\item
			Die Determinante von $T_x$ nennt man die \emphdef{Norm} $N_{L/K}(x)$.
	\end{enumerate}
\end{df}

% Bsp + Bem
\begin{nt} \label{8.2}
	\begin{enumerate}[a)]
		\item
			Spur und Determinante von Vektorraumendomorphismen sind unabhängig von der Wahl der Basis.
		\item
			Ist
			\[
				\chi_{T_x}(t)
				= t^n - a_1 t^{n-1} + \dotsb + (-1)^n a_n
			\]
			mit $a_1 \in K$ das charakteristische Polynom von $T_x$, dann ist $\Tr_{L/K}(x) = a_1$ und $N_{L/K}(x) = a_n$.

			Ist $M_x = (m_{ij})$ die zu $T_x$ gehörige Matrix in einer Matrixdarstellung nach Wahl einer Basis, dann ist $\Tr_{L/K}(x) = \sum_{i=1}^n m_{ij}$, $n = \dim_K L$ und $N_{L/K}(x) = \det M_x$.
		\item
			Sei $L = \Q(\sqrt d)$ mit $d \in \Z$ quadratfrei.
			Wähle $\Set{1, \sqrt d}$ als $\Q$-Basis.
			Setze $x = a + b \sqrt d$ mit $a, b \in \Q$.
			Dann ist $M_x = \Matrix{a & bd \\ b & a}$ wegen $x \sqrt d = a \sqrt d + b d$.
			Also ist
			\begin{align*}
				\Tr_{L/\Q}(x) &= 2a, &
				N_{L/\Q}(x) &= a^2 - db^2.
			\end{align*}
	\end{enumerate}
\end{nt}

% Lem 8.3
\begin{lem} \label{8.3}
	\begin{enumerate}[a)]
		\item
			Die Abbildung $L \to K, x \mapsto \Tr_{L/K}(x)$ ist ein Gruppenhomomorphismus $(L, +) \to (K, +)$ und es gilt
			\[
				\Tr_{L/K}(kx) = k \Tr_{L/K}(x)
			\]
			für $k \in K$.
		\item
			Die Abbildung $L^* \to K^*, x \mapsto N_{L/K}(x)$ ist ein Gruppenhomomorphismus $(L^*, \cdot) \to (K^*, \cdot)$ zwischen den Einheitengruppen von $L$ und $K$.
	\end{enumerate}
	\begin{proof}
		Einfach nachzurechnen (Spur ist additiv, Determinante multiplikativ, \dots).
	\end{proof}
\end{lem}

% Wiederholung ... 8.4
\begin{nt}[Wiederholung von Begriffen und Sätzen aus der Algebra] \label{8.4}
	\begin{enumerate}[a)]
		\item
			Eine Körpererweiterung $L / K$ heißt \emphdef{separabel}, wenn jedes Element $x \in L$ ein Minimalpolynom $\mu_x \in K[x]$ besitzt, welches in einem algebraisch abgeschlossenen Körper $\_K$ keine mehrfachen Nullstellen hat.
		\item
			Körpererweiterungen über Körper der Charakteristik 0 sind separabel.
			Insbesondere ist $L / \Q$ mit algebraischem Zahlkörper $L$ stets separabel.
			Für algebraische Zahlkörper kann man $\_K = \C$ wählen.
		\item
			Sei $L / K$ separabel und $\dim_K L = n$.
			Dann gibt es genau $n$ verschiedene $K$-Homomorphismen $\sigma_i: L \to \_K$.
			Dabei ist ein $K$-Homomorphismus ein Körperhomomorphismus, der $K$ elementweise festlässt, d.h. $\sigma_i(k) = k$ für alle $k \in K$.
	\end{enumerate}
\end{nt}

% St 8.5
\begin{st} \label{8.5}
	Sei $L / K$ separabel, $n = \dim_K L$ und $\Hom_K(L, \_K) = \Set{\sigma_1, \dotsc, \sigma_n}$ die Menge der $K$-Homomorphismen nach $\_K$.
	Dann gilt für ein $x \in L$, dass
	\begin{enumerate}[(i)]
		\item
			$\chi_{T_x}(t) = \prod_{i=1}^n (t - \sigma_i(x))$,
		\item
			$\Tr_{L/K}(x) = \sum_{i=1}^n \sigma_i(x)$,
		\item
			$N_{L/K}(x) = \prod_{i=1}^n \sigma_i(x)$.
	\end{enumerate}
	\begin{proof}
		(ii) und (iii) folgen unmittelbar aus (i).
	\end{proof}
	\begin{note}
		Ist $L / K$ sogar galoissch, dann bilden $\Set{\sigma_1, \dotsc, \sigma_n}$ die Galoisgruppe und $\sigma_i$ sind insbesondere Körperautomorphismen von $L$.
	\end{note}
\end{st}

\coursetimestamp{10}{07}{2014}

% Folgerung 8.6
\begin{kor} \label{8.6}
	Sei $L / K$ eine Körpererweiterung, $A$ der Ring der ganzen Zahlen in $K$ und $B$ der ganzzahlige Abschluss von $A$ in $L$.
	In diesem Fall gilt
	\[
		\Tr_{L/K}(B) \subset A.
	\]
	Insbesondere gilt dies für Zahlkörper $K = \Q$, $A = \Z$.
	\begin{proof}
		Sei $b \in B$. Dann ist $b^n + k_1 b^{n-1} + \dotsb + k_n = 0$ mit $k_i \in A$.
		Sicherlich ist $\Tr_{L/K}(b) \in K$.
		$\sigma_j(b)$ ist auch ganz über $K$ für jeden Index $j$, denn weil $\sigma_j$ ein $K$-Homomorphismus ist, gilt
		\begin{align*}
			\sigma_j(0)
			= 0
			&= \sigma_j(b^n + k_1b^{n-1} + \dotsb + k_n) \\
			&= \sigma_j(b)^n + k_1 \sigma_j(b)^{n-1} + \dotsb + k_n.
		\end{align*}
		Nach \ref{8.5} gilt
		\[
			\Tr_{L/K}(b)
			= \underbrace{\sum_{i=1}^n \underbrace{\sigma_i(b)}_{\text{ganz}}}_{\text{ganz}}
			\in A.
		\]
	\end{proof}
\end{kor}

% Def 8.7
\begin{df} \label{8.7}
	Sei $L/K$ separabel, $S = \Set{\alpha_1, \dotsc, \alpha_n}$ eine $K$-Basis von $L$ und $\Hom_{K}(L, \_K) = \Set{ \sigma_1, \dotsc, \sigma_n}$.
	Dann heißt
	\[
		d(s) = \det((\sigma_i(\alpha_j))^2)
	\]
	\emphdef{Diskriminante} von $S$.
\end{df}

% Lem 8.8
\begin{lem} \label{8.8}
	Sei $L/K$ separabel und $S = \Set{\alpha_1, \dotsc, \alpha_n}$ eine $K$-Basis von $L$.
	\begin{enumerate}[a)]
		\item
			Dann gilt $d(S) = \det T$ mit $T = (t_{ij})$ und $t_{ij} = \Tr_{L/K}(\alpha_i\alpha_j)$.
		\item
			Ist $S = \Set{1, \theta, \dotsc, \theta^{n-1}}$, dann ist
			\[
				d(S) = \prod_{i<j} (\sigma_i(\theta) - \sigma_j(\theta))^2.
			\]
	\end{enumerate}
	\begin{proof}
		\begin{enumerate}[a)]
			\item
				Schreibe mit \ref{8.5} (ii)
				\begin{align*}
					\Tr_{L/K}(\alpha_i \alpha_j)
					= \sum_{k=1}^n \sigma_k (\alpha_i \alpha_j)
					= \sum_{k=1}^n \sigma_k(\alpha_i) \sigma_k(\alpha_j).
				\end{align*}
				Setze $B = (b_{ij})$ mit $b_{ij} = \sigma_i(\alpha_j)$.
				Dann ist $BB^t = T$ mit $T = (\Tr_{L/K}(\alpha_i \alpha_j))$.
				Es folgt
				\[
					d(S) = \det(B)^2
					= \det(B)\det(B^t)
					= \det T.
				\]
			\item
				Nach Definition und mit Multiplikativität ist
				\[
					d(S) = \det \Matrix{
						1 & \sigma_1(\theta) & \cdots & \sigma_1(\theta)^{n-1} \\
						\vdots & \vdots & \ddots & \vdots \\
						1 & \sigma_n(\theta) & \cdots & \sigma_n(\theta)^{n-1}
					}^2.
				\]
				Dies ist eine Vandermonde-Matrix, für die sich die Determinante genau wie in der Behauptung ergibt.
		\end{enumerate}
	\end{proof}
\end{lem}

% Lem 8.9
\begin{lem} \label{8.9}
	Sei $L / K$ eine Körpererweiterung, $\dim_K L = n$, $A$ ganz abgeschlossen in $K$ und $B$ der ganze Abschluss von $A$ in $L$.
	Sei $S = \Set{\beta_1, \dotsc, \beta_n}$ eine in $B$ gelegene $K$-Basis von $L$.
	Sei $d = d(S)$ die Diskriminante von $S$.
	Dann gilt
	\[
		d B \subset \< \beta_1, \dotsc, \beta_n\>_A.
	\]
	\begin{proof}
		Sei $\beta \in B$, dann ist $\beta = b_1 \beta_1 + \dotsb + b_n \beta_n$ mit $b_j \in K$.
		Es genügt zu zeigen, dass $db_j \in A$ für $1 \le j \le n$, d.h. $d \beta \in \< \beta_1, \dotsc, \beta_n\>_A$.
		Für $1 \le i \le n$ gilt
		\[
			\Tr_{L/K}(\beta_i\beta)
			\stack{\ref{8.3}a)}= \sum_{j} \Tr_{L/K} (\beta_i b_j \beta_j)
			= \sum_{j} b_j \Tr_{L/K}(\beta_i \beta_j).
		\]
		Betrachte das lineare Gleichungssystem der Form
		\[
			\Tr_{L/K}(\beta_i\beta) = \sum_{j=1}^n \Tr_{L/K}(\beta_i\beta_j) x_j
		\]
		für $1 \le j \le n$.
		Die $b_j$ sind nach vorigem Lösungen dieses Gleichungssystems.
		Nach der Cramerschen Regel sind die $b_j$ Quotienten eines in $A$ gelegenen Zählers, da die Koeffizienten $\Tr_{L/K}(\beta_i\beta_j) \in A$ nach \ref{8.6} und $\det(\Tr_{L/K}(\beta_i\beta_j)) = d$ nach \ref{8.8}a).
		Damit ist $d\beta \in \<\beta_1, \dotsc, \beta_n\>$, denn $d \beta_j \in A$.
	\end{proof}
\end{lem}

% St 8.10
\begin{st}[Existenz von Ganzheitsbasen] \label{8.10}
	Sei $L / K$ endlich und separabel, $A$ Hauptidealbereich und ganz abgeschlossen in $K$, $B$ der ganze Abschluss von $A$ in $L$.
	Dann ist jeder endlich erzeugte $B$-Teilmodul $M$ von $L$ ($L$ kann als Einschränkung als $B$-Modul betrachtet werden) ein freier $A$-Modul ($M$ wird durch weitere Einschränkung auch als $A$-Modul betrachtet) vom Rang $n = \dim_K L$.

	Insbesondere gilt dies für $M = B$, d.h. $B$ ist ein freier $A$-Modul vom Rang $n$ und besitzt somit eine Ganzheitsbasis (siehe \ref{5.12}, \ref{5.13}).
	\begin{proof}
		Sei $\Set{\alpha_1, \dotsc, \alpha_n}$ eine $K$-Basis von $L$.
		Dann ist wegen $\alpha_i = \f{x_i}{y_i}$, $x_i \in B, y_i \in A$ (nach \ref{5.9}) liefert Durchmultiplizieren mit $y := \prod_{i=1}^n y_i$, dass $S := \Set{y\alpha_1, \dotsc, y\alpha_n}$ eine $K$-Basis von $L$, welche in $B$ liegt.

		Mit \ref{8.9} ist $d(S)B \subset \<y\alpha_1, \dotsc, y\alpha_n\>_A$.
		Sei nun $M$ ein endlich erzeugter $B$-Teilmodul von $L$ und $M = \< \gamma_1, \dotsc, \gamma_r \>_B$.
		Analog wie vorher liefert Durchmultiplizieren mit $\prod_{i=1}^r w_i$, wenn $\gamma_i = \f {\delta_i}{w_i}$, $\delta_i \in B, w_i \in A$, dass $wM = \<w\gamma_1, \dotsc, w\gamma_r\> \subset B$.

		Es ist dann $d(S) w M \subset d(S) B \subset \<y \alpha_1, \dotsc, y \alpha_n\>_A$.
		$\<y\alpha_1, \dotsc, y \alpha_n\>_A$ ist offensichtlich ein freier $A$-Modul mit Basis $\Set{y\alpha_1, \dotsc, y\alpha_n}$.
		Da $A$ eine Hauptidealbereich nach Voraussetzung, sind Teilmoduln von freien $A$-Moduln vom Rang $m$ ebenfalls frei vom Rang $\le m$.
		Damit ist auch $d(S) w M$ ein freier $A$-Modul.
		Sicherlich ist $d(S) w M \isomorphic M$ als $A$-Modul und $\Rang M = \Rang d(S) w M \le \Rang \<y\alpha_1, \dotsc, y\alpha_n \> = n$.

		Für $m \in M$ ist $Bm \subset M$ und $Bm \isomorphic B$, also $\Rang B \le \Rang M$.
		Insegamt ist also
		\[
			\Rang M \le n \le \Rang B \le \Rang M,
		\]
		d.h. $n = \Rang M = \Rang B$.
	\end{proof}
\end{st}

% Bemerkungen 8.11
\begin{nt} \label{8.11}
	\begin{enumerate}[a)]
		\item
			\ref{8.10} kann man auf Zahlkörpersituationen anwenden $K = \Q, A = \Z$, denn $\Z$ ist eine Hauptidealbereich
			Die im Beweis von \ref{8.10} verwendete Eigenschaft von freien Moduln über Hauptidealbereichen überträgt sich im Fall $A = \Z$, dazu, dass Untergruppen freier abelscher Gruppen vom Range $n$ frei sind vom Rang $\le n$.
		\item
			Die Diskriminante einer Ganzheitsbasis $B$ eines Zahlkörpers $L$ nennt man auch \emphdef{Diskriminante} von $L$.
			Diese ist unabhängig von der Wahl der Ganzheitsbasis, denn seien $\Set{\alpha_1, \dotsc, \alpha_n}$ und $\Set{\alpha_1', \dotsc, \alpha_1'}$ Ganzheitsbasen von $B$, dann ist die Transformationsmatrix $T = (t_{ij})$ mit $\alpha_i' = \sum_{j=1}^n t_{ij} \alpha_j$ und ihre Inverse ganzzahlig, d.h. $\det T = \pm 1$.
			Damit ist
			\[
				d(\Set{\alpha_1', \dotsc \alpha_n'})
				= \det((\sigma_i(\alpha_j'))^2)
				= (\det T)^2 \det(\sigma_i(\alpha_i))^2
				= d(\Set{\alpha_1, \dotsc, \alpha_n}).
			\]
		\item
			Im Allgemeinen ist es nicht leicht, Ganzheitsbasen zu bestimmen.
			Im Fall von Kreisteilungskörpern $\Q(\zeta)$, mit einer primitiven $n$-ten Einheitswurzel $\zeta$, kann man zeigen, dass $\Set{1, \zeta, \dotsc, \zeta^{\phi(n) - 1}}$ eine Ganzheitsbasis ist, d.h.
			$\Z[\zeta]$ sind die ganzen Zahlen von $\Q(\zeta)$.
	\end{enumerate}
\end{nt}

% Bemerkungen 8.12
\begin{nt} \label{8.12}
	Wir haben gesehen, dass in den ganzen Zahlen quadritischer Zahlkörper die Eindeutigkeit in die Zerlegung von Primelementen in der Regel nicht gegeben ist.

	E. Kummer hat postuliert:
	Sei $L$ ein algebraischer Zahlkörper, $B$ die ganzen Zahlen von $L$, dann sollte es einen Bereich $\hat B$ geben, so dass in $\hat B$ eine eindeutige Primfaktorzerlegung via „idealer Primzahlen“ möglich ist.

	Präziser sollte folgendes gelten:
	Sei $\scr a$ eine eine ideale Zahl, dann soll sie mit $a, b, \lambda \in B$ folgende Teilbarkeitsregeln erfüllen
	\begin{enumerate}[{Postulat P}1]
		\item
			\begin{enumerate}[(i)]
				\item
					$\scr a$ „teile“ $a$ und $\scr a$ „teile“ $b$, dann auch $\scr a$ teile $a \pm b$,
				\item
					$\scr a$ „teile“ $a$, dann auch $a$ „teile“ $\lambda a$.
			\end{enumerate}
		\item
			$\scr a = \Set{ a \in B & \scr a \text{ „teile“ } a}$.
	\end{enumerate}
	R. Dedekind hat „ideale Zahlen“ eingeführt als Ideale mit „teile“ als Inklusion und idealen Primzahlen als Primidealen.
\end{nt}

% Wiederholung
\begin{df*}
	Sei $R$ ein kommutativer Ring.
	Ein Ideal $I \IdealOf R$ heißt \emphdef{Primideal}, wenn für Ideale $A$ und $B$ gilt
	\[
		I \supset A B \implies I \supset A \lor I \supset B
	\]
	(analog zu $p \divs ab \implies p\divs a \lor p \divs b$).
\end{df*}

% Lemma 8.13
\begin{lem} \label{8.13}
	Sei $R$ ein kommutativer Ring, $I \IdealOf R$ ein Ideal.
	\begin{enumerate}[a)]
		\item
			$I$ ist Primideal genau dann, wenn $R / I$ Integritätsbereich ist,
		\item
			$I$ ist maximales Ideal genau dann, wenn $R / I$ ein Körper ist.
	\end{enumerate}
	Insbesondere ist jedes maximale Ideal ein Primideal.
	\begin{note}
		In $\Z$ sind die Primideale von der Form $p\Z$ für eine Primzahl $p \in Z$, oder das Nullideal.

		In $\R$ nicht-triviale Primideale sind maximal, es existiert ein Primideal, welches nicht maximal ist, in jedem Integritätsbereich ist das Nullideal ein Primideal.
	\end{note}
\end{lem}
