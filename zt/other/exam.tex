\documentclass{scrartcl}

\usepackage[
	a4paper,
	outer          = 21mm,
	inner          = 21mm,
	top            = 20mm,
	bottom         = 30mm,
	marginparwidth = 18mm,
	marginparsep   =  3mm,
]{geometry}
\usepackage{mymath}

\usepackage{fontspec}
\usepackage{polyglossia}


\DeclareDocumentCommand{\question}{m}{\textbf{\color{blue!40!black}#1}}

\title{Zahlentheorie, Prüfung}
\author{}
\date{}

\begin{document}

\maketitle

\begin{enumerate}
	\item
		\question{Was sind Diophantische Gleichungen?}
		Man sucht ganzzahlige Lösungen für polynomielle Gleichungen.
	\item
		\question{Wir haben uns eingehend mit der Pellschen Gleichung beschäftigt. Wie sieht diese aus?}
		$x^2 - y^2 d = 1$
	\item
		\question{Woher stammt diese? Warum wollten wir sie lösen?}
		Wir wollten Einheiten in $\Z[\alpha_d]$ suchen und haben $N(u) = 1$ angesetzt.
	\item
		\question{Was ist denn $\Z[\alpha_d]$?}
		$\Z[\alpha_d]$ ist der Ring der ganzen Zahlen im quadratischen Zahlkörper $\Q(\sqrt d)$, insbesondere ein Teilring.
	\item
		\question{Und was hat das $\alpha_d$ zu bedeuten?}
		$\alpha_d = \sqrt d$ für $d \equiv 2,3 \bmod 4$ und $\alpha_d = \frac {1+\sqrt d}{2}$ für $d \equiv 1 \bmod 4$.
	\item
		\question{Welche Lösungen hat denn die Pellsche Gleichung?}
		Im Allgemeinen für negatives $d$ nur endlich viele, für positives $d$ unendlich viele.
	\item
		\question{Können Sie eine konkrete angeben?}
		$(\pm 1, 0)$ sind stets Lösungen.
	\item
		\question{Sie sagten, für negative $d$ hat sie nur endlich viele Lösungen. Wie sieht in dem Fall $U(\Z[\alpha_d])$ aus?}
		Beispielsweise für $d = -1$, die Gaußschen Zahlen: $U(\Z[i]) = (\{\pm 1, \pm i\}, \cdot) \isomorphic C_4$.
	\item
		\question{Gibt es für negative $d$ noch andere interessante Ringe ganzer Zahlen?}
		Für $d = -3$, die Eisensteinzahlen.
	\item
		\question{Wie sieht dort $U(\Z[\alpha_d])$ aus?}
		$U(\Z[\alpha_d]) \isomorphic C_6$.
	\item
		\question{Sie meinten, für positive $d$ gäbe es unendlich viele Lösungen, wie sehen diese aus? Wir haben diese ja klassifiziert.}
		$x + y\sqrt d = \pm (\xi + \eta \sqrt d)^n$ für $n \in \Z$, mit dem negativen $n$ hat man dann auch die Inversen Lösungen dabei.
	\item
		\question{Wie hatten wir das bewiesen?}
		Wir haben dazu das Kapitel über die Kettenbrüche entwickelt, man hat die Kettenbruchentwicklung von $\sqrt d$ genutzt.
	\item
		\question{Wie ist das mit den Kettenbrüchen? Gibt es endliche Kettenbrüche?}
		Ja, für alle rationalen Zahlen.
	\item
		\question{Also für irrationale Zahlen nicht. Gibt es auch periodische Kettenbrüche?}
		Ja, $\alpha \in \R$ liefert genau dann einen periodischen Kettenbruch, wenn es algebraisch vom Grad 2 über $\Q$ ist.
	\item
		\question{Ja, also im Grunde alle Wurzeln. Können Sie mal $\sqrt 2$ in einen Kettenbruch entwickeln?}
		(bischen gerechnet, ich hatte zum Schluss stehen:) $\sqrt 2 = [1; \_2]$.
	\item
		\question{Wir hatten da den Dirichletschen Einheitensatz. Wie sieht der aus und was hat er mit der Pellschen Gleichung zu tun?}
		Der Satz besagt: in einem Ring ganzer Zahlen $B$ eines quadratischen Zahlkörpers $L$ ergibt sich für die Einheitengruppe $U(B) \isomorphic \mu(L) \times \Z^{r+s-1}$.
		Dabei ist $\mu(L)$ die Gruppe der Einheitswurzeln in $L$ und $r = |\Hom_\Q(L,\R)|, s = |\Hom_\Q(L,\C) \setminus \Hom_\Q(L, \R)|$.
	\item
		\question{Was sind den diese $\Q$-Homomorphismen?}
		Das sind $\Q$-lineare Körperhomomorphismen.
	\item
		\question{Was passiert denn mit $\Q$ unter solchen Homomorphismen?}
		Da $\Q$-linear, wird $\Q$ fix gelassen.
	\item
		\question{Ja, alle Körperautomorphismen von Erweiterungskörpern von $\Q$ müssen $\Q$ fix lassen, $\Q$ lässt sich nur auf eine Art und Weise in $\R$ oder $\C$ einbetten, wir hätten also auch das $\Q$-linear streichen können. Wie sieht denn jetzt $r,s$ aus für positives $d$?}
		$r = 2$, da über einem quadratischen Zahlkörper, $s = 0$, da $\Z[\alpha_d] \subset \R$.
	\item
		\question{Also haben wir als Einheiten $U(\Z[\alpha_d]) = \{\pm 1\} \times \Z^1$, genau die Lösungen der Pellschen Gleichung. Was ist mit negativem $d$?}
		Zum Beispiel für $d = -1$ ist $\mu(L) = \{\pm 1, \pm i\}$ und $s = 1$.
		Für $r$ …
	\item
		\question{Naja, sie können ja wohl nicht etwas Komplexes in $\Q$ einbetten, oder?}
		Ja, das ergibt Sinn, dann ist $r = 0$ und für die Einheiten ergibt sich $\{\pm 1, \pm i\}$ wie oben.
	\item
		\question{Also, damit haben wir jetzt die Einheiten geklärt. Es gibt ja noch Primzahlen. Wie viele hat es denn davon?}
		Abzählbar unendlich viele.
	\item
		\question{Und genauer? Kennen sie was zur Verteilung der Primzahlen?}
		Wir hatten da den Primzahlsatz, der besagt dass für $\pi(x)$, die Anzahl der Primzahlen kleiner gleich $x$ gilt: $\pi(x) \sim \frac{x}{\log x}$.
	\item
		\question{Seit wann kennt man den Satz, oder wann wurde er bewiesen?}
		Hmm … Tschebycheff hatte 1850 eine Abschätzung bewiesen, aber wann der Satz bewiesen wurde …
	\item
		\question{Was für eine Abschätzung hat Tschebycheff denn bewiesen?}
		$a \frac{x}{\log x} \le \pi(x) \le A \frac{x}{\log x}$ für irgendwelche Konstanten $a < 1$ und $A > 1$, genau weiß ich es nicht mehr.
		\question{Wir hatten $a = \f 14$, $A = 6$ bewiesen.}
		Ja und Tschebyscheff hatte bessere Konstanten geliefert.
		\question{Ja, es geht sogar noch besser mit $a + A = 2$.}
	\item
		\question{Wer hat den Primzahlsatz bewiesen? Eigentlich waren es zwei Personen.}
		Hmm, einer war Franzose, aber ich weiß die Namen jetzt nicht.
	\item
		\question{Hadamard und de la Vallée-Poussin. Wer hat denn noch wichtige Vorarbeit geleistet, abgesehen von Tschebyscheff? Es war ein Deutscher.}
		(hatte keine Ahnung in dem Moment) …
		\question{Das war Riemann mit der Riemannschen Zetafunktion.}
		Ah, natürlich.
	\item
		\question{Es gibt besondere Primzahlen, z.B. die Mersenne-Primzahlen, wie sehen die aus?}
		$M_p = 2^p - 1$
	\item
		\question{Und was weiß man über sie? Wie viele gibt es?}
		Also bekannt sind sicherlich nur endlich viele, es gibt schließlich Wettbewerbe, um die nächstgrößere Primzahl zu entdecken.
	\item
		\question{Ja, man findet immer neue und vermutet dass es unendlich viele davon gibt. Können sie das irgendwie begründen?}
		Man kann das probabilistisch mit Hilfe des Primzahlsatzes machen: $P(x \in \P) \approx \frac{1}{\log x}$ und damit summiert $\sum_{p=0}^\infty \frac{1}{\log{M_p}} \approx \log 2 \sum_{p=0}^\infty \frac{1}{p}$ und die Reihe mit den Primzahlen divergiert, wie wir in einer Übung gezeigt haben.
	\item
		\question{Erinnern sie sich an eine ähnliche Begründung für Primzahlzwillinge?}
		Nein, leider nicht, man vermutet, dass es unendlich viele Primzahlzwillinge gibt.
		\question{Wir hatten da den Satz von Brun, dort konvergiert die Reihe über die Kehrwerte der Primzahlzwillinge nämlich.}
		Aber trotzdem vermutet man, dass es unendlich viele Primzahlzwillinge gibt?
		\question{Ja, die meisten Mathematiker vermuten das}.
	\item
		\question{Damit hätten wir die Primzahlen auch abgehandelt. Wie ist das mit $\Z[\alpha_d]$? Ist der Ring immer faktoriell?}
		Nein.
	\item
		\question{Also haben wir keine eindeutige Primfaktorzerlegung. Wie behilft man sich dann?}
		Man nutzt die eindeutige Zerlegung in Primideale.
	\item
		\question{Wann gibt es die denn?}
		Dazu wurden der Dedekindring eingeführt, dort ist die eindeutige Zerlegung sichergestellt.
		Jedes $\Z[\alpha_d]$ ist ein Dedekindring.
	\item
		\question{Was macht soeinen Dedekindring aus?}
		Integritätsbereich, noethersch, $\Gamma_{\mathrm{Quot}(R)}(R) = R$ und jedes Primideal ist maximal.
	\item
		\question{Wirklich \emph{jedes} Primideal?}
		Außer dem Nullideal.
	\item
		\question{Ja, wie zeigt man denn für die ganzen Zahlen, dass jedes Primideal maximal ist?}
		Hmm … wir hatten angefangen, dass für ein Primideal $P$ auch $P \cap \Z$ ein Primideal von $\Z$ ist.
	\item
		\question{Ja, im ausführlichen Beweis haben wir das so gemacht. Wichtig war, dass wir gezeigt haben, dass etwas ein Körper ist. Was war das?}
		Wir haben gezeigt, dass $R / P$ ein Körper ist, das ist äquivalent dazu, dass $P$ ein maximales Ideal ist (hier hatte er mir zwischendrin irgendwie geholfen, dann war die Zeit um und wir haben plötzlich abgebrochen).
\end{enumerate}


\end{document}
