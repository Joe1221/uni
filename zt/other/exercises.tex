\documentclass{scrartcl}

\usepackage[
	a4paper,
	outer          = 21mm,
	inner          = 21mm,
	top            = 20mm,
	bottom         = 30mm,
	marginparwidth = 18mm,
	marginparsep   =  3mm,
]{geometry}
\usepackage{mymath}

\usepackage{fontspec}
\usepackage{polyglossia}


\title{Zahlentheorie, Prüfungsvorbereitung}
\author{}
\date{}

\begin{document}

\maketitle

\setcounter{section}{-1}
\section{Einführung}

\begin{itemize}
	\item
		Womit beschäftigt sich die Zahlentheorie?
\end{itemize}

\section{Integritätsbereiche, irreduzibel Elemente, Primzahlen}

\begin{itemize}
	\item
		Was ist eine Primzahl?
		Warum ist die 1 keine Primzahl?
		Wie viele Primzahlen gibt es?
		Wie beweist man das?
	\item
		Was versteht man unter prim, irreduzibel?
		Wie hängen die Begriffe zusammen?
		Was ist ein faktorieller Ring, ein euklidischer Ring, ein Hauptidealring?
	\item
		Was besagt der Fundamentalsatz der Arithmetik?
		Wie beweist man ihn?
	\item
		Was besagt der Chinesische Restsatz, wie zerlegt man $\Z/n\Z$?
	\item
		Kann man auch $U(\Z/n\Z)$ zerlegen?
		Zerlege $U(\Z / 3\cdot 5 \cdot 7 \Z)$.
	\item
		Wie beweist man den Chinesischen Restsatz?
\end{itemize}

\section{Arithmetik modulo $n$}

\begin{itemize}
	\item
		Ist $\Z/p\Z$ zyklisch?
		Was ist mit einer endlichen Untergruppe von $K^*$ (Einheitengruppe eines Körpers $K$)?
	\item
		Wie ist die Struktur von $U(\Z/n\Z)$?
	\item
		Was ist das Legendre-Symbol?
		Wie lautet das quadratische Reziprozitätsgesetz?
		Wo haben wir das Legendre-Symbol verwendet?
\end{itemize}

\section{Kryptographie, Primzahltests}

\begin{itemize}
	\item
		Was besagt der kleine Fermat?
		Wie beweist man ihn?
	\item
		Existiert eine Umkehrung für den kleinen Fermat?
		Gegenbeispiel?
		Was sind Carmichaelzahlen?
	\item
		Wie lautet eine äquivalente Charakterisierung der Carmichaelzahlen?
		Wie beweist man diese?
\end{itemize}

\section{Arithmetik in quadratischen Zahlkörpern}

\begin{itemize}
	\item
		Was ist ein quadratischer Zahlkörper, $\Q(\sqrt d)$?
	\item
		Ist $\Q(\sqrt d) \to \Q(\sqrt d) : \beta \to \beta'$ ein Körperisomorphismus?
	\item
		Was was sind die ganzen Zahlen $\Z[\alpha_d]$ eines quadratischen Zahlkörpers $\Q(\sqrt d)$?
		Wie definiert man $\alpha$ in Abhängigkeit von $d$?
	\item
		Bilden die ganzen Zahlen im Allgemeinen einen euklidischen Ring, einen Dedekind-Ring?
		Ist die Primfaktorzerlegung in $\Z[\alpha_d]$ eindeutig?
		Welche bilden euklidische Ringe?
	\item
		Was ist die Normabbildung und wofür ist sie nützlich?
	\item
		Was passiert mit Primzahlen aus $\Z$ in $\Z[\alpha_d]$?
		Was versteht man unter „träge“, „verzweigt“, „zelegt“?
		Man nenne jeweils Beispiele für $\Z[i]$.
	\item
		Wie ist die Struktur von $U(\Z[\alpha_d])$?
	\item
		Wie sieht die Pellsche Gleichung aus?
		Wie viele Lösungen hat sie für $d > 0$?
		Wie viele Lösungen hat die negative Pell-Gleichung?
		Wie ist der Zusammenhang zur Einheitengruppe?
\end{itemize}

\section{Ganzheit und endlich erzeugte Moduln}

\begin{itemize}
	\item
		Was ist ein Modul?
		Wann ist dieser ganz abgeschlossen?
\end{itemize}

\section{Kettenbrüche}

\begin{itemize}
	\item
		Wie kann man Lösungen für Pellsche Gleichungen ermitteln?
\end{itemize}

\section{Primzahlsatz}

\begin{itemize}
	\item
		Was besagt der Primzahlsatz?
		Nennen sie Abschätzungen für $\pi(x)$.
		Was hat Tschebycheff dazu gezeigt? Wann war das?
		Welches Vermutung hat Tschebycheff noch bewiesen?
		Welchen Ansatz verwendet man für den Beweis?
\end{itemize}

\section{Einführung in die algebraische Zahlentheorie}

\begin{itemize}
	\item
		Was ist ein Dedekindring / Wann ist ein Primideal maximal?
		Welche Primideale hat $\Z$?
	\item
		Was besagt der Dirichletsche Einheitensatz?
		Verdeutliche an einem Beispiel.
	\item
		Was beschreibt $\mu(L)$?
		Zusammenhang zu Kapitel 4?
\end{itemize}


\end{document}
