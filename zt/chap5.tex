\label{chap:5}
\chapter{Ganzheit und endlich erzeugte Moduln}

\coursetimestamp{05}{06}{2014}

Sei $R$ ein kommutativer Ring.
„Ein $R$-Modul ist ein Vektorraum, dessen Skalarbereich ein Ring ist“.
Genauer nennt man eine abelsche Gruppe $(M, +)$ $R$-Modul, wenn die Skalarmultiplikation
\begin{align*}
	R \times M &\to M \\
	(r,m) &\mapsto rm
\end{align*}
folgende Eigenschaften erfüllt
\begin{enumerate}[M1)]
	\item
		$1\cdot m = m$ für alle $m \in M$,
	\item
		$r_1 (r_2 m) = (r_1 r_2) m$ für alle $r_1, r_2 \in R, m\in M$,
	\item
		$r(m_ + m_2) = rm_1 + rm_2$ für alle $r \in R, m_1, m_2 \in M$,
	\item
		$(r_1 + r_2)m = r_1 m + r_2 m$ für alle $r_1, r_2 \in R, m \in M$.
\end{enumerate}
für alle $m, m_1, m_2 \in M$
\begin{note}
	\begin{itemize}
		\item
			Ist $R$ nicht kommutativ, dann nennt man $(M, +)$ mit obigen Eigenschaften einen $R$-Linksmodul.
			Ersetzt man M1) durch $r_1(r_2m) = (r_2r_1) m$, dann spricht man von einem $R$-Rechtsmodul.
		\item
			Ein $R$-Modul $F$ heißt \emphdef{frei}, wenn er eine Basis $B$ hat, d.h. $B \subset F$ und jedes $m \in F$ lässt sich eindeutig als $R$-Linksmodul aus Elementen von $B$ schreiben: $m = \sum_{i=1}^k r_i b_i$ mit $b_i \in B$.
		\item
			Ein $R$-Modul $M$ heißt \emphdef{endlich erzeugt}, wenn es eine endliche Teilmenge $S$ von $M$ gibt, so dass sich jedes $m \in M$ als $R$-Linearkombination aus $S$ schreiben lässt.
		\item
			Eine Untergruppe $(T, +)$ des $R$-Moduls $M$ nennt man \emphdef{Teilmodul}, wenn für alle $r \in R, m \in T$ gilt $rm \in T$.
			Man schreibt dann $T \le M$.
		\item
			Ist $T \le M$, dann kann man die abelsche Faktorgruppe $M / T$ bilden.
			$M / T$ wird auf natürlicher Weise zu einem $R$-Modul durch $r(mT) := (rm)T$.
			$M / T$ heißt \emphdef{Faktormodul}, oder auch \emphdef{Quotientenmodul}.
		\item
			Seien $M, N$ $R$-Moduln und $\phi: M \to N$ ein Gruppenhomomorphismus.
			Dann nennt man $\phi$ $R$-linear, oder auch \emphdef{$R$-Modulhomomorphismus}, wenn
			\[
				\phi(rm) = r\phi(m).
			\]
	\end{itemize}
\end{note}

\begin{ex*}
	\begin{itemize}
		\item
			Jeder Vektorraum über einem Körper $K$ ist ein $K$-Modul.
		\item
			Jede abelsche Gruppe $A$ wird in natürlicher Weise zu einem $\Z$-Modul.
			Sei hierzu $z \in \Z, a \in A$ und setze
			\[
				za := \eps(\underbrace{a + \dotsc + a}_{|z|\text{-mal}})
			\]
			mit $\eps := \sign(z)$.
			Die Begriffe der abelschen Gruppe und des $\Z$-Moduls stimmen überein.
		\item
			Sei $R$ ein Ring.
			Dann wird $R$ zu einem $R$-Modul, indem man die Ringmultiplikation als Skalarmultiplikation nimmt.
			Dieser $R$-Modul ist frei mit Basis $\{1\}$.

			Etwas allgemeiner ist $R^n = \bigtimes_{i=1}^n R$ mit
			\[
				r(a_1, \dotsc, a_n) = (ra_1, \dotsc, ra_n)
			\]
			ist ein freier $R$-Modul mit Basis $\{ (1, 0, \dotsc, 0), (0, 1, 0, \dotsc, 0), \dotsc, (0, \dotsc, 0, 1) \}$.
		\item
			Ist $R$ ein Ring und $r \in R$, dann bezeichnet $\Z[r]$ die Menge aller ganzzahligen Linearkombinationen, also
			\[
				\Big\{ \sum_{i=0}^m z_i r^i : m\in \N_0, z_i \in \Z \Big\},
			\]
			wobei $z_i r^i$ die $\Z$-Operation von $\Z$ auf $(R, +)$ ist.
		\item
			Ist $R$ ein Ring, $S \le R$ ein Teilring und $r \in R$.
			Dann bezeichnet $S[r]$ die Menge aller $S$-Linearkombinationen der Form $\sum_{i=0}^m s_i r^i$ mit $m \in \N_0, s_i \in S$.
			Ist $\Z$ ein Teilring von $R$, dann ist $\Z[r] = S[r]$.

			Ist $R$ kommutativ, dann ist $S[r]$ der kleinste Teilring, der $S$ und $r$ enthält.
	\end{itemize}
\end{ex*}

% Lem 5.1
\begin{lem} \label{5.1}
	Sei $R$ ein Ring, $S \le R$ ein Teilring und $r \in R$.
	\begin{enumerate}[a)]
		\item
			$r$ ist genau dann ganz, wenn $\Z[r]$ als $\Z$-Modul endlich erzeugt ist.
		\item
			$r$ ist genau dann ganz über $S$, wenn $S[r]$ als $S$-Modul endlich erzeugt ist.
	\end{enumerate}
	\begin{proof}
		\begin{enumerate}[a)]
			\item
				Ist $r$ ganz, dann erfüllt $r$ eine normierte Gleichung der Form
				\[
					r^n + a_1 r^{n-1} + \dotsc + a_{n-1} r + a_n = 0
				\]
				mit $a_i \in \Z$.
				Es folgt, dass $\Z[r] = \<1, r, \dotsc, r^{n-1}\>_\Z$, also endlich erzeugt.

				Sei umgekehrt
				\[
					\Z[r] = \< f_1, \dotsc, f_m \>_\Z.
				\]
				mit Polynomen $f_i$ in $r = \sum_{j=0}^m a_{ij} r^j$.
				Jedes $f_i$ hat endlichen Grad.
				Wähle $n > \max \{ \deg f_i \}$.
				Dann ist
				\[
					r^n = b_1 f_1 + \dotsb + b_m f_m
				\]
				mit $b_i \in \Z$, da $\Z[r]$ von den $f_i$ erzeugt ist.
				Es folgt
				\[
					r^n - b_1 f_1 - \dotsb - b_m f_m = 0,
				\]
				also erfüllt $r$ ein normiertes Polynom mit Koeffizienten in $\Z$ und $r$ ist ganz.
			\item
				Verläuft analog.
		\end{enumerate}
	\end{proof}
\end{lem}

\begin{st} \label{5.2}
	Sei $R$ ein Ring.
	\begin{enumerate}[a)]
		\item
			Seien $r, t \in R$ ganz und $rt = tr$, dann sind $r - t$ und $rt$ ganz.
		\item
			Ist $R$ kommutativ, dann bilden die ganzen Elemente von $R$ einen Teilring, den Ring der ganzen Zahlen von $R$.
	\end{enumerate}
	\begin{proof}
		\begin{enumerate}[a)]
			\item
				$r, t$ sind ganz, also $\Z[r] = \<1, r, \dotsc, r^{m-1}\>_\Z$ und $\Z[t] = \<1, t, \dotsc, t^{m-1}\>_\Z$.
				Dann ist
				\begin{align*}
					\Z[r, t] &= \Big\{ \sum_{i,j} a_{ij} a^i r^i t^j : a_{ij} \in \Z \Big\} \\
					&= \text{kleinster Teilring von $R$, der $\Z[r]$ und $\Z[T]$ enthält} \\
					&= \<1, r, \dotsc, r^{m-1}, t, \dotsc, t^{n-1}, rt, \dotsc, r^{m-1}t, \dotsc, r^{m-1}t^{n-1}\>_\Z.
				\end{align*}
				ein endlich erzeugter $Z$-Modul.
				$\Z[rt]$ und $\Z[r-t]$ sind Teilmoduln von $\Z[r,t]$.
				Mit \ref{5.3} sind $\Z[rt]$ und $\Z[r-t]$ endlich erzeugte $\Z$-Moduln und nach \ref{5.1} a) sind $rt$ und $r-t$ ganz.
			\item
				Folgt unmittelbar aus a), da wegen $R$ kommutativ dann beliebige Produkte, bzw. Differenzen von ganzen Elementen wieder ganz sind.
		\end{enumerate}
	\end{proof}
\end{st}

% St 5.3
\begin{st} \label{5.3}
	Sei $M$ endlich erzeugter $\Z$-Modul, $T \le M$ ein Teilmodul von $M$.
	Dann ist $T$ endlich erzeugt.
	\begin{note}
		Man kann \ref{5.3} aus dem Hauptsatz über endlich erzeugte abelsche Gruppen herleiten ($A \isomorphic \Z^m \times \Z / a_1 \Z \times \dotsb \times \Z / a_k \Z$ mit $a_1 \divs a_2 \divs a_3 \dotsc$).
		Später wird in \ref{chap:5} für \ref{5.3} ein separater Beweis erfolgen.

		Zunächst wird ein zu \ref{5.2} b) verwandtes Resultat bewiesen.
	\end{note}
\end{st}

\begin{st} \label{5.4}
	Sei $R$ kommutativ und $S \le R$ Teilring.
	Endlich viele Elemente $r_1, \dotsc, r_n \in R$ sind genau dann ganz über $S$, wenn der $S$-Modul
	\[
		S[r_1, \dotsc, r_n] = \text{kleinster Teilring von $R$, der $S[r_i]$ für $1 \le i \le n$ enthält}
	\]
	endlich erzeugt ist.
	\begin{proof}
		Seien $r_1, \dotsc, r_n$ ganz über $S$.
		Nach \ref{5.1} b) ist $S[r_1]$ ein endlich erzeugter $S$-Modul.
		$r_2$ ist ganz über $S$, also wegen $S \subset S[r_1]$ ist $r_2$ ganz über $S[r_1]$.
		Mit \ref{5.1} b) ist $S[r_2]$ ein endlich erzeugter $S[r_1]$-Modul.
		Damit ist $S[r_1][r_2] = S[r_1, r_2]$ ein endlich erzeugter $S$-Modul (siehe Bemerkung).
		Man erkennt, dass sich diese Richtung leicht per Induktion zeigen lässt.

		Umgekehrt sei $S[r_1, \dotsc, r_n]$ ein endlich erzeugter $S$-Modul mit Erzeugendensystem $\{m_1, \dotsc, m_k\}$
		Sei $r \in S[r_1, \dotsc, r_n]$, dann ist für $1 \le i \le k$
		\[
			rm_i = \sum_{j= 1}^k \alpha_{ij} m_j
		\]
		mit $\alpha_{ij} \in S$.
		In Matrixschreibweise ist
		\[
			r E \Vector{m_1 & \dots & m_k}
			= A \Vector{m_1 & \dots & m_k}
		\]
		mit $A = (a_{ij})$, bzw. mit $m = (m_1, \dotsc, m_k), (rE - A)m = 0$.
		In der linearen Algebrau zeigt man, dass für $A \in R^{k\times k}$ mit kommutativem Ring $R$ gilt: $A^* A = \det A E$ mit adjungierter Matrix $A^*$.
		Für $\phi = (x_1, \dotsc, x_k)$ gilt dann $A \phi = 0 \implies \det A \cdot \phi = 0$.
		Angewandt auf $(rE - A)m = 0$ folgt, dass $\det (aE - A) m_i = 0$.
		Da $1 = \sum_{i=1}^k c_i m_i$ mit $c_i \in S$, folgt $\det(rE - A) 1 = 0$, also $\det(rE - A) = 0$.
		Die Determinante stellt eine normierte Gleichung für $r$ dar mit Koeffizienten aus $S$, also ist $r$ ganz.
	\end{proof}
	\begin{note}
		Ist $A$ ein Teilring von $B$ und $M$ ein endlich erzeugter $B$-Modul und $B$ als $A$-Modul endlich erzeugt.
		Dann ist $M$ ein endlich erzeugter $A$-Modul (durch Einschränkung).
		\begin{proof}
			Für $m \in M$, $M = \<b_1, \dotsc, b_k\>_B, B = \<\beta_1, \dotsc, \beta_l\>_A$ ist
			\[
				m = \sum_{i=1}^k b_i m_i
				= \sum_{i=1}^k \Big( \sum_{j=1}^l a_{ji} \beta_j \Big) m_i
				= \sum_{i,j} a_{ji} \underbrace{\beta_j m_i}_{\in M},
			\]
			also $M = \< \{ \beta_j m_i : 1 \le j \le l, 1 \le i \le k \} \>$.
		\end{proof}
	\end{note}
\end{st}

% Folg 5.5
\begin{kor} \label{5.5}
	Sei $R$ ein kommutativer Ring, $S \le R$ ein Teilring.
	Dann bilden die ganzen Zahlen von $R$ über $S$ einen Teilring $\Gamma_R(S)$.
	\begin{proof}
		Folgt aus $S[r_1, \dotsc, r_n] = S[r_1, \dotsc, r_n, r]$ mit $r \in S[r_1, \dotsc, r_n]$ und \ref{5.4}.
	\end{proof}
	\begin{note}
		Insbesondere bilden die algebraischen Zahlen $\A$, also die ganzen Zahlen $\C$ über $\Q$, einen Teilring von $\C$.
		Genauso bilden die ganzen algebraischen Zahlen, also die ganzen Zahlen von $\C$ über $\Z$ einen Teilring von $\C$.

		Der Unterschied zu \ref{5.2} ist nur, dass $\Z$ hier ein Teilring ist.

		$\A$ ist abzählbar.
	\end{note}
\end{kor}

\coursetimestamp{16}{06}{2014}

% Def 5.6
\begin{df} \label{5.6}
	Sei $R$ ein kommutativer Ring, $S \le R$ ein Teilring.
	\begin{enumerate}[a)]
		\item
			$\Gamma_R(S)$ heißt \emphdef{ganzer Abschluss} von $S$ in $R$.
		\item
			$S$ heißt \emphdef{ganz abgeschlossen} in $R$, wenn $\Gamma_R(S) = S$ ist.
		\item
			$R$ heißt \emphdef{ganz über $S$}, wenn alle Elemente von $R$ ganz über $S$ sind.
	\end{enumerate}
\end{df}

% Bem 5.7
\begin{nt} \label{5.7}
	Sei $R_1$ ein kommutativer Ring, $R_1$ ganz über $R_2$ und $R_3$ ganz über $R_3$.
	Dann ist auch $R_1$ ganz über $R_3$.
	\begin{proof}
		Sei $c \in R_1$.
		Da $R_1$ über $R_2$ ganz ist, erfüllt $c$ eine Gleichung der Form
		\[
			c^n + b_1 c^{n-1} + \dotsb + b_n = 0,
		\]
		wobei $b_i \in R_2$.
		Setze $\tilde R := R_3[b_1, \dotsc, b_n] \subset R_2$.
		$c$ ist offensichtlich ganz über $\tilde R$.
		Nach \ref{5.1} b) ist $\tilde R[c]$ ein endlich erzeugter $\tilde R$-Modul.
		Da $\tilde R$ endlich erzeugt über $R_3$ ist, ist $\tilde R[c]$ ebenfalls ein endlich erzeugter $R_3$-Modul.
		Mit \ref{5.1} b) ist also $c$ ganz über $R_3$.
	\end{proof}
\end{nt}

\begin{lem} \label{5.8}
	Ist $R$ faktoriell und $K = \Quot(R)$.
	Dann ist $R$ in $K$ ganz abgeschlossen.
	\begin{proof}
		Sei $\f ab \in K$ ganz über $R$ mit $a,b \in R$.
		Dann erfüllt $\f ab$ eine Gleichung der Form
		\[
			(\f ab)^n + \alpha_1 (\f ab)^{n-1} + \dotsb + \alpha_n = 0,
		\]
		wobei $\alpha_i \in R$.
		Multipliziert man mit $b^n$, erhält man
		\[
			a^n + \alpha_1 b a^{n-1} + \dotsb + \alpha_n b^n = 0,
		\]
		d.h. $a^n = bx$ mit einem $x \in R$.
		Jedes Primelement, das $b \in R$ teilt, teilt also auch $a$.
		Mit Kürzung folgt dann, dass $b$ eine Einheit von $R$, d.h. $\f ab \in R$.
	\end{proof}
\end{lem}

% Bem 5.9
\begin{nt} \label{5.9}
	\begin{enumerate}[a)]
		\item
			Sei $L$ endlichdimensionaler Erweiterungskörper des Körpers $K$, d.h. $\dim_K L < \infty$.
			Sei $A$ ganz abgeschlossen in $K$ und sei $B$ der ganze Abschluss von $A$ in $L$, d.h. $B = \Gamma_L(A)$.

			Dann gilt $\Gamma_L(B) = B$.
		\item
			Jedes Element $\beta \in L$ hat die Gestalt $\beta = \f ba$ mit $b \in B, a \in A$.			
	\end{enumerate}
	\begin{proof}
		\begin{enumerate}[a)]
			\item
				$\Gamma_L(B)$ ist ganz über $B$ und $B$ ist ganz über $A$, somit ist nach \ref{5.7} $\Gamma_L(B)$ ganz über $A$.
				Also ist $\Gamma_L(B) \subset \Gamma_L(A) = B \subset \Gamma_L(B)$ und somit $\Gamma_L(B) = B$.
			\item
				Es ist $\dim_K L < \infty$ und $\beta \in L$, also existieren $c_i \in K$ mit
				\[
					\beta^n + c_1 \beta^{n-1} + \dotsb + c_n = 0.
				\]
				Durchmultiplizieren mit dem Hauptnenner ergibt eine Form
				\[
					a_n \beta^n + a_{n-1} \beta^{n-1} + \dotsb + a_0 = 0
				\]
				mit $a_i \in A$.
				Multipliziert man mit $a_n^{n-1}$, ergibt sich
				\begin{align*}
					a_n^n \beta^n + \dotsb a_i a_n^{n-1} \beta^i + \dotsb + a_n^{n-1} a_0 &= 0 \\
					\iff \quad
					(a_n \beta)^n + \dotsb \tilde a_i (a_n \beta)^i + \dotsb + \tilde a_n a_0 &= 0.
				\end{align*}
				$a_n \beta$ ist also ganz über $A$, $a_n \beta \in B$ und somit $\beta = \f b{a_n}$ mit $b \in B$.
		\end{enumerate}
	\end{proof}
\end{nt}

Die Situation ist anwendbar auf die klassische Zahlkörpersituation: $A = \Z, K = \Q$, $L$ ein algebraischer Zahlkörper und $B$ der ganze Abschluss von $\Z$ in $L$ (in der Literatur oft auch als $O_B$ bezeichnet).

% St 5.10
\begin{st} \label{5.10}
	Sei $R$ ein noetherscher Ring und $M$ ein endlich erzeugter $R$-Modul.
	Dann ist jeder Teilmodul von $M$ endlich erzeugt.
	\begin{proof}
		Sei $M = \<m_1, m_2, \dotsc, m_k\>_R$, wir zeigen die Aussage induktiv.
		\begin{seg}{$k = 1$}
			Sei $T$ ein Teilmodul von $M$.
			Setze $I_T = \Set{ r \in R & rm_1 \in T }$.
			Dann sieht man leicht, dass $I_T$ ein Ideal von $R$ ist.
			Für Teilmoduln $T_1 \subset T_2$ von $M$ folgt $I_{T_1} \subset I_{T_2}$.
			Ist $T_0 \subset T_1 \subset \dotsc$ eine austeigende Kette von Teilmoduln, dann ist $I_{T_0} \subset I_{T_1} \subset \dotsc$ eine aufsteigende Kette von Idealen, die abbricht, da $R$ noethersch.

			Ist $I_{T_1} = I_{T_2}$, dann gilt auch $T_1 = T_2$, denn jedes Element von $M$ ist von der Form $r m_1$ mit $m_1 \in R$, da $M = \< m_1 \>_R$ zyklisch ist.
			Also bricht die Kette $T_0 \subset T_1 \subset T_2 \subset \dotsc$ ab und nach \ref{5.11} ist jeder Teilmodul von $M$ endlich erzeugt.
		\end{seg}
		\begin{seg}{$k > 1$}
			Sei $U$ ein Teilmodul von $M$.
			Setze
			\[
				I_k(U) := \Set{ r \in R & \exists u \in U : u = \Big( \sum_{i=1}^{k-1} \alpha_i m_i \Big) + rm_k }.
			\]
			Dann verifiziert man leicht, dass $I_k(U)$ ein Ideal von $R$ ist.
			Sind $U_i, U_{i+1} \le M$ Teilmoduln mit $U_i \subset U_{i+1}$, dann auch $I_k(U_i) \le I_k(U_{i+1})$.
			Zu einer gegebenen Kette $U_0 \le U_1 \dotsb \le U_{i} \le U_{i+1} \le \dotsb$ von Teilmodul von $U$ gibt es ein Index $i_0$ mit $I_k(U_{i_0}) = I_k(U_m)$ für alle $m \ge i_0$, da $R$ noethersch.

			Wir behaupten
			\[
				U_m = \< U_{i_0}, U_m \cap \<m_1, \dotsc, m_{k-1}\>_R \>_R,
			\]
			wobei $U_m \cap \<m_1, \dotsc, m_{k-1}\>$ Teilmodul von $\<m_1, \dotsc, m_{k-1}\>_R$ ist und nach Induktion endlich erzeugt ist.

			Zeige nun die Behauptung.
			„$\supset$“ ist trivial.
			Sei $u = \sum_{i=1}^{k-1} \alpha_i m_i + r m_k \in U_m$, $\alpha_i, r \in R$.
			Dann existierte $\tilde u \in U_{i_0}$ mitm
			\[
				\tilde n = \sum_{i=1}^{k-1} \beta_i m_i + r m_k
			\]
			wegen $I_k(U_{i_0}) = I_k(U_m), \beta_i \in R$.
			Also ist wegen $m \ge i_0$
			\[
				u - \tilde u = \sum_{i=1}^{k-1} (\alpha_i - \beta_i) m_i  \in U_m \cap \< m_1, \dotsc, m_{k-1} \>.
			\]
			und somit gilt auch „$\subset$“.

			Setze nun $\tilde U_m = U_m \cap \<m_1, \dotsc, m_{k-1}$.
			Dann gilt sicherlich $\tilde U_{i_0} \le \tilde U_{i_0 + 1} \le \dotsb \le \tilde U_m \le \tilde U_{m+1} \le \dotsc $.
			Dies ist eine aufsteigende Kette von Teilmoduln von $\<m_1, \dotsc, m_{k-1}\>$.
			Nach Induktion sind alle Teilmoduln von $\<m_1, \dotsc, m_{k-1}\>$ endlich erzeugt.
			Mit \ref{5.11} existiert ein Index $j_0$ mit $\tilde U_{j_0} = \tilde U_m$ für alle $m \ge j_0$.
			Mit der Behauptung folgt $U_m = \<U_{i_0}, \tilde U_{j_0}\> = \<U_{i_0}, U_{j_0}\> = U_{j_0}$ für alle $m \ge j_0$.
			Die Ausgangskette bricht also ab und mit \ref{5.11} ist der Teilmodul $U$ endlich erzeugt.
		\end{seg}
	\end{proof}
	\begin{note}
		Aus \ref{5.10} folgt unmittelbar \ref{5.3}, denn $\Z$ ist ein noetherscher Ring.
		Damit ist der Beweis von \ref{5.2} a) und b) abgeschlossen.

		Insbesondere bilden die ganzen Elemente eines kommutativen Ringes $R$ einen Teilring von $R$ unabhängig davon, ob $\Z$ ein Teilring von $R$ ist.

		Die Folgerung \ref{5.5} kann man für $\Z$ nur anwenden, wenn $\Z$ ein Teilring ist.
	\end{note}
\end{st}

% Lem 5.11
\begin{lem} \label{5.11}
	Sei $M$ ein $R$-Modul und $R$ ein Ring.
	Dann ist jeder Teilmodul von $M$ endlich erzeugt genau dann, wenn jede aufsteigende Kette von Teilmoduln bricht ab.
	\begin{proof}
		Verläuft analog zum Beweis von \ref{4.6}.
		Andererseits ist \ref{4.6} ein Spezialfall von \ref{5.11}, denn betrachtet man $R$ über sich selbst als Modul (durch Linksmultiplikation), dann sind die (Links-)Ideale genau die Teilmoduln.
	\end{proof}
\end{lem}

% Def 5.12
\begin{st} \label{5.12}
	Sei $A$ ganz abgeschlossen in $K = \Quot(A)$ und $L$ eine Körpererweiterung von $K$ mit $\dim_K L < \infty$.
	Sei $B$ der ganze Abschluss von $A$ in $L$ (nach \ref{5.9} ist $B$ also ganz abgeschlossen in $L$).

	Eine Teilmenge $S = \Set{ w_1, \dotsc, w_n } \subset B$ heißt \emphdef{Ganzheitsbasis} von $B$ über $A$, wenn $B$ ein freier $A$-Modul mit Basis $S$ ist.
\end{st}

Gibt es Ganzheitsbasen?

\begin{nt} \label{5.13}
	\begin{enumerate}[a)]
		\item
			Sei $S$ eine Ganzheitsbasis von $B$ über $A$.
			Dann ist $S$ auch eine $K$-Basis von $L$, d.h. $n = \dim_K L$.
			\begin{proof}
				\ref{5.9} ist beim Beweis nützlich. \Exercise
			\end{proof}
		\item
			Es gibt nicht immer Ganzheitsbasen.
			In der Situation der Zahlkörpern $A = \Z, K = \Q$ gilt:
			Der Ring der ganzen Zahlen in einem algebraischen Zahlkörper besitzt Ganzheitsbasen.
		\item
			Sei $B$ über $A$ ein endlich erzeugter $A$-Modul (schwächer als Existenz von Ganzheitsbasen, aber die Existenz von Ganzheitsbasen impliziert, dass $B$ endlich erzeugter $A$-Modul ist, also gilt dies in der Zahlkörpersituation).
			Dann ist jedes Ideal von $B$ als $A$-Modul endlich erzeugt, wenn $A$ noethersch ist (nach \ref{5.10}).
			Dies gilt insbesondere für $A = \Z$.
			Ein Ring $R$ ganzer algebraischer Zahlen in einem algebraischen Zahlkörper ist also stets noethersch.
			Die Eigenschaft noethersch „vererbt“ sich auf $B$.

			Beachte: $B$ ist nicht immer ein Hauptidealbereich.
		\item
			Für $R = \Z$ gelten stärkere Aussagen als in \ref{5.10}, beispielsweise:

			Ist $U \le F$ mit einem freien $\Z$-Modul $F$ von endlichem Rang $n$, dann ist auch $U$ frei von Rang $\le n$.
	\end{enumerate}

\end{nt}
