\chapter{Integritätsbereiche, irreduzible Elemente, Primzahlen}


\begin{df*}
	Ein \emphdef{Ring} $R = (R, +, \cdot)$ ist eine Menge mit zwei Verknüpfungen $+: R \times R \to R$, $\cdot: R \times R \to R$, so dass $(R, +)$ eine abelsche Gruppe ist, $(R, \cdot)$ ein Monoid ist und die Distributivgesetze
	\begin{align*}
		(r+s)t &= rt + st \\
		r(s+t) &= rs + rt
	\end{align*}
	für alle $r, s, t \in R$ gelten.
	Mit $0$ und $1$ bezeichnet man die neutralen Elemente bezüglich $+$, bzw. $\cdot$.

	$R$ nennt man \emphdef[Ring!kommutativ]{kommutativ}, wenn $rs = sr$ für alle $r,s \in R$.
	\begin{note}
		Nach unserer Definition hat jeder Ring ein Einselement.
	\end{note}
\end{df*}

\begin{ex*}
	\begin{itemize}
		\item
			$\Z$,
		\item
			$\Z / n \Z$,
		\item
			$(K)^{n\times m}$ mit $K$ Körper oder Ring,
		\item
			Körper
			\begin{itemize}
				\item
					$\Q$,
				\item
					$\R$,
				\item
					$\C$,
				\item
					$\Z / p \Z$ mit Primzahl $p$.
			\end{itemize}
		\item
			$K[x]$, der Polynomring in einer Variablen mit Koeffizienten in $K$,
		\item
			$(K)^{n\times m}$ mit Körper $K$ ist isomorph zu $\End_K(V)$, wobei $V$ ein endlichdimensionaler $K$-Vektorraum,
		\item
			$\H, (K)^{n\times n}$ für $n \ge 2$ sind nicht kommutativ,
		\item
			$\Z[i] = \{ z_1 + z_2 i : z_1, z_2 \in \Z \}$.
	\end{itemize}
\end{ex*}

\begin{df*}
	Eine abelsche Untergruppe $(I, +)$ eines Ringes $(R, +, \cdot)$ nennt man \emphdef{Linksideal}, wenn
	\[
		\forall r \in R, m \in I : rm \in I
	\]
	und \emphdef[Ideal!Rechtsideal]{Rechtsideal}, wenn
	\[
		\forall r \in R, m \in I : m r \in I
	\]
	und \emphdef[Ideal!zweiseitiges Ideal]{zweiseitiges Ideal}, wenn es zugleich Links- und Rechtsideal ist.
	In kommutativen Ringen spricht man meistens nur von \emphdef[Ideal]{Idealen}
\end{df*}

\begin{ex*}
	\begin{itemize}
		\item
			In $\Z$ sind für festes $n \in \N$
			\[
				\Z n = \{ z \in \Z : z = x n \text{ für ein $x \in \Z$} \}
				= \{ z n : z \in \Z \}
				= n \Z
			\]
			Ideale.
		\item
			In jedem Ring $R$ ist für festes $s \in R$
			\begin{align*}
				Rs &= \{ r s : r \in R \}, &
				sR &= \{ sr : r \in R \}
			\end{align*}
			ein Links- bzw. Rechtsideal.
			Solche Ideale nennt man \emphdef[Hauptideal]{Hauptideale}.
	\end{itemize}
\end{ex*}

% Def 1.1
\begin{df} \label{1.1}
	Sei $R$ ein Ring und $R^* := R \setminus \{0\}$.
	\begin{enumerate}[a)]
		\item
			$a \in R^*$ nennt man \emphdef{Nullteiler}, wenn es ein $b \in R^*$ gibt mit $ab = 0$ oder $ba = 0$.
		\item
			$\unit u \in R^*$ nennt man eine \emphdef{Einheit}, wenn es $v, w \in R^*$ gibt mit $v \unit u = 1$ und $\unit u w = 1$
		\item
			$R$ nennt man \emphdef{Integritätsbereich}, wenn $R$ kommutativ ist und keine Nullteiler besitzt.
		\item
			$R$ heißt \emphdef{Hauptidealring}, wenn jedes Ideal von $R$ ein Hauptideal ist.
		\item
			$R$ heißt \emphdef{Hauptidealbereich}, wenn $R$ Integritätsbereich und Hauptidealring ist.
	\end{enumerate}
	\begin{note}
		Die Menge aller Einheiten bildet mit der Multiplikation eine Gruppe, die sogenannte \emphdef{Einheitengruppe} von $R$, geschrieben $U(R)$.
	\end{note}
\end{df}

% Bsp 1.2
\begin{ex} \label{1.2}
	\begin{enumerate}[a)]
		\item
			Körper $K$ (z.B. $\Q, \R, \C$) sind Integritätsbereiche und Hauptidealbereiche, $U(K) = K^* = K \setminus \{0\}$ (bei Körper verlangt man $0 \neq 1$, d.h. Körper haben mindestens zwei Elemente).
		\item
			$\Z$ ist Integritätsbereich und Hauptidealbereich $U(\Z) = \{ \pm 1 \} \isomorphic C_2$
		\item
			$\Z[x]$ ist Integritätsbereich, aber kein Hauptidealbereich.
			Wenn $K$ ein Körper ist, dann ist $K[x]$ ein Hauptidealbereich.
		\item
			$\Z / 4 \Z$ ist kein Integritätsbereich ($\_ 2 \cdot \_ 2 = 0$)
		\item
			$K^{2+2}$ hat Nullteiler, z.B. $\Matrix{0 & 1 \\ 0 & 0 } \Matrix{0 & 1 \\ 0 & 0} = 0$,
		\item
			Wie bereichnet man Inverse in $\Z / n \Z$?

			Mit Hilfe des Euklidischen Algorthmus (Division mit Rest) kann man zu gegebenen Zahlen $a,b \in R$ Zahlen $z_1, z_2 \in R$ berechnen mit $z_1 a + z_2 b = \ggT(a,b)$ (Lemma von Bezout).

			Sei $n \in \N$ vorgegeben und $a \in \N$ mit $\ggT(a, n) = 1$ vorgegeben.
			Nach Bezout existieren $z_1, z_2 \in \Z$ mit $z_1 a + z_2 n = 1$.
			Modulo $n$ ergibt sich
			\[
				\_ z_1 \cdot \_ a + \underbrace{\_ z_2 \_ n}_{=\_ 0} = \_ 1
			\]
			also ist $\_z_1 \_ a = \_ 1$ und $\_ z_1$ Invers zu $\_ a$.

			Beispiel für $n = 31, a = 7$:
			\begin{align*}
				31 &= 4 \cdot 7 + 3 \\
				7 &= 2 \cdot 3 + 1
			\end{align*}
			und
			\begin{align*}
				2 \cdot 31 &= 2 \cdot 4 \cdot 7 + 2 \cdot 3 = 2 \cdot 4 \cdot 7 + 7 - 1 \\
				2 \cdot 31 &= 9 \cdot 7 - 1 \\
				1 &= 9 \cdot 7 - 2 \cdot 31
			\end{align*}
			also ist $\_ 9$ invers zu $\_ 7$.
	\end{enumerate}
\end{ex}

% Def 1.3
\begin{df} \label{1.3}
	Sei $R$ ein kommutativer Ring und $r, s \in R$.
	\begin{enumerate}[a)]
		\item
			$r$ heißt \emphdef{Teiler} von $s$, wenn $q \in R$ existiert mit $s = r q$, wir schreiben dann $r \divs  s$.
		\item
			$r$ heißt \emphdef[Teiler!echt]{echter Teiler} von $s$, wenn $r\divs s$, $s \ndivs  r$ und $r \not\in U(R)$, wir schreiben dann $r \divs*  s$.
		\item
			$r$ heißt \emphdef{irreduzibel}, wenn $r \neq 0$, $r \not\in U(R)$ und $r$ besitzt keine echten Teiler.
		\item
			$r$ heißt \emphdef{prim}, wenn $r \neq 0$, $r \not\in U(R)$ und
			\[
				\forall a,b \in R : r \divs  ab \implies r \divs  a \lor r \divs  b.
			\]
	\end{enumerate}
	\begin{note}
		$1$ ist weder irreduzibel, noch prim.
	\end{note}
\end{df}

% Lem 1.4
\begin{lem} \label{1.4}
	Für einen Integritätsbereich $R$ gilt in $(R^*, \cdot)$ die Kürzungsregel.
	\begin{proof}
		Aus $r s = r t$ folgt $r(s-t) = 0$ und da $r \in R^*$ und $R$ Integritätsbereich folgt $s-t = 0$, also $s = t$.
	\end{proof}
\end{lem}

% Lem 1.5
\begin{lem} \label{1.5}
	Sei $R$ ein Integritätsbereich. Es gilt
	\begin{enumerate}[a)]
		\item
			Sei $r, s \in R^*$.
			Dann gilt:
			\[
				r\divs s  \land  s\divs r
				\implies
				\exists \unit u \in U(R) : s = \unit u r
			\]
		\item
			Ist $r$ prim, so auch irreduzibel
		\item
			Ist $p$ prim und $\unit u \in U(R)$, dann ist auch $\unit u p$ prim.
		\item
			Sind $p$ und $q$ prim und $p \divs  q$, dann gilt $p = \unit u q$ für ein $\unit u \in U(R)$.
	\end{enumerate}
	\begin{proof}
		\begin{enumerate}[a)]
			\item
				Sei $s = rq$ und $r = s\tilde q$, dann ist $s = s \tilde q q$ und mit \ref{1.4} $1 = \tilde q q$ und damit $\tilde q, q \in U(R)$.
			\item
				Sei $r$ prim aber nicht irreduzibel.
				Es gilt dann $r \in R^*$, $r = st$ und $s \divs*  r$.
				Da $r$ prim, folgt $r \divs  s$ oder $r \divs  t$.
				$r \divs  s$ ist wegen $s \divs*  r$ unmöglich, also $r \divs  t$.
				Schreibe $t = \tilde t r$, also $r = s \tilde t r$ und mit \ref{1.4} $1 = s \tilde t$.
				Also ist $s$ eine Einheit, ein Widerspruch zu $s \divs*  r$.
		\end{enumerate}
	\end{proof}
\end{lem}

% Satz 1.6
\begin{st}[Eindeutigkeit der Zerlegung in Primelemente] \label{1.6}
	Sei $R$ ein Integritätsbereich und $r \in R$.
	Sei $r = p_1 \dotsc p_m$ und $r = q_1 \dotsc, p_n$ mit $p_1, \dotsc, p_m, q_1, \dotsc, q_n$ prim.
	Dann gilt $m = n$ und nach geeigneter Nummerierung gilt $p_i = \unit u_i \cdot q_i$ mit $\unit u_1 \in U(R)$.
	\begin{proof}
		Da $p_1$ prim, gilt $p_1 \divs  q_{i_0}$ für einen Index $i_0$.
		Sei \oBdA $i_0 = 1$.
		Nach \ref{1.5} d) gilt dann $q_1 = \unit u_1 p_1$ mit $\unit u_1 \in U(R)$.
		Also ist
		\[
			p_1 \dotsb p_m = \unit u_1 p_1 q_{\pi(2)} \dotsb q_{\pi(n)}.
		\]
		Wähle $\tilde q_i := q_{\pi(i)}$ für $i \ge 3$ und $\tilde q_2 = u_1 q_{\pi(2)}$.
		und nach der Kürzungsregel $p_2 \dotsb p_m = \tilde q_2 \dotsb \tilde q_n$ mit $p_i, \tilde q_j$ prim.
		Nach \ref{1.5} c) ist $\tilde q_2$ ebenfalls prim.

		Wäre $m = 1$, dann ist $1 = \unit u_1 q_{\pi(2)} q_{\pi(n)}$, woraus für $n \ge 2$ $q_{\pi(1)} \in U(R)$ folgt, ein Widerspruch.
		Also $m = 1 \iff n = 1$.

		Per Induktion ist $m - 1 = n - 1$ und $p_k = \tilde u_k \tilde q_k$ und $\tilde u_k \in U(R)$ für $k \ge 2$.
	\end{proof}
\end{st}

Gemäß unserer Definition ist eine Primzahl in $\Z$ ein irreduzibles Element von $\Z$.
Um zu einer eindeutigen Primfaktorzerlegung mittels \ref{1.6} zu gelangen, bräuchten wir, dass aus irreduzibel prim folgt.

\coursetimestamp{14}{04}{2014}

% Lem 1.7
\begin{lem} \label{1.7}
	Sei $R$ ein Hauptidealbereich, dann ist $p \in R$ genau dann irreduzibel, wenn $p$ prim ist.
	\begin{proof}
		\begin{segnb}{$\impliedby$}
			Gilt nach \ref{1.5} b).
		\end{segnb}
		\begin{segnb}{$\implies$}
			Sei $q \in R$ irreduzibel und $q$ teile $ab$ für $a, b \in R$.
			$Rq + Ra$ ist ein Ideal, also existiert da $R$ Hauptidealbereich, $d \in R$ mit $Rq + Ra = Rd$.
			Also gilt $a,q \in Rd$ und $q = sd$ für ein $s \in R$.
			Da $q$ irreduzibel, folgt $s \in U(R)$ oder $d \in U(R)$.
			Wir unterscheiden beide Fälle:
			\begin{segnb}{$s\in U(R)$}
				Es gilt $d = s^{-1}q$, d.h. $q \divs d$.
				Da auch $d \divs a$, gilt also $q\divs a$.
			\end{segnb}
			\begin{segnb}{$d\in U(R)$}
				Dann gilt $Rd = R$, also $1 = r_1 q + r_2 a$.
				$q$ teilt $ba$, also teilt $q$ auch $b r_2 a$.
				Da $b = b r_1 q + b r_2 a$ gilt somit $q \divs  b$.
			\end{segnb}
			Also folgt insgesamt, dass $q$ prim ist.
		\end{segnb}
	\end{proof}
\end{lem}

%Lem 1.8
\begin{lem} \label{1.8}
	Sei $R$ ein Integritätsbereich.
	Falls jede aufsteigende Kette (bezüglich der Inklusion) von Hauptidealen ein maximales Element besitzt (also abbricht).
	Dann lässt sich jedes $s \in R, s \neq 0, s \not \in U(R)$ in ein endliches Produkt von irreduziblen Elementen zerlegen.
	\begin{proof}
		$s$ sei nicht irreduzibel, dann besitzt $s$ einen echten Teiler: $s = a_1 b_1$ mit $a_1 \divs*  s$.

		Es gilt $s \in Ra_1$ und $Rs \subset Ra_1$.
		Angenommen $Rs = Ra_1$, dann ist $a_1 = s c$, also $s \divs  a_1$, ein Widerspruch zu $a_1 \divs*  s$.
		Also ist $Rs \subsetneq Ra_1$.

		% fixme: Klären: Hier wird verwendet, dass R Integritätsbereich ist (Kürzungsregel bei s), ist aber nicht vorausgesetzt!
		Ähnlich sieht man $b_1 \divs*  s$, denn wenn $b_1 \divs  s$ und $s \divs  b_1$, dann ist $s = a_1 b = a_1 x s$, also $a_1 x = 1$ und $a_1 \in U(R)$, ein Widerspruch zu $a_1$ irreduzibel.
		Also ist auch $Rs \subsetneq Rb_1$.

		Sind $a_1$, bzw. $b_1$ nicht irreduzibel, so kann man das Argument wiederholen.
		Induktiv ergibt sich dann eine aufsteigende Kette von Hauptidealen, welche nach Voraussetzung abbricht.
		Also muss $s$ ein endliches Produkt von irreduziblen Elementen sein.
	\end{proof}
\end{lem}

%Lem 1.9
\begin{lem} \label{1.9}
	$\Z$ ist ein Hauptdealbereich.
	\begin{proof}
		Sei $I$ ein Ideal von $\Z$, $I \neq 0, I \neq \Z$.
		Dann gibt es ein $d \in I$ mit $1 \le |d| \le |s|$ für alle $s \in I$.
		($|d| \ge 2$, denn aus $|d|=1$ folgt $d = \pm 1$ und $I = \Z$)

		Sei $m \in I$, dann $m = zd + r$ mit $0 \le r < d$, $z \in \Z$.
		Wegen $m \in I, d\in I$ ist $r = zd - m \in I$.
		Also $|r| < |d|$, ein Widerspruch zur Minimalität von $d$, falls $r \neq 0$.
		Also folgt $r = 0$ und $m \in Rd$.
		Für beliebiges $m$ ist $I \subset Rd \subset I$, da $d \in I$ und somit $I = Rd$.
		Also ist $I$ ein Hauptideal.
	\end{proof}
\end{lem}

%St 1.10
\begin{st}[Fundamentalsatz der Arithmetik] \label{1.10}
	In $\N$ lässt sich jede Zahl $n \ge 2$ eindeutig als ein Produkt von Primzahlen, bzw. Primzahlpotenzen aus $\N$ schreiben, also
	\[
		n = p_1^{\alpha_1} \dotsb p_k^{\alpha_k}
	\]
	mit $p_1, \dotsc, p_k$ Primzahlen und $\alpha_i \ge 1$.
	\begin{proof}
		Man muss sich nur überzeugen, dass aufsteigende Ketten von Hauptidealen in $\Z$ abbrechen.
		$\Z m_1 \subset \Z m_2 \implies |m_2| \divs  |m_1|$, also ist \ref{1.8} offensichtlich erfüllt.
		\ref{1.10} folgt dann aus \ref{1.6}, \ref{1.7} und \ref{1.9} und $U(\Z) = \pm 1$.
	\end{proof}
\end{st}

\begin{nt*}
	Die Aussage „Ist $p$ eine Primzahl und $p \divs  ab$, so gilt $p \divs  a$ oder $p \divs  b$“ war bereits Euklid bekannt.
\end{nt*}

\begin{st}[Zur Unendlichkeit von $\P$]
	\begin{enumerate}[a)]
		\item
			Euklid.
			Angenommen $\P$ sei endlich: $\P = \{p_1, p_2, \dotsc, p_k\}$.
			Setze $n = \prod_{i=1}^k p_i$ und betrachte $n + 1$.
			Nach \ref{1.10} lässt sich $n +1$ in ein Produkt von Primzahlen zerlegen.
			Sei $p$ eine solche, dass ist $p \divs n + 1$ und $p \divs  n$ und $p \divs  n + 1 - n = 1$.
		\item
			Euler.
			\[
				\prod_{p\in \P} \Big( \sum_{r=0}^\infty p^{-r} \Big)
				= \prod_{p\in \P} (1 - \f 1p)^{-1}
				\stack{\ref{1.10}}= \sum_{n\in \N} \f 1n
			\]
			ist das Produkt endlich, dann konvergiert die linke Seite als endliches Produkt von absolut konvergenten Reihen.
			Die rechte Seite divergiert jedoch.
		\item
			Polya.
			Betrachte die Zahl $F_n := 2^{2^n} + 1$ ($n$-te Fermatzahl).
			$\{F_n : n \in \N_0\}$ ist offensichtlich unendlich.

			$F_n$ und $F_m$ sind teilerfremd für $n < m$:
			Sei $p$ ein Teiler von $F_n$, dann $2^{2^n} \equiv -1 \mod p$ und $2^{2^m} \equiv 1 \mod p$.
			Also ist $2^{2^m} + 1 \equiv 2 \mod p$, $p$ teilt also $2$, also $p = 2$.
			Aber $F_n$ und $F_m$ sind ungerade, also teilt $p$ nicht $F_m$.

			Mit \ref{1.10} folgt dann, dass $\P$ unendlich ist.
	\end{enumerate}
\end{st}

\begin{df*}
	$f: \N \to \C$ nennt man \emphdef{multiplikative zahlentheoretische Funktion}, wenn $f(1) = 1$ und $f(m n) = f(m) f(n)$ für $(m,n) = 1$.
\end{df*}

\begin{nt}[Kongruenzen und Rechnen in $\Z / n\Z$]
	\begin{enumerate}[a)]
		\item
			$a \equiv b \mod n$ genau dann, wenn $n \divs  ab$.
			$a \equiv b \mod n$ genau dann $\_a = \_b$ in $\Z / n\Z$.
		\item
			Ist $\ggT(c, n) = 1$ (statt $\ggT(a,b)$ schreibt man häufig kurz $(a,b)$), dann ist
			\begin{align*}
				a \equiv b &\mod n
				&\iff ca \equiv cb \mod n
			\end{align*}
			(denn aus $a \equiv b \mod n$ folgt $a -b = n x$, also $c(a-b) = cnx$ und $ca \equiv cb \mod n$.
			Ist $ca = cb = d$, also $c(a-b) = dn$ und wegen $(c,n) = 1$ gilt $c\divs d$ und $dn = \tilde d c n$.
			Es gilt $c(a-b) = c \tilde d n$, also $ab = \tilde n$ und $a \equiv b \mod n$)

			Nur für die Rückrichtung wird $(c,n) = 1$ gebraucht.

			Rechnet man in $\Z / n \Z$, dann ist $(c,n) = 1$ genau dann, wenn $\_c \in U(\Z / n \Z)$.
			Aus $\_a = \_b$ folgt $\_c \_a = \_c \_b$ und wegen $\_c \in U(\Z / n\Z)$ auch $\_c^{-1} \_c \_a = \_c^{-1} \_c \_b$, also $\_a = \_b$.

			Ist $(c, n) \neq 1$, dann ist die Rückrichtung im Allgemeinen falsch, z.B. $n = 4, c = 2$: $2 \cdot 1 = 2 \cdot 3 \mod 4$, aber $1 \not\equiv 3 \mod 4$.
		\item
			Die \emphdef{Eulersche $\phi$-Funktion}, $\phi: \N \to \N$ ist definiert durch
			\[
				\phi(n) := \begin{cases}
					1 & n = 1 \\
					| \{m \in \N : 1 \le m < n \land (m,n) = 1 \} | = |U(\Z / n\Z)| & \text{sonst}
				\end{cases}.
			\]
			$\phi$ ist eine multiplikative zahlentheoretische Funktion.
	\end{enumerate}
\end{nt}

\coursetimestamp{17}{04}{2014}

% St 1.13
\begin{st}[Chinesischer Restsatz] \label{1.13}
	Seien $m_1, \dotsc, m_n$ paarweise teilerfremde natürliche Zahlen und $a_1, \dotsc, a_n \in \Z$.
	Dann existiert $x \in \Z$, so dass
	\begin{align*}
		x &\equiv a_1 \mod m_1,&
		&\dotsc,&
		x &\equiv a_n \mod m_n
	\end{align*}
	($x$ löst simultan diese $n$ Kongruenzen).

	$x$ ist modulo $\tilde m = \prod_{i=1}^n m_i$ eindeutig bestimmt, d.h. ist $y \in \Z$ eine andere Lösung der Kongruenzen, dann ist $y \equiv x \mod \tilde m$.
	\begin{proof}
		Zeige zunächst die Eindeutigkeit.
		Sei also $y \equiv a_1 \mod m_1, \dotsc y \equiv a_n \mod m_n$, dann ist $y -x \equiv \mod m_1, y-x \equiv 0 \mod m_n$.
		Also gilt $m_i \divs y-x$ für alle $i$.
		Da $m_1, \dotsc, m_n$ paarweise teilerfremd, folgt dann mit \ref{1.10}, dass
		\[
			\tilde m = \prod_{i=1}^n m_i \divs y - x,
		\]
		also $y = x \mod \tilde m$.

		Nun zur Existenz:
		Definiere $\phi: \Z / \tilde m \Z \to (\Z / m_1 \Z) \times \dotsb \times (\Z / m_n \Z)$ durch
		\[
			x \mod \tilde m
			\mapsto
			\big( x \mod m_1, \dotsc, x \mod m_n \big).
		\]
		$\phi$ ist injektiv, denn sei $\phi(x \mod \tilde m) = \phi(y \mod \tilde m)$, dann ist
		\[
			\big( x \mod m_1, \dotsc, x \mod m_n \big)
			= \big( y \mod m_1, \dotsc, y \mod m_n \big)
		\]
		und damit $x - y \equiv 0 \mod m_i$ für alle $i$ und damit mit \ref{1.10} $x - y \equiv 0 \mod \tilde m$ auch $x \mod \tilde m = y \mod \tilde m$.
		$\phi$ ist wohldefiniert, denn sei $x \equiv \tilde x \mod \tilde m$, dann ist $x - \tilde x \equiv 0 \mod \tilde m$ und $\tilde m \divs x - \tilde x$, also $m_i \divs x - \tilde x$ für jedes $i$ und somit $x \equiv \tilde x \mod m_i$ für jedes $i$.

		Es gilt
		\[
			\big| \Z / \tilde m \Z \big|
			= \tilde m
			= m_1 m_2 \dotsb m_n,
		\]
		andererseits ist
		\[
			\Big| \Z/ m_1 \Z \times \dotsb \times \Z / m_n \Z \Big|
			= | \Z / m_1 \Z | \cdot \dotsb \cdot | \Z / m_n \Z|
			= m_1 m_2 \dotsb m_n.
		\]
		Beide Mengen sind endlich und gleichmächtig, also ist $\phi$ surjektiv.

		Zu $b := (a_1 \mod m_1, a_2 \mod m_2, \dotsc, a_n \mod m_n)$ existiert also $x \mod \tilde m$ mit $\phi(x \mod \tilde m) = b$.
		Dieses $x$ löst simultan die gegebenen $n$ Kongruenzen.
	\end{proof}
\end{st}

% St 1.14
\begin{st} \label{1.14}
	\begin{enumerate}[a)]
		\item
			Es existiert ein Ringhomomorphismus
			\[
				\Z / m_1m_2 \dotsb m_n \Z
				\isomorphic
				\Z /m_1 \Z \times \Z / m_n \Z.
			\]
		\item
			Es existiert ein Gruppenhomomorphismus
			\[
				U(\Z / m_1 m_2 \dotsb m_n \Z)
				\isomorphic
				U(\Z / m_1 \Z) \times U(\Z / m_n \Z).
			\]
	\end{enumerate}
	\begin{proof}
		\begin{enumerate}[a)]
			\item
				Verwende die im Beweis von \ref{1.13} konstruierte Bijektion $\phi$.
				Zeige noch, dass $\phi$ additiv, multiplikativ ist und $\phi(1) = 1$ gilt.

				Beispielsweise gilt
				\begin{align*}
					y(x \mod \tilde m + y \mod \tilde m)
					&= \phi\big( (x+y) \mod \tilde m \big) \\
					&= \big(x + y \mod m_1, \dotsc, x + y \mod m_n \big) \\
					&= \big( x \mod m_1, \dotsc, x \mod m_n\big) + \big(y \mod m_1, \dotsc, y \mod m_n \big) \\
					&= \phi(x \mod \tilde m) + \phi(y \mod \tilde m)
				\end{align*}
			\item
				Isomorphe Ringe haben isomorphe Einheitengruppen und es gilt
				\[
					U(\Z / m_1 \Z \times \dotsb \times \Z / m_n \Z)
					\isomorphic U(\Z / m_1 \Z) \times \dotsb \times U(\Z / m_n \Z).
				\]
		\end{enumerate}
	\end{proof}
\end{st}

% Folg 1.15
\begin{kor} \label{1.15}
	Die Eulersche $\phi$-Funktion ist eine multiplikative zahlentheoretische Funktion.
	Es gilt
	\[
		\phi( p_1^{a_1} \dotsb p_k^{a_k} )
		=\phi( p_1^{a_1}) \dotsb \phi(p_k^{a_k})
		= \prod_{i=1}^k (p_1 - 1) p_i^{a_i - 1}.
	\]
	\begin{proof}
		$\phi(1) = 1$ gilt per definitionem.
		Ist $n = p^m$ mit Primzahl $p$, dann ist $\phi(p^m) = p^m - p^{m-1}$.
		$p^{m-1}$ sind die Anzahl der natürlichen Zahlen $\le p^m$, welche teilbar durch $p$ sind.
		$p^m - p^{m-1}$ ist die Anzahl der natürlichen Zahlen $< p^m$, welche teilerfremd zu $p^m$ sind.
		Es gilt
		\[
			\phi(n)
			= | U(\Z / n\Z) |
			\stack{\ref{1.14}}= |U(\Z /m _1 \Z) | \cdot \dotsb \cdot | U(\Z /m_l \Z) |
		\]
		für $n = m_1 \dotsb m_l$ und $m_i$ sind paarweise teilerfremd.
	\end{proof}
\end{kor}

% St 1.16
\begin{st}[Teilbarkeitskriterien] \label{1.16}
	Sei $m \in \N$ mit Dezimaldarstellung
	\[
		m = a_n \cdot 10^n + a_1 \cdot 10 + a_0
	\]
	mit $0 \le a_i \le 9$ und $a_n \neq 0$.
	Dann gilt
	\begin{enumerate}[a)]
		\item
			$3 \divs m \iff 3 \divs a_0 + a_1 + a_n$,
		\item
			$9 \divs m \iff 9 \divs a_0 + a_1 + a_n$,
		\item
			$11 \divs m \iff 11 \divs a_0 - a_1 \pm \dotsc + (-1)^n a_n$ ($n$ teilt alternierende Quersumme).
	\end{enumerate}
	\begin{proof}
		\begin{enumerate}[a)]
			\item
				Es gilt $10^d \equiv 1 \mod 3$ für alle $d \in \N$ und damit
				\[
					m \equiv a_n + a_{n-1} + \dotsc a_0 \mod 3
				\]
			\item
				analog.
			\item
				ähnlich.
		\end{enumerate}
	\end{proof}
\end{st}
