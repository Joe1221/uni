\chapter{Integritätsbereiche, irreduzible Elemente, Primzahlen}


\begin{df}
	Ein \emphdef{Ring} $R = (R, +, \cdot)$ ist eine Menge mit zwei Verknüpfungen $+: R \times R \to R$, $\cdot: R \times R \to R$, so dass $(R, +)$ eine abelsche Gruppe ist, $(R, \cdot)$ ein Monoid ist und die Distributivgesetze
	\begin{align*}
		(r+s)t &= rt + st \\
		r(s+t) &= rs + rt
	\end{align*}
	für alle $r, s, t \in R$ gelten.
	Mit $0$ und $1$ bezeichnet man die neutralen Elemente bezüglich $+$, bzw. $\cdot$.

	$R$ nennt man \emphdef[Ring!kommutativ]{kommutativ}, wenn $rs = sr$ für alle $r,s \in R$.
	\begin{note}
		Nach unserer Definition hat jeder Ring ein Einselement.
	\end{note}
\end{df}

\begin{ex}
	\begin{itemize}
		\item
			$\Z$,
		\item
			$\Z / n \Z$,
		\item
			$(K)^{n\times m}$ mit $K$ Körper oder Ring,
		\item
			Körper
			\begin{itemize}
				\item
					$\Q$,
				\item
					$\R$,
				\item
					$\C$,
				\item
					$\Z / p \Z$ mit Primzahl $p$.
			\end{itemize}
		\item
			$K[x]$, der Polynomring in einer Variablen mit Koeffizienten in $K$,
		\item
			$(K)^{n\times m}$ mit Körper $K$ ist isomorph zu $\End_K(V)$, wobei $V$ ein endlichdimensionaler $K$-Vektorraum,
		\item
			$\H, (K)^{n\times n}$ für $n \ge 2$ sind nicht kommutativ,
		\item
			$\Z[i] = \{ z_1 + z_2 i : z_1, z_2 \in \Z \}$.
	\end{itemize}
\end{ex}

\begin{df}
	Eine abelsche Untergruppe $(I, +)$ eines Ringes $(R, +, \cdot)$ nennt man \emphdef{Linksideal}, wenn
	\[
		\forall r \in R, m \in I : rm \in I
	\]
	und \emphdef[Ideal!Rechtsideal]{Rechtsideal}, wenn
	\[
		\forall r \in R, m \in I : m r \in I
	\]
	und \emphdef[Ideal!zweiseitiges Ideal]{zweiseitiges Ideal}, wenn es zugleich Links- und Rechtsideal ist.
	In kommutative Ringen spricht man meistens nur von \emphdef[Ideal]{Idealen}
\end{df}

\begin{ex}
	\begin{itemize}
		\item
			In $\Z$ sind für festes $n \in \N$
			\[
				\Z n = \{ z \in \Z : z = x n \text{ für ein $x \in \Z$} \}
				= \{ z n : z \in \Z \}
				= n \Z
			\]
			Ideale.
		\item
			In jedem Ring $R$ ist für festes $s \in R$
			\begin{align*}
				Rs &= \{ r s : r \in R \}, &
				sR &= \{ sr : r \in R \}
			\end{align*}
			ein Links- bzw. Rechtsideal.
			Solche Ideale nennt man \emphdef[Hauptideal]{Hauptideale}.
	\end{itemize}
\end{ex}

% Def 1.1
\begin{df}
	Sei $R$ ein Ring und $R^* := R \setminus \{0\}$.
	\begin{enumerate}[a)]
		\item
			$a \in R^*$ nennt man \emphdef{Nullteiler}, wennn es eine $b \in R^*$ gibt mit $ab = 0$ oder $ba = 0$.
		\item
			$\_ u \in R^*$ nennt man eine \emphdef{Einheit}, wenn es $v, w \in R^*$ gibt mit $v \_ u = 1$ und $\_ u w = 1$
		\item
			$R$ nennt man \emphdef{Integritätsbereich}, wenn $R$ kommutativ ist und keine Nullteiler besitzt.
		\item
			$R$ heißt \emphdef{Hauptidealring}, wenn jedes Ideal von $R$ ein Hauptideal ist.
		\item
			$R$ heißt \emphdef{Hauptidealbereich}, wenn $R$ Integritätsbereich und Hauptidealring ist.
	\end{enumerate}
	\begin{note}
		Die Menge aller Einheiten bildet mit der Multiplikation eine Gruppe, die sogenannte \emphdef{Einheitengruppe} von $R$, geschrieben $U(R)$.
	\end{note}
\end{df}

% Bsp 1.2
\begin{ex}
	\begin{enumerate}[a)]
		\item
			Körper $K$ ($\Q, \R, \C$) sind Integritätsbereiche und Hauptidealbereiche, $U(K) = K^*$ (bei Körper verlangt man $0 \neq 1$, d.h. Körper haben mindestens zwei Elemente).
		\item
			$\Z$ ist Integritätsbereich und Hauptidealbereich $U(\Z) = \{ \pm 1 \} \isomorphic C_2$
		\item
			$\Z[x]$ ist Integritätsbereich, aber kein Hauptidealbereich.
			Wenn $K$ ein Körper ist, dann ist $K[x]$ ein Hauptidealbereich.
		\item
			$\Z / 4 \Z$ ist kein Integritätsbereich ($\_ 2 \cdot \_ 2 = 0$)
		\item
			$K^{2+2}$ hat Nullteiler, z.B. $\Matrix{0 & 1 \\ 0 & 0 } \Matrix{0 & 1 \\ 0 & 0} = 0$,
		\item
			Wie bereichnet man Inverse in $\Z / n \Z$?

			Mit Hilfe des Euklidischen Algorthmus (Division mit Rest) kann man zu gegebenen Zahlen $a,b \in R$ Zahlen $z_1, z_2 \in R$ berechnen mit $z_1 a + z_2 b = \ggT(a,b)$ (Lemma von Bezout).

			Sei $n \in \N$ vorgegeben und $a \in \N$ mit $\ggT(a, n) = 1$ vorgegeben.
			Nach Bezout existieren $z_1, z_2 \in \Z$ mit $z_1 a + z_2 n = 1$.
			Modulo $n$ ergibt sich
			\[
				\_ z_1 \cdot \_ a + \underbrace{\_ z_2 \_ n}_{=\_ 0} = \_ 1
			\]
			also ist $\_z_1 \_ a = \_ 1$ und $\_ z_1$ Invers zu $\_ a$.

			Beispiel für $n = 31, a = 7$:
			\begin{align*}
				31 &= 4 \cdot 7 + 3 \\
				7 &= 2 \cdot 3 + 1
			\end{align*}
			und
			\begin{align*}
				2 \cdot 31 &= 2 \cdot 4 \cdot 7 + 2 \cdot 3 = 2 \cdot 4 \cdot 7 + 7 - 1 \\
				2 \cdot 31 &= 9 \cdot 7 - 1 \\
				1 &= 9 \cdot 7 - 2 \cdot 31
			\end{align*}
			also ist $\_ 9$ invers zu $\_ 7$.
	\end{enumerate}
\end{ex}

% Def 1.3
\begin{df}
	Sei $R$ ein kommutativer Ring und $r, s \in R$.
	\begin{enumerate}[a)]
		\item
			$r$ heißt \emphdef{Teiler} von $s$, wenn $q \in R$ existiert mit $s = r q$, wir schreiben dann $r | s$.
		\item
			$r$ heißt \emphdef[Teiler!echt]{echter Teiler} von $s$, wenn $r|s$, $s \not | r$ und $r \not\in U(R)$, wir schreiben dann $r \| s$.
		\item
			$r$ heißt \emphdef{irreduzibel}, wenn $r \neq 0$, $r \not\in U(R)$ und $r$ besitzt keine echten Teiler.
		\item
			$r$ heißt \emphdef{prim}, wenn $r \neq 0$, $r \not\in U(R)$ und
			\[
				\forall a,b \in R : r | ab \implies r | a \lor r | b.
			\]
	\end{enumerate}
	\begin{note}
		$1$ ist weder irreduzibel, noch prim.
	\end{note}
\end{df}

% Lem 1.4
\begin{lem}
	In einem Integritätsbereich $R$ gilt in $(R^*, \cdot)$ die Kürzungsregel.
	\begin{proof}
		Aus $r s = r t$ folgt $r(s-t) = 0$ und da $r \in R^*$ und $R$ Integritätsbereich folgt $s-t = 0$, also $s = t$.
	\end{proof}
\end{lem}

% Lem 1.5
\begin{lem}
	Sei $R$ ein Integritätsbereich. Es gilt
	\begin{enumerate}[a)]
		\item
			$r, s \in R^*$. Aus $r | s$ und $s | r$ folgt:
			\[
				\exists \_ u \in U(R) : s \_ u r
			\]
		\item
			Ist $r$ prim, so auch irreduzibel
		\item
			Ist $p$ prim und $\_ u \in U(R)$, dann ist auch $\_ u p$ prim.
		\item
			Sind $p$ und $q$ prim und $p | q$, dann gilt $p = \_ u q$ für ein $\_ u \in U(R)$.
	\end{enumerate}
	\begin{proof}
		\begin{enumerate}[a)]
			\item
				Sei $s = rq$ und $r = s\tilde q$, dann ist $s = s \tilde q q$ und mit \ref{1.4} $1 = \tilde q q$ und damit $\tilde q, q \in U(R)$.
			\item
				Sei $r$ prim aber nicht irreduzibel.
				Es gilt dann $r \in R^*$, $r = st$ und $s \| r$.
				Da $r$ prim, folgt $r | s$ oder $r | t$.
				$r | s$ ist wegen $s \| r$ unmöglich, also $r | t$.
				Schreibe $t = \tilde t r$, also $r = s \tilde t r$ und mit \ref{1.4} $1 = s \tilde t$.
				Also ist $s$ eine Einheit, ein Widerspruch zu $s \| r$.
		\end{enumerate}
	\end{proof}
\end{lem}

% Satz 1.6
\begin{st}[Eindeutigkeit der Zerlegung in Primelemente]
	Sei $R$ ein Integritätsbereich und $r \in R$.
	Sei $r = p_1 \dotsc p_m$ und $r = q_1 \dotsc, p_n$ mit $p_1, \dotsc, p_m, q_1, \dotsc, q_n$ prim.
	Dann gilt $m = n$ und nach geeigneter Nummerierung gilt $p_i = \_ u_i \cdot q_i$ mit $\_ u_1 \in U(R)$.
	\begin{proof}
		Funktioniert mit Induktion
	\end{proof}
\end{st}

Gemäß unserer Definition ist eine Primzahl in $\Z$ ein irreduzibles Element von $\Z$.
Um zu einer eindeutigen Primfaktorzerlegung mittels \ref{1.6} zu gelangen, bräuchten wir, dass aus irreduzibel prim folgt.
