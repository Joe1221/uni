\chapter{Integritätsbereiche, irreduzible Elemente, Primzahlen}

\section{Wiederholung}

\begin{df*}
	Ein \emphdef{Ring} $R = (R, +, \cdot)$ ist eine Menge mit zwei Verknüpfungen $+: R \times R \to R$, $\cdot: R \times R \to R$, so dass $(R, +)$ eine abelsche Gruppe ist, $(R, \cdot)$ ein Monoid (Neutrales Element, Assoziativität) ist und die Distributivgesetze
	\begin{align*}
		(r+s)t &= rt + st, \\
		r(s+t) &= rs + rt
	\end{align*}
	für alle $r, s, t \in R$ gelten.
	Mit $0$ und $1$ bezeichnet man die neutralen Elemente bezüglich $+$, bzw. $\cdot$.

	$R$ nennt man \emphdef[Ring!kommutativ]{kommutativ}, wenn die Multiplikation kommutativ ist, d.h. $rs = sr$ für alle $r,s \in R$.
	\begin{note}
		Nach unserer Definition hat jeder Ring ein Einselement.
		In anderer Literatur wird dies nicht immer vorausgesetzt.
	\end{note}
\end{df*}

\begin{ex*}
	\begin{itemize}
		\item
			$\Z$ ist ein kommutativer Ring,
		\item
			$\Z / n \Z$ ist ein kommutativer Ring,
		\item
			Sei $K$ ein Ring, dann ist der Matrizenraum $K^{n\times n}$ ein Ring.
			Ist $K$ zusätzlich ein Körper, so ist $K^{n\times n} \isomorphic \End_K(V)$ mit $n$-dimensionalem $K$-Vektorraum $V$.
		\item
			Alle Körper sind insbesondere Ringe, z.B.
			\begin{itemize*}[label=]
				\item
					$\Q$,
				\item
					$\R$,
				\item
					$\C$,
				\item
					$\Z / p \Z$ mit Primzahl $p$.
			\end{itemize*}
		\item
			$K[x]$, der Polynomring in einer Variablen mit Koeffizienten in $K$,
		\item
			$\H, (K)^{n\times n}$ für $n \ge 2$ sind nicht-kommutative Ringe,
		\item
			Die \emphdef{Gaußsche Zahlen} $\Z[i] = \{ z_1 + z_2 i : z_1, z_2 \in \Z \}$ bilden einen Ring mit der üblichen (von $\C$ geerbten) Multiplikation.
	\end{itemize}
\end{ex*}

\begin{df*}
	Eine abelsche Untergruppe $(I, +)$ eines Ringes $(R, +, \cdot)$ nennt man \emphdef{Linksideal}, wenn
	\[
		\forall r \in R, m \in I : rm \in I
	\]
	und \emphdef[Ideal!Rechtsideal]{Rechtsideal}, wenn analog
	\[
		\forall r \in R, m \in I : m r \in I
	\]
	und \emphdef[Ideal!zweiseitiges Ideal]{zweiseitiges Ideal}, wenn es zugleich Links- und Rechtsideal ist.
	In kommutativen Ringen spricht man meistens nur von \emphdef[Ideal]{Idealen}
\end{df*}

\begin{ex*}
	\begin{itemize}
		\item
			$0$ und $R$ sind stets Ideale, sogenannte \emphdef{triviale Ideale}.
		\item
			In $\Z$ ist für festes $n \in \N$
			\[
				\Z n = \Set{ z \in \Z & z = x n \text{ für ein $x \in \Z$} }
				= \Set{ z n & z \in \Z }
				= n \Z
			\]
			ein Ideal.
		\item
			In jedem Ring $R$ ist für festes $s \in R$
			\begin{align*}
				Rs &= \Set{ r s & r \in R }, &
				sR &= \Set{ sr & r \in R }
			\end{align*}
			ein Links- bzw. Rechtsideal.
			In kommutativen Ringen sind beide äquivalent und man nennt sie \emphdef[Hauptideal]{Hauptideale}.
	\end{itemize}
\end{ex*}

\begin{alg*}[Lemma von Bézout, Euklidischer Algorithmus]
	\begin{algorithmic}
		\Require{$a, b \in \N$, $a \ge b$}
		\Ensure{$z_1 a + z_2 b = d = \ggT(a,b) \in \N; z_1, z_2 \in \Z$}
		\State{$(r_0, r_1) \gets (a, b)$}
		\State{$k \gets 1$}
		\While{$r_k > 0$}
			\State{$r_{k-1} = p_k r_k + r_{k+1}$} \Comment{Division mit Rest}
			\State{$k \gets k+1$}
		\EndWhile
		\State{$d \gets r_{k-1}$} \Comment{$= \ggT(a,b)$}
		\State{$k \gets k - 2$}
		\While{$k > 0$}
			\State{$d = s_k r_k + t_{k-1} r_{k-1}$} \Comment{Mittels Einsetzen von oben}
			\State{$k \gets k-1$}
		\EndWhile
		\State{$(z_1, z_2) \gets (s_1, t_0)$}
	\end{algorithmic}
\end{alg*}

\section{Grundlagen}

% Def 1.1
\begin{df} \label{1.1}
	Sei $R$ ein Ring.
	Wir schreiben $R^* := R \setminus \{0\}$.
	\begin{enumerate}[a)]
		\item
			$a \in R^*$ nennt man \emphdef{Nullteiler}, wenn es ein $b \in R^*$ gibt mit $ab = 0$ oder $ba = 0$.
		\item
			$\unit u \in R^*$ nennt man eine \emphdef{Einheit} oder \emphdef{invertierbar}, wenn es $v, w \in R^*$ gibt mit $v \unit u = 1$ und $\unit u w = 1$
		\item
			$R$ nennt man \emphdef{Integritätsbereich}, wenn $R$ kommutativ ist und keine Nullteiler besitzt.
		\item
			$R$ heißt \emphdef{Hauptidealring}, wenn $R$ ein kommutativer Ring ist und jedes Ideal von $R$ ein Hauptideal ist.
		\item
			$R$ heißt \emphdef{Hauptidealbereich}, wenn $R$ Integritätsbereich und Hauptidealring ist.
	\end{enumerate}
	Die Menge aller Einheiten von $R$ bildet mit der Multiplikation auf $R$ eine Gruppe, die sogenannte \emphdef{Einheitengruppe} von $R$, geschrieben $U(R)$.
\end{df}

% Bsp 1.2
\begin{ex} \label{1.2}
	\begin{enumerate}[a)]
		\item
			Jeder Körper $K$ (z.B. $\Q, \R, \C$) ist ein Hauptidealbereich.
			Es gilt $U(K) = K^* = K \setminus \{0\}$ (bei Körper verlangt man $0 \neq 1$, d.h. Körper haben mindestens zwei Elemente).
		\item
			$\Z$ ist ein Hauptidealbereich mit $U(\Z) = \{ \pm 1 \} \isomorphic C_2$.
		\item
			$\Z[x]$ ist ein Integritätsbereich, jedoch kein Hauptidealbereich.
			Ist $K$ ein Körper, dann ist $K[x]$ ein Hauptidealbereich.
		\item
			$\Z / 4 \Z$ ist kein Integritätsbereich ($\_ 2 \cdot \_ 2 = 0$)
		\item
			$K^{2\times 2}$ hat Nullteiler, z.B. $\Matrix{0 & 1 \\ 0 & 0 } \Matrix{0 & 1 \\ 0 & 0} = 0$,
		\item
			Wie berechnet man Inverse in $\Z / n \Z$?

			Mit Hilfe des Euklidischen Algorithmus kann man zu gegebenen Zahlen $a,b \in R$ Zahlen $z_1, z_2 \in R$ berechnen mit $z_1 a + z_2 b = \ggT(a,b)$ (Lemma von Bezout).

			Seien $n \in \N$ und $a \in \N$ mit $\ggT(a, n) = 1$ vorgegeben.
			Nach Bezout existieren $z_1, z_2 \in \Z$ mit $z_1 a + z_2 n = 1$.
			Modulo $n$ ergibt sich
			\[
				\_ z_1 \cdot \_ a + \underbrace{\_ z_2 \_ n}_{=\_ 0} = \_ 1
			\]
			also ist $\_z_1 \_ a = \_ 1$ und $\_ z_1$ Inverse zu $\_ a$.

			Beispiel für $n = 31, a = 7$:
			Wir rechnen
			\begin{align*}
				31 &= 4 \cdot 7 + 3 \\
				7 &= 2 \cdot 3 + 1
			\end{align*}
			und
			\begin{align*}
				1 &= 7 - 2 \cdot 3 \\
				&= 7 - 2 (31 - 4 \cdot 7) \\
				&= -2 \cdot 31 + 9 \cdot 7.
			\end{align*}
			Also $9 \cdot 7 = 1 + 2 \cdot 31 \equiv 1 \bmod 31$, d.h. $\_ 9$ invers zu $\_ 7$ in $\Z / 31 \Z$.
	\end{enumerate}
\end{ex}

% Def 1.3
\begin{df} \label{1.3}
	Sei $R$ ein kommutativer Ring und $r, s \in R$.
	\begin{enumerate}[a)]
		\item
			$r$ heißt \emphdef{Teiler} von $s$, wenn $q \in R$ existiert mit $s = r q$, wir schreiben dann $r \divs  s$.
		\item
			$r$ heißt \emphdef[Teiler!echt]{echter Teiler} von $s$, wenn $r\divs s$, $s \ndivs  r$ und $r \not\in U(R)$, wir schreiben dann $r \divs*  s$.
		\item
			$r$ heißt \emphdef{irreduzibel} oder \emphdef{irreduzibles Element}, wenn $r \neq 0$, $r \not\in U(R)$ und $r$ besitzt keine echten Teiler.
		\item
			$r$ heißt \emphdef{prim} oder \emphdef{Primelement}, wenn $r \neq 0$, $r \not\in U(R)$ und
			\[
				\forall a,b \in R : r \divs  ab \implies r \divs  a \lor r \divs  b.
			\]
	\end{enumerate}
	\begin{note}
		$1 \in U(R)$ ist weder irreduzibel, noch prim.
	\end{note}
\end{df}

% Lem 1.4
\begin{lem} \label{1.4}
	Für einen Integritätsbereich $R$ gilt in $(R^*, \cdot)$ die Kürzungsregel, d.h. für $s, t \in R, r \in R^*$ gilt $r s = r t \implies s = t$.
	\begin{proof}
		Aus $r s = r t$ folgt $r(s-t) = 0$ und da $R$ Integritätsbereich und $r \in R^*$ folgt $s-t = 0$, also $s = t$.
	\end{proof}
\end{lem}

% Lem 1.5
\begin{lem} \label{1.5}
	Sei $R$ ein Integritätsbereich. Es gilt
	\begin{enumerate}[a)]
		\item
			Seien $r, s \in R^*$.
			Dann gilt:
			\[
				r\divs s  \land  s\divs r
				\implies
				\exists \unit u \in U(R) : s = \unit u r.
			\]
		\item
			Ist $r$ prim, so auch irreduzibel.
		\item
			Ist $p$ prim und $\unit u \in U(R)$, dann ist auch $\unit u p$ prim.
		\item
			Sind $p$ und $q$ prim und $p \divs  q$, dann gilt $p = \unit u q$ für ein $\unit u \in U(R)$.
	\end{enumerate}
	\begin{proof}
		\begin{enumerate}[a)]
			\item
				Sei $s = rq$ und $r = s\tilde q$, dann ist $s = s \tilde q q$.
				Mit \ref{1.4} folgt $1 = \tilde q q$ und damit $\tilde q, q \in U(R)$.
			\item
				Angenommen $r$ sei prim aber nicht irreduzibel.
				Es gilt dann $r \in R^*$, $r = st$ und $s \divs*  r$.
				Da $r$ prim, folgt $r \divs  s$ oder $r \divs  t$.
				$r \divs  s$ ist wegen $s \divs*  r$ unmöglich, also $r \divs  t$.
				Schreibe damit $t = \tilde t r$, also $r = s \tilde t r$ und mit \ref{1.4} folgt $1 = s \tilde t$.
				Also ist $s$ eine Einheit, ein Widerspruch zu $s \divs* r$.
		\end{enumerate}
	\end{proof}
\end{lem}

\section{Primfaktorzerlegung}

% Satz 1.6
\begin{st}[Eindeutigkeit der Zerlegung in Primelemente] \label{1.6}
	Sei $R$ ein Integritätsbereich und $r \in R$.
	Sei $r = p_1 \dotsb p_m$ und $r = q_1 \dotsb q_n$ mit $p_1, \dotsc, p_m, q_1, \dotsc, q_n$ prim.
	Dann gilt $m = n$ und nach geeigneter Nummerierung gilt $p_i = \unit u_i q_i$ mit $\unit u_1 \in U(R)$.
	\begin{proof}
		Da $p_1$ prim und $p_1 \divs r = q_1 \dotsb q_n$, gilt $p_1 \divs  q_{i_0}$ für einen Index $i_0$.
		Sei \oBdA $i_0 = 1$.
		Nach \ref{1.5} d) gilt dann $q_1 = p_1 \unit u_1$ mit $\unit u_1 \in U(R)$.
		Also ist
		\[
			p_1 \dotsb p_m = p_1 \unit u_1 q_{2} q_3 \dotsb q_{n}.
		\]
		Nach der Kürzungsregel folgt dann
		\[
			p_2 \dotsb p_m = \unit u_1 q_2 \dotsb q_n.
		\]
		Da $p_2$ prim und $p_2 \ndivs \unit u_1$ (sonst $p_2 \in U(R)$) gilt wie oben \oBdA $p_2 \divs q_2$, $q_2 = p_2 \unit u_2$.
		Das Argument lässt sich induktiv mit $q_i = p_i \unit u_i$ fortsetzen.

		Wäre \oBdA $m < n$, dann schließlich zu $m = 0$, also $1 = \unit u_1 \dotsb \unit u_m q_{m+1} \dotsb q_{n}$, d.h. $q_{i} \in U(R)$, ein Widerspruch, also $m = n$.
	\end{proof}
\end{st}

Primzahlen in $\Z$ haben nur $1$ und sich selbst als Teiler und entsprechen gerade den (positiven) irreduziblen Elementen von $\Z$.
\ref{1.6} liefert die Eindeutigkeit von Zerlegungen allerdings nur für Primelemente.
Wann sind irreduzible Elemente prim?

\coursetimestamp{14}{04}{2014}

% Lem 1.7
\begin{lem} \label{1.7}
	Sei $R$ ein Hauptidealbereich, dann ist $p \in R$ genau dann irreduzibel, wenn $p$ prim ist.
	\begin{proof}
		\begin{segnb}{$\impliedby$}
			Gilt nach \ref{1.5} b).
		\end{segnb}
		\begin{segnb}{$\implies$}
			Sei $p \in R$ irreduzibel und $p \divs ab$ für $a, b \in R$, zeige $p \divs a$ oder $p \divs b$.
			$Rp + Ra$ ist ein Ideal, also existiert da $R$ Hauptidealbereich, $d \in R$ mit $Rp + Ra = Rd$.
			Also gilt $a,p \in Rd$ und $p = sd$ für ein $s \in R$.
			Da $p$ irreduzibel, folgt $s \in U(R)$ oder $d \in U(R)$.
			Wir unterscheiden beide Fälle:
			\begin{segnb}{$s\in U(R)$}
				Es gilt $d = s^{-1}p$, d.h. $p \divs d$.
				Da auch $d \divs a$, gilt also $p\divs a$.
			\end{segnb}
			\begin{segnb}{$d\in U(R)$}
				Dann gilt $R = Rd = Rp + Ra$, also $1 = r_1 p + r_2 a$ für $r_1, r_2 \in R$.
				Wegen $p \divs ba$ also
				\[
					p \divs
					b r_1 p + r_2 a b
					= b ( r_1 p + r_2 a)
					= b.
				\]
			\end{segnb}
			Somit ist $p$ prim.
		\end{segnb}
	\end{proof}
\end{lem}

%Lem 1.8
\begin{lem}[Existenz der Zerlegung in irreduzible Elemente] \label{1.8}
	Sei $R$ ein Integritätsbereich und jede aufsteigende Kette von Hauptidealen ($Ra_1 \subset Ra_2 \subset \dotsc$) besitze ein maximales Element.

	Dann lässt sich jedes $s \in R^* \setminus U(R)$ in ein endliches Produkt von irreduziblen Elementen zerlegen.
	\begin{proof}
		Sei $\oBdA$ $s$ nicht irreduzibel, dann besitzt $s$ einen echten Teiler, d.h. $s = a_1 b_1$ mit $a_1 \divs*  s$.

		Es gilt $s \in Ra_1$ und $Rs \subset Ra_1$.
		Angenommen $Rs = Ra_1$, dann $s \divs  a_1$, ein Widerspruch zu $a_1 \divs*  s$.
		Also ist $Rs \subsetneq Ra_1$.

		Ähnlich sieht man $b_1 \divs*  s$, denn wenn $b_1 \divs  s$ und $s \divs  b_1$, dann ist $s = a_1 b_1 = a_1 x s$, also $a_1 x = 1$ und $a_1 \in U(R)$, ein Widerspruch zu $a_1 \divs* s$.
		Also ist auch $Rs \subsetneq Rb_1$.

		Sind $a_1$, bzw. $b_1$ nicht irreduzibel, so kann man das Argument wiederholen.
		Induktiv ergeben sich dann aufsteigende Ketten von Hauptidealen, welche nach Voraussetzung abbrechen.
		Also muss $s$ ein endliches Produkt von irreduziblen Elementen sein.
	\end{proof}
\end{lem}

%Lem 1.9
\begin{lem} \label{1.9}
	$\Z$ ist ein Hauptidealbereich.
	\begin{proof}
		Sei $I$ ein nicht-triviales Ideal von $\Z$ ($0$ und $\Z$ sind trivialerweise Hauptideale).
		Dann gibt es ein $d \in I$ mit $2 \le |d| \le |s|$ für alle $s \in I \setminus \Set 0$.
		($|d| \ge 2$, denn aus $|d|=1$ folgt $d = \pm 1$ und $I = \Z$, ein triviales Ideal)

		Sei $m \in I$ beliebig, dann existieren $m = zd + r$ mit $0 \le r < |d|$, $z \in \Z$.
		Wegen $m \in I, d\in I$ ist $r = zd - m \in I$.
		Also $|r| < |d|$, ein Widerspruch zur Minimalität von $d$, falls $r \neq 0$.
		Also folgt $r = 0$ und $m = zd \in Rd$.
		Da $m$ beliebig gewählt war, folgt $I \subset Rd \subset I$ und somit $I = Rd$.
		Also ist $I$ ein Hauptideal und $\Z$ Hauptidealbereich.
	\end{proof}
\end{lem}

%St 1.10
\begin{st}[Fundamentalsatz der Arithmetik] \label{1.10}
	In $\N$ lässt sich jede Zahl $n \ge 2$ eindeutig als ein Produkt von Primzahlen, bzw. Primzahlpotenzen aus $\N$ schreiben, also
	\[
		n = p_1^{\alpha_1} \dotsb p_k^{\alpha_k}
	\]
	mit paarweise verschiedenen Primzahlen $p_1, \dotsc, p_k$ und $\alpha_i \ge 1$.
	\begin{proof}
		Mit \ref{1.9}, \ref{1.7} sind die Primzahlen in $\N$ sowohl prim, als auch irreduzibel in $\Z$.
		Um \ref{1.8}, \ref{1.6} anzuwenden muss man sich nur überzeugen, dass aufsteigende Ketten von Hauptidealen in $\Z$ abbrechen.
		Wegen $\Z m_1 \subset \Z m_2 \implies |m_2| \divs  |m_1|$ (Division mit Rest) ist dies der Fall.
		\ref{1.10} folgt schließlich aus der Überlegung $U(\Z) = \{ \pm 1 \}$.
	\end{proof}
\end{st}

\begin{nt*}
	Die Aussage „Ist $p$ eine Primzahl und $p \divs  ab$, so gilt $p \divs  a$ oder $p \divs  b$“ war bereits Euklid bekannt.
\end{nt*}

\begin{st}[Zur Unendlichkeit von $\P$]
	$\P$ ist unendlich.
	\begin{proof}[nach Euklid]
		Angenommen $\P$ sei endlich: $\P = \{p_1, p_2, \dotsc, p_k\}$.
		Setze $n = \prod_{i=1}^k p_i$ und betrachte $n + 1$.
		Nach \ref{1.10} lässt sich $n +1$ in ein Produkt von Primzahlen zerlegen.
		Sei $p$ eine solche, dann gilt $p \divs n + 1$ und $p \divs  n$, also $p \divs  n + 1 - n = 1$, ein Widerspruch zu $p \not\in U(\Z)$.
	\end{proof}
	\begin{proof}[nach Euler]
		Angenommen $\P$ sei endlich, dann ist
		\[
			\infty
			> \prod_{p\in \P} \dfrac 1{1 - \f 1p}
			= \prod_{p\in \P} \Big( \sum_{r=0}^\infty p^{-r} \Big)
			\stack{\ref{1.10}}= \sum_{n\in \N} \f 1n
			= \infty,
		\]
		ein Widerspruch.
	\end{proof}
	\begin{proof}[nach Polya]
		Betrachte die $n$-te Fermatzahl $F_n := 2^{2^n} + 1$.
		Die Menge aller Fermatzahlen $\{F_n : n \in \N_0\}$ ist offensichtlich unendlich.

		Wir zeigen $F_n$ und $F_m$ sind teilerfremd für $n < m$.
		Sei $p$ ein Teiler von $F_n$, dann ist $2^{2^n} \equiv -1 \bmod p$ und durch Potenzieren $2^{2^m} \equiv 1 \bmod p$.
		Angenommen $0 \equiv 2^{2^m} + 1 \equiv 2 \bmod p$, dann ist $p = 2$.
		$F_n$ ist jedoch ungerade, ein Widerspruch, also sind $F_n$ und $F_m$ teilerfremd.

		Mit \ref{1.10} folgt dann, dass $\P$ unendlich ist.
	\end{proof}
\end{st}

\section{Kongruenzen und Teilbarkeit}

\begin{conv*}
	Wir verwenden die Notation $(a,b) := \ggT(a,b)$.
\end{conv*}

\begin{df*}
	$f: \N \to \C$ nennt man \emphdef{multiplikative zahlentheoretische Funktion}, wenn
	\begin{enumerate}[i)]
		\item
			$f(1) = 1$,
		\item
			$f(m n) = f(m) f(n)$ für $(m,n) = 1$.
	\end{enumerate}
\end{df*}

\begin{nt}[Kongruenzen und Rechnen in $\Z / n\Z$] \label{1.12}
	\begin{enumerate}[a)]
		\item
			Wir schreiben $a \equiv b \bmod n$ für $n \divs a - b$.

			Es gilt $a \equiv b \bmod n$ genau dann, wenn $\_a = \_b$ in $\Z / n\Z$.
		\item
			Es gilt
			\begin{align*}
				a &\equiv b \mod n &
				&\begin{aligned}\implies \\ \underset{\mathclap{(c, n) = 1}}{\impliedby}\end{aligned}&
				ca &\equiv cb \mod n.
			\end{align*}
			Ist $(c, n) \neq 1$, dann ist die zweite Implikation im Allgemeinen falsch, z.B. für $c = 4, n = 2$ gilt $2 \cdot 1 = 2 \cdot 3 \bmod 4$, aber $1 \not\equiv 3 \bmod 4$.
			\begin{proof}
				denn aus $a \equiv b \bmod n$ folgt $a -b = n x$, also $c(a-b) = cnx$ und $ca \equiv cb \bmod n$.
				Ist $ca - cb = dn$, also $c(a-b) = dn$ und wegen $(c,n) = 1$ gilt $c\divs d$, d.h. $c(a-b) = c \tilde d n$ für ein $\tilde d \in \Z$, gekürzt $a - b = \tilde d n$, also $n \divs a-b$ und $a \equiv b \bmod n$.
			\end{proof}
		\item
			$(c,n) = 1$ genau dann, wenn $c \bmod n \in U(\Z / n \Z)$.
			\begin{proof}
				\begin{align*}
					(c,n) = 1
					&\iff \exists z_1, z_2 \in \Z : 1 = z_1 c + z_2 n \\
					&\iff \exists z_1 \in \Z : 1 \equiv z_1 c \bmod n \\
					&\iff c \in U(\Z / n \Z).
				\end{align*}
			\end{proof}
		\item
			Es gilt
			\begin{align*}
				a &\equiv b \bmod n, m &
				&\begin{aligned}\overset{\mathclap{(n,m) = 1}}{\implies} \\ \impliedby\end{aligned}&
				a & \equiv b \bmod nm.
			\end{align*}
			Mittels \ref{1.10} lässt sich dies auf endlich viele Kongruenzen (paarweise teilerfremd) verallgemeinern.
			\begin{proof}
				\begin{segnb}{\ProofImplication}
					Es gilt $n,m \divs a - b$.
					Wegen $\kgV(n,m) = \f{nm}{\ggT(n,m)} = nm$ daher auch $nm \divs a - b$, also $a \equiv b \bmod nm$.
				\end{segnb}
				\begin{segnb}{\ProofImplication*}
					Wegen $nm \divs a - b$ ist $n,m \divs nm \divs a - b$, also $a \equiv b \bmod n,m$.
				\end{segnb}
			\end{proof}
		\item
			Die \emphdef{Eulersche $\phi$-Funktion}, $\phi: \N \to \N$ ist definiert durch $\phi(1) := 1$ und
			\[
				\phi(n)
				:= \Big| \big\{m \in \N : 1 \le m \le n \land (m,n) = 1 \big\} \Big|
				= |U(\Z / n\Z)|
			\]
			für $n > 1$.
			$\phi$ ist eine multiplikative zahlentheoretische Funktion (siehe \ref{1.15}).
	\end{enumerate}
\end{nt}

\coursetimestamp{17}{04}{2014}

% St 1.13
\begin{st}[Chinesischer Restsatz] \label{1.13}
	Seien $m_1, \dotsc, m_n \in \N$ paarweise teilerfremd und $a_1, \dotsc, a_n \in \Z$.
	Dann existiert $x \in \Z$, so dass
	\begin{align*}
		x &\equiv a_1 \mod m_1,&
		&\dotsc,&
		x &\equiv a_n \mod m_n,
	\end{align*}
	mit anderen Worten: $x$ löst simultan diese $n$ Kongruenzen.

	$x$ ist modulo $\prod_{i=1}^n m_i$ eindeutig bestimmt, d.h. ist $y \in \Z$ eine andere Lösung der Kongruenzen, dann ist $y \equiv x \bmod \prod_{i=1}^n m_i$.
	\begin{proof}
		\begin{segnb}{Eindeutigkeit}
			Sei $y \equiv a_1 \bmod m_1, \dotsc, y \equiv a_n \bmod m_n$, dann ist $x \equiv y \bmod m_1, \dotsc, m_n$.
			Mit \ref{1.12} d) folgt $y \equiv x \bmod \prod_{i=1}^n m_i =: m_i$.
		\end{segnb}
		\begin{seg}{Existenz}
			Definiere $\phi: \Z / \tilde m \Z \to (\Z / m_1 \Z) \times \dotsb \times (\Z / m_n \Z)$ durch
			\[
				x \bmod \tilde m
				\mapsto
				\Vector{x \bmod m_1 & \dots & x \bmod m_n}.
			\]
			$\phi$ ist wohldefiniert, denn sei $x \equiv \tilde x \bmod \tilde m$, dann ist $x - \tilde x \equiv 0 \bmod \tilde m$ und $\tilde m \divs x - \tilde x$, also $m_i \divs x - \tilde x$ für jedes $i$ und somit $x \equiv \tilde x \bmod m_i$ für jedes $i$.
			$\phi$ ist injektiv (wieder nach \ref{1.12} d)).

			Es gilt
			\[
				\big| \Z / \tilde m \Z \big|
				= \tilde m
				= m_1 m_2 \dotsb m_n,
			\]
			andererseits ist
			\[
				\Big| \Z/ m_1 \Z \times \dotsb \times \Z / m_n \Z \Big|
				= | \Z / m_1 \Z | \cdot \dotsb \cdot | \Z / m_n \Z|
				= m_1 m_2 \dotsb m_n.
			\]
			Beide Mengen sind endlich und gleichmächtig, also ist $\phi$ surjektiv.

			Zu $b := (a_1 \bmod m_1, a_2 \bmod m_2, \dotsc, a_n \bmod m_n)$ existiert also $x \bmod \tilde m$ mit $\phi(x \bmod \tilde m) = b$.
			Dieses $x$ löst simultan die gegebenen $n$ Kongruenzen.
		\end{seg}
	\end{proof}
\end{st}

% St 1.14
\begin{st} \label{1.14}
	Seien $m_1, \dotsc, m_n \in \N$ paarweise teilerfremd.
	Dann gilt:
	\begin{enumerate}[a)]
		\item
			Es existiert ein Ringisomorphismus
			\[
				\Z / m_1m_2 \dotsb m_n \Z
				\isomorphic
				\Z /m_1 \Z \times \dotsb \times \Z / m_n \Z.
			\]
			Insbesondere sind die jeweiligen additiven zyklischen Gruppen isomorph.
		\item
			Es existiert ein Gruppenisomorphismus
			\[
				U(\Z / m_1 m_2 \dotsb m_n \Z)
				\isomorphic
				U(\Z / m_1 \Z) \times \dotsb \times U(\Z / m_n \Z).
			\]
	\end{enumerate}
	\begin{proof}
		\begin{enumerate}[a)]
			\item
				Verwende die im Beweis von \ref{1.13} konstruierte Bijektion $\phi$.
				Zeige noch, dass $\phi$ additiv, multiplikativ ist und $\phi(1) = 1$ gilt.

				Für die Additivität:
				\begin{align*}
					\phi(x \bmod \tilde m + y \bmod \tilde m)
					&= \phi\big( (x+y) \bmod \tilde m \big) \\
					&= \big(x + y \bmod m_1, \dotsc, x + y \bmod m_n \big) \\
					&= \big( x \bmod m_1, \dotsc, x \bmod m_n\big) \\
					&\quad + \big(y \bmod m_1, \dotsc, y \bmod m_n \big) \\
					&= \phi(x \bmod \tilde m) + \phi(y \bmod \tilde m).
				\end{align*}
				Die Multiplikativität verläuft analog und $\phi(1) = (1, \dotsc, 1) = 1 \in (\Z / m_1 \Z) \times \dotsb \times (\Z / m_n \Z)$.
			\item
				Isomorphe Ringe haben isomorphe Einheitengruppen und es gilt
				\[
					U(\Z / m_1 \Z \times \dotsb \times \Z / m_n \Z)
					\isomorphic U(\Z / m_1 \Z) \times \dotsb \times U(\Z / m_n \Z),
				\]
				da die Multiplikation komponentenweise auf den Faktoren operiert.
		\end{enumerate}
	\end{proof}
\end{st}

% Folg 1.15
\begin{kor} \label{1.15}
	Die Eulersche $\phi$-Funktion ist eine multiplikative zahlentheoretische Funktion.
	Es gilt außerdem für Primzahlen $p_i$ und $a_i \in \N$.
	\[
		\phi( p_1^{a_1} \dotsb p_k^{a_k} )
		=\phi( p_1^{a_1}) \dotsb \phi(p_k^{a_k})
		= \prod_{i=1}^k (p_i - 1) p_i^{a_i - 1}.
	\]
	\begin{proof}
		$\phi(1) = 1$ gilt per definitionem.
		Für $n = m_1 \dotsb m_l$ mit $m_i$ paarweise teilerfremd gilt
		\begin{align*}
			\phi(n)
			= | U(\Z / n\Z) |
			&\stack{\ref{1.14}}= |U(\Z /m _1 \Z) | \cdot \dotsb \cdot | U(\Z /m_l \Z) | \\
			&= \phi(m_1)\phi(m_2) \dotsb \phi(m_l),
		\end{align*}
		also ist $\phi$ eine multiplikative zahlentheoretische Funktion.

		Es reicht also $\phi(p^m) = p^m - p^{m-1}$ zu zeigen:
		$p^{m-1}$ sind die Anzahl der natürlichen Zahlen $\le p^m$, welche teilbar durch $p$ sind,
		also ist $p^m - p^{m-1}$ ist die Anzahl der natürlichen Zahlen $\le p^m$, welche teilerfremd zu $p^m$ sind.
	\end{proof}
\end{kor}

% St 1.16
\begin{st}[Teilbarkeitskriterien] \label{1.16}
	Sei $m \in \N$ gegeben mit Dezimaldarstellung
	\[
		m = \sum_{i=0}^n a_i \cdot 10^i,
	\]
	wobei $0 \le a_i \le 9$ und $a_n \neq 0$.
	Dann gilt
	\begin{enumerate}[a)]
		\item
			$3 \divs m \iff 3 \divs \sum_{i=0}^n a_i$,
		\item
			$9 \divs m \iff 9 \divs \sum_{i=0}^n a_i$,
		\item
			$11 \divs m \iff 11 \divs \sum_{i=0}^n (-1)^i a_i$ ($11$ teilt alternierende Quersumme).
	\end{enumerate}
	\begin{proof}
		\begin{enumerate}[a)]
			\item
				Es gilt $10^d \equiv 1 \bmod 3$ für alle $d \in \N$ und damit
				\[
					m \equiv a_n + a_{n-1} + \dotsb + a_0 \mod 3
				\]
			\item
				Analog ist $10^d \equiv 1 \bmod 9$.
			\item
				Ähnlich ist $10^d \equiv (-1)^d \bmod 11$.
		\end{enumerate}
	\end{proof}
\end{st}

