\chapter{Maß- und Integrationstheorie (MIT)}

\coursetimestamp{17}{10}{2013}

\section{Sigma-Algebren und Maße}

Sei $\Omega \neq \emptyset$.
Unser Ziel ist, für $A \in \Omega$ den Ausdruck $\P(A)$ zu definieren, d.h. wir müssen $A$ „ausmessen“.
Im Allgemeinen ist dies nur auf einem echten Teilsystem von $\scr P(\Omega)$ möglich.

\begin{df} \label{1.1.1}
	\begin{itemize}
		\item
			Ein Mengensystem $\scr F \subset \scr P(\Omega)$ heißt \emph{$\sigma$-Algebra} (über $\Omega$), falls
			\begin{enumerate}[(i)]
				\item
					$\Omega \in \scr F$,
				\item
					$A \in \scr F \implies A^C := \Omega \setminus A \in \scr F$,
				\item
					$A_1, A_2, \dotsc \in \scr F \implies \bigcup_{i=1}^\infty A_i \in \scr F$.
			\end{enumerate}
		\item
			$A \in \scr F$ heißt \emph{messbare Menge}.
		\item
			$(\Omega, \scr F)$ heißt \emph{messbarer Raum}.
	\end{itemize}
\end{df}

\begin{nt} \label{1.1.2}
	\begin{enumerate}[(a)]
		\item
			$\scr P(\Omega)$ und $\{\emptyset, \Omega\}$ sind $\sigma$-Algebren (sog. „triviale $\sigma$-Algebren“).
		\item
			Sei $\scr F$ $\sigma$-Algebra, dann gilt
			\[
				A_1, A_2, \dotsc \in \scr F \implies \bigcup_{i=1}^\infty A_i \in \scr F.
			\]
		\item
			Schnitte von $\sigma$-Algebren sind wieder $\sigma$-Algebren, Vereinigungen jedoch im Allgemeinen nicht.
	\end{enumerate}
	\begin{proof}
		\begin{enumerate}[(a)]
			\item
				Hierzu prüfe man (i-iii) aus \ref{1.1.1} nach.
			\item
				Wir schreiben
				\[
					\bigcap_{i=1}^\infty A_i
					= \bigg( \bigcup_{i=1}^\infty A_i^C \bigg)^C.
				\]
				Nach (ii) und (iii) ist damit $\bigcap_{i=1}^\infty A_i \in \scr F$.
			\item
				Übung.
		\end{enumerate}
	\end{proof}
\end{nt}

\begin{lem} \label{1.1.3}
	Sei $C \subset \scr P(\Omega)$.
	Dann ist
	\[
		\sigma(C)
		:= \bigcap \Big\{ A : \text{$A$ ist $\sigma$-Algebra über $\Omega$ mit $C \subset A$} \Big\}
	\]
	die kleinste $\sigma$-Algebra, die $C$ umfasst.
	Wir nennen diese die \emph{von $C$ erzeugte $\sigma$-Algebra}.
	\begin{proof}
		\begin{enumerate}[1)]
			\item
				$\sigma(C)$ ist $\sigma$-Algebra nach \ref{1.1.2}(c).
			\item
				$C \subset \sigma(C)$ nach Definition.
			\item
				$\forall B$ $\sigma$-Algebra mit $C \subset B$ gilt $\sigma(C) \subset B$ nach Definition.
		\end{enumerate}
	\end{proof}
\end{lem}

\begin{nt} \label{1.1.4}
	Es gelten folgende Eigenschaften:
	\begin{enumerate}[(i)]
		\item
			$C \subset \sigma(C)$,
		\item
			$C = \sigma(C) \iff C$ ist $\sigma$-Algebra,
		\item
			$C_1 \subset C_2 \implies \sigma(C_1) \subset \sigma(C_2)$.
	\end{enumerate}
\end{nt}

\begin{nt} \label{1.1.5}
	Sei $\{\scr F_i\}_{i\in I}$ eine Familie von $\sigma$-Algebren über $\Omega$, so ist wie bereits bekannt $\bigcup_{i\in I} \scr F_i$ im Allgemeinen keine $\sigma$-Algebra über $\Omega$.
	Man setzt jedoch
	\begin{align*}
		\scr F_1 \vee \scr F_2 &:= \sigma(\scr F_1 \cup \scr F_2), \\
		\bigvee_{i\in I} \scr F_i &:= \sigma\bigg( \bigcup_{i\in I} \scr F_i \bigg).
	\end{align*}
\end{nt}

\begin{ex}[Borel-$\sigma$-Algebra] \label{1.1.6}
	\begin{enumerate}[(a)]
		\item
			Setze im $\R^1$
			\[
				O^1
				:= \{ U \in \R^1 : \text{$U$ ist offen}.
			\]
			Offensichtlich ist $O^1 \subset \scr P(\R^1)$.
			Man nennt
			\[
				B^1 := \sigma(O^1)
			\]
			\emph{Borel-$\sigma$-Algebra} über $\R^1$.

			Es gibt auch andere Erzeugendensysteme für $B^1$, z.B.
			\begin{align*}
				B^1
				&= \sigma(\{(a,b) : -\infty \le a < b \le \infty\}) \\
				&= \sigma(\{(-\infty,t] : t \in \R\}) \\
				&= \sigma(\{(-\infty,t] : t \in \Q\}) \\
				&= \sigma(\{[a,b) : -\infty < a < b < \infty \}). \\
			\end{align*}
		\item
			Analog setzt man im $\R^n$
			\[
				O^n
				:= \{ U \subset \R^n : \text{$U$ ist offen}
			\]
			mit der \emph{Borel-$\sigma$-Algebra} über $\R^n$:
			\[
				B^n := \sigma(O^n).
			\]

			Es gibt auch andere Erzeugendensysteme für $B^n$:
			\begin{align*}
				B^n
				&= \sigma \bigg( \Big\{ [a_1,b_1) \times \dotsb \times [a_n,b_n] : -\infty < a < b < \infty, i = 1, \dotsc, n \Big\} \bigg) \\
				&= \sigma( \{F \subset \R^n \text{ abgeschlossen} ) \\
				&= \sigma( \{K \subset \R^n \text{ abgeschlossen} )
			\end{align*}
		\item
			In metrischen Räumen $(X,d)$ lässt sich analog eine Borel-$\sigma$-Algebra definieren.
			Setze
			\[
				B(x,r) := \{ y \in X : d(x,y) < r \},
			\]
			dann definiert man $O \subset X$ als offen genau dann, wenn
			\[
				\forall x \in O \exists \eps > 0 : B(x,\eps) \subset O.
			\]
			Setze
			\begin{align*}
				\tau_d &:= \{ O \subset X \text{ offen} \}, \\
				B(X,d) &:= \sigma(\tau_d).
			\end{align*}
		\item
			Sei $\tau$ eine Topologie, dann ist
			\[
				B := \sigma(\tau)
			\]
			eine Borel-$\sigma$-Algebra über $\tau$.
	\end{enumerate}
\end{ex}

% fixme: \scr F vs F??
\begin{nt}[Spur-$\sigma$-Algebra] \label{1.1.7}
	Sei $\scr F$ eine $\sigma$-Algebra über $\Omega$ und $E \subset \Omega$.
	Man nennt
	\[
		\scr F\Big|_E := \{ F \cap E : F \in \scr F \}
	\]
	\emph{Spur-$\sigma$-Algebra} von $\scr F$ auf $E$.
	Es gilt
	\begin{enumerate}[(i)]
		\item
			$\scr F\big|_E$ ist tatsächlich eine $\sigma$-Algebra
		\item
			Falls $E \in \scr F$, so ist
			\[
				\scr F\Big|_{E} = \{ A \in \scr F : A \subset E \}
			\]
	\end{enumerate}
	\begin{proof}
		Übung.
	\end{proof}
\end{nt}

\begin{df}[Maß, Maßraum, Wahrscheinlichkeitsraum] \label{1.1.8}
	Sei $(\Omega, \scr F)$ ein messbarer Raum.
	\begin{enumerate}[(a)]
		\item
			Eine Abbildung $\my : \scr F \to [0,\infty]$ heißt \emph{Maß}, falls
			\begin{enumerate}[(i)]
				\item
					$\my(\emptyset) = 0$,
				\item
					$\my$ ist $\sigma$-additiv, d.h.
					\[
						\my \bigg( \bigcup_{i=1}^\infty A_n \bigg)
						= \sum_{i=1}^\infty \my(A_i)
					\]
					für $A_i$ paarweise disjunkt und $A_i \subset \scr F$.
			\end{enumerate}

			Das Tripel $(\Omega, \scr F, \my)$ heißt \emph{Maßraum}.
		\item
			Ein Maß $\my$ heißt \emph{$\sigma$-endlich}, falls eine Folge $(\Omega_n)$ existiert mit
			\begin{itemize}
				\item
					$\Omega_i \subset \Omega_{i+1}$ und	$\Omega = \bigcup_{i=1}^\infty \Omega_i$
					(Man schreibt dafür auch $\Omega_i \nearrow \Omega$).
				\item
					$\Omega_i \in \scr F$ und $\my(\Omega_i) < \infty$ für alle $i \in \N$.
			\end{itemize}
		\item
			Das Maß $\my$ heißt \emph{endlich}, falls $\my(\Omega) < \infty$.
		\item
			Falls $\my(\Omega) = 1$, so heißt $\my$ \emph{Wahrscheinlichkeitsmaß}.
			Man schreibt oft auch $\P$ statt $\my$.

			Das Tripel $(\Omega, \scr F, \P)$ nennt man \emph{Wahrscheinlichkeitsraum}.
	\end{enumerate}
\end{df}

\begin{ex}[Lebesgue-Maß] \label{1.1.9}
	Sei $B^n$ die Borel-$\sigma$-Algebra über $\R^n$.
	Wir suchen nun einen „Volumenbegriff“ für Borel-Mengen.

	Wir definieren $\lambda^n$ durch
	\[
		\lambda^n \Big( [a_1,b_1) \times \dotsb \times [a_n,b_n) \Big)
		:= \prod_{i=1}^n (b_i-a_i)
	\]
	wobei $-\infty < a_i < b_i < \infty$ für alle $i = 1, \dotsc, n$.
	Dies legt $\lambda^n$ auf einem Erzeugendensystem von $B^n$ fest.
	% fixme: ref
	Später werden wir sehen, das sich $\lambda^n$ zu einem Maß auf $B^n$ fortsetzen lässt.
\end{ex}

\begin{lem}[Eigenschaften von Maßen] \label{1.1.10}
	Sei $(\Omega, \scr F, \my)$ ein Maßraum.
	Dann gelten folgende Eigenschaften
	\begin{enumerate}[(a)]
		\item
			Falls $A_n \nearrow A$, $A_n \in \scr F$ ($\forall n\in \N$), so ist $A \in \scr F$ und
			\[
				\lim_{n\to \infty} \my (A_n) = \my(A).
			\]
			Man sagt „$\my$ ist stetig von unten“.
		\item
			Falls $A_n \searrow A$, $A_n \in \scr F$ ($\forall n\in \N$) \emph{und} $\my(A_n) < \infty$ für ein $n \in \N$ so ist $A \in \scr F$ und
			\[
				\lim_{n\to \infty} \my(A_n) = \my(A).
			\]
			Man sagt „$\my$ ist stetig von oben“.
		\item
			Es gilt für $A_i \in \scr F$ ($i \in \N$) die sogenannte \emph{$\sigma$-Subadditivität}
			\[
				\my \bigg( \bigcup_{i=1}^\infty A_i \bigg)
				\le \sum_{i=1}^\infty \my (A_i).
			\]
		\item
			Für $A, B \in \scr F$ gilt die \emph{Monotonität}
			\[
				A \subset B
				\implies
				\my(A) \le \my(B).
			\]
	\end{enumerate}
\coursetimestamp{23}{10}{2013}
	\begin{proof}
		\begin{enumerate}[(a)]
			\item
				Sei $A_n \nearrow A$, d.h. $A_n \subset A_{n+1}$ und $\bigcup_{n}A_n = \sup_n A_n = A$.
				Da $A_n \in \scr F$ für alle $n$, gilt $A = \bigcup_n A_n \in \scr F$ (also $\my(A)$ definiert).
				Es bleibt zu zeigen, dass $\lim_{n\to \infty} \my(A_n) = \my(A)$.

				Setze $A_0 = \emptyset$ und $B_n = A_n \setminus A_{n-1}$ für $n \ge 1$.
				Dann gelten
				\begin{enumerate}[(i)]
					\item
						$(B_n)_{n\in\N}$ sind paaweise disjunkt per Konstruktion
					\item
						$B_n \in \scr F$ für alle $n$ (denn $A_{n-1},A_n \in \scr F \implies A_n, A_{n-1} \in \scr F \implies A_n \cap A_{n-1}^C = A_n \setminus A_{n-1} \in \scr F$)
					\item
						$\bigcup_{n=1}^\infty B_n = A$, denn
						\[
							\bigcup_{n=1}^\infty B_n = \bigcup_{n=1}^\infty (A_n \setminus A_{n-1}) = \bigcup_{n=1}^\infty A_n = A.
						\]
				\end{enumerate}
				Es folgt
				\[
					\my(A)
					= \my(\bigcup_{n=1}^\infty B_n)
					= \sum_{n=1}^\infty \my(B_n)
					= \lim_{m\to\infty} \sum_{n=1}^m \my(B_n)
					= \lim_{m\to \infty} \my(\bigcup_{n=1}^m B_n)
					= \lim_{m\to \infty} \my(A_m)
				\]
			\item
				Funktioniert ähnlich wie, oder mit (a).
			\item
				Zeige zunächst
				\[
					\forall n \in \N : \my(\bigcup_{i=1}^n A_i) \le \sum_{i=1}^n \my(A_i)
				\]
				per vollständiger Induktion (Übung).

				Es gilt $\bigcup_{i=1}^n A_i \nearrow \bigcup_{i=1}^\infty A_i$, also
				\[
					\my(\bigcup_{i=1}^\infty A_i)
					= \lim_{n\to \infty} \my(\bigcup_{i=1}^n A_i)
					\le \lim_{n\to \infty} \sum_{i=1}^n \my(A_i)
					= \sum_{i=1}^\infty \my (A_i).
				\]
			\item
				Schreibe $B = A \cupdot (B \setminus A)$, dann ist
				\[
					\my(B) = \my(A) + \underbrace{\my(B\setminus A)}_{\ge 0} \ge \my(A).
				\]
		\end{enumerate}
	\end{proof}
\end{lem}

\begin{df} \label{1.1.11}
	Sei $(\Omega, \scr F, \P)$ ein Wahrscheinlichkeitsraum und $(A_i)_{i\in\N}$ eine Folge messbarer Mengen in $\scr F$.
	Definiere
	\begin{align*}
		\limsup_{n\to\infty} A_n
			&:= \bigcap_{n=1}^\infty \bigcup_{k=n}^\infty A_k, \\
		\liminf_{n\to\infty} A_n
			&:= \bigcup_{n=1}^\infty \bigcap_{k=n}^\infty A_k.
	\end{align*}
\end{df}

\begin{nt} \label{1.1.12}
	Wir versuchen \ref{1.1.11} zu interpretieren:
	\begin{align*}
		\limsup_{n\to\infty} A_n
			&= \Big\{ \omega \in \Omega : \forall n \in \N \exists k \ge n : \omega \in A_k \Big\} \\
			&= \Big\{ \omega \in \Omega : \omega \in A_k \text{ für unendlich viele $k \ge 1$} \Big\}, \\
		\liminf_{n\to\infty} A_n
			&= \Big\{ \omega \in \Omega : \exists n \in \N \forall k \ge n : \omega \in A_k \Big\} \\
			&= \Big\{ \omega \in \Omega : \omega \in A_k \text{ für fast alle $k$}.
	\end{align*}
\end{nt}

\begin{st}[I. Lemma von Borel-Cantelli] \label{1.1.13}
	Sei $(\Omega, \scr F, \P)$ ein Wahrscheinlichkeitsraum und $(A_i)_{i\in\N}$ eine Folge messbarer Mengen in $\scr F$.
	Dann gilt
	\[
		\sum_{n=1}^\infty \P(A_n) < \infty
		\implies
		\P(\limsup_{n\to \infty} A_n) = 0.
	\]
	\begin{nt*}[Interpretation]
		Wenn die Ereignisse $(A_n)$ „hinreichend disjunkt“ sind, so liegen fast alle $\omega \in \Omega$ in nur endlich vielen der $(A_n)$.
	\end{nt*}
	\begin{proof}
		Setze $B_n := \bigcup_{k \ge n} A_k$, dann ist $B_n \searrow B = \bigcap_{n=1}^\infty B_n$, d.h.
		\[
			B
			= \bigcap_{n=1}^\infty \bigcup_{k\ge n} A_k
			= \limsup_{n\to\infty} A_n
		\]
		und
		\[
			\P(\limsup_{n} A_n)
			= \P(B)
			\le \P(B_j)
		\]
		für alle $j \in \N$.
		Also gilt für beliebige $n \in \N$.
		\[
			\P(B)
			\le \P(B_n)
			\le \underbrace{\sum_{k=n}^\infty \P(A_k)}_{\to 0 (n\to\infty)}.
		\]
		also ist $\P(B_n) \to 0$ und somit wegen $\P(B) \le \P(B_n)$ für alle $n\in \N$, gilt $\P(B) = 0$.
	\end{proof}
\end{st}

\begin{ex}[Dirac-Maß] \label{1.1.14}
	Sei $\Omega \neq \emptyset$.
	Fixiere $x \in \Omega$.
	Setze für $A \in \Omega$
	\[
		\delta_x(A) := \Ind_A(x) = \begin{cases}
			1 & x \in A \\
			0 & x \not\in A
		\end{cases}.
	\]
	dann ist
	\begin{align*}
		\delta_0([-1,1]) &= 1 &
		\delta_0(\R) &= 1 & \\
		\delta_0((0,\infty)) &= 0 &
		\delta_0(\{0\}) &= 1
	\end{align*}
	$\delta_0$ heißt \emph{Dirac-Maß in $x$}.

	$\delta_0$ ist ein Wahrscheinlichkeitsmaß auf $(\Omega, \scr P(\Omega))$.
	\begin{proof}
		Siehe $\ref{1.1.8}$ für die zu prüfenden Eigenschaften.
		Zunächst ist offensichtlich $(\Omega, \scr P(\Omega))$ ein $\sigma$-Algebra.
		Außerdem gilt
		\begin{enumerate}[(a)]
			\item
				$\delta_x (\emptyset) = 0$, da $x \not\in \emptyset$,
			\item
				Seien $(A_n)$ paarweise disjunkt, $A_n \subset \Omega$.
				Für $x \not\in A_n$ für alle $n$ ist die Aussage klar, sei also $x \in A_j$ für ein $j \in \N$.
				Da die $A_n$ paarweise disjunkt, gilt $\sum_{n=1}^\infty \delta_x (A_n) = 1 = \delta_x (\bigcup_{n=1}^\infty A_n)$.
			\item
				$\delta_x(\Omega) = 1$.
		\end{enumerate}
	\end{proof}
\end{ex}

\section{Einschub: Wahrscheinlichkeitstheorie I., Diskrete Wahrscheinlichkeitsräume}

Sei $\Omega \neq \emptyset$ und $p: \Omega \to [0,1]$, wobei $\Omega$ höchstens Abzählbar und $\sum_{\omega \in \Omega} p(\omega) == 1$.
Für $A \subset \Omega$ setzen wir
\[
	\P(A) := \sum_{\omega \in A} p(\omega).
\]
Dann ist $\P: \scr P(\Omega) \to [0,1]$ ein Wahrscheinlichkeitsmaß (Übung: nachprüfen).

\begin{ex}[Laplace'scher Wahrscheinlichkeitsraum] \label{1.2.1}
	Sei $\# \Omega < \infty$ und für alle $\omega$
	\[
		p(\omega) = \P(\{\omega\}) = \f 1{\# \Omega}.
	\]
	Dann ist für $A \subset \Omega$
	\[
		\P(A)
		= \sum_{\omega \in A} p(\omega)
		= \sum_{\omega \not\in A} \f 1{\Omega}
		= \f 1{\# \Omega} \sum_{\omega \in A} 1
		= \f {\# A}{\# \Omega}.
	\]
	Man nennt $\P$ \emph{diskrete Gleichverteilung}.
\end{ex}

\begin{ex}[Bernoulli-Verteilung ($\Bin_{1,p}$)] \label{1.2.1}
	Sei $p \in [0,1]$ ein fester Parameter.
	Ein einmaliges Experiment
	\begin{align*}
		\P(\{1\}) &= \P(\text{„Erfolg“}) = p. \\
		\P(\{0\}) &= \P(\text{„Misserfolg“}) = 1 -p.
	\end{align*}
	Mit $\Omega = \{0,1\}$ und $p_k = \P({k})$, $k \in \Omega$ ist dies ein diskreter Wahrscheinlichkeitsraum.
\end{ex}

\begin{ex}[Binomial-Verteilung ($\Bin_{n,p}$)] \label{1.2.3}
	Ein Bernoulli-Experiment $n$-mal ausgeführt, jeweils unabhängig voneinander.
	\begin{align*}
		\P(\text{„genau $k$ Erfolge“}) = \binom{n}{k} p^k (1-p)^{n-k}
	\end{align*}
	$\P$ ist tatsächlich eine Verteilung, d.h.
	\begin{enumerate}[(i)]
		\item
			$p_k \in [0,1]$
		\item
			$\sum_{k=0}^n p_k = 1$
	\end{enumerate}
	\begin{proof}
		Zeige zunächst (ii):
		\[
			\sum_{k=0}^n p_k
			= \sum_{k=0}^n \binom{n}{k} p^k (1-p)^{n-k}
			= (p + (1-p))^n
			= 1.
		\]
		Es gilt $p_k \ge 0$ und damit mit (ii) $p_k \le 1$.
	\end{proof}
\end{ex}

\begin{ex}[Hypergeometrische Verteilung ($\Hyp_{n;S,W}$)] \label{1.2.4}
	Betrachte eine Urne mit $S$ schwarzen und $W$ weißen Kugeln.
	Ziehe $n \le S + W$ Kügeln ohne Zurücklegen.

	Wie hoch ist die Wahrscheinlichkeit, genau $s$ schwarze (und somit $n-s =w$ weiße) Kugeln zu ziehen?

	Zunächst ist $s \le S, s \ge 0, w \le W, s \le n$, oder statt $w \le W$, $s \ge n - W$.
	\[
		\max \{0, (n-w)\} \le s \le \min\{ n, S \}.
	\]
	Es ist
	\[
		\P(\{s\})
		= \f {\binom{S}{s} \binom{W}{n-s}}{\binom{S+W}{n}}
	\]
	eine Wahrscheinlichkeitsverteilung (der Nachweis ist kompliziert).
\end{ex}

\begin{ex*}[Skat]
	Wir suchen $\P(\text{3 Buben beim Geber})$ und wenden dazu \ref{1.2.4} an.
	\begin{align*}
		S &= \#\text{Buben} = 4 &
		W &= \#\text{Nicht-Buben} = 28
	\end{align*}
	Berechne
	\[
		\Hyp_{10;4,28}(3)
		= \f {\binom 43 \binom{28}{3}}{\binom {32}{10}} = \f {66}{899} \approx 7,3\%.
	\]
\end{ex*}
