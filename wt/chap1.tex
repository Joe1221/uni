\chapter{Maß- und Integrationstheorie (MIT)}

\coursetimestamp{17}{10}{2013}

\section{Sigma-Algebren und Maße}

Sei $\Omega \neq \emptyset$.
Unser Ziel ist, für $A \in \Omega$ den Ausdruck $\P(A)$ zu definieren, d.h. wir müssen $A$ „ausmessen“.
Im Allgemeinen ist dies nur auf einem echten Teilsystem von $P(\Omega)$ möglich.

\begin{df} \label{1.1.1}
	\begin{itemize}
		\item
			Ein Mengensystem $F \subset \P(\Omega)$ heißt \emph{$\sigma$-Algebra} (über $\Omega$), falls
			\begin{enumerate}[(i)]
				\item
					$\Omega \in F$,
				\item
					$A \in F \implies A^C := \Omega \setminus A \in F$,
				\item
					$A_1, A_2, \dotsc \in F \implies \bigcup_{i=1}^\infty A_i \in F$.
			\end{enumerate}
		\item
			$A \in F$ heißt \emph{messbare Menge}.
		\item
			$(\Omega, F)$ heißt \emph{messbarer Raum}.
	\end{itemize}
\end{df}

\begin{nt} \label{1.1.2}
	\begin{enumerate}[(a)]
		\item
			$P(\Omega)$ und $\{\emptyset, \Omega\}$ sind $\sigma$-Algebren (sog. „triviale $\sigma$-Algebren“).
		\item
			Sei $F$ $\sigma$-Algebra, dann gilt
			\[
				A_1, A_2, \dotsc \in F \implies \bigcup_{i=1}^\infty A_i \in F.
			\]
		\item
			Schnitte von $\sigma$-Algebren sind wieder $\sigma$-Algebren, Vereinigungen jedoch im Allgemeinen nicht.
	\end{enumerate}
	\begin{proof}
		\begin{enumerate}[(a)]
			\item
				Hierzu prüfe man (i-iii) aus \ref{1.1.1} nach.
			\item
				Wir schreiben
				\[
					\bigcap_{i=1}^\infty A_i
					= \bigg( \bigcup_{i=1}^\infty A_i^C \bigg)^C.
				\]
				Nach (ii) und (iii) ist damit $\bigcap_{i=1}^\infty A_i \in F$.
			\item
				Übung.
		\end{enumerate}
	\end{proof}
\end{nt}

\begin{lem} \label{1.1.3}
	Sei $C \subset P(\Omega)$.
	Dann ist
	\[
		\sigma(C)
		:= \bigcap \Big\{ A : \text{$A$ ist $\sigma$-Algebra über $\Omega$ mit $C \subset A$} \Big\}
	\]
	die kleinste $\sigma$-Algebra, die $C$ umfasst.
	Wir nennen diese die \emph{von $C$ erzeugte $\sigma$-Algebra}.
	\begin{proof}
		\begin{enumerate}[1)]
			\item
				$\sigma(C)$ ist $\sigma$-Algebra nach \ref{1.1.2}(c).
			\item
				$C \subset \sigma(C)$ nach Definition.
			\item
				$\forall B$ $\sigma$-Algebra mit $C \subset B$ gilt $\sigma(C) \subset B$ nach Definition.
		\end{enumerate}
	\end{proof}
\end{lem}

\begin{nt} \label{1.1.4}
	Es gelten folgende Eigenschaften:
	\begin{enumerate}[(i)]
		\item
			$C \subset \sigma(C)$,
		\item
			$C = \sigma(C) \iff C$ ist $\sigma$-Algebra,
		\item
			$C_1 \subset C_2 \implies \sigma(C_1) \subset \sigma(C_2)$.
	\end{enumerate}
\end{nt}

\begin{nt} \label{1.1.5}
	Sei $\{F_i\}_{i\in I}$ eine Familie von $\sigma$-Algebren über $\Omega$, so ist wie bereits bekannt $\bigcup_{i\in I} F_i$ im Allgemeinen keine $\sigma$-Algebra über $\Omega$.
	Man setzt jedoch
	\begin{align*}
		F_1 \vee F_2 &:= \sigma(F_1 \cup F_2), \\
		\bigvee_{i\in I} F_i &:= \sigma\bigg( \bigcup_{i\in I} F_i \bigg).
	\end{align*}
\end{nt}

\begin{ex}[Borel-$\sigma$-Algebra] \label{1.1.6}
	\begin{enumerate}[(a)]
		\item
			Setze im $\R^1$
			\[
				O^1
				:= \{ U \in \R^1 : \text{$U$ ist offen}.
			\]
			Offensichtlich ist $O^1 \subset P(\R^1)$.
			Man nennt
			\[
				B^1 := \sigma(O^1)
			\]
			\emph{Borel-$\sigma$-Algebra} über $\R^1$.

			Es gibt auch andere Erzeugendensysteme für $B^1$, z.B.
			\begin{align*}
				B^1
				&= \sigma(\{(a,b) : -\infty \le a < b \le \infty\}) \\
				&= \sigma(\{(-\infty,t] : t \in \R\}) \\
				&= \sigma(\{(-\infty,t] : t \in \Q\}) \\
				&= \sigma(\{[a,b) : -\infty < a < b < \infty \}). \\
			\end{align*}
		\item
			Analog setzt man im $\R^n$
			\[
				O^n
				:= \{ U \subset \R^n : \text{$U$ ist offen}
			\]
			mit der \emph{Borel-$\sigma$-Algebra} über $\R^n$:
			\[
				B^n := \sigma(O^n).
			\]

			Es gibt auch andere Erzeugendensysteme für $B^n$:
			\begin{align*}
				B^n
				&= \sigma \bigg( \Big\{ [a_1,b_1) \times \dotsb \times [a_n,b_n] : -\infty < a < b < \infty, i = 1, \dotsc, n \Big\} \bigg) \\
				&= \sigma( \{F \subset \R^n \text{ abgeschlossen} ) \\
				&= \sigma( \{K \subset \R^n \text{ abgeschlossen} )
			\end{align*}
		\item
			In metrischen Räumen $(X,d)$ lässt sich analog eine Borel-$\sigma$-Algebra definieren.
			Setze
			\[
				B(x,r) := \{ y \in X : d(x,y) < r \},
			\]
			dann definiert man $O \subset X$ als offen genau dann, wenn
			\[
				\forall x \in O \exists \eps > 0 : B(x,\eps) \subset O.
			\]
			Setze
			\begin{align*}
				\tau_d &:= \{ O \subset X \text{ offen} \}, \\
				B(X,d) &:= \sigma(\tau_d).
			\end{align*}
		\item
			Sei $\tau$ eine Topologie, dann ist
			\[
				B := \sigma(\tau)
			\]
			eine Borel-$\sigma$-Algebra über $\tau$.
	\end{enumerate}
\end{ex}

% fixme: \scr F vs F??
\begin{nt}[Spur-$\sigma$-Algebra] \label{1.1.7}
	Sei $F$ eine $\sigma$-Algebra über $\Omega$ und $E \subset \Omega$.
	Man nennt
	\[
		\scr F\Big|_E := \{ F \cap E := F \in \scr F \}
	\]
	\emph{Spur-$\sigma$-Algebra} von $\scr F$ auf $E$.
	Es gilt
	\begin{enumerate}[(i)]
		\item
			$\scr F\big|_E$ ist tatsächlich eine $\sigma$-Algebra
		\item
			Falls $E \in \scr F$, so ist
			\[
				\scr F\Big|_{E} = \{ A \in \scr F : A \subset E \}
			\]
	\end{enumerate}
	\begin{proof}
		Übung.
	\end{proof}
\end{nt}

\begin{df}[Maß, Maßraum, Wahrscheinlichkeitsraum] \label{1.1.8}
	Sei $(\Omega, \scr F)$ ein messbarer Raum.
	\begin{enumerate}[(a)]
		\item
			Eine Abbildung $\my : \scr F \to [0,\infty]$ heißt \emph{Maß}, falls
			\begin{enumerate}[(i)]
				\item
					$\my(\emptyset) = 0$,
				\item
					$\my$ ist $\sigma$-additiv, d.h.
					\[
						\my \bigg( \bigcup_{i=1}^\infty A_n \bigg)
						= \sum_{i=1}^\infty \my(A_i)
					\]
					für $A_i$ paarweise disjunkt und $A_i \subset \scr F$.
			\end{enumerate}

			Das Tripel $(\Omega, \scr F, \my)$ heißt \emph{Maßraum}.
		\item
			Ein Maß $\my$ heißt \emph{$\sigma$-endlich}, falls eine Folge $(\Omega_n)$ existiert mit
			\begin{itemize}
				\item
					$\Omega_i \subset \Omega_{i+1}$ und	$\Omega = \bigcup_{i=1}^\infty \Omega_i$
					(Man schreibt dafür auch $\Omega_i \nearrow \Omega$).
				\item
					$\Omega_i \in F$ und $\my(\Omega_i) < \infty$ für alle $i \in \N$.
			\end{itemize}
		\item
			Das Maß $\my$ heißt \emph{endlich}, falls $\my(\Omega) < \infty$.
		\item
			Falls $\my(\Omega) = 1$, so heißt $\my$ \emph{Wahrscheinlichkeitsmaß}.
			Man schreibt oft auch $\P$ statt $\my$.

			Das Tripel $(\Omega, \scr F, \P)$ nennt man \emph{Wahrscheinlichkeitsraum}.
	\end{enumerate}
\end{df}

\begin{ex}[Lebesgue-Maß] \label{1.1.9}
	Sei $B^n$ die Borel-$\sigma$-Algebra über $\R^n$.
	Wir suchen nun einen „Volumenbegriff“ für Borel-Mengen.

	Wir definieren $\lambda^n$ durch
	\[
		\lambda^n \Big( [a_1,b_1) \times \dotsb \times [a_n,b_n) \Big)
		:= \prod_{i=1}^n (b_i-a_i)
	\]
	wobei $-\infty < a_i < b_i < \infty$ für alle $i = 1, \dotsc, n$.
	Dies legt $\lambda^n$ auf einem Erzeugendensystem von $B^n$ fest.
	% fixme: ref
	Später werden wir sehen, das sich $\lambda^n$ zu einem Maß auf $B^n$ fortsetzen lässt.
\end{ex}

\begin{lem}[Eigenschaften von Maßen] \label{1.1.10}
	Sei $(\Omega, \scr F, \my)$ ein Maßraum.
	Dann gelten folgende Eigenschaften
	\begin{enumerate}[(a)]
		\item
			Falls $A_n \nearrow A$, $A_n \in \scr F$ ($\forall n\in \N$), so ist $A \in \scr F$ und
			\[
				\lim_{n\to \infty} \my (A_n) = \my(A).
			\]
			Man sagt „$\my$ ist stetig von unten“.
		\item
			Falls $A_n \searrow A$, $A_n \in \scr F$ ($\forall n\in \N$) \emph{und} $\my(A_n) < \infty$ für ein $n \in \N$ so ist $A \in \scr F$ und
			\[
				\lim_{n\to \infty} \my(A_n) = \my(A).
			\]
			Man sagt „$\my$ ist stetig von oben“.
		\item
			Es gilt für $A_i \in \scr F$ ($i \in \N$) die sogenannte \emph{$\sigma$-Subadditivität}
			\[
				\my \bigg( \bigcup_{i=1}^\infty A_i \bigg)
				\le \sum_{i=1}^\infty \my (A_i).
			\]
		\item
			Für $A, B \in \scr F$ gilt die \emph{Monotonität}
			\[
				A \subset B
				\implies
				\my(A) \le \my(B).
			\]
	\end{enumerate}
	\begin{proof}
		\begin{enumerate}[(a)]
			\item
			\item
			\item
			\item
				Schreibe $B = A \cupdot (B \setminus A)$, dann ist
				\[
					\my(B) = \my(A) + \underbrace{\my(B\setminus A)}_{\ge 0} \ge \my(A).
				\]
		\end{enumerate}
	\end{proof}

\end{lem}
