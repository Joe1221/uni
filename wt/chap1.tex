\chapter{Maß- und Integrationstheorie (MIT)}

\coursetimestamp{17}{10}{2013}

\section{Sigma-Algebren und Maße}

Sei $\Omega \neq \emptyset$.
Unser Ziel ist, für $A \in \Omega$ den Ausdruck $\P(A)$ zu definieren, d.h. wir müssen $A$ „ausmessen“.
Im Allgemeinen ist dies nur auf einem echten Teilsystem von $\scr P(\Omega)$ möglich.

\begin{df} \label{1.1.1}
	\begin{itemize}
		\item
			Ein Mengensystem $\scr F \subset \scr P(\Omega)$ heißt \emph{$\sigma$-Algebra} (über $\Omega$), falls
			\begin{enumerate}[(i)]
				\item
					$\Omega \in \scr F$,
				\item
					$A \in \scr F \implies A^C := \Omega \setminus A \in \scr F$,
				\item
					$A_1, A_2, \dotsc \in \scr F \implies \bigcup_{i=1}^\infty A_i \in \scr F$.
			\end{enumerate}
		\item
			$A \in \scr F$ heißt \emph{messbare Menge}.
		\item
			$(\Omega, \scr F)$ heißt \emph{messbarer Raum}.
	\end{itemize}
\end{df}

\begin{nt} \label{1.1.2}
	\begin{enumerate}[(a)]
		\item
			$\scr P(\Omega)$ und $\{\emptyset, \Omega\}$ sind $\sigma$-Algebren (sog. „triviale $\sigma$-Algebren“).
		\item
			Sei $\scr F$ $\sigma$-Algebra, dann gilt
			\[
				A_1, A_2, \dotsc \in \scr F \implies \bigcup_{i=1}^\infty A_i \in \scr F.
			\]
		\item
			Schnitte von $\sigma$-Algebren sind wieder $\sigma$-Algebren, Vereinigungen jedoch im Allgemeinen nicht.
	\end{enumerate}
	\begin{proof}
		\begin{enumerate}[(a)]
			\item
				Hierzu prüfe man (i-iii) aus \ref{1.1.1} nach.
			\item
				Wir schreiben
				\[
					\bigcap_{i=1}^\infty A_i
					= \bigg( \bigcup_{i=1}^\infty A_i^C \bigg)^C.
				\]
				Nach (ii) und (iii) ist damit $\bigcap_{i=1}^\infty A_i \in \scr F$.
			\item
				Übung.
		\end{enumerate}
	\end{proof}
\end{nt}

\begin{lem} \label{1.1.3}
	Sei $C \subset \scr P(\Omega)$.
	Dann ist
	\[
		\sigma(C)
		:= \bigcap \Big\{ A : \text{$A$ ist $\sigma$-Algebra über $\Omega$ mit $C \subset A$} \Big\}
	\]
	die kleinste $\sigma$-Algebra, die $C$ umfasst.
	Wir nennen diese die \emph{von $C$ erzeugte $\sigma$-Algebra}.
	\begin{proof}
		\begin{enumerate}[1)]
			\item
				$\sigma(C)$ ist $\sigma$-Algebra nach \ref{1.1.2}(c).
			\item
				$C \subset \sigma(C)$ nach Definition.
			\item
				$\forall B$ $\sigma$-Algebra mit $C \subset B$ gilt $\sigma(C) \subset B$ nach Definition.
		\end{enumerate}
	\end{proof}
\end{lem}

\begin{nt} \label{1.1.4}
	Es gelten folgende Eigenschaften:
	\begin{enumerate}[(i)]
		\item
			$C \subset \sigma(C)$,
		\item
			$C = \sigma(C) \iff C$ ist $\sigma$-Algebra,
		\item
			$C_1 \subset C_2 \implies \sigma(C_1) \subset \sigma(C_2)$.
	\end{enumerate}
\end{nt}

\begin{nt} \label{1.1.5}
	Sei $\{\scr F_i\}_{i\in I}$ eine Familie von $\sigma$-Algebren über $\Omega$, so ist wie bereits bekannt $\bigcup_{i\in I} \scr F_i$ im Allgemeinen keine $\sigma$-Algebra über $\Omega$.
	Man setzt jedoch
	\begin{align*}
		\scr F_1 \vee \scr F_2 &:= \sigma(\scr F_1 \cup \scr F_2), \\
		\bigvee_{i\in I} \scr F_i &:= \sigma\bigg( \bigcup_{i\in I} \scr F_i \bigg).
	\end{align*}
\end{nt}

\begin{ex}[Borel-$\sigma$-Algebra] \label{1.1.6}
	\begin{enumerate}[(a)]
		\item
			Setze im $\R^1$
			\[
				O^1
				:= \{ U \in \R^1 : \text{$U$ ist offen}.
			\]
			Offensichtlich ist $O^1 \subset \scr P(\R^1)$.
			Man nennt
			\[
				B^1 := \sigma(O^1)
			\]
			\emph{Borel-$\sigma$-Algebra} über $\R^1$.

			Es gibt auch andere Erzeugendensysteme für $B^1$, z.B.
			\begin{align*}
				B^1
				&= \sigma(\{(a,b) : -\infty \le a < b \le \infty\}) \\
				&= \sigma(\{(-\infty,t] : t \in \R\}) \\
				&= \sigma(\{(-\infty,t] : t \in \Q\}) \\
				&= \sigma(\{[a,b) : -\infty < a < b < \infty \}). \\
			\end{align*}
		\item
			Analog setzt man im $\R^n$
			\[
				O^n
				:= \{ U \subset \R^n : \text{$U$ ist offen}
			\]
			mit der \emph{Borel-$\sigma$-Algebra} über $\R^n$:
			\[
				B^n := \sigma(O^n).
			\]

			Es gibt auch andere Erzeugendensysteme für $B^n$:
			\begin{align*}
				B^n
				&= \sigma \bigg( \Big\{ [a_1,b_1) \times \dotsb \times [a_n,b_n] : -\infty < a < b < \infty, i = 1, \dotsc, n \Big\} \bigg) \\
				&= \sigma( \{F \subset \R^n \text{ abgeschlossen} ) \\
				&= \sigma( \{K \subset \R^n \text{ abgeschlossen} )
			\end{align*}
		\item
			In metrischen Räumen $(X,d)$ lässt sich analog eine Borel-$\sigma$-Algebra definieren.
			Setze
			\[
				B(x,r) := \{ y \in X : d(x,y) < r \},
			\]
			dann definiert man $O \subset X$ als offen genau dann, wenn
			\[
				\forall x \in O \exists \eps > 0 : B(x,\eps) \subset O.
			\]
			Setze
			\begin{align*}
				\tau_d &:= \{ O \subset X \text{ offen} \}, \\
				B(X,d) &:= \sigma(\tau_d).
			\end{align*}
		\item
			Sei $\tau$ eine Topologie, dann ist
			\[
				B := \sigma(\tau)
			\]
			eine Borel-$\sigma$-Algebra über $\tau$.
	\end{enumerate}
\end{ex}

% fixme: \scr F vs F??
\begin{nt}[Spur-$\sigma$-Algebra] \label{1.1.7}
	Sei $\scr F$ eine $\sigma$-Algebra über $\Omega$ und $E \subset \Omega$.
	Man nennt
	\[
		\scr F\Big|_E := \{ F \cap E : F \in \scr F \}
	\]
	\emph{Spur-$\sigma$-Algebra} von $\scr F$ auf $E$.
	Es gilt
	\begin{enumerate}[(i)]
		\item
			$\scr F\big|_E$ ist tatsächlich eine $\sigma$-Algebra
		\item
			Falls $E \in \scr F$, so ist
			\[
				\scr F\Big|_{E} = \{ A \in \scr F : A \subset E \}
			\]
	\end{enumerate}
	\begin{proof}
		Übung.
	\end{proof}
\end{nt}

\begin{df}[Maß, Maßraum, Wahrscheinlichkeitsraum] \label{1.1.8}
	Sei $(\Omega, \scr F)$ ein messbarer Raum.
	\begin{enumerate}[(a)]
		\item
			Eine Abbildung $\my : \scr F \to [0,\infty]$ heißt \emph{Maß}, falls
			\begin{enumerate}[(i)]
				\item
					$\my(\emptyset) = 0$,
				\item
					$\my$ ist $\sigma$-additiv, d.h.
					\[
						\my \bigg( \bigcup_{i=1}^\infty A_n \bigg)
						= \sum_{i=1}^\infty \my(A_i)
					\]
					für $A_i$ paarweise disjunkt und $A_i \subset \scr F$.
			\end{enumerate}

			Das Tripel $(\Omega, \scr F, \my)$ heißt \emph{Maßraum}.
		\item
			Ein Maß $\my$ heißt \emph{$\sigma$-endlich}, falls eine Folge $(\Omega_n)$ existiert mit
			\begin{itemize}
				\item
					$\Omega_i \subset \Omega_{i+1}$ und	$\Omega = \bigcup_{i=1}^\infty \Omega_i$
					(Man schreibt dafür auch $\Omega_i \nearrow \Omega$).
					%fixme: extra notation
				\item
					$\Omega_i \in \scr F$ und $\my(\Omega_i) < \infty$ für alle $i \in \N$.
			\end{itemize}
		\item
			Das Maß $\my$ heißt \emph{endlich}, falls $\my(\Omega) < \infty$.
		\item
			Falls $\my(\Omega) = 1$, so heißt $\my$ \emph{Wahrscheinlichkeitsmaß}.
			Man schreibt oft auch $\P$ statt $\my$.

			Das Tripel $(\Omega, \scr F, \P)$ nennt man \emph{Wahrscheinlichkeitsraum}.
	\end{enumerate}
\end{df}

\begin{ex}[Lebesgue-Maß] \label{1.1.9}
	Sei $B^n$ die Borel-$\sigma$-Algebra über $\R^n$.
	Wir suchen nun einen „Volumenbegriff“ für Borel-Mengen.

	Wir definieren $\lambda^n$ durch
	\[
		\lambda^n \Big( [a_1,b_1) \times \dotsb \times [a_n,b_n) \Big)
		:= \prod_{i=1}^n (b_i-a_i)
	\]
	wobei $-\infty < a_i < b_i < \infty$ für alle $i = 1, \dotsc, n$.
	Dies legt $\lambda^n$ auf einem Erzeugendensystem von $B^n$ fest.
	% fixme: ref
	Später werden wir sehen, das sich $\lambda^n$ zu einem Maß auf $B^n$ fortsetzen lässt.
\end{ex}

\begin{lem}[Eigenschaften von Maßen] \label{1.1.10}
	Sei $(\Omega, \scr F, \my)$ ein Maßraum.
	Dann gelten folgende Eigenschaften
	\begin{enumerate}[(a)]
		\item
			Falls $A_n \nearrow A$, $A_n \in \scr F$ ($\forall n\in \N$), so ist $A \in \scr F$ und
			\[
				\lim_{n\to \infty} \my (A_n) = \my(A).
			\]
			Man sagt „$\my$ ist stetig von unten“.
		\item
			Falls $A_n \searrow A$, $A_n \in \scr F$ ($\forall n\in \N$) \emph{und} $\my(A_n) < \infty$ für ein $n \in \N$ so ist $A \in \scr F$ und
			\[
				\lim_{n\to \infty} \my(A_n) = \my(A).
			\]
			Man sagt „$\my$ ist stetig von oben“.
		\item
			Es gilt für $A_i \in \scr F$ ($i \in \N$) die sogenannte \emph{$\sigma$-Subadditivität}
			\[
				\my \bigg( \bigcup_{i=1}^\infty A_i \bigg)
				\le \sum_{i=1}^\infty \my (A_i).
			\]
		\item
			Für $A, B \in \scr F$ gilt die \emph{Monotonität}
			\[
				A \subset B
				\implies
				\my(A) \le \my(B).
			\]
	\end{enumerate}
\coursetimestamp{23}{10}{2013}
	\begin{proof}
		\begin{enumerate}[(a)]
			\item
				Sei $A_n \nearrow A$, d.h. $A_n \subset A_{n+1}$ und $\bigcup_{n}A_n = \sup_n A_n = A$.
				Da $A_n \in \scr F$ für alle $n$, gilt $A = \bigcup_n A_n \in \scr F$ (also $\my(A)$ definiert).
				Es bleibt zu zeigen, dass $\lim_{n\to \infty} \my(A_n) = \my(A)$.

				Setze $A_0 = \emptyset$ und $B_n = A_n \setminus A_{n-1}$ für $n \ge 1$.
				Dann gelten
				\begin{enumerate}[(i)]
					\item
						$(B_n)_{n\in\N}$ sind paaweise disjunkt per Konstruktion
					\item
						$B_n \in \scr F$ für alle $n$ (denn $A_{n-1},A_n \in \scr F \implies A_n, A_{n-1} \in \scr F \implies A_n \cap A_{n-1}^C = A_n \setminus A_{n-1} \in \scr F$)
					\item
						$\bigcup_{n=1}^\infty B_n = A$, denn
						\[
							\bigcup_{n=1}^\infty B_n = \bigcup_{n=1}^\infty (A_n \setminus A_{n-1}) = \bigcup_{n=1}^\infty A_n = A.
						\]
				\end{enumerate}
				Es folgt
				\[
					\my(A)
					= \my(\bigcup_{n=1}^\infty B_n)
					= \sum_{n=1}^\infty \my(B_n)
					= \lim_{m\to\infty} \sum_{n=1}^m \my(B_n)
					= \lim_{m\to \infty} \my(\bigcup_{n=1}^m B_n)
					= \lim_{m\to \infty} \my(A_m)
				\]
			\item
				Funktioniert ähnlich wie, oder mit (a).
			\item
				Zeige zunächst
				\[
					\forall n \in \N : \my(\bigcup_{i=1}^n A_i) \le \sum_{i=1}^n \my(A_i)
				\]
				per vollständiger Induktion (Übung).

				Es gilt $\bigcup_{i=1}^n A_i \nearrow \bigcup_{i=1}^\infty A_i$, also
				\[
					\my(\bigcup_{i=1}^\infty A_i)
					= \lim_{n\to \infty} \my(\bigcup_{i=1}^n A_i)
					\le \lim_{n\to \infty} \sum_{i=1}^n \my(A_i)
					= \sum_{i=1}^\infty \my (A_i).
				\]
			\item
				Schreibe $B = A \cupdot (B \setminus A)$, dann ist
				\[
					\my(B) = \my(A) + \underbrace{\my(B\setminus A)}_{\ge 0} \ge \my(A).
				\]
		\end{enumerate}
	\end{proof}
\end{lem}

\begin{df} \label{1.1.11}
	Sei $(\Omega, \scr F, \P)$ ein Wahrscheinlichkeitsraum und $(A_i)_{i\in\N}$ eine Folge messbarer Mengen in $\scr F$.
	Definiere
	\begin{align*}
		\limsup_{n\to\infty} A_n
			&:= \bigcap_{n=1}^\infty \bigcup_{k=n}^\infty A_k, \\
		\liminf_{n\to\infty} A_n
			&:= \bigcup_{n=1}^\infty \bigcap_{k=n}^\infty A_k.
	\end{align*}
\end{df}

\begin{nt} \label{1.1.12}
	Wir versuchen \ref{1.1.11} zu interpretieren:
	\begin{align*}
		\limsup_{n\to\infty} A_n
			&= \Big\{ \omega \in \Omega : \forall n \in \N \exists k \ge n : \omega \in A_k \Big\} \\
			&= \Big\{ \omega \in \Omega : \omega \in A_k \text{ für unendlich viele $k \ge 1$} \Big\}, \\
		\liminf_{n\to\infty} A_n
			&= \Big\{ \omega \in \Omega : \exists n \in \N \forall k \ge n : \omega \in A_k \Big\} \\
			&= \Big\{ \omega \in \Omega : \omega \in A_k \text{ für fast alle $k$}.
	\end{align*}
\end{nt}

\begin{st}[I. Lemma von Borel-Cantelli] \label{1.1.13}
	Sei $(\Omega, \scr F, \P)$ ein Wahrscheinlichkeitsraum und $(A_i)_{i\in\N}$ eine Folge messbarer Mengen in $\scr F$.
	Dann gilt
	\[
		\sum_{n=1}^\infty \P(A_n) < \infty
		\implies
		\P(\limsup_{n\to \infty} A_n) = 0.
	\]
	\begin{nt*}[Interpretation]
		Wenn die Ereignisse $(A_n)$ „hinreichend disjunkt“ sind, so liegen fast alle $\omega \in \Omega$ in nur endlich vielen der $(A_n)$.
	\end{nt*}
	\begin{proof}
		Setze $B_n := \bigcup_{k \ge n} A_k$, dann ist $B_n \searrow B = \bigcap_{n=1}^\infty B_n$, d.h.
		\[
			B
			= \bigcap_{n=1}^\infty \bigcup_{k\ge n} A_k
			= \limsup_{n\to\infty} A_n
		\]
		und
		\[
			\P(\limsup_{n} A_n)
			= \P(B)
			\le \P(B_j)
		\]
		für alle $j \in \N$.
		Also gilt für beliebige $n \in \N$.
		\[
			\P(B)
			\le \P(B_n)
			\le \underbrace{\sum_{k=n}^\infty \P(A_k)}_{\to 0 (n\to\infty)}.
		\]
		also ist $\P(B_n) \to 0$ und somit wegen $\P(B) \le \P(B_n)$ für alle $n\in \N$, gilt $\P(B) = 0$.
	\end{proof}
\end{st}

\begin{ex}[Dirac-Maß] \label{1.1.14}
	Sei $\Omega \neq \emptyset$.
	Fixiere $x \in \Omega$.
	Setze für $A \in \Omega$
	\[
		\delta_x(A) := \Ind_A(x) = \begin{cases}
			1 & x \in A \\
			0 & x \not\in A
		\end{cases}.
	\]
	dann ist
	\begin{align*}
		\delta_0([-1,1]) &= 1 &
		\delta_0(\R) &= 1 & \\
		\delta_0((0,\infty)) &= 0 &
		\delta_0(\{0\}) &= 1
	\end{align*}
	$\delta_0$ heißt \emph{Dirac-Maß in $x$}.

	$\delta_0$ ist ein Wahrscheinlichkeitsmaß auf $(\Omega, \scr P(\Omega))$.
	\begin{proof}
		Siehe $\ref{1.1.8}$ für die zu prüfenden Eigenschaften.
		Zunächst ist offensichtlich $(\Omega, \scr P(\Omega))$ ein $\sigma$-Algebra.
		Außerdem gilt
		\begin{enumerate}[(a)]
			\item
				$\delta_x (\emptyset) = 0$, da $x \not\in \emptyset$,
			\item
				Seien $(A_n)$ paarweise disjunkt, $A_n \subset \Omega$.
				Für $x \not\in A_n$ für alle $n$ ist die Aussage klar, sei also $x \in A_j$ für ein $j \in \N$.
				Da die $A_n$ paarweise disjunkt, gilt $\sum_{n=1}^\infty \delta_x (A_n) = 1 = \delta_x (\bigcup_{n=1}^\infty A_n)$.
			\item
				$\delta_x(\Omega) = 1$.
		\end{enumerate}
	\end{proof}
\end{ex}

\section{Einschub: Wahrscheinlichkeitstheorie I., Diskrete Wahrscheinlichkeitsräume}

Sei $\Omega \neq \emptyset$ und $p: \Omega \to [0,1]$, wobei $\Omega$ höchstens Abzählbar und $\sum_{\omega \in \Omega} p(\omega) == 1$.
Für $A \subset \Omega$ setzen wir
\[
	\P(A) := \sum_{\omega \in A} p(\omega).
\]
Dann ist $\P: \scr P(\Omega) \to [0,1]$ ein Wahrscheinlichkeitsmaß (Übung: nachprüfen).

\begin{ex}[Laplace'scher Wahrscheinlichkeitsraum] \label{1.2.1}
	Sei $\# \Omega < \infty$ und für alle $\omega$
	\[
		p(\omega) = \P(\{\omega\}) = \f 1{\# \Omega}.
	\]
	Dann ist für $A \subset \Omega$
	\[
		\P(A)
		= \sum_{\omega \in A} p(\omega)
		= \sum_{\omega \not\in A} \f 1{\Omega}
		= \f 1{\# \Omega} \sum_{\omega \in A} 1
		= \f {\# A}{\# \Omega}.
	\]
	Man nennt $\P$ \emph{diskrete Gleichverteilung}.
\end{ex}

\begin{ex}[Bernoulli-Verteilung ($\Bin_{1,p}$)] \label{1.2.2}
	Sei $p \in [0,1]$ ein fester Parameter.
	Ein einmaliges Experiment
	\begin{align*}
		\P(\{1\}) &= \P(\text{„Erfolg“}) = p. \\
		\P(\{0\}) &= \P(\text{„Misserfolg“}) = 1 -p.
	\end{align*}
	Mit $\Omega = \{0,1\}$ und $p_k = \P({k})$, $k \in \Omega$ ist dies ein diskreter Wahrscheinlichkeitsraum.
\end{ex}

\begin{ex}[Binomial-Verteilung ($\Bin_{n,p}$)] \label{1.2.3}
	Ein Bernoulli-Experiment $n$-mal ausgeführt, jeweils unabhängig voneinander.
	\begin{align*}
		\P(\text{„genau $k$ Erfolge“}) = \binom{n}{k} p^k (1-p)^{n-k}
	\end{align*}
	$\P$ ist tatsächlich eine Verteilung, d.h.
	\begin{enumerate}[(i)]
		\item
			$p_k \in [0,1]$
		\item
			$\sum_{k=0}^n p_k = 1$
	\end{enumerate}
	\begin{proof}
		Zeige zunächst (ii):
		\[
			\sum_{k=0}^n p_k
			= \sum_{k=0}^n \binom{n}{k} p^k (1-p)^{n-k}
			= (p + (1-p))^n
			= 1.
		\]
		Es gilt $p_k \ge 0$ und damit mit (ii) $p_k \le 1$.
	\end{proof}
\end{ex}

\begin{ex}[Hypergeometrische Verteilung ($\Hyp_{n;S,W}$)] \label{1.2.4}
	Betrachte eine Urne mit $S$ schwarzen und $W$ weißen Kugeln.
	Ziehe $n \le S + W$ Kügeln ohne Zurücklegen.

	Wie hoch ist die Wahrscheinlichkeit, genau $s$ schwarze (und somit $n-s =w$ weiße) Kugeln zu ziehen?

	Zunächst ist $s \le S, s \ge 0, w \le W, s \le n$, oder statt $w \le W$, $s \ge n - W$.
	\[
		\max \{0, (n-W)\} \le s \le \min\{ n, S \}.
	\]
	Es ist
	\[
		\P(\{s\})
		= \Hyp_{n;S,W}(s)
		= \dfrac {\binom{S}{s} \binom{W}{n-s}}{\binom{S+W}{n}}
	\]
	eine Wahrscheinlichkeitsverteilung (der Nachweis ist kompliziert).
\end{ex}

\begin{ex*}[Skat]
	Wir suchen $\P(\text{3 Buben beim Geber})$ und wenden dazu \ref{1.2.4} an.
	\begin{align*}
		S &= \#\text{Buben} = 4 &
		W &= \#\text{Nicht-Buben} = 28
	\end{align*}
	Berechne
	\[
		\Hyp_{10;4,28}(3)
		= \f {\binom 43 \binom{28}{3}}{\binom {32}{10}} = \f {66}{899} \approx 7,3\%.
	\]
\end{ex*}

\coursetimestamp{24}{10}{2013}

\begin{ex}[Geometrische Verteilung ($\Geo_p$)] \label{1.2.5}
	Wir spielen ein Bernoulli-Experiment mit Erfolgschance $p\in (0,1]$ und untersuchen, wann der erste Erfolg eintritt.
	Die Verteilung ist gegeben durch
	\[
		p_k
		= \Geo_p(k)
		= \P(\text{1 Erfolg in $k$-tem Versuch})
		= \P(\text{$(k-1)$ Misserfolge, dann Erfolg)}
		= (1-p)^{k-1} p
	\]
	\begin{proof}
		\begin{enumerate}[(i)]
			\item
				$p_k \in [0,1]$ ist erfüllt
			\item
				Es gilt
				\[
					\sum_{k=1}^\infty p_k
					= \sum_{k=1}^\infty (1-p)^{k-1} p
					= p \sum_{k=0}^\infty (1-p)^k
					= p \f 1{1-(1-p)}
					= 1
				\]
		\end{enumerate}
	\end{proof}
\end{ex}

\begin{ex}[Poisson-Verteilung ($\Pois_\alpha$)] \label{1.2.6}
	Gegeben ein Parameter $\alpha \in (0,\infty)$, betrachten wir eine Verteilung auf $\N_0$, definiert durch
	\[
		p_k
		= \Pois_\alpha(k)
		= e^{-\alpha} \f {\alpha^k}{k!}
	\]
	\begin{proof}
		\begin{enumerate}[(i)]
			\item
				Wegen $p_k \ge 0$ und $\sum_{k=0}^\infty p_k = 1$, muss auch $p_k \le 1$.
			\item
				Es gilt
				\[
					\sum_{k=0}^\infty p_k = e^{-\alpha} \sum_{k=0}^\infty \f {\alpha^k}{k!} = 1.
				\]
		\end{enumerate}
	\end{proof}
	Man verwendet sie zur Modellierung sehr vieler, sehr seltener Ereignisse.
\end{ex}

\begin{st}[Poisson'scher Grenzwertsatz, Poisson-Approximation der Binomialverteilung] \label{1.2.7}
	Falls $p_n \to 0$ für $n \to \infty$, sodass $\lim_{n\to\infty} p_n n = \alpha \in (0,\infty)$, so gilt für alle $k \in \N_0$
	\[
		\lim_{n\to \infty} \Bin_{n,p_n}(k)
		= \Pois_\alpha(k).
	\]
	\begin{proof}
		Sei $k \in \N_0$ fest.
		Wir beweisen zunächst zwei Hilfsaussagen
		\begin{itemize}
			\item
				Für $n > k$ gilt
				\[
					\binom{n}{k}
					= \f {n!}{k!(n-k)!}
					= \f {n(n-1)(n-2)\dotsb(n-k+1)}{k!},
				\]
				also
				\[
					\binom{n}{k} \f {k!}{n^k}
					= \f nn \f {n-1}{n} \f{n-2}n \dotsb \f {n-k+1}n
				\]
				und damit
				\[
					\lim_{n\to\infty} \binom{n}{k} \f {k!}{n^k} = 1.
				\]
			\item
				Sei $t \in \R$, dann ist
				\[
					\lim_{t\to 0} \f {\ln(1+t)}{t}
					= \lim_{t\to 0} \f 1{1+t}
					= 1,
				\]
				also
				\[
					1
					= \lim_{t\to 0} \f 1t \ln(1+t)
					= \lim_{t \to 0} \ln((1+t)^{\f 1t})
				\]
				und damit
				\[
					\lim_{t\to 0} (1+t)^{\f 1t}
					= e
				\]
		\end{itemize}
		Es sei \oBdA $n > k$.
		Mit diesen Aussagen gilt jetzt
		\begin{align*}
			\lim_{n\to\infty} \Bin_{n,p_n}(k)
			&= \lim_{n\to\infty} \binom{n}{k}p_n^k (1-p_n)^{n-k} \\
			&= \lim_{n\to\infty} \underbrace{\binom{n}{k} \f {k!}{n^k}}_{\to 1 \text{ (s.o.)}}\f {n^k}{k!} p_n^k (1-p_n)^n  \underbrace{(1-p_n)^{-k}}_{\to 1 \text{(s.o.)}}. \\
			&= \lim_{n\to \infty} \f {(p_n n)^k}{k!}  \Big( \underbrace{(1-p_n)^{-\f 1{p_n}}}_{\to e} \Big)^{-n p_n} \\
			&= \lim_{n\to \infty} \f {(p_n n)^k}{k!} e^{p_n n} \\
			&= \f {\alpha^k}{k!} e^{-\alpha} \\
			&= \Pois_{\alpha}(k.)
		\end{align*}
	\end{proof}
\end{st}

\begin{ex}[Urnenmodelle] \label{1.2.8}
	Urnenmodelle sind spezielle Laplace'sche Wahrscheinlichkeitsräume.

	Wir betrachten das Modell mit $n$ Kugeln, nummeriert von $1, \dotsc, n$.
	Ziehe $k$ Kugeln, notiere jeweils die Zahl auf der Kugel.
	\begin{enumerate}[(i)]
		\item
			\textbf{mit Zurücklegen, mit Reihenfolge}
			\begin{align*}
				\Omega_1 &= \{1, \dotsc, n\}^k,
				\# \Omega_1 &= n^k.
			\end{align*}
		\item
			\textbf{ohne Zurücklegen, mit Reihenfolge, $k \le n$}
			\begin{align*}
				\Omega_2 &= \{(\omega_1,\dotsc, \omega_k) \in \Omega_1 : \omega_i \neq \omega_j \text{ für } i \neq j \},
				\# \Omega_2 &= \# \Omega_3 k! = \f {n!}{(n-k)!}.
			\end{align*}
		\item
			\textbf{ohne Zurücklegen, ohne Reihenfolge, $k \le n$}
			Da die Reihenfolge keine Rolle spielt, sortieren wir
			\begin{align*}
				\Omega_3 &= \{(\omega_1,\dotsc, \omega_k) \in \Omega_1 : \omega_1 < \dotsb < \omega_k  \},
				\# \Omega_3 &= \binom{n}{k}.
			\end{align*}
		\item
			\textbf{mit Zurücklegen, ohne Reihenfolge}
			Wieder sortieren:
			\begin{align*}
				\Omega_4 &= \{ (\omega_1, \dotsc, \omega_k) \in \Omega_1 : \omega_1 \le \omega_2 \le \dotsb \le \omega_k \{ \\
				\# \Omega_4 &= \binom{n+k-1}{k}
			\end{align*}
			\begin{proof}
				Übungsaufgabe, Hinweis:
				Betrachte $f: \Omega_4 \to \Omega_3'$ mit
				\[
					\Omega_3' = \{ (\omega_1, \dotsc, \omega_k) \in \{1,2\dotsc, n+k-1\}^k : \omega_1 < \dotsb < \omega_k \}
				\]
				definiert durch
				\[
					(\omega_1, \dotsc, \omega_k) \mapsto (\omega_1', \dotsc, \omega_k'),
				\]
				wobei $\omega_i' := \omega_i + i- 1$.
				Zeige, dass $f$ eine Bijektion ist.
			\end{proof}
	\end{enumerate}
\end{ex}


\section{Kontruktion von Maßen}


\begin{df} \label{1.3.1}
	Sei $\Omega \neq \emptyset$, $\scr C \subset \scr P(\Omega)$, $\scr \neq \emptyset$.
	$\scr C$ heißt
	\begin{enumerate}[(a)]
		\item
			\emph{durchschnitts-stabil}, falls
			\[
				A,B \in \scr C
				\implies A \cap B \in \scr C.
			\]
		\item
			\emph{Dynken-System}, falls
			\begin{enumerate}[(i)]
				\item
					$\Omega \in \scr C$,
				\item
					$A \in \scr C \implies A^C \in \scr C$,
				\item
					Für $(A_n) \subset \scr C$ paarweise disjunkt ist $\bigcup_{n} A_n \in \scr C$.
			\end{enumerate}
	\end{enumerate}
	\begin{nt}
		\begin{itemize}
			\item
				Einziger Unterschied von $\sigma$-Algebra und Dynken-System ist „paarweise disjunkt“ in (iii).
			\item
				Jede $\sigma$-Algebra ist ein Dynken-System, aber nicht umgekehrt (z.B. $\Omega = \{1,2,3,4\}$ und $\scr C = \{A \subset \Omega : \#A \in \{0,2,4\}$).
		\end{itemize}
	\end{nt}
\end{df}

\begin{st} \label{1.2.2}
	Jedes $\cap$-stabile Dynken-System ist eine $\sigma$-Algebra.
	\begin{proof}[Beweisskizze]
		Sei $\scr C$ ein $\cap$-stabiles Dynken-System, es ist zu zeigen, dass für $(A_n) \subset \scr C$ auch $A := \bigcup A_n \in \scr C$.

		Setze $A_0 := \emptyset$ und für $n \ge 1$
		\[
			B_n := A_n \setminus ( A_1 \cup \dotsb \cup A_{n-1} ).
		\]
		Die $B_n$ sind paarweise disjunkt und es gilt
		\[
			\bigcup_{i=1}^m = \bigcup_{i=1}^m A_i.
		\]
		Zeige, dass für alle $m \in \N$ gilt (Rest Übungsaufgabe)
		\[
			\bigcup_{n=1}^m A_n \in \scr C.
		\]
		Nutze Induktion über $m$.
		Der Anfang ist klar, betrachte den Induktionsschritt:
		Sei $A_1 \cup \dotsb \cup A_{m-1} \in \scr C$.
		Nun ist
		\[
			B_m = A_m \cup \underbrace{( A_1 \cup \dotsb \cup A_{m-1} )^C}_{\in \scr C} \in \scr C
		\]
		und damit
		\[
			(A_1 \cup \dotsb \cup A_{m-1}) \cupdot B_m \in \scr C.
		\]
	\end{proof}
	\begin{nt*}
		Analog zu Lemma \ref{1.1.3} setzt man für $\scr C \subset \scr P(\Omega)$
		\[
			\delta(\scr C)
			:= \bigcap \{ \scr A \text{ Dynken-System}, \scr C \subset \scr A \}
		\]
		als das \emph{von $\scr C$ erzeugte Dynken-System}.
	\end{nt*}
\end{st}

\begin{st} \label{1.3.3}
	Sei $\scr C$ $\cap$-stabil, dann ist
	\[
		\delta(\scr C) = \sigma(\scr C)
	\]
	\begin{proof}
		% fixme: bibref
		Siehe Klenke, Satz 1.19
	\end{proof}
\end{st}

Wir betrachten jetzt Mengensysteme mit noch weniger vorraussetzenden Eigenschaften.

\begin{df} \label{1.3.4}
	Für $\Omega \neq \emptyset$ heißt $\scr A \subset \scr P(\Omega)$ \emph{Algebra}, falls
	\begin{enumerate}[(i)]
		\item
			$\Omega \in \scr A$,
		\item
			$A \in \scr A \implies A^C \in \scr A$,
		\item
			$A,B \in \scr A \implies A \cup B \in \scr A$.
	\end{enumerate}
\end{df}

\begin{nt} \label{1.3.5}
	Jede $\sigma$-Algebra ist eine Algebra, aber nicht umgekehrt.
	\begin{enumerate}[(a)]
		\item
			Man könnte statt (iii) auch fordern, dass
			\[
				A,B \in \scr A \implies A \cap B \in \scr A.
			\]
		\item
			Man könnte statt (ii) auch fordern, dass
			\[
				A,B \in \scr A \implies A \setminus B \in \scr A.
			\]
			(Übungsaufgabe)
	\end{enumerate}
\end{nt}

Man betrachtet auch Mengenfunktionen auf Algebren, z.B. den Inhalt.

\begin{df} \label{1.3.6}
	\begin{enumerate}[(a)]
		\item
			Sei $\scr A$ eine Algebra über $\Omega$.
			Eine Abbildung $\my: \scr A \to [0,\infty]$ heißt \emph{Inhalt}, falls
			\begin{enumerate}[(i)]
				\item
					$\my(\emptyset) = 0$,
				\item
					$\my(A \cup B) = \my(A) + \my(B)$ für $A,B \in \scr A$ mit $A \cap B = \emptyset$.
			\end{enumerate}
		\item
			Ein $\sigma$-additiver Inhalt heißt \emph{Prä-Maß}.
	\end{enumerate}
	\begin{nt*}
		\begin{itemize}
			\item
				Fall $\scr A$ eine $\sigma$-Algebra ist, so „fehlt“ dem Inhalt lediglich die $\sigma$-Additivität, um Maß zu sein.
			\item
				Falls $\scr A$ eine $\sigma$-Algebra ist, so ist das Prä-Maß ein Maß.
		\end{itemize}
	\end{nt*}
\end{df}

\begin{nt}[Eigenschaften von Inhalten] \label{1.3.7}
	Sei $\my$ ein Inhalt auf $(\Omega, \scr A)$.
	Dann gelten
	\begin{enumerate}[(i)]
		\item
			Monotonie: $A,B \in \scr A, A \subset B \implies \my(A) \le \my(B)$,
		\item
			Endliche Additivität: für $A_i \in \scr A$ paarweise disjunkt gilt
			\[
				\my\bigg(\bigcup_{i=1}^m A_i\bigg)
				= \sum_{i=1}^n \my(A_i),
			\]
		\item
			Endliche Sub-Additivität: für $A_i \in \scr A$ gilt
			\[
				\my\bigg(\bigcup_{i=1}^m A_i\bigg)
				\le \sum_{i=1}^n \my(A_i).
			\]
	\end{enumerate}
	\begin{proof}
		Übungsaufgabe.
	\end{proof}
\end{nt}

Die $\sigma$-Additivität ist äquivalent zur Stetigkeit in $\emptyset$.

\begin{lem} \label{1.3.8}
	Sei $\my$ ein endlicher Inhalt auf einer Algebra $\scr A$ (d.h. $\my(\Omega) < \infty$).
	Dann sind äquivalent
	\begin{enumerate}[(i)]
		\item
			$\my$ ist $\sigma$-additiv (d.h. $\my$ ist Prä-Maß),
		\item
			$\my$ ist stetig in $\emptyset$, d.h. für $A_n \in \scr A$ mit $A_n \searrow \emptyset$ gilt $\my(A_n) \to 0$.
	\end{enumerate}
	\begin{proof}
		Übungsaufgabe.
	\end{proof}
\end{lem}

\coursetimestamp{30}{10}{2013}

\begin{st}[Caratheodory] \label{1.3.9}
	Sei $\my$ ein $\sigma$-endliches Prä-Maß auf einer Algebra $A$.
	Dann existiert genau ein Maß $\tilde \my$ auf $\sigma(A)$, das $\my$ fortsetzt.
	\begin{proof}
		Existenz von $\tilde \my$: Klenke, Satz 1.41 (sogar allgemeiner), Bauer MIT, Satz 5.1. % fixme: reference
		Eindeutigkeit: siehe \ref{1.3.10}
	\end{proof}
\end{st}

\begin{st}[Eindeutigkeitssatz] \label{1.3.10}
	Es seien
	\begin{itemize}
		\item
			$\scr T$ eine $\sigma$-Algebra über $\Omega$,
		\item
			$\scr C$ ein $\cap$-stabiles Erzeugendensystem von $\scr F$,
		\item
			$\my, \ny$ zwei Maße auf $\scr F$, die auf $\scr C$ übereinstimmen
		\item
			$(\Omega_n)_{n\in\N} \subset \scr C$ mit $\Omega_n \nearrow \Omega$ und $\my(\Omega_n) = \ny(\Omega_n) < \infty$ für alle $n \in \N$.
	\end{itemize}
	Dann gilt $\my = \ny$ auf ganz $\scr F$.
	\begin{proof}
		Seihe Klenke, Lemma 1.42. % fixme: reference
	\end{proof}
\end{st}

\begin{kor} \label{1.3.11}
	Falls zwei endliche Maße $\my, \ny$ auf $(\Omega, \scr F)$ auf einem $\cap$-stabilen Erzeugendensystem übereinstimmen und $\my(\Omega) = \ny(\Omega)$ gilt, so folgt $\my = \ny$ (auf ganz $\scr F$).

	Insbesondere gilt: falls zwei Wahrscheinlichkeitsmaße $\P_1$ und $\P_2$ auf $(\Omega, \scr F)$ auf einem $\cap$-stabilen Erzeugendensystem übereinstimmen, so gilt $\P_1 = \P_2$.
\end{kor}

\begin{df} \label{1.3.12}
	Sei $(\Omega, \scr F, \my)$ ein Maßraum.
	$N \subset \Omega$ heißt \emph{$\my$-Nullmenge}, falls
	\[
		\exists A \in \scr F : \my(A) = 0 \land N \subset A.
	\]

	$(\Omega, \scr F, \my)$ heißt \emph{vollständig}, falls alle Nullmengen messbar sind ($N \in \scr F$).
\end{df}

\begin{nt} \label{1.3.13}
	\begin{enumerate}[(i)] % fixme: start with 0
		\item
			Eine Nullmenge muss selbst nicht messbar sein.
		\item
			$\emptyset$ ist immer eine Nullmenge.
		\item
			Falls $(N_i)_{i \in \N}$ eine Folge von Nullmengen ist, so ist auch $\bigcup_{i=1}^\infty N_i$ Nullmenge.
		\item
			Jeder Maßraum $(\Omega, \scr F, \my)$ kann \emph{vervollständigt} werden.
			Definiere dazu
			\[
				\scr G := \{ A \subset \Omega : \exists B \in \scr F : A \symdiff B \text{ $\my$-Nullmenge} \}.
			\]
			Es gilt
			\begin{itemize}
				\item
					$\scr G$ ist eine $\sigma$-Algebra,
				\item
					$\scr F \subset \scr G$,
				\item
					$\my$ kann auf $\scr G$ fortgesetzt werden.
					Definiere dazu $\tilde \my: \scr G \to [0,\infty]$
					\begin{enumerate}[(a)]
						\item
							$\tilde \my := \my$ auf $\scr F$,
						\item
							Falls $A \in \scr G \setminus \scr F$, so existiert $B \in \scr F$ mit $A \symdiff B$ $\my$-Nullmenge.
							Setze $\tilde \my(A) := \my(B)$.
					\end{enumerate}
					$\tilde \my$ ist wohldefiniert (d.h. unabhängig von $\scr B$).
			\end{itemize}
			Der Raum $(\Omega, \scr G, \tilde \my)$ ist ein vollständiger Maßraum und heißt \emph{Vervollständigung} von $(\Omega, \scr F, \my)$.
		\item
			Die Vervollständigung hängt wesentlich von $\my$ ab.
	\end{enumerate}
\end{nt}


% 1.4.
\section{Verteilungsfunktionen und Maße}

Sei $\P$ ein Wahrscheinlichkeitsmaß auf $(\R^1, \scr B^1)$ und definiere $F: \R \to [0,1]$ durch
\[
	\scr F(t)
	:= \scr P((-\infty, t]).
\]
$F$ ist monoton wachsend.
Wir wollen den Zusammenhang zwischen $\P$ und $F$ untersuchen.

\begin{df} \label{1.4.1}
	$F: \R \to [0,1]$ heißt \emph{Verteilungsfunktion}, falls
	\begin{enumerate}[(i)]
		\item
			$F$ monoton wachsend: $x \le y \implies F(x) \le F(y)$,
		\item
			$F$ ist rechtsseitig stetig: $x_n \searrow x \implies F(x_n) \to F(x)$,
		\item
			$\lim_{x\to -\infty} F(x) = 0, \lim_{x\to \infty} F(x) = 1$.
	\end{enumerate}
\end{df}

\begin{st} \label{1.4.2}
	\begin{enumerate}[(a)]
		\item
			Sei $\P$ ein Wahrscheinlichkeitsmaß auf $(\R^1, \scr B^1)$.
			Dann existiert genau eine Verteilungsfunktion $F$ mit
			\[
				F(b) - F(a)
				= \P((a,b]),
				\qquad
				a,b\in\R, a<b.
			\]
		\item
			Für alle Verteilungsfunktionen $F$ existiert genau ein Wahrscheinlichkeitsmaß $\P$ auf $(\R^1, \scr B^1)$ mit
			\[
				F(b) - F(a)
				= \P((a,b]),
				\qquad
				a,b\in\R, a<b.
			\]
	\end{enumerate}
	\begin{proof}
		\begin{enumerate}[(a)]
			\item
				\begin{seg}[Existenz]
					Sei $F(x) := \P((-\infty,x])$ für $x \in \R$ und $a < b$.
					Es gilt
					\[
						F(b) - F(a)
						= \P((-\infty,b]) - \P((-\infty,a])
						= \P((a,b]).
					\]
				\end{seg}
				\begin{seg}[Verteilungsfunktion]
					$F$ ist tatsächlich eine Verteilungsfunktion.
					\begin{enumerate}[(i)] %  start with 0
						\item
							$F : \R \to [0,1]$ erfüllt per Definition
						\item
							Sei $x \le y$, dann ist $(-\infty, x] \subset (-\infty, y]$, also $\P((-\infty, x]) \le \P((-\infty, y])$ (Monotonie von Maßen) und damit $F(x) \le F(y)$.
						\item
							Sei $x \in \R$ fest und $x_n \searrow x$.
							Setze $A_n := (-\infty, x_n], A := (-\infty, x]$, dann ist $A_n \searrow A$ (ferner $\P(A_n) \le 1 < \infty$ für alle $n\in \N$).
							Verwende \ref{1.1.10} (b), dann ist
							\[
								\lim_{n\to\infty} F(x_n)
								= \lim_{n\to \infty} \P((-\infty,x_n])
								= \lim_{n\to\infty} \P(A_n)
								= \P(A)
								= \P((-\infty, x])
								= F(x).
							\]
							Linkstetigkeit ließe sich hier nicht zeigen!
						\item
							Sei $x_n \to -\infty$, also insbesondere $\limsup_{n\to\infty} = -\infty$ und $y_n := \sup_{k\ge n} x_k \to -\infty$ und sogar $y_n \searrow -\infty$.
							Also $(-\infty, y_n] \searrow \emptyset$ und wegen $x_n \le y_n \implies F(x_n) \le F(y_n)$ gilt
							\[
								0
								\le \lim_{n\to \infty} F(x_n)
								\le \lim_{n\to \infty} F(y_n)
								= \lim_{n\to \infty} \P((-\infty, y_n])
								\stack{\text{\ref{1.1.10} (b)}}= \P(\emptyset)
								= 0
							\]
							und somit $F(x_n) \to 0$.
							$\lim_{x\to\infty} F(x) = 1$ verläuft analog.
					\end{enumerate}
				\end{seg}
				\begin{seg}[Eindeutigkeit]
					Sei $F^*$ eine weitere Verteilungsfunktion mit der gewünschten Eigenschaft, dann gilt
					\begin{align*}
						F(b) - F(0) = \P((0,b]) = F^*(b) - F^*(0), &\qquad b > 0; \\
						F(0) - F(b) = \P((b,0]) = F^*(0) - F^*(b), &\qquad b < 0. \\
					\end{align*}
					und damit für alle $b \in \R$
					\[
						F(b) = F^*(b) + \underbrace{F(0) - F^*(0)}_{=:c}
						= F^* + c.
					\]
					Es gilt für alle $n \in \N$ $F(n) = F^*(n) + c$, jedoch nach (iii)
					\[
						1
						= \lim_{n\to\infty} F(n)
						= \lim_{n\to\infty} F^*(n) + c
						= 1 + c
					\]
					und somit $c = 0$, also $F = F^*$.
				\end{seg}
			\item
				Lässt sich mit dem Fortsetzungsatz \ref{1.3.9} zeigen, siehe Klenke, Satz 1.60. % fixme: reference
		\end{enumerate}
	\end{proof}
\end{st}

Warum ist die Verteilungsfunktion $F$ nicht stetig?
Es können Sprünge auftreten.

\begin{ex} \label{1.4.3}
	Betrachte das \emph{Diracmaß} in $0$ (vgl. \ref{1.1.14}):
	\begin{align*}
		\delta_0(A)
		&= \Ind_{\{0\in A\}} \\
		F(x)
		&:= \delta_0((-\infty,x])
		= \Ind_{\{0\in(-\infty,x]\}}
		= \Ind_{\{0 \le x\}}
	\end{align*}
	$F$ besitzt einen Sprung in $x=0$, ist dort aber rechtsstetig.
\end{ex}


\section{Einschub WT II: Bedingte Wahrscheinlichkeiten und Unabhängigkeit von Ereignissen}


\subsection{Bedingte Wahrscheinlichkeit}

\begin{description}
	\item[Bedingte Wahrscheinlichkeit]
		Oft besitzt man Vorinformationen und möchte $\P(A)$ ermitteln unter der Voraussetzung, dass ein anderes Ereignis $B$ bekannt ist.
		Mann berechnet dann $\P(A|B)$.
\end{description}

\begin{ex} \label{1.5.1}
	Es wird eine Umfrage unter 60 Stundenten gemacht, ob sie Sport treiben und ob sie Rauchen.
	\begin{table}[H]
		\centering
		\caption{Umfrageergebnisse zum Rauchen und Sport treiben}
		\begin{tabular}{r|cc}
			& Raucher & Nichtraucher \\\hline
			Sport & 10 & 20 \\
			kein Sport & 20 & 10
		\end{tabular}
	\end{table}
	\begin{enumerate}[(a)]
		\item
			Berechne
			\begin{align*}
				&\P(\text{zufällig gewählter Student raucht}) \\
				&\quad= \f {\# \text{Raucher}}{\# \text{alle}}
				= \f {30}{60}
				= \f 12.
			\end{align*}
		\item
			Berechne
			\begin{align*}
				&\P(\text{zufällig gewählter Student raucht und bekannt, dass er Sport treibt}) \\
				&\quad= \f {\# \text{Raucher und Sportler}}{\# \text{alle Sportler}}
				= \f {10}{30}
				= \f 13.
			\end{align*}
	\end{enumerate}
	Setze $A := $ „Person raucht“, $B := $ „Person treibt Sport“.
	In (b) wurde berechnet
	\[
		\f {10}{30}
		= \f {\# (A\cap B)}{\# B}
		= \f {\f{\#(A\cap B)}{\#\Omega}}{\f {\# B}{\#\Omega}}
		= \f {\P(A \cap B)}{\P(B)}
		=: \P(A|B)
	\]
\end{ex}

\begin{df} \label{1.5.2}
	Sei $(\Omega, \scr F, \P)$ ein Wahrscheinlichkeitsraum und $A,B \in \scr F$ mit $\P(B) > 0$.
	Dann heißt
	\[
		\P(A | B) := \f {\P(A\cap B)}{\P(B)}
	\]
	die \emph{bedingte Wahrscheinlichkeit von $A$ unter $B$}.
\end{df}

\coursetimestamp{31}{10}{2013}

\begin{st} \label{1.5.3}
	Sei $(\Omega, \scr F, \P)$ ein Wahrscheinlichkeitsraum und $B \in \scr F$ mit $\P(B) > 0$.
	Dann gelten
	\begin{enumerate}[(i)]
		\item
			$\P(\argdot | B) : \scr F \to [0,1]$ ist ein Wahrscheinlichkeitsmaß
		\item
			Sei $A \in \scr F, \P(A) > 0$, dann ist
			\[
				\P(A|B) = \P(B|A) \f {\P(A)}{\P(B)}.
			\]
		\item
			Seien $A_1, \cdots, A_n \in \scr F$ mit $\P(\bigcup_{i=1}^{n-1} A_i) > 0$, dann gilt
			\[
				\P\bigg(\bigcap_{i=1}^m A_i\bigg)
				= \P(A_1) \cdot \P(A_2|A_1) \cdot \P(A_3|A_1\cap A_2) \cdot \dotsb \cdot \P(A_n|A_1 \cap \dotsb \cap A_{n-1}).
			\]
	\end{enumerate}
	\begin{proof}
		\begin{enumerate}[(i)]
			\item
				Übungsaufgabe % fixme: ref
			\item
				Es gilt
				\[
					\P(A|B)
					= \f{\P(A\cap B)}{\P(B)}
					= \f{\P(A\cap B)}{\P(A)}\f{\P(A)}{\P(B)}
					= \P(B|A) \f{\P(A)}{\P(B)}.
				\]
			\item
				Übungsaufgabe % fixme: ref
		\end{enumerate}
	\end{proof}
\end{st}

\begin{st} \label{1.5.4}
	Sei $(\Omega, \scr F, \P)$ ein Wahrscheinlichkeitsraum und $(B_i)_{i\in I}$ eine abzählbare Zerlegung in $\Omega$ mit $\P(B_i) > 0$ für alle $i \in I$.
	Dann gilt
	\begin{enumerate}[(i)]
		\item
			\emph{Totale Wahrscheinlichkeit}:
			\[
				\P(A) = \sum_{i\in I} \P(A | B_i) \P(B_i).
			\]
		\item
			Falls $A \in \scr F$ mit $\P(A) > 0$, dann gilt die \emph{Formel von Bayes}:
			\[
				\P(B_n | A)
				= \f{\P(A|B_n) \P(B_n)}{\sum_{i\in I} \P(A|B_i) \P(B_i)}.
			\]
	\end{enumerate}
	\begin{proof}
		\begin{enumerate}[(i)]
			\item
				Es gilt
				\[
					A = A \cap \Omega = A \cap \bigg( \bigcupdot_{i \in I} B_i \bigg)
					= \bigcupdot_{i\in I} (A \cap B_i),
				\]
				also
				\[
					\P(A)
					= \P\bigg(\bigcupdot_{i\in I} (A \cap B_i)\bigg)
					= \sum_{i\in I}\P(A \cap B_i)
					= \sum_{i \in I} \P(A | B_i) \P(B_i),
				\]
				denn $\P(A | B_i) = \f {\P(A\cap B)}{\P(B_i)}$.
			\item
				Es gilt
				\[
					\P(B_n | A)
					\stack{\text{\ref{1.5.3} (ii)}}=
					\P(A | B_n) \f {\P(B_n)}{\P(A)}.
				\]
				Die Behauptung folgt dann mit (i).
		\end{enumerate}
	\end{proof}
\end{st}

\begin{ex}[Test auf seltene Krankheit] \label{1.5.5}
	Bekannt sei
	\begin{enumerate}[1)]
		\item
			Krankheit $K$ liegt bei 0.5\% der Bevölkerung vor,
		\item
			Test positiv bei 99\% der Kranken,
		\item
			Test positiv bei 2\% der Gesunden.
	\end{enumerate}
	Mit welcher Wahrscheinlichkeit ist eine positiv getestete Person wirklich krank?

	Setze $B_1 := \text{„Person ist krank“}, B_2 := B_1^c, A := \text{„Test positiv“}$.
	Gesucht ist $\P(B_1 | A)$, bekannt ist
	\begin{enumerate}[1)]
		\item
			$\P(B_1) = 0.005$
		\item
			$\P(A|B_1) = 0.99$
		\item
			$\P(A|B_2) = 0.02$
	\end{enumerate}
	Benutze Bayes nach \ref{1.5.4} (ii) mit $\Omega = B_1 \dunion B_2$:
	\[
		\P(B_1|A)
		= \f {\P(A|B_1) \P(B_1)}{\P(A|B_1) \P(B_1) + \P(A|B_2) \P(B_2)}
		= \f {0.99 \cdot 0.005}{0.99 \cdot 0.005 + 0.02 \cdot (1-0.005)}
		= \f {495}{2485}
		\approx 19.9 \%
	\]
	\begin{nt}
		\begin{enumerate}[1)]
			\item
				Es besteht kein großer Grund zur Beunruhigung.
			\item
				Mit welcher Wahrscheinlichkeit wird eine vorhandene Krankheit durch den Test nicht entdeckt?
				Berechne
				\[
					\P(A^C | B_1)
					\stack{\text{\ref{1.5.3} (i)}} = 1 - \P(A | B_1)
					= 0.01
					= 1 \%.
				\]
				Diese Irrtumswahrscheinlichkeit wird in der Medizin in der Regel minimiert.
		\end{enumerate}
	\end{nt}
\end{ex}

\begin{ex}[Skat] \label{1.5.6}
	Gesucht ist $\P(\text{„jeder Spieler hat genau ein As“})$.
	Setze $A_i := \text{„$i$-ter Spieler hat genau ein As“}$ und berechne
	\[
		\P(A_1 \cap A_2 \cap A_3)
		\stack{\text{\ref{1.5.3} (iii)}} = \P(A_1) \P(A_2 | A_1) \P(A_3 | A_1 \cap A_2)
		= \f {\binom{4}{1} \binom{28}{9}}{\binom{32}{10}} \f {\binom{3}{1} \binom{19}{9}}{\binom{22}{10}} \f {\binom{2}{1}\binom{10}{9}}{\binom{12}{10}}
		\approx 5.56 \%.
	\]
\end{ex}

\subsection{Unabhängigkeit von Ereignissen}

$A, B$ wollen wir unabhängig nennen, wenn das Eintreten von $B$ \emph{keinen} Einfluss auf das Eintreten von $A$ haben (und umgekehrt), also $\P(A | B) = \P(A)$, oder
\[
	\P(A \cap B) = \P(A) \P(B).
\]

% \indep: kopfstehendes \pi

\begin{df} \label{1.5.7}
	Sei $(\Omega, \scr F, \P)$ ein Wahrscheinlichkeitraum
	\begin{enumerate}[(i)]
		\item
			$A, B \in \scr F$ heißen \emph{unabhängig} - wir schreiben $A \indep B$, falls
			\[
				\P(A \cap B) = \P(A) \P(B).
			\]
		\item
			Eine Familie $(A_i)_{i\in I} \subset \scr F$ heißt \emph{unabhängig}, $(A_i) \indep$, falls für jede endliche Teilmenge $\scr J \subset I$ die folgende Produktformel gilt
			\[
				\P\bigg(\bigcap_{j\in \scr J} A_j \bigg)
				= \prod_{j\in \scr J} \P(A_j).
			\]
	\end{enumerate}
\end{df}

\begin{ex}[zurück zu \ref{1.5.1}] \label{1.5.8}
	Sei $A = \text{„Raucher“}$, $\P(A) = \text{„Raucher“}$.
	Es galt $\P(A) = \f 12, \P(B) = \f 12$ und daher
	\[
		\P(A \cap B) = \f 16 \neq \f 14 = \P(A) \P(B),
	\]
	also sind beide Ereignisse nicht unabhängig.
\end{ex}

\begin{ex} \label{1.5.9}
	Es sei eine Urne mit $s$ schwarzen, $w$ weißen Kugeln gegeben ($s, w \ge 1$).
	Sei $A_1 := \text{„1. Kugel weiß“}, A_2 = \text{„2. Kugel weiß“}$.
	Es gilt
	\[
		A_1 \indep A_2
		\iff
		\text{„mit Zurücklegen“}.
	\]
\end{ex}

\begin{nt} \label{1.5.10}
	\begin{enumerate}[(i)]
		\item
			Jede Teilfamilie unabhängiger Ereignisse ist unabhängig.
		\item
			Es gibt den schwächeren Begriff der \emph{paarweisen Unabhängigkeit}:
			Eine Familie $(A_i)_{i\in I} \subset \scr F$ heißt \emph{paarweise unabhängig}, falls
			\[
				\forall i,y \in I, i \neq j :
				\P(A_i \cap A_j) = \P(A_i) \P(A_j).
			\]
			Jede unabhängige Familie ist paarweise unabhängig, aber im Allgemeinen nicht umgekehrt.
			Betrachte als Gegenbeispiel $\Omega := \{1,2,3,4\}, p_i := p(i) := \f 14, A = \{1,2\}, B = \{2,3\}, C = \{1,3\}$.
			Es gilt $\P(A) = \P(B) = \P(C) = \f 12$ und
			\begin{align*}
				\P(A \cap B) &= \f 14 = \P(A) \P(B) \\
				\P(B \cap C) &= \f 14 = \P(B) \P(C) \\
				\P(C \cap A) &= \f 14 = \P(C) \P(A),
			\end{align*}
			also sind $A,B,C$ paarweise disjunkt, aber
			\[
				\P(A\cap B \cap C) = 0 \neq \f 18 = \P(A)\P(B)\P(C).
			\]
		\item
			Für welche $A$ ist $A \indep A$, also $\P(A \cap A) = \P(A) \P(A)$?
			Es muss gelten
			\[
				\P(A) = \P(A)^2,
			\]
			also $\P(A) \in \{0,1\}$ und somit $A \indep A$ genau dann, wenn $\P(A) \in {0,1}$.
		\item
			Unabhängigkeit von Ereignissen kann trotz Kausalität vorliegen.

			Würfele zwei mal hintereinander, $X = \text{„Augensumme beider Würfe“}, Y = \text{„Ergebnis des ersten Wurfels“}$.
			Sei $A = \text{„Die Augensumme ist 7“}$, $B := \text{„Der erste Wurf ist 6“}, \Omega = \{1,2,\dotsc, 6\} \times \{1, \dotsc, 6\}$.
			Es gilt
			\[
				\P(A\cap B)
				= \P({(1,6)})
				= \f 1{36}
			\]
			und
			\[
				\P(B) = \f 16 = \P(A).
			\]
			Also ist $A \indep B$.

			Sei $A = \text{„Die Augensumme ist 2“}, B := \text{„Der erste Wurf ist 6“}, \Omega = \{1,2,\dotsc, 6\} \times \{1, \dotsc, 6\}$.
			Es gilt
			\[
				\P(A\cap B)
				= 0
			\]
			und
			\begin{align*}
				\P(B) &= \f 16, &
				\P(A) &= \f 1{36}.
			\end{align*}
			Also ist $A \not\indep B$.
		\item
			Man verwechsle nicht Disjunktheit ($A \cap B = \emptyset$) und Unabhängigkeit ($\P(A\cap B) = \P(A)\P(B)$).
			Disjunktheit ist in der Regel sogar eine starke Form der Abhängigkeit.

			Zwei disjunkte Ereignisse $A,B$ sind nur dann voneinander Unabhängig, wenn eines von beiden unmöglich ist ($\P(A) = 0$ oder $\P(B) = 0$).
		\item
			Das Gelten der Produktformel für gewisse endlich $\scr J \subset I$ impliziert \emph{nicht} das Gelten der Produktformel för $\scr J' \subset \scr J$.

			Betrachte als Gegenbeispiel einen dreimaligen Münzwurf (Kopf/Zahl, fair).
			$\Omega = \{K, Z\}^3, \#\Omega = 8, \forall \omega \in \Omega : p(\omega) = \f 18$.
			Sei $A = \text{„mindestens 2x Kopf“}, B = \text{„beim 1. Wurf Kopf“}, C = \text{2. und 3. Wurf gleich}$.
			Es gilt
			\begin{align*}
				\P(A)
				&= \f {\#\{(K,K,Z), (K,Z,K), (Z,K,K), (K,K,K)}{8}
				= \f 48
				= \f 12, \\
				\P(B)
				&= \f 12, \\
				\P(C)
				&= \f 12.
			\end{align*}
			Es gilt
			\[
				\P(A\cap B \cap C)
				= \P(\{(K,K,K)\})
				= \f 18
				= \P(A) \P(B) \P(C),
			\]
			jedoch
			\[
				\P(A \cap B)
				= \f 38
				\neq \f 14
				= \P(A) \P(B).
			\]
	\end{enumerate}
\end{nt}
