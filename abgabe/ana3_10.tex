\documentclass[a4paper]{scrartcl}
\usepackage{mathe-blatt}
\blattanadrei

\begin{document}

	\setcounter{section}{10}

	\begin{aufgabe}~

		\begin{enumerate}[{Teil} A:]
			\item
				\begin{enumerate}[(i)]
					\item
						Sei
						\[
							f(t) = \sum_{k=-\infty}^\infty c_k e^{ikt}
						\]
						die Fourierreihenentwicklung von $f$ mit Fourierkoeffizienten $c_k$.
						Durch Ableiten und Koeffizientenvergleich ergeben sich für $f'$ die Fourierkoeffizienten $ikc_k$.
						Es gilt wegen der Vorraussetzung:
						\[
							c_0 = \f 1{2\pi}\int_0^{2\pi} f(t) dt = 0
						\]
						Da $f$ beschränkt auf allen Teilintervallen, auf denen $f$ stetig ist, ist $f$ damit auch beschränkt auf dem Gesamtintervall $[0,2\pi]$, ebenso $f'$.
						Also lässt sich die Parseval'sche Gleichung auf $f$ und $f'$ anwenden:
						\begin{align*}
							\int_0^{2\pi} |f'(t)|^2 dt
							= 2\pi \sum_{\substack{k=-\infty\\k\neq 0}}^\infty |ikc_k|^2
							\ge 2\pi \sum_{\substack{k=-\infty\\k\neq 0}}^\infty |c_k|^2
							= 2\pi \sum_{-\infty}^\infty |c_k|^2
							= \int_0^{2\pi} |f(t)|^2 dt
						\end{align*}
					\item
						Man betrachte die Ungleichung in (i), Gleichheit ergibt sich genau dann, wenn
						\begin{align*}
							\sum_{k=-\infty}^\infty |k||c_k|^2 = \sum_{k=\infty}^\infty |c_k|^2
							\quad \iff \quad \sum_{-\infty}^\infty (|k|-1)|c_k|^2 = 0 
							\quad \iff \quad \forall k\in \Z\setminus\{-1,1\} : c_k = 0
						\end{align*}
						Identifiziert man jetzt $a := c_{-1}$ und $b := c_1$, ergibt sich sofort die Behauptung.
				\end{enumerate}
			\item
				\begin{enumerate}[(i)]
					\item
						Sei $f = \Id$ die identische Abbildung, dann ist $\div f = 2$, also
						\[
							|S| = \int_S dx = \f 12 \int_S \div f dx = \f 12 \int_\gamma f \cdot n \; ds = \f 12 \int_0^{2\pi} \gamma_1(t) n_1(t) + \gamma_2(t) n_2(t) \; dt
						\]
						mit Normalenvektor $n$. Man kann das auch schreiben als
						\[
							= \f 12 \int_0^{2\pi} \gamma_1 \gamma_2'(t) - \gamma_2 \gamma_1'(t) \;dt
						\]
					\item
						Zunächst verschieben wir die Kurve $\gamma$ so, dass $\int_0^{2\pi} \gamma_1(t) dt = \int_0^{2\pi} \gamma_2(t) dt = 0$ ist (eine Art Zentrierung, der Flächeninhalt ändert sich offensichtlich nicht).

						Der Integrand in der ersten Darstellung der Fläche in (i) wird maximal, wenn man als Normalenvektor $n = \f {\gamma(t)}{\|\gamma(t)\|}$ wählt (betrachte dazu das Skalarprodukt).
						Also
						\begin{align*}
							|S| 
							&= \f 12 \int_\gamma f \cdot n \; ds 
							\le \f 12 \int_0^{2\pi} \f 1{\|\gamma(t)\|} \Big(\gamma_1(t)^2 + \gamma_2(t)^2\Big) \; dt
						\intertext{
							Ich vermute, dass sich das $\f 1{\|\gamma(t)\|}$ irgendwie wegdiskutieren lässt, aber kann es nicht genau begründen
						}
							&\stackrel ?\le \f 12 \int_0^{2\pi} \gamma_1(t)^2 + \gamma_2(t)^2 \; dt
						\intertext{
							Definiere $g(t) = \gamma_1(t) + i \gamma_2(t)$.
							Dank unserer Verschiebung vorhin ist die Vorraussetzung $\int_0^{2\pi} g(t) dt = 0$ erfüllt, die anderen ebenfalls.
							Wir wenden also Teil $A$ an:
						}
							&= \f 12 \int_0^{2\pi} |g(t)|^2 \; dt 
							\le \f 12 \int_0^{2\pi} |g'(t)|^2 \; dt
						\intertext{
							Da $\gamma$ die Parametrisierung nach der Bogenlänge war, ist $g'(t) = \|\gamma'(t)\| = 1$:
						}
							&= \f 12 \int_0^{2\pi} dt 
							= \pi 
						\end{align*}
					\item
						Die eine Richtung sollte klar sein (ist $S$ ein Kreis, dann hat er Radius 1 und Flächeninhalt $\pi$).

						Sei der Flächeninhalt also $\pi$, dann gilt Gleichheit in der Abschätzung in (ii).
						Insbesondere ist dann $\|\gamma(t)\| = 1$ und $n(t) = \gamma(t)$ (aus dem Skalarprodukt).
						Es ergibt sich damit anschaulich ein Kreis.
				\end{enumerate}
		\end{enumerate}
	\end{aufgabe}

\end{document}


