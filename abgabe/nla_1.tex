\documentclass{scrartcl}
\usepackage{mathe-blatt}

\blattnla

\begin{document}

\setcounter{aufgabe}{0}

\begin{aufgabe}~

	Zunächst scheint die Folge gegen eine Zahl in der Nähe von $3.7$ zu konvergieren.
	Ab $n=10^{17}$ ergeben sich jedoch auf meinem Rechner Werte von $1$, was nicht mathematisch korrekt ist.
Es gilt nämlich:
\[
\left(1-\frac 1n\right)^n=e^{n\ln\left(1-\frac 1n\right)}
\]
und da:
\[
\lim_{n\to\infty}n\ln\left(1-\frac 1n\right)
=\lim_{n\to\infty}\frac{\ln\left(1-\frac 1n\right)}{\frac 1n}
=\lim_{n\to\infty}\frac{\frac 1{n^2}\cdot\frac 1{1-\frac 1n}}{-\frac 1{n^2}}
=\lim_{n\to\infty}-\frac{n^2}{n^2\left(1-\frac 1n\right)}
=-1
\]
gilt, weil $e^x$ stetig:
\[
\lim_{n\to\infty}\left(1-\frac 1n\right)^n=\frac 1e
\]

\end{aufgabe}

\begin{aufgabe}
\begin{lem*}
Sei $r$ der Wert der Zahl $a\cdot b$, nachdem $\frac 12 10^{-N}$ addiert wurde, mit den ersten $N$ Dezimalstellen gleich $0$.
Da $0\le r<10^{-N}$, gilt:
\[
\left|a\ast b-a\cdot b\right|
=\left|a\cdot b + \frac 12 10^{-N} - r -a\cdot b\right|
=\left|\frac 12 10^{-N}-r\right|
\le 10^{-N}
\]
\end{lem*}
\begin{enumerate}[a)]
\item
\begin{align*}
\left| (a\ast b)\ast c - a\cdot b\cdot c\right|
&= \left| (a\cdot b+x_1)\cdot c + x_2 - a\cdot b\cdot c\right|\\
&=\left| a\cdot b\cdot c +x_1\cdot c+x_2-a \cdot b\cdot c\right| \qquad (-10^{-N}\le x_1,x_2\le 10^{-N})\\
&=\left| x_1\cdot c+x_2\right|\\
&\le c|x_1|+|x_2|\\
&\le 2\cdot 10^{-N}
\end{align*}
\item
\begin{align*}
&\left|(a\ast b)\ast c - a\ast(b\ast c)\right|\\
&\quad =\left|(a\cdot b+x_1)\cdot c+x_2-(a\cdot(b\cdot c + x_3)+x_4)\right| \qquad (-10^{-N}\le x_1,x_2,x_3,x_4\le 10^{-N})\\
&\quad =\left|c\cdot x_1+x_2-a\cdot x_3+x_4\right|\\
&\quad \le\underbrace{\left|c\cdot x_1-a\cdot x_3\right|}_{\le 10^{-N}} + \underbrace{\left|x_2-x_4\right|}_{\le 10^{-N}}
\le 2\cdot 10^{-N}
\end{align*}
\end{enumerate}
\end{aufgabe}

\begin{aufgabe}~

Wähle $B=10, m=1$, und setze:
\begin{align*}
a &= B^{e_a}\sum_{k=-m}^{-1}a_kB^k=10^1\cdot 1\cdot 10^{-1} \qquad (=1)\\
c=b &= B^{e_b}\sum_{k=-m}^{-1}b_kB^k=10^0\cdot 5\cdot 10^{-1} \qquad (=0,5)
\end{align*}
dann gilt:
\[
a+b
=10^1(1\cdot 10^{-1}+5\cdot 10^{-2})
=10^1\cdot 10^{-1}
=a
\]
Also gilt nach der Rundung $a+b=a$ und damit auch $(a+b)+c=a=1$.
Werden die Klammern jedoch anders gesetzt:
\[
a+(b+c)
=a+(10^0\cdot (5+5)\cdot 10^{-1})
=a+(10^1\cdot 1 \cdot 10^{-1})
=10^1\cdot (1+1) \cdot 10^{-1}
=10^1\cdot 2\cdot 10^{-1}
=2
\]
Also:
\[
a+(b+c)=2\neq 1=(a+b)+c
\]
\end{aufgabe}

\begin{aufgabe}
	\begin{enumerate}[a)]
\item
\[
|\cos(x)-1|
=\left|\sum_{k=0}^\infty(-1)^k\frac {x^{2k}}{(2k)!}-1\right|
=\left|\sum_{k=1}^\infty(-1)^k\frac {x^{2k}}{(2k)!}\right|
=\sum_{k=1}^\infty(-1)^{k+1}\frac {x^{2k}}{(2k)!}
\le \frac {x^2}2
=\frac 12\left|x^2\right|
\]
Also gilt $\cos(x)-1=\mathcal O(x^n)$ für $x\to 0$ für alle $n\le 2$
(denn $\forall x\in [-1,1]:\left|x^n\right|\le \left|x^{n-1}\right|$).

\item
Es gilt
\[
\left|x^2\right|\le c|x|
\]
genau dann, wenn $c\ge|x|$. 
Es lässt sich also für jedes $c$ ein entsprechendes Intervall $[-c,c]$ wählen,
so dass obiges erfüllt ist.

Also folgt aus (a), dass
\[
|\cos(x)-1|\le c_1\left|x^2\right| \le c_1c_2|x|=c|x|
\]
Man wähle zu gegebenem $c$ und existierendem $c_1$, $c_2=\frac c{c_1}$ und setze
$U_\varepsilon(0)$ als die kleinere der beiden existierenden Umgebungen für
$|cos(x)-1|\le c_1\left|x^2\right|$ und $|x^2|\le c_2|x|$.
Obige Bedingung ist dann für alle $x\in U_\varepsilon(0)$ erfüllt.
Also gilt $\cos(x)-1=o(x^n)$ für $x\to 0$ für alle $n=1$.
\item
$\sqrt{\frac x{x^2+1}} = \mathcal O(x^n)$ gilt für keine $n\in\N$.
Angenommen es existiere ein $n>0$, sodass die Bedingung erfüllt ist, dann müsste ($x\ge 0$)
\begin{align*}
\sqrt{\frac x{x^2+1}} &\le cx^n\\
c&\ge \frac 1{\sqrt{x^2+1}\sqrt x x^{n-1}} \to \infty \qquad (x\to 0)\\
\implies c&\not\in\R
\end{align*}



\end{enumerate}
\end{aufgabe}
\end{document}




















