\documentclass{mywork}
\blattnla

\begin{document}

\begin{aufgabe}
	\begin{enumerate}[a)]
		\item
			Zwei äquivalente Normen induzieren den gleichen Konvergenzbegriff, also:
			\[
				\infty = \lim_{n\to \infty}\kappa_{\|\cdot\|_A}(A_n) = \lim_{n\to\infty}\|A_n\|_A\cdot \lim_{n\to\infty}\|A_n^{-1}\|_A = \lim_{n\to\infty}\|A_n\|_B\cdot\lim_{n\to\infty}\|A_n^{-1}\|_B=\lim_{n\to \infty} \kappa_{\|\cdot\|_B}(A_n)
			\]
		\item
			Nein, gibt es nicht, denn für jede Matrix $A=LL^T$ gilt:
			\[
				A^T = (LL^T)^T = LL^T = A
			\]
			Also ist jede solche Matrix $A$ symmetrisch.
		\item
			Da die Einträge über der Diagonalen von $A$ alle gleich 0 sind, verändern sich die Diagonalelemente während der LR-Zerlegung nicht.
			Damit sind die Diagonaleinträge von $A$ genau die Pivotelemente.
			Da keines von ihnen gleich Null ist, lässt sich der Gauß-Algorithmus ohne Pivotisierung ausführen.
		\item ~
			\vspace{7em}
	\end{enumerate}
\end{aufgabe}

\setcounter{aufgabe}{1}
\begin{aufgabe}
	\begin{enumerate}[a)]
		\item
			Da $\|B\|<1$ gilt für die Norm:
			\[
				\|(I+B)x\| = \|x+Bx\| \ge \|x\| - \|Bx\| \ge (1-\|B\|)\|x\| \ge 0
			\]
			Für $(I+B)x=0$ folgt also $x=0$, womit alle Zeilen der Matrix $(I+B)$ linear unabhängig sind und somit $(I+B)$ invertierbar ist.
		\item
			Zeige: $\|(I+B)^{-1}\| \cdot (1-\|B\|) \le 1$
			\begin{align*}
				\|(I+B)^{-1}\| \cdot (1-\|B\|) &\le
				\|(I+B)^{-1}\|-\|B(I-B)^{-1}\| \\
				&\le \|(I+B)^{-1} + B(I+B)^{-1}\| \\
				&= \|(I+B)(I+B)^{-1}\| = 1
			\end{align*}
		\item
			$A$ ist wegen $\det(A)\neq 0$ invertierbar, also $x=A^{-1}b$.
			Außerdem, da $(I+B)$ nach a) invertierbar ist, gilt:
			\begin{align*}
				A(I+B)(x+\Delta x) &= b\\
			\iff (I+B)(x+\Delta x) &= x\\
			 \implies (I+B)(\Delta x) &= -Bx\\
				  \implies \|\Delta x \| & =\|(I+B)^{-1}Bx\|\\
																	   &\le \|(I+B)^{-1}\|\cdot\|B\|\cdot\|x\|\\
				\implies \f{\|\Delta x\|}{\|x\|} &\le \f{\|B\|}{1-\|B\|}
			\end{align*}
	\end{enumerate}

\end{aufgabe}

\end{document}

