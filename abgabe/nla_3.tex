\documentclass{mywork}
\blattnla

\begin{document}
\begin{aufgabe}\\
	Zur Notation: Schreibe zu Beginn $(E|A)$ und führe die LR-Zerlegung aus.
	Auf der linken Seite steht zum Schluss $L$ und auf der rechten $R$.
	(Damit ist sichergestellt, dass in jedem Schritt das Produkt der linken mit der rechten Matrix die ursprüngliche Matrix $A$ ergibt, was die Kontrolle vereinfacht)
	\begin{enumerate}[a)]
		\item
			\begin{align*}
				\begin{pmatrix}[rrrr|rrrr]
					1&0&0&0  &-1&3&-2&-1\\
					0&1&0&0  &3&-15&7&3\\
					0&0&1&0  &5&-21&14&9\\
					0&0&0&1  &2&-12&5&6
	 \end{pmatrix}&\leadsto
				\begin{pmatrix}[rrrr|rrrr]
					1&0&0&0  &-1&3&-2&-1\\
				-3&1&0&0  &0&-6&1&0\\
				-5&0&1&0  &0&-6&4&4\\
				-2&0&0&1  &0&-6&1&4
				\end{pmatrix}\\
				&\leadsto
				\begin{pmatrix}[rrrr|rrrr]
					1&0&0&0  &-1&3&-2&-1\\
				-3&1&0&0  &0&-6&1&0\\
				-5&1&1&0  &0&0&3&4\\
				-2&1&0&1  &0&0&0&4
				\end{pmatrix}=(L|R)
			\end{align*}
		\item
			\begin{align*}
				\begin{pmatrix}[rrrr|rrrr]
					1&0&0&0  &-1&3&-2&-1\\
					0&1&0&0  &3&-15&7&3\\
					0&0&1&0  &2&-12&5&6\\
					0&0&0&1  &5&-21&14&9
	 \end{pmatrix}&\leadsto
				\begin{pmatrix}[rrrr|rrrr]
					1&0&0&0  &-1&3&-2&-1\\
				-3&1&0&0  &0&-6&1&0\\
				-2&0&1&0  &0&-6&1&4\\
				-5&0&0&1  &0&-6&4&4
				\end{pmatrix}\\
				&\leadsto
				\begin{pmatrix}[rrrr|rrrr]
					1&0&0&0  &-1&3&-2&-1\\
				-3&1&0&0  &0&-6&1&0\\
				-2&1&1&0  &0&0&0&4\\
				-5&1&0&1  &0&0&3&4
				\end{pmatrix}
			\end{align*}
			Im nächsten Schritt müsste man durch die Null auf der Hauptdiagonalen der rechten Matrix teilen.
			Also ist keine LR-Zerlegung (ohne Pivotisierung) durchführbar.
	\end{enumerate}
\end{aufgabe}

\setcounter{aufgabe}{2}
\begin{aufgabe}\\
	Wegen der Multiplikativität der Determinante gilt:
	\[
		\det A = \det (LR) = \det L \cdot \det R
	\]
	$L$ und $R$ sind linke untere und rechte obere Dreiecksmatrizen, also berechnet sich deren Determinanten als Produkt der Diagonalelemente.
	Da die Diagonalelemente von $L$ jedoch nach Vorraussetzung alle $1$ sind, ist auch $\det L=1$.
	Also gilt:
	\[
		\det A = \det R = \prod_{j=1}^nr_{jj}
	\]
\end{aufgabe}
\newpage
\begin{aufgabe}
	\begin{lem*}
		Die Matrix diagonaldominante Matrix $A$ bleibt während jedem Schritt der LR-Zerlegung diagonaldominant.
		(Ohne Beweis)
	\end{lem*}\vspace{1em}

	Beweise durch Induktion, dass in keinem Schritt durch Null geteilt wird und somit die LR-Zerlegung durchführbar ist:
	
	\underline{Induktionsanfang ($n=1$):}\\
	Da $a_{ii}>\sum_{j\neq i}|a_{ij}| \forall i=1,\dotsc,n$ gilt insbesondere:
	\[
		|a_{11}|>\sum_{j\neq 1}|a_{ij}| \ge 0 \implies a_{11} \neq 0
	\]
	Damit ist das Diagonalelement, durch das im ersten Schritt geteilt wird ungleich Null.
	
	\underline{Induktionsschritt:}\\
	Unsere Induktionsvorraussetzung lautet: $A^{(k)}\ni a_{kk}\neq 0$.
	Wir zeigen, dass das Diagonalelement $a_{k+1,k+1}\in A^{(k+1)}$ aus der Matrix im nächsten Schritt ungleich Null ist und somit ein weiterer Schritt ausgeführt werden kann:
	Nach dem Algorithmus der LR-Zerlegung gilt folgendes:
	\begin{align*}
		l_{k+1,k} &:= \frac {a_{k+1,k}^{(k)}}{a_{k,k}^{(k)}}\\
		a_{k+1,k+1}^{(k+1)} &:= a_{k+1,k+1}^{(k)} - l_{k+1,k}\cdot a_{k,k+1}^{(k-1)}
	\end{align*}
	Man erhält also das gesuchte Diagonalelement $a_{k+1,k+1}^{(k+1)}$ des nächsten Schritts durch 
	\[
		a_{k+1,k+1}^{(k+1)} := a_{k+1,k+1}^{(k)} - a_{k+1,k}^{(k)}\cdot \frac{ a_{k,k+1}^{(k-1)}}{a_{k,k}^{(k)}}
	\]
	Es gilt:
	\[
		a_{kk} > \sum_{j\neq k}|a_{kj}| \ge |a_{ik}|
	\]
	Da nach Vorraussetzung $a_{kk}\neq 0$, darf durch $a_{kk}$ geteilt werden und es gilt $\left|\frac{a_{k,k+1}}{a_{k,k}}\right|<1$, also
	\begin{align*}
		\left|a_{k+1,k+1}^{(k+1)}\right| &\ge \left|a_{k+1,k+1}^{(k)}\right| - \left|a_{k+1,k}^{(k)}\cdot \frac{ a_{k,k+1}^{(k-1)}}{a_{k,k}^{(k)}}\right|\\
																										&\ge \left|a_{k+1,k+1}^{(k)}\right| - \left|a_{k+1,k}^{(k)}\right| > 0
	\end{align*}
	Also ist auch das Diagonalelement, durch das im nächsten Schritt geteilt wird ungleich Null.
\end{aufgabe}

\begin{aufgabe}
	\begin{enumerate}
		\item[c)]
			\begin{tabular}{r|r}
				n & $\frac{\|\tilde x-x\|_2}{\|x\|_2}$\\
				\hline
				10 & 2.5947e-16\\
			   100 & 2.1322e-16\\
				   500& 1.4314e-16\\
				  1000& 1.2642e-16
			\end{tabular}
		\item[d)]
			\begin{tabular}{r|r}
				n & $\frac{\|\tilde x-x\|_2}{\|x\|_2}$\\
				\hline
				5 & 2.9283e-12\\
				10& 5.4744e-05\\
				15& 1.8273\\
				20& 2.5607
			\end{tabular}
	\end{enumerate}

\end{aufgabe}
\end{document}
