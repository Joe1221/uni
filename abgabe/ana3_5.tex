\documentclass[a4paper]{scrartcl}
\usepackage{mathe-blatt}
\blattanadrei

\begin{document}

	\setcounter{section}{5}

	\begin{aufgabe}~
		
		Da $f_0$ an jedem Punkt in $U_0$ in eine Potenzreihe entwickelt werden kann, ist $f_0$ offensichtlich beliebig oft differenzierbar in $U_0$.
		Sei $x_0\in U_0$, dann ist in der Umgebung von $x_0$ die Taylorentwicklung
		\[
			f_0(x) = \sum_{n=0}^\infty \f {f_0^{(n)}(x_0)}{n!}(x-x_0)^n
		\]
		die eindeutige Potenzreihe von $f_0$.
		Bezeichne ihren Konvergenzradius mit $r_{x_0}$.

		Definiere jetzt
		\[
			U := \bigg( \Big\{ \bigcup_{x\in U} K_{r_x}(x) \Big\} \setminus \R \bigg) \cup U_0
			 =\bigg( \Big\{ \bigcup_{x\in U} K_{r_x}(x) \Big\} \cap (\C \setminus \R)\bigg) \cup U_0
		\]
		$U$ ist damit offen, da $\C\setminus \R$ und $U_0$ offen sind.
		Außerdem ist $U\cap \R = U_0$ erfüllt.

		Wähle jetzt für beliebiges $z\in U$ ein $x_z\in U_0$, so dass $z\in K_{r_{x_z}}(x_z)$, also im Konvergenzkreis der Potenzreihe um $x_z$ liegt (Existenz von $x_z$ ist durch die Definition von $U$ gewährleistet).
		Definiere jetzt
		\[
			f(z) := \sum_{n=0}^\infty \f {f_0^{(n)}(x_z)}{n!} (z-x_z)^n
		\]
		Die Reihe ist wegen der Wahl von $x_z$ konvergent und auch für zwei verschiedene Entwicklungspunkte $x_z$ und $x_z'$ konvergieren die beiden Reihen für ein $z$ im Konvergenzradius gegen den selben Wert.
		Also ist $f$ wohldefiniert.

		Offensichtlich gilt mit dieser Definition $f\big|_{U_0} = f_0$.

		Um die Holomorphie von $f$ nachzuweisen, zeigen wir, dass sich $f$ an einem beliebigen Punkt $z_0\in U$ in eine Potenzreihe entwickeln lässt.
		Definiere $x_{z_0}$ so wie oben als den Entwicklungspunkt für $z_0$ auf der reellen Achse.
		Dann gilt für alle $z\in K_{r_{x_{z_0}}}(x_{z_0})$ : 
		\begin{align*}
			f(z) &= \sum_{n=0}^\infty \f {f_0^{(n)}(x_{z_0})}{n!} (z-x_{z_0})^n \\
			&= \sum_{n=0}^\infty \f {f_0^{(n)}(x_{z_0})}{n!} (z-z_0+z_0-x_{z_0})^n \\
			&= \sum_{n=0}^\infty \f {f_0^{(n)}(x_{z_0})}{n!} \sum_{k=0}^{n} \binom nk (z_0-x_{z_0})^{n-k}(z-z_0)^k
		\intertext{
			Durch umsortieren der Glieder (innerhalb des Konvergenzradius erlaubt, da absolut konvergent) und gruppieren all derer mit gleichem Exponenten für $(z-z_0)^k$, ergibt sich (eine Skizze für die Summationsindizes ist zum Nachvollziehen hilfreich)
		}
			&= \sum_{k=0}^\infty \left( \sum_{m=0}^\infty \f {f_0^{(m)}(x_{z_0})}{m!}\binom mk (z_0-x_{z_0})^{m-k}\right) (z-z_0)^k
		\end{align*}
		Dies ist eine Potenzreihe mit einem Konvergenzradius von mindestens $r_{x_{z_0}}-|z_0-x_{z_0}| > 0$ um den Entwicklungspunkt $z_0$.
		Also ist $f$ holomorph auf $U$.

		\newpage
		Zur Frage, ob man für $U_0 = (-1,1)$ immer $U:=K_1(0)$ wählen kann: 
		Nein! Betrachte
		\[
			f(x) = \f 1{1+4x^2}
		\]
		auf $U_0 = (-1,1)$.
		$f$ lässt sich auf jedem Punkt $U_0$ in eine Potenzreihe entwickeln (reell differenzierbar und keine reellen Polstellen), aber $U=K_1(0)$ enthält die Polstelle $\f i2$, weswegen $f$ auf $U$ nicht holomorph sein kann.
	\end{aufgabe}

	\begin{aufgabe}
		\begin{enumerate}[a)]
			\item
				Weil $f_n$ kompakt gegen $f$ konvergiert, konvergiert $f_n$ auf einer kompakten Dreiecksfläche $D$ gleichmäßig gegen $f$.
				Zeige für die Vorraussetzung des Satzes von Morera, dass $\int_{\d D}f_n(z)dz = 0$ gilt:
				\begin{align*}
					\lim_{f_n\to f} \l| \int_{\d D}f_n(z) dz - \int_{\d D}f(z) dz \r|
					\le \lim_{f_n \to f} \Big( \underbrace{\max_{z\in \d D}|f_n(z) - f(z)|}_{\to 0 \text{ (gleichmäßig)}} \cdot \underbrace{L(\d D)}_{=\const} \Big) = 0
				\end{align*}								
				Also $\int_{\d D}f(z) dz = \lim\limits_{f_n\to f} \int_{\d D}f_n(z) dz = 0$ ($f_n$ holomorph).
				Damit ist $f$ nach Morera holomorph.
			\item
				Definiere auf $\C\setminus \big(\{z\in \C : |z| = 1 \}\cup \{0\}\big)$:
				\[
					f_N(z) := \sum_{n=0}^N \f 1{z^n + z^{-n}} = \f 12 + \sum_{n=1}^N \f 1{z^n + z^{-n}}
				\]
				Zeige jetzt die gleichmäßige Konvergenz mit Hilfe des Weierstraß'schen M-Tests für $|z|< r <1$ (für $|z|>r >1$ analog).
				\[
					\l | \f 1{z^n+z^{-n}} \r| \le \f 1{|z|^{-n}-|z|^n} = \f 1{|z|^{-n}(1-|z|^{2n})} \le \f 1{1-r^2} r^n =: M_n
				\]
				$\sum_{n=1}^\infty M_n$ konvergiert offensichtlich wegen $r<1$.

				Damit konvergiert $f_N(z)$ auf jeder kompakten Teilmenge von $\C\setminus \big(\{z\in \C : |z| = 1 \}\cup \{0\}\big)$ gleichmäßig gegen $f$ (wähle einfach ein $r$, das genügend nah an $1$ liegt, so dass die kompakte Menge enthalten ist).
				Mit anderen Worten: $f_N$ konvergiert kompakt gegen $f$.
				$f_N$ ist als endliche Summe von Kompositionen holomorpher Funktionen auf $\C\setminus \big(\{z\in \C : |z| = 1 \}\cup \{0\}\big)$ holomorph.

				Damit ist die Bedingung des Satzes aus a) erfüllt, und $f$ ist holomorph.

				Zeige für die holomorphe Fortsetzung, dass $f(z)$ in $K_r(0)$ mit $r<1$ beschränkt ist:
				\begin{align*}
					\l| \sum_{n=0}^\infty \f 1{z^n + z^{-n}} \r|
					&\le \f 12 + \sum_{n=1}^\infty \f 1{|z^n + z^{-n}|} \\
					&\le \f 12 + \sum_{n=1}^\infty \f 1{|z|^{-n}-|z|^n} \\
					&= \f 12 + \sum_{n=1}^\infty \f 1{1-|z|^{2n}} \cdot |z|^n \\
					&\le \f 12 + \f 1{1-|z|^2} \sum_{n=1}^\infty |z|^n \\
					&\le \f 12 + \f 1{1-r^2} \sum_{n=1}^\infty r^n \\
					&= \f 12 + \f 1{1-r^2} \l( \f 1{1-r} - 1\r) =: M
				\end{align*}
				damit sind die Vorraussetzungen für den Riemannschen Hebbarkeitssatz erfüllt und $f$ kann im Punkt $0$ holomorph fortgesetzt werden.
		\end{enumerate}
	\end{aufgabe}
	

\end{document}


