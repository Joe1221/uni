\documentclass{mywork}
\blattpse

\renewcommand{\"}{\textquotedbl}

\begin{document}

\begin{aufgabe}~

	\begin{enumerate}[a)]
		\item
			$aabbcc \in L(G_1)$, denn es lässt sich folgendermaßen produzieren: $S \to aSBC \to aaBCBC \to aaBBCC \to aabBCC \to aabbCC \to aabbcC \to aabbcc$.
		\item
			$S
			\to aB
			\to aaBB
			\to aaaBBB
			\to aaabSBB
			\to aaabbABB
			\to aaabbaBB
			\to aaabbabB
			\to aaabbabbS
			\to aaabbabbbA
			\to aaabbabbba$
		\item
			$L(G_3) = \Big\{ a\{ab,ba\}^nb : n\in \N_0 \Big\}$
		\item
			$L(G_4) = \Big\{ a_1a_2 \dotso a_{n-1}a_na_na_{n-1} \dotso a_2a_1 : n \in \N, \forall k\in \{1,\dotsc,n\} a_k\in\{0,1\} \Big\}$
		\item
			$L(G_5) = \{a^n ba : n\in \N_0\}$
		\item
			$L(G_6) = \{a^n ba : n\in \N_0\}$
		\item
			$G_7 = \Big( \{a,b\}, \{S\}, \{S\to ab, S\to SS\}, S \Big)$
		\item
			$G_8 = \Big( \{a,b\}, \{S\}, \{S\to ab, S\to aSb\}, S \Big)$
	\end{enumerate}
\end{aufgabe}

\begin{aufgabe}
	\begin{enumerate}[a)]
		\item
			Deutsche Adresse = (Person | Firma), ((Straße, Hausnummer) | Postfach inkl. Nummer), PLZ, Stadt, [Tel.], [Fax], [E-mail] ;
		\item
			Grammatik = $(\"0\" | \"1\"), \{(\"0\" | \"1\")\} ;$
		\item
			Ziffer = $\"0\" | \"1\" | \"2\" | \"3\" | \"4\" | \"5\" | \"6\" | \"7\" | \"8\" | \"9\" ;$ \\
			GanzeZahl = $[\"$+$\" | \"$-$\"],$ Ziffer$, [\{$Ziffer$\}] ;$ \\
			DezimalZahl = GanzeZahl$, [\",\", \{$Ziffer$\}] ;$ \\
			FließkommaZahl = DezimalZahl$, \"$E$\", $GanzeZahl$ ;$
		\item
			Die Sprache ist gegeben durch ein oder mehr „a“s, gefolgt von ein oder mehr „b“s, gefolgt von ein oder mehr „c“s.
	\end{enumerate}
\end{aufgabe}

\begin{aufgabe}
	\begin{enumerate}[a)]
		\item
			Die Sprache lässt sich rekursiv beschreiben durch: $S = \Big(\"a\" | (\"a\", S, \"a\") | (S, \"a\") \Big)$
		\item
			Die Sprache ist beschrieben durch die Zeichenkette gegegeben durch entweder „a“, „b“ oder „c“, gefolgt von einer beliebig langen Binärzahl.
	\end{enumerate}
\end{aufgabe}

\end{document}
