\documentclass[a4paper]{scrartcl}
\usepackage{mathe-blatt}
\blattpse

\begin{document}

\setcounter{aufgabe}{1}

\begin{aufgabe}~

	Siehe Quellcode im Verzeichnis „aufgabe2“.
	\begin{verbatim}
Die größte Zahl ist: 5
Die mittlere Zahl ist: 4
Die kleinste Zahl ist: 2
drei
	\end{verbatim}
\end{aufgabe}

\begin{aufgabe}~

	Siehe Quellcode im Verzeichnis „aufgabe3“.

	Die Ausgabe des Programms sieht folgendermaßen aus:
	\begin{verbatim}
Gerade Zahlen zwischen 4 und 13:
4
6
8
10
12
Gerade Zahlen zwischen 4 und 13:
4
6
8
10
12
Die Fakultät von 5 ist 120
	\end{verbatim}
\end{aufgabe}

\begin{aufgabe}~

	Siehe Quellcode im Verzeichnis „aufgabe4“.

	Die Ausgabe des Programms lautet:
	\begin{verbatim}
Index   Element
---------------
0       4
1       12
2       16
3       18
4       20
Index   Element
---------------
4       20
3       18
2       16
1       12
0       4
	\end{verbatim}
\end{aufgabe}

\begin{aufgabe}~

		
\end{aufgabe}
	Siehe Quellcode im Verzeichnis „aufgabe5“.

	Die Ausgabe des Programms lautet:
	\begin{verbatim}
Der Song BugsBunny wird hinzugefügt
Der Song Tour wird hinzugefügt
CountItems=2
Song [name=BugsBunny, type=Musik, duration=35]
Song [name=Tour, type=Rock, duration=90]
Der Song1 wird gelöscht
CountItems=1
Song [name=BugsBunny, type=Musik, duration=35]
Das Video Terminator wird hinzugefügt
Das Video Matrix wird hinzugefügt
CountItems=3
Video [name=Terminator, type=Action, duration=5400]
Video [name=Matrix, type=Action, duration=6300]
Die Playlist von ISTE-MediaPlayer
=================================
Song [name=BugsBunny, type=Musik, duration=35]
Video [name=Terminator, type=Action, duration=5400]
Video [name=Matrix, type=Action, duration=6300]		
	\end{verbatim}
\end{document}
