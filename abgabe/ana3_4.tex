\documentclass[a4paper]{scrartcl}
\usepackage{mathe-blatt}
\blattanadrei

\begin{document}

	\setcounter{section}{4}
	\begin{aufgabe}
		\begin{enumerate}[a)]
			\item
				Wähle ein festes $z_0\in O$ und setze als Kandidaten für die Stammfunktion:
				\[
					F(z) := \int_{[z_0,z]} f(\zeta) d\zeta \qquad z\in O
				\]
				Wegen $[z_0,z]\in O$ (da $O$ sternförmig) ist $F(z)$ wohldefiniert.

				Da $f$ holomorph, gilt nach Hilfssatz 2.17 für jede abgeschlossene Dreiecksfläche $D\in O$ mit geschlossener Randkurve $\d D$:
				\[
					\int_{\d D} f(z) dz = 0
				\]
				Verfahre nun analog zum Beweis vom Satz von Morera, um zu zeigen, dass $F$ tatsächlich eine Stammfunktion von $f$ ist.
				Nutze dazu obige Aussage und die Stetigkeit von $f$:
				\begin{align*}
					\l| \f {F(w)-F(z)}{w-z} - f(z) \r|
					&= \l| \f 1{w-z} \l( \int_{[z_0,w]}f(\zeta) d\zeta - \int_{[z_0,z]}f(\zeta)d\zeta - (w-z) f(z) \r) \r| \\
					&= \l| \f 1{w-z} \l( \int\limits_{[z_0,w]}f(\zeta) d\zeta - \int\limits_{[z_0,z]}f(\zeta)d\zeta - \int\limits_{[z,w]}f(z) d\zeta + \int\limits_{[w,z]}f(\zeta) d\zeta + \int\limits_{[z,w]} f(\zeta) d\zeta \r)\r| \\
					&= \l| \f 1{w-z} \int_{[z,w]} f(\zeta) - f(z) d\zeta \r| \\
					&\le \f 1{|w-z|} \underbrace{\max_{\zeta\in [z,w]} |f(\zeta) - f(z)|}_{< \eps \text{ für $|w-z|<\delta$}} \cdot L([z,w])
					< \eps \qquad |w-z| < \delta
				\end{align*}
				Also besitzt $f$ die Stammfunktion $F$.
			\item
				Wir zeigen zunächst, die Vorraussetzung für den Satz von Morera, nämlich dass für jede Dreiecksrandkurve $\d D$ in $\C$ gilt:
				\[
					\int_{\d D} f(z) dz = 0
				\]
				Dabei unterscheiden wir folgende Fälle:

				\begin{seg}{$\d D \cap \{z\in C : \Re z = 0\} = \emptyset$}
					In diesem Fall folgt die Aussage direkt aus dem Hilfssatz 2.17, da $f$ auf $\{z\in C : \Re z = 0\}$ holomorph ist.										
				\end{seg}
				\begin{seg}{Die Gerade $i\R$ teilt das Dreieck $D$ in zwei Teilflächen}
					Eine Teilfläche ist ein Dreieck und besitzt eine Randkante auf der Geraden $i\R$.
					Die andere Teilfläche lässt sich so in drei Dreiecke teilen, dass eines wieder eine Randkante auf der Geraden hat und die beiden anderen jeweils eine Ecke auf der Geraden.
					Diese Teildreiecke können dann separat nach dem kommenden Fall behandelt werden.
				\end{seg}
				\begin{seg}{Das Dreieck liegt mit einer Ecke oder einer Kante auf der Geraden $i\R$}
					Wir nähern die Dreiecksrandkurve $\gamma$ mit einer Kurve $\gamma_\eps$ an.

					Sei $z_0$ ein Eckpunkt des Dreiecks, der nicht auf der Geraden $i\R$ liegt und $z_1,z_2$ die anderen beiden.
					Wir wählen $\gamma_\eps$ so, dass wir ein Dreieck $z_0,z_1+\eps(z_0-z_1), z_2+\eps(z_0-z_2)$ erhalten (mit anderen Worten: man rückt immer näher an die $z_1z_2$-Seite).
					
					Da $f$ stetig ist, ist $f$ auf dem kompakten Dreieck $D$ gleichmäßig stetig.
					Es ergibt sich
					\begin{align*}
						&\lim_{\eps\to 0} \l| \int_\gamma f(z) dz - \int_{\gamma_\eps} f(z) dz \r| 
						= \lim_{\eps\to 0} \l| \int_0^1 f(\gamma(t)) \cdot \gamma'(t) dt - \int_0^1 f(\gamma_\eps'(t) dt \r| \\
						&\qquad= \lim_{\eps\to 0} \l| \int_0^1 f(\gamma(t)) \gamma'(t) - f(\gamma_\eps(t)) \gamma'(t) + f(\gamma_\eps(t)) \gamma'(t) - f(\gamma_\eps(t)) \gamma_\eps'(t) \;dt\r| \\
						&\qquad= \lim_{\eps\to 0} \l| \int_0^1 \Big(f(\gamma(t))-f(\gamma_\eps(t))\Big)\gamma'(t) + f(\gamma_\eps(t))\Big(\gamma'(t) - \gamma_\eps'(t)\Big) \; dt \r|\\
						&\qquad\le \lim_{\eps\to 0} \max_{t\in [0,1]} \Big(\underbrace{|f(\gamma(t))-f(\gamma_\eps(t))|}_{\to 0 \; \text{(glm. stetig)}}\cdot\underbrace{|\gamma'(t)|}_{\le C_1} + \underbrace{|f(\gamma_\eps(t))|}_{\le C_2}\cdot|\gamma'(t)-\gamma_\eps'(t)| \Big) \\
						&\qquad\le C_2 \cdot \lim_{\eps\to 0} \max_{t\in [0,1]} |\gamma'(t)-\gamma_\eps'(t)|
						\intertext{
							Seien die Randkurvenparametrisierungen von $\gamma$ o.B.d.A. durch stückweise definierte Wege mit jeweils konstanter Geschwindigkeit ($(z_1-z_0$, $z_2-z_1$ oder $z_0-z_2$)) gegeben.
							Dann approximiert $\gamma_\eps'(t)$ offensichtlich $\gamma'(t)$ gleichmäßig, also gilt:
						}
						&\qquad= 0
					\end{align*}
					Die Kurve $\gamma_\eps$ liegt in $\{z\in \C : \Re z \neq 0\}$ und es gilt $\int_{\gamma_\eps} f(z) dz = 0$, also auch für die Kurve $\gamma$.
				\end{seg}
				Damit sind die Vorraussetzungen für den Satz von Morera erfüllt und $f$ ist damit holomorph auf ganz $\C$.
		\end{enumerate}


	\end{aufgabe}

	\begin{aufgabe}

		\begin{enumerate}[a)]
			\item
				Sei $u = \Re f, v= \Im f$.
				Da $\Phi\in C^2(O\to \R)$ vertauschen die partiellen Ableitungen.
				Außerdem gelten die Cauchy-Riemannschen DGLs:
				\[
					\d_x^2 u(x,y) = \d_xd_y v(x,y) = \d_y\d_x v(x,y) = -\d_y^2 u(x,y)
				\]
				Also $\d_x^2 u(x,y) + \d_y^2 x(x,y) = 0$ und damit ist $u(x,y) = \Re f(x,y)$ harmonisch.
				Für $\Im f(x,y)$ analog.
			\item
		\end{enumerate}
		



	\end{aufgabe}

\end{document}


