\documentclass{mywork}
\blattanadrei

\begin{document}

\setcounter{section}{14}

\begin{aufgabe}
	\begin{enumerate}[(a)]
		\item
			Setze $u_1(x) := y(x)$ und $u_2(x) := y'(x)$.
			Man erhält die Bedingungen $u_1'(x) = u_2(x)$ und (aus der DGL) $u_2'(x) = -6 u_1(x) + 5 u_2(x)$ und damit das DGL-System:
			\[
				u' = Au = \begin{pmatrix}
					0 & 1 \\
					-6 & 5
				\end{pmatrix} u
			\]
			Als Eigenwerte der Matrix ergeben sich $\lambda_1 = 2, \lambda_2 = 3$ mit Eigenvektoren $v_1 = (1,2)^T, v_2 = (1,3)^T$.
			Da $A$ damit diagonalisierbar ist, ergibt sich die Lösung des Systems durch
			\[
				u(x) = \sum_{j=1}^2 c_j e^{\lambda_j x} v_j = c_1 e^{2x} \begin{pmatrix}
					1 \\
					2
				\end{pmatrix} + c_2e^{3x} \begin{pmatrix}
					1 \\
					3
				\end{pmatrix}
			\]
			Für die ursprüngliche DGL also
			\[
				y(x) = u_1(x) = c_1e^{2x} + c_2e^{3x}
			\]
		\item
			Verfahre wie in (a) und man erhält das DGL-System
			\[
				u' = Ax = \begin{pmatrix}
					0 & 1 \\
					-16 & 8
				\end{pmatrix} u
			\]
			Berechne die Jordannormalform:
			\[
				A = TJT^{-1},
				\qquad T = \begin{pmatrix}
					1 & 0 \\
					4 & 1
				\end{pmatrix},
				\qquad J = \begin{pmatrix}
					4 & 1 \\
					0 & 4
				\end{pmatrix}
			\]
			Betrachte das Jordansystem:
			\[
				w' = \begin{pmatrix}
					4 & 1 \\
					0 & 4
				\end{pmatrix} w
			\]
			welches die Lösung
			\[
				w(x) = \begin{pmatrix}
					c_1 e^{4x} + c_2 x e^{4x} \\
					c_2 e^{4x}
				\end{pmatrix}
			\]
			besitzt. Resubstituiere $u(x) = T w(x)$:
			\[
				u(x) = \begin{pmatrix}
					1 & 0 \\
					4 & 1
				\end{pmatrix} \begin{pmatrix}
					c_1 e^{4x} + c_2 x e^{4x} \\
					c_2 e^{4x}
				\end{pmatrix}
			\]
			Als Lösung der ursprünglichen DGL ergibt sich dann
			\[
				y(x) = u_1(x) = c_1 e^{4x} + c_2 x e^{4x}
			\]
	\end{enumerate}
\end{aufgabe}

\newpage

\begin{aufgabe}~

	Die Matrix hat Eigenwerte $\lambda_1 = 1, \lambda_2 = 2, \lambda_3 = 3$ und Eigenvektoren
	\[
		v_1 = \begin{pmatrix}
			1 \\ 1 \\ 1
		\end{pmatrix}, \qquad
		v_2 = \begin{pmatrix}
			1 \\ 1 \\ 0
		\end{pmatrix}, \qquad
		v_3 = \begin{pmatrix}
			1 \\ -1 \\ 1
		\end{pmatrix}
	\]
	Damit ist die Matrix diagonalisierbar und die Lösung des DGL-Systems ergibt sich als
	\[
		y(x) 
		= \sum_{j=1}^3 c_j e^{\lambda_j x}v_j
		= c_1 e^x \begin{pmatrix}
			1 \\ 1 \\ 1
		\end{pmatrix} + c_2 e^{2x} \begin{pmatrix}
			1 \\ 1 \\ 0
		\end{pmatrix} + c_3 e^{3x} \begin{pmatrix}
			1 \\ -1 \\ 1
		\end{pmatrix}
		= \begin{pmatrix}
			c_1 e^x + c_2 e^{2x} + c_3 e^{3x} \\
			c_1 e^x + c_2 e^{2x} - c_3 e^{3x} \\
			c_1 e^x + c_3 e^{3x}
		\end{pmatrix}
	\]
\end{aufgabe}

\end{document}


