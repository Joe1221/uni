\documentclass{mywork}
\blattpse

\begin{document}

\setcounter{aufgabe}{3}

\begin{aufgabe}~

	\begin{itemize}
		\item
			\begin{itemize}
				\item
					Dual-Core
				\item
					Fertigungstechnik: 32nm
				\item
					Befehlssatzunterstützung: MMX, SSE, SSE2, SSE3, SSSE3, Intel 64, XD-Bit, Hyper-Threading, EIST
				\item
					Taktrate: 1,60 GHz
				\item
					L2-Cache: 2x512 KiB
				\item
					TDP: 3,5 Watt
			\end{itemize}
		\item ~
			\begin{table}[h]
				\centering
				\begin{tabular}{p{8cm}|p{8cm}}
					Vorteile & Nachteile \\ \hline
					Vielseitig einsetzbar & Von-Neumann-Flaschenhals (SISD, sequentielle Verarbeitung) \\
					deterministischer Programmablauf durch sequentiellen Ablauf & Memory-Wall (langsamer Speicher wird zunehmend zum Flaschenhals)

				\end{tabular}
			\end{table}
		\item ~
			\begin{table}[h]
				\centering
				\begin{tabular}{l|c|c}
					 & entpsricht VNA & entspricht nicht VNA \\ \hline
					 EVA-Prinzip & & X \\
					 SISD & X &  \\
					 Tertiärspeicher &  & X \\
					 Binärkodierung & X & 
				\end{tabular}
			\end{table}
		\item ~
			\begin{enumerate}[a)]
				\item ~
					\begin{table}[h]
						\centering
						\begin{tabular}{l|l}
							Register & Funktion \\ \hline
							Arbeitsregister & Nimmt Daten und manchmal auch Adressen auf \\
							Befehlszählregister & Nimmt die Adresse des nächsten Befehls auf \\
							Befehlsregister & Nimmt binäre Maschinenbefehle auf
						\end{tabular}
					\end{table}
				\item
					Das Steuerwerk entscheidet über die Verarbeitung von eingegebenen Daten und liest diese in den Speicher ein.
					Diese Daten können später vom Rechenwerk bearbeitet werden und das Ergebnis wiederum im Speicher gespeichert werden.
					Das Steuerwerk entscheidet wiederum über die Ausgabe der Daten aus dem Speicher.
			\end{enumerate}
	\end{itemize}
\end{aufgabe}

\end{document}
