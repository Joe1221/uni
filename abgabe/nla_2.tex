\documentclass{mywork}
\blattnla

\begin{document}

\begin{lem*}
Ist $\lambda\neq 0$ Eigenwert von $A$, dann ist $\lambda^{-1}$ Eigenwert von $A^{-1}$.
\begin{proof}
Sei $\lambda$ Eigenwert von $A$ zum Eigenvektor $v$, dann gilt:
\begin{align*}
Av &= \lambda v\\
\iff v &= A^{-1}\lambda v\\
\iff \lambda^{-1}v &= A^{-1}v
\end{align*}
Also ist $\lambda^{-1}$ Eigenwert von $A^{-1}$.
\end{proof}
\end{lem*}
\begin{lem*}
Die Matrix $A$ hat die selben Eigenwerte wie $A^T$.
\begin{proof}
Die Eigenwerte sind durch die Nullstellen des charakteristischen Polynoms gegeben.
Dieses ist aber identisch mit dem für $A^T$:
\[
\det(A-\lambda I)=\det((A-\lambda I)^T)=\det(A^T-\lambda I)
\]
\end{proof}
\end{lem*}

\begin{aufgabe}
\begin{enumerate}[(a)]
\item
Es gilt:
\[
\|AB\|
=\sup_{x\neq 0}\frac{\|ABx\|}{\|x\|}
=\sup_{x\neq 0}\left(\frac{\|ABx\|}{\|Bx\|}\frac{\|Bx\|}{\|x\|}\right)
=\sup_{x\neq 0}\frac{\|ABx\|}{\|Bx\|}\cdot\underbrace{\sup\frac{\|Bx\|}{\|x\|}}_{=\|B\|}
\]
Da jedoch $\{Bx|x\neq 0\}\subset\{y|y\neq 0\}$, gilt die Abschätzung:
\[
\sup_{x\neq 0}\frac{\|ABx\|}{\|Bx\|}\cdot\|B\|
\le \sup_{x\neq 0}\frac{\|Ay\|}{\|y\|}\cdot\|B\|
= \|A\|\cdot\|B\|
\]
Und somit
\[
\|AB\|\le \|A\|\cdot\|B\|
\]
\item
\[
\|Ax\| 
= \frac{\|Ax\|}{\|x\|}\cdot \|x\|
\le \sup_{y\neq 0}\frac{\|Ay\|}{\|y\|}\cdot \|x\|
= \|A\|\cdot\|x\|
\]
\item
	Da $\|\cdot\|_M$ mit $\|\cdot\|$ verträglich, gilt für alle $y\neq 0$: $\|A\|_M\ge \frac {\|Ay\|}{|y|}$, also auch:
\[
\|A\|_M
\ge \sup_{x\neq 0}\frac{\|Ax\|}{\|x\|}
= \|A\|
\]
\end{enumerate}

\end{aufgabe}
\begin{aufgabe}
	\begin{align*}
		\|A\|_1 &= \max_{j=1,\dots,n}\sum_{i=1}^n|a_{ij}| = 6\\
  \|A\|_\infty &= \max_{i=1,\dots,n}\sum_{j=1}^n|a_{ij}| = 5
	\end{align*}
	Berechne $A^{-1}$:
	\begin{align*}
		\begin{pmatrix}[rrr|rrr]1&4&0&1&0&0\\0&2&3&0&1&0\\0&0&3&0&0&1\end{pmatrix}
		&\leadsto \begin{pmatrix}[rrr|rrr]1&0&0&1&-2&0\\0&1&0&0&\frac 12&0\\0&0&1&0&0&\frac 13\end{pmatrix}\\
		\implies A^{-1} &= \begin{pmatrix}[rrr]1&-2&0\\0&\frac 12&0\\0&0&\frac 13\end{pmatrix}
	\end{align*}
	Also gilt:
	\begin{align*}
		\kappa_1(A)&=\|A\|_1\|A^{-1}\|_1=6\cdot \frac 52=15\\
   \kappa_\infty(A)&=\|A\|_\infty\|A^{-1}\|_\infty=5\cdot 3=15
	\end{align*}
	Berechne für die Spektralnorm die Eigenwerte von
	\[
		A^TA=\begin{pmatrix}1&4&0\\4&20&0\\0&0&9\end{pmatrix}
	\]
	Es gilt:
	\begin{align*}
		\det(A^TA-\lambda I)&=\begin{vmatrix}1-\lambda&4&0\\4&20-\lambda&0\\0&0&9-\lambda\end{vmatrix}
		=(9-\lambda)\begin{vmatrix}1-\lambda&4\\4&20-\lambda\end{vmatrix}
		=(9-\lambda)(\lambda^2-21\lambda+4)
	\end{align*}
	Die Eigenwerte sind also:
	\begin{align*}
		\lambda_1&=9\\
		\lambda_2&=\frac 12(21+5\sqrt{17})\\
		\lambda_3&=\frac 12(21-5\sqrt{17})
	\end{align*}
	Es ergibt sich:
	\[
		\|A\|_2=\sqrt{\lambda_{\text{max}}(A^TA)}=\sqrt{\frac 12(21+5\sqrt{17})}
	\]
	Da nach den anfangs bewiesenen Lemmas gilt:
	\begin{align*}
		\lambda_{\text{max}}((A^{-1})^TA^{-1})
		=\lambda_{\text{max}}((AA^T)^{-1})
		=\lambda_{\text{max}}((A^TA)^{-1})
		=\frac 1{\lambda_{\text{min}}(A^TA)}
	\end{align*}
	ergibt sich für die Kondition bezüglich der Spektralnorm:
	\[
		\kappa_2(A)
		=\|A\|_2\cdot\|A^{-1}\|_2
		=\frac{\sqrt{\lambda_{\text{max}}(A^TA)}}{\sqrt{\lambda_{\text{min}}(A^TA)}}
		=\frac {\sqrt{\frac 12(21+5\sqrt{17})}}{\sqrt{\frac 12(21-5\sqrt{17})}}
		=\sqrt{\frac{(21+5\sqrt{17})^2}{16}}
		=\frac 14(21+5\sqrt{17})
	\]

\end{aufgabe}

\begin{aufgabe}
	\begin{enumerate}[(a)]
		\item
			\[
				\kappa(\alpha A)=\|\alpha A\|_2\cdot\|(\alpha A)^{-1}\|=\alpha\|A\|_2\cdot\frac 1\alpha\|A\|_2=\kappa(A)
			\]
		\item
			\begin{align*}
				\kappa(AB)&=\|AB\|_2\cdot\|B^{-1}A^{-1}\|\\
						  &\le\|A\|_2\cdot\|B\|_2\cdot\|B^{-1}\|_2\cdot\|A^{-1}\|_2=\kappa(A)\cdot\kappa(B)
			\end{align*}
		\item
			Da $A$ orthogonal, gilt $A^T=A^{-1}$, also:
			\begin{align*}
				\kappa(A)
				&=\|A\|_2\cdot\|A^{-1}\|_2\\
				=\|A\|_2\cdot\|A^T\|_2\\
				&=\sqrt{\lambda_{\text{max}}(A^TA)}\sqrt{\lambda_{\text{max}}(AA^T)}\\
				&=\lambda_{\text{max}}(A^TA)\\
				&=\lambda_{\text{max}}(A^{-1}A)\\
				&=\lambda_{\text{max}}(E)=1
			\end{align*}
		\item
			Für einen Eigenwert $\lambda$ zu einem Eigenvektor $x\neq 0$ gilt $Ax=\lambda x$, also auch
			\[
				\|Ax\|=\|\lambda x\| \longrightarrow |\lambda|=\frac {\|Ax\|}{\|x\|}
			\]
			und damit
			\[
				\lambda_{\text{max}}=\sup_{x\neq 0}\frac {\|Ax\|}{\|x\|}
			\]
			Da jedoch nach Aufgabe 1c) $\|A\|=\sup_{x\neq 0}\frac {\|Ax\|}{\|x\|}$ minimal ist, gilt
			\begin{align*}
				\kappa(A)&=\|A\|_2\cdot\|A^{-1}\|_2\\
					&\ge \|A\|\cdot \|A^{-1}\|\\
					&=\lambda_{\text{max}}(A)\cdot \lambda_{\text{max}}(A^{-1})
			\end{align*}
	\end{enumerate}
\end{aufgabe}

\begin{aufgabe}
	\\Man teile die Formel
	\begin{equation}
		\label{pq_ns}
		x=-p+\sqrt{p^2+q}
	\end{equation} 
	in kleinere Einzeloperationen auf:
	\begin{align*}
		s&=p^2\\
		t&=s +q\\
		u&=\sqrt{t}\\
		x&=-p+u
	\end{align*}
	Es ergibt sich für die Fehler der Einzeloperationen (Produkte von Fehlern werden ignoriert):
	\begin{align*}
		\Delta s&=(p+\Delta p)^2-p^2=2p\Delta p\\
		\Delta t&=(s +\Delta s) + (q +\Delta q) = \Delta s + \Delta q = 2p\Delta p + \Delta q\\
		\Delta u&=\sqrt{t+\Delta t}-\sqrt{t}\\
			    &=\frac 1{\sqrt{t}}\sqrt{t^2+t\Delta t+\left(\frac 12\Delta t\right)^2}-\sqrt{t}
				 =\frac 1{\sqrt{t}}\left(t+\frac 12\Delta t\right)-\sqrt{t}\\
		   &=\frac 12\frac{\Delta t}{\sqrt{t}}
				 =\frac {2p\Delta p+\Delta q}{2\sqrt{p^2+q}}\\
		\Delta x&=-(p+\Delta p)+(u+\Delta u)-(-p+u)
				 =-\Delta p +\Delta u\\
				 &=-\Delta p +\frac {2p\Delta p+\Delta q}{2\sqrt{p^2+q}}
	\end{align*}
	Jetzt schätzt man ab:
	\begin{align*}
		|\Delta x|&=\left|-\Delta p+ \frac {2p\Delta p}{2\sqrt{p^2+q}}+\frac{\Delta q}{2\sqrt{p^2+q}}\right|\\
				   &\le |\Delta p|+\left|\frac{2p\Delta p}{2\sqrt{p^2}}\right|+\left|\frac{\Delta q}{2\sqrt{q}}\right|\\
				   &\le 2\Delta p + \frac {\Delta q}{\sqrt q}
	\end{align*}
	und damit weil $p,q>0 \implies x=-p+\sqrt{p^2+q}>0$
	\[
		\frac{|\Delta x|}{|x|}\le2\frac{|\Delta p|}{|x|}+\frac{|\Delta q|}{|x|\sqrt{q}}
	\]
	Für \eqref{pq_ns} bedeutet das, dass sie nicht rückwärtsstabil ist, da der Ausdruck
	$\frac {|\Delta x|}{|x|}$ nicht durch ein konstantes Vielfaches der Maschinengenauigkeit eps beschränkt ist.
	Man erkennt an \eqref{pq_ns}, dass für kleine $q$ nahe der Auslöschung operiert wird, was sich auch in der Fehlerabschätzung wieder findet: für kleine $q$ wird der zweite Summand groß.
	
	\begin{equation}
		\label{pq_s}
		x=\frac q{p+\sqrt{p^2+q}}
	\end{equation}
	Es gilt, da $q>0$
	\[
		x=\frac q{p+\sqrt{p^2+q}}=\frac{q(p-\sqrt{p^2+q})}{p^2-(p^2+q)}=-p+\sqrt{p^2+q}
	\]
	womit die Äquivalenz gezeigt wäre.

	Für die Fehlerabschätzung zerlegt man ähnlich wie oben:
	\begin{align*}
		s&=p^2\\
		t&=s +q\\
		u&=\sqrt{t}\\
		w&=p+u\\
		v&=\frac 1w\\
		x&=q\cdot v
	\end{align*}
	Von vorhin weiß man bereits
	\[
		\Delta u=\frac{2p\Delta p+\Delta q}{2\sqrt{p^2+q}}
	\]
	weiterhin:
	\begin{align*}
		\Delta w&=(p + \Delta p)+(u + \Delta u)-(p+u)=\Delta p + \Delta u =\Delta p+\frac{2p\Delta p+\Delta q}{2\sqrt{p^2+q}}\\
\frac{\Delta v}v&=\frac{\frac{1}{\Delta w + w}-\frac 1w}{\frac 1w}=\frac w{\Delta w + w}-1=\frac{\Delta w}{\Delta w +w}\\
  \frac {\Delta y}y&=\frac{(\Delta q + q)(\Delta v +v)-qv}{qv}=\frac{\Delta qv+\Delta vq}{qv}=\frac {\Delta q}q+\frac {\Delta v}v = \frac {\Delta q}q + \frac{\Delta w}{\Delta w + w}
	\end{align*}
	Ich mache folgende Annahme, die, wenn die Fehler im Verhältnis zu $p$ und $q$ klein sind erfüllt ist:
	\[
		|\Delta w|=\left|\Delta p+\frac{2p\Delta p+\Delta q}{2\sqrt{p^2+q}}\right|<p<\sqrt{p^2+q}
	\]
	Damit kann man Abschätzen:
	\begin{align*}
		\left|\frac{\Delta v}v\right| &= \frac{|\Delta w|}{|\Delta w + w|}\\
								&=\frac{\left|\Delta p+\frac{2p\Delta p+\Delta q}{2\sqrt{p^2+q}}\right|}
		{|\Delta w + w|}\\%{\left|\Delta p+\frac{2p\Delta p+\Delta q}{2\sqrt{p^2+q}}+p+\sqrt{p^2+q}\right|}
		&\le \frac{|\Delta p|}{|\Delta w +w|}+\left|\frac{\Delta p}{\Delta w +w}+\frac {\frac 1{2\sqrt q}\Delta q}{\Delta w + w}\right|\\
		&\le \left|\frac{2\Delta p}{\Delta w + w}\right|+\left|\frac{\frac 1{2\sqrt{q}}\Delta q}{\Delta w+w}\right|\\
		&\le \left|\frac{2\Delta p}{p}\right|+\left|\frac{\frac 1{2\sqrt{q}}\Delta q}{\sqrt{p^2+q}}\right|\\
		&\le 2\frac{|\Delta p|}{p}+\frac{|\Delta q|}{q}
	\end{align*}
	Also auch
	\[
		\frac{|\Delta y|}{|y|}=\frac {|\Delta q|}q+\frac{|\Delta v|}v\le 2\frac{|\Delta p|}{p}+2\frac{|\Delta q|}{q}
	\]
	Da gilt:
	\[
		2\frac{|\Delta p|}p+2\frac{|\Delta q|}q\le 4\cdot \text{ eps}
	\]
	ist $\frac{|\Delta x|}{|x|}$ durch ein konstantes Vielfaches der Maschinengenauigkeit abgeschätzt.
	Damit ist \eqref{pq_s} rückwärtsstabil.
\end{aufgabe}
\end{document}
