\documentclass{mywork}
\blattnumzwei

\begin{document}


\setcounter{section}{2}
\setcounter{aufgabe}{5}

\begin{aufgabe}~

	\begin{enumerate}[a)]
		\item
			Berechne $y^{(1)}, y^{(2)}, y^{(3)}$ mit der Picard-Lindelöf-Iteration.
			Setze dazu $y^{(0)}(x) = y_0 = 0$, dann gilt
			\begin{align*}
				y^{(1)}(t) &= y_0 + \int_0^t y^{(0)}(\tau)^2 + \tau^2 d\tau = \int_0^t t^2 dt  \\
				&= \f 13 t^3 \\
				y^{(2)}(t) &= \int_0^t \f 19 \tau^6 + \tau^2 d\tau \\
				&= \f 1{63} t^7 + \f 13 t^3 \\
				y^{(3)}(t) &= \int_0^t \Big( \f 1{63}\tau^7 + \f 13 \tau^3 \Big)^2 + \tau^2 d\tau \\
				&= \int_0^t \f 1{3969}\tau^{14} + \f 2{189} \tau^{10}  + \f 19 \tau^6 + \tau^2 d\tau \\ 
				&= \f 1{59535}t^{15} + \f 2{2079} t^{11} + \f 1{63} t^7 + \f 13 t^3
			\end{align*}
		\item
			Gehe vor wie in der a):
			\begin{align*}
				y^{(1)}(t) &= \begin{pmatrix}
					1 \\ 1 \\ 0
				\end{pmatrix} + \int_0^t \begin{pmatrix}
					0 \\ 0 \\ 1
				\end{pmatrix} d\tau
				= \begin{pmatrix}
					1 \\ 1 \\ t
				\end{pmatrix} \\
				y^{(2)}(t) &= \begin{pmatrix}
					1 \\ 1 \\ 0
				\end{pmatrix} + \int_0^t \begin{pmatrix}
					\tau \\ - \tau \\ 1
				\end{pmatrix} = \begin{pmatrix}
					1 + \f 12 t^2  \\ 1 - \f 12 t^2 \\ t
				\end{pmatrix} \\
				y^{(3)}(t) &= \begin{pmatrix}
					1 \\ 1 \\ 0
				\end{pmatrix} + \int_0^t \begin{pmatrix}
					\tau + \f 12 \tau^3 \\ - \tau + \f 12 \tau^3 \\ 1
				\end{pmatrix} = \begin{pmatrix}
					1 + \f 12 t^2 + \f 16 t^4 \\ 1 - \f 12 t^2 + \f 16 t^4 \\ t
				\end{pmatrix}
			\end{align*}
			Aufgrund der Taylorentwicklung von $e^x = \sum_{k=0}^\infty \f {x^k}{k!}$ raten wir die Lösung:
			\[
				y(t) = \begin{pmatrix}
					e^{\f {t^2}2} \\ e^{- \f {t^2}2} \\ t
				\end{pmatrix}
			\]
			Diese ist tatsächlich Lösung des AWPs, denn:
			\[
				y'(t) = \begin{pmatrix}
					t e^{\f {t^2}2} \\
					-t e^{\f {-t^2}2} \\
					1
				\end{pmatrix} = \begin{pmatrix}
					y_1(t) y_3(t) \\
					-y_2(t) y_3(t) \\
					1
				\end{pmatrix}
			\]
	\end{enumerate}

\end{aufgabe}
\end{document}
