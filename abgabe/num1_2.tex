\documentclass{mywork}
\blattnumeins

\begin{document}

\begin{aufgabe}~

	Für $n=1$ sind die äquidistanten Knoten $x_0=-1, x_1=1$.
	Es gilt
	\begin{align*}
		\Lambda_1 
		&:= \max_{x\in [-1,1]} \left\{\sum_{n=0}^1 |L_k^{1}(x)| \right\}\\
		&= \max_{x\in [-1,1]} \left\{\left(\left|\f{x-1}2\right| + \left|\f{x+1}2\right|\right)\right\}\\
		&= \f 12 \max_{x\in [-1,1]} \left\{ |x-1| + |x+1|\right\} = 1
	\end{align*}
	
	Für $n=2$ sind die äquidistanten Knoten $x_0=-1,x_1=0,x_2=1$.
	Es gilt dann
	\begin{align*}
		\Lambda_2
		&:= \max_{x\in [-1,1]} \left\{\sum_{n=0}^2 |L_k^{2}(x)|\right\} \\
		&= \max_{x\in [-1,1]} \left\{ \f12 |x(x-1)| + |(x+1)(x-1)| + \f 12 |(x+1)x|\right\} \\
	\intertext{
		Da diese Funktion gerade ist und das Intervall $[-1,1]$, genügt es den Fall $x>0$ zu betrachten
	}
		&= \max_{x\in [-1,1]} \left\{ \f12 x(x-1) + (x+1)(x-1) + \f 12 (x+1)x\right\} \\
		&= \max_{x\in [-1,1]} \left\{ -x^2 +x+1\right\} \\
	\intertext{Das Maximum wird bei $x=\f 12$ angenommen und beträgt $\f 54$}
		&= \f 54
	\end{align*}
\end{aufgabe}

\begin{aufgabe}~

	Zunächst wird $p(x_j) =f(x_j)$ für $j=0,\dotsc,n$ erfüllt:
	\begin{align*}
		p(x_j)
		&= q(x_j) + \sum_{k=0}^n (f(x_k)-q(x_k)) L_k(x_j)\\
		&= q(x_j) + (f(x_j) -q(x_j) \prod_{\substack{i=0\\i\neq j}}^n \f{x_j-x_i}{x_j-x_i}\\
		&= f(x_j)
	\end{align*}

	Für die Abschätzung bezeichne zunächst 
	\[
		\xi := \min_{\substack{i,j=0,\dotsc,n\\i\neq j}} \{|x_i - x_j|\}
	\]
	den minimalen Abstand zweier Stützstellen voneinander.
	Da $q$ die Funktion $f$ beliebig genau approximiert, wähle $q$ so, dass
	\[
		\max_{x\in [a,b]}|q(x)-f(x)| < \delta
	\]
	mit 
	\[
		\delta := \f \eps{1+ (\f {b-a}\xi)^n(n+1)}
	\]
	Dann gilt mit der vorgegebenen Wahl von $p(x)$:
	\begin{align*}
		\max_{x\in [a,b]} |p(x)-f(x)| 
		&\le \max_{x\in [a,b]}\left\{ |q(x)-f(x)| + \left| \sum_{k=0}^n (f(x_k) - q(x_k)) L_k(x) \right| \right\} \\
		&\le \max_{x\in [a,b]}\left\{ |q(x)-f(x)| + \sum_{k=0}^n \left(|f(x_k) - q(x_k)| \left|\prod_{\substack{i=0\\i\neq k}}^n \f {x-x_i}{x_k-x_i} \right|\right) \right\} \\
		&< \delta + \sum_{k=0}^n \left(\delta \prod_{\substack{i=0\\i\neq k}}^n \f {(b-a)}{\xi} \right) \\
		&\le \delta + \left(\f{b-a}{\xi}\right)^n \sum_{k=0}^n \delta  \\
		&\le \delta \left( 1 + \left(\f{b-a}{\xi}\right)^n (n+1)\right)  \\
		&= \eps
	\end{align*}

	Das Resultat besagt im Grunde, dass man eine stetige Funktion mit Hilfe einer Approximierenden beliebig genau interpolieren kann.
	
	Zur numerischen Konstruktion eignet sich diese Aussage allerdings nur bedingt, da die Funktion $q$ und die kostspieligen Lagrangepolynome mehrfach ausgewertet werden müssen.
\end{aufgabe}
	
\begin{aufgabe}
	\begin{enumerate}[a)]
		\item		
			Wegen $T_n(x) = \cos(n\cos^{-1}(x))$ für $x\in [-1,1]$ und dem Additionstheorem $2\cos(a)\cos(b) = \cos(a+b) + \cos(a-b)$ gilt
			\begin{align*}
				2 T_n(x) T_m(x) &= 2\cos(n\cos^{-1}(x)) \cos(m\cos^{-1}(x))\\
				&=\cos((n+m)\cos^{-1}(x)) + \cos((n-m)\cos^{-1}(x))\\
				&= T_{n+m}(x) + T_{n-m}(x)
			\end{align*}
		\item
			\begin{itemize}
				\item
					Wegen $T_n(x) \le 1$ und $|w|<1$ konvergiert die Reihe $\sum_{n=0}^\infty T_n(x)w^n$ offensichtlich.
					Es gilt deshalb
					\begin{align*}
						\sum_{n=0}^\infty T_n(x)w^n
						&= T_0(x)w^0 + T_1(x)w^1 + \sum_{n=2}^\infty (2xT_{n-1}(x) - T_{n-2}(x))w^n\\
						&= 1 + wx + 2wx \sum_{n=2}^\infty T_{n-1}(x)w^{n-1} - w^2 \sum_{n=2}^\infty T_{n-2}(x) w^{n-2}\\
						&= 1 + wx + 2wx\left( - 1 + \sum_{n=0}^\infty T_n(x)w^n\right) - w^2 \sum_{n=2}^\infty T_n(x)w^n\\
						&= 1 - wx + (2wx - w^2) \sum_{n=0}^\infty T_n(x) w^n
					\end{align*}
					Also
					\[
						(1-2wx +w^2)\sum_{n=0}^\infty T_n(x)w^n = 1-wx
					\]
					und deshalb für $(1-2wx+w^2)\neq 0$
					\[
						\sum_{n=0}^\infty T_n(x) w^n = \f {1-wx}{1-2wx +w^2}
					\]
				\item
					Beweise induktiv nach $n$.
					Induktionsanfang ($n=0$):
					\begin{align*}
						T_0(-x) &= 1 = (-1)^0T_0(x)\\
						T_1(-x) &= x = (-1)(-x) = (-1)^1T_1(-x)
					\end{align*}
					Induktionsschritt:
					\begin{align*}
						T_{n+1}(-x) &= 2(-x)T_{n}(-x) - T_{n-1}(-x) \\
						&= -2x(-1)^nT_n(x) - (-1)^{n-1}T_{n-1}(x) \\
						&= (-1)^{n+1}\cdot 2xT_n(x) - (-1)^{n+1}\cdot T_{n-1}(x) \\
						&= (-1)^{n+1}T_{n+1}(x)
					\end{align*}
			\end{itemize}
	\end{enumerate}
\end{aufgabe}

\end{document}
