\documentclass[a4paper]{scrartcl}
\usepackage{mathe-blatt}
\blattanadrei

\begin{document}

	\setcounter{section}{9}

	\begin{aufgabe}~

		In der c) und d) sind die partiellen Ableitungen vertauschbar, da jeweils zwei mal stetig differenzierbar, der Rest sollte selbsterklärend sein.
		\begin{enumerate}[a)]
			\item
				\begin{align*}
					\nabla \cdot (f u)
					= \sum_{j=1}^3 \d_j(f u_j)
					= \sum_{j=1}^3 \big( (\d_j f) u_j + f (\d_j u_j) \big)
					= \sum_{j=1}^3 (\d_j f) u_j + f \sum_{j=1}^3 \d_j u_j
					= \nabla f \cdot u + f(\nabla \cdot u)
				\end{align*}
			\item
				\begin{align*}
					\nabla \times (fu)
					= \begin{pmatrix}
						\d_2 (fu_3) - \d_3 (fu_2) \\
						\d_3 (fu_1) - \d_1 (fu_3) \\
						\d_1 (fu_2) - \d_2 (fu_1) 
					\end{pmatrix}
					= \begin{pmatrix}
						u_3 \d_2 f - u_2 \d_3 f + f \d_2 u_3 - f \d_3 u_2 \\
						u_1 \d_3 f - u_3 \d_1 f + f \d_3 u_1 - f \d_1 u_3 \\
						u_2 \d_1 f - u_1 \d_2 f + f \d_1 u_2 - f \d_2 u_1
					\end{pmatrix}
					= (\nabla f) \times u + f(\nabla \times u)
				\end{align*}
				\begin{align*}
					\nabla \cdot (u \times v)
					&= \d_1(u_2v_3) - \d_1(u_3 v_2) + \d_2 (u_3v_1) - \d_2(u_1 v_3) + \d_3(u_1 v_2) - \d_3 (u_2 v_1) \\
					&= v_3 \d_1 u_2 + u_2 \d_1 v_3 - v_2 \d_1 u_3 - u_3 \d_1 v_2 + v_1 \d_2 u_3 + u_3 \d_2 v_1 - v_3 \d_2 u_1 - u_1 \d_2 v_3  \\
					&\qquad + v_2 \d_3 u_1 + u_1 \d_3 v_2 - v_1 \d_3 u_2 - u_2 \d_3 v_1 \\
					&= v \cdot \begin{pmatrix}
						\d_2 u_3 - \d_3 u_2 \\
						\d_3 u_1 - \d_1 u_3 \\
						\d_1 u_2 - \d_2 u_1 
					\end{pmatrix}
					- u \cdot \begin{pmatrix}
						\d_2 v_3 \\
						\d_3 v_1 \\
						\d_1 v_2
					\end{pmatrix}
					= v \cdot (\nabla \times u) - u \cdot (\nabla \times v)
				\end{align*}
			\item
				\begin{align*}
					\nabla \times (\nabla f)
					= \begin{pmatrix}
						\d_2 \d_3 f - \d_3 \d_2 f \\
						\d_3 \d_1 f - \d_1 \d_3 f \\
						\d_1 \d_2 f - \d_2 \d_1 f
					\end{pmatrix}
					= 0
				\end{align*}
				\begin{align*}
					\nabla \cdot (\nabla \times u)
					= \d_1\d_2 u_3 - \d_1 \d_3 u_2 + \d_2 \d_3 u_1 - \d_2 \d_1 u_3 + \d_3 \d_1 u_2 - \d_3 \d_2 u_1
					= 0
				\end{align*}
			\item
				\begin{align*}
					\nabla \times (\nabla \times u)
					&= \begin{pmatrix}
						\d_2 \d_1 u_2 - \d_2^2 u_1 - \big( \d_3^2 u_1 - \d_3 \d_1 u_3\big) \\
						\d_3 \d_2 u_3 - \d_3^2 u_2 - \big( \d_1^2 u_2 - \d_1 \d_2 u_1 \big) \\
						\d_1 \d_3 u_1 - \d_1^2 u_3 - \big( \d_2^2 u_3 - \d_2 \d_3 u_2 \big)
					\end{pmatrix}
					= \begin{pmatrix}
						\d_1 \d_2 u_2 + \d_1 \d_3 u_3 + \d_1^2 u_1 \\
						\d_2 \d_3 u_3 + \d_1 \d_2 u_1 + \d_2^2 u_2 \\
						\d_1 \d_3 u_1 + \d_2 \d_3 u_2 + \d_3^2 u_3
					\end{pmatrix}
					- \Delta u \\
					&= \begin{pmatrix}
						\d_1 (\nabla \cdot u) \\
						\d_2 (\nabla \cdot u) \\
						\d_3 (\nabla \cdot u)
					\end{pmatrix}
					 - \Delta u
					 = \nabla (\nabla \cdot u) - \Delta u
				\end{align*}
		\end{enumerate}
	\end{aufgabe}

	\newpage

	\begin{aufgabe}~

		\begin{enumerate}[a)]
			\item
				
		Nach einer Umdrehung hat das Rad seinen Umfang sozusagen einmal „abgewickelt“ und sich um diese Strecke in $x$-Richtung bewegt.
		Man erhält aus $2\pi r = v T$ also für die Umlaufzeit
		\[
			T = \f {2\pi r}v
		\]
		Sei
		\[
			K(t) := \begin{pmatrix}
				- r \sin (\tf {2\pi t}T) \\
				- r \cos (\tf {2\pi t}T) 
			\end{pmatrix}
			= \begin{pmatrix}
				- r \sin (\tf {tv}r) \\
				- r \cos (\tf {tv}r)
			\end{pmatrix}
			\qquad
			M(t) := \begin{pmatrix}
				v t \\
				r
			\end{pmatrix}
		\]
		$K(t)$ beschreibt dann den Ort des Peripheriepunktes $P$ auf einer sich drehenden Kreisscheibe um den Ursprung und $M(t)$ den Ort des Mittelpunktes der rollenden Kreisscheibe.
		$x(t)$ ergibt sich also zusammengesetzt:
		\[
			x(t) := K(t) + M(t) = \begin{pmatrix}
				vt - r \sin(\tf {tv}r) \\
				r - r \cos (\tf {tv}r)
			\end{pmatrix}
		\]

	\item
		Man rechnet
		\begin{align*}
			x'(t) = \begin{pmatrix}
				v- v \cos(\tf{tv}r) \\
				v \sin(\tf{tv}r)
			\end{pmatrix}
		\end{align*}
		und damit
		\begin{align*}
			\| x'(t) \|
			&= \sqrt{v^2\big(1-\cos (\tf{tv}r)\big)^2 + v^2 \sin^2(\tf{tv}r)} \\
			&= v \sqrt{1- 2 \cos (\tf{tv}r) + \cos^2 (\tf {tv}r) + \sin^2 (\tf {tv}r)} 
			= v \sqrt{2} \sqrt{1 - \cos (\tf {tv}r)}
		\end{align*}
		also
		\begin{align*}
			\max_{t\in \R} \|x'(t)\| &= 2v \\
			\min_{t\in \R} \|x'(t)\| &= 0
		\end{align*}
	\item
		Mit der Darstellung für $\|x'(t)\|$ aus b) gilt:
		\begin{align*}
			L_P = \int_0^T \|x'(t)\| dt
			&= \sqrt 2 v \int_0^T \sqrt{1 -\cos (\tf {tv}r)} dt
			= \sqrt 2 r \int_0^{2\pi} \sqrt{1 - \cos u} \; du \\
			&= \sqrt 2 r \int_0^{2\pi} \sqrt{2 \sin^2 (\tf u2)} du
			= 4 r \int_0^{\pi} \sin(w) dw 
			= 8 r
		\end{align*}
		Für die Länge des zurückgelegten Weges von $M$ gilt (eigentlich auch offensichtlich)
		\[
			L_M 
			= \int_0^T \|M'(t)\| dt 
			= \int_0^T v dt 
			= \big[ v t \big]_{t=0}^{t=T}
			= v T
			= 2 \pi r
		\]
		und damit $L_M = \f \pi 4 L_p$.
		\end{enumerate}
	\end{aufgabe}

\end{document}


