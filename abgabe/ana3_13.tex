\documentclass[a4paper]{scrartcl}
\usepackage{mathe-blatt}
\blattanadrei

\begin{document}

\setcounter{section}{13}

\begin{aufgabe}
	\begin{enumerate}[(a)]
		\item
			Da $(a_1,a_2)$ eine Basis ist, können wir $(\tilde a_1, \tilde a_2)$ schreiben als
			\begin{align*}
				\tilde a_1 &= \lambda_{11} a_1 + \lambda_{12} a_2 \\
				\tilde a_2 &= \lambda_{21} a_1 + \lambda_{22} a_2
			\end{align*}
			Damit gilt
			\begin{align*}
				\tilde a_1 \times \tilde a_2
				&= (\lambda_{11}a_1 + \lambda_{12}a_2) \times (\lambda_{21}a_1 + \lambda_{22}a_2) \\
				&= \lambda_{11}\lambda_{21}\underbrace{a_1\times a_1}_{=0} + \lambda_{11} \lambda_{22}a_1 \times a_2 + \lambda_{12}\lambda_{21} \underbrace{a_2 \times a_1}_{= - a_1\times a_2}+ \lambda_{12}\lambda_{22} \underbrace{a_2\times a_2}_{=0} \\
				&= \underbrace{(\lambda_{11}\lambda_{22} - \lambda_{12}\lambda_{21})}_{=:\lambda}a_1\times a_2
			\end{align*}
			Also $\tilde a_1 \times \tilde a_2 = \lambda a_1 \times a_2$ mit $\lambda = \lambda_{11}\lambda_{22} - \lambda_{21} \lambda_{12}$.
			Jetzt ist
			\begin{align*}
				(a_1,a_2), (\tilde a_1,a_2) \text{ gleich orientiert}
				\qquad&\iff\qquad 0 < \det \begin{pmatrix}
					\lambda_{11} & \lambda_{12} \\
					\lambda_{21} & \lambda_{22}
				\end{pmatrix} = \lambda_{11}\lambda_{22} - \lambda_{21}\lambda_{12} = \lambda
			\end{align*}
			Womit die Behauptung sofort dasteht.
		\item
			Wir zeigen zunächst ein kleines nützliches Lemma:
			\begin{lem*}
				Zwei Karten $(\phi, U), (\psi, V)$ einer 2-dimensionales $C^1$-Mannigfaltigkeit sind genau dann gleich orientiert, wenn die Basen der Tangentialräume für jeden Punkt $x \in U\cap V$:
				\[
					\bigg( \f{\d}{\d y_1} \phi(\phi^{-1}(x)), \f{\d}{\d y_2} \phi(\phi^{-1}(x)) \bigg)
					\qquad \text{und}\qquad
					\bigg( \f{\d}{\d y_1} \psi(\psi^{-1}(x)), \f{\d}{\d y_2} \psi(\psi^{-1}(x)) \bigg)
				\]
				gleich orientiert sind.
				\begin{proof}
					Wegen
					\begin{align*}
						\f{\d}{\d y_i} \phi(\phi^{-1}(x))
						&= \f{\d}{\d y_i} (\psi \circ \psi^{-1} \circ \phi)(\phi^{-1}(x)) \\
						&= \sum_{j=1}^2 \f {\d}{\d y_i} \psi((\psi^{-1}\circ \phi)(\phi^{-1}(x))) \cdot \underbrace{\bigg( \f{\d}{\d z} (\psi^{-1} \circ \phi)(\phi^{-1}(x))\bigg)_j}_{=:\lambda_j} \\
						&= \sum_{j=1}^2 \lambda_j \f{\d}{\d y_i}\psi(\psi^{-1}(x))
					\end{align*}
					ist die Matrix
					\[
						\f{\d}{\d z} (\psi^{-1}\circ \phi)(\phi^{-1}(x))
					\]
					gerade die Basiswechselmatrix.
					Der Rest folgt aus den jeweiligen Definitionen der Orientiertheit.
				\end{proof}
			\end{lem*}
			\begin{seg}[„$\implies$“]
				Für $x\in S$ und einer beliebigen Karte $(\phi, U)$ in $x$ aus einem orientiertem Atlas definieren wir
				\begin{align*}
					\hat n(x) &:= \f{\d}{\d y_1}\phi(\phi^{-1}(x)) \times \f{\d}{\d y_2} \phi(\phi^{-1}(x)) \\
					n(x) &:= \f{\hat n(x)}{\|\hat n(x)\|}
				\end{align*}
				Weil $\phi$ bijektiv ist, sind die Vektoren im Kreuzprodukt von $\hat n(x)$ stets linear unabhängig und damit der Nenner in $n(x)$ stets ungleich Null.
				Außerdem spielt die Wahl der Karte $\phi$ keine Rolle, da nach der (a) und dem Lemma $\hat n(x)$ stets in die gleiche Richtung zeigt.

				Offensichtlich ist mit dieser Konstruktion von $n(x)$ gewährleistet, dass $n(x)$ senkrecht auf $T(x)$ steht und $\|n(x)\|=1$.
				Die Stetigkeit folgt als Komposition stetiger Abbildungen.
			\end{seg}
			\begin{seg}[„$\Longleftarrow$“]
				Wir definieren einen Atlas auf $S$.
				Dazu nehmen wir alle Karten $(\phi, U)$ ($U\subset S$) in den Atlas auf für die gilt
				\[
					\f{\d}{\d y_1}\phi(\phi^{-1}(x)) \times \f{\d}{\d y_2} \phi(\phi^{-1}(x))
					= \lambda_x n(x) \qquad 
				\]
				für beliebiges $x \in U$ und mit $\lambda_x > 0$.
				
				Da $\f{\d}{\d y_1}\phi(\phi^{-1}(x)) \times \f{\d}{\d y_2} \phi(\phi^{-1}(x))$ stetig ist und stets ungleich $0$ ($\phi$ bijektiv) und $n(x)$ ebenfalls stetig und ungleich $0$ ist, gilt die Aussage sofort für alle $x\in U$, falls sie für eines gilt (die Wahl von $x\in U$ ist für die Definition unerheblich).
				Außerdem ist die Existenz solcher Karten ist klar (siehe Votieraufgabe 12.1.).
				Damit ist der Atlas wohldefiniert.

				Seien nun $(\phi, U), (\psi, V)$ zwei beliebige Karten aus diesem Atlas mit $U\cap V \neq \emptyset$ und sei $x\in U\cap V$.
				Dann zeigen nach Atlasdefinition
				\[
					\f{\d}{\d y_1}\phi(\phi^{-1}(x)) \times \f{\d}{\d y_2} \phi(\phi^{-1}(x))
					\qquad\text{und}\qquad
					\f{\d}{\d y_1}\psi(\psi^{-1}(x)) \times \f{\d}{\d y_2} \psi(\psi^{-1}(x))
				\]
				in die gleiche Richtung (Richtung $n(x)$).
				Nach der (a) sind also die Basen
				\[
					\bigg( \f{\d}{\d y_1} \phi(\phi^{-1}(x)), \f{\d}{\d y_2} \phi(\phi^{-1}(x)) \bigg)
					\qquad \text{und}\qquad
					\bigg( \f{\d}{\d y_1} \psi(\psi^{-1}(x)), \f{\d}{\d y_2} \psi(\psi^{-1}(x)) \bigg)
				\]
				gleich orientiert und nach dem Lemma damit auch $(\phi, U)$ und $(\psi, V)$.
			\end{seg}
	\end{enumerate}
\end{aufgabe}

\end{document}


