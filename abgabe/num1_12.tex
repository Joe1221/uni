\documentclass[a4paper]{scrartcl}
\usepackage{mathe-blatt}
\usepackage{graphicx}
\blattnumeins

\begin{document}


\begin{aufgabe}~

	\begin{enumerate}[a)]
		\item
			Es soll gelten
			\[
				F(x_1,x_2) = \begin{pmatrix}
					x_1x_2 + x_1 - x_2 - 1 \\ x_1x_2
				\end{pmatrix}
				\stackrel != 0
			\]
			Im Fall $x_1 = 0$, ergibt sich sofort $x_2 = -1$.
			Im Fall $x_1 \neq 0$, ergibt sich sofort $x_2 = 0$ und daraus $x_1 = 1$.

			Damit sind die Nullstellen gegeben durch die Menge
			\[
				N = \l\{ \begin{pmatrix}
					0 \\ -1 
				\end{pmatrix}, \begin{pmatrix}
					1 \\ 0
				\end{pmatrix} \r\}
			\]
		\item
			Die Jacobi-Matrix ist gegeben durch
			\[
				DF(x_1,x_2) = \begin{pmatrix}
					x_2 + 1 & x_1 - 1 \\
					x_2 & x_1
				\end{pmatrix}
			\]
			Für die Newton-Iterationsvorschrift invertiere man die Matrix:
			\[
				\Big(DF(x_1,x_2)\Big)^{-1} = \f 1{x_1+x_2} \begin{pmatrix}
					x_1 & 1-x_1 \\
					-x_2 & x_2+1
				\end{pmatrix}
			\]
			Damit ergibt sich die Newtoniterationsvorschrift durch
			\begin{align*}
				x^{(k+1)} = \begin{pmatrix}
					x_1^{(k+1)} \\ x_2^{(k+1)}
				\end{pmatrix} :&= x^{(k)} - \Big(DF(x^{(k)}\Big)^{-1}\cdot F(x^{(k)}) \\ 
				&= \begin{pmatrix}
					x_1^{(k)} \\ x_2^{(k)}
				\end{pmatrix} - \f 1{x^{(k)}-x_2^{(k)}} \begin{pmatrix}
					x_1^{(k)} & 1 - x_1^{(k)} \\
					-x_2^{(k)} & x_2^{(k)} + 1
				\end{pmatrix} \cdot \begin{pmatrix}
					x_1^{(k)} x_2^{(k)} + x_1^{(k)} - x_2^{(k)} - 1 \\
					x_1^{(k)} x_2^{(k)}
				\end{pmatrix} \\ 
				&= \begin{pmatrix}
					x_1^{(k)} \\ x_2^{(k)}
				\end{pmatrix} - \f 1{x^{(k)} + x_2^{(k)}} \begin{pmatrix}
					\big(x_1^{(k)}\big)^2 - x_1^{(k)} \\
					\big(x_2^{(k)}\big)^2 - x_2^{(k)}
				\end{pmatrix}  \\
				&= \f 1{x_1^{(k)} + x_2^{(k)}} \begin{pmatrix}
					x_1^{(k)} x_2^{(k)} + x_1^{(k)} \\
					x_1^{(k)} x_2^{(k)} - x_2^{(k)} \\
				\end{pmatrix}
			\end{align*}
		\item
			Für den Startwert $x^{(0)} = (2,2)^T$:
			\begin{align*}
				x^{(1)} &= \f 1{2+2} \begin{pmatrix}
					2 \cdot 2 + 2 \\
					2 \cdot 2 - 2
				\end{pmatrix} = \begin{pmatrix}
					\f 32 \\[0.4em]
					\f 12
				\end{pmatrix} \\
				x^{(2)} &= \f 1{\f 32 + \f 12} \begin{pmatrix}
					\f 34 + \f 32 \\[0.4em]
					\f 34 - \f 12
				\end{pmatrix} = \begin{pmatrix}
					\f 98 \\[0.4em]
					\f 18
				\end{pmatrix}\\
				x^{(3)} &= \f 1{\f 98 + \f 18} \begin{pmatrix}
					\f 9{64} + \f 98 \\[0.4em]
					\f 9{64} - \f 18
				\end{pmatrix} = \begin{pmatrix}
					\f {81}{80} \\[0.4em]
					\f 1{80}
				\end{pmatrix}
			\end{align*}
			Die nächste Nullstelle ist $(1,0)^T$, es ergibt sich einen Fehler von
			\[
				\|x^{(3)} - x^*\| = \sqrt{\big(\tf 1{80}\big)^2 + \big(\tf 1{80}\big)^2} = \f {\sqrt 2}{80}
			\]
			Für den Startwert $x^{(0)} = (0,-2)^T$:
			\begin{align*}
				x^{(1)} &= \f 1{0-2} \begin{pmatrix}
					0 \\ 0 - (-2)
				\end{pmatrix} = \begin{pmatrix}
					0 \\[0.4em]
					-1
				\end{pmatrix} = x^* \qquad \implies \qquad 
				x^{(2)} = x^{(3)} = x^*
			\end{align*}
			Der Fehler ist dann offensichtlich
			\[
				\|x^{(3)} -x^*\| = 0
			\]
	\end{enumerate}
\end{aufgabe}

\begin{aufgabe}~

	\begin{enumerate}[a)]
		\item
			Da $\Phi(x) \in C^2$, kann man den Grenzwert einfach über L'Hospital berechnen:  
			\begin{align*}
				\lim_{k\to \infty} \f {x^{(k+1)}-x^*}{(x^{(k)}-x^*)^2}
				= \lim_{x \to x^*} \f {\Phi(x) - x^*}{(x - x^*)^2} 
				= \lim_{x \to x^*} \f {\Phi'(x)}{2(x-x^*)}
				= \lim_{x \to x^*} \f {\Phi''(x)}2
				= \f {\Phi''(x^*)}2
			\end{align*}
		\item
			In der Iteration
			\[
				\Psi(x) := x - \f {(\Phi(x) - x)^2}{\Phi(\Phi(x)) - 2\Phi(x) + x}
			\]
			entwickle $\Phi(\Phi(x))$ an der Stelle $x$.
			Für den Nenner ergibt sich dann
			\begin{align*}
				\Phi(\Phi(x)) - 2 \Phi(x) + x
				&= \Phi(x) + \Phi'(\xi)\Big(\Phi(x)-x\Big) - 2 \Phi(x) + x \\
				&= (\Phi(x) - x)(\Phi'(\xi) - 1)
			\end{align*}
			Dabei liegt $\xi$ zwischen $x$ und $\Phi(x)$.
			Im Folgenden betrachten wir deshalb mal $\xi$ als eine differenzierbare Funktion, die von $x$ abhängt (ohne Bedenken des Autors).

			Die Iteration vereinfacht sich damit zu
			\[
				\Psi(x) = x - \f {\Phi(x) - x}{\Phi'(\xi(x)) - 1}
			\]
			Die erste Ableitung dann
			\[
				\Psi'(x) = 1 - \f{(\Phi'(x)-1)(\Phi'(\xi(x))-1) - (\Phi(x) - x)\big(\Phi''(\xi(x)) \xi'(x) \big)}{\big(\Phi'(\xi(x))-1\big)^2}
			\]
			Betrachtet man den Grenzwert $x \to x^*$ ergibt sich $\xi(x) = x^*$ ($\xi$ lag immer zwischen $x$ und $\Phi(x)$).
			Da außerdem $\Phi'(x^*) \neq 1$ gilt, berechnet sich $\Psi'(x^*)$ als
			\[
				\Psi'(x^*) = \lim_{x\to x^*}\Psi'(x) = 1 - \f{(\Phi'(x^*)-1)(\Phi'(x^*)-1) - \overbrace{(\Phi(x^*) - x^*)}^{=0}\big(\Phi''(x^*) \xi'(x^*) \big)}{\big(\Phi'(x^*)-1\big)^2}
				= 1 - 1 = 0
			\]
			Damit hat die Iteration $\Psi$ mindestens Konvergenzordnung $2$.
	\end{enumerate}

\end{aufgabe}


\begin{aufgabe}~

	$M = [\f 74, 2]$ ist nichtleer und abgeschlossen.

	Da $F$ monoton auf $M$, reicht es, die Grenzen zu überprüfen:
	\begin{align*}
		F(\tf 74) &= 1 + \f 47 + \f {16}{49} = 1 + \f {44}{49} \in M &
		F(2) &= 1 + \f 12 + \f 14 = \f 74 \in M
	\end{align*}
	Also ist $F$ selbstabbildend.
	Außerdem ist wegen
	\[
		\f {|F(x) - F(y)|}{|x-y|} 
		= |F'(\xi)| 
		= \bigg| - \f 1{\xi^2} - \f 2{\xi^3} \bigg| \stackrel{\xi \in (\f 74, 2)}
		= \underbrace{\f 1{\xi^2} + \f 2{\xi^3} }_{\text{mon. fallend}} 
		\stackrel{\xi \in (\f 74, 2)}< \f 1{(\f 74)^2} + \f 2{(\f 74)^3} 
		= \underbrace{\f {240}{343}}_{=:L} < 1
	\]
	$F(x)$ lipschitzstetig mit Lipschitzkonstante $L = \f {240}{343}$.
	Die Vorraussetzungen für den Fixpunktsatz von Banach sind damit erfüllt.
\end{aufgabe}

\setcounter{aufgabe}{3}
 
\begin{aufgabe}
	
	\begin{enumerate}[a)]
		\item
			Gewünschte Routine:
			\lstinputlisting[language=matlab]{num1_11_4/sekd.m}
		\item
			Ausgabe:
			\lstinputlisting[language=matlab]{num1_11_4/b.m}
			\begin{table}[h]
				\caption{Werte ohne und mit Dämpfung für $x^{(0)}=4, x^{(1)}=4.01, k_{\text{max}}=20, \eps_{\text{tol}}=10^{-10}$}
				\centering
				\begin{tabular}{c|c|c}
					$k$ & Ohne Dämpfung & Mit Dämpfung \\ \hline
					0 & 4 & 4 \\
					1 & 4.0100 & 4.0100 \\
					2 & -8.5279 & 1.3303 \\
					3 & -1.7600 & 0.94014 \\
					4 & 32.303 & 0.98125 \\
					5 & 13.325 & 0.99712 \\
					6 & -563.43 & 0.99982 \\
					7 & -267.59 & 1.00000 \\
					8 & 2.3730e+05 & 1.00000 \\
					9 & 1.1838e+05 & 1.00000 \\
					10 & -4.4124e+10 & 1.00000 \\
					11 & -2.2062e+10 &  \\
					12 & 1.5291e+21 &  \\
					13 & 7.6455e+20 & \\
					14 & Inf & \\
					15 & NaN
				\end{tabular}				
			\end{table}
			Man sieht deutlich, dass die Iteration ohne Dämpfung divergiert, während sie mit Dämpfung gegen die Nullstelle $x^* = 1$ konvergiert.

	\end{enumerate}
\end{aufgabe}
\end{document}
