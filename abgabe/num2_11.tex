\documentclass{mywork}
\blattnumzwei

\begin{document}


\setcounter{section}{11}

\begin{aufgabe}
	\begin{enumerate}[$A$:]
		\item
			Für das charakteristische Polynom von $A$ ergibt sich
			\[
				\chi_A(\lambda) 
				= \det(A-\lambda I) 
				= (3-\lambda)(2i - \lambda)(-1-\lambda)
			\]
			und somit als Eigenwerte von $A$: $\lambda_1=3, \lambda_2 = 2i, \lambda_3 = -1$.
			Die Eigenwerte sind paarweise verschieden,
			damit ist $A$ diagonalisierbar.
			Für die Eigenräume zu den Eigenwerten erhält man (als Lösungen der jeweiligen homogenen linearen Gleichungssysteme $A-\lambda_i I = 0$):
			\[
				\ker(A-3I) = \l\< \begin{pmatrix}
					52 \\ 10 + 11i \\ 13
				\end{pmatrix}\r\>, \qquad
				\ker(A-2i I) = \l\< \begin{pmatrix}
					0 \\ 1 \\ 0
				\end{pmatrix}\r\>, \qquad
				\ker(A + I) = \l\< \begin{pmatrix}
					0 \\ -2 -i \\ 5
				\end{pmatrix}\r\>
			\]
			Somit lässt sich die Diagonalmatrix angeben als $D = X_A^{-1} A X_A$, wobei
			\[
				D = \begin{pmatrix}
					3 & 0 & 0 \\
					0 & 2i & 0 \\
					0 & 0 & -1
				\end{pmatrix}, \qquad
				X_A = \begin{pmatrix}
					52 & 0 & 0 \\
					10 + 11i & 1 & -2 -i \\
					13 & 0 & 5
				\end{pmatrix}
			\]
		\item
			Es gilt
			\[
				\chi_B = -(3-\lambda)^2 \lambda,
			\]
			also ergeben sich die Eigenwerte: $\lambda_{1,2} = 3, \lambda_3 = 0$.
			Der Eigenwert $3$ hat algebraische Vielfachheit $2$, allerdings ist der Eigenraum zu Eigenwert $3$,
			\[
				\ker (A-3I) = \l\< \begin{pmatrix}
					1 \\ 1 \\ 0
				\end{pmatrix}\r\>,
			\]
			nur eindimensional.
			Also kann $B$ nicht diagonalsierbar sein.
	\end{enumerate}
\end{aufgabe}

\begin{aufgabe}
	\begin{enumerate}[a)]
		\item
			Für $A$:
			\begin{align*}
				K_1 &= \{ \zeta \in \C : | \zeta - 3 | \le 0 \}, \\
				K_2 &= \{ \zeta \in \C : | \zeta - 2i | \le 2 \}, \\
				K_3 &= \{ \zeta \in \C : | \zeta + 1 | \le 1 \}.
			\end{align*}
			Und für $A^T$:
			\begin{align*}
				K_1' &= \{ \zeta \in \C : | \zeta - 3 | \le 2 \}, \\
				K_2' &= \{ \zeta \in \C : | \zeta - 2i | \le 0 \}, \\
				K_3' &= \{ \zeta \in \C : | \zeta + 1 | \le 1 \}.
			\end{align*}
		\item
			Wegen $\sigma(A) = \sigma(A^T)$ wäre eine möglichst kleine im Kontext der Greschgorinkreise, die Eigenwerte von $A$ einschließende Menge
			\[
				M := \Bigg (\bigcup_{i=1}^3 K_i\Bigg ) \cap \Bigg (\bigcup_{i=1}^3 K'_i\Bigg ) = \{3, 2_i\} \cup K_3\cup (K_2 \cap K'_1)
			\]
	\end{enumerate}
\end{aufgabe}

\end{document}
