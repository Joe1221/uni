\documentclass[a4paper]{scrartcl}
\usepackage{mathe-blatt}
\usepackage{graphicx}
\blattnumeins

\begin{document}


\begin{aufgabe}~
	\begin{align*}
		f(x) &= x^4 + 13x^3 + 7x^2 - 165x \\
		f'(x) &= 4x^3 + 39x^2 + 14x - 165 \\
		f''(x) &= 12x^2 + 78x^2 + 14
	\end{align*}
	Man findet die Nullstellen $\{-11,-5,0,3\}$ von $f$ und wähle als Intervalle
	\begin{align*}
		I'_1 = [-12,-10], \qquad
		I'_2 = [-6,-4], \qquad
		I'_3 = [-1,1], \qquad
		I'_4 = [2,4]
	\end{align*}
	Ein Blick auf das Schaubild von $f'$ bestätigt, dass $\min_{x\in [a,b]} |f'(x)| > 0$ erfüllt ist.
	Also wenden wir den Satz zur Konvergenz des Newton-Verfahrens an:

	Berechne/finde für die einzelnen Intervalle
	\begin{align*}
		m &:= \min_{x\in [a,b]}|f'(x)| > 0 \\
		M &:= \max_{x\in [a,b]}|f''(x)| \\
		\delta &< \f {2m}M
	\end{align*}
	\begin{enumerate}[${I'}_1$:]
		\item $m=405, M=806, \delta < \f{405}{403}$, also konvergiert das Verfahren in ganz $[-12,10]$.
		\item $m=147, M=106, \delta < \f{147}{106}$, also konvergiert das Verfahren in ganz $[-6,-4]$.
		\item $m=108, M=104, \delta < \f{27}{13}$, also konvergiert das Verfahren in ganz $[-1,1]$.
		\item $m=51, M=518, \delta < \f{51}{259}$, wähle $\delta = \f 1{10}$ und es ergibt sich das Intervall $[2.9, 3.1]$.
	\end{enumerate}
	Es ergeben sich also folgende Intervalle in denen das Newtonverfahren lokal gegen die jeweilige Nullstelle konvergiert:
	\begin{align*}
		I'_1 = [-12,10], \qquad
		I'_2 = [-6,-4], \qquad
		I'_3 = [-1,1], \qquad
		I'_4 = [2.9, 3.1]
	\end{align*}
\end{aufgabe}

\setcounter{aufgabe}{2}
\begin{aufgabe}~

	Ich verwende die äquivalente Definition $\max_{\|x\|=1}\f {\|Ax\|}{\|x\|}$ für die induzierte Matrixnorm.
	\begin{itemize}
		\item
			Es gilt
			\[
				\|A\|_1 := \max_{\|x\|_1=1} \|Ax\|_1 = \max_{\|x\|_1=1}\sum_{i=1}^n \Big| \sum_{j=1}^n a_{ij} x_j \Big|
			\]
			das ist einerseits
			\[
				\le \max_{\|x\|_1=1} \sum_{i,j=1}^n |a_{ij}| |x_j| \le \max_{\|x\|_1=1} \sum_{j=1}^n \bigg( |x_j| \max_{k=1,\dotsc,n} \sum_{i=1}^n |a_{ik}| \bigg) \le \max_{k=1,\dotsc,n} \sum_{i=1}^n |a_{ik}|
			\]
			und andererseits, indem man $l$ so wählt, dass $\displaystyle \sum_{j=1}^n |a_{jl}| = \max_{j=1,\dotsc,n} \sum_{i=1}^n |a_{ij}|$ und $x := e_l$ wählt:
			\[
				\ge \sum_{i=1}^n \Big| \sum_{j=1}^n a_{ij} \delta_{jl} \Big| = \sum_{i=1}^n |a_{il} | = \max_{j=1,\dotsc,n} \sum_{i=1}^n |a_{ij}|
			\]
			damit folgt also $\displaystyle \|A\|_1 = \max_{i=1,\dotsc,n} \sum_{i=1}^n |a_{ij}|$
		\item
			Es gilt
			\[
				\|A\|_\infty := \max_{\|x\|_\infty=1} \|Ax\|_\infty = \max_{\|x\|_\infty=1}\max_{i=1,\dotsc,n} \Big| \sum_{j=1}^n a_{ij} x_j \Big|
			\]
			das ist einerseits
			\[
				\le \max_{\|x\|_\infty=1} \max_{i=1,\dotsc,n} \sum_{j=1}^n |a_{ij}||x_j|
				\le \max_{i=1,\dotsc,n} \sum_{j=1}^n |a_{ij}|
			\]
			und andererseits, indem man $l$ so wählt, dass $\displaystyle \sum_{j=1}^n |a_{jl}| = \max_{i=1,\dotsc,n} \sum_{j=1}^n |a_{ij}|$ und $x := \begin{cases}1 & a_{li} \ge 0 \\ -1 & a_{li} < 0 \end{cases}$ wählt:
			\[
				\ge \max_{i=1,\dotsc,n} \Big| \sum_{j=1}^n a_{ij} x_j \Big|
				\ge \Big| \sum_{j=1}^n \underbrace{a_{lj} x_j}_{\ge 0} \Big| = \sum_{j=1}^n |a_{lj}| = \max_{i=1,\dotsc,n} \sum_{j=1}^n |a_{ij}|
			\]
			damit folgt also $\displaystyle \|A\|_\infty = \max_{j=1,\dotsc,n} \sum_{j=1}^n |a_{ij}|$

	\end{itemize}
\end{aufgabe}

\begin{aufgabe}~

	\begin{enumerate}[a)]
		\item
			Bisektionsverfahren:
			\lstinputlisting[language=matlab]{num1_10_4/bisektion.m}
		\item
			Newtonverfahren:
			\lstinputlisting[language=matlab]{num1_10_4/newton.m}
		\item
			Test mit $f(x) = \f{\cos(x)}{\sin(x)}$
			\lstinputlisting[language=matlab]{num1_10_4/c.m}
			\begin{table}[h]
				\centering
				\caption{Ergebnisse}
				\begin{tabular}{c|c|c|c|c}
					$n$ & Bisektion & Bisektion (Fehler) & Newton (Fehler) & Newton \\ \hline
1 & 1.5000000 & 0.0707963 & 0.1161476 & 1.4546487\\
2 & 1.7500000 & 0.1792037 & 0.0010418 & 1.5697546\\
3 & 1.6250000 & 0.0542037 & 0.0000000 & 1.5707963\\
4 & 1.5625000 & 0.0082963 & 0.0000000 & 1.5707963\\
5 & 1.5937500 & 0.0229537 & 0.0000000 & 1.5707963\\
6 & 1.5781250 & 0.0073287 & 0.0000000 & 1.5707963\\
7 & 1.5703125 & 0.0004838 & 0.0000000 & 1.5707963\\
8 & 1.5742188 & 0.0034224 & 0.0000000 & 1.5707963\\
9 & 1.5722656 & 0.0014693 & 0.0000000 & 1.5707963\\
10 & 1.5712891 & 0.0004927 & 0.0000000 & 1.5707963\\
11 & 1.5708008 & 0.0000045 & 0.0000000 & 1.5707963\\
12 & 1.5705566 & 0.0002397 & 0.0000000 & 1.5707963\\
13 & 1.5706787 & 0.0001176 & 0.0000000 & 1.5707963\\
14 & 1.5707397 & 0.0000566 & 0.0000000 & 1.5707963\\
15 & 1.5707703 & 0.0000261 & 0.0000000 & 1.5707963\\
16 & 1.5707855 & 0.0000108 & 0.0000000 & 1.5707963\\
17 & 1.5707932 & 0.0000032 & 0.0000000 & 1.5707963\\
18 & 1.5707970 & 0.0000006 & 0.0000000 & 1.5707963\\
19 & 1.5707951 & 0.0000013 & 0.0000000 & 1.5707963\\
20 & 1.5707960 & 0.0000003 & 0.0000000 & 1.5707963\\
21 & 1.5707965 & 0.0000002 & 0.0000000 & 1.5707963\\
22 & 1.5707963 & 0.0000001 & 0.0000000 & 1.5707963\\
23 & 1.5707964 & 0.0000000 & 0.0000000 & 1.5707963\\
24 & 1.5707963 & 0.0000000 & 0.0000000 & 1.5707963\\
25 & 1.5707963 & 0.0000000 & 0.0000000 & 1.5707963\\
26 & 1.5707963 & 0.0000000 & 0.0000000 & 1.5707963\\
27 & 1.5707963 & 0.0000000 & 0.0000000 & 1.5707963\\
				\end{tabular}
			\end{table}
	\end{enumerate}
\end{aufgabe}
\end{document}
