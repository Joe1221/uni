\documentclass[a4paper]{scrartcl}
\usepackage{mathe-blatt}
\blattanadrei

\begin{document}

	\setcounter{section}{11}

	\begin{aufgabe}~
		
		Sei $k$ die Dimension der Mannigfaltigkeit $X$.
		Zu $x = (\tilde x, x_{n+1}) \in X \times (0,1]$ (mit $\tilde x \in X$, $x_{n+1} \in (0,1]$ und Karte $(\tilde\phi_{\tilde x}, \tilde U(\tilde x)$) existiert $(\phi_x, U(x))$, definiert durch
		\begin{alignat*}{2}
			\phi_x &:\; &K_1^{(k+1)+}(0) &\to U(x) \\
			 &&(\tilde y, y_{k+1}) &\mapsto \big( \tilde \phi(\tilde y), 1-y_{k+1} \big)
		\end{alignat*}
		und $U(x) := \tilde U(\tilde x) \times (0,1]$.
		$\phi_x, \phi_x^{-1}$ sind damit ebenfalls $C^m$ und außerdem ist $\phi_x$ bijektiv ($\tilde \phi$ und $1-y_{k+1}$ bijektiv).
		Weiter ist 
		\[
			U(x) = \tilde U(\tilde x) \times (0,1] = \big(\underbrace{\tilde U(\tilde x) \times (0,2)}_{\text{offen}} \big) \cap \big(X \times (0,1]\big)
		\]
		eine offene Umgebung auf $X \times (0,1]$.
		Also ist die Karte $(\phi_x, U(x))$ wohldefiniert. 
		
		Es ist ebenfalls klar, dass endlich viele Karten ausreichen, um $X \times (0,1]$ zu beschreiben (für $X$ reichen endlich viele und die gesamte Spannweite der $(n+1)$-ten Koordinate liegt im Bild jeder obig definierten Karte).
		Also ist $X\times (0,1]$ eine Mannigfaltigkeit von Dimension $k+1$.

		Für den Rand gilt
		\begin{align*}
			\partial \big(X \times (0,1] \big) &= \Big\{ x \in X\times (0,1] : \big( \phi_x^{-1}(x) \big)_{k+1} = 0 \Big\} \\
			&= \Big\{ x \in X\times(0,1] : x_{n+1} = 1 \Big\}
		\end{align*}
	\end{aufgabe}

	\begin{aufgabe}~

		\begin{enumerate}[a)]
			\item
				Die beiden Basen der Tangentialräume sind gegeben durch
				\[
					\Big\{ \d_{x_1} \phi_1(x_1,x_2), \d_{x_2} \phi_1(x_1,x_2) \Big\}
					= \left\{ \begin{pmatrix}
						1 \\ 0 \\ \f{-x_1}{\sqrt{1-x_1^2 -x_2^2}}
					\end{pmatrix}, \begin{pmatrix}
						0 \\ 1 \\ \f{-x_2}{\sqrt{1-x_1^2 -x_2^2}}
					\end{pmatrix} \right\}
				\]
				und
				\[
					\Big\{ \d_{\psi} \phi_2(\psi,\theta), \d_{\theta} \phi_2(\psi,\theta) \Big\}
					= \left\{ \begin{pmatrix}
							-\sin \psi \cos \theta \\ \cos \psi \cos \theta \\ 0
					\end{pmatrix}, \begin{pmatrix}
						-\cos \psi \sin \theta \\ -\sin \psi \sin \theta \\ \cos \theta
					\end{pmatrix} \right\}
				\]
			\item
				Die beiden Karten überlappen sich an der Stelle $\tilde x = \big(\f{\sqrt 2}2, 0, \f{\sqrt 2}2\big)$, denn $\phi_1(\f{\sqrt 2}2, 0) = \tilde x = \phi_2(0, \f \pi 4)$.
				Betrachte jetzt
				\[
					\phi_1^{-1} \circ \phi_2 (\psi, \theta) = \begin{pmatrix}
						\cos \psi \cos \theta \\
						\sin \psi \cos \theta
					\end{pmatrix}
				\]
				Die Matrix bestehend aus den partiellen Ableitungen:
				\[
					J_{\phi_1^{-1}\circ \phi_2} (\psi, \theta) = \begin{pmatrix}
						-\sin \psi \cos \theta & -\cos \psi  \sin \theta \\
						\cos \psi \cos \theta & -\sin \psi \sin \theta
					\end{pmatrix}
				\]
				ausgewertet an der Stelle $\phi_2^{-1}(\tilde x) = (0, \f \pi 4)$:
				\[
					J_{\phi_1^{-1}\circ \phi_2} (0, \tf \pi 4) = \begin{pmatrix}
						0 & -\f {\sqrt 2}2 \\
						\f {\sqrt 2}2 & 0
					\end{pmatrix}
				\]
				hat Determinante $\det(J_{\phi_1^{-1}\circ \phi_2}) = \f 12 > 0$.
				Also sind $\phi_1$ und $\phi_2$ gleich orientiert und damit kompatibel.

				\newpage

				Da $\phi_2$ und $\phi_1$ gleich orientiert sind, können wir $\phi_2$ auf dem Rand betrachten:
				\[
					\phi_2(\psi, 0) = \begin{pmatrix}
						\cos \psi \\
						\sin \psi \\
						0
					\end{pmatrix}
				\]
				$\partial S$ muss also gegen den Uhrzeigersinn durchlaufen werden.

		\end{enumerate}
	\end{aufgabe}

\end{document}


