\documentclass{mywork}
\blattfsa

\begin{document}

\begin{aufgabe}
	\begin{enumerate}[a)]
		\item
			Die Grammatik $G = (V, \Sigma, P, S)$ mit $V=\{S,M,F\}$ und Produktionsregeln:
			\begin{align*}
				P = \Big\{
				S &\to FaMaF, & F &\to a, & M &\to MF, \\
				S &\to FbMbF, & F &\to b, & M &\to \eps \Big\}
			\end{align*}
			erzeugt die gewünschte Sprache ($a_2 = a_{n-1}$ ist stets erfüllt, $a_1$ und $a_n$ sind beliebig und der Rest kann frei gewählt werden).

			Es gilt $aaaaa \in L_1$ und $aaaba \not\in L_1$.
		\item
			Die Grammatik $G = (V, \Sigma, P, S)$ mit $V=\{S,A,B\}$ und Produktionsregeln:
			\begin{align*}
				P = \Big\{
				S &\to SAAB, & AB &\to BA, & A &\to a, \\
				S &\to \eps, & BA &\to AB, & B &\to b \Big\}
			\end{align*}
			erzeugt die gewünschte Sprache (es gibt stets doppelt so viel $A$s wie $B$s, welche beliebig umsortiert werden können, bevor sie in $a$, bzw. $b$ umgewandelt werden).

			Es gilt $aaaabb \in L_2$ und $aaaaa \not\in L_2$. 
		\item
			Die Grammatik $G = (V, \Sigma, P, S)$ mit $V=\{S,A,B,L,M,R\}$ und Produktionsregeln:
			\begin{align*}
				P = \Big\{
				S &\to aLMR, & La &\to aL, & LB &\to bL, & M &\to \eps \\
				M &\to aMBA, & AB &\to BA, & AR &\to Ra, & LR &\to \eps \Big\}
			\end{align*}
			erzeugt die gewünschte Sprache:

			Erzeuge zuerst links ein zusätzliches $a$ mit Links- und Rechtsmarkern ($L$ und $R$) und anschließend mit $M\to aMBA$ beliebig viele $a$, $B$, $A$ (aber in gleicher Anzahl).
			Alle $a$s fallen nach links durch und die $A$ und $B$s werden sortiert ($a\dotso aLB\dotso BA\dotso AR$).
			$LB \to bL$ und $AR \to Ra$ lösen von links und rechts die $B$s und $A$s auf bis sie aufeinandertreffen.

			Es gilt $aaabbaa \in L_3$ und $aaaaa \not\in L_3$. 
	\end{enumerate}
\end{aufgabe}

\begin{aufgabe}~

	Die Anzahl Vertauschungen $AB \to BA$, die man benötigt, um
	\[
		\underbrace{AA\dotso AA}_{n}\underbrace{BB\dotso BB}_{m}
	\]
	in die Form $B \dotso BA \dotso A$ zu bringen beträgt genau $n\cdot m$, wie sich leicht nachvollziehen lässt.

	Zunächst ist $n^2-1 = (n+1)(n-1)$.
	Die Idee ist jetzt einfach: Produziere zunächst einen Ausdruck der Form
	\[
		AAL\underbrace{AA \dotso AA}_{n-1} \underbrace{BB \dotso BB}_{n-1}R
		\qquad n \in \N
	\]
	durch die Regeln $S \to AALMR$, $M \to AMB$, $M \to \eps$.
	Vertausche nun $A$ und $B$ und produziere dabei immer ein $a$:
	\[
		AB \to aBA
	\]
	dabei sollen alle $a$s stets nach links zurückfallen:
	\begin{align*}
		Aa &\to aA, &
		Ba &\to aB, &
		La &\to aL
	\end{align*}
	Ist die Form
	\[
		\underbrace{aa \dotso aa}_{n^2-1} L \underbrace{B \dotso B}_{n-1} \underbrace{A \dotso A}_{n-1} R
	\]
	erreicht (oder auch früher), kann aufgeräumt werden:
	\begin{align*}
		LB &\to L, &
		AR &\to R, &
		LR &\to \eps
	\end{align*}

	Die Grammatik $G = (V, \Sigma, P, S)$ mit $V=\{S,A,B,L,M,R\}$ mit Produktionsregeln
	\begin{align*}
		P = \Big\{
		S &\to AALMR, & AB &\to aBA, & Aa &\to aA, & LB &\to L, \\
		M &\to AMB, & & & Ba &\to aB, & AR &\to R, \\
		M &\to \eps, & & & La &\to aL, & LR &\to \eps \Big\}
	\end{align*}
	erzeugt also die gewünschte Sprache.
\end{aufgabe}

\end{document}
