\documentclass[a4paper]{scrartcl}
\usepackage{mathe-blatt}
\blattpse

\begin{document}

\setcounter{aufgabe}{1}

\begin{aufgabe}~

	Siehe Quellcode im Verzeichnis „aufgabe2“.

	Die Ausgabe des Programms lautet:
	\begin{verbatim}
Gerade Zahlen zwischen 4 und 13:
4
6
8
10
12
Die Summe der ganzen Zahlen von 15 bis 100:
4945
	\end{verbatim}
\end{aufgabe}

\begin{aufgabe}~

	Siehe Quellcode im Verzeichnis „aufgabe3“.

	Die Ausgabe des Programms lautet:
	\begin{verbatim}
27.900000000000002
67.5
Die Pizza    Gesamtpreis
------------------------
Thunfisch    9.3
Salami       8.75
Paprika      7.5
Zwiebel      6.75

	\end{verbatim}
\end{aufgabe}

\begin{aufgabe}~

	Siehe Quellcode im Verzeichnis „aufgabe4“.

	Die Ausgabe des Programms lautet:
	\begin{verbatim}
Der Song BugsBunny wird hinzugefügt
Der Song Tour wird hinzugefügt
Der Song Maschendrahtzaun wird hinzugefügt
CountItems=3
Song [name=BugsBunny, type=Musik, duration=35]
Song [name=Tour, type=Rock, duration=90]
Song [name=Maschendrahtzaun, type=Country, duration=120]
Der Song1 wird gelöscht
CountItems=2
Song [name=BugsBunny, type=Musik, duration=35]
Song [name=Maschendrahtzaun, type=Country, duration=120]
Das Video Terminator wird hinzugefügt
Das Video Matrix wird hinzugefügt
CountItems=4
Video [name=Terminator, type=Action, duration=5400]
Video [name=Matrix, type=Action, duration=6300]
Die Playlist von ISTE-MediaPlayer
=================================
Song [name=BugsBunny, type=Musik, duration=35]
Song [name=Maschendrahtzaun, type=Country, duration=120]
Video [name=Terminator, type=Action, duration=5400]
Video [name=Matrix, type=Action, duration=6300]
Creating mp3 player..

Playing mediadateien/BugsBunny.mp3...
Player: playing...

Playing mediadateien/Tour.mp3...
Player: playing...

Playing mediadateien/Maschendrahtzaun.mp3...
Player: playing...

Playing mediadateien/BugsBunny.mp3...
Player: playing...

Playing mediadateien/Maschendrahtzaun.mp3...
Player: playing...
	\end{verbatim}
\end{aufgabe}

\end{document}
