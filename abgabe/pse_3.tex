\documentclass[a4paper]{scrartcl}
\usepackage{mathe-blatt}
\blattpse

\begin{document}

\setcounter{aufgabe}{1}

\begin{aufgabe}~

	Siehe Quellcode in der Datei „aufgabe2/Song.java“.
\end{aufgabe}

\begin{aufgabe}~

	Siehe Quellcode in der Datei „aufgabe3/Playlist.java“.
	Die Ausgabe des Programms sieht folgendermaßen aus:
	\begin{verbatim}
Der Song BugsBunny wird gespielt...
Der Song BugsBunny pausiert...
Der Song BugsBunny hält an...
Der Song MaschenDrahtZaun wird gespielt...
Der Song MaschenDrahtZaun pausiert...
Der Song MaschenDrahtZaun hält an...
Der Song Tour wird gespielt...
Der Song Tour pausiert...	
Der Song Tour hält an...
	\end{verbatim}
\end{aufgabe}

\begin{aufgabe}
	\begin{itemize}
		\item
			Methode 1 ist als \verb|public| definiert, während Methode 2 als \verb|private| definiert ist.
			Damit ist Methode 1 auch außerhalb der Klasse und außerhalb des Packages aufrufbar, während Methode 2 nur innerhalb der Klasse aufrufbar ist.
		\item
			Die Fehlerausgabe lautet:
			\begin{verbatim}
Exception in thread "main" java.lang.Error: Unresolved compilation problems: 
	The field Song.name is not visible
	The field Song.type is not visible
	The field Song.duration is not visible
	The field Song.name is not visible
	The field Song.type is not visible
	The field Song.duration is not visible
	The field Song.name is not visible
	The field Song.type is not visible
	The field Song.duration is not visible

	at Playlist.main(Playlist.java:32)	
			\end{verbatim}
			Weil die Attribute als \verb|private| definiert wurden, sind sie außerhalb der Klasse nicht mehr zugänglich, also auch nicht von der \verb|main()|-Methode in „aufgabe3/Playlist.java“.
		\item
			Siehe Quellcode in der Datei „aufgabe4/Song.java“.
		\item
			Siehe Quellcode in der Datei „aufgabe4/Playlist.java“.
		\item
			Die Ausgabe lautet:
			\begin{verbatim}
Der Song ist= BugsBunny
Der Song ist vom Typ: Musik
Der Song ist 35 Sekunden lang.
Der Song ist= MaschenDrahtZaun
Der Song ist vom Typ: Musik
Der Song ist 132 Sekunden lang.
Der Song ist= Tour
Der Song ist vom Typ: Musik
Der Song ist 151 Sekunden lang.	
			\end{verbatim}
	\end{itemize}
\end{aufgabe}

\begin{aufgabe}
	\begin{itemize}
		\item
			Siehe Quellcode in der Datei „aufgabe5/Song.java“.
		\item
			Siehe Quellcode in der Datei „aufgabe5/Playlist.java“.
		\item
			Die Ausgabe lautet:
			\begin{verbatim}
Song Album ist Sicksation
			\end{verbatim}
		\item
			Der Zugriff auf das statische Attribut und der Aufruf der statischen Methode auf das instanziierte Objekt funktioniert und es wird das statische Attribut, bzw. die statische Methode der zugrundeliegenden Song-Klasse aufgerufen.

			Allerdings gibt der Compiler Warnings aus, in denen er fordert, dass auf statische Attribute und Methoden auch in einer statischen Art und Weise zugegriffen werden soll (d.h. über den Klassennamen: \verb|Song.attribute|, bzw. \verb|Song.method()|).
			Das ist verständlich, da der Zugriff auf statische Methoden/Attribute nicht Befehle an das Objekt sind, sondern an die zugrundeliegende Klasse und er damit Auswirkungen für alle von ihr instanziierten Objekte hat.
			Der Aufruf über ein Objekt erzeugt also unnötige Verwirrung.
		\item
			Der Compiler beschwert sich darüber, dass auf nicht-statische Attribute und Methoden zugegriffen wird, obwohl die Klasse noch nicht instanziiert wurde.
			Die \verb|main()|-Methode ist verständlicherweise statisch und beim Ausführen dieser existiert keine aktuelle Instanz der Klasse, also können auch nicht auf lokale nicht-statische Attribute oder Methoden zugegriffen werden.
	\end{itemize}

\end{aufgabe}

\begin{aufgabe}~

	Siehe Quellcode im Verzeichnis „aufgabe6“.
	Die Ausgabe des Programms lautet folgendermaßen:
	\begin{verbatim}
Maximal 10 Titel zulässig in einer Playlist.
Maximale Fenstertitellänge: 55 Zeichen
Musiktitel 'Tolle Musik' zur Playlist hinzugefügt
Mische die Playlist ...
Der Titel von myMusic lautet: Tolle Musik
Die Bewertung für myMovie beträgt 99 Sterne.
Ändere die Größe des Fensters auf: 500x300
	\end{verbatim}
\end{aufgabe}


\end{document}
