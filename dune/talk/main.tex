\documentclass{beamer}
\usepackage{fontspec}
\usepackage{polyglossia}
\setmainlanguage{german}
\usepackage{xparse}
\usepackage{mymath}
\usepackage{stmaryrd}

\usetheme{Luebeck}
\setbeamertemplate{theorems}[normal font]
\newtheorem{satz}[theorem]{Satz}


\title{Reaktion-Diffusion\\ mit Singulären Quellen, bzw. Senken}
\subtitle{Ein Dune-Fem Programmierprojekt}


\author{Stephan Hilb}
\date{\today}
\institute{Universität Stuttgart}


\begin{document}


\begin{frame}
    \titlepage
\end{frame}


\section{Setting}

\begin{frame}
    \frametitle{Das Gebiet}
    \begin{center}
        \begin{tikzpicture}[scale=2]
            \node at (-0.8,0.8) {$\Omega$};
            \draw (-1,-1) -- (1,-1) -- (1,1) -- (-1,1) -- cycle;
            \coordinate[label=below:{$P_1$}] (p1) at (-0.33333,0.4);
            \coordinate[label=below:{$P_2$}] (p2) at (0.43,-0.5);
            \fill[blue] (p1) circle[radius=0.5pt];
            \fill[blue] (p2) circle[radius=0.5pt];
            \coordinate (g1) at (-0.7,-1);
            \coordinate (g2) at (0.8,1);
            \draw[blue] (g1) -- (g2) {};
            \node at (0.5,0.3) {$\Gamma_F$};
            \draw[very thick] (-1,-1) -- (-1,1);
            \node at (-1.2,0) {$\Gamma_D$};
        \end{tikzpicture}
    \end{center}
\end{frame}


%

\end{document}
